%% Generated by Sphinx.
\def\sphinxdocclass{report}
\documentclass[letterpaper,10pt,english]{sphinxmanual}
\ifdefined\pdfpxdimen
   \let\sphinxpxdimen\pdfpxdimen\else\newdimen\sphinxpxdimen
\fi \sphinxpxdimen=.75bp\relax

\usepackage[utf8]{inputenc}
\ifdefined\DeclareUnicodeCharacter
 \ifdefined\DeclareUnicodeCharacterAsOptional
  \DeclareUnicodeCharacter{"00A0}{\nobreakspace}
  \DeclareUnicodeCharacter{"2500}{\sphinxunichar{2500}}
  \DeclareUnicodeCharacter{"2502}{\sphinxunichar{2502}}
  \DeclareUnicodeCharacter{"2514}{\sphinxunichar{2514}}
  \DeclareUnicodeCharacter{"251C}{\sphinxunichar{251C}}
  \DeclareUnicodeCharacter{"2572}{\textbackslash}
 \else
  \DeclareUnicodeCharacter{00A0}{\nobreakspace}
  \DeclareUnicodeCharacter{2500}{\sphinxunichar{2500}}
  \DeclareUnicodeCharacter{2502}{\sphinxunichar{2502}}
  \DeclareUnicodeCharacter{2514}{\sphinxunichar{2514}}
  \DeclareUnicodeCharacter{251C}{\sphinxunichar{251C}}
  \DeclareUnicodeCharacter{2572}{\textbackslash}
 \fi
\fi
\usepackage{cmap}
\usepackage[T1]{fontenc}
\usepackage{amsmath,amssymb,amstext}
\usepackage{babel}
\usepackage{times}
\usepackage[Conny]{fncychap}
\usepackage[dontkeepoldnames]{sphinx}

\usepackage{geometry}

% Include hyperref last.
\usepackage{hyperref}
% Fix anchor placement for figures with captions.
\usepackage{hypcap}% it must be loaded after hyperref.
% Set up styles of URL: it should be placed after hyperref.
\urlstyle{same}

\addto\captionsenglish{\renewcommand{\figurename}{Figure}}
\addto\captionsenglish{\renewcommand{\tablename}{Table}}
\addto\captionsenglish{\renewcommand{\literalblockname}{Code}}

\addto\captionsenglish{\renewcommand{\literalblockcontinuedname}{continued from previous page}}
\addto\captionsenglish{\renewcommand{\literalblockcontinuesname}{continues on next page}}

\addto\extrasenglish{\def\pageautorefname{page}}

\setcounter{tocdepth}{1}

\usepackage{hyperref}

\title{CLM5 Documentation}
\date{Jan 16, 2018}
\release{}
\author{}
\newcommand{\sphinxlogo}{\vbox{}}
\renewcommand{\releasename}{Release}
\makeindex

\begin{document}

\maketitle
\sphinxtableofcontents
\phantomsection\label{\detokenize{index::doc}}


This document has two major sections.


\chapter{CLM User’s Guide}
\label{\detokenize{users_guide/index:welcome-to-the-clm-documentation}}\label{\detokenize{users_guide/index:users-guide}}\label{\detokenize{users_guide/index::doc}}\label{\detokenize{users_guide/index:clm-user-s-guide}}

\section{Overview}
\label{\detokenize{users_guide/overview/index:overview}}\label{\detokenize{users_guide/overview/index::doc}}\label{\detokenize{users_guide/overview/index:overview-section}}\phantomsection\label{\detokenize{users_guide/overview/introduction:introduction}}
\sphinxstylestrong{User’s Guide to version 5.0 of the Community Land Model (CLM)}

\sphinxstylestrong{Authors: Benjamin Andre, Erik Kluzek, William Sacks}

The National Center for Atmospheric Research (NCAR) is operated by the
nonprofit University Corporation for Atmospheric Research (UCAR) under
the sponsorship of the National Science Foundation. Any opinions,
findings, conclusions, or recommendations expressed in this publication
are those of the author(s) and do not necessarily reflect the views of
the National Science Foundation.

National Center for Atmospheric Research
P. O. Box 3000, Boulder, Colorado 80307-300


\subsection{Introduction}
\label{\detokenize{users_guide/overview/introduction:rst-users-guide-introduction}}\label{\detokenize{users_guide/overview/introduction::doc}}\label{\detokenize{users_guide/overview/introduction:id1}}
The Community Land Model (CLM5.0 in CESM2.0) is the latest in a
series of global land models developed by t he CESM Land Model Working
Group (LMWG) and maintained at the National Center for Atmospheric
Research (NCAR). This guide is intended to instruct both the novice
and experienced user on running CLM. This guide pertains to the latest
version CLM5.0 in CESM2.0 available for download from the public
release subversion repository as a part of CESM1.2.0. Documentation
may be different if you are using an older version, you should either
use the documentation for that release version, update to the latest
version, or use the documentation inside your own source tree. There
is information in the ChangeLog file and in the \sphinxhref{CLM-URL}{What is new with
CLM5.0 in CESM2.0 since previous public releases?}
regarding the changes from previous versions of CESM.

\begin{sphinxadmonition}{note}{Note:}
This release of CLM5.0 in CESM1.2.0 includes BOTH CLM4.0
physics used in previous releases as well as the updated CLM4.5
physics. Both CLM as well as CLM support tools allow you to trigger
between the two physics modes. Most often when we refer to CLM4.0 we
are referring to the CLM4.0 physics in CLM4.5 in CESM1.2.0 rather
than to a specific version of CLM4.0 (where we would give the exact
version). Likewise, when referring to CLM4.5 we are referring to the
CLM4.5 physics in CLM4.5 in CESM1.2.0.
\end{sphinxadmonition}

The novice user should read \sphinxhref{CLM-URL}{Chapter 1} in detail before
beginning work, while the expert user should read \sphinxhref{CLM-URL}{What is new with
CLM4.5 in CESM1.2.0 since previous public releases?} and
\sphinxhref{CLM-URL}{Quickstart to using CLM4.5} chapters, and then use the
more detailed chapters as reference. Before novice users go onto more
technical problems covered in \sphinxhref{CLM-URL}{Chapter 2}, \sphinxhref{CLM-URL}{Chapter 3}, \sphinxhref{CLM-URL}{Chapter 4}, or \sphinxhref{CLM-URL}{Chapter 5} they
should know the material covered in \sphinxhref{CLM-URL}{Chapter 1} and be able
to replicate some of the examples given there.

All users should read the \sphinxhref{CLM-URL}{How to Use This Document} and
\sphinxhref{CLM-URL}{Other resources to get help from} sections to understand
the document conventions and the various ways of getting help on using
CLM4.5. Users should also read the \sphinxhref{CLM-URL}{What is scientifically validated
and functional in CLM4.5 in CESM1.2.0?} section to see if
their planned use of the model is something that has been
scientifically validated and well tested. Users that are NOT using
NCAR machines or our list of well tested machines should also read the
What are the UNIX utilities required to use CLM4.5? section to make
sure they have all the required UNIX utilities on the system they want
to do their work.

Developers that are making changes to CLM either for their own
development or for development that they hope will eventually become a
part of the main CLM should read the \sphinxhref{CLM-URL}{Chapter 8}
chapter. We have a suite of test scripts that automatically test many
different model configurations and namelist options, as well as
ensuring things like restarts are bit-for-bit and the like. It’s
helpful to use these scripts to ensure your changes are working
correctly. As well as being a required part of the process to bring in
new code developments. And it’s far easier to use the automated
scripts rather than having to figure out, what to test, how to do it,
and then finally do it by hand. If you are using non supported
machines you may also want to use the test scripts to make sure your
machine is working correctly.


\subsection{What is New with CLM4.5}
\label{\detokenize{users_guide/overview/introduction:what-is-new-with-clm4-5}}\label{\detokenize{users_guide/overview/introduction:id3}}
The CESM1.2.0 \sphinxhref{CLM-URL}{What’s New Science} and \sphinxhref{CLM-URL}{What’s New Software} pages gives a synopsis of the changes to all CESM components since the CESM1.1.1 release.
More details are given in the \sphinxhref{CLM-URL}{CLM ChangeLog file}.

Previous release pages give similar list of changes for previous versions of the model.
The \sphinxhref{CLM-URL}{CLM4 in CESM1.0.5 User’s Guide} gives information on the updates for versions up to CLM4 in CESM1.0.5.


\subsection{Overview of User’s Guide}
\label{\detokenize{users_guide/overview/introduction:overview-of-user-s-guide}}\label{\detokenize{users_guide/overview/introduction:users-guide-overview}}
In this introduction we first give a simple guide to understand the document conventions in \sphinxhref{CLM-URL}{How to Use This Document}.
The next section \sphinxhref{CLM-URL}{What is new with CLM4.5 in CESM1.2.0 since previous public releases?} gives references to describe the differences between CLM4.5 in CESM1.2.0 and previous CESM releases both from a scientific as well as a software engineering point of view.
For information on previous releases of CLM4.5 before CLM4.5 in CESM1.2.0 see the CESM1.1.1 documentation.
The next section \sphinxhref{CLM-URL}{Quickstart to using CLM4.5} is for users that are already experts in using CLM and gives a quickstart guide to the bare details on how to use CLM4.5.
The next \sphinxhref{CLM-URL}{What is scientifically validated and functional in CLM4.5 in CESM1.2.0?} tells you about what has been extensively tested and scientifically validated (and maybe more importantly) what has NOT.
\sphinxhref{CLM-URL}{What are the UNIX utilities required to use CLM4.5?} lists the UNIX utilities required to use CLM4.5 and is important if you are running on non-NCAR machines, generic local machines, or machines NOT as well tested by us at NCAR.
Next we have \sphinxhref{CLM-URL}{Important Notes and Best Practices for Usage of CLM4.5} to detail some of the best practices for using CLM4.5 for science.
The last introductory section is \sphinxhref{CLM-URL}{Other resources} to get help from which lists different resources for getting help with CESM1.0 and CLM4.5.

\sphinxhref{CLM-URL}{Chapter 1} goes into detail on how to setup and run simulations with CLM4.5 and especially how to customize cases.
Details of cesm\_setup modes and build-namelist options as well as namelist options are given in this chapter.

\sphinxhref{CLM-URL}{Chapter 2} gives instructions on the CLM tools for either CLM4.0 or CLM4.5 physics for creating input datasets for use by CLM, for the expert user.
There’s an overview of what each tool does, and some general notes on how to build the FORTRAN tools.
Then each tool is described in detail along with different ways in which the tool might be used.
A final section on how to customize datasets for observational sites for very savvy expert users is given as the last section of this chapter.

As a followup to the tools chapter, \sphinxhref{CLM-URL}{Chapter 3} tells how to add files to the XML database for build-namelist to use.
This is important if you want to use the XML database to automatically select user-created input files that you have created when you setup new cases with CLM (both CLM4.0 and CLM4.5 physics).

In \sphinxhref{CLM-URL}{Chapter 4}, again for the expert user, we give details on how to do some particularly difficult special cases.
For example, we give the protocol for spinning up the CLM4.5-BGC and CLMCN models as well as CLM with dynamic vegetation active (CNDV).
We give instructions to do a spinup case from a previous case with Coupler history output for atmospheric forcing.
We also give instructions on running both the prognostic crop and irrigation models.
Lastly we tell the user how to use the DATM model to send historical CO2 data to CLM.

\sphinxhref{CLM-URL}{Chapter 5} outlines how to do single-point or regional simulations using CLM4.5.
This is useful to either compare CLM4.5 simulations with point observational stations, such as tower sites (which might include your own atmospheric forcing), or to do quick simulations with CLM for example to test a new parameterization.
There are several different ways given on how to perform single-point simulations which range from simple PTS\_MODE to more complex where you create all your own datasets, tying into \sphinxhref{CLM-URL}{Chapter 2} and also \sphinxhref{CLM-URL}{Chapter 3} to add the files into the build-namelist XML database.
The PTCLM python script to run single-point simulations was added back in for this release (but it has bugs that don’t allow it to work out of the box).
CLM4 in CESM1.0.5 has a fully working versions of PTCLM.

Need \sphinxhref{CLM-URL}{Chapter 6} blurb…

\sphinxhref{CLM-URL}{Chapter 7} gives some guidance on trouble-shooting problems when using CLM4.5.
It doesn’t cover all possible problems with CLM, but gives you some guidelines for things that can be done for some common problems.

\sphinxhref{CLM-URL}{Chapter 8}  goes over the automated testing scripts for validating that the CLM is working correctly.
The test scripts run many different configurations and options with CLM4.0 physics as well and CLM4.5 physics making sure that they work, as well as doing automated testing to verify restarts are working correctly, and testing at many different resolutions.
In general this is an activity important only for a developer of CLM4.5, but could also be used by users who are doing extensive code modifications and want to ensure that the model continues to work correctly.

In the appendices we talk about some issues that are useful for advanced users and developers of CLM4.5.

Finally in \sphinxhref{CLM-URL}{Appendix A} we give instructions on how to build the documentation associated with CLM4.5 (i.e. how to build this document).
This document is included in every CLM distribution and can be built so that you can view a local copy rather than having to go to the CESM website.
This also could be useful for developers who need to update the documentation due to changes they have made.


\subsection{Best Practices}
\label{\detokenize{users_guide/overview/introduction:best-practices}}\label{\detokenize{users_guide/overview/introduction:best-practices-for-usage}}\begin{itemize}
\item {} 
CLM4.5 includes BOTH the old CLM4.0 physics AND the new CLM4.5 physics and you can toggle between two.
The “standard” practice for CLM4.0 is to run with CN on, and with Qian atmospheric forcing.
While the “standard” practice for CLM4.5 is to run with BGC on, and CRUNCEP atmospheric forcing.
“BGC” is the new CLM4.5 biogeochemistry and include CENTURY-like pools, vertical resolved carbon, as well as Nitrification and de-Nitrification (see \sphinxhref{CLM-URL}{the Section called Some Acronym’s and Terms We’ll be Using in Other resources to get help from} ).

\item {} 
When running with CLMCN (either CLM4.0 or CLM4.5 physics) or CLM4.5-BGC, it is critical to begin with initial conditions that are provided with the release or to spin the model up following the CN spinup procedure before conducting scientific runs (see \sphinxhref{CLM-URL}{the Section called Spinning up the CLM4.5 biogeochemistry (CLMBGC spinup) in Chapter 4} for CLM4.5 or \sphinxhref{CLM-URL}{the Section called Spinning up the CLM4.0 biogeochemistry Carbon-Nitrogen Model (CN spinup) in Chapter 4} for CLM4.0).
Simulations without a proper spinup will effectively be starting from an unvegetated world.
See \sphinxhref{CLM-URL}{the Section called Setting Your Initial Conditions File in Chapter 1} for information on how to provide initial conditions for your simulation.

\item {} 
Initial condition files are provided for CLM4.0-CN as before, for fully coupled BCN and offline ICN cases for 1850 and 2000 at finite volume grids: 1deg (0.9x1.25), 2deg (1.9x2.5), and T31 resolutions.
We also have interpolated initial conditions for BCN for 1850 and 2000 for two finite volume grids: 10x15, 4x5 and two HOMME grids (ne30np4 and ne120np4).
There’s also an initial condition file for ICN with the prognostic crop model for 2000 at 2deg resolution, and one with CLMSP for 2000 at 2deg resolution.
We also have initial conditions for offline CNDV for 1850.
The 1850 initial condition files are in ‘reasonable’ equilibrium.
The 2000 initial condition files represent the model state for the year 2000, and have been taken from transient simulations.
Therefore, by design the year 2000 initial condition files do not represent an equilibrium state.
Note also that spinning the 2000 initial conditions out to equilibrium will not reflect the best estimate of the real carbon/nitrogen state for the year 2000.

\item {} 
Initial condition files are also provided for CLM4.5 for several configurations and resolutions.
For CLM4.5-SP and CLM4.5-BGC with CRUNCEP forcing we have initial conditions at 1deg resolution for both 1850 and 2000.
The CLM4.5-BGC initial conditions for 1850 (again with CRUNCEP forcing) were also interpolated to 2deg, CRUNCEP half degree (360x720cru), and ne30np4 resolutions.
Also the CLM4.5-BGC initial conditions for 1850 (with CRUNCEP forcing) were interpolated to 1deg CLM4.5-BGC-DV and 2deg CLM4.5-BGC-Crop.

\item {} 
Users can generate initial condition files at different resolutions by using the CLM tool interpinic to interpolate from one of the provided resolutions to the resolution of interest.
Interpolated initial condition files may no longer be in ‘reasonable’ equilibrium.

\item {} 
In CLM4.5 for both CLM4.5-CN and CLM4.5-BGC the new fire model requires lightning frequency data, and human population density (both are read inside of CLM).
By default we have provided a climatology dataset for lightning frequency and a dataset with coverage from 1850 to 2010 for population density.
Both of these datasets are interpolated from the native resolution of the datasets to the resolution you are running the model on.
If you are running with an atmosphere model or forcing that is significantly different than present day \textendash{} the lightning frequency may NOT appropriately correspond to your atmosphere forcing and fire initiation would be inappropriate.

\item {} 
Aerosol deposition is a required field to both CLM4.0 and CLM4.5 physics, sent from the atmosphere model.
Simulations without aerosol deposition will exhibit unreasonably high snow albedos.
The model sends aerosol deposition from the atmospheric model (either CAM or DATM).
When running with prescribed aerosol the atmosphere model will interpolate the aerosols from 2-degree resolution to the resolution the atmosphere model is running at.

\end{itemize}


\subsection{How To Use This Document}
\label{\detokenize{users_guide/overview/introduction:id16}}\label{\detokenize{users_guide/overview/introduction:id17}}
Links to descriptions and definitions have been provided in the code below. We use the same conventions used in the CESM documentation as outlined below.

\begin{sphinxVerbatim}[commandchars=\\\{\}]
Throughout the document this style is used to indicate shell
commands and options, fragments of code, namelist variables, etc.
Where examples from an interactive shell session are presented, lines
starting with \PYGZgt{} indicate the shell prompt.  A backslash \PYGZdq{}\PYGZbs{}\PYGZdq{} at the end
of a line means the line continues onto the next one (as it does in
standard UNIX shell).  Note that \PYGZdl{}EDITOR\PYGZdq{} is used to refer to the
text editor of your choice. \PYGZdl{}EDITOR is a standard UNIX environment
variable and should be set on most UNIX systems. Comment lines are
signaled with a \PYGZdq{}\PYGZsh{}\PYGZdq{} sign, which is the standard UNIX comment sign as well.
\PYGZdl{}CSMDATA is used to denote the path to the inputdata directory for
your CESM data.

\PYGZgt{} This is a shell prompt with commands \PYGZbs{}
that continues to the following line.
\PYGZgt{} \PYGZdl{}EDITOR filename \PYGZsh{} means you are using a text editor to edit \PYGZdq{}filename\PYGZdq{}
\PYGZsh{} This is a comment line
\end{sphinxVerbatim}


\subsection{Quickstart}
\label{\detokenize{users_guide/overview/quickstart::doc}}\label{\detokenize{users_guide/overview/quickstart:quickstart}}\label{\detokenize{users_guide/overview/quickstart:id1}}
Running the CLM requires a suite of UNIX utilities and programs and you should make sure you have all of these available before trying to go forward with using it.
If you are missing one of these you should contact the systems administrator for the machine you wish to run on and make sure they are installed.

List of utilities required for CESM in the “CESM1.2.0 Software/Operating System Prerequisites” section in \sphinxhref{CLM-URL}{http://www.cesm.ucar.edu/models/cesm1.2//cesm/doc/usersguide/book1.html}
- UNIX bash shell (for some of the CLM tools scripts)
- NCL (for some of the offline tools for creating/modifying CLM input datasets see \sphinxhref{CLM-URL}{Chapter 2} for more information on NCL)
- Python (optional, needed for PTCLM)
- xsltproc, docbook and docbook utilities (optional, needed to build the Users-Guide)

Before working with CLM4.5 read the QuickStart Guide in the \sphinxhref{CLM-URL}{CESM1.2.0 Scripts User’s Guide}. Once you are familiar with how to setup cases for any type of simulation with CESM you will want to direct your attention to the specifics of using CLM.

For some of the details of setting up cases for CLM4.5 read the README and text files available from the “models/lnd/clm/doc” directory (see the “CLM Web pages” section for a link to the list of these files). Here are the important ones that you should be familiar with.
\begin{enumerate}
\item {} 
\sphinxhref{CLM-URL}{README file} describing the directory structure.

\item {} 
\sphinxhref{CLM-URL}{Quickstart.userdatasets} file describing how to use your own datasets in the model (also see \sphinxhref{CLM-URL}{the Section called Creating your own single-point/regional surface datasets in Chapter 5}).

\item {} 
\sphinxhref{CLM-URL}{models/lnd/clm/doc/KnownBugs} file describing known problems in CLM4.5 (that we expect to eventually fix).

\item {} 
\sphinxhref{CLM-URL}{models/lnd/clm/doc/KnownLimitationss} file describing known limitations in CLM4.5 and workarounds that we do NOT expect to fix.

\end{enumerate}

The IMPORTANT\_NOTES file talks about important things for users to know about using the model scientifically. It content is given in the next chapter on \sphinxhref{CLM-URL}{“What is scientifically validated and functional in CLM4.5 in CESM1.2.0?”}.

The ChangeLog/ChangeSum talk about advances in different versions of CLM. The content of these files is largely explained in the previous chapter on \sphinxhref{CLM-URL}{“What is new with CLM4.5 in CESM1.2.0 since previous public releases?”}.

Note other directories have README files that explain different components and tools used when running CLM and are useful in understanding how those parts of the model work and should be consulted when using tools in those directories. For more details on configuring and customizing a case with CLM see \sphinxhref{CLM-URL}{Chapter 1}.

The Quickstart.GUIDE (which can be found in \sphinxcode{models/lnd/clm/doc}) is repeated here.

\begin{sphinxVerbatim}[commandchars=\\\{\}]
       Quick\PYGZhy{}Start to Using cpl7 Scripts for clm4\PYGZus{}5

       Assumptions: You want to use yellowstone with clm4\PYGZus{}5 BGC
           to do a clm simulation with data atmosphere and the
           latest CRUNCEP atm forcing files and settings. You also want to cycle
           the CRUNCEP atm data between 1901 to 1920 and you want to run at
           0.9x1.25 degree resolution.

       Process:

       \PYGZsh{} Create the case

       cd scripts

       ./create\PYGZus{}newcase \PYGZhy{}case \PYGZlt{}testcase\PYGZgt{} \PYGZhy{}mach yellowstone\PYGZus{}intel \PYGZhy{}res f09\PYGZus{}g16 \PYGZhy{}compset  I1850CRUCLM45BGC
       (./create\PYGZus{}newcase \PYGZhy{}help \PYGZhy{}\PYGZhy{} to get help on the script)

       \PYGZsh{} Setup the case

cd \PYGZlt{}testcase\PYGZgt{}
./xmlchange id1=val1,id2=val2  \PYGZsh{} to make changes to any settings in the env\PYGZus{}*.xml files
./cesm\PYGZus{}setup
(./cesm\PYGZus{}setup \PYGZhy{}help \PYGZhy{}\PYGZhy{} to get help on the script, this creates the ./\PYGZlt{}testcase\PYGZgt{}.run  \PYGZbs{}
script)

\PYGZsh{} Add any namelist changes to the user\PYGZus{}nl\PYGZus{}* files

\PYGZdl{}EDITOR user\PYGZus{}nl\PYGZus{}*

\PYGZsh{} Compile the code

./\PYGZlt{}testcase\PYGZgt{}.build

\PYGZsh{} Submit the run

./\PYGZlt{}testcase\PYGZgt{}.submit
\end{sphinxVerbatim}

Information on Compsets:
\begin{quote}

“I” compsets are the ones with clm and datm7 without ice and ocean. They
specify either CLM4.0 physics or CLM4.5 physics.
Most of the “I” compsets for CLM4.0 use the CLM\_QIAN data with solar following
the cosine of solar zenith angle, precipitation constant, and other
variables linear interpolated in time (and with appropriate time-stamps on
the date).  Useful “I” compsets for CLM4.5 use the CRUNCEP data in place
of the CLM\_QIAN data.

To list all the compsets use:
./create\_newcase -list compsets

Some of the CLM4.5 I compsets are:

Alias               Description
1850CRUCLM45        CLM4.5 to simulate year=1850 with CLMN45SP (Satellite Phenology)
I1850CRUCLM45BGC    CLM4.5 to simulate year=1850 with CLM45BGC biogeophysics model (BGC)
I20TRCRUCLM45BGC    CLM4.5 with BGC on with transient PFT over 1850-2000

While some of the CLM4 I compsets are:

Alias               Description
ICN                 CLM4.0 to simulate year=2000 with Carbon-Nitrogen BGC model (CN)
I1850CN             CLM4.0 to simulate year=1850 with Carbon-Nitrogen BGC model (CN)
I20TRCN             CLM4.0 with CN on with transient PFT over 1850-2000
IRCP26CN            CLM4.0 with CN on with transient PFT over 1850-2100 for RCP=2.6 scenario
IRCP45CN            CLM4.0 with CN on with transient PFT over 1850-2100 for RCP=4.5 scenario
IRCP60CN            CLM4.0 with CN on with transient PFT over 1850-2100 for RCP=6.0 scenario
IRCP85CN            CLM4.0 with CN on with transient PFT over 1850-2100 for RCP=8.5 scenario
\end{quote}

Automatically resubmitting jobs:
\begin{quote}

After doing a short simulation that you believe is correct

./xmlchange CONTINUE\_RUN=TRUE

\# Change RESUBMIT to number greater than 0, and CONTINUE\_RUN to TRUE…

./\textless{}testcase\textgreater{}.submit
\end{quote}


\subsection{Scientific Validation}
\label{\detokenize{users_guide/overview/scientific_validation:scientific-validiation}}\label{\detokenize{users_guide/overview/scientific_validation:scientific-validation}}\label{\detokenize{users_guide/overview/scientific_validation::doc}}
In this section we go over what has been extensively tested and scientifically validated with CLM4.5, and maybe more importantly what has NOT been tested and may NOT be scientifically validated. You can use all features of CLM, but need to realize that some things haven’t been tested extensively or validated scientifically. When you use these features you may run into trouble doing so, and will need to do your own work to make sure the science is reasonable.


\subsubsection{Standard Configuration and Namelist Options that are Validated}
\label{\detokenize{users_guide/overview/scientific_validation:standard-configuration-and-namelist-options-that-are-validated}}
See
\sphinxurl{http://www.cesm.ucar.edu/models/cesm1.2/clm/CLM\_configurations\_CESM1.2.pdf} for an explanation of what configurations are scientifically validated for CLM4.5. For CLM4.0 changes to the science of the model are minimal since CESM1.1.1 so we expect answers to be very similar to using it.

In the sections below we go through configuration and/or namelist options or modes that the user should be especially wary of using. You are of course free to use these options, and you may find that they work functionally. Although in some cases you will find issues even with functionality of using them. If so you will need to test, debug and find solutions for these issues on your own. But in every case you will need to go through more extensive work to validate these options from a scientific standpoint. Some of these options are only for CLM4.5 while others are for both CLM4.0 AND CLM4.5 we explicitly say which they apply to.


\subsubsection{Configure Modes NOT scientifically validated, documented, supported or, in some cases, even advised to be used:}
\label{\detokenize{users_guide/overview/scientific_validation:configure-modes-not-scientifically-validated-documented-supported-or-in-some-cases-even-advised-to-be-used}}
These are options that you would add to \sphinxcode{CLM\_CONFIG\_OPTS}.
\begin{enumerate}
\item {} 
exlaklayers on{[}CLM4.5 only{]} This mode is NOT tested and may NOT be even functional.

\item {} 
snicar\_frc on{[}CLM4.0 AND CLM4.5{]} This mode is tested and functional, but is NOT constantly scientifically validated, and should be considered experimental.

\item {} 
vichydro on{[}CLM4.5 only{]} This mode is tested and functional, but does NOT have long scientific validation simulations run with it so, should be considered experimental.

\item {} 
vsoilc\_centbgc{[}CLM4.5 only{]} This option is extensively tested for both “on” and “off”. The “no-vert” option has limited testing performed on it, but isn’t scientifically validated (and it currently has a bug \textendash{} see 1746 and 1672 in \sphinxhref{CLM-URL}{models/lnd/clm/doc/KnownBugs}). The “no-cent” and “no-nitrif” options are NOT tested and as such may NOT ben even functional.

\end{enumerate}


\subsubsection{Namelist options that should NOT be exercised:}
\label{\detokenize{users_guide/overview/scientific_validation:namelist-options-that-should-not-be-exercised}}

\subsubsection{Build-Namelist options that should NOT be exercised:}
\label{\detokenize{users_guide/overview/scientific_validation:build-namelist-options-that-should-not-be-exercised}}
1. -irrig with -bgc cn and -phys clm4\_0 We have only run the irrigation model with CLMSP (i.e. without the CN model). We recommend that if you want to run the irrigation model with CN, that you do a spinup. But, more than that you may need to make adjustments to irrig\_factor in models/lnd/clm/src/biogeophys/CanopyFluxesMod.F90. See the notes on this in the description of the irrigation model in the
\sphinxhref{CLM-URL}{Technical Descriptions of the Interactive Crop Management and Interactive Irrigation Models}.
\begin{enumerate}
\setcounter{enumi}{1}
\item {} 
-irrig with -crop on and -phys clm4\_0 Irrigation doesn’t work with the prognostic crop model. Irrigation is only applied to generic crop currently, which negates it’s practical usage. We also have a known problem when both are on (see bug 1326 in the \sphinxhref{CLM-URL}{models/lnd/clm/doc/KnownBugs} file). If you try to run in this mode, the CLM build-namelist will return with an error.

\end{enumerate}


\subsubsection{Namelist items that should NOT be exercised:}
\label{\detokenize{users_guide/overview/scientific_validation:namelist-items-that-should-not-be-exercised}}
suplnitro=’ALL’ The suplnitro namelist option to the CN Biogeochemistry model supplies unlimited nitrogen and therefore vegetation is over-productive in this mode.

urban\_traffic:Not currently functional

allowlakeprod:Considered experimental.

anoxia\_wtsat:Considered experimental (deprecated will be removed).

atm\_c14\_filename:Considered experimental (dataset not provided).

exponential\_rooting\_profile:Considered experimental.

fin\_use\_fsat:Considered experimental.

glc\_dyntopo:Not currently functional.

lake\_decomp\_fact:Considered experimental.

more\_vertlayers:Considered experimental.

no\_frozen\_nitrif\_denitrif:Considered experimental.

perchroot:Considered experimental.

perchroot\_alt:Considered experimental.

replenishlakec:Considered experimental.

use\_c14\_bombspike:Considered experimental (dataset not provided).

usefrootc:Considered experimental.


\subsection{Getting Help}
\label{\detokenize{users_guide/overview/getting-help:getting-help}}\label{\detokenize{users_guide/overview/getting-help::doc}}\label{\detokenize{users_guide/overview/getting-help:id1}}
In addition to this users-guide there are several other resources that are available to help you use CLM5.0. The first one is the CESM1.2.0 User’s-Guide, which documents the entire process of creating cases with CESM1.2.0. The next is the CESM bulletin board which is a web-site for exchanging information between users of CESM. There are also CLM web-pages specific for CLM, and finally there is an email address to report bugs that you find in CESM1.2.0.


\subsubsection{The CESM User’s-Guide}
\label{\detokenize{users_guide/overview/getting-help:the-cesm-user-s-guide}}
CLM5.0 in CESM2.0 is always run from within the standard CESM2.0 build and run scripts. Therefore, the user of CLM4.5 should familiarize themselves with the CESM1.2.0 scripts and understand how to work with them. User’s-Guide documentation on the CESM1.2.0 scripts are available from the following web-page. The purpose of this CLM4.5 in CESM1.2.0 User’s Guide is to give the CLM4.5 user more complete details on how to work with CLM and the set of tools that support CLM, as well as to give examples that are unique to the use of CLM. However, the CESM1.2.0 Scripts User’s-Guide remains the primary source to get detailed information on how to build and run the CESM system.

\sphinxhref{CLM-URL}{cesmrel; Scripts User’s-Guide (http://www.cesm.ucar.edu/models/cesm1.2/cesm/doc/usersguide/book1.html)}


\subsubsection{The CESM Bulletin Board}
\label{\detokenize{users_guide/overview/getting-help:the-cesm-bulletin-board}}
There is a rich and diverse set of people that use the CESM, and often it is useful to be in contact with others to get help in solving problems or trying something new. To facilitate this we have an online Bulletin Board for questions on the CESM. There are also different sections in the Bulletin Board for the different component models or for different topics.

\sphinxhref{http://bb.cgd.ucar.edu/}{CESM Online Bulletin Board}


\subsubsection{The CLM web pages}
\label{\detokenize{users_guide/overview/getting-help:the-clm-web-pages}}
The main CLM web page contains information on the CLM, it’s history, developers, as well as downloads for previous model versions. There are also documentation text files in the models/lnd/clm/doc directory that give some quick information on using CLM.

\sphinxhref{http://www.cgd.ucar.edu/tss/clm/}{CLM web page}
\sphinxhref{CLM-URL}{CLM Documentation Text Files}

Also note that several of the XML database files can be viewed in a web browser to get a nice table of namelist options, namelist defaults, or compsets. Simply view them as a local file and bring up one of the following files:
\begin{itemize}
\item {} 
\sphinxhref{CLM-URL}{models/lnd/clm/bld/namelist\_files/namelist\_definition\_clm4\_0.xml} \textendash{} definition of CLM4.0 namelist items.

\item {} 
\sphinxhref{CLM-URL}{models/lnd/clm/bld/namelist\_files/namelist\_definition\_clm4\_5.xml} \textendash{} definition of CLM4.0 namelist items.

\item {} 
\sphinxhref{CLM-URL}{models/lnd/clm/bld/namelist\_files/namelist\_defaults\_clm4\_0.xml} \textendash{} default values for CLM4.0 namelist items.

\item {} 
\sphinxhref{CLM-URL}{models/lnd/clm/bld/namelist\_files/namelist\_defaults\_clm4\_5.xml} \textendash{} default values for CLM4.5 namelist items.

\item {} 
\sphinxhref{CLM-URL}{scripts/ccsm\_utils/Case.template/config\_definition.xml} \textendash{} definition of all env\_*.xml items.

\item {} 
\sphinxhref{CLM-URL}{scripts/ccsm\_utils/Case.template/config\_compsets.xml} \textendash{} definition of all the compsets.

\item {} 
\sphinxhref{CLM-URL}{models/lnd/clm/bld/namelist\_files/history\_fields\_clm4\_0.xml} \textendash{} definition of CLM4.0 history fields.

\item {} 
\sphinxhref{CLM-URL}{models/lnd/clm/bld/namelist\_files/history\_fields\_clm4\_5.xml} \textendash{} definition of CLM4.5 history fields.

\end{itemize}


\subsubsection{Reporting bugs in CLM4.5}
\label{\detokenize{users_guide/overview/getting-help:reporting-bugs-in-clm4-5}}
If you have any problems, additional questions, bug reports, or any other feedback, please send an email to \textless{}\sphinxhref{mailto:cesmhelp@cgd.ucar.edu}{cesmhelp@cgd.ucar.edu}\textgreater{}. If you find bad, wrong, or misleading information in this users guide send an email to \textless{}\sphinxhref{mailto:erik@ucar.edu}{erik@ucar.edu}\textgreater{}. The current list of known issues for CLM4.5 in CESM1.2.0 is in the models/lnd/clm/doc/KnownBugs file, and the list of issues for CESM1.2.0 is at…
\sphinxurl{http://www.cesm.ucar.edu/models/cesm1.2//tags/cesm1\_2\_0/\#PROBLEMS}.


\subsubsection{Some Acronym’s and Terms We’ll be Using}
\label{\detokenize{users_guide/overview/getting-help:some-acronym-s-and-terms-we-ll-be-using}}\begin{description}
\item[{CAM}] \leavevmode
Community Atmosphere Model (CAM). The prognostically active atmosphere model component of CESM.

\item[{CESM}] \leavevmode
Community Earth System Model (CESM). The coupled earth system model that CLM is a component of.

\item[{CLM}] \leavevmode
Community Land Model (CLM). The prognostically active land model component of CESM.

\item[{CLMBGC}] \leavevmode
Community Land Model (CLM4.5) with BGC Biogeochemistry. Uses CN Biogeochemistry with vertically resolved soil Carbon, CENTURY model like pools, and Nitrification/De-Nitrification. The CLM\_CONFIG\_OPTS option for this is

\sphinxcode{./xmlchange CLM\_CONFIG\_OPTS="phys clm4\_5 -bgc cn -vsoilc\_centbgc on -clm4me on"}

\item[{CLMCN}] \leavevmode
Community Land Model (CLM) with Carbon Nitrogen (CN) Biogeochemistry (either CLM4.0 or CLM4.5) The CLM\_CONFIG\_OPTS option for this is

\sphinxcode{./xmlchange CLM\_CONFIG\_OPTS="-bgc cn" -append}

\item[{CLMSP}] \leavevmode
Community Land Model (CLM) with Satellite Phenology (SP) (either CLM4.0 or CLM4.5) The CLM\_CONFIG\_OPTS option for this is

\sphinxcode{./xmlchange CLM\_CONFIG\_OPTS="-bgc none" -append}

\item[{CLMU}] \leavevmode
Community Land Model (CLM) Urban Model (either CLM4.0 or CLM4.5). The urban model component of CLM is ALWAYS active (unless you create special surface datasets that have zero urban percent, or for regional/single-point simulations for a non-urban area).

\item[{CRUNCEP}] \leavevmode
The Climate Research Unit (CRU) analysis of the NCEP atmosphere reanalysis atmosphere forcing data. This can be used to drive CLM with atmosphere forcing from 1901 to 2010. We also DO expect to be able to update this dataset beyond 2010 as newer data becomes available.

\item[{DATM}] \leavevmode
Data Atmosphere Model (DATM) the prescribed data atmosphere component for CESM. Forcing data that we provide are either the Qian or CRUNCEP forcing datasets (see below).

\item[{DV}] \leavevmode
Dynamic global vegetation, where fractional PFT (see PFT below) changes in time prognostically. Can NOT be used with prescribed transient PFT (requires either CLM4.5-BGC or CLMCN for either CLM4.0 or CLM4.5). The CLM\_CONFIG\_OPTS option for this is

\sphinxcode{./xmlchange CLM\_CONFIG\_OPTS="-bgc cndv" -append}

\item[{ESMF}] \leavevmode
Earth System Modeling Framework (ESMF). They are a software project that provides a software library to support Earth System modeling. We provide interfaces for ESMF as well as use their regridding capabilities for offline CLM tools.

\item[{NCAR}] \leavevmode
National Center for Atmospheric Research (NCAR). This is the research facility that maintains CLM with contributions from other national labs and Universities.

\item[{NCEP}] \leavevmode
The National Center for Environmental Prediction (NCEP). In this document this normally refers to the reanalysis atmosphere data produced by NCEP.

\item[{PFT}] \leavevmode
Plant Function Type (PFT). A type of vegetation that CLM parameterizes.

\item[{PTCLM}] \leavevmode
PoinT CLM (PTCLM) a python script that operates on top of CLM for CLM4.5 to run single point simulations for CLM.

\item[{Qian}] \leavevmode
The Qian et. al. analysis of the NCEP forcing data. This can be used to drive CLM with atmosphere forcing from 1948 to 2004. We do NOT expect to be able to update this dataset beyond 2004.

\item[{SCRIP}] \leavevmode
Spherical Coordinate Remapping and Interpolation Package (SCRIP). We use it’s file format for specifying both grid coordinates as well as mapping between different grids.

\item[{VIC}] \leavevmode
Variable Infiltration Capacity (VIC) model for hydrology. This is an option to CLM4.5 in place of the standard CLM4.5 hydrology. The CLM\_CONFIG\_OPTS option for this is

\sphinxcode{./xmlchange CLM\_CONFIG\_OPTS="-vichydro on" -append}

\end{description}


\section{Setting Up and Running a Case}
\label{\detokenize{users_guide/setting-up-and-running-a-case/index:customizing-section}}\label{\detokenize{users_guide/setting-up-and-running-a-case/index:setting-up-and-running-a-case}}\label{\detokenize{users_guide/setting-up-and-running-a-case/index::doc}}

\subsection{Choosing a compset}
\label{\detokenize{users_guide/setting-up-and-running-a-case/choosing-a-compset:choosing-a-compset}}\label{\detokenize{users_guide/setting-up-and-running-a-case/choosing-a-compset::doc}}\label{\detokenize{users_guide/setting-up-and-running-a-case/choosing-a-compset:id1}}
When setting up a new case one of the first choices to make is which “component set” (or compset) to use.
The compset refers to which component models are used as well as specific settings for them.
We label the different types of compsets with a different letter of the alphabet from “A” (for all data model) to “X” (for all dead model).
The compsets of interest when working with CLM are the “I” compsets (which contain CLM with a data atmosphere model and a stub ocean, and stub sea-ice models), “E” and “F” compsets (which contain CLM with the active atmosphere model (CAM), prescribed sea-ice model, and a data ocean model), and “B” compsets which have all active components.
Below we go into details on the “I” compsets which emphasize CLM as the only active model, and just mention the two other categories.

When working with CLM you usually want to start with a relevant “I” compset before moving to the more complex cases that involve other active model components.
The “I” compsets can exercise CLM in a way that is similar to the coupled modes, but with much lower computational cost and faster turnaround times.


\subsubsection{Compsets coupled to data atmosphere and stub ocean/sea-ice (“I” compsets)}
\label{\detokenize{users_guide/setting-up-and-running-a-case/choosing-a-compset:compsets-coupled-to-data-atmosphere-and-stub-ocean-sea-ice-i-compsets}}
\sphinxhref{CLM-URL}{Supported CLM Configurations} are listed in \sphinxhref{CLM-1.1-Choosing-a-compset-using-CLM\#table-1-1-scientifically-supported-i-compsets}{Table 1-1} for the Scientifically Supported compsets (have been scientifically validated with long simulations) and in \sphinxhref{CLM-1.1-Choosing-a-compset-using-CLM\#table-1-2-functionally-supported-i-compsets}{Table 1-2} for the Functionally Supported compsets (we’ve only checked that they function).

\sphinxstylestrong{Scientifically Supported I Compsets:}


\begin{savenotes}\sphinxattablestart
\centering
\begin{tabulary}{\linewidth}[t]{|T|T|T|T|T|}
\hline
\sphinxstylethead{\sphinxstyletheadfamily 
Short Name
\unskip}\relax &\sphinxstylethead{\sphinxstyletheadfamily 
Description
\unskip}\relax &\sphinxstylethead{\sphinxstyletheadfamily 
Atm.
Forcing
\unskip}\relax &\sphinxstylethead{\sphinxstyletheadfamily 
Compset Alias
Name
\unskip}\relax &\sphinxstylethead{\sphinxstyletheadfamily 
Period
\unskip}\relax \\
\hline
CLM4.5SP
&
Satellite
phenology
with new
biogeophys
&
CRUNCEP
&
1850CRUCLM45
&
1850
\\
\hline\sphinxmultirow{2}{11}{%
\begin{varwidth}[t]{\sphinxcolwidth{1}{5}}
CLM4.5SP
\par
\vskip-\baselineskip\strut\end{varwidth}%
}%
&\sphinxmultirow{2}{12}{%
\begin{varwidth}[t]{\sphinxcolwidth{1}{5}}
New
biogeophys
+ CENTURY-
like
vertically
resolved
soil
BGC + CH4
emissions,
nitrogen
updates
\par
\vskip-\baselineskip\strut\end{varwidth}%
}%
&\sphinxmultirow{2}{13}{%
\begin{varwidth}[t]{\sphinxcolwidth{1}{5}}
CRUNCEP
\par
\vskip-\baselineskip\strut\end{varwidth}%
}%
&
I1850CRUCLM45BGC
&
1850
\\
\cline{4-5}\sphinxtablestrut{11}&\sphinxtablestrut{12}&\sphinxtablestrut{13}&
I20TRCRUCLM45BGC
&
20th Century
\\
\hline
CLM4.5CN
&
New
biogeophys
+ CN soil
BGC, updates
&
CRUNCEP
&
I1850CRUCLM45CN
&
1850
\\
\hline\sphinxmultirow{3}{23}{%
\begin{varwidth}[t]{\sphinxcolwidth{1}{5}}
CLM4SP
\par
\vskip-\baselineskip\strut\end{varwidth}%
}%
&\sphinxmultirow{3}{24}{%
\begin{varwidth}[t]{\sphinxcolwidth{1}{5}}
As in
CCSM4/CESM1
release
\par
\vskip-\baselineskip\strut\end{varwidth}%
}%
&\sphinxmultirow{3}{25}{%
\begin{varwidth}[t]{\sphinxcolwidth{1}{5}}
Qian
\par
\vskip-\baselineskip\strut\end{varwidth}%
}%
&
I1850
&
1850
\\
\cline{4-5}\sphinxtablestrut{23}&\sphinxtablestrut{24}&\sphinxtablestrut{25}&
I
&
2000
\\
\cline{4-5}\sphinxtablestrut{23}&\sphinxtablestrut{24}&\sphinxtablestrut{25}&
I20TR
&
20th Century
\\
\hline\sphinxmultirow{7}{32}{%
\begin{varwidth}[t]{\sphinxcolwidth{1}{5}}
CLM4CN
\par
\vskip-\baselineskip\strut\end{varwidth}%
}%
&\sphinxmultirow{7}{33}{%
\begin{varwidth}[t]{\sphinxcolwidth{1}{5}}
As in
CCSM4/CESM1
release
\par
\vskip-\baselineskip\strut\end{varwidth}%
}%
&\sphinxmultirow{7}{34}{%
\begin{varwidth}[t]{\sphinxcolwidth{1}{5}}
Qian
\par
\vskip-\baselineskip\strut\end{varwidth}%
}%
&
I1850CN
&
1850
\\
\cline{4-5}\sphinxtablestrut{32}&\sphinxtablestrut{33}&\sphinxtablestrut{34}&
ICN
&
2000
\\
\cline{4-5}\sphinxtablestrut{32}&\sphinxtablestrut{33}&\sphinxtablestrut{34}&
I20TRCN
&
20th Century
\\
\cline{4-5}\sphinxtablestrut{32}&\sphinxtablestrut{33}&\sphinxtablestrut{34}&
IRCP26CN
&
RCP 2.6 to 2100
\\
\cline{4-5}\sphinxtablestrut{32}&\sphinxtablestrut{33}&\sphinxtablestrut{34}&
IRCP45CN
&
RCP 4.5 to 2100
\\
\cline{4-5}\sphinxtablestrut{32}&\sphinxtablestrut{33}&\sphinxtablestrut{34}&
IRCP60CN
&
RCP 6.0 to 2100
\\
\cline{4-5}\sphinxtablestrut{32}&\sphinxtablestrut{33}&\sphinxtablestrut{34}&
IRCP85CN
&
RCP 8.5 to 2100
\\
\hline
\end{tabulary}
\par
\sphinxattableend\end{savenotes}

\sphinxstylestrong{Functionally Supported I Compsets:}


\begin{savenotes}\sphinxattablestart
\centering
\begin{tabulary}{\linewidth}[t]{|T|T|T|T|T|}
\hline
\sphinxstylethead{\sphinxstyletheadfamily 
Short Name
\unskip}\relax &\sphinxstylethead{\sphinxstyletheadfamily 
Description
\unskip}\relax &\sphinxstylethead{\sphinxstyletheadfamily 
Atm.
Forcing
\unskip}\relax &\sphinxstylethead{\sphinxstyletheadfamily 
Compset Alias
Name
\unskip}\relax &\sphinxstylethead{\sphinxstyletheadfamily 
Period
\unskip}\relax \\
\hline
CLM4.5BGC-
CROP
&
ICRUCLM45BGCCROP
&
New biogeophys +
CENTURY-like
vertically
resolved soil
BGC + CH4
emissions,
nitrogen updates
with prognostic
CROP
&
CRUNCEP
&
2000
\\
\hline
CLM4.5BGC-
DV
&
I1850CRUCLM45BGCDV
&
New biogeophys
+ CENTURY-like
vertically
resolved soil
BGC + CH4
emissions,
nitrogen updates
with DV
&
CRUNCEP
&
1850
\\
\hline
CLM4.5SP-
VIC
&
ICLM45VIC
&
Satellite
phenology with new
biogeophys with
VIC hydrology
&
Qian
&
2000
\\
\hline
CLM4CN-CROP
&
ICNCROP
&
As in CCSM4/CESM1
release
&
Qian
&
2000
\\
\hline
CLM4CN-DV
&
ICNDV
&
As in CCSM4/CESM1
release
&
Qian
&
1850
\\
\hline
\end{tabulary}
\par
\sphinxattableend\end{savenotes}

Here is the entire list of compsets available.
Note that using the “-user\_compset” option even more combinations are possible.
In the list below we give the alias name and then the long name which describes each component in parenthesis.
Alias (Long-name with time-period and each component)
\begin{enumerate}
\item {} 
\sphinxcode{I} (2000\_DATM\%QIA\_CLM40\%SP\_SICE\_SOCN\_RTM\_SGLC\_SWAV)

\item {} 
\sphinxcode{I1850} (1850\_DATM\%QIA\_CLM40\%SP\_SICE\_SOCN\_RTM\_SGLC\_SWAV)

\item {} 
\sphinxcode{I1850CLM45} (1850\_DATM\%QIA\_CLM45\%SP\_SICE\_SOCN\_RTM\_SGLC\_SWAV)

\item {} 
\sphinxcode{I1850CLM45BGC} (1850\_DATM\%QIA\_CLM45\%BGC\_SICE\_SOCN\_RTM\_SGLC\_SWAV)

\item {} 
\sphinxcode{I1850CLM45CN} (1850\_DATM\%QIA\_CLM45\%CN\_SICE\_SOCN\_RTM\_SGLC\_SWAV)

\item {} 
\sphinxcode{I1850CLM45CNF} (1850\_DATM\%QIA\_CLM45\%CN\_SICE\_SOCN\_RTM\%FLOOD\_SGLC\_SWAV)

\item {} 
\sphinxcode{I1850CN} (1850\_DATM\%QIA\_CLM40\%CN\_SICE\_SOCN\_RTM\_SGLC\_SWAV)

\item {} 
\sphinxcode{I1850CRU} (1850\_DATM\%CRU\_CLM40\%SP\_SICE\_SOCN\_RTM\_SGLC\_SWAV)

\item {} 
\sphinxcode{I1850CRUCLM45} (1850\_DATM\%CRU\_CLM45\%SP\_SICE\_SOCN\_RTM\_SGLC\_SWAV)

\item {} 
\sphinxcode{I1850CRUCLM45BGC} (1850\_DATM\%CRU\_CLM45\%BGC\_SICE\_SOCN\_RTM\_SGLC\_SWAV)

\item {} 
\sphinxcode{I1850CRUCLM45BGCDV} (1850\_DATM\%CRU\_CLM45\%BGCDV\_SICE\_SOCN\_RTM\_SGLC\_SWAV)

\item {} 
\sphinxcode{I1850CRUCLM45CN} (1850\_DATM\%CRU\_CLM45\%CN\_SICE\_SOCN\_RTM\_SGLC\_SWAV)

\item {} 
\sphinxcode{I1850CRUCN} (1850\_DATM\%CRU\_CLM40\%CN\_SICE\_SOCN\_RTM\_SGLC\_SWAV)

\item {} 
\sphinxcode{I1850SPINUPCLM45BGC} (1850\_DATM\%S1850\_CLM45\%BGC\_SICE\_SOCN\_RTM\_SGLC\_SWAV)

\item {} 
\sphinxcode{I1850SPINUPCN} (1850\_DATM\%S1850\_CLM40\%CN\_SICE\_SOCN\_RTM\_SGLC\_SWAV)

\item {} 
\sphinxcode{I1PT} (2000\_DATM\%1PT\_CLM40\%SP\_SICE\_SOCN\_RTM\_SGLC\_SWAV)

\item {} 
\sphinxcode{I1PTCLM45} (2000\_DATM\%1PT\_CLM45\%SP\_SICE\_SOCN\_RTM\_SGLC\_SWAV)

\item {} 
\sphinxcode{I20TR} (20TR\_DATM\%QIA\_CLM40\%SP\_SICE\_SOCN\_RTM\_SGLC\_SWAV)

\item {} 
\sphinxcode{I20TRCLM45} (20TR\_DATM\%QIA\_CLM45\%SP\_SICE\_SOCN\_RTM\_SGLC\_SWAV)

\item {} 
\sphinxcode{I20TRCLM45CN} (20TR\_DATM\%QIA\_CLM45\%CN\_SICE\_SOCN\_RTM\_SGLC\_SWAV)

\item {} 
\sphinxcode{I20TRCN} (20TR\_DATM\%QIA\_CLM40\%CN\_SICE\_SOCN\_RTM\_SGLC\_SWAV)

\item {} 
\sphinxcode{I20TRCRU} (20TR\_DATM\%CRU\_CLM40\%SP\_SICE\_SOCN\_RTM\_SGLC\_SWAV)

\item {} 
\sphinxcode{I20TRCRUCLM45} (20TR\_DATM\%CRU\_CLM45\%SP\_SICE\_SOCN\_RTM\_SGLC\_SWAV)

\item {} 
\sphinxcode{I20TRCRUCLM45BGC} (20TR\_DATM\%CRU\_CLM45\%BGC\_SICE\_SOCN\_RTM\_SGLC\_SWAV)

\item {} 
\sphinxcode{I20TRCRUCLM45CN} (20TR\_DATM\%CRU\_CLM45\%CN\_SICE\_SOCN\_RTM\_SGLC\_SWAV)

\item {} 
\sphinxcode{I20TRCRUCN} (20TR\_DATM\%CRU\_CLM40\%CN\_SICE\_SOCN\_RTM\_SGLC\_SWAV)

\item {} 
\sphinxcode{I4804} (4804\_DATM\%QIA\_CLM40\%SP\_SICE\_SOCN\_RTM\_SGLC\_SWAV)

\item {} 
\sphinxcode{I4804CLM45} (4804\_DATM\%QIA\_CLM45\%SP\_SICE\_SOCN\_RTM\_SGLC\_SWAV)

\item {} 
\sphinxcode{I4804CLM45CN} (4804\_DATM\%QIA\_CLM45\%CN\_SICE\_SOCN\_RTM\_SGLC\_SWAV)

\item {} 
\sphinxcode{I4804CN} (4804\_DATM\%QIA\_CLM40\%CN\_SICE\_SOCN\_RTM\_SGLC\_SWAV)

\item {} 
\sphinxcode{ICLM45} (2000\_DATM\%QIA\_CLM45\%SP\_SICE\_SOCN\_RTM\_SGLC\_SWAV)

\item {} 
\sphinxcode{ICLM45BGC} (2000\_DATM\%QIA\_CLM45\%BGC\_SICE\_SOCN\_RTM\_SGLC\_SWAV)

\item {} 
\sphinxcode{ICLM45BGCCROP} (2000\_DATM\%QIA\_CLM45\%BGC-CROP\_SICE\_SOCN\_RTM\_SGLC\_SWAV)

\item {} 
\sphinxcode{ICLM45BGCDV} (2000\_DATM\%QIA\_CLM45\%BGCDV\_SICE\_SOCN\_RTM\_SGLC\_SWAV)

\item {} 
\sphinxcode{ICLM45BGCDVCROP} (2000\_DATM\%QIA\_CLM45\%BGCDV-CROP\_SICE\_SOCN\_RTM\_SGLC\_SWAV)

\item {} 
\sphinxcode{ICLM45BGCNoVS} (2000\_DATM\%QIA\_CLM45\%NoVS\_SICE\_SOCN\_RTM\_SGLC\_SWAV)

\item {} 
\sphinxcode{ICLM45CN} (2000\_DATM\%QIA\_CLM45\%CN\_SICE\_SOCN\_RTM\_SGLC\_SWAV)

\item {} 
\sphinxcode{ICLM45CNCROP} (2000\_DATM\%QIA\_CLM45\%CN-CROP\_SICE\_SOCN\_RTM\_SGLC\_SWAV)

\item {} 
\sphinxcode{ICLM45CNDV} (2000\_DATM\%QIA\_CLM45\%CNDV\_SICE\_SOCN\_RTM\_SGLC\_SWAV)

\item {} 
\sphinxcode{ICLM45CNTEST} (2003\_DATM\%QIA\_CLM45\%CN\_SICE\_SOCN\_RTM\_SGLC\_SWAV\_TEST)

\item {} 
\sphinxcode{ICLM45CRUBGC} (2000\_DATM\%CRU\_CLM45\%BGC\_SICE\_SOCN\_RTM\_SGLC\_SWAV)

\item {} 
\sphinxcode{ICLM45GLCMEC} (2000\_DATM\%QIA\_CLM45\%CN\_SICE\_SOCN\_RTM\_CISM1\_SWAV\_TEST)

\item {} 
\sphinxcode{ICLM45SNCRFRC} (2000\_DATM\%QIA\_CLM45\%SP-SNCR\_SICE\_SOCN\_RTM\_SGLC\_SWAV)

\item {} 
\sphinxcode{ICLM45USUMB} (2000\_DATM\%1PT\_CLM45\%SP\_SICE\_SOCN\_RTM\_SGLC\_SWAV\_CLMUSRDAT\%1x1\_US-UMB)

\item {} 
\sphinxcode{ICLM45VIC} (2000\_DATM\%QIA\_CLM45\%SP-VIC\_SICE\_SOCN\_RTM\_SGLC\_SWAV)

\item {} 
\sphinxcode{ICLM45alaskaCN} (2000\_DATM\%QIA\_CLM45\%CN\_SICE\_SOCN\_RTM\_SGLC\_SWAV\_CLMUSRDAT\%13x12pt\_f19\_alaskaUSA)

\item {} 
\sphinxcode{ICN} (2000\_DATM\%QIA\_CLM40\%CN\_SICE\_SOCN\_RTM\_SGLC\_SWAV)

\item {} 
\sphinxcode{ICNCROP} (2000\_DATM\%QIA\_CLM40\%CN-CROP\_SICE\_SOCN\_RTM\_SGLC\_SWAV)

\item {} 
\sphinxcode{ICNDV} (2000\_DATM\%QIA\_CLM40\%CNDV\_SICE\_SOCN\_RTM\_SGLC\_SWAV)

\item {} 
\sphinxcode{ICNDVCROP} (2000\_DATM\%QIA\_CLM40\%CNDV-CROP\_SICE\_SOCN\_RTM\_SGLC\_SWAV)

\item {} 
\sphinxcode{ICNTEST} (2003\_DATM\%QIA\_CLM40\%CN\_SICE\_SOCN\_RTM\_SGLC\_SWAV\_TEST)

\item {} 
\sphinxcode{ICRU} (2000\_DATM\%CRU\_CLM40\%SP\_SICE\_SOCN\_RTM\_SGLC\_SWAV)

\item {} 
\sphinxcode{ICRUCLM45} (2000\_DATM\%CRU\_CLM45\_SICE\_SOCN\_RTM\_SGLC\_SWAV)

\item {} 
\sphinxcode{ICRUCLM45BGC} (2000\_DATM\%CRU\_CLM45\%BGC\_SICE\_SOCN\_RTM\_SGLC\_SWAV)

\item {} 
\sphinxcode{ICRUCLM45BGCCROP} (2000\_DATM\%CRU\_CLM45\%BGC-CROP\_SICE\_SOCN\_RTM\_SGLC\_SWAV)

\item {} 
\sphinxcode{ICRUCLM45BGCTEST} (2003\_DATM\%CRU\_CLM45\%BGC\_SICE\_SOCN\_RTM\_SGLC\_SWAV\_TEST)

\item {} 
\sphinxcode{ICRUCLM45CN} (2000\_DATM\%CRU\_CLM45\%CN\_SICE\_SOCN\_RTM\_SGLC\_SWAV)

\item {} 
\sphinxcode{ICRUCN} (2000\_DATM\%CRU\_CLM40\%CN\_SICE\_SOCN\_RTM\_SGLC\_SWAV)

\item {} 
\sphinxcode{IG} (2000\_DATM\%QIA\_CLM40\%SP\_SICE\_SOCN\_RTM\_CISM1\_SWAV)

\item {} 
\sphinxcode{IG1850} (1850\_DATM\%QIA\_CLM40\%SP\_SICE\_SOCN\_RTM\_CISM1\_SWAV)

\item {} 
\sphinxcode{IG1850CLM45} (1850\_DATM\%QIA\_CLM45\%SP\_SICE\_SOCN\_RTM\_CISM1\_SWAV)

\item {} 
\sphinxcode{IG1850CLM45CN} (1850\_DATM\%QIA\_CLM45\%CN\_SICE\_SOCN\_RTM\_CISM1\_SWAV)

\item {} 
\sphinxcode{IG1850CN} (1850\_DATM\%QIA\_CLM40\%CN\_SICE\_SOCN\_RTM\_CISM1\_SWAV)

\item {} 
\sphinxcode{IG20TR} (20TR\_DATM\%QIA\_CLM40\%SP\_SICE\_SOCN\_RTM\_CISM1\_SWAV)

\item {} 
\sphinxcode{IG20TRCLM45} (20TR\_DATM\%QIA\_CLM45\%SP\_SICE\_SOCN\_RTM\_CISM1\_SWAV)

\item {} 
\sphinxcode{IG20TRCLM45CN} (20TR\_DATM\%QIA\_CLM45\%CN\_SICE\_SOCN\_RTM\_CISM1\_SWAV)

\item {} 
\sphinxcode{IG20TRCN} (20TR\_DATM\%QIA\_CLM40\%CN\_SICE\_SOCN\_RTM\_CISM1\_SWAV)

\item {} 
\sphinxcode{IG4804} (4804\_DATM\%QIA\_CLM40\%SP\_SICE\_SOCN\_RTM\_CISM1\_SWAV)

\item {} 
\sphinxcode{IG4804CLM45} (4804\_DATM\%QIA\_CLM45\%SP\_SICE\_SOCN\_RTM\_CISM1\_SWAV)

\item {} 
\sphinxcode{IG4804CLM45CN} (4804\_DATM\%QIA\_CLM45\%CN\_SICE\_SOCN\_RTM\_CISM1\_SWAV)

\item {} 
\sphinxcode{IG4804CN} (4804\_DATM\%QIA\_CLM40\%CN\_SICE\_SOCN\_RTM\_CISM1\_SWAV)

\item {} 
\sphinxcode{IGCLM45} (2000\_DATM\%QIA\_CLM45\%SP\_SICE\_SOCN\_RTM\_CISM1\_SWAV)

\item {} 
\sphinxcode{IGCLM45CN} (2000\_DATM\%QIA\_CLM45\%CN\_SICE\_SOCN\_RTM\_CISM1\_SWAV)

\item {} 
\sphinxcode{IGCN} (2000\_DATM\%QIA\_CLM40\%CN\_SICE\_SOCN\_RTM\_CISM1\_SWAV)

\item {} 
\sphinxcode{IGLCMEC} (2000\_DATM\%QIA\_CLM40\%CN\_SICE\_SOCN\_RTM\_CISM1\_SWAV\_TEST)

\item {} 
\sphinxcode{IGRCP26CLM45CN} (RCP2\_DATM\%QIA\_CLM45\%CN\_SICE\_SOCN\_RTM\_CISM1\_SWAV)

\item {} 
\sphinxcode{IGRCP26CN} (RCP2\_DATM\%QIA\_CLM40\%CN\_SICE\_SOCN\_RTM\_CISM1\_SWAV)

\item {} 
\sphinxcode{IGRCP45CLM45CN} (RCP4\_DATM\%QIA\_CLM45\%CN\_SICE\_SOCN\_RTM\_CISM1\_SWAV)

\item {} 
\sphinxcode{IGRCP45CN} (RCP4\_DATM\%QIA\_CLM40\%CN\_SICE\_SOCN\_RTM\_CISM1\_SWAV)

\item {} 
\sphinxcode{IGRCP60CLM45CN} (RCP6\_DATM\%QIA\_CLM45\%CN\_SICE\_SOCN\_RTM\_CISM1\_SWAV)

\item {} 
\sphinxcode{IGRCP60CN} (RCP6\_DATM\%QIA\_CLM40\%CN\_SICE\_SOCN\_RTM\_CISM1\_SWAV)

\item {} 
\sphinxcode{IGRCP85CLM45CN} (RCP8\_DATM\%QIA\_CLM45\%CN\_SICE\_SOCN\_RTM\_CISM1\_SWAV)

\item {} 
\sphinxcode{IGRCP85CN} (RCP8\_DATM\%QIA\_CLM40\%CN\_SICE\_SOCN\_RTM\_CISM1\_SWAV)

\item {} 
\sphinxcode{IRCP26CLM45CN} (RCP2\_DATM\%QIA\_CLM45\%CN\_SICE\_SOCN\_RTM\_SGLC\_SWAV)

\item {} 
\sphinxcode{IRCP26CN} (RCP2\_DATM\%QIA\_CLM40\%CN\_SICE\_SOCN\_RTM\_SGLC\_SWAV)

\item {} 
\sphinxcode{IRCP45CLM45CN} (RCP4\_DATM\%QIA\_CLM45\%CN\_SICE\_SOCN\_RTM\_SGLC\_SWAV)

\item {} 
\sphinxcode{IRCP45CN} (RCP4\_DATM\%QIA\_CLM40\%CN\_SICE\_SOCN\_RTM\_SGLC\_SWAV)

\item {} 
\sphinxcode{IRCP60CLM45CN} (RCP6\_DATM\%QIA\_CLM45\%CN\_SICE\_SOCN\_RTM\_SGLC\_SWAV)

\item {} 
\sphinxcode{IRCP60CN} (RCP6\_DATM\%QIA\_CLM40\%CN\_SICE\_SOCN\_RTM\_SGLC\_SWAV)

\item {} 
\sphinxcode{IRCP85CLM45CN} (RCP8\_DATM\%QIA\_CLM45\%CN\_SICE\_SOCN\_RTM\_SGLC\_SWAV)

\item {} 
\sphinxcode{IRCP85CN} (RCP8\_DATM\%QIA\_CLM40\%CN\_SICE\_SOCN\_RTM\_SGLC\_SWAV)

\item {} 
\sphinxcode{ISNCRFRC} (2000\_DATM\%QIA\_CLM40\%SP-SNCR\_SICE\_SOCN\_RTM\_SGLC\_SWAV)

\item {} 
\sphinxcode{ITEST} (2003\_DATM\%QIA\_CLM40\%SP\_SICE\_SOCN\_RTM\_SGLC\_SWAV\_TEST)

\item {} 
\sphinxcode{ITESTCLM45} (2003\_DATM\%QIA\_CLM45\%SP\_SICE\_SOCN\_RTM\_SGLC\_SWAV\_TEST)

\item {} 
\sphinxcode{IUSUMB} (2000\_DATM\%1PT\_CLM40\%SP\_SICE\_SOCN\_RTM\_SGLC\_SWAV\_CLMUSRDAT\%1x1\_US-UMB)

\item {} 
\sphinxcode{IalaskaCN} (2000\_DATM\%QIA\_CLM40\%CN\_SICE\_SOCN\_RTM\_SGLC\_SWAV\_CLMUSRDAT\%13x12pt\_f19\_alaskaUSA)

\end{enumerate}


\subsubsection{Compsets coupled to active atmosphere with data ocean}
\label{\detokenize{users_guide/setting-up-and-running-a-case/choosing-a-compset:compsets-coupled-to-active-atmosphere-with-data-ocean}}
CAM compsets are compsets that start with “E” or “F” in the name. They are described more fully in the scripts documentation or the CAM documentation. “E” compsets have a slab ocean model while “F” compsets have a data ocean model.


\subsubsection{Fully coupled compsets with fully active ocean, sea-ice, and atmosphere}
\label{\detokenize{users_guide/setting-up-and-running-a-case/choosing-a-compset:fully-coupled-compsets-with-fully-active-ocean-sea-ice-and-atmosphere}}
Fully coupled compsets are compsets that start with “B” in the name. They are described more fully in the scripts documentation.


\subsubsection{Conclusion to choosing a compset}
\label{\detokenize{users_guide/setting-up-and-running-a-case/choosing-a-compset:conclusion-to-choosing-a-compset}}
We’ve introduced the basic type of compsets that use CLM and given some further details for the “standalone CLM” (or “I” compsets).
The \sphinxhref{CLM-URL}{config\_compsets.xml} lists all of the compsets and gives a full description of each of them.
In the next section we look into customizing the setup time options for compsets using CLM.


\subsection{Customizing CLM’s Configuration}
\label{\detokenize{users_guide/setting-up-and-running-a-case/customizing-the-clm-configuration:configuring-clm}}\label{\detokenize{users_guide/setting-up-and-running-a-case/customizing-the-clm-configuration::doc}}\label{\detokenize{users_guide/setting-up-and-running-a-case/customizing-the-clm-configuration:customizing-clm-s-configuration}}
The “Creating a Case” section of the \sphinxhref{link-CESM-UG}{CESM1.2.0 Scripts User’s-Guide} gives instructions on creating a case. What is of interest here is how to customize your use of CLM for the case that you created.

For CLM when \sphinxstylestrong{preview\_namelist}, \sphinxstylestrong{\$CASE.build}, or \sphinxstylestrong{\$CASE.run} are called there are two steps that take place:
\begin{enumerate}
\item {} 
The CLM “\sphinxstylestrong{configure}” script is called to setup the build-time configuration for CLM (more information on \sphinxstylestrong{configure} is given in \sphinxhref{CLM-URL}{the Section called More information on the CLM configure script}). The env variables for \sphinxstylestrong{configure} are locked after the \sphinxstylestrong{\$CASE.build} step. So the results of the CLM \sphinxstylestrong{configure} are locked after the build has taken place.

\item {} 
The CLM “\sphinxstylestrong{build-namelist}” script is called to generate the run-time namelist for CLM (more information on \sphinxstylestrong{build-namelist} is given below in \sphinxhref{CLM-URL}{the Section called Definition of Namelist items and their default values}.

\end{enumerate}

When customizing your case at the \sphinxstyleemphasis{8cesm\_setup*} step you are able to modify the process by effecting either one or both of these steps. The CLM “\sphinxstylestrong{configure}” and “\sphinxstylestrong{build-namelist}” scripts are both available in the “models/lnd/clm/bld” directory in the distribution. Both of these scripts have a “-help” option that is useful to examine to see what types of options you can give either of them.

There are five different types of customization for the configuration that we will discuss: CLM4.5 in CESM1.2.0 build-time options, CLM4.5 in CESM1.2.0 run-time options, User Namelist, other noteworthy CESM1.2.0 configuration items, the CLM \sphinxstylestrong{configure} script options, and the CLM \sphinxstylestrong{build-namelist} script options.

Information on all of the script, configuration, build and run items is found under \sphinxcode{scripts/ccsm\_utils/Case.template} in the \sphinxhref{CLM-URL}{config\_definition.xml} file.


\subsubsection{CLM Script configuration items}
\label{\detokenize{users_guide/setting-up-and-running-a-case/customizing-the-clm-configuration:clm-script-configuration-items}}
Below we list each of the CESM configuration items that are specific to CLM. All of these are available in your: \sphinxcode{env\_build.xml} and \sphinxcode{env\_run.xml} files.

\begin{sphinxVerbatim}[commandchars=\\\{\}]
\PYG{n}{CLM\PYGZus{}CONFIG\PYGZus{}OPTS}
\PYG{n}{CLM\PYGZus{}BLDNML\PYGZus{}OPTS}
\PYG{n}{CLM\PYGZus{}NAMELIST\PYGZus{}OPTS}
\PYG{n}{CLM\PYGZus{}FORCE\PYGZus{}COLDSTART}
\PYG{n}{CLM\PYGZus{}NML\PYGZus{}USE\PYGZus{}CASE}
\PYG{n}{CLM\PYGZus{}USRDAT\PYGZus{}NAME}
\PYG{n}{CLM\PYGZus{}CO2\PYGZus{}TYPE}
\end{sphinxVerbatim}

For the precedence of the different options to \sphinxstylestrong{build-namelist} see the section on precedence below.

The first item \sphinxcode{CLM\_CONFIG\_OPTS} has to do with customizing the CLM build-time options for your case, the rest all have to do with generating the namelist.
\begin{description}
\item[{CLM\_CONFIG\_OPTS}] \leavevmode
The option \sphinxcode{CLM\_CONFIG\_OPTS} is all about passing command line arguments to the CLM \sphinxstylestrong{configure} script.
It is important to note that some compsets, may already put a value into the \sphinxcode{CLM\_CONFIG\_OPTS} variable.
You can still add more options to your \sphinxcode{CLM\_CONFIG\_OPTS} but make sure you add to what is already there rather than replacing it.
Hence, we recommend using the “-append” option to the xmlchange script.
In \sphinxhref{CLM-URL}{the Section called More information on the CLM configure script} below we will go into more details on options that can be customized in the CLM “\sphinxstylestrong{configure}” script.
It’s also important to note that the \sphinxstylestrong{clm.buildnml.csh} script may already invoke certain CLM \sphinxstylestrong{configure} options and as such those command line options are NOT going to be available to change at this step (nor would you want to change them).
The options to CLM \sphinxstylestrong{configure} are given with the “-help” option which is given in \sphinxhref{CLM-URL}{the Section called More information on the CLM configure script}.
.. note:: \sphinxcode{CLM\_CONFIG\_OPTS} is locked after the \sphinxstylestrong{\$CASE.build} script is run. If you want to change something in \sphinxcode{CLM\_CONFIG\_OPTS} you’ll need to clean the build and rerun \sphinxstylestrong{\$CASE.build}. The other env variables can be changed at run-time so are never locked.

\item[{CLM\_NML\_USE\_CASE}] \leavevmode
\sphinxcode{CLM\_NML\_USE\_CASE} is used to set a particular set of conditions that set multiple namelist items, all centering around a particular usage of the model. To list the valid options do the following:

\begin{sphinxVerbatim}[commandchars=\\\{\}]
\PYG{o}{\PYGZgt{}} \PYG{n}{cd} \PYG{n}{models}\PYG{o}{/}\PYG{n}{lnd}\PYG{o}{/}\PYG{n}{clm}\PYG{o}{/}\PYG{n}{doc}
\PYG{o}{\PYGZgt{}} \PYG{o}{.}\PYG{o}{.}\PYG{o}{/}\PYG{n}{bld}\PYG{o}{/}\PYG{n}{build}\PYG{o}{\PYGZhy{}}\PYG{n}{namelist} \PYG{o}{\PYGZhy{}}\PYG{n}{use\PYGZus{}case} \PYG{n+nb}{list}
\end{sphinxVerbatim}

The output of the above command is:

\begin{sphinxVerbatim}[commandchars=\\\{\}]
CLM build\PYGZhy{}namelist \PYGZhy{} use cases: 1850\PYGZhy{}2100\PYGZus{}rcp2.6\PYGZus{}glacierMEC\PYGZus{}transient 1850\PYGZhy{}2100\PYGZus{}rcp2.6\PYGZus{}transient  \PYGZbs{}
1850\PYGZhy{}2100\PYGZus{}rcp4.5\PYGZus{}glacierMEC\PYGZus{}transient 1850\PYGZhy{}2100\PYGZus{}rcp4.5\PYGZus{}transient  \PYGZbs{}
1850\PYGZhy{}2100\PYGZus{}rcp6\PYGZus{}glacierMEC\PYGZus{}transient 1850\PYGZhy{}2100\PYGZus{}rcp6\PYGZus{}transient  \PYGZbs{}
1850\PYGZhy{}2100\PYGZus{}rcp8.5\PYGZus{}glacierMEC\PYGZus{}transient 1850\PYGZhy{}2100\PYGZus{}rcp8.5\PYGZus{}transient 1850\PYGZus{}control  \PYGZbs{}
1850\PYGZus{}glacierMEC\PYGZus{}control 2000\PYGZhy{}2100\PYGZus{}rcp8.5\PYGZus{}transient 2000\PYGZus{}control 2000\PYGZus{}glacierMEC\PYGZus{}control  \PYGZbs{}
20thC\PYGZus{}glacierMEC\PYGZus{}transient 20thC\PYGZus{}transient glacierMEC\PYGZus{}pd stdurbpt\PYGZus{}pd
Use cases are:...

1850\PYGZhy{}2100\PYGZus{}rcp2.6\PYGZus{}glacierMEC\PYGZus{}transient = Simulate transient land\PYGZhy{}use, and aerosol deposition changes  \PYGZbs{}
with historical data from 1850 to 2005 and then with the RCP2.6 scenario from IMAGE

1850\PYGZhy{}2100\PYGZus{}rcp2.6\PYGZus{}transient = Simulate transient land\PYGZhy{}use, and aerosol deposition changes with  \PYGZbs{}
historical data from 1850 to 2005 and then with the RCP2.6 scenario from IMAGE

1850\PYGZhy{}2100\PYGZus{}rcp4.5\PYGZus{}glacierMEC\PYGZus{}transient = Simulate transient land\PYGZhy{}use, and aerosol deposition changes  \PYGZbs{}
with historical data from 1850 to 2005 and then with the RCP4.5 scenario from MINICAM

1850\PYGZhy{}2100\PYGZus{}rcp4.5\PYGZus{}transient = Simulate transient land\PYGZhy{}use, and aerosol deposition changes with  \PYGZbs{}
historical data from 1850 to 2005 and then with the RCP4.5 scenario from MINICAM

1850\PYGZhy{}2100\PYGZus{}rcp6\PYGZus{}glacierMEC\PYGZus{}transient = Simulate transient land\PYGZhy{}use, and aerosol deposition changes  \PYGZbs{}
with historical data from 1850 to 2005 and then with the RCP6 scenario from AIM

1850\PYGZhy{}2100\PYGZus{}rcp6\PYGZus{}transient = Simulate transient land\PYGZhy{}use, and aerosol deposition changes with  \PYGZbs{}
historical data from 1850 to 2005 and then with the RCP6 scenario from AIM

1850\PYGZhy{}2100\PYGZus{}rcp8.5\PYGZus{}glacierMEC\PYGZus{}transient = Simulate transient land\PYGZhy{}use, and aerosol deposition changes  \PYGZbs{}
with historical data from 1850 to 2005 and then with the RCP8.5 scenario from MESSAGE

1850\PYGZhy{}2100\PYGZus{}rcp8.5\PYGZus{}transient = Simulate transient land\PYGZhy{}use, and aerosol deposition changes with  \PYGZbs{}
historical data from 1850 to 2005 and then with the RCP8.5 scenario from MESSAGE

1850\PYGZus{}control = Conditions to simulate 1850 land\PYGZhy{}use
1850\PYGZus{}glacierMEC\PYGZus{}control = Running an IG case for 1850 conditions with the ice sheet model glimmer
2000\PYGZhy{}2100\PYGZus{}rcp8.5\PYGZus{}transient = Simulate transient land\PYGZhy{}use, and aerosol deposition changes with  \PYGZbs{}
historical data from 2000 to 2005 and then with the RCP8.5 scenario from MESSAGE

2000\PYGZus{}control = Conditions to simulate 2000 land\PYGZhy{}use
2000\PYGZus{}glacierMEC\PYGZus{}control = Running an IG case for 2000 conditions with the ice sheet model glimmer
20thC\PYGZus{}glacierMEC\PYGZus{}transient = Simulate transient land\PYGZhy{}use, and aerosol deposition changes from 1850  \PYGZbs{}
to 2005
20thC\PYGZus{}transient = Simulate transient land\PYGZhy{}use, and aerosol deposition changes from 1850 to 2005
glacierMEC\PYGZus{}pd = Running an IG case with the ice sheet model glimmer
stdurbpt\PYGZus{}pd = Standard Urban Point Namelist Settings

.. note::See {}`the Section called Precedence of Options \PYGZlt{}CLM\PYGZhy{}URL\PYGZgt{}{}`\PYGZus{} section for the precedence of this option relative to the others.
\end{sphinxVerbatim}

\item[{CLM\_BLDNML\_OPTS}] \leavevmode
The option CLM\_BLDNML\_OPTS is for passing options to the CLM “build-namelist” script.
As with the CLM “configure” script the CLM clm.buildnml.csh may already invoke certain options and as such those options will NOT be available to be set here. The best way to see what options can be sent to the “build-namelist” script is to do

\begin{sphinxVerbatim}[commandchars=\\\{\}]
\PYG{o}{\PYGZgt{}} \PYG{n}{cd} \PYG{n}{models}\PYG{o}{/}\PYG{n}{lnd}\PYG{o}{/}\PYG{n}{clm}\PYG{o}{/}\PYG{n}{bld}
\PYG{o}{\PYGZgt{}} \PYG{o}{.}\PYG{o}{/}\PYG{n}{build}\PYG{o}{\PYGZhy{}}\PYG{n}{namelist} \PYG{o}{\PYGZhy{}}\PYG{n}{help}
\end{sphinxVerbatim}

Here is the output from the above.

\begin{sphinxVerbatim}[commandchars=\\\{\}]
./SYNOPSIS
build\PYGZhy{}namelist [options]

Create the namelist for CLM
OPTIONS
\PYGZhy{}[no\PYGZhy{}]chk\PYGZus{}res            Also check [do NOT check] to make sure the resolution and
                         land\PYGZhy{}mask is valid.
\PYGZhy{}clm\PYGZus{}demand \PYGZdq{}list\PYGZdq{}       List of variables to require on clm namelist besides the usuals.
                         \PYGZdq{}\PYGZhy{}clm\PYGZus{}demand list\PYGZdq{} to list valid options.
                         (can include a list member \PYGZdq{}null\PYGZdq{} which does nothing)
\PYGZhy{}clm\PYGZus{}startfile \PYGZdq{}file\PYGZdq{}    CLM restart file to start from.
\PYGZhy{}clm\PYGZus{}start\PYGZus{}type \PYGZdq{}type\PYGZdq{}   Start type of simulation
                         (default, cold, arb\PYGZus{}ic, startup, continue, or branch)
                         (default=do the default type for this configuration)
                         (cold=always start with arbitrary initial conditions)
                         (arb\PYGZus{}ic=start with arbitrary initial conditions if
                          initial conditions don\PYGZsq{}t exist)
                         (startup=ensure that initial conditions are being used)
\PYGZhy{}clm\PYGZus{}usr\PYGZus{}name     \PYGZdq{}name\PYGZdq{} Dataset resolution/descriptor for personal datasets.
                         Default: not used
                         Example: 1x1pt\PYGZus{}boulderCO\PYGZus{}c090722 to describe location,
                                  number of pts, and date files created
\PYGZhy{}co2\PYGZus{}type \PYGZdq{}value\PYGZdq{}        Set CO2 the type of CO2 variation to use.
\PYGZhy{}co2\PYGZus{}ppmv \PYGZdq{}value\PYGZdq{}        Set CO2 concentration to use when co2\PYGZus{}type is constant (ppmv).
\PYGZhy{}config \PYGZdq{}filepath\PYGZdq{}       Read the given CLM configuration cache file.
                         Default: \PYGZdq{}config\PYGZus{}cache.xml\PYGZdq{}.
\PYGZhy{}csmdata \PYGZdq{}dir\PYGZdq{}           Root directory of CESM input data.
                         Can also be set by using the CSMDATA environment variable.
\PYGZhy{}d \PYGZdq{}directory\PYGZdq{}           Directory where output namelist file will be written
                         Default: current working directory.
\PYGZhy{}drydep                  Produce a drydep\PYGZus{}inparm namelist that will go into the
                         \PYGZdq{}drv\PYGZus{}flds\PYGZus{}in\PYGZdq{} file for the driver to pass dry\PYGZhy{}deposition to the atm.
                         Default: \PYGZhy{}no\PYGZhy{}drydep
                         (Note: buildnml.csh copies the file for use by the driver)
\PYGZhy{}glc\PYGZus{}grid \PYGZdq{}grid\PYGZdq{}         Glacier model grid and resolution when glacier model,
                         Only used if glc\PYGZus{}nec \PYGZgt{} 0 for determining fglcmask
                         Default:  gland5UM
                         (i.e. gland20, gland10 etcetera)
\PYGZhy{}glc\PYGZus{}nec \PYGZlt{}name\PYGZgt{}          Glacier number of elevation classes [0 \textbar{} 3 \textbar{} 5 \textbar{} 10 \textbar{} 36]
                         (default is 0) (standard option with land\PYGZhy{}ice model is 10)
\PYGZhy{}glc\PYGZus{}smb \PYGZlt{}value\PYGZgt{}         Only used if glc\PYGZus{}nec \PYGZgt{} 0
                         If .true., pass surface mass balance info to GLC
                         If .false., pass positive\PYGZhy{}degree\PYGZhy{}day info to GLC
                         Default: true
\PYGZhy{}help [or \PYGZhy{}h]            Print usage to STDOUT.
\PYGZhy{}ignore\PYGZus{}ic\PYGZus{}date          Ignore the date on the initial condition files
                         when determining what input initial condition file to use.
\PYGZhy{}ignore\PYGZus{}ic\PYGZus{}year          Ignore just the year part of the date on the initial condition files
                            when determining what input initial condition file to use.
\PYGZhy{}infile \PYGZdq{}filepath\PYGZdq{}       Specify a file (or list of files) containing namelists to
                         read values from.

                         If used with a CLM build with multiple ensembles (ninst\PYGZus{}lnd\PYGZgt{}1)
                         and the filename entered is a directory to files of the
                         form filepath/filepath and filepath/filepath\PYGZus{}\PYGZdl{}n where \PYGZdl{}n
                         is the ensemble member number. the \PYGZdq{}filepath/filepath\PYGZdq{}
                         input namelist file is the master input namelist file
                         that is applied to ALL ensemble members.

                         (by default for CESM this is setup for files of the
                          form \PYGZdl{}CASEDIR/user\PYGZus{}nl\PYGZus{}clm/user\PYGZus{}nl\PYGZus{}clm\PYGZus{}????)
\PYGZhy{}inputdata \PYGZdq{}filepath\PYGZdq{}    Writes out a list containing pathnames for required input datasets in

                            file specified.
\PYGZhy{}irrig \PYGZdq{}value\PYGZdq{}           If .true. turn irrigation on with namelist logical irrigate (for CLM4.5 physics)
                         (requires crop to be on in the clm configuration)
                         Seek surface datasets with irrigation turned on.  (for CLM4.0 physics)
                         Default: .false.
\PYGZhy{}l\PYGZus{}ncpl \PYGZdq{}LND\PYGZus{}NCPL\PYGZdq{}       Number of CLM coupling time\PYGZhy{}steps in a day.
\PYGZhy{}lnd\PYGZus{}frac \PYGZdq{}domainfile\PYGZdq{}   Land fraction file (the input domain file)
\PYGZhy{}mask \PYGZdq{}landmask\PYGZdq{}         Type of land\PYGZhy{}mask (default, navy, gx3v5, gx1v5 etc.)
                         \PYGZdq{}\PYGZhy{}mask list\PYGZdq{} to list valid land masks.
\PYGZhy{}namelist \PYGZdq{}namelist\PYGZdq{}     Specify namelist settings directly on the commandline by supplying
                         a string containing FORTRAN namelist syntax, e.g.,
                            \PYGZhy{}namelist \PYGZdq{}\PYGZam{}clm\PYGZus{}inparm dt=1800 /\PYGZdq{}
\PYGZhy{}no\PYGZhy{}megan                DO NOT PRODUCE a megan\PYGZus{}emis\PYGZus{}nl namelist that will go into the
                         \PYGZdq{}drv\PYGZus{}flds\PYGZus{}in\PYGZdq{} file for the driver to pass VOCs to the atm.
                         MEGAN (Model of Emissions of Gases and Aerosols from Nature)
                         (Note: buildnml.csh copies the file for use by the driver)
\PYGZhy{}[no\PYGZhy{}]note               Add note to output namelist  [do NOT add note] about the
                         arguments to build\PYGZhy{}namelist.
\PYGZhy{}rcp \PYGZdq{}value\PYGZdq{}             Representative concentration pathway (rcp) to use for
                         future scenarios.
                         \PYGZdq{}\PYGZhy{}rcp list\PYGZdq{} to list valid rcp settings.
\PYGZhy{}res \PYGZdq{}resolution\PYGZdq{}        Specify horizontal grid.  Use nlatxnlon for spectral grids;
                         dlatxdlon for fv grids (dlat and dlon are the grid cell size
                         in degrees for latitude and longitude respectively)
                         \PYGZdq{}\PYGZhy{}res list\PYGZdq{} to list valid resolutions.
\PYGZhy{}s                       Turns on silent mode \PYGZhy{} only fatal messages issued.
\PYGZhy{}sim\PYGZus{}year \PYGZdq{}year\PYGZdq{}         Year to simulate for input datasets
                         (i.e. 1850, 2000, 1850\PYGZhy{}2000, 1850\PYGZhy{}2100)
                         \PYGZdq{}\PYGZhy{}sim\PYGZus{}year list\PYGZdq{} to list valid simulation years
\PYGZhy{}bgc\PYGZus{}spinup \PYGZdq{}on\textbar{}off\PYGZdq{}     CLM 4.5 Only. For CLM 4.0, spinup is controlled from configure.
                         Turn on given spinup mode for BGC setting of CN
                         on : Turn on Accelerated Decomposition (spinup\PYGZus{}state = 1)
                         off : run in normal mode (spinup\PYGZus{}state = 0)

                         Default is off.

                         Spinup is now a two step procedure. First, run the model
                         with spinup = \PYGZdq{}on\PYGZdq{}. Then run the model for a while with
                         spinup = \PYGZdq{}off\PYGZdq{}. The exit spinup step happens automatically
                         on the first timestep when using a restart file from spinup
                         mode.

                         The spinup state is saved to the restart file.
                         If the values match between the model and the restart
                         file it proceeds as directed.

                         If the restart file is in spinup mode and the model is in
                         normal mode, then it performs the exit spinup step
                         and proceeds in normal mode after that.

                         If the restart file has normal mode and the model is in
                         spinup, then it enters spinup. This is useful if you change
                         a parameter and want to rapidly re\PYGZhy{}equilibrate without doing
                         a cold start.

\PYGZhy{}test                    Enable checking that input datasets exist on local filesystem.
\PYGZhy{}verbose [or \PYGZhy{}v]         Turn on verbose echoing of informational messages.
\PYGZhy{}use\PYGZus{}case \PYGZdq{}case\PYGZdq{}         Specify a use case which will provide default values.
                         \PYGZdq{}\PYGZhy{}use\PYGZus{}case list\PYGZdq{} to list valid use\PYGZhy{}cases.
\PYGZhy{}version                 Echo the SVN tag name used to check out this CLM distribution.



Note: The precedence for setting the values of namelist variables is (highest to lowest):
 0. namelist values set by specific command\PYGZhy{}line options, like, \PYGZhy{}d, \PYGZhy{}sim\PYGZus{}year
        (i.e.  CLM\PYGZus{}BLDNML\PYGZus{}OPTS env\PYGZus{}run variable)
 1. values set on the command\PYGZhy{}line using the \PYGZhy{}namelist option,
        (i.e. CLM\PYGZus{}NAMELIST\PYGZus{}OPTS env\PYGZus{}run variable)
 2. values read from the file(s) specified by \PYGZhy{}infile,
        (i.e.  user\PYGZus{}nl\PYGZus{}clm files)
 3. datasets from the \PYGZhy{}clm\PYGZus{}usr\PYGZus{}name option,
        (i.e.  CLM\PYGZus{}USRDAT\PYGZus{}NAME env\PYGZus{}run variable)
 4. values set from a use\PYGZhy{}case scenario, e.g., \PYGZhy{}use\PYGZus{}case
        (i.e.  CLM\PYGZus{}NML\PYGZus{}USE\PYGZus{}CASE env\PYGZus{}run variable)
 5. values from the namelist defaults file.
\end{sphinxVerbatim}

\end{description}

The \sphinxstylestrong{clm.buildnml.csh} script already sets the resolution and mask as well as the CLM \sphinxstylestrong{configure} file, and defines an input namelist and namelist input file, and the output namelist directory, and sets the start-type (from \sphinxcode{RUN\_TYPE}), namelist options (from \sphinxcode{CLM\_NAMELIST\_OPTS}), co2\_ppmv (from \sphinxcode{CCSM\_CO2\_PPMV}, co2\_type (from \sphinxcode{CLM\_CO2\_TYPE}), lnd\_frac (from \sphinxcode{LND\_DOMAIN\_PATH} and \sphinxcode{LND\_DOMAIN\_FILE}), l\_ncpl (from \sphinxcode{LND\_NCPL}, glc\_grid, glc\_smb, glc\_nec (from \sphinxcode{GLC\_GRID}, \sphinxcode{GLC\_SMB}, and \sphinxcode{GLC\_NEC}), and “clm\_usr\_name” is set (to \sphinxcode{CLM\_USRDAT\_NAME \textgreater{}{}`{}`when the grid is set to {}`{}`CLM\_USRDAT\_NAME}.
Hence only the following different options can be set:

1.
-bgc\_spinup
\begin{enumerate}
\item {} 
-chk\_res

\item {} 
-clm\_demand

\item {} 
-drydep

\item {} 
-ignore\_ic\_date

\item {} 
-ignore\_ic\_year

\item {} 
-irrig

\item {} 
-no-megan

\item {} 
-note

\item {} 
-rcp

\item {} 
-sim\_year

\item {} 
-verbose

\end{enumerate}

“-bgc\_spinup” is an option only available for CLM4.5 for any configuration when CN is turned on (so either CLMCN or CLMBGC). It can be set to “on” or “off”. If “on” the model will go into Accelerated Decomposition mode, while for “off” (the default) it will have standard decomposition rates. If you are starting up from initial condition files the model will check what mode the initial condition file is in and do the appropriate action on the first time-step to change the Carbon pools to the appropriate spinup setting. See \sphinxhref{CLM-URL}{the Section called Spinning up the CLM4.5 biogeochemistry (CLMBGC spinup) in Chapter 4} for an example using this option.

“-chk\_res” ensures that the resolution chosen is supported by CLM. If the resolution is NOT supported it will cause the CLM \sphinxstylestrong{build-namelist} to abort when run. So when either \sphinxstylestrong{preview\_namelist}, \sphinxstylestrong{\$CASE.build} or \sphinxstylestrong{\$CASE.run} is executed it will abort early. Since, the CESM scripts only support certain resolutions anyway, in general this option is NOT needed in the context of running CESM cases.

“-clm\_demand” asks the \sphinxstylestrong{build-namelist} step to require that the list of variables entered be set. Typically, this is used to require that optional filenames be used and ensure they are set before continuing. For example, you may want to require that fpftdyn be set to get dynamically changing vegetation types. To do this you would do the following.

\begin{sphinxVerbatim}[commandchars=\\\{\}]
\PYGZgt{} ./xmlchange CLM\PYGZus{}BLDNML\PYGZus{}OPTS=\PYGZdq{}\PYGZhy{}clm\PYGZus{}demand fpftdyn\PYGZdq{}{}`{}`
\end{sphinxVerbatim}

To see a list of valid variables that you could set do this:

\begin{sphinxVerbatim}[commandchars=\\\{\}]
\PYG{o}{\PYGZgt{}} \PYG{n}{cd} \PYG{n}{models}\PYG{o}{/}\PYG{n}{lnd}\PYG{o}{/}\PYG{n}{clm}\PYG{o}{/}\PYG{n}{doc}
\PYG{o}{\PYGZgt{}} \PYG{o}{.}\PYG{o}{.}\PYG{o}{/}\PYG{n}{bld}\PYG{o}{/}\PYG{n}{build}\PYG{o}{\PYGZhy{}}\PYG{n}{namelist} \PYG{o}{\PYGZhy{}}\PYG{n}{clm\PYGZus{}demand} \PYG{n+nb}{list}
\end{sphinxVerbatim}

\begin{sphinxadmonition}{note}{Note:}
Using a 20th-Century transient compset or the \sphinxcode{20thC\_transient} use-case using \sphinxcode{CLM\_NML\_USE\_CASE} would set this as well, but would also use dynamic nitrogen and aerosol deposition files, so using \sphinxcode{-clm\_demand} would be a way to get \sphinxstyleemphasis{just} dynamic vegetation types and NOT the other files as well.
\end{sphinxadmonition}

“-drydep” adds the dry-deposition namelist to the driver. This is a driver namelist, but adding the option here has CLM \sphinxstylestrong{build-namelist} create the \sphinxcode{drv\_flds\_in} file that the driver will copy over and use. Invoking this option does have an impact on performance even for I compsets and will slow the model down. It’s also only useful when running with an active atmosphere model that makes use of this information.

“-ignore\_ic\_date” ignores the Initial Conditions (IC) date completely for finding initial condition files to startup from. Without this option or the “-ignore\_ic\_year” option below, the date of the file comes into play.

“-ignore\_ic\_year” ignores the Initial Conditions (IC) year for finding initial condition files to startup from. The date is used, but the year is ignored. Without this option or the “-ignore\_ic\_date” option below, the date and year of the file comes into play.

When “-irrig” is used \sphinxstylestrong{build-namelist} will try to find surface datasets that have the irrigation model enabled.

“no-megan” means do NOT add the MEGAN model Biogenic Volatile Organic Compounds (BVOC) namelist to the driver. This namelist is created by default, so normally this WILL be done. This is a driver namelist, so unless “no-megan” is specified the CLM \sphinxstylestrong{build-namelist} will create the \sphinxcode{drv\_flds\_in} file that the driver will copy over and use (if you are running with CAM and CAM produces this file as well, it’s file will have precedence).

“-note” adds a note to the bottom of the namelist file, that gives the details of how \sphinxstylestrong{build-namelist} was called, giving the specific command-line options given to it.

“-rcp” is used to set the representative concentration pathway for the future scenarios you want the data-sets to simulate conditions for, in the input datasets. To list the valid options do the following:

\begin{sphinxVerbatim}[commandchars=\\\{\}]
\PYG{o}{\PYGZgt{}} \PYG{n}{cd} \PYG{n}{models}\PYG{o}{/}\PYG{n}{lnd}\PYG{o}{/}\PYG{n}{clm}\PYG{o}{/}\PYG{n}{doc}
\PYG{o}{\PYGZgt{}} \PYG{o}{.}\PYG{o}{.}\PYG{o}{/}\PYG{n}{bld}\PYG{o}{/}\PYG{n}{build}\PYG{o}{\PYGZhy{}}\PYG{n}{namelist} \PYG{o}{\PYGZhy{}}\PYG{n}{rcp} \PYG{n+nb}{list}
\end{sphinxVerbatim}

“-sim\_year” is used to set the simulation year you want the data-sets to simulate conditions for in the input datasets. The simulation “year” can also be a range of years in order to do simulations with changes in the dataset values as the simulation progresses. To list the valid options do the following:

\begin{sphinxVerbatim}[commandchars=\\\{\}]
\PYG{o}{\PYGZgt{}} \PYG{n}{cd} \PYG{n}{models}\PYG{o}{/}\PYG{n}{lnd}\PYG{o}{/}\PYG{n}{clm}\PYG{o}{/}\PYG{n}{doc}
\PYG{o}{\PYGZgt{}} \PYG{o}{.}\PYG{o}{.}\PYG{o}{/}\PYG{n}{bld}\PYG{o}{/}\PYG{n}{build}\PYG{o}{\PYGZhy{}}\PYG{n}{namelist} \PYG{o}{\PYGZhy{}}\PYG{n}{sim\PYGZus{}year} \PYG{n+nb}{list}
\end{sphinxVerbatim}
\begin{description}
\item[{\sphinxcode{CLM\_NAMELIST\_OPTS}}] \leavevmode
passes namelist items into one of the CLM namelists.

\begin{sphinxadmonition}{note}{Note:}
For character namelist items you need to use “\&apos;” as quotes for strings so that the scripts don’t get confused with other quotes they use.
\end{sphinxadmonition}

Example, you want to set \sphinxcode{hist\_dov2xy} to \sphinxcode{.false.} so that you get vector output to your history files. To do so edit \sphinxcode{env\_run.xml} and add a setting for \sphinxcode{hist\_dov2xy}. So do the following:

\begin{sphinxVerbatim}[commandchars=\\\{\}]
\PYG{o}{\PYGZgt{}} \PYG{o}{.}\PYG{o}{/}\PYG{n}{xmlchange} \PYG{n}{CLM\PYGZus{}NAMELIST\PYGZus{}OPTS}\PYG{o}{=}\PYG{l+s+s2}{\PYGZdq{}}\PYG{l+s+s2}{hist\PYGZus{}dov2xy=.false.}\PYG{l+s+s2}{\PYGZdq{}}
\end{sphinxVerbatim}

Example, you want to set \sphinxcode{hist\_fincl1} to add the variable ‘HK’ to your history files. To do so edit \sphinxcode{env\_run.xml} and add a setting for \sphinxcode{hist\_fincl1}. So do the following:

\begin{sphinxVerbatim}[commandchars=\\\{\}]
\PYG{o}{\PYGZgt{}} \PYG{o}{.}\PYG{o}{/}\PYG{n}{xmlchange} \PYG{n}{CLM\PYGZus{}NAMELIST\PYGZus{}OPTS}\PYG{o}{=}\PYG{l+s+s2}{\PYGZdq{}}\PYG{l+s+s2}{hist\PYGZus{}fincl1=\PYGZam{}apos;HK\PYGZam{}apos;}\PYG{l+s+s2}{\PYGZdq{}}
\end{sphinxVerbatim}

For a list of the history fields available see \sphinxhref{CLM-URL}{CLM History Fields}.

\item[{\sphinxcode{CLM\_FORCE\_COLDSTART}}] \leavevmode
when set to on, \sphinxstyleemphasis{requires} that your simulation do a cold start from arbitrary initial conditions. If this is NOT set, it will use an initial condition file if it can find an appropriate one, and otherwise do a cold start. \sphinxcode{CLM\_FORCE\_COLDSTART} is a good way to ensure that you are doing a cold start if that is what you want to do.

\item[{\sphinxcode{CLM\_USRDAT\_NAME}}] \leavevmode
Provides a way to enter your own datasets into the namelist.
The files you create must be named with specific naming conventions outlined in: \sphinxhref{CLM-URL}{the Section called Creating your own single-point/regional surface datasets in Chapter 5}.
To see what the expected names of the files are, use the \sphinxstylestrong{queryDefaultNamelist.pl} to see what the names will need to be.
For example if your \sphinxcode{CLM\_USRDAT\_NAME} will be “1x1\_boulderCO”, with a “navy” land-mask, constant simulation year range, for 1850, the following will list what your filenames should be:

\begin{sphinxVerbatim}[commandchars=\\\{\}]
\PYGZgt{} cd models/lnd/clm/bld
\PYGZgt{} queryDefaultNamelist.pl \PYGZhy{}usrname \PYGZdq{}1x1\PYGZus{}boulderCO\PYGZdq{} \PYGZhy{}options mask=navy,sim\PYGZus{}year=1850,sim\PYGZus{}year\PYGZus{}range=\PYGZdq{}constant\PYGZdq{}  \PYGZhy{}csmdata \PYGZdl{}CSMDATA

An example of using {}`{}`CLM\PYGZus{}USRDAT\PYGZus{}NAME{}`{}` for a simulation is given in {}`Example 5\PYGZhy{}4 \PYGZlt{}CLM\PYGZhy{}URL\PYGZgt{}{}`\PYGZus{}.
\end{sphinxVerbatim}

\item[{\sphinxcode{CLM\_CO2\_TYPE}}] \leavevmode
sets the type of input CO2 for either “constant”, “diagnostic” or prognostic”.
If “constant” the value from \sphinxcode{CCSM\_CO2\_PPMV} will be used.
If “diagnostic” or “prognostic” the values MUST be sent from the atmosphere model.
For more information on how to send CO2 from the data atmosphere model see \sphinxhref{CLM-URL}{the Section called Running stand-alone CLM with transient historical CO2 concentration in Chapter 4}.

\end{description}


\subsubsection{User Namelist}
\label{\detokenize{users_guide/setting-up-and-running-a-case/customizing-the-clm-configuration:user-namelist}}
\sphinxcode{CLM\_NAMELIST\_OPTS} as described above allows you to set any extra namelist items you would like to appear in your namelist. However, it only allows you a single line to enter namelist items, and strings must be quoted with \&apos; which is a bit awkward. If you have a long list of namelist items you want to set (such as a long list of history fields) a convenient way to do it is to add to the \sphinxcode{user\_nl\_clm} that is created after the \sphinxstylestrong{cesm\_setup} command runs. The file needs to be in valid FORTRAN namelist format (with the exception that the namelist name \&namelist and the end of namelist marker “/” are excluded”. The \sphinxstylestrong{preview\_namelist} or \sphinxstylestrong{\$CASE.run} step will abort if there are syntax errors. All the variable names must be valid and the values must be valid for the datatype and any restrictions for valid values for that variable. Here’s an example \sphinxcode{user\_nl\_clm} namelist that sets a bunch of history file related items, to create output history files monthly, daily, every six and 1 hours.


\paragraph{Example: user\_nl\_clm namelist file}
\label{\detokenize{users_guide/setting-up-and-running-a-case/customizing-the-clm-configuration:example-user-nl-clm-namelist-file}}
\begin{sphinxVerbatim}[commandchars=\\\{\}]
!\PYGZhy{}\PYGZhy{}\PYGZhy{}\PYGZhy{}\PYGZhy{}\PYGZhy{}\PYGZhy{}\PYGZhy{}\PYGZhy{}\PYGZhy{}\PYGZhy{}\PYGZhy{}\PYGZhy{}\PYGZhy{}\PYGZhy{}\PYGZhy{}\PYGZhy{}\PYGZhy{}\PYGZhy{}\PYGZhy{}\PYGZhy{}\PYGZhy{}\PYGZhy{}\PYGZhy{}\PYGZhy{}\PYGZhy{}\PYGZhy{}\PYGZhy{}\PYGZhy{}\PYGZhy{}\PYGZhy{}\PYGZhy{}\PYGZhy{}\PYGZhy{}\PYGZhy{}\PYGZhy{}\PYGZhy{}\PYGZhy{}\PYGZhy{}\PYGZhy{}\PYGZhy{}\PYGZhy{}\PYGZhy{}\PYGZhy{}\PYGZhy{}\PYGZhy{}\PYGZhy{}\PYGZhy{}\PYGZhy{}\PYGZhy{}\PYGZhy{}\PYGZhy{}\PYGZhy{}\PYGZhy{}\PYGZhy{}\PYGZhy{}\PYGZhy{}\PYGZhy{}\PYGZhy{}\PYGZhy{}\PYGZhy{}\PYGZhy{}\PYGZhy{}\PYGZhy{}\PYGZhy{}\PYGZhy{}\PYGZhy{}\PYGZhy{}\PYGZhy{}\PYGZhy{}\PYGZhy{}\PYGZhy{}\PYGZhy{}\PYGZhy{}\PYGZhy{}\PYGZhy{}\PYGZhy{}\PYGZhy{}\PYGZhy{}\PYGZhy{}\PYGZhy{}\PYGZhy{}
! Users should add all user specific namelist changes below in the form of
! namelist\PYGZus{}var = new\PYGZus{}namelist\PYGZus{}value
!
! Include namelist variables for drv\PYGZus{}flds\PYGZus{}in ONLY if \PYGZhy{}megan and/or \PYGZhy{}drydep options
! are set in the CLM\PYGZus{}NAMELIST\PYGZus{}OPTS env variable.
!
! EXCEPTIONS:
! Set co2\PYGZus{}ppmv           with CCSM\PYGZus{}CO2\PYGZus{}PPMV                      option
! Set dtime              with L\PYGZus{}NCPL                             option
! Set fatmlndfrc         with LND\PYGZus{}DOMAIN\PYGZus{}PATH/LND\PYGZus{}DOMAIN\PYGZus{}FILE    options
! Set finidat            with RUN\PYGZus{}REFCASE/RUN\PYGZus{}REFDATE/RUN\PYGZus{}REFTOD options for hybrid or branch cases
!                        (includes \PYGZdl{}inst\PYGZus{}string for multi\PYGZhy{}ensemble cases)
! Set glc\PYGZus{}grid           with GLC\PYGZus{}GRID                           option
! Set glc\PYGZus{}smb            with GLC\PYGZus{}SMB                            option
! Set maxpatch\PYGZus{}glcmec    with GLC\PYGZus{}NEC                            option
!\PYGZhy{}\PYGZhy{}\PYGZhy{}\PYGZhy{}\PYGZhy{}\PYGZhy{}\PYGZhy{}\PYGZhy{}\PYGZhy{}\PYGZhy{}\PYGZhy{}\PYGZhy{}\PYGZhy{}\PYGZhy{}\PYGZhy{}\PYGZhy{}\PYGZhy{}\PYGZhy{}\PYGZhy{}\PYGZhy{}\PYGZhy{}\PYGZhy{}\PYGZhy{}\PYGZhy{}\PYGZhy{}\PYGZhy{}\PYGZhy{}\PYGZhy{}\PYGZhy{}\PYGZhy{}\PYGZhy{}\PYGZhy{}\PYGZhy{}\PYGZhy{}\PYGZhy{}\PYGZhy{}\PYGZhy{}\PYGZhy{}\PYGZhy{}\PYGZhy{}\PYGZhy{}\PYGZhy{}\PYGZhy{}\PYGZhy{}\PYGZhy{}\PYGZhy{}\PYGZhy{}\PYGZhy{}\PYGZhy{}\PYGZhy{}\PYGZhy{}\PYGZhy{}\PYGZhy{}\PYGZhy{}\PYGZhy{}\PYGZhy{}\PYGZhy{}\PYGZhy{}\PYGZhy{}\PYGZhy{}\PYGZhy{}\PYGZhy{}\PYGZhy{}\PYGZhy{}\PYGZhy{}\PYGZhy{}\PYGZhy{}\PYGZhy{}\PYGZhy{}\PYGZhy{}\PYGZhy{}\PYGZhy{}\PYGZhy{}\PYGZhy{}\PYGZhy{}\PYGZhy{}\PYGZhy{}\PYGZhy{}\PYGZhy{}\PYGZhy{}\PYGZhy{}\PYGZhy{}
hist\PYGZus{}fincl2    = \PYGZsq{}TG\PYGZsq{},\PYGZsq{}TBOT\PYGZsq{},\PYGZsq{}FIRE\PYGZsq{},\PYGZsq{}FIRA\PYGZsq{},\PYGZsq{}FLDS\PYGZsq{},\PYGZsq{}FSDS\PYGZsq{},
                 \PYGZsq{}FSR\PYGZsq{},\PYGZsq{}FSA\PYGZsq{},\PYGZsq{}FGEV\PYGZsq{},\PYGZsq{}FSH\PYGZsq{},\PYGZsq{}FGR\PYGZsq{},\PYGZsq{}TSOI\PYGZsq{},
                 \PYGZsq{}ERRSOI\PYGZsq{},\PYGZsq{}BUILDHEAT\PYGZsq{},\PYGZsq{}SABV\PYGZsq{},\PYGZsq{}SABG\PYGZsq{},
                 \PYGZsq{}FSDSVD\PYGZsq{},\PYGZsq{}FSDSND\PYGZsq{},\PYGZsq{}FSDSVI\PYGZsq{},\PYGZsq{}FSDSNI\PYGZsq{},
                 \PYGZsq{}FSRVD\PYGZsq{},\PYGZsq{}FSRND\PYGZsq{},\PYGZsq{}FSRVI\PYGZsq{},\PYGZsq{}FSRNI\PYGZsq{},
                 \PYGZsq{}TSA\PYGZsq{},\PYGZsq{}FCTR\PYGZsq{},\PYGZsq{}FCEV\PYGZsq{},\PYGZsq{}QBOT\PYGZsq{},\PYGZsq{}RH2M\PYGZsq{},\PYGZsq{}H2OSOI\PYGZsq{},
                 \PYGZsq{}H2OSNO\PYGZsq{},\PYGZsq{}SOILLIQ\PYGZsq{},\PYGZsq{}SOILICE\PYGZsq{},
                 \PYGZsq{}TSA\PYGZus{}U\PYGZsq{}, \PYGZsq{}TSA\PYGZus{}R\PYGZsq{},
                 \PYGZsq{}TREFMNAV\PYGZus{}U\PYGZsq{}, \PYGZsq{}TREFMNAV\PYGZus{}R\PYGZsq{},
                 \PYGZsq{}TREFMXAV\PYGZus{}U\PYGZsq{}, \PYGZsq{}TREFMXAV\PYGZus{}R\PYGZsq{},
                 \PYGZsq{}TG\PYGZus{}U\PYGZsq{}, \PYGZsq{}TG\PYGZus{}R\PYGZsq{},
                 \PYGZsq{}RH2M\PYGZus{}U\PYGZsq{}, \PYGZsq{}RH2M\PYGZus{}R\PYGZsq{},
                 \PYGZsq{}QRUNOFF\PYGZus{}U\PYGZsq{}, \PYGZsq{}QRUNOFF\PYGZus{}R\PYGZsq{},
                 \PYGZsq{}SoilAlpha\PYGZus{}U\PYGZsq{},
                 \PYGZsq{}Qanth\PYGZsq{}, \PYGZsq{}SWup\PYGZsq{}, \PYGZsq{}LWup\PYGZsq{}, \PYGZsq{}URBAN\PYGZus{}AC\PYGZsq{}, \PYGZsq{}URBAN\PYGZus{}HEAT\PYGZsq{}
hist\PYGZus{}fincl3 = \PYGZsq{}TG:I\PYGZsq{}, \PYGZsq{}FSA:I\PYGZsq{}, \PYGZsq{}SWup:I\PYGZsq{}, \PYGZsq{}URBAN\PYGZus{}AC:I\PYGZsq{}, \PYGZsq{}URBAN\PYGZus{}HEAT:I\PYGZsq{},
              \PYGZsq{}TG\PYGZus{}U:I\PYGZsq{}, \PYGZsq{}TG\PYGZus{}R:I\PYGZsq{},
hist\PYGZus{}fincl4 = \PYGZsq{}TG\PYGZsq{}, \PYGZsq{}FSA\PYGZsq{}, \PYGZsq{}SWup\PYGZsq{}, \PYGZsq{}URBAN\PYGZus{}AC\PYGZsq{}, \PYGZsq{}URBAN\PYGZus{}HEAT\PYGZsq{}
hist\PYGZus{}mfilt  = 1, 30,  28, 24
hist\PYGZus{}nhtfrq = 0, \PYGZhy{}24, \PYGZhy{}6, \PYGZhy{}1
\end{sphinxVerbatim}

\sphinxstylestrong{Note:} The comments at the top are some guidance given in the default user\_nl\_clm and just give some guidance on how to set variables and use the file.

\sphinxstylestrong{Note:} See \sphinxhref{CLM-link}{the Section called Precedence of Options} section for the precedence of this option relative to the others.

\sphinxstylestrong{Note:} You do NOT need to specify the namelist group that the variables are in because the CLM \sphinxstylestrong{build-namelist} knows the namelist that specific variable names belong to, and it puts them there.

Obviously, all of this would be difficult to put in the CLM\_NAMELIST\_OPTS variable, especially having to put \&apos; around all the character strings. For more information on the namelist variables being set here and what they mean, see the section on CLM namelists below, as well as the namelist definition that gives details on each variable.


\paragraph{Precedence of Options}
\label{\detokenize{users_guide/setting-up-and-running-a-case/customizing-the-clm-configuration:precedence-of-options}}
Note: The precedence for setting the values of namelist variables with the different env\_build.xml, env\_run.xml options is (highest to lowest):
\begin{enumerate}
\item {} 
Namelist values set by specific command-line options, like, -d, -sim\_year (i.e. CLM\_BLDNML\_OPTS env\_build.xml variable)

\item {} 
Values set on the command-line using the -namelist option, (i.e. CLM\_NAMELIST\_OPTS env\_run.xml variable)

\item {} 
Values read from the file specified by -infile, (i.e. user\_nl\_clm file)

\item {} 
Datasets from the -clm\_usr\_name option, (i.e. CLM\_USRDAT\_NAME env\_run.xml variable)

\item {} 
Values set from a use-case scenario, e.g., -use\_case (i.e. CLM\_NML\_USE\_CASE env\_run.xml variable)

\item {} 
Values from the namelist defaults file.

\end{enumerate}

Thus a setting in \sphinxcode{CLM\_BLDNML\_OPTS} will override a setting for the same thing given in a use case with \sphinxcode{CLM\_NML\_USE\_CASE}. Likewise, a setting in \sphinxcode{CLM\_NAMELIST\_OPTS} will override a setting in \sphinxcode{user\_nl\_clm}.


\paragraph{Setting Your Initial Conditions File}
\label{\detokenize{users_guide/setting-up-and-running-a-case/customizing-the-clm-configuration:setting-your-initial-conditions-file}}
Especially with CLMBGC and CLMCN starting from initial conditions is very important. Even with CLMSP it takes many simulation years to get the model fully spunup. There are a couple different ways to provide an initial condition file.
\begin{itemize}
\item {} 
\sphinxhref{CLM-URL}{the Section called Doing a hybrid simulation to provide initial conditions}

\item {} 
\sphinxhref{CLM-URL}{the Section called Doing a branch simulation to provide initial conditions}

\item {} 
\sphinxhref{CLM-URL}{the Section called Providing a finidat file in your user\_nl\_clm file}

\item {} 
\sphinxhref{CLM-URL}{the Section called Adding a finidat file to the XML database}

\sphinxstylestrong{Note:} Your initial condition file MUST agree with the surface dataset you are using to run the simulation. If the two files do NOT agree you will get a run-time about a mis-match in PFT weights, or in the number of PFT’s or columns. To get around this you’ll need to use the \sphinxhref{CLM-URL}{Section called Using interpinic to interpolate initial conditions to different resolutions in Chapter 2} to interpolate your initial condition dataset.

\end{itemize}


\paragraph{Doing a hybrid simulation to provide initial conditions}
\label{\detokenize{users_guide/setting-up-and-running-a-case/customizing-the-clm-configuration:doing-a-hybrid-simulation-to-provide-initial-conditions}}
The first option is to setup a hybrid simulation and give a \sphinxcode{RUN\_REFCASE} and \sphinxcode{RUN\_REFDATE} to specify the reference case simulation name to use. When you setup most cases, at the standard resolutions of “f09” or “f19” it will already do this for you. For example, if you run an “I2000CN” compset at “f09\_g16” resolution the following settings will already be done for you.

\sphinxcode{./xmlchange RUN\_TYPE=hybrid,RUN\_REFCASE=I2000CN\_f09\_g16\_c100503,RUN\_REFDATE=0001-01-01,GET\_REFCASE=TRUE}

Setting the \sphinxcode{GET\_REFCASE} option to \sphinxcode{TRUE means} it will copy the files from the: \sphinxcode{\$DIN\_LOC\_ROOT/ccsm4\_init/I2000CN\_f09\_g16\_c100503/0001-01-01} directory. Note, that the \sphinxcode{RUN\_REFCASE} and \sphinxcode{RUN\_REFDATE} variables are expanded to get the directory name above. If you do NOT set \sphinxcode{GET\_REFCASE} to \sphinxcode{TRUE} then you will need to have placed the file in your run directory yourself. In either case, the file is expected to be named: \sphinxcode{\$RUN\_REFCASE.clm2.r.\$RUN\_REFDATE-00000.nc} with the variables expanded of course.


\paragraph{Doing a branch simulation to provide initial conditions}
\label{\detokenize{users_guide/setting-up-and-running-a-case/customizing-the-clm-configuration:doing-a-branch-simulation-to-provide-initial-conditions}}
The setup for running a branch simulation is essentially the same as for a hybrid. With the exception of setting \sphinxcode{RUN\_TYPE} to branch rather than hybrid. A branch simulation runs the case essentially as restarting from it’s place before to exactly reproduce it (but possibly output more or different fields on the history files). While a hybrid simulation allows you to change the configuration or run-time options, as well as use a different code base than the original case that may have fewer fields on it than a full restart file. The \sphinxcode{GET\_REFCASE} option works similarly for a branch case as for a hybrid.


\paragraph{Providing a finidat file in your user\_nl\_clm file}
\label{\detokenize{users_guide/setting-up-and-running-a-case/customizing-the-clm-configuration:providing-a-finidat-file-in-your-user-nl-clm-file}}
Setting up a branch or hybrid simulation requires the initial condition file to follow a standard naming convention, and a standard input directory if you use the \sphinxcode{GET\_REFCASE} option. If you want to name your file willy nilly and place it anywhere, you can set it in your \sphinxcode{user\_nl\_clm} file as in this example.

\begin{sphinxVerbatim}[commandchars=\\\{\}]
\PYG{n}{finidat}    \PYG{o}{=} \PYG{l+s+s1}{\PYGZsq{}}\PYG{l+s+s1}{/glade/home/\PYGZdl{}USER/myinitdata/clmi\PYGZus{}I1850CN\PYGZus{}f09\PYGZus{}g16\PYGZus{}0182\PYGZhy{}01\PYGZhy{}01.c120329.nc}\PYG{l+s+s1}{\PYGZsq{}}
\end{sphinxVerbatim}

Note, if you provide an initial condition file \textendash{} you can NOT set \sphinxcode{CLM\_FORCE\_COLDSTART} to \sphinxcode{TRUE}.


\paragraph{Adding a finidat file to the XML database}
\label{\detokenize{users_guide/setting-up-and-running-a-case/customizing-the-clm-configuration:adding-a-finidat-file-to-the-xml-database}}
Like other datasets, if you want to use a given initial condition file to be used for all (or most of) your cases you’ll want to put it in the XML database so it will be used by default. The initial condition files, are resolution dependent, and dependent on the number of PFT’s and other variables such as GLC\_NEC or if irrigation is on or off. See Chapter 3 for more information on this.


\paragraph{Other noteworthy configuration items}
\label{\detokenize{users_guide/setting-up-and-running-a-case/customizing-the-clm-configuration:other-noteworthy-configuration-items}}
For running “I” cases there are several other noteworthy configuration items that you may want to work with.
Most of these involve settings for the DATM, but one \sphinxcode{CCSM\_CO2\_PPMV} applies to all models.
If you are running an B, E, or F case that doesn’t use the DATM obviously the DATM\_* settings will not be used. All of the settings below are in your \sphinxcode{env\_build.xml} and \sphinxcode{env\_run.xml} files

\begin{sphinxVerbatim}[commandchars=\\\{\}]
\PYG{n}{CCSM\PYGZus{}CO2\PYGZus{}PPMV}
\PYG{n}{CCSM\PYGZus{}VOC}
\PYG{n}{DATM\PYGZus{}MODE}
\PYG{n}{DATM\PYGZus{}PRESAERO}
\PYG{n}{DATM\PYGZus{}CLMNCEP\PYGZus{}YR\PYGZus{}ALIGN}
\PYG{n}{DATM\PYGZus{}CLMNCEP\PYGZus{}YR\PYGZus{}START}
\PYG{n}{DATM\PYGZus{}CLMNCEP\PYGZus{}YR\PYGZus{}END}
\PYG{n}{DATM\PYGZus{}CPL\PYGZus{}CASE}
\PYG{n}{DATM\PYGZus{}CPL\PYGZus{}YR\PYGZus{}ALIGN}
\PYG{n}{DATM\PYGZus{}CPL\PYGZus{}YR\PYGZus{}START}
\PYG{n}{DATM\PYGZus{}CPL\PYGZus{}YR\PYGZus{}END}
\end{sphinxVerbatim}
\begin{description}
\item[{\sphinxcode{CCSM\_CO2\_PPMV}}] \leavevmode
Sets the mixing ratio of CO2 in parts per million by volume for ALL CESM components to use. Note that most compsets already set this value to something reasonable. Also note that some compsets may tell the atmosphere model to override this value with either historic or ramped values. If the CCSM\_BGC variable is set to something other than “none” the atmosphere model will determine CO2, and CLM will listen and use what the atmosphere sends it. On the CLM side the namelist item co2\_type tells CLM to use the value sent from the atmosphere rather than a value set on it’s own namelist.

\item[{\sphinxcode{CCSM\_VOC}}] \leavevmode
Enables passing of the Volatile Organic Compounds (VOC) from CLM to the atmospheric model. This of course is only important if the atmosphere model is a fully active model that can use these fields in it’s chemistry calculations.

\item[{\sphinxcode{DATM\_MODE}}] \leavevmode
Sets the mode that the DATM model should run in this determines how data is handled as well as what the source of the data will be. Many of the modes are setup specifically to be used for ocean and/or sea-ice modeling. The modes that are designed for use by CLM are:

\begin{sphinxVerbatim}[commandchars=\\\{\}]
\PYG{n}{CLMCRUNCEP}
\PYG{n}{CLM\PYGZus{}QIAN}
\PYG{n}{CLM1PT}\PYG{o}{\PYGZgt{}}
\PYG{n}{CPLHIST3HrWx}
\end{sphinxVerbatim}

\item[{\sphinxcode{CLMCRUNCEP}}] \leavevmode
The standard mode for CLM4.5 of using global atmospheric data that was developed by CRU using NCEP data from 1901 to 2010.
See \sphinxhref{CLM-URL}{the Section called CLMCRUNCEP mode and it’s DATM settings} for more information on the DATM settings for \sphinxcode{CLMCRUNCEP} mode.

\item[{\sphinxcode{CLM\_QIAN}}] \leavevmode
The standard mode for CLM4.0 of using global atmospheric data that was developed by Qian et. al. for CLM using NCEP data from 1948 to 2004. See the \sphinxhref{CLM-URL}{Section called CLM\_QIAN mode and it’s DATM settings} for more information on the DATM settings for \sphinxcode{CLM\_QIAN} mode. \sphinxcode{CLM1PT} is for the special cases where we have single-point tower data for particular sites. Right now we only have data for three urban locations: MexicoCity Mexico, Vancouver Canada, and the urban-c alpha site. And we have data for the US-UMB AmeriFlux tower site for University of Michigan Biological Station. See \sphinxhref{CLM-URL}{the Section called CLM1PT mode and it’s DATM settings} for more information on the DATM settings for \sphinxcode{CLM1PT} mode. \sphinxcode{CPLHIST3HrWx} is for running with atmospheric forcing from a previous CESM simulation. See \sphinxhref{CLM-URL}{the Section called CPLHIST3HrWx mode and it’s DATM settings} for more information on the DATM settings for \sphinxcode{CPLHIST3HrWx} mode.

\item[{\sphinxcode{DATM\_PRESAERO}}] \leavevmode
sets the prescribed aerosol mode for the data atmosphere model. The list of valid options include:

\sphinxcode{clim\_1850} = constant year 1850 conditions

\sphinxcode{clim\_2000} = constant year 2000 conditions

\sphinxcode{trans\_1850-2000} = transient 1850 to year 2000 conditions

\sphinxcode{rcp2.6} = transient conditions for the rcp=2.6 W/m2 future scenario

\sphinxcode{rcp4.5} = transient conditions for the rcp=4.5 W/m2 future scenario

\sphinxcode{rcp6.0} = transient conditions for the rcp=6.0 W/m2 future scenario

\sphinxcode{rcp8.5} = transient conditions for the rcp=8.5 W/m2 future scenario

\sphinxcode{pt1\_pt1} = read in single-point or regional datasets

\item[{DATM\_CLMNCEP\_YR\_START}] \leavevmode
\sphinxcode{DATM\_CLMNCEP\_YR\_START} sets the beginning year to cycle the atmospheric data over for \sphinxcode{CLM\_QIAN} or \sphinxcode{CLMCRUNCEP} modes.

\item[{DATM\_CLMNCEP\_YR\_END}] \leavevmode
\sphinxcode{DATM\_CLMNCEP\_YR\_END} sets the ending year to cycle the atmospheric data over for \sphinxcode{CLM\_QIAN} or \sphinxcode{CLMCRUNCEP} modes.

\item[{DATM\_CLMNCEP\_YR\_ALIGN}] \leavevmode
\sphinxcode{DATM\_CLMNCEP\_YR\_START} and \sphinxcode{DATM\_CLMNCEP\_YR\_END} determine the range of years to cycle the atmospheric data over, and \sphinxcode{DATM\_CLMNCEP\_YR\_ALIGN} determines which year in that range of years the simulation will start with.

\item[{DATM\_CPL\_CASE}] \leavevmode
\sphinxcode{DATM\_CPL\_CASE} sets the casename to use for the \sphinxcode{CPLHIST3HrWx} mode.

\item[{DATM\_CPL\_YR\_START}] \leavevmode
\sphinxcode{DATM\_CPL\_YR\_START} sets the beginning year to cycle the atmospheric data over for the \sphinxcode{CPLHIST3HrWx} mode.

\item[{DATM\_CPL\_YR\_END}] \leavevmode
\sphinxcode{DATM\_CPL\_YR\_END} sets the ending year to cycle the atmospheric data over for the \sphinxcode{CPLHIST3HrWx} mode.

\item[{DATM\_CPL\_YR\_ALIGN}] \leavevmode
\sphinxcode{DATM\_CPL\_YR\_START} and \sphinxcode{DATM\_CPL\_YR\_END} determine the range of years to cycle the atmospheric data over, and \sphinxcode{DATM\_CPL\_YR\_ALIGN} determines which year in that range of years the simulation will start with.

\end{description}


\paragraph{Downloading DATM Forcing Data}
\label{\detokenize{users_guide/setting-up-and-running-a-case/customizing-the-clm-configuration:downloading-datm-forcing-data}}
In Chapter One of the \sphinxhref{link-to-CESM-UG}{CESM User’s Guide} there is a section on “Downloading input data”. The normal process of setting up cases will use the “scripts/ccsm\_utils/Tools/check\_input\_data” script to retrieve data from the CESM subversion inputdata repository. This is true for the standard \sphinxtitleref{CLM\_QIAN} forcing as well.

The \sphinxtitleref{CLMCRUNCEP} data is uploaded into the subversion inputdata repository as well \textendash{} but as it is 1.1 Terabytes of data downloading it is problematic (\sphinxstyleemphasis{IT WILL TAKE SEVERAL DAYS TO DOWNLOAD THE ENTIRE DATASET USING SUBVERSION}). Because of it’s size you may also need to download it onto a separate disk space. We have done that on yellowstone for example where it resides in \sphinxcode{\$ENV\{CESMROOT\}/lmwg} while the rest of the input data resides in \sphinxcode{\$ENV\{CESMDATAROOT\}/inputdata}. The data is also already available on: janus, franklin, and hopper. If you download the data, we recommend that you break your download into several chunks, by setting up a case and setting the year range for \sphinxcode{DATM\_CPL\_YR\_START} and \sphinxcode{DATM\_CPL\_YR\_END} in say 20 year sections over 1901 to 2010, and then use \sphinxstylestrong{check\_input\_data} to export the data.

The \sphinxcode{CPLHIST3HrWx} DATM forcing data is unique \textendash{} because it is large compared to the rest of the input data, and we only have a disk copy on yellowstone. The DATM assumes the path for the previous NCAR machine yellowstone of \sphinxcode{/glade/p/cesm/shared\_outputdata/cases/ccsm4/\$DATM\_CPLHIST\_CASE} for the data. So you will need to change this path in order to run on any other machine. You can download the data itself from NCAR HPSS from \sphinxcode{/CCSM/csm/\$DATM\_CPLHIST\_CASE}.


\paragraph{Customizing via the build script files}
\label{\detokenize{users_guide/setting-up-and-running-a-case/customizing-the-clm-configuration:customizing-via-the-build-script-files}}
The final thing that the user may wish to do before \sphinxstylestrong{cesm\_setup} is run is to edit the build script files which determine the configuration and namelist. The variables in \sphinxcode{env\_build.xml} or \sphinxcode{env\_run.xml} typically mean you will NOT have to edit build script files. But, there are rare instances where it is useful to do so. The build script files are copied to your case directory and are available under Buildconf. The list of build script files you might wish to edit are:

\sphinxstylestrong{clm.buildexe.csh}
\sphinxstylestrong{clm.buildnml.csh}
\sphinxstylestrong{datm.buildexe.csh}
\sphinxstylestrong{datm.buildnml.csh}


\paragraph{More information on the CLM configure script}
\label{\detokenize{users_guide/setting-up-and-running-a-case/customizing-the-clm-configuration:more-information-on-the-clm-configure-script}}
The CLM \sphinxstylestrong{configure} script defines the details of a clm configuration and summarizes it into a \sphinxcode{config\_cache.xml} file. The \sphinxcode{config\_cache.xml} will be placed in your case directory under \sphinxcode{Buildconf/clmconf}. The \sphinxhref{CLM-URL}{config\_definition.xml} in \sphinxcode{models/lnd/clm/bld/config\_files} gives a definition of each CLM configuration item, it is viewable in a web-browser. Many of these items are things that you would NOT change, but looking through the list gives you the valid options, and a good description of each. Below we repeat the \sphinxcode{config\_definition.xml} files contents:


\subparagraph{Help on CLM configure}
\label{\detokenize{users_guide/setting-up-and-running-a-case/customizing-the-clm-configuration:help-on-clm-configure}}
Coupling this with looking at the options to CLM \sphinxstylestrong{configure} with “-help” as below will enable you to understand how to set the different options.

\begin{sphinxVerbatim}[commandchars=\\\{\}]
\PYG{o}{\PYGZgt{}} \PYG{n}{cd} \PYG{n}{models}\PYG{o}{/}\PYG{n}{lnd}\PYG{o}{/}\PYG{n}{clm}\PYG{o}{/}\PYG{n}{bld}
\PYG{o}{\PYGZgt{}} \PYG{n}{configure} \PYG{o}{\PYGZhy{}}\PYG{n}{help}
\end{sphinxVerbatim}

The output to the above command is as follows:

\begin{sphinxVerbatim}[commandchars=\\\{\}]
SYNOPSIS
  configure [options]

  Configure CLM in preparation to be built.
OPTIONS
  User supplied values are denoted in angle brackets (\PYGZlt{}\PYGZgt{}).  Any value that contains
  white\PYGZhy{}space must be quoted.  Long option names may be supplied with either single
  or double leading dashes.  A consequence of this is that single letter options may
  NOT be bundled.

  \PYGZhy{}bgc \PYGZlt{}name\PYGZgt{}            Build CLM with BGC package [ none \textbar{} cn \textbar{} cndv ]
                         (default is none).
  \PYGZhy{}cache \PYGZlt{}file\PYGZgt{}          Name of output cache file (default: config\PYGZus{}cache.xml).
  \PYGZhy{}cachedir \PYGZlt{}file\PYGZgt{}       Name of directory where output cache file is written
                         (default: CLM build directory).
  \PYGZhy{}clm4me \PYGZlt{}name\PYGZgt{}         Turn Methane model: [on \textbar{} off]
                           Requires bgc=cn/cndv (Carbon Nitrogen model)
                         (ONLY valid for CLM4.5!)
  \PYGZhy{}clm\PYGZus{}root \PYGZlt{}dir\PYGZgt{}        Root directory of clm source code
                         (default: directory above location of this script)
  \PYGZhy{}cppdefs \PYGZlt{}string\PYGZgt{}      A string of user specified CPP defines.  Appended to
                         Makefile defaults.  e.g. \PYGZhy{}cppdefs \PYGZsq{}\PYGZhy{}DVAR1 \PYGZhy{}DVAR2\PYGZsq{}
  \PYGZhy{}vichydro \PYGZlt{}name\PYGZgt{}       Turn VIC hydrologic parameterizations : [on \textbar{} off] (default is off)
  \PYGZhy{}crop \PYGZlt{}name\PYGZgt{}           Toggle for prognostic crop model. [on \textbar{} off] (default is off)
                         (can ONLY be turned on when BGC type is CN or CNDV)
  \PYGZhy{}comp\PYGZus{}intf \PYGZlt{}name\PYGZgt{}      Component interface to use (ESMF or MCT) (default MCT)
  \PYGZhy{}defaults \PYGZlt{}file\PYGZgt{}       Specify full path to a configuration file which will be used
                         to supply defaults instead of the defaults in bld/config\PYGZus{}files.
                         This file is used to specify model configuration parameters only.
                         Parameters relating to the build which are system dependent will
                         be ignored.
  \PYGZhy{}exlaklayers \PYGZlt{}name\PYGZgt{}    Turn on extra lake layers (25 layers instead of 10) [on \textbar{} off]
                         (ONLY valid for CLM4.5!)
  \PYGZhy{}help [or \PYGZhy{}h]          Print usage to STDOUT.
  \PYGZhy{}nofire                Turn off wildfires for BGC setting of CN
                         (default includes fire for CN)
  \PYGZhy{}noio                  Turn history output completely off (typically for testing).
  \PYGZhy{}phys \PYGZlt{}name\PYGZgt{}           Value of clm4\PYGZus{}0 or clm4\PYGZus{}5 (default is clm4\PYGZus{}0)
  \PYGZhy{}silent [or \PYGZhy{}s]        Turns on silent mode \PYGZhy{} only fatal messages issued.
  \PYGZhy{}sitespf\PYGZus{}pt \PYGZlt{}name\PYGZgt{}     Setup for the given site specific single\PYGZhy{}point resolution.
  \PYGZhy{}snicar\PYGZus{}frc \PYGZlt{}name\PYGZgt{}     Turn on SNICAR radiative forcing calculation. [on \textbar{} off]
                         (default is off)
  \PYGZhy{}spinup \PYGZlt{}name\PYGZgt{}         CLM 4.0 Only. For CLM 4.5, spinup is controlled from  build\PYGZhy{}namelist.
                         Turn on given spinup mode for BGC setting of CN               (level)
                           AD            Turn on Accelerated Decomposition from            (2)
                                         bare\PYGZhy{}soil
                           exit          Jump directly from AD spinup to normal mode       (1)
                           normal        Normal decomposition (\PYGZdq{}final spinup mode\PYGZdq{})        (0)
                                         (default)
                         The recommended sequence is 2\PYGZhy{}1\PYGZhy{}0
  \PYGZhy{}usr\PYGZus{}src \PYGZlt{}dir1\PYGZgt{}[,\PYGZlt{}dir2\PYGZgt{}[,\PYGZlt{}dir3\PYGZgt{}[...]]]
                         Directories containing user source code.
  \PYGZhy{}verbose [or \PYGZhy{}v]       Turn on verbose echoing of settings made by configure.
  \PYGZhy{}version               Echo the SVN tag name used to check out this CLM distribution.
  \PYGZhy{}vsoilc\PYGZus{}centbgc \PYGZlt{}name\PYGZgt{} Turn on vertical soil Carbon profile, CENTURY model decomposition, \PYGZbs{}

                         split Nitrification/de\PYGZhy{}Nitrification into two mineral
                         pools for NO3 and NH4 (requires clm4me Methane model), and
                         eliminate inconsistent duplicate soil hydraulic
                         parameters used in soil biogeochem.
                         (requires either CN or CNDV)
                         (ONLY valid for CLM4.5!)
                         [on,off or colon delimited list of no options] (default off)
                           no\PYGZhy{}vert     Turn vertical soil Carbon profile off
                           no\PYGZhy{}cent     Turn CENTURY off
                           no\PYGZhy{}nitrif   Turn the Nitrification/denitrification off
                         [no\PYGZhy{}vert,no\PYGZhy{}cent,no\PYGZhy{}nitrif,no\PYGZhy{}vert:no\PYGZhy{}cent]
\end{sphinxVerbatim}

We’ve given details on how to use the options in env\_build.xml and env\_run.xml to interact with the CLM “configure” and “build-namelist” scripts, as well as giving a good understanding of how these scripts work and the options to them.
In the next section we give further details on the CLM namelist. You could customize the namelist for these options after “cesm\_setup” is run.


\subsection{Customizing CLM’s namelist}
\label{\detokenize{users_guide/setting-up-and-running-a-case/customizing-the-clm-namelist::doc}}\label{\detokenize{users_guide/setting-up-and-running-a-case/customizing-the-clm-namelist:customizing-a-case}}\label{\detokenize{users_guide/setting-up-and-running-a-case/customizing-the-clm-namelist:customizing-clm-s-namelist}}
Once a case is \sphinxstylestrong{cesm\_setup}, we can then customize the case further, by editing the run-time namelist for CLM. First let’s list the definition of each namelist item and their valid values, and then we’ll list the default values for them. Next for some of the most used or tricky namelist items we’ll give examples of their use, and give you example namelists that highlight these features.

In the following, various examples of namelists are provided that feature the use of different namelist options to customize a case for particular uses.
Most the examples revolve around how to customize the output history fields.
This should give you a good basis for setting up your own CLM namelist.


\subsubsection{Definition of Namelist items and their default values}
\label{\detokenize{users_guide/setting-up-and-running-a-case/customizing-the-clm-namelist:definition-of-namelist-items-and-their-default-values}}
Here we point to you where you can find the definition of each namelist item and separately the default values for them. The default values may change depending on the resolution, land-mask, simulation-year and other attributes. Both of these files are viewable in your web browser, and then expand each in turn.
\begin{enumerate}
\item {} 
\sphinxhref{CLM-URL}{Definition of Namelists Relevant for CLM4.5}

\item {} 
\sphinxhref{CLM-URL}{Default values of each CLM4.0 Namelist Item}

\item {} 
\sphinxhref{CLM-URL}{Default values of each CLM4.5 Namelist Item}

\end{enumerate}


\paragraph{List of fields that can be added to your output history files by namelist}
\label{\detokenize{users_guide/setting-up-and-running-a-case/customizing-the-clm-namelist:list-of-fields-that-can-be-added-to-your-output-history-files-by-namelist}}
One set of the namelist items allows you to add fields to the output history files: \sphinxcode{hist\_fincl1}, \sphinxcode{hist\_fincl2}, \sphinxcode{hist\_fincl3}, \sphinxcode{hist\_fincl4}, \sphinxcode{hist\_fincl5}, and \sphinxcode{hist\_fincl6}. The following links for \sphinxhref{CLM-URL}{CLM4.0 History Fields} and \sphinxhref{CLM-URL}{CLM4.5 History Fields} documents all of the history fields available and gives the long-name and units for each. The table below lists all the CLM4.5 history fields.


\paragraph{Definition of CLM history variables}
\label{\detokenize{users_guide/setting-up-and-running-a-case/customizing-the-clm-namelist:definition-of-clm-history-variables}}
Included in the table are the following pieces of information:
\begin{itemize}
\item {} 
Variable name.

\item {} 
Long name description.

\item {} 
units

\end{itemize}


\paragraph{Table 1-3. CLM History Fields}
\label{\detokenize{users_guide/setting-up-and-running-a-case/customizing-the-clm-namelist:table-1-3-clm-history-fields}}
Table goes here.


\subsubsection{Examples of using different namelist features}
\label{\detokenize{users_guide/setting-up-and-running-a-case/customizing-the-clm-namelist:examples-of-using-different-namelist-features}}
Below we will give examples of user namelists that activate different commonly used namelist features. We will discuss the namelist features in different examples and then show a user namelist that includes an example of the use of these features. First we will show the default namelist that doesn’t activate any user options.


\paragraph{The default namelist}
\label{\detokenize{users_guide/setting-up-and-running-a-case/customizing-the-clm-namelist:the-default-namelist}}
Here we give the default namelist as it would be created for an “I1850CRUCLM45BGC” compset at 0.9x1.25 resolution with a gx1v6 land-mask on yellowstone. To edit the namelist you would edit the \sphinxcode{user\_nl\_clm} user namelist with just the items you want to change. For simplicity we will just show the CLM namelist and NOT the entire file. In the sections below, for simplicity we will just show the user namelist (\sphinxcode{user\_nl\_clm}) that will add (or modify existing) namelist items to the namelist.


\paragraph{Example 1-2. Default CLM Namelist}
\label{\detokenize{users_guide/setting-up-and-running-a-case/customizing-the-clm-namelist:example-1-2-default-clm-namelist}}
\begin{sphinxVerbatim}[commandchars=\\\{\}]
\PYG{o}{\PYGZam{}}\PYG{n}{clm\PYGZus{}inparm}
 \PYG{n}{albice} \PYG{o}{=} \PYG{l+m+mf}{0.60}\PYG{p}{,}\PYG{l+m+mf}{0.40}
 \PYG{n}{co2\PYGZus{}ppmv} \PYG{o}{=} \PYG{l+m+mf}{284.7}
 \PYG{n}{co2\PYGZus{}type} \PYG{o}{=} \PYG{l+s+s1}{\PYGZsq{}}\PYG{l+s+s1}{constant}\PYG{l+s+s1}{\PYGZsq{}}
 \PYG{n}{create\PYGZus{}crop\PYGZus{}landunit} \PYG{o}{=} \PYG{o}{.}\PYG{n}{false}\PYG{o}{.}
 \PYG{n}{dtime} \PYG{o}{=} \PYG{l+m+mi}{1800}
 \PYG{n}{fatmlndfrc} \PYG{o}{=} \PYG{l+s+s1}{\PYGZsq{}}\PYG{l+s+s1}{/glade/p/cesm/cseg/inputdata/share/domains/domain.lnd.fv0.9x1.25\PYGZus{}gx1v6.090309.nc}\PYG{l+s+s1}{\PYGZsq{}}
 \PYG{n}{finidat} \PYG{o}{=} \PYG{l+s+s1}{\PYGZsq{}}\PYG{l+s+s1}{\PYGZdl{}DIN\PYGZus{}LOC\PYGZus{}ROOT/lnd/clm2/initdata\PYGZus{}map/clmi.I1850CRUCLM45BGC.0241\PYGZhy{}01\PYGZhy{}01.0.9x1.25\PYGZus{}g1v6\PYGZus{}simyr1850\PYGZus{}c130531.nc}\PYG{l+s+s1}{\PYGZsq{}}
 \PYG{n}{fpftcon} \PYG{o}{=} \PYG{l+s+s1}{\PYGZsq{}}\PYG{l+s+s1}{/glade/p/cesm/cseg/inputdata/lnd/clm2/pftdata/pft\PYGZhy{}physiology.c130503.nc}\PYG{l+s+s1}{\PYGZsq{}}
 \PYG{n}{fsnowaging} \PYG{o}{=} \PYG{l+s+s1}{\PYGZsq{}}\PYG{l+s+s1}{/glade/p/cesm/cseg/inputdata/lnd/clm2/snicardata/snicar\PYGZus{}drdt\PYGZus{}bst\PYGZus{}fit\PYGZus{}60\PYGZus{}c070416.nc}\PYG{l+s+s1}{\PYGZsq{}}
 \PYG{n}{fsnowoptics} \PYG{o}{=} \PYG{l+s+s1}{\PYGZsq{}}\PYG{l+s+s1}{/glade/p/cesm/cseg/inputdata/lnd/clm2/snicardata/snicar\PYGZus{}optics\PYGZus{}5bnd\PYGZus{}c090915.nc}\PYG{l+s+s1}{\PYGZsq{}}
 \PYG{n}{fsurdat} \PYG{o}{=} \PYG{l+s+s1}{\PYGZsq{}}\PYG{l+s+s1}{/glade/p/cesm/cseg/inputdata/lnd/clm2/surfdata\PYGZus{}map/surfdata\PYGZus{}0.9x1.25\PYGZus{}simyr1850\PYGZus{}c130415.nc}\PYG{l+s+s1}{\PYGZsq{}}
 \PYG{n}{maxpatch\PYGZus{}glcmec} \PYG{o}{=} \PYG{l+m+mi}{0}
 \PYG{n}{more\PYGZus{}vertlayers} \PYG{o}{=} \PYG{o}{.}\PYG{n}{false}\PYG{o}{.}
 \PYG{n}{nsegspc} \PYG{o}{=} \PYG{l+m+mi}{20}
 \PYG{n}{spinup\PYGZus{}state} \PYG{o}{=} \PYG{l+m+mi}{0}
 \PYG{n}{urban\PYGZus{}hac} \PYG{o}{=} \PYG{l+s+s1}{\PYGZsq{}}\PYG{l+s+s1}{ON}\PYG{l+s+s1}{\PYGZsq{}}
 \PYG{n}{urban\PYGZus{}traffic} \PYG{o}{=} \PYG{o}{.}\PYG{n}{false}\PYG{o}{.}
\PYG{o}{/}
\PYG{o}{\PYGZam{}}\PYG{n}{ndepdyn\PYGZus{}nml}
 \PYG{n}{ndepmapalgo} \PYG{o}{=} \PYG{l+s+s1}{\PYGZsq{}}\PYG{l+s+s1}{bilinear}\PYG{l+s+s1}{\PYGZsq{}}
 \PYG{n}{stream\PYGZus{}fldfilename\PYGZus{}ndep} \PYG{o}{=} \PYG{l+s+s1}{\PYGZsq{}}\PYG{l+s+s1}{/glade/p/cesm/cseg/inputdata/lnd/clm2/ndepdata/fndep\PYGZus{}clm\PYGZus{}hist\PYGZus{}simyr1849\PYGZhy{}2006\PYGZus{}1.9x2.5\PYGZus{}c100428.nc}\PYG{l+s+s1}{\PYGZsq{}}
 \PYG{n}{stream\PYGZus{}year\PYGZus{}first\PYGZus{}ndep} \PYG{o}{=} \PYG{l+m+mi}{1850}
 \PYG{n}{stream\PYGZus{}year\PYGZus{}last\PYGZus{}ndep} \PYG{o}{=} \PYG{l+m+mi}{1850}
\PYG{o}{/}
\PYG{o}{\PYGZam{}}\PYG{n}{popd\PYGZus{}streams}
 \PYG{n}{popdensmapalgo} \PYG{o}{=} \PYG{l+s+s1}{\PYGZsq{}}\PYG{l+s+s1}{bilinear}\PYG{l+s+s1}{\PYGZsq{}}
 \PYG{n}{stream\PYGZus{}fldfilename\PYGZus{}popdens} \PYG{o}{=} \PYG{l+s+s1}{\PYGZsq{}}\PYG{l+s+s1}{\PYGZdl{}DIN\PYGZus{}LOC\PYGZus{}ROOT/lnd/clm2/firedata/clmforc.Li\PYGZus{}2012\PYGZus{}hdm\PYGZus{}0.5x0.5\PYGZus{}AVHRR\PYGZus{}simyr1850\PYGZhy{}2010\PYGZus{}c130401.nc}\PYG{l+s+s1}{\PYGZsq{}}
 \PYG{n}{stream\PYGZus{}year\PYGZus{}first\PYGZus{}popdens} \PYG{o}{=} \PYG{l+m+mi}{1850}
 \PYG{n}{stream\PYGZus{}year\PYGZus{}last\PYGZus{}popdens} \PYG{o}{=} \PYG{l+m+mi}{1850}
\PYG{o}{/}
\PYG{o}{\PYGZam{}}\PYG{n}{light\PYGZus{}streams}
 \PYG{n}{lightngmapalgo} \PYG{o}{=} \PYG{l+s+s1}{\PYGZsq{}}\PYG{l+s+s1}{bilinear}\PYG{l+s+s1}{\PYGZsq{}}
 \PYG{n}{stream\PYGZus{}fldfilename\PYGZus{}lightng} \PYG{o}{=} \PYG{l+s+s1}{\PYGZsq{}}\PYG{l+s+s1}{/glade/p/cesm/cseg/inputdata/atm/datm7/NASA\PYGZus{}LIS/clmforc.Li\PYGZus{}2012\PYGZus{}climo1995\PYGZhy{}2011.T62.lnfm\PYGZus{}c130327.nc}\PYG{l+s+s1}{\PYGZsq{}}
 \PYG{n}{stream\PYGZus{}year\PYGZus{}first\PYGZus{}lightng} \PYG{o}{=} \PYG{l+m+mi}{0001}
 \PYG{n}{stream\PYGZus{}year\PYGZus{}last\PYGZus{}lightng} \PYG{o}{=} \PYG{l+m+mi}{0001}
\PYG{o}{/}
\PYG{o}{\PYGZam{}}\PYG{n}{clm\PYGZus{}hydrology1\PYGZus{}inparm}
\PYG{o}{/}
\PYG{o}{\PYGZam{}}\PYG{n}{clm\PYGZus{}soilhydrology\PYGZus{}inparm}
\PYG{o}{/}
\PYG{o}{\PYGZam{}}\PYG{n}{ch4par\PYGZus{}in}
 \PYG{n}{fin\PYGZus{}use\PYGZus{}fsat} \PYG{o}{=} \PYG{o}{.}\PYG{n}{true}\PYG{o}{.}
\PYG{o}{/}
\end{sphinxVerbatim}


\paragraph{Adding/removing fields on your primary history file}
\label{\detokenize{users_guide/setting-up-and-running-a-case/customizing-the-clm-namelist:adding-removing-fields-on-your-primary-history-file}}
The primary history files are output monthly, and contain an extensive list of fieldnames, but the list of fieldnames can be added to using \sphinxcode{hist\_fincl1} or removed from by adding fieldnames to \sphinxcode{hist\_fexcl1}.
A sample user namelist \sphinxcode{user\_nl\_clm} adding few new fields (cosine of solar zenith angle, and solar declination) and excluding a few standard fields is (ground temperature, vegetation temperature, soil temperature and soil water).:


\paragraph{Example 1-3. Example user\_nl\_clm namelist adding and removing fields on primary history file}
\label{\detokenize{users_guide/setting-up-and-running-a-case/customizing-the-clm-namelist:example-1-3-example-user-nl-clm-namelist-adding-and-removing-fields-on-primary-history-file}}
\begin{sphinxVerbatim}[commandchars=\\\{\}]
\PYG{n}{hist\PYGZus{}fincl1} \PYG{o}{=} \PYG{l+s+s1}{\PYGZsq{}}\PYG{l+s+s1}{COSZEN}\PYG{l+s+s1}{\PYGZsq{}}\PYG{p}{,} \PYG{l+s+s1}{\PYGZsq{}}\PYG{l+s+s1}{DECL}\PYG{l+s+s1}{\PYGZsq{}}
\PYG{n}{hist\PYGZus{}fexcl1} \PYG{o}{=} \PYG{l+s+s1}{\PYGZsq{}}\PYG{l+s+s1}{TG}\PYG{l+s+s1}{\PYGZsq{}}\PYG{p}{,} \PYG{l+s+s1}{\PYGZsq{}}\PYG{l+s+s1}{TV}\PYG{l+s+s1}{\PYGZsq{}}\PYG{p}{,} \PYG{l+s+s1}{\PYGZsq{}}\PYG{l+s+s1}{TSOI}\PYG{l+s+s1}{\PYGZsq{}}\PYG{p}{,} \PYG{l+s+s1}{\PYGZsq{}}\PYG{l+s+s1}{H2OSOI}\PYG{l+s+s1}{\PYGZsq{}}
\end{sphinxVerbatim}


\paragraph{Adding auxiliary history files and changing output frequency}
\label{\detokenize{users_guide/setting-up-and-running-a-case/customizing-the-clm-namelist:adding-auxiliary-history-files-and-changing-output-frequency}}
The \sphinxcode{hist\_fincl2} through \sphinxcode{hist\_fincl6} set of namelist variables add given history fieldnames to auxiliary history file “streams”, and \sphinxcode{hist\_fexcl2} through \sphinxcode{hist\_fexcl6} set of namelist variables remove given history fieldnames from history file auxiliary “streams”.
A history “stream” is a set of history files that are produced at a given frequency.
By default there is only one stream of monthly data files.
To add more streams you add history fieldnames to \sphinxcode{hist\_fincl2} through \sphinxcode{hist\_fincl6}.
The output frequency and the way averaging is done can be different for each history file stream.
By default the primary history files are monthly and any others are daily. You can have up to six active history streams, but you need to activate them in order. So if you activate stream “6” by setting \sphinxcode{hist\_fincl6}, but if any of \sphinxcode{hist\_fincl2} through \sphinxcode{hist\_fincl5} are unset, only the history streams up to the first blank one will be activated.

The frequency of the history file streams is given by the namelist variable \sphinxcode{hist\_nhtfrq} which is an array of rank six for each history stream. The values of the array \sphinxcode{hist\_nhtfrq} must be integers, where the following values have the given meaning:

\sphinxstyleemphasis{Positive value} means the output frequency is the number of model steps between output.
\sphinxstyleemphasis{Negative value} means the output frequency is the absolute value in hours given (i.e -1 would mean an hour and -24 would mean a full day). Daily (-24) is the default value for all auxiliary files.
\sphinxstyleemphasis{Zero} means the output frequency is monthly. This is the default for the primary history files.

The number of samples on each history file stream is given by the namelist variable \sphinxcode{hist\_mfilt} which is an array of rank six for each history stream. The values of the array \sphinxcode{hist\_mfilt} must be positive integers. By default the primary history file stream has one time sample on it (i.e. output is to separate monthly files), and all other streams have thirty time samples on them.

A sample user namelist \sphinxcode{user\_nl\_clm} turning on four extra file streams for output: daily, six-hourly, hourly, and every time-step, leaving the primary history files as monthly, and changing the number of samples on the streams to: yearly (12), thirty, weekly (28), daily (24), and daily (48) is:


\paragraph{Example: user\_nl\_clm namelist adding auxiliary history files and changing output frequency}
\label{\detokenize{users_guide/setting-up-and-running-a-case/customizing-the-clm-namelist:example-user-nl-clm-namelist-adding-auxiliary-history-files-and-changing-output-frequency}}
\begin{sphinxVerbatim}[commandchars=\\\{\}]
\PYG{n}{hist\PYGZus{}fincl2} \PYG{o}{=} \PYG{l+s+s1}{\PYGZsq{}}\PYG{l+s+s1}{TG}\PYG{l+s+s1}{\PYGZsq{}}\PYG{p}{,} \PYG{l+s+s1}{\PYGZsq{}}\PYG{l+s+s1}{TV}\PYG{l+s+s1}{\PYGZsq{}}
\PYG{n}{hist\PYGZus{}fincl3} \PYG{o}{=} \PYG{l+s+s1}{\PYGZsq{}}\PYG{l+s+s1}{TG}\PYG{l+s+s1}{\PYGZsq{}}\PYG{p}{,} \PYG{l+s+s1}{\PYGZsq{}}\PYG{l+s+s1}{TV}\PYG{l+s+s1}{\PYGZsq{}}
\PYG{n}{hist\PYGZus{}fincl4} \PYG{o}{=} \PYG{l+s+s1}{\PYGZsq{}}\PYG{l+s+s1}{TG}\PYG{l+s+s1}{\PYGZsq{}}\PYG{p}{,} \PYG{l+s+s1}{\PYGZsq{}}\PYG{l+s+s1}{TV}\PYG{l+s+s1}{\PYGZsq{}}
\PYG{n}{hist\PYGZus{}fincl5} \PYG{o}{=} \PYG{l+s+s1}{\PYGZsq{}}\PYG{l+s+s1}{TG}\PYG{l+s+s1}{\PYGZsq{}}\PYG{p}{,} \PYG{l+s+s1}{\PYGZsq{}}\PYG{l+s+s1}{TV}\PYG{l+s+s1}{\PYGZsq{}}
\PYG{n}{hist\PYGZus{}nhtfrq} \PYG{o}{=} \PYG{l+m+mi}{0}\PYG{p}{,} \PYG{o}{\PYGZhy{}}\PYG{l+m+mi}{24}\PYG{p}{,} \PYG{o}{\PYGZhy{}}\PYG{l+m+mi}{6}\PYG{p}{,} \PYG{o}{\PYGZhy{}}\PYG{l+m+mi}{1}\PYG{p}{,} \PYG{l+m+mi}{1}
\PYG{n}{hist\PYGZus{}mfilt}  \PYG{o}{=} \PYG{l+m+mi}{12}\PYG{p}{,} \PYG{l+m+mi}{30}\PYG{p}{,} \PYG{l+m+mi}{28}\PYG{p}{,} \PYG{l+m+mi}{24}\PYG{p}{,} \PYG{l+m+mi}{48}
\end{sphinxVerbatim}


\paragraph{Removing all history fields}
\label{\detokenize{users_guide/setting-up-and-running-a-case/customizing-the-clm-namelist:removing-all-history-fields}}
Sometimes for various reasons you want to remove all the history fields either because you want to do testing without any output, or you only want a very small custom list of output fields rather than the default extensive list of fields.
By default only the primary history files are active, so technically using \sphinxcode{hist\_fexcl1} explained in the first example, you could list ALL of the history fields that are output in \sphinxcode{hist\_fexcl1} and then you wouldn’t get any output.
However, as the list is very extensive this would be a cumbersome thing to do.
So to facilitate this \sphinxcode{hist\_empty\_htapes} allows you to turn off all default output.
You can still use \sphinxcode{hist\_fincl1} to turn your own list of fields on, but you then start from a clean slate.
A sample user namelist \sphinxcode{user\_nl\_clm} turning off all history fields and then activating just a few selected fields (ground and vegetation temperatures and absorbed solar radiation) is:


\paragraph{Example 1-5. Example user\_nl\_clm namelist removing all history fields}
\label{\detokenize{users_guide/setting-up-and-running-a-case/customizing-the-clm-namelist:example-1-5-example-user-nl-clm-namelist-removing-all-history-fields}}
\begin{sphinxVerbatim}[commandchars=\\\{\}]
\PYG{n}{hist\PYGZus{}empty\PYGZus{}htapes} \PYG{o}{=} \PYG{o}{.}\PYG{n}{true}\PYG{o}{.}
\PYG{n}{hist\PYGZus{}fincl1} \PYG{o}{=} \PYG{l+s+s1}{\PYGZsq{}}\PYG{l+s+s1}{TG}\PYG{l+s+s1}{\PYGZsq{}}\PYG{p}{,} \PYG{l+s+s1}{\PYGZsq{}}\PYG{l+s+s1}{TV}\PYG{l+s+s1}{\PYGZsq{}}\PYG{p}{,} \PYG{l+s+s1}{\PYGZsq{}}\PYG{l+s+s1}{FSA}\PYG{l+s+s1}{\PYGZsq{}}
\end{sphinxVerbatim}


\paragraph{Various ways to change history output averaging flags}
\label{\detokenize{users_guide/setting-up-and-running-a-case/customizing-the-clm-namelist:various-ways-to-change-history-output-averaging-flags}}
There are two ways to change the averaging of output history fields.
The first is using \sphinxcode{hist\_avgflag\_pertape} which gives a default value for each history stream, the second is when you add fields using \sphinxcode{hist\_fincl*}, you add an averaging flag to the end of the field name after a colon (for example ‘TSOI:X’, would output the maximum of TSOI).
The types of averaging that can be done are:
\begin{itemize}
\item {} 
\sphinxstyleemphasis{A} Average, over the output interval.

\item {} 
\sphinxstyleemphasis{I} Instantaneous, output the value at the output interval.

\item {} 
\sphinxstyleemphasis{X} Maximum, over the output interval.

\item {} 
\sphinxstyleemphasis{M} Minimum, over the output interval.

\end{itemize}

The default averaging depends on the specific fields, but for most fields is an average.
A sample user namelist \sphinxcode{user\_nl\_clm} making the monthly output fields all averages (except TSOI for the first two streams and FIRE for the 5th stream), and adding auxiliary file streams for instantaneous (6-hourly), maximum (daily), minimum (daily), and average (daily).
For some of the fields we diverge from the per-tape value given and customize to some different type of optimization.


\paragraph{Example: user\_nl\_clm namelist with various ways to average history fields}
\label{\detokenize{users_guide/setting-up-and-running-a-case/customizing-the-clm-namelist:example-user-nl-clm-namelist-with-various-ways-to-average-history-fields}}
\begin{sphinxVerbatim}[commandchars=\\\{\}]
\PYG{n}{hist\PYGZus{}empty\PYGZus{}htapes} \PYG{o}{=} \PYG{o}{.}\PYG{n}{true}\PYG{o}{.}
\PYG{n}{hist\PYGZus{}fincl1} \PYG{o}{=} \PYG{l+s+s1}{\PYGZsq{}}\PYG{l+s+s1}{TSOI:X}\PYG{l+s+s1}{\PYGZsq{}}\PYG{p}{,} \PYG{l+s+s1}{\PYGZsq{}}\PYG{l+s+s1}{TG}\PYG{l+s+s1}{\PYGZsq{}}\PYG{p}{,}   \PYG{l+s+s1}{\PYGZsq{}}\PYG{l+s+s1}{TV}\PYG{l+s+s1}{\PYGZsq{}}\PYG{p}{,}   \PYG{l+s+s1}{\PYGZsq{}}\PYG{l+s+s1}{FIRE}\PYG{l+s+s1}{\PYGZsq{}}\PYG{p}{,}   \PYG{l+s+s1}{\PYGZsq{}}\PYG{l+s+s1}{FSR}\PYG{l+s+s1}{\PYGZsq{}}\PYG{p}{,} \PYG{l+s+s1}{\PYGZsq{}}\PYG{l+s+s1}{FSH}\PYG{l+s+s1}{\PYGZsq{}}\PYG{p}{,}
              \PYG{l+s+s1}{\PYGZsq{}}\PYG{l+s+s1}{EFLX\PYGZus{}LH\PYGZus{}TOT}\PYG{l+s+s1}{\PYGZsq{}}\PYG{p}{,} \PYG{l+s+s1}{\PYGZsq{}}\PYG{l+s+s1}{WT}\PYG{l+s+s1}{\PYGZsq{}}
\PYG{n}{hist\PYGZus{}fincl2} \PYG{o}{=} \PYG{l+s+s1}{\PYGZsq{}}\PYG{l+s+s1}{TSOI:X}\PYG{l+s+s1}{\PYGZsq{}}\PYG{p}{,} \PYG{l+s+s1}{\PYGZsq{}}\PYG{l+s+s1}{TG}\PYG{l+s+s1}{\PYGZsq{}}\PYG{p}{,}   \PYG{l+s+s1}{\PYGZsq{}}\PYG{l+s+s1}{TV}\PYG{l+s+s1}{\PYGZsq{}}\PYG{p}{,}   \PYG{l+s+s1}{\PYGZsq{}}\PYG{l+s+s1}{FIRE}\PYG{l+s+s1}{\PYGZsq{}}\PYG{p}{,}   \PYG{l+s+s1}{\PYGZsq{}}\PYG{l+s+s1}{FSR}\PYG{l+s+s1}{\PYGZsq{}}\PYG{p}{,} \PYG{l+s+s1}{\PYGZsq{}}\PYG{l+s+s1}{FSH}\PYG{l+s+s1}{\PYGZsq{}}\PYG{p}{,}
              \PYG{l+s+s1}{\PYGZsq{}}\PYG{l+s+s1}{EFLX\PYGZus{}LH\PYGZus{}TOT}\PYG{l+s+s1}{\PYGZsq{}}\PYG{p}{,} \PYG{l+s+s1}{\PYGZsq{}}\PYG{l+s+s1}{WT}\PYG{l+s+s1}{\PYGZsq{}}
\PYG{n}{hist\PYGZus{}fincl3} \PYG{o}{=} \PYG{l+s+s1}{\PYGZsq{}}\PYG{l+s+s1}{TSOI}\PYG{l+s+s1}{\PYGZsq{}}\PYG{p}{,}   \PYG{l+s+s1}{\PYGZsq{}}\PYG{l+s+s1}{TG:I}\PYG{l+s+s1}{\PYGZsq{}}\PYG{p}{,} \PYG{l+s+s1}{\PYGZsq{}}\PYG{l+s+s1}{TV}\PYG{l+s+s1}{\PYGZsq{}}\PYG{p}{,}   \PYG{l+s+s1}{\PYGZsq{}}\PYG{l+s+s1}{FIRE}\PYG{l+s+s1}{\PYGZsq{}}\PYG{p}{,}   \PYG{l+s+s1}{\PYGZsq{}}\PYG{l+s+s1}{FSR}\PYG{l+s+s1}{\PYGZsq{}}\PYG{p}{,} \PYG{l+s+s1}{\PYGZsq{}}\PYG{l+s+s1}{FSH}\PYG{l+s+s1}{\PYGZsq{}}\PYG{p}{,}
              \PYG{l+s+s1}{\PYGZsq{}}\PYG{l+s+s1}{EFLX\PYGZus{}LH\PYGZus{}TOT}\PYG{l+s+s1}{\PYGZsq{}}\PYG{p}{,} \PYG{l+s+s1}{\PYGZsq{}}\PYG{l+s+s1}{WT}\PYG{l+s+s1}{\PYGZsq{}}
\PYG{n}{hist\PYGZus{}fincl4} \PYG{o}{=} \PYG{l+s+s1}{\PYGZsq{}}\PYG{l+s+s1}{TSOI}\PYG{l+s+s1}{\PYGZsq{}}\PYG{p}{,}   \PYG{l+s+s1}{\PYGZsq{}}\PYG{l+s+s1}{TG}\PYG{l+s+s1}{\PYGZsq{}}\PYG{p}{,}   \PYG{l+s+s1}{\PYGZsq{}}\PYG{l+s+s1}{TV:I}\PYG{l+s+s1}{\PYGZsq{}}\PYG{p}{,} \PYG{l+s+s1}{\PYGZsq{}}\PYG{l+s+s1}{FIRE}\PYG{l+s+s1}{\PYGZsq{}}\PYG{p}{,}   \PYG{l+s+s1}{\PYGZsq{}}\PYG{l+s+s1}{FSR}\PYG{l+s+s1}{\PYGZsq{}}\PYG{p}{,} \PYG{l+s+s1}{\PYGZsq{}}\PYG{l+s+s1}{FSH}\PYG{l+s+s1}{\PYGZsq{}}\PYG{p}{,}
              \PYG{l+s+s1}{\PYGZsq{}}\PYG{l+s+s1}{EFLX\PYGZus{}LH\PYGZus{}TOT}\PYG{l+s+s1}{\PYGZsq{}}\PYG{p}{,} \PYG{l+s+s1}{\PYGZsq{}}\PYG{l+s+s1}{WT}\PYG{l+s+s1}{\PYGZsq{}}
\PYG{n}{hist\PYGZus{}fincl5} \PYG{o}{=} \PYG{l+s+s1}{\PYGZsq{}}\PYG{l+s+s1}{TSOI}\PYG{l+s+s1}{\PYGZsq{}}\PYG{p}{,}   \PYG{l+s+s1}{\PYGZsq{}}\PYG{l+s+s1}{TG}\PYG{l+s+s1}{\PYGZsq{}}\PYG{p}{,}   \PYG{l+s+s1}{\PYGZsq{}}\PYG{l+s+s1}{TV}\PYG{l+s+s1}{\PYGZsq{}}\PYG{p}{,}   \PYG{l+s+s1}{\PYGZsq{}}\PYG{l+s+s1}{FIRE:I}\PYG{l+s+s1}{\PYGZsq{}}\PYG{p}{,} \PYG{l+s+s1}{\PYGZsq{}}\PYG{l+s+s1}{FSR}\PYG{l+s+s1}{\PYGZsq{}}\PYG{p}{,} \PYG{l+s+s1}{\PYGZsq{}}\PYG{l+s+s1}{FSH}\PYG{l+s+s1}{\PYGZsq{}}\PYG{p}{,}
              \PYG{l+s+s1}{\PYGZsq{}}\PYG{l+s+s1}{EFLX\PYGZus{}LH\PYGZus{}TOT}\PYG{l+s+s1}{\PYGZsq{}}\PYG{p}{,} \PYG{l+s+s1}{\PYGZsq{}}\PYG{l+s+s1}{WT}\PYG{l+s+s1}{\PYGZsq{}}
\PYG{n}{hist\PYGZus{}avgflag\PYGZus{}pertape} \PYG{o}{=} \PYG{l+s+s1}{\PYGZsq{}}\PYG{l+s+s1}{A}\PYG{l+s+s1}{\PYGZsq{}}\PYG{p}{,} \PYG{l+s+s1}{\PYGZsq{}}\PYG{l+s+s1}{I}\PYG{l+s+s1}{\PYGZsq{}}\PYG{p}{,} \PYG{l+s+s1}{\PYGZsq{}}\PYG{l+s+s1}{X}\PYG{l+s+s1}{\PYGZsq{}}\PYG{p}{,}   \PYG{l+s+s1}{\PYGZsq{}}\PYG{l+s+s1}{M}\PYG{l+s+s1}{\PYGZsq{}}\PYG{p}{,} \PYG{l+s+s1}{\PYGZsq{}}\PYG{l+s+s1}{A}\PYG{l+s+s1}{\PYGZsq{}}
\PYG{n}{hist\PYGZus{}nhtfrq} \PYG{o}{=} \PYG{l+m+mi}{0}\PYG{p}{,} \PYG{o}{\PYGZhy{}}\PYG{l+m+mi}{6}\PYG{p}{,} \PYG{o}{\PYGZhy{}}\PYG{l+m+mi}{24}\PYG{p}{,} \PYG{o}{\PYGZhy{}}\PYG{l+m+mi}{24}\PYG{p}{,} \PYG{o}{\PYGZhy{}}\PYG{l+m+mi}{24}
\end{sphinxVerbatim}

In the example we put the same list of fields on each of the tapes: soil-temperature, ground temperature, vegetation temperature, emitted longwave radiation, reflected solar radiation, sensible heat, total latent-heat, and total water storage.
We also modify the soil-temperature for the primary and secondary auxiliary tapes by outputting them for a maximum instead of the prescribed per-tape of average and instantaneous respectively.
For the tertiary auxiliary tape we output ground temperature instantaneous instead of as a maximum, and for the fourth auxiliary tape we output vegetation temperature instantaneous instead of as a minimum.
Finally, for the fifth auxiliary tapes we output \sphinxcode{FIRE} instantaneously instead of as an average.

\begin{sphinxadmonition}{note}{Note:}
We also use \sphinxcode{hist\_empty\_htapes} as in the previous example, so we can list ONLY the fields that we want on the primary history tapes.
\end{sphinxadmonition}


\paragraph{Outputting history files as a vector in order to analyze the plant function types within gridcells}
\label{\detokenize{users_guide/setting-up-and-running-a-case/customizing-the-clm-namelist:outputting-history-files-as-a-vector-in-order-to-analyze-the-plant-function-types-within-gridcells}}
By default the output to history files are the grid-cell average of all land-units, and vegetation types within that grid-cell, and output is on the full 2D latitude/longitude grid with ocean masked out.
Sometimes it’s important to understand how different land-units or vegetation types are acting within a grid-cell.
The way to do this is to output history files as a 1D-vector of all land-units and vegetation types.
In order to display this, you’ll need to do extensive post-processing to make sense of the output.
Often you may only be interested in a few points, so once you figure out the 1D indices for the grid-cells of interest, you can easily view that data.
1D vector output can also be useful for single point datasets, since it’s then obvious that all data is for the same grid cell.

To do this you use \sphinxcode{hist\_dov2xy} which is an array of rank six for each history stream.
Set it to \sphinxcode{.false.} if you want one of the history streams to be a 1D vector.
You can also use \sphinxcode{hist\_type1d\_pertape} if you want to average over all the: Plant-Function-Types, columns, land-units, or grid-cells.
A sample user namelist \sphinxcode{user\_nl\_clm} leaving the primary monthly files as 2D, and then doing grid-cell (GRID), column (COLS), and no averaging over auxiliary tapes output daily for a single field (ground temperature) is:


\paragraph{Example: user\_nl\_clm namelist outputting some files in 1D Vector format}
\label{\detokenize{users_guide/setting-up-and-running-a-case/customizing-the-clm-namelist:example-user-nl-clm-namelist-outputting-some-files-in-1d-vector-format}}
\begin{sphinxVerbatim}[commandchars=\\\{\}]
\PYG{n}{hist\PYGZus{}fincl2} \PYG{o}{=} \PYG{l+s+s1}{\PYGZsq{}}\PYG{l+s+s1}{TG}\PYG{l+s+s1}{\PYGZsq{}}
\PYG{n}{hist\PYGZus{}fincl3} \PYG{o}{=} \PYG{l+s+s1}{\PYGZsq{}}\PYG{l+s+s1}{TG}\PYG{l+s+s1}{\PYGZsq{}}
\PYG{n}{hist\PYGZus{}fincl4} \PYG{o}{=} \PYG{l+s+s1}{\PYGZsq{}}\PYG{l+s+s1}{TG}\PYG{l+s+s1}{\PYGZsq{}}
\PYG{n}{hist\PYGZus{}fincl5} \PYG{o}{=} \PYG{l+s+s1}{\PYGZsq{}}\PYG{l+s+s1}{TG}\PYG{l+s+s1}{\PYGZsq{}}
\PYG{n}{hist\PYGZus{}fincl6} \PYG{o}{=} \PYG{l+s+s1}{\PYGZsq{}}\PYG{l+s+s1}{TG}\PYG{l+s+s1}{\PYGZsq{}}
\PYG{n}{hist\PYGZus{}dov2xy} \PYG{o}{=} \PYG{o}{.}\PYG{n}{true}\PYG{o}{.}\PYG{p}{,} \PYG{o}{.}\PYG{n}{false}\PYG{o}{.}\PYG{p}{,} \PYG{o}{.}\PYG{n}{false}\PYG{o}{.}\PYG{p}{,} \PYG{o}{.}\PYG{n}{false}\PYG{o}{.}
\PYG{n}{hist\PYGZus{}type2d\PYGZus{}pertape} \PYG{o}{=} \PYG{l+s+s1}{\PYGZsq{}}\PYG{l+s+s1}{ }\PYG{l+s+s1}{\PYGZsq{}}\PYG{p}{,} \PYG{l+s+s1}{\PYGZsq{}}\PYG{l+s+s1}{GRID}\PYG{l+s+s1}{\PYGZsq{}}\PYG{p}{,} \PYG{l+s+s1}{\PYGZsq{}}\PYG{l+s+s1}{COLS}\PYG{l+s+s1}{\PYGZsq{}}\PYG{p}{,} \PYG{l+s+s1}{\PYGZsq{}}\PYG{l+s+s1}{ }\PYG{l+s+s1}{\PYGZsq{}}
\PYG{n}{hist\PYGZus{}nhtfrq} \PYG{o}{=} \PYG{l+m+mi}{0}\PYG{p}{,} \PYG{o}{\PYGZhy{}}\PYG{l+m+mi}{24}\PYG{p}{,} \PYG{o}{\PYGZhy{}}\PYG{l+m+mi}{24}\PYG{p}{,} \PYG{o}{\PYGZhy{}}\PYG{l+m+mi}{24}
\end{sphinxVerbatim}

\begin{sphinxadmonition}{warning}{Warning:}
LAND and COLS are also options to the pertape averaging, but currently there is a bug with them and they fail to work.
\end{sphinxadmonition}

\begin{sphinxadmonition}{note}{Note:}
Technically the default for hist\_nhtfrq is for primary files output monthly and the other auxiliary tapes for daily, so we don’t actually have to include hist\_nhtfrq, we could use the default for it. Here we specify it for clarity.
\end{sphinxadmonition}

\begin{sphinxadmonition}{caution}{Caution:}
LAND and COLS are also options to the pertape averaging, but currently there is a bug with them and they fail to work.
\end{sphinxadmonition}

Visualizing global 1D vector files will take effort.
You’ll probably want to do some post-processing and possibly just extract out single points of interest to see what is going on.
Since, the output is a 1D vector, of only land-points traditional plots won’t be helpful.
The number of points per grid-cell will also vary for anything, but grid-cell averaging.
You’ll need to use the output fields pfts1d\_ixy, and pfts1d\_jxy, to get the mapping of the fields to the global 2D array.
pfts1d\_itype\_veg gives you the PFT number for each PFT.
Most likely you’ll want to do this analysis in a data processing tool (such as NCL, Matlab, Mathmatica, IDL, etcetera that is able to read and process NetCDF data files).


\subsection{Customizing the DATM namelist}
\label{\detokenize{users_guide/setting-up-and-running-a-case/customizing-the-datm-namelist:customizing-the-datm-namelist}}\label{\detokenize{users_guide/setting-up-and-running-a-case/customizing-the-datm-namelist::doc}}
When running “I” compsets with CLM you use the DATM model to give atmospheric forcing data to CLM. There are two ways to customize DATM:
\begin{enumerate}
\item {} 
\sphinxstylestrong{DATM Main Namelist and Stream Namlist gorup} (\sphinxcode{datm\_in})

\item {} 
\sphinxstylestrong{DATM stream files}

\end{enumerate}

The \sphinxhref{CLM-URL}{Data Model Documentation} gives the details of all the options for the data models and for DATM specifically.
It goes into detail on all namelist items both for DATM and for DATM streams.
So here we won’t list ALL of the DATM namelist options, nor go into great details about stream files.
But, we will talk about a few of the different options that are relevant for running with CLM.
All of the options for changing the namelists or stream files is done by editing the \sphinxcode{user\_nl\_datm} file.

Because, they aren’t useful for work with CLM we will NOT discuss any of the options for the main DATM namelist. Use the DATM Users Guide at the link above to find details of that. For the streams namelist we will discuss three items:
\begin{enumerate}
\item {} 
\sphinxcode{mapalgo}

\item {} 
\sphinxcode{taxmode}

\item {} 
\sphinxcode{tintalgo}

\end{enumerate}

And for the streams file itself we will discuss:
\begin{description}
\item[{offset}] \leavevmode
Again everything else (and including the above items) are discussed in the Data Model User’s Guide. Of the above the last three: offset, taxmode and tintalgo are all closely related and have to do with the time interpolation of the DATM data.

\item[{mapalgo}] \leavevmode
\sphinxcode{mapalgo} sets the spatial interpolation method to go from the DATM input data to the output DATM model grid. The default is \sphinxcode{bilinear}. For \sphinxcode{CLM1PT} we set it to \sphinxcode{nn} to just select the nearest neighbor. This saves time and we also had problems running the interpolation for single-point mode.

\item[{taxmode}] \leavevmode
\sphinxcode{taxmode} is the time axis mode.
For CLM we usually have it set to \sphinxcode{cycle} which means that once the end of the data is reached it will start over at the beginning.
The extend modes is used have it use the last time-step of the forcing data once it reaches the end of forcing data (or use the first time-step before it reaches where the forcing data starts). See the warning below about the extend mode.

\end{description}

\begin{sphinxadmonition}{warning}{Warning:}
\sphinxstyleemphasis{THE extend OPTION NEEDS TO BE USED WITH CAUTION!} It is only invoked by default for the CLM1PT mode and is only intended for the supported urban datasets to extend the data for a single time-step. If you have the model \sphinxstyleemphasis{run extensively through periods in this mode you will effectively be repeating that last time-step over that entire period}. This means the output of your simulation will be worthless.
\end{sphinxadmonition}
\begin{description}
\item[{offset (in the stream file)}] \leavevmode
\sphinxcode{offset} is the time offset in seconds to give to each stream of data. Normally it is NOT used because the time-stamps for data is set correctly for each stream of data. Note, the \sphinxcode{offset} may NEED to be adjusted depending on the \sphinxcode{taxmode} described above, or it may need to be adjusted to account for data that is time-stamped at the END of an interval rather than the middle or beginning of interval. The \sphinxcode{offset} can is set in the stream file rather than on the stream namelist. For data with a \sphinxcode{taxmode} method of \sphinxcode{coszen} the time-stamp needs to be for the beginning of the interval, while for other data it should be the midpoint. The \sphinxcode{offset} can be used to adjust the time-stamps to get the data to line up correctly.

\item[{tintalgo}] \leavevmode
\sphinxcode{tintalgo} is the time interpolation algorithm. For CLM we usually use one of three modes: \sphinxcode{coszen}, \sphinxcode{nearest}, or \sphinxcode{linear}. We use \sphinxcode{coszen} for solar data, nearest for precipitation data, and linear for everything else. If your data is half-hourly or hourly, nearest will work fine for everything. The \sphinxcode{coszen} scaling is useful for longer periods (three hours or more) to try to get the solar to match the cosine of the solar zenith angle over that longer period of time. If you use linear for longer intervals, the solar will cut out at night-time anyway, and the straight line will be a poor approximation of the cosine of the solar zenith angle of actual solar data. nearest likewise would be bad for longer periods where it would be much higher than the actual values.

\end{description}

\begin{sphinxadmonition}{note}{Note:}
For \sphinxcode{coszen} the time-stamps of the data should correspond to the beginning of the interval the data is measured for. Either make sure the time-stamps on the datafiles is set this way, or use the \sphinxcode{offset} described above to set it.
\end{sphinxadmonition}

\begin{sphinxadmonition}{note}{Note:}
For nearest and linear the time-stamps of the data should correspond to the middle of the interval the data is measured for. Either make sure the time-stamps on the datafiles is set this way, or use the \sphinxcode{offset} described above to set it.
\end{sphinxadmonition}

In the sections below we go over each of the relevant \sphinxcode{DATM\_MODE} options and what the above DATM settings are for each. This gives you examples of actual usage for the settings. We also describe in what ways you might want to customize them for your own case.


\subsubsection{CLMCRUNCEP mode and it’s DATM settings}
\label{\detokenize{users_guide/setting-up-and-running-a-case/customizing-the-datm-namelist:clmcruncep-mode-and-it-s-datm-settings}}
In \sphinxcode{CLMCRUNCEP} mode the CRUNCEP dataset is used and all of it’s data is on a 6-hourly interval.
Like \sphinxcode{CLM\_QIAN} the dataset is divided into those three data streams: solar, precipitation, and everything else (temperature, pressure, humidity and wind).
The time-stamps of the data were also adjusted so that they are the beginning of the interval for solar, and the middle for the other two.
Because, of this the \sphinxcode{offset} is set to zero, and the \sphinxcode{tintalgo} is: \sphinxcode{coszen}, \sphinxcode{nearest}, and \sphinxcode{linear} for the solar, precipitation and other data respectively.
\sphinxcode{taxmode} is set to \sphinxcode{cycle} and \sphinxcode{mapalgo} is set to \sphinxcode{bilinear} so that the data is spatially interpolated from the input exact half degree grid to the grid the atmosphere model is being run at (to run at this same model resolution use the 360x720cru\_360x720cru resolution).

\begin{sphinxadmonition}{note}{Note:}
The “everything else” data stream (of temperature, pressure, humidity and wind) also includes the data for longwave downward forcing as well. Our simulations showed sensitivity to this field, so we backed off in using it, and let DATM calculate longwave down from the other fields.
\end{sphinxadmonition}

For more information on CRUNCEP forcing see \sphinxurl{http://dods.extra.cea.fr/data/p529viov/cruncep/}.


\subsubsection{CLM\_QIAN mode and it’s DATM settings}
\label{\detokenize{users_guide/setting-up-and-running-a-case/customizing-the-datm-namelist:clm-qian-mode-and-it-s-datm-settings}}
In \sphinxcode{CLM\_QIAN} mode the Qian dataset is used which has 6-hourly solar and precipitation data, and 3-hourly for everything else.
The dataset is divided into those three data streams: solar, precipitation, and everything else (temperature, pressure, humidity and wind).
The time-stamps of the data were also adjusted so that they are the beginning of the interval for solar, and the middle for the other two.
Because, of this the \sphinxcode{offset} is set to zero, and the \sphinxcode{tintalgo} is: \sphinxcode{coszen}, \sphinxcode{nearest}, and \sphinxcode{linear} for the solar, precipitation and other data respectively.
\sphinxcode{taxmode} is set to \sphinxcode{cycle} and \sphinxcode{mapalgo} is set to \sphinxcode{bilinear} so that the data is spatially interpolated from the input T62 grid to the grid the atmosphere model is being run at.

Normally you wouldn’t customize the \sphinxcode{CLM\_QIAN} settings, but you might replicate it’s use for your own global data that had similar temporal characteristics.


\subsubsection{CLM1PT mode and it’s DATM settings}
\label{\detokenize{users_guide/setting-up-and-running-a-case/customizing-the-datm-namelist:clm1pt-mode-and-it-s-datm-settings}}
In \sphinxcode{CLM1PT} mode the model is assumed to have half-hourly or hourly data for a single-point.
For the supported datasets that is exactly what it has.
But, if you add your own data you may need to make adjustments accordingly.
Using the \sphinxcode{CLM\_USRDAT\_NAME} resolution you can easily extend this mode for your own datasets that may be regional or even global and could be at different temporal frequencies.
If you do so you’ll need to make adjustments to your DATM settings.
The dataset has all data in a single stream file.
The time-stamps of the data were also adjusted so that they are at the middle of the interval.
Because, of this the \sphinxcode{offset} is set to zero, and the \sphinxcode{tintalgo} is set to \sphinxcode{nearest}.
\sphinxcode{taxmode} is set to \sphinxcode{extend} and \sphinxcode{mapalgo} is set to \sphinxcode{nn} so that simply the nearest point is used.

If you are using your own data for this mode and it’s not at least hourly you’ll want to adjust the DATM settings for it. If the data is three or six hourly, you’ll need to divide it up into separate streams like in \sphinxcode{CLM\_QIAN} mode which will require fairly extensive changes to the DATM namelist and streams files. For an example of doing this see \sphinxhref{CLM-URL}{Example 5-8}.


\subsubsection{CPLHIST3HrWx mode and it’s DATM settings}
\label{\detokenize{users_guide/setting-up-and-running-a-case/customizing-the-datm-namelist:cplhist3hrwx-mode-and-it-s-datm-settings}}
In \sphinxcode{CPLHIST3HrWx} mode the model is assumed to have 3-hourly for a global grid from a previous CESM simulation.
Like \sphinxcode{CLM\_QIAN} mode the data is divided into three streams: one for precipitation, one for solar, and one for everything else.
The time-stamps for Coupler history files for CESM is at the end of the interval, so the \sphinxcode{offset} needs to be set in order to adjust the time-stamps to what it needs to be for the \sphinxcode{tintalgo} settings.
For precipitation \sphinxcode{taxmode} is set to \sphinxcode{nearest} so the \sphinxcode{offset} is set to \sphinxcode{-5400} seconds so that the ending time-step is adjusted by an hour and half to the middle of the interval.
For solar \sphinxcode{taxmode} is set to \sphinxcode{coszen} so the offset is set to \sphinxcode{-10800} seconds so that the ending time-step is adjust by three hours to the beginning of the interval.
For everything else \sphinxcode{taxmode} is set to \sphinxcode{linear} so the offset is set to \sphinxcode{-5400} seconds so that the ending time-step is adjusted by an hour and half to the middle of the interval.
For an example of such a case see \sphinxhref{CLM-URL}{the Section called Running with MOAR data as atmospheric forcing to spinup the model in Chapter 4}.

Normally you wouldn’t modify the DATM settings for this mode.
However, if you had data at a different frequency than 3-hours you would need to modify the \sphinxcode{offset} and possibly the \sphinxcode{taxmode}.
The other two things that you might modify would be the path to the data or the domain file for the resolution (which is currently hardwired to f09).
For data at a different input resolution you would need to change the domain file in the streams file to use a domain file to the resolution that the data comes in on.


\section{Using CLM tools}
\label{\detokenize{users_guide/using-clm-tools/index:using-clm-tools}}\label{\detokenize{users_guide/using-clm-tools/index:using-clm-tools-section}}\label{\detokenize{users_guide/using-clm-tools/index::doc}}

\subsection{What are the CLM tools}
\label{\detokenize{users_guide/using-clm-tools/what-are-the-clm-tools:what-are-the-clm-tools}}\label{\detokenize{users_guide/using-clm-tools/what-are-the-clm-tools::doc}}\label{\detokenize{users_guide/using-clm-tools/what-are-the-clm-tools:id1}}
There are several tools provided with CLM that allow you to create your own input datasets at resolutions you choose, or to interpolate initial conditions to a different resolution, or used to compare CLM history files between different cases.
The tools are all available in the \sphinxcode{models/lnd/clm/tools} directory.
Most of the tools are FORTRAN stand-alone programs in their own directory, but there is also a suite of NCL scripts in the \sphinxcode{shared/ncl\_scripts} directory, and some of the tools are scripts that may also call the ESMF regridding program.
Some of the NCL scripts are very specialized and not meant for general use, and we won’t document them here.
They still contain documentation in the script itself and the README file in the tools directory.

The tools are divided into three directories for three categories: clm4\_0, clm4\_5, and shared.
The first two are of course for tools that are designed to work with either the CLM4.0 or CLM4.5 versions of the model.
The last one are shared utilities that can be used by either, or have a “-phys” option so you can specify which version you want to use.

The list of generally important scripts and programs are as follows.
\begin{enumerate}
\item {} 
\sphinxstyleemphasis{tools/cprnc} (relative to top level directory) to compare NetCDF files with a time axis.

\item {} 
\sphinxstyleemphasis{shared/mkmapgrids} to create SCRIP grid data files from old CLM format grid files that can then be used to create new CLM datasets (deprecated). There is also a NCL script (\sphinxcode{shared/mkmapgrids/mkscripgrid.ncl} to create SCRIP grid files for regular latitude/longitude grids.

\item {} 
\sphinxstyleemphasis{shared/mkmapdata} to create SCRIP mapping data file from SCRIP grid files (uses ESMF).

\item {} 
\sphinxstyleemphasis{shared/gen\_domain} to create a domain file for datm from a mapping file. The domain file is then used by BOTH datm AND CLM to define the grid and land-mask.

\item {} 
\sphinxstyleemphasis{mksurfdata\_map} to create surface datasets from grid datasets (clm4\_0 and clm4\_5 versions).

\item {} 
\sphinxstyleemphasis{shared/mkprocdata\_map} to interpolate output unstructured grids (such as the CAM HOMME dy-core “ne” grids like ne30np4) into a 2D regular lat/long grid format that can be plotted easily. Can be used by either clm4\_0 or clm4\_5.

\end{enumerate}

In the sections to come we will go into detailed description of how to use each of these tools in turn.
First, however we will discuss the common environment variables and options that are used by all of the FORTRAN tools.
Second, we go over the outline of the entire file creation process for all input files needed by CLM for a new resolution, then we turn to each tool.
In the last section we will discuss how to customize files for particular observational sites.

The tools run either one of two ways, with a namelist to provide options, or with command line arguments (and NOT both).
\sphinxstylestrong{gen\_domain} and \sphinxstylestrong{cprnc} run with command line arguments, and the other tools run with namelists.

In the following sections, we will outline how to make these files available for build-namelist so that you can easily create simulations that include them.
In the chapter on single-point and regional datasets we also give an alternative way to enter new datasets without having to edit files.


\subsubsection{Running FORTRAN tools with namelists}
\label{\detokenize{users_guide/using-clm-tools/what-are-the-clm-tools:running-fortran-tools-with-namelists}}
\sphinxstylestrong{mksurfdata\_map} and \sphinxstylestrong{mkmapgrids} run with namelists that are read from standard input.
Hence, you create a namelist and then run them by redirecting the namelist file into standard input as follows:

\begin{sphinxVerbatim}[commandchars=\\\{\}]
\PYG{o}{.}\PYG{o}{/}\PYG{n}{program} \PYG{o}{\PYGZlt{}} \PYG{n}{namelist}
\end{sphinxVerbatim}

For programs with namelists there is at least one sample namelist with the name “program”.namelist (i.e.
\sphinxcode{mksurfdata\_map.namelist} for the \sphinxstylestrong{mksurfdata\_map} program).
There may also be other sample namelists that end in a different name besides “namelist”.
Namelists that you create should be similar to the example namelist.
The namelist values are also documented along with the other namelists in the:

\begin{sphinxVerbatim}[commandchars=\\\{\}]
models/lnd/clm/bld/namelist\PYGZus{}files/namelist\PYGZus{}definition.xml{}`{}` file
     and default values in the:
models/lnd/clm/bld/namelist\PYGZus{}files/namelist\PYGZus{}defaults\PYGZus{}clm\PYGZus{}tools.xml{}`{}` file.
\end{sphinxVerbatim}


\subsubsection{Running FORTRAN tools with command line options}
\label{\detokenize{users_guide/using-clm-tools/what-are-the-clm-tools:running-fortran-tools-with-command-line-options}}
\sphinxstylestrong{gen\_domain}, and \sphinxstylestrong{cprnc} run with command line arguments.
The detailed sections below will give you more information on the command line arguments specific to each tool.
Also running the tool without any arguments will give you a general synopsis on how to run the tool.


\subsubsection{Running FORTRAN tools built with SMP=TRUE}
\label{\detokenize{users_guide/using-clm-tools/what-are-the-clm-tools:running-fortran-tools-built-with-smp-true}}
When you enable \sphinxcode{SMP=TRUE} on your build of one of the tools that make use of it, you are using OpenMP for shared memory parallelism (SMP).
In SMP loops are run in parallel with different threads run on different processors all of which access the same memory (called on-node).
Thus you can only usefully run up to the number of processors that are available on a single-node of the machine you are running on.
For example, on the NCAR machine yellowstone there are 16 processors per node, but the SMT hardware on the machine allows you to submit twice as many threads or 32 threads.


\subsubsection{Using NCL}
\label{\detokenize{users_guide/using-clm-tools/what-are-the-clm-tools:using-ncl}}
In the tools directory \sphinxcode{models/lnd/clm/tools/shared/ncl\_scripts} and in a few other locations there are scripts that use NCAR Command Language (NCL).
Unlike the FORTRAN tools, you will need to get a copy of NCL in order to use them.
You also won’t have to build an executable in order to use them, hence no Makefile is provided.
NCL is provided for free download as either binaries or source code from: \sphinxurl{http://www.ncl.ucar.edu/}.
The NCL web-site also contains documentation on NCL and it’s use. These scripts are stand-alone and at most use environment variables to control how to use them. In some cases there are perl scripts with command line arguments that call the NCL scripts to control what they do.


\subsection{Building the CLM tools}
\label{\detokenize{users_guide/using-clm-tools/building-the-clm-tools::doc}}\label{\detokenize{users_guide/using-clm-tools/building-the-clm-tools:building-the-clm-tools}}
The FORTRAN tools all have similar makefiles, and similar options for building.
All of the Makefiles use GNU Make extensions and thus require that you use GNU make to use them.
They also auto detect the type of platform you are on, using “uname -s” and set the compiler, compiler flags and such accordingly.
There are also environment variables that can be set to set things that must be customized.
All the tools use NetCDF and hence require the path to the NetCDF libraries and include files.
On some platforms (such as Linux) multiple compilers can be used, and hence there are env variables that can be set to change the FORTRAN and/or “C” compilers used.
The tools other than cprnc also allow finer control, by also allowing the user to add compiler flags they choose, for both FORTRAN and “C”, as well as picking the compiler, linker and and add linker options.
Finally the tools other than \sphinxstylestrong{cprnc} allow you to turn optimization on (which is off by default but on for the \sphinxstylestrong{mksurfdata\_map} and \sphinxstylestrong{interpinic} programs) with the OPT flag so that the tool will run faster.
To get even faster performance, the \sphinxstylestrong{interpinic}, program allows you to also use the SMP to turn on multiple shared memory processors.
When \sphinxcode{SMP=TRUE} you set the number of threads used by the program with the \sphinxcode{OMP\_NUM\_THREADS} environment variable.

Options used by all: \sphinxstylestrong{cprnc}, \sphinxstylestrong{interpinic}, and \sphinxstylestrong{mksurfdata\_map}
\begin{itemize}
\item {} 
\sphinxcode{LIB\_NETCDF} \textendash{} sets the location of the NetCDF library.

\item {} 
\sphinxcode{INC\_NETCDF} \textendash{} sets the location of the NetCDF include files.

\item {} 
\sphinxcode{USER\_FC} \textendash{} sets the name of the FORTRAN compiler.

\end{itemize}

Options used by: \sphinxstylestrong{interpinic}, \sphinxstylestrong{mkprocdata\_map}, \sphinxstylestrong{mkmapgrids}, and \sphinxstylestrong{mksurfdata\_map}
\begin{itemize}
\item {} 
\sphinxcode{MOD\_NETCDF} \textendash{} sets the location of the NetCDF FORTRAN module.

\item {} 
\sphinxcode{USER\_LINKER} \textendash{} sets the name of the linker to use.

\item {} 
\sphinxcode{USER\_CPPDEFS} \textendash{} adds any CPP defines to use.

\item {} 
\sphinxcode{USER\_CFLAGS} \textendash{} add any “C” compiler flags to use.

\item {} 
\sphinxcode{USER\_FFLAGS} \textendash{} add any FORTRAN compiler flags to use.

\item {} 
\sphinxcode{USER\_LDFLAGS} \textendash{} add any linker flags to use.

\item {} 
\sphinxcode{USER\_CC} \textendash{} sets the name of the “C” compiler to use.

\item {} 
\sphinxcode{OPT} \textendash{} set to TRUE to compile the code optimized (TRUE or FALSE)

\item {} 
\sphinxcode{SMP} \textendash{} set to TRUE to turn on shared memory parallelism (i.e. OpenMP) (TRUE or FALSE)

\item {} 
\sphinxcode{Filepath} \textendash{} list of directories to build source code from.

\item {} 
\sphinxcode{Srcfiles} \textendash{} list of source code filenames to build executable from.

\item {} 
\sphinxcode{Makefile} \textendash{} customized makefile options for this particular tool.

\item {} 
\sphinxcode{mkDepends} \textendash{} figure out dependencies between source files, so make can compile in order..

\item {} 
\sphinxcode{Makefile.common} \textendash{} General tool Makefile that should be the same between all tools.

\end{itemize}

Options used only by \sphinxstylestrong{cprnc}:
\begin{itemize}
\item {} 
\sphinxcode{EXEDIR} \textendash{} sets the location where the executable will be built.

\item {} 
\sphinxcode{VPATH} \textendash{} colon delimited path list to find the source files.

\end{itemize}

More details on each environment variable.
\begin{description}
\item[{\sphinxcode{LIB\_NETCDF}}] \leavevmode
This variable sets the path to the NetCDF library file (\sphinxcode{libnetcdf.a}). If not set it defaults to \sphinxcode{/usr/local/lib}. In order to use the tools you need to build the NetCDF library and be able to link to it. In order to build the model with a particular compiler you may have to compile the NetCDF library with the same compiler (or at least a compatible one).

\item[{\sphinxcode{INC\_NETCDF}}] \leavevmode
This variable sets the path to the NetCDF include directory (in order to find the include file \sphinxcode{netcdf.inc}). if not set it defaults to \sphinxcode{/usr/local/include}.

\item[{\sphinxcode{MOD\_NETCDF}}] \leavevmode
This variable sets the path to the NetCDF module directory (in order to find the NetCDF FORTRAN-90 module file when NetCDF is used with a FORTRAN-90 \sphinxstylestrong{use statement}. When not set it defaults to the \sphinxcode{LIB\_NETCDF} value.

\item[{\sphinxcode{USER\_FC}}] \leavevmode
This variable sets the command name to the FORTRAN-90 compiler to use when compiling the tool. The default compiler to use depends on the platform. And for example, on the AIX platform this variable is NOT used

\item[{\sphinxcode{USER\_LINKER}}] \leavevmode
This variable sets the command name to the linker to use when linking the object files from the compiler together to build the executable. By default this is set to the value of the FORTRAN-90 compiler used to compile the source code.

\item[{\sphinxcode{USER\_CPPDEFS}}] \leavevmode
This variable adds additional optional values to define for the C preprocessor. Normally, there is no reason to do this as there are very few CPP tokens in the CLM tools. However, if you modify the tools there may be a reason to define new CPP tokens.

\item[{\sphinxcode{USER\_CC}}] \leavevmode
This variable sets the command name to the “C” compiler to use when compiling the tool. The default compiler to use depends on the platform. And for example, on the AIX platform this variable is NOT used

\item[{\sphinxcode{USER\_CFLAGS}}] \leavevmode
This variable adds additional compiler options for the “C” compiler to use when compiling the tool. By default the compiler options are picked according to the platform and compiler that will be used.

\item[{\sphinxcode{USER\_FFLAGS}}] \leavevmode
This variable adds additional compiler options for the FORTRAN-90 compiler to use when compiling the tool. By default the compiler options are picked according to the platform and compiler that will be used.

\item[{\sphinxcode{USER\_LDFLAGS}}] \leavevmode
This variable adds additional options to the linker that will be used when linking the object files into the executable. By default the linker options are picked according to the platform and compiler that is used.

\item[{\sphinxcode{SMP}}] \leavevmode
This variable flags if shared memory parallelism (using OpenMP) should be used when compiling the tool. It can be set to either TRUE or FALSE, by default it is set to FALSE, so shared memory parallelism is NOT used. When set to TRUE you can set the number of threads by using the OMP\_NUM\_THREADS environment variable. Normally, the most you would set this to would be to the number of on-node CPU processors. Turning this on should make the tool run much faster.

\end{description}

\begin{sphinxadmonition}{warning}{Warning:}
Note, that depending on the compiler answers may be different when SMP is activated.
\end{sphinxadmonition}
\begin{description}
\item[{\sphinxcode{OPT}}] \leavevmode
This variable flags if compiler optimization should be used when compiling the tool. It can be set to either \sphinxcode{TRUE} or \sphinxcode{FALSE}, by default it is set to \sphinxcode{FALSE} for \sphinxstylestrong{mkmapgrids} and \sphinxcode{TRUE} for \sphinxstylestrong{mksurfdata\_map}, \sphinxstylestrong{mkprocdata\_map} and \sphinxstylestrong{interpinic}. Turning this on should make the tool run much faster.

\end{description}

\begin{sphinxadmonition}{warning}{Warning:}
Note, you should expect that answers will be different when \sphinxcode{OPT} is activated.
\end{sphinxadmonition}
\begin{description}
\item[{\sphinxcode{Filepath}}] \leavevmode
All of the tools are stand-alone and don’t need any outside code to operate. The Filepath is the list of directories needed to compile and hence is always simply “.” the current directory. Several tools use copies of code outside their directory that is in the CESM distribution (either \sphinxcode{csm\_share} code or CLM source code).

\item[{\sphinxcode{Srcfiles}}] \leavevmode
The \sphinxcode{Srcfiles} lists the filenames of the source code to use when building the tool.

\item[{\sphinxcode{Makefile}}] \leavevmode
The \sphinxcode{Makefile} is the custom GNU Makefile for this particular tool. It will customize the \sphinxcode{EXENAME} and the optimization settings for this particular tool.

\item[{\sphinxcode{Makefile.common}}] \leavevmode
The \sphinxcode{Makefile.common} is the copy of the general GNU Makefile for all the CLM tools. This file should be identical between the different tools. This file has different sections of compiler options for different Operating Systems and compilers.

\item[{\sphinxcode{mkDepends}}] \leavevmode
The \sphinxcode{mkDepends} is the copy of the perl script used by the \sphinxcode{Makefile.common} to figure out the dependencies between the source files so that it can compile in the necessary order. This file should be identical between the different tools.

\item[{\sphinxcode{EXEDIR}}] \leavevmode
The cprnc tool uses this variable to set the location of where the executable will be built. The default is the current directory.

\item[{\sphinxcode{VPATH}}] \leavevmode
The \sphinxstylestrong{cprnc} tool uses this variable to set the colon delimited pathnames of where the source code exists. The default is the current directory.

\end{description}

\begin{sphinxadmonition}{note}{Note:}
There are several files that are copies of the original files from either models{}`{}`/lnd/clm/src/util\_share{}`{}`, \sphinxcode{models/csm\_share/shr}, or copies from other tool directories. By having copies the tools can all be made stand-alone, but any changes to the originals will have to be put into the tool directories as well.
\end{sphinxadmonition}

The \sphinxstyleemphasis{README.filecopies} (which can be found in \sphinxcode{models/lnd/clm/tools}) is repeated here.

\begin{sphinxVerbatim}[commandchars=\\\{\}]
   \PYG{n}{models}\PYG{o}{/}\PYG{n}{lnd}\PYG{o}{/}\PYG{n}{clm}\PYG{o}{/}\PYG{n}{tools}\PYG{o}{/}\PYG{n}{README}\PYG{o}{.}\PYG{n}{filecopies}                          \PYG{n}{Jun}\PYG{o}{/}\PYG{l+m+mi}{04}\PYG{o}{/}\PYG{l+m+mi}{2013}

\PYG{n}{There} \PYG{n}{are} \PYG{n}{several} \PYG{n}{files} \PYG{n}{that} \PYG{n}{are} \PYG{n}{copies} \PYG{n}{of} \PYG{n}{the} \PYG{n}{original} \PYG{n}{files} \PYG{k+kn}{from}
\PYG{n+nn}{either} \PYG{n}{models}\PYG{o}{/}\PYG{n}{lnd}\PYG{o}{/}\PYG{n}{clm}\PYG{o}{/}\PYG{n}{src}\PYG{o}{/}\PYG{n}{main}\PYG{p}{,} \PYG{n}{models}\PYG{o}{/}\PYG{n}{csm\PYGZus{}share}\PYG{o}{/}\PYG{n}{shr}\PYG{p}{,}
\PYG{n}{models}\PYG{o}{/}\PYG{n}{csm\PYGZus{}share}\PYG{o}{/}\PYG{n}{unit\PYGZus{}testers}\PYG{p}{,} \PYG{o+ow}{or} \PYG{n}{copies} \PYG{k+kn}{from} \PYG{n+nn}{other} \PYG{n}{tool}
\PYG{n}{directories}\PYG{o}{.} \PYG{n}{By} \PYG{n}{having} \PYG{n}{copies} \PYG{n}{the} \PYG{n}{tools} \PYG{n}{can} \PYG{n+nb}{all} \PYG{n}{be} \PYG{n}{made} \PYG{n}{stand}\PYG{o}{\PYGZhy{}}\PYG{n}{alone}\PYG{p}{,}
\PYG{n}{but} \PYG{n+nb}{any} \PYG{n}{changes} \PYG{n}{to} \PYG{n}{the} \PYG{n}{originals} \PYG{n}{will} \PYG{n}{have} \PYG{n}{to} \PYG{n}{be} \PYG{n}{put} \PYG{n}{into} \PYG{n}{the} \PYG{n}{tool}
\PYG{n}{directories} \PYG{k}{as} \PYG{n}{well}\PYG{o}{.}

\PYG{n}{I}\PYG{o}{.} \PYG{n}{Files} \PYG{n}{that} \PYG{n}{are} \PYG{n}{IDENTICAL}\PYG{p}{:}

   \PYG{l+m+mf}{1.} \PYG{n}{csm\PYGZus{}share} \PYG{n}{files} \PYG{n}{copied} \PYG{n}{that} \PYG{n}{should} \PYG{n}{be} \PYG{n}{identical} \PYG{n}{to} \PYG{n}{models}\PYG{o}{/}\PYG{n}{csm\PYGZus{}share}\PYG{o}{/}\PYG{n}{shr}\PYG{p}{:}

       \PYG{n}{shr\PYGZus{}const\PYGZus{}mod}\PYG{o}{.}\PYG{n}{F90}
       \PYG{n}{shr\PYGZus{}log\PYGZus{}mod}\PYG{o}{.}\PYG{n}{F90}
       \PYG{n}{shr\PYGZus{}timer\PYGZus{}mod}\PYG{o}{.}\PYG{n}{F90}

   \PYG{l+m+mf}{2.} \PYG{n}{csm\PYGZus{}share} \PYG{n}{files} \PYG{n}{copied} \PYG{n}{that} \PYG{n}{should} \PYG{n}{be} \PYG{n}{identical} \PYG{n}{to} \PYG{n}{models}\PYG{o}{/}\PYG{n}{csm\PYGZus{}share}\PYG{o}{/}\PYG{n}{unit\PYGZus{}testers}\PYG{p}{:}

       \PYG{n}{test\PYGZus{}mod}\PYG{o}{.}\PYG{n}{F90}

   \PYG{l+m+mf}{3.} \PYG{n}{clm}\PYG{o}{/}\PYG{n}{src} \PYG{n}{files} \PYG{n}{copied} \PYG{n}{that} \PYG{n}{should} \PYG{n}{be} \PYG{n}{identical} \PYG{n}{to} \PYG{n}{models}\PYG{o}{/}\PYG{n}{lnd}\PYG{o}{/}\PYG{n}{clm}\PYG{o}{/}\PYG{n}{src}\PYG{o}{/}\PYG{n}{util\PYGZus{}share}\PYG{p}{:}

       \PYG{n}{nanMod}\PYG{o}{.}\PYG{n}{F90}

\PYG{n}{II}\PYG{o}{.} \PYG{n}{Files} \PYG{k}{with} \PYG{n}{differences}

   \PYG{l+m+mf}{1.} \PYG{n}{csm\PYGZus{}share} \PYG{n}{files} \PYG{n}{copied} \PYG{k}{with} \PYG{n}{differences}\PYG{p}{:}

       \PYG{n}{shr\PYGZus{}kind\PYGZus{}mod}\PYG{o}{.}\PYG{n}{F90} \PYG{o}{\PYGZhy{}}\PYG{o}{\PYGZhy{}}\PYG{o}{\PYGZhy{}} \PYG{n}{SHR\PYGZus{}KIND\PYGZus{}CXX} \PYG{o+ow}{is} \PYG{n}{new}
       \PYG{n}{shr\PYGZus{}sys\PYGZus{}mod}\PYG{o}{.}\PYG{n}{F90} \PYG{o}{\PYGZhy{}}\PYG{o}{\PYGZhy{}}\PYG{o}{\PYGZhy{}}\PYG{o}{\PYGZhy{}} \PYG{n}{Remove} \PYG{n}{mpi} \PYG{n}{abort} \PYG{o+ow}{and} \PYG{n}{reference} \PYG{n}{to} \PYG{n}{shr\PYGZus{}mpi\PYGZus{}mod}\PYG{o}{.}\PYG{n}{F90}\PYG{o}{.}
       \PYG{n}{shr\PYGZus{}infnan\PYGZus{}mod}\PYG{o}{.}\PYG{n}{F90} \PYG{o}{\PYGZhy{}} \PYG{n}{Earlier} \PYG{n}{version}
       \PYG{n}{shr\PYGZus{}string\PYGZus{}mod}\PYG{o}{.}\PYG{n}{F90} \PYG{o}{\PYGZhy{}} \PYG{n}{Earlier} \PYG{n}{version}
       \PYG{n}{shr\PYGZus{}file\PYGZus{}mod}\PYG{o}{.}\PYG{n}{F90} \PYG{o}{\PYGZhy{}}\PYG{o}{\PYGZhy{}}\PYG{o}{\PYGZhy{}} \PYG{n}{mkprocdata\PYGZus{}map} \PYG{n}{version} \PYG{o+ow}{is} \PYG{n}{stripped} \PYG{n}{down}
       \PYG{n}{clm\PYGZus{}varctl}\PYG{o}{.}\PYG{n}{F90} \PYG{o}{\PYGZhy{}}\PYG{o}{\PYGZhy{}}\PYG{o}{\PYGZhy{}}\PYG{o}{\PYGZhy{}}\PYG{o}{\PYGZhy{}} \PYG{n}{Earlier} \PYG{n}{version}

   \PYG{l+m+mf}{2.} \PYG{n}{clm}\PYG{o}{/}\PYG{n}{src} \PYG{n}{files} \PYG{k}{with} \PYG{n}{differences}\PYG{p}{:}

       \PYG{n}{fileutils}\PYG{o}{.}\PYG{n}{F90} \PYG{o}{\PYGZhy{}}\PYG{o}{\PYGZhy{}}\PYG{o}{\PYGZhy{}} \PYG{n}{Remove} \PYG{n}{use} \PYG{n}{of} \PYG{n}{masterproc} \PYG{o+ow}{and} \PYG{n}{spmdMod} \PYG{o+ow}{and} \PYG{n}{endrun} \PYG{o+ow}{in} \PYG{n}{abortutils}\PYG{o}{.}

   \PYG{l+m+mf}{4.} \PYG{n}{Files} \PYG{o+ow}{in} \PYG{n}{mkmapgrids}

      \PYG{n}{domainMod}\PYG{o}{.}\PYG{n}{F90} \PYG{o}{\PYGZhy{}}\PYG{o}{\PYGZhy{}}\PYG{o}{\PYGZhy{}}\PYG{o}{\PYGZhy{}} \PYG{n}{Highly} \PYG{n}{customized} \PYG{n}{based} \PYG{n}{off} \PYG{n}{an} \PYG{n}{earlier} \PYG{n}{version} \PYG{n}{of} \PYG{n}{clm} \PYG{n}{code}\PYG{o}{.}
                         \PYG{n}{Remove} \PYG{n}{use} \PYG{n}{of} \PYG{n}{abortutils}\PYG{p}{,} \PYG{n}{spmdMod}\PYG{o}{.} \PYG{n}{clm} \PYG{n}{version} \PYG{n}{uses} \PYG{n}{latlon}
                         \PYG{n}{this} \PYG{n}{version} \PYG{n}{uses} \PYG{n}{domain} \PYG{o+ow}{in} \PYG{n}{names}\PYG{o}{.} \PYG{n}{Distributed} \PYG{n}{memory}
                         \PYG{n}{parallelism} \PYG{o+ow}{is} \PYG{n}{removed}\PYG{o}{.}

   \PYG{l+m+mf}{5.} \PYG{n}{Files} \PYG{o+ow}{in} \PYG{n}{mksurfdata\PYGZus{}map}

       \PYG{n}{mkvarpar}\PYG{o}{.}\PYG{n}{F90} \PYG{o}{\PYGZhy{}}\PYG{o}{\PYGZhy{}}\PYG{o}{\PYGZhy{}} \PYG{n}{clm4\PYGZus{}0} \PYG{o+ow}{and} \PYG{n}{clm4\PYGZus{}5} \PYG{n}{versions} \PYG{n}{are} \PYG{n}{different} \PYG{o+ow}{and} \PYG{n}{different} \PYG{k+kn}{from} \PYG{n+nn}{main} \PYG{n}{clm} \PYG{n}{versions}\PYG{o}{.}
\end{sphinxVerbatim}


\subsection{Creating input for surface dataset generation}
\label{\detokenize{users_guide/using-clm-tools/creating-input-for-surface-dataset-generation:creating-input-for-surface-dataset-generation}}\label{\detokenize{users_guide/using-clm-tools/creating-input-for-surface-dataset-generation:creating-maps-for-mksurfdata}}\label{\detokenize{users_guide/using-clm-tools/creating-input-for-surface-dataset-generation::doc}}

\subsubsection{1. Generating SCRIP grid files}
\label{\detokenize{users_guide/using-clm-tools/creating-input-for-surface-dataset-generation:generating-scrip-grid-files}}
The utility \sphinxcode{mkmapdata.sh} requires SCRIP format input files to describe the input and output grids that maps are generated for. CLM provides a utility, \sphinxcode{mkmapgrids} that generates those files.
The program converts old formats of CAM or CLM grid files to SCRIP grid format. There is also a NCL script (\sphinxcode{mkscripgrid.ncl}) to create regular latitude longitude regional or single-point grids at the resolution the user desires.

SCRIP grid files for all the standard model resolutions and the raw surface datasets have already been done and the files are in the XML database. Hence, this step doesn’t need to be done \textendash{} EXCEPT WHEN YOU ARE CREATING YOUR OWN GRIDS. If you have a CLM grid or CAM file from previous versions and you want to convert it you can use \sphinxstylestrong{mkmapgrids}.


\paragraph{Using mknocnmap.pl to create grid and maps for single-point regional grids}
\label{\detokenize{users_guide/using-clm-tools/creating-input-for-surface-dataset-generation:using-mknocnmap-pl-to-create-grid-and-maps-for-single-point-regional-grids}}
If you want to create a regular latitude/longitude single-point or regional grid, we suggest you use \sphinxstylestrong{mknoocnmap.pl} in \sphinxcode{models/lnd/clm/tools/shared/mkmapdata} which will create both the SCRIP grid file you need (using \sphinxcode{models/lnd/clm/tools/shared/mkmapgrids/mkscripgrid.ncl} AND an identity mapping file assuming there is NO ocean in your grid domain. If you HAVE ocean in your domain you could modify the mask in the SCRIP grid file for ocean, and then use \sphinxstylestrong{ESMF\_RegridWeightGen} to create the mapping file, and \sphinxstylestrong{gen\_domain} to create the domain file. Like other tools, \sphinxcode{shared/mkmapdata/mknoocnmap.pl} has a help option with the following:

\begin{sphinxVerbatim}[commandchars=\\\{\}]
\PYG{n}{SYNOPSIS}
     \PYG{n}{mknoocnmap}\PYG{o}{.}\PYG{n}{pl} \PYG{p}{[}\PYG{n}{options}\PYG{p}{]}     \PYG{n}{Gets} \PYG{n+nb}{map} \PYG{o+ow}{and} \PYG{n}{grid} \PYG{n}{files} \PYG{k}{for} \PYG{n}{a} \PYG{n}{single} \PYG{n}{land}\PYG{o}{\PYGZhy{}}\PYG{n}{only} \PYG{n}{point}\PYG{o}{.}
\PYG{n}{REQUIRED} \PYG{n}{OPTIONS}
     \PYG{o}{\PYGZhy{}}\PYG{n}{centerpoint} \PYG{p}{[}\PYG{o+ow}{or} \PYG{o}{\PYGZhy{}}\PYG{n}{p}\PYG{p}{]} \PYG{o}{\PYGZlt{}}\PYG{n}{lat}\PYG{p}{,}\PYG{n}{lon}\PYG{o}{\PYGZgt{}} \PYG{n}{Center} \PYG{n}{latitude}\PYG{p}{,}\PYG{n}{longitude} \PYG{n}{of} \PYG{n}{the} \PYG{n}{grid} \PYG{n}{to} \PYG{n}{create}\PYG{o}{.}
     \PYG{o}{\PYGZhy{}}\PYG{n}{name} \PYG{p}{[}\PYG{o}{\PYGZhy{}}\PYG{o+ow}{or} \PYG{o}{\PYGZhy{}}\PYG{n}{n}\PYG{p}{]} \PYG{o}{\PYGZlt{}}\PYG{n}{name}\PYG{o}{\PYGZgt{}}       \PYG{n}{Name} \PYG{n}{to} \PYG{n}{use} \PYG{n}{to} \PYG{n}{describe} \PYG{n}{point}

\PYG{n}{OPTIONS}
     \PYG{o}{\PYGZhy{}}\PYG{n}{dx} \PYG{o}{\PYGZlt{}}\PYG{n}{number}\PYG{o}{\PYGZgt{}}                   \PYG{n}{Size} \PYG{n}{of} \PYG{n}{total} \PYG{n}{grid} \PYG{o+ow}{in} \PYG{n}{degrees} \PYG{o+ow}{in} \PYG{n}{longitude} \PYG{n}{direction}
                                    \PYG{p}{(}\PYG{n}{default} \PYG{o+ow}{is} \PYG{l+m+mf}{0.1}\PYG{p}{)}
     \PYG{o}{\PYGZhy{}}\PYG{n}{dy} \PYG{o}{\PYGZlt{}}\PYG{n}{number}\PYG{o}{\PYGZgt{}}                   \PYG{n}{Size} \PYG{n}{of} \PYG{n}{total} \PYG{n}{grid} \PYG{o+ow}{in} \PYG{n}{degrees} \PYG{o+ow}{in} \PYG{n}{latitude} \PYG{n}{direction}
                                    \PYG{p}{(}\PYG{n}{default} \PYG{o+ow}{is} \PYG{l+m+mf}{0.1}\PYG{p}{)}
     \PYG{o}{\PYGZhy{}}\PYG{n}{silent} \PYG{p}{[}\PYG{o+ow}{or} \PYG{o}{\PYGZhy{}}\PYG{n}{s}\PYG{p}{]}             \PYG{n}{Make} \PYG{n}{output} \PYG{n}{silent}
     \PYG{o}{\PYGZhy{}}\PYG{n}{help} \PYG{p}{[}\PYG{o+ow}{or} \PYG{o}{\PYGZhy{}}\PYG{n}{h}\PYG{p}{]}               \PYG{n}{Print} \PYG{n}{usage} \PYG{n}{to} \PYG{n}{STDOUT}\PYG{o}{.}
     \PYG{o}{\PYGZhy{}}\PYG{n}{verbose} \PYG{p}{[}\PYG{o+ow}{or} \PYG{o}{\PYGZhy{}}\PYG{n}{v}\PYG{p}{]}                    \PYG{n}{Make} \PYG{n}{output} \PYG{n}{more} \PYG{n}{verbose}\PYG{o}{.}
     \PYG{o}{\PYGZhy{}}\PYG{n}{nx} \PYG{o}{\PYGZlt{}}\PYG{n}{number}\PYG{o}{\PYGZgt{}}                   \PYG{n}{Number} \PYG{n}{of} \PYG{n}{longitudes} \PYG{p}{(}\PYG{n}{default} \PYG{o+ow}{is} \PYG{l+m+mi}{1}\PYG{p}{)}
     \PYG{o}{\PYGZhy{}}\PYG{n}{ny} \PYG{o}{\PYGZlt{}}\PYG{n}{number}\PYG{o}{\PYGZgt{}}                   \PYG{n}{Number} \PYG{n}{of} \PYG{n}{latitudes}  \PYG{p}{(}\PYG{n}{default} \PYG{o+ow}{is} \PYG{l+m+mi}{1}\PYG{p}{)}
\end{sphinxVerbatim}

See \sphinxhref{CLM-URL}{Figure 2-5} for a visual representation of this process.


\subsubsection{2. Creating mapping files for mksurfdata\_map}
\label{\detokenize{users_guide/using-clm-tools/creating-input-for-surface-dataset-generation:creating-mapping-files-for-mksurfdata-map}}
\sphinxcode{mkmapdata.sh} uses the above SCRIP grid input files to create SCRIP mapping data files (uses ESMF).

The bash shell script \sphinxcode{models/lnd/clm/tools/shared/mkmapgrids/mkmapdata.sh} uses \sphinxstylestrong{ESMF\_RegridWeightGen} to create a list of maps from the raw datasets that are input to \sphinxstylestrong{mksurfdata\_map}.
Each dataset that has a different grid, or land-mask needs a different mapping file for it, but many different raw datasets share the same grid/land-mask as other files.
Hence, there doesn’t need to be a different mapping file for EACH raw dataset \textendash{} just for each DIFFERENT raw dataset.
See \sphinxhref{CLM-URL}{Figure 2-3} for a visual representation of how this works.
The bash script figures out which mapping files it needs to create and then runs \sphinxstylestrong{ESMF\_RegridWeightGen} for each one.
You can then either enter the datasets into the XML database (see \sphinxhref{CLM-URL}{Chapter 3} or leave the files in place, and use the “-res usrspec -usr\_gname -usr\_gdate” options to \sphinxstylestrong{mksurfdata\_map} (see \sphinxhref{CLM-URL}{the Section called Running mksurfdata.pl} below).
Use the “-phys” option to specify if you are creating mapping files for clm4\_0 or clm4\_5 (the list of raw datafiles is somewhat different between the two).
mkmapdata.sh has a help option with the following

\begin{sphinxVerbatim}[commandchars=\\\{\}]
\PYG{o}{.}\PYG{o}{.}\PYG{o}{/}\PYG{o}{.}\PYG{o}{.}\PYG{o}{/}\PYG{n}{tools}\PYG{o}{/}\PYG{n}{shared}\PYG{o}{/}\PYG{n}{mkmapdata}\PYG{o}{/}\PYG{n}{mkmapdata}\PYG{o}{.}\PYG{n}{sh}

\PYG{o}{*}\PYG{o}{*}\PYG{o}{*}\PYG{o}{*}\PYG{o}{*}\PYG{o}{*}\PYG{o}{*}\PYG{o}{*}\PYG{o}{*}\PYG{o}{*}\PYG{o}{*}\PYG{o}{*}\PYG{o}{*}\PYG{o}{*}\PYG{o}{*}\PYG{o}{*}\PYG{o}{*}\PYG{o}{*}\PYG{o}{*}\PYG{o}{*}\PYG{o}{*}\PYG{o}{*}
\PYG{n}{usage} \PYG{n}{on} \PYG{n}{yellowstone}\PYG{p}{:}
\PYG{o}{.}\PYG{o}{/}\PYG{n}{mkmapdata}\PYG{o}{.}\PYG{n}{sh}

\PYG{n}{valid} \PYG{n}{arguments}\PYG{p}{:}
\PYG{p}{[}\PYG{o}{\PYGZhy{}}\PYG{n}{f}\PYG{o}{\textbar{}}\PYG{o}{\PYGZhy{}}\PYG{o}{\PYGZhy{}}\PYG{n}{gridfile} \PYG{o}{\PYGZlt{}}\PYG{n}{gridname}\PYG{o}{\PYGZgt{}}\PYG{p}{]}
     \PYG{n}{Full} \PYG{n}{pathname} \PYG{n}{of} \PYG{n}{model} \PYG{n}{SCRIP} \PYG{n}{grid} \PYG{n}{file} \PYG{n}{to} \PYG{n}{use}
     \PYG{n}{This} \PYG{n}{variable} \PYG{n}{should} \PYG{n}{be} \PYG{n+nb}{set} \PYG{k}{if} \PYG{n}{this} \PYG{o+ow}{is} \PYG{o+ow}{not} \PYG{n}{a} \PYG{n}{supported} \PYG{n}{grid}
     \PYG{n}{This} \PYG{n}{variable} \PYG{n}{will} \PYG{n}{override} \PYG{n}{the} \PYG{n}{automatic} \PYG{n}{generation} \PYG{n}{of} \PYG{n}{the}
     \PYG{n}{filename} \PYG{n}{generated} \PYG{k+kn}{from} \PYG{n+nn}{the} \PYG{o}{\PYGZhy{}}\PYG{n}{res} \PYG{n}{argument}
     \PYG{n}{the} \PYG{n}{filename} \PYG{o+ow}{is} \PYG{n}{generated} \PYG{n}{ASSUMING} \PYG{n}{that} \PYG{n}{this} \PYG{o+ow}{is} \PYG{n}{a} \PYG{n}{supported}
     \PYG{n}{grid} \PYG{n}{that} \PYG{n}{has} \PYG{n}{entries} \PYG{o+ow}{in} \PYG{n}{the} \PYG{n}{file} \PYG{n}{namelist\PYGZus{}defaults\PYGZus{}clm}\PYG{o}{.}\PYG{n}{xml}
     \PYG{n}{the} \PYG{o}{\PYGZhy{}}\PYG{n}{r}\PYG{o}{\textbar{}}\PYG{o}{\PYGZhy{}}\PYG{o}{\PYGZhy{}}\PYG{n}{res} \PYG{n}{argument} \PYG{n}{MUST} \PYG{n}{be} \PYG{n}{specied} \PYG{k}{if} \PYG{n}{this} \PYG{n}{argument} \PYG{o+ow}{is} \PYG{n}{specified}
\PYG{p}{[}\PYG{o}{\PYGZhy{}}\PYG{n}{r}\PYG{o}{\textbar{}}\PYG{o}{\PYGZhy{}}\PYG{o}{\PYGZhy{}}\PYG{n}{res} \PYG{o}{\PYGZlt{}}\PYG{n}{res}\PYG{o}{\PYGZgt{}}\PYG{p}{]}
     \PYG{n}{Model} \PYG{n}{output} \PYG{n}{resolution} \PYG{p}{(}\PYG{n}{default} \PYG{o+ow}{is} \PYG{l+m+mi}{10}\PYG{n}{x15}\PYG{p}{)}
\PYG{p}{[}\PYG{o}{\PYGZhy{}}\PYG{n}{t}\PYG{o}{\textbar{}}\PYG{o}{\PYGZhy{}}\PYG{o}{\PYGZhy{}}\PYG{n}{gridtype} \PYG{o}{\PYGZlt{}}\PYG{n+nb}{type}\PYG{o}{\PYGZgt{}}\PYG{p}{]}
     \PYG{n}{Model} \PYG{n}{output} \PYG{n}{grid} \PYG{n+nb}{type}
     \PYG{n}{supported} \PYG{n}{values} \PYG{n}{are} \PYG{p}{[}\PYG{n}{regional}\PYG{p}{,}\PYG{k}{global}\PYG{p}{]}\PYG{p}{,} \PYG{p}{(}\PYG{n}{default} \PYG{o+ow}{is} \PYG{k}{global}\PYG{p}{)}
\PYG{p}{[}\PYG{o}{\PYGZhy{}}\PYG{n}{p}\PYG{o}{\textbar{}}\PYG{o}{\PYGZhy{}}\PYG{o}{\PYGZhy{}}\PYG{n}{phys} \PYG{o}{\PYGZlt{}}\PYG{n}{CLM}\PYG{o}{\PYGZhy{}}\PYG{n}{version}\PYG{o}{\PYGZgt{}}\PYG{p}{]}
     \PYG{n}{Whether} \PYG{n}{to} \PYG{n}{generate} \PYG{n}{mapping} \PYG{n}{files} \PYG{k}{for} \PYG{n}{clm4\PYGZus{}0} \PYG{o+ow}{or} \PYG{n}{clm4\PYGZus{}5}
     \PYG{n}{supported} \PYG{n}{values} \PYG{n}{are} \PYG{p}{[}\PYG{n}{clm4\PYGZus{}0}\PYG{p}{,}\PYG{n}{clm4\PYGZus{}5}\PYG{p}{]}\PYG{p}{,} \PYG{p}{(}\PYG{n}{default} \PYG{o+ow}{is} \PYG{n}{clm4\PYGZus{}5}\PYG{p}{)}
\PYG{p}{[}\PYG{o}{\PYGZhy{}}\PYG{n}{b}\PYG{o}{\textbar{}}\PYG{o}{\PYGZhy{}}\PYG{o}{\PYGZhy{}}\PYG{n}{batch}\PYG{p}{]}
     \PYG{n}{Toggles} \PYG{n}{batch} \PYG{n}{mode} \PYG{n}{usage}\PYG{o}{.}
\PYG{n}{If} \PYG{n}{you} \PYG{n}{want} \PYG{n}{to} \PYG{n}{run} \PYG{o+ow}{in} \PYG{n}{batch} \PYG{n}{mode}
     \PYG{n}{you} \PYG{n}{need} \PYG{n}{to} \PYG{n}{have} \PYG{n}{a} \PYG{n}{separate} \PYG{n}{batch} \PYG{n}{script} \PYG{k}{for} \PYG{n}{a} \PYG{n}{supported} \PYG{n}{machine}
     \PYG{n}{that} \PYG{n}{calls} \PYG{n}{this} \PYG{n}{script} \PYG{n}{interactively} \PYG{o}{\PYGZhy{}} \PYG{n}{you} \PYG{n}{cannot} \PYG{n}{submit} \PYG{n}{this}
     \PYG{n}{script} \PYG{n}{directory} \PYG{n}{to} \PYG{n}{the} \PYG{n}{batch} \PYG{n}{system}
\PYG{p}{[}\PYG{o}{\PYGZhy{}}\PYG{n}{l}\PYG{o}{\textbar{}}\PYG{o}{\PYGZhy{}}\PYG{o}{\PYGZhy{}}\PYG{n+nb}{list}\PYG{p}{]}
     \PYG{n}{List} \PYG{n}{mapping} \PYG{n}{files} \PYG{n}{required} \PYG{p}{(}\PYG{n}{use} \PYG{n}{check\PYGZus{}input\PYGZus{}data} \PYG{n}{to} \PYG{n}{get} \PYG{n}{them}\PYG{p}{)}
     \PYG{n}{also} \PYG{n}{writes} \PYG{n}{data} \PYG{n}{to} \PYG{n}{clm}\PYG{o}{.}\PYG{n}{input\PYGZus{}data\PYGZus{}list}
\PYG{p}{[}\PYG{o}{\PYGZhy{}}\PYG{n}{d}\PYG{o}{\textbar{}}\PYG{o}{\PYGZhy{}}\PYG{o}{\PYGZhy{}}\PYG{n}{debug}\PYG{p}{]}
     \PYG{n}{Toggles} \PYG{n}{debug}\PYG{o}{\PYGZhy{}}\PYG{n}{only} \PYG{p}{(}\PYG{n}{don}\PYG{l+s+s1}{\PYGZsq{}}\PYG{l+s+s1}{t actually run mkmapdata just echo what would happen)}
\PYG{p}{[}\PYG{o}{\PYGZhy{}}\PYG{n}{h}\PYG{o}{\textbar{}}\PYG{o}{\PYGZhy{}}\PYG{o}{\PYGZhy{}}\PYG{n}{help}\PYG{p}{]}
     \PYG{n}{Displays} \PYG{n}{this} \PYG{n}{help} \PYG{n}{message}
\PYG{p}{[}\PYG{o}{\PYGZhy{}}\PYG{n}{v}\PYG{o}{\textbar{}}\PYG{o}{\PYGZhy{}}\PYG{o}{\PYGZhy{}}\PYG{n}{verbose}\PYG{p}{]}
     \PYG{n}{Toggle} \PYG{n}{verbose} \PYG{n}{usage} \PYG{o}{\PYGZhy{}}\PYG{o}{\PYGZhy{}} \PYG{n}{log} \PYG{n}{more} \PYG{n}{information} \PYG{n}{on} \PYG{n}{what} \PYG{o+ow}{is} \PYG{n}{happening}

 \PYG{n}{You} \PYG{n}{can} \PYG{n}{also} \PYG{n+nb}{set} \PYG{n}{the} \PYG{n}{following} \PYG{n}{env} \PYG{n}{variables}\PYG{p}{:}
  \PYG{n}{ESMFBIN\PYGZus{}PATH} \PYG{o}{\PYGZhy{}} \PYG{n}{Path} \PYG{n}{to} \PYG{n}{ESMF} \PYG{n}{binaries}
                 \PYG{p}{(}\PYG{n}{default} \PYG{o+ow}{is} \PYG{o}{/}\PYG{n}{contrib}\PYG{o}{/}\PYG{n}{esmf}\PYG{o}{\PYGZhy{}}\PYG{l+m+mf}{5.3}\PYG{o}{.}\PYG{l+m+mi}{0}\PYG{o}{\PYGZhy{}}\PYG{l+m+mi}{64}\PYG{o}{\PYGZhy{}}\PYG{n}{O}\PYG{o}{/}\PYG{n+nb}{bin}\PYG{p}{)}
  \PYG{n}{CSMDATA} \PYG{o}{\PYGZhy{}}\PYG{o}{\PYGZhy{}}\PYG{o}{\PYGZhy{}}\PYG{o}{\PYGZhy{}}\PYG{o}{\PYGZhy{}}\PYG{o}{\PYGZhy{}} \PYG{n}{Path} \PYG{n}{to} \PYG{n}{CESM} \PYG{n+nb}{input} \PYG{n}{data}
                 \PYG{p}{(}\PYG{n}{default} \PYG{o+ow}{is} \PYG{o}{/}\PYG{n}{glade}\PYG{o}{/}\PYG{n}{p}\PYG{o}{/}\PYG{n}{cesm}\PYG{o}{/}\PYG{n}{cseg}\PYG{o}{/}\PYG{n}{inputdata}\PYG{p}{)}
  \PYG{n}{MPIEXEC} \PYG{o}{\PYGZhy{}}\PYG{o}{\PYGZhy{}}\PYG{o}{\PYGZhy{}}\PYG{o}{\PYGZhy{}}\PYG{o}{\PYGZhy{}}\PYG{o}{\PYGZhy{}} \PYG{n}{Name} \PYG{n}{of} \PYG{n}{mpirun} \PYG{n}{executable}
                 \PYG{p}{(}\PYG{n}{default} \PYG{o+ow}{is} \PYG{n}{mpirun}\PYG{o}{.}\PYG{n}{lsf}\PYG{p}{)}
  \PYG{n}{REGRID\PYGZus{}PROC} \PYG{o}{\PYGZhy{}}\PYG{o}{\PYGZhy{}} \PYG{n}{Number} \PYG{n}{of} \PYG{n}{MPI} \PYG{n}{processors} \PYG{n}{to} \PYG{n}{use}
                 \PYG{p}{(}\PYG{n}{default} \PYG{o+ow}{is} \PYG{l+m+mi}{8}\PYG{p}{)}

\PYG{o}{*}\PYG{o}{*}\PYG{k}{pass} \PYG{n}{environment} \PYG{n}{variables} \PYG{n}{by} \PYG{n}{preceding} \PYG{n}{above} \PYG{n}{commands}
  \PYG{k}{with} \PYG{l+s+s1}{\PYGZsq{}}\PYG{l+s+s1}{env var1=setting var2=setting }\PYG{l+s+s1}{\PYGZsq{}}
\PYG{o}{*}\PYG{o}{*}\PYG{o}{*}\PYG{o}{*}\PYG{o}{*}\PYG{o}{*}\PYG{o}{*}\PYG{o}{*}\PYG{o}{*}\PYG{o}{*}\PYG{o}{*}\PYG{o}{*}\PYG{o}{*}\PYG{o}{*}\PYG{o}{*}\PYG{o}{*}\PYG{o}{*}\PYG{o}{*}\PYG{o}{*}\PYG{o}{*}\PYG{o}{*}\PYG{o}{*}
\end{sphinxVerbatim}

\begin{sphinxadmonition}{warning}{Warning:}
Make sure you specify with the “-phys” option if you are creating files for CLM4.0! The default is CLM4.5.
\end{sphinxadmonition}


\paragraph{Figure 2-3. Details of running mkmapdata.sh}
\label{\detokenize{users_guide/using-clm-tools/creating-input-for-surface-dataset-generation:figure-2-3-details-of-running-mkmapdata-sh}}
Insert figure 2-3

Each of the raw datasets for \sphinxstylestrong{mksurfdata\_map} needs a mapping file to map from the output grid you are running on to the grid and land-mask for that dataset. This is what \sphinxstylestrong{mkmapdata.sh} does. To create the mapping files you need a SCRIP grid file to correspond with each resolution and land mask that you have a raw data file in \sphinxstylestrong{mksurfdata\_map}. Some raw datasets share the same grid and land mask \textendash{} hence they can share the same SCRIP grid file. The output maps created here go into \sphinxstylestrong{mksurfdata\_map} see \sphinxhref{CLM-URL}{Figure 2-6}.


\subsection{Creating Surface Datasets}
\label{\detokenize{users_guide/using-clm-tools/creating-surface-datasets:creating-surface-datasets}}\label{\detokenize{users_guide/using-clm-tools/creating-surface-datasets::doc}}\label{\detokenize{users_guide/using-clm-tools/creating-surface-datasets:id1}}
When just creating a replacement file for an existing one, the relevant tool should be used directly to create the file. When you are creating a set of files for a new resolution there are some dependencies between the tools that you need to keep in mind when creating them. The main dependency is that you MUST create a SCRIP grid file first as the SCRIP grid dataset is then input into the other tools. Also look at \sphinxhref{CLM-URL}{Table 3-1} which gives information on the files required and when. \sphinxhref{CLM-URL}{Figure 2-1} shows an overview of the general data-flow for creation of the fsurdat datasets.


\subsubsection{Figure 2-1. Data Flow for Creation of Surface Datasets from Raw SCRIP Grid Files}
\label{\detokenize{users_guide/using-clm-tools/creating-surface-datasets:figure-2-1-data-flow-for-creation-of-surface-datasets-from-raw-scrip-grid-files}}
Insert figure 2-1

Starting from a SCRIP grid file that describes the grid you will run the model on, you first run \sphinxstylestrong{mkmapdata.sh} to create a list of mapping files. See \sphinxhref{CLM-URL}{Figure 2-3} for a more detailed view of how \sphinxstylestrong{mkmapdata.sh} works. The mapping files tell \sphinxstylestrong{mksurfdata\_map} how to map between the output grid and the raw datasets that it uses as input. The output of \sphinxstylestrong{mksurfdata\_map} is a surface dataset that you then use for running the model. See \sphinxhref{CLM-URL}{Figure 2-6} for a more detailed view of how \sphinxstylestrong{mksurfdata\_map} works.

\sphinxhref{CLM-URL}{Figure 2-2} is the legend for this figure (\sphinxhref{CLM-URL}{Figure 2-1}) and other figures in this chapter (\sphinxhref{CLM-URL}{Figure 2-4}, \sphinxhref{CLM-URL}{Figure 2-5}, and \sphinxhref{CLM-URL}{Figure 2-6}).
Figure 2-2. Legend for Data Flow Figures
Insert figure 2-2

Green arrows define the input to a program, while red arrows define the output. Cylinders define files that are either created by a program or used as input for a program. Boxes are programs.

You start with a description of a SCRIP grid file for your output grid file and then create mapping files from the raw datasets to it. Once, the mapping files are created \sphinxstylestrong{mksurfdata\_map} is run to create the surface dataset to run the model.


\subsubsection{Creating a Complete Set of Files for Input to CLM}
\label{\detokenize{users_guide/using-clm-tools/creating-surface-datasets:creating-a-complete-set-of-files-for-input-to-clm}}\begin{enumerate}
\item {} 
Create SCRIP grid datasets (if NOT already done)

First you need to create a descriptor file for your grid, that includes the locations of cell centers and cell corners. There is also a “mask” field, but in this case the mask is set to one everywhere (i.e. all of the masks for the output model grid are “nomask”). An example SCRIP grid file is: \$CSMDATA/lnd/clm2/mappingdata/grids/SCRIPgrid\_10x15\_nomask\_c110308.nc. The mkmapgrids and mkscripgrid.ncl NCL script in the models/lnd/clm/tools/shared/mkmapgrids directory can help you with this. SCRIP grid files for all the standard CLM grids are already created for you. See the Section called Creating an output SCRIP grid file at a resolution to run the model on for more information on this.

\item {} 
Create domain dataset (if NOT already done)

Next use gen\_domain to create a domain file for use by DATM and CLM. This is required, unless a domain file was already created. See the Section called Creating a domain file for CLM and DATM for more information on this.

\item {} 
Create mapping files for mksurfdata\_map (if NOT already done)

Create mapping files for mksurfdata\_map with mkmapdata.sh in models/lnd/clm/tools/shared/mkmapdata. See the Section called Creating mapping files that mksurfdata\_map will use for more information on this.

\item {} 
Create surface datasets

Next use mksurfdata\_map to create a surface dataset, using the mapping datasets created on the previous step as input. There is a version for either clm4\_0 or clm4\_5 for this program. See the Section called Using mksurfdata\_map to create surface datasets from grid datasets for more information on this.

\item {} 
Create some sort of initial condition dataset

You then need to do one of the following three options to have an initial dataset to start from.
\begin{enumerate}
\item {} 
Use spinup-procedures to create initial condition datasets

The first option is to do the spinup procedures from arbitrary initial conditions to get good initial datasets. This is the most robust method to use. See the Section called Spinning up the Satellite Phenology Model (CLMSP spinup) in Chapter 4, the Section called Spinning up the CLM4.0 biogeochemistry Carbon-Nitrogen Model (CN spinup) in Chapter 4, or the Section called Spinning up the CLM4.0 Carbon-Nitrogen Dynamic Global Vegetation Model (CNDV spinup) in Chapter 4 for more information on this.

\item {} 
Use interpinic to interpolate existing initial condition datasets

The next option is to interpolate from spunup datasets at a different resolution, using interpinic. There is a version for either clm4\_0 or clm4\_5 for this program. See the Section called Using interpinic to interpolate initial conditions to different resolutions for more information on this.

\item {} 
Start up from arbitrary initial conditions

The last alternative is to run from arbitrary initial conditions without using any spun-up datasets. This is inappropriate when using CLM4.5-BGC or CLMCN (bgc=cn or cndv) as it takes a long time to spinup Carbon pools.

\end{enumerate}

\end{enumerate}

\begin{sphinxadmonition}{warning}{Warning:}
This is NOT recommended as many fields in CLM take a long time to equilibrate.
\end{sphinxadmonition}
\begin{enumerate}
\setcounter{enumi}{5}
\item {} 
Enter the new datasets into the build-namelist XML database
The last optional thing to do is to enter the new datasets into the build-namelist XML database. See Chapter 3 for more information on doing this. This is optional because the user may enter these files into their namelists manually. The advantage of entering them into the database is so that they automatically come up when you create new cases.

\end{enumerate}

The \sphinxcode{models/lnd/clm/tools/README} goes through the complete process for creating input files needed to run CLM. We repeat that file here:

\begin{sphinxVerbatim}[commandchars=\\\{\}]
models/lnd/clm/tools/README                                  Jun/04/2013

CLM tools for analysis of CLM history files \PYGZhy{}\PYGZhy{} or for creation or
modification of CLM input files.

I.  General directory structure:

 clm4\PYGZus{}0
     mksurfdata\PYGZus{}map \PYGZhy{}\PYGZhy{}\PYGZhy{} Create surface datasets.
     interpinic \PYGZhy{}\PYGZhy{}\PYGZhy{}\PYGZhy{}\PYGZhy{}\PYGZhy{}\PYGZhy{} Interpolate initial datasets to a different resolution.
                        (has optimized and OMP options)
 clm4\PYGZus{}5
     mksurfdata\PYGZus{}map \PYGZhy{}\PYGZhy{}\PYGZhy{} Create surface datasets.
     interpinic \PYGZhy{}\PYGZhy{}\PYGZhy{}\PYGZhy{}\PYGZhy{}\PYGZhy{}\PYGZhy{} Interpolate initial datasets to a different resolution.
                        (has optimized and OMP options)

 shared
     mkmapgrids \PYGZhy{}\PYGZhy{}\PYGZhy{}\PYGZhy{}\PYGZhy{}\PYGZhy{}\PYGZhy{} Create SCRIP grid files needed by mkmapdata
                        [input is CLM grid files]
                        (deprecated)
     mkmapdata \PYGZhy{}\PYGZhy{}\PYGZhy{}\PYGZhy{}\PYGZhy{}\PYGZhy{}\PYGZhy{}\PYGZhy{} Create SCRIP mapping data from SCRIP grid files (uses ESMF)
     gen\PYGZus{}domain \PYGZhy{}\PYGZhy{}\PYGZhy{}\PYGZhy{}\PYGZhy{}\PYGZhy{}\PYGZhy{} Create data model domain datasets from SCRIP mapping datasets.
                        (also in the top level mapping directory [../../../../tools/mapping])
     mkprocdata\PYGZus{}map \PYGZhy{}\PYGZhy{}\PYGZhy{} Convert output unstructured grids into a 2D format that
                        can be plotted easily
     ncl\PYGZus{}scripts \PYGZhy{}\PYGZhy{}\PYGZhy{}\PYGZhy{}\PYGZhy{}\PYGZhy{} NCL post or pre processing scripts.


 Note that there are different versions of mksurfdata\PYGZus{}map and interpinic for
 CLM4.0 vs. CLM4.5. Other tools are shared between the two model
 versions.

 However, note that mkmapdata makes mapping files for CLM4.5 by default; to
 make mapping files for CLM4.0, run the tool with the option:
     \PYGZhy{}p clm4\PYGZus{}0

II. Notes on building/running for each of the above tools:

 Each tool that has FORTRAN source code has the following files:

     README \PYGZhy{}\PYGZhy{}\PYGZhy{}\PYGZhy{}\PYGZhy{}\PYGZhy{}\PYGZhy{}\PYGZhy{}\PYGZhy{}\PYGZhy{}\PYGZhy{}\PYGZhy{}\PYGZhy{}\PYGZhy{} Specific help for using the specific tool and help on specific
                           files in that directory.
     src/Filepath \PYGZhy{}\PYGZhy{}\PYGZhy{}\PYGZhy{}\PYGZhy{}\PYGZhy{}\PYGZhy{}\PYGZhy{} List of directories needed to build the tool
                           (some files in ../src directories are required).
     src/Makefile \PYGZhy{}\PYGZhy{}\PYGZhy{}\PYGZhy{}\PYGZhy{}\PYGZhy{}\PYGZhy{}\PYGZhy{} Customization of the make for the particular tool in question
     src/Makefile.common \PYGZhy{} General GNU Makefile for creating FORTRAN tools
                           (these are identical between tools).
     src/Srcfiles \PYGZhy{}\PYGZhy{}\PYGZhy{}\PYGZhy{}\PYGZhy{}\PYGZhy{}\PYGZhy{}\PYGZhy{} List of source files that are needed.
     src/Mkdepends \PYGZhy{}\PYGZhy{}\PYGZhy{}\PYGZhy{}\PYGZhy{}\PYGZhy{}\PYGZhy{} Dependency generator program

 mkmapdata and ncl\PYGZus{}scripts only contain scripts so don\PYGZsq{}t have the above build files.

 Most tools have copies of files from other directories \PYGZhy{}\PYGZhy{} see the README.filecopies
 file for more information on this.

 Tools may also have files with the directory name followed by: namelist, or runoptions.

     \PYGZlt{}directory\PYGZgt{}.namelist \PYGZhy{}\PYGZhy{}\PYGZhy{}\PYGZhy{}\PYGZhy{}\PYGZhy{} Namelist to create a global file.
     \PYGZlt{}directory\PYGZgt{}.runoptions \PYGZhy{}\PYGZhy{}\PYGZhy{}\PYGZhy{} Command line options to use the given tool.

 These files are also used by the test scripts to test the tools (see the
 README.testing) file.

 NOTE: Be sure to change the path of the datasets references by these namelists to
 point to where you have exported your CESM inputdata datasets.

 To build:

     cd \PYGZlt{}directory\PYGZgt{}
     setenv INC\PYGZus{}NETCDF \PYGZlt{}path\PYGZhy{}to\PYGZhy{}NetCDF\PYGZhy{}include\PYGZhy{}files\PYGZgt{}
     setenv LIB\PYGZus{}NETCDF \PYGZlt{}path\PYGZhy{}to\PYGZhy{}NetCDF\PYGZhy{}library\PYGZhy{}files\PYGZgt{}
     gmake

 The process will create a file called \PYGZdq{}Depends\PYGZdq{} which has the dependencies
 for the build of each file on other files.

   By default some codes may be compiled non\PYGZhy{}optimized
   so that you can use the debugger, and with bounds\PYGZhy{}checking, and float trapping on.
   To speed up do the following...

gmake OPT=TRUE  (by default already on for interpinic and mksurfdata\PYGZus{}map)

   Also some of the tools allow for OpenMP shared memory parallelism
   (such as interpinic) with

gmake SMP=TRUE

 To run a program with a namelist:

     ./program \PYGZlt{} namelist

 To get help on running a program with command line options (e.g., interpinic):

     ./program

 To run a program built with SMP=TRUE:

     setenv OMP\PYGZus{}NUM\PYGZus{}THREADS=\PYGZlt{}number\PYGZus{}of\PYGZus{}threads\PYGZus{}to\PYGZus{}use\PYGZgt{}

     run normally as above

III. Process sequence to create input datasets needed to run CLM

NOTE: The following assumes you want to create files for CLM4.5. If you want to
   use CLM4.0, you will need to do the following:
 \PYGZhy{} In the following commands, change references to the clm4\PYGZus{}5 directory to clm4\PYGZus{}0
 \PYGZhy{} Add the option \PYGZsq{}\PYGZhy{}p clm4\PYGZus{}0\PYGZsq{} to the mkmapdata.sh command.

 1.) Create SCRIP grid files (if needed)

    a.) For standard resolutions these files will already be created. (done)

    b.) To create regular lat\PYGZhy{}lon regional/single\PYGZhy{}point grids run mknoocnmap.pl

     This will create both SCRIP grid files and a mapping file that will
     be valid if the region includes NO ocean whatsoever (so you can skip step 2).
     You can also use this script to create SCRIP grid files for a region
     (or even a global grid) that DOES include ocean if you use step 2 to
     create mapping files for it (simply discard the non\PYGZhy{}ocean map created by
     this script).

     Example, for single\PYGZhy{}point over Boulder Colorado.

        cd shared/mkmapdata
        ./mknoocnmap.pl \PYGZhy{}p 40,255 \PYGZhy{}n 1x1\PYGZus{}boulderCO

    c.) General case

     You\PYGZsq{}ll need to convert or create SCRIP grid files on your own (using scripts
     or other tools) for the general case where you have an unstructured grid, or
     a grid that is not regular in latitude and longitude.

    example format
      ==================
       netcdf fv1.9x2.5\PYGZus{}090205 \PYGZob{}
       dimensions:
            grid\PYGZus{}size = 13824 ;
            grid\PYGZus{}corners = 4 ;
            grid\PYGZus{}rank = 2 ;
       variables:
            double grid\PYGZus{}center\PYGZus{}lat(grid\PYGZus{}size) ;
                    grid\PYGZus{}center\PYGZus{}lat:units = \PYGZdq{}degrees\PYGZdq{} ;
            double grid\PYGZus{}center\PYGZus{}lon(grid\PYGZus{}size) ;
                    grid\PYGZus{}center\PYGZus{}lon:units = \PYGZdq{}degrees\PYGZdq{} ;
            double grid\PYGZus{}corner\PYGZus{}lat(grid\PYGZus{}size, grid\PYGZus{}corners) ;
                    grid\PYGZus{}corner\PYGZus{}lat:units = \PYGZdq{}degrees\PYGZdq{} ;
            double grid\PYGZus{}corner\PYGZus{}lon(grid\PYGZus{}size, grid\PYGZus{}corners) ;
                    grid\PYGZus{}corner\PYGZus{}lon:units = \PYGZdq{}degrees\PYGZdq{} ;
            int grid\PYGZus{}dims(grid\PYGZus{}rank) ;
            int grid\PYGZus{}imask(grid\PYGZus{}size) ;
                    grid\PYGZus{}imask:units = \PYGZdq{}unitless\PYGZdq{} ;

 2.) Create ocean to atmosphere mapping file (if needed)

     a.) Standard resolutions (done)

     If this is a standard resolution with a standard ocean resolution \PYGZhy{}\PYGZhy{} this
     step is already done, the files already exist.

     b.) Region without Ocean (done in step 1.b)

     IF YOU RAN mknoocnmap.pl FOR A REGION WITHOUT OCEAN THIS STEP IS ALREADY DONE.

     c.) New atmosphere or ocean resolution

     If the region DOES include ocean, use gen\PYGZus{}domain to create a
     mapping file for it.

 Example:

 cd ../../../../tools/mapping/gen\PYGZus{}domain\PYGZus{}files/src
 ./gen\PYGZus{}domain \PYGZhy{}m \PYGZdl{}MAPFILE \PYGZhy{}o \PYGZdl{}OCNGRIDNAME \PYGZhy{}l \PYGZdl{}ATMGRIDNAME


 3.) Add SCRIP grid file(s) created in (1) into XML database in CLM (optional)

     See the \PYGZdq{}Adding New Resolutions or New Files to the build\PYGZhy{}namelist Database\PYGZdq{}
     Chapter in the CLM User\PYGZsq{}s Guide

     http://www.cesm.ucar.edu/models/cesm1.0/clm/models/lnd/clm/doc/UsersGuide/book1.html

      If you don\PYGZsq{}t do this step, you\PYGZsq{}ll need to specify the file to mkmapdata
      in step (3) using the \PYGZdq{}\PYGZhy{}f\PYGZdq{} option.

 4.) Create mapping files for use by mksurfdata\PYGZus{}map with mkmapdata
     (See mkmapdata/README for more help on doing this)

    \PYGZhy{} this step uses the results of (1) that were entered into the XML database
      by step (3). If you don\PYGZsq{}t enter datasets in, you need to specify the
      SCRIP grid file using the \PYGZdq{}\PYGZhy{}f\PYGZdq{} option to mkmapdata.sh.

    \PYGZhy{} note that mkmapdata generates maps for CLM4.5 by default; to generate
      mapping files for CLM4.0, add the option \PYGZsq{}\PYGZhy{}p clm4\PYGZus{}0\PYGZsq{}

    Example: to generate all necessary mapping files for the ne30np4 grid

        cd shared/mkmapdata
        ./mkmapdata.sh \PYGZhy{}r ne30np4

 5.) Add mapping file(s) created in step (4) into XML database in CLM (optional)

    See notes on doing this in step (3) above.
    Edit ../bld/namelist\PYGZus{}files/namelist\PYGZus{}defaults\PYGZus{}clm.xml to incorporate new
    mapping files.

    If you don\PYGZsq{}t do this step, you\PYGZsq{}ll need to specify the grid resolution name
    and file creation dates to mksurfdata\PYGZus{}map in step (5) below.

 6.) Convert map of ocean to atm for use by DATM and CLM with gen\PYGZus{}domain
     (See tools/mapping/README for more help on doing this)

    \PYGZhy{} gen\PYGZus{}domain uses the map from step (2) (or previously created CESM maps)

    Example:

     cd ../../../../tools/mapping/gen\PYGZus{}domain\PYGZus{}files/src
     gmake
     cd ..
     setenv CDATE       090206
     setenv OCNGRIDNAME gx1v6
     setenv ATMGRIDNAME fv1.9x2.5
     setenv MAPFILE \PYGZdl{}CSMDATA/cpl/cpl6/map\PYGZus{}\PYGZdl{}\PYGZob{}OCNGRIDNAME\PYGZcb{}\PYGZus{}to\PYGZus{}\PYGZdl{}\PYGZob{}ATMGRIDNAME\PYGZcb{}\PYGZus{}aave\PYGZus{}da\PYGZus{}\PYGZdl{}\PYGZob{}CDATE\PYGZcb{}.nc
     ./gen\PYGZus{}domain \PYGZhy{}m \PYGZdl{}MAPFILE \PYGZhy{}o \PYGZdl{}OCNGRIDNAME \PYGZhy{}l \PYGZdl{}ATMGRIDNAME

     Normally for I compsets running CLM only you will discard the ocean domain
     file, and only use the atmosphere domain file for datm and as the fatmlndfrc
     file for CLM. Output domain files will be named according to the input OCN/LND
     gridnames.

 7.) Create surface datasets with mksurfdata\PYGZus{}map
     (See mksurfdata\PYGZus{}map/README for more help on doing this)

    \PYGZhy{} Run clm4\PYGZus{}5/mksurfdata\PYGZus{}map/mksurfdata.pl
    \PYGZhy{} This step uses the results of step (4) entered into the XML database
      in step (5).
    \PYGZhy{} If datasets were NOT entered into the XML database, set the resolution
      to \PYGZdq{}usrspec\PYGZdq{} and use the \PYGZdq{}\PYGZhy{}usr\PYGZus{}gname\PYGZdq{}, and \PYGZdq{}\PYGZhy{}usr\PYGZus{}gdate\PYGZdq{} options.

    Example: for 0.9x1.25 resolution

    cd clm4\PYGZus{}5/mksurfdata\PYGZus{}map/src
    gmake
    cd ..
    ./mksurfdata.pl \PYGZhy{}r 0.9x1.25

    NOTE that surface dataset will be used by default for fatmgrid \PYGZhy{} and it will
    contain the lat,lon,edges and area values for the atm grid \PYGZhy{} ASSUMING that
    the atm and land grid are the same

 8.) Interpolate initial conditions using interpinic (optional)
     (See interpinic/README for more help on doing this)
     IMPORTANT NOTE on interpinic!!!:: BE SURE TO USE NetCDF4.3 WHEN BUILDING!
        If your template file was written using pnetcdf \PYGZhy{}\PYGZhy{} interpinic will corrupt
        the resulting file and make it unusable!

 9.) Add new files to XML data or using user\PYGZus{}nl\PYGZus{}clm (optional)

    See notes on doing this in step (3) above.

IV. Example of creating single\PYGZhy{}point datasets without entering into XML database.

 Here we apply the process described in III. for a single\PYGZhy{}point dataset
 where we don\PYGZsq{}t enter the datasets into the XML database (thus skipping
 steps 3, 5 and 9), but use the needed command line options to specify where the
 files are. This also skips step (2) since step 1 creates the needed mapping file.
 We also skip step (8) and do NOT create a finidat file.

 0.) Set name of grid to use and the creation date to be used later...
    setenv GRIDNAME 1x1\PYGZus{}boulderCO
    setenv CDATE    {}`date +\PYGZpc{}y\PYGZpc{}m\PYGZpc{}d{}`
 1.) SCRIP grid and atm to ocn mapping file
    cd shared/mkmapdata
    ./mknoocnmap.pl \PYGZhy{}p 40,255 \PYGZhy{}n \PYGZdl{}GRIDNAME
    \PYGZsh{} Set pointer to MAPFILE that will be used in step (6)
    setenv MAPFILE {}`pwd{}`/map\PYGZus{}\PYGZdl{}\PYGZob{}GRIDNAME\PYGZcb{}\PYGZus{}noocean\PYGZus{}to\PYGZus{}\PYGZdl{}\PYGZob{}GRIDNAME\PYGZcb{}\PYGZus{}nomask\PYGZus{}aave\PYGZus{}da\PYGZus{}\PYGZdl{}\PYGZob{}CDATE\PYGZcb{}.nc
    cd ../..
 2.) skip
 3.) skip
 4.) Mapping files needed for mksurfdata\PYGZus{}map
    cd shared/mkmapdata
    setenv GRIDFILE ../mkmapgrids/SCRIPgrid\PYGZus{}\PYGZdl{}\PYGZob{}GRIDNAME\PYGZcb{}\PYGZus{}nomask\PYGZus{}\PYGZdl{}\PYGZob{}CDATE\PYGZcb{}.nc
    ./mkmapdata.sh \PYGZhy{}r \PYGZdl{}GRIDNAME \PYGZhy{}f \PYGZdl{}GRIDFILE \PYGZhy{}t regional
    cd ..
 5.) skip
 6.) Generate domain file for datm and CLM
     cd ../../../../tools/mapping/gen\PYGZus{}domain\PYGZus{}files/src
     gmake
     cd ..
     setenv OCNDOM domain.ocn\PYGZus{}noocean.nc
     setenv ATMDOM domain.lnd.\PYGZob{}\PYGZdl{}GRIDNAME\PYGZcb{}\PYGZus{}noocean.nc
     ./gen\PYGZus{}domain \PYGZhy{}m \PYGZdl{}MAPFILE \PYGZhy{}o \PYGZdl{}OCNDOM \PYGZhy{}l \PYGZdl{}ATMDOM
     cd ../../../../lnd/clm/tools
 7.) Create surface dataset for CLM
    cd clm4\PYGZus{}5/mksurfdata\PYGZus{}map/src
    gmake
    cd ..
    ./mksurfdata.pl \PYGZhy{}r usrspec \PYGZhy{}usr\PYGZus{}gname \PYGZdl{}GRIDNAME \PYGZhy{}usr\PYGZus{}gdate \PYGZdl{}CDATE
 8.) skip
 9.) skip

V.  Notes on which input datasets are needed for CLM

    global or regional/single\PYGZhy{}point grids
      \PYGZhy{} need fsurdata and fatmlndfrc

   fsurdata \PYGZhy{}\PYGZhy{}\PYGZhy{}\PYGZhy{} from mksurfdata\PYGZus{}map in step (III.7)
   fatmlndfrc \PYGZhy{}\PYGZhy{} use the domain.lnd file from gen\PYGZus{}domain in step (III.6)
   (NOTE: THIS FILE IS POINTED TO USING ATM\PYGZus{}DOMAIN\PYGZus{}PATH/ATM\PYGZus{}DOMAIN\PYGZus{}FILE/LND\PYGZus{}DOMAIN\PYGZus{}PATH/ \PYGZbs{}
LND\PYGZus{}DOMAIN\PYGZus{}FILE
          env\PYGZus{}run.xml variables \PYGZhy{}\PYGZhy{} do NOT simply add this to your user\PYGZus{}nl\PYGZus{}clm as it will fail)
\end{sphinxVerbatim}


\subsection{Datasets for Observational Sites}
\label{\detokenize{users_guide/using-clm-tools/datasts-for-observational-sites:datasets-for-observational-sites}}\label{\detokenize{users_guide/using-clm-tools/datasts-for-observational-sites::doc}}
There are two ways to customize datasets for a particular observational site.
The first is to customize the input to the tools that create the dataset, and the second is to over-write the default data after you’ve created a given dataset.
Depending on the tool it might be easier to do it one way or the other.
In \sphinxhref{CLM-URL}{Table 3-1} we list the files that are most likely to be customized and the way they might be customized.
Of those files, the ones you are most likely to customize are: fatmlndfrc, fsurdat, faerdep (for DATM), and stream\_fldfilename\_ndep.
Note \sphinxstylestrong{mksurfdata\_map} as documented previously has options to overwrite the vegetation and soil types.
For more information on this also see \sphinxhref{CLM-URL}{the Section called Creating your own single-point/regional surface datasets in Chapter 5}.
And PTCLM uses these methods to customize datasets see \sphinxhref{CLM-URL}{Chapter 6}.

Another aspect of customizing your input datasets is customizing the input atmospheric forcing datasets.
See \sphinxhref{CLM-URL}{the Section called Running with your own atmosphere forcing in Chapter 5} for more information on this.
Also the chapter on PTCLM in \sphinxhref{CLM-URL}{the Section called Converting AmeriFlux Data for use by PTCLM in Chapter 6} has information on using the AmeriFlux tower site data as atmospheric forcing.


\subsection{Creating CLM domain files}
\label{\detokenize{users_guide/using-clm-tools/creating-domain-files:creating-clm-domain-files}}\label{\detokenize{users_guide/using-clm-tools/creating-domain-files::doc}}\label{\detokenize{users_guide/using-clm-tools/creating-domain-files:creating-domain-files}}
\sphinxstyleemphasis{gen\_domain} to create a domain file for datm from a mapping file. The domain file is then used by BOTH DATM AND CLM to define the grid and land-mask. The general data flow is shown in two figures. \sphinxhref{CLM-URL}{Figure 2-4} shows the general flow for a general global case (or for a regional grid that DOES include ocean). \sphinxhref{CLM-URL}{Figure 2-5} shows the use of \sphinxstylestrong{mknoocnmap.pl} (see \sphinxhref{CLM-URL}{the Section called Using mknocnmap.pl to create grid and maps for single-point regional grids}) to create a regional or single-point map file that is then run through \sphinxstylestrong{gen\_domain} to create the domain file for it. As stated before \sphinxhref{CLM-URL}{Figure 2-2} is the legend for both of these figures. See \sphinxhref{CLM-URL}{the
tools/mapping/gen\_domain\_files/README} file for more help on \sphinxstylestrong{gen\_domain}.

Here we create domain files for a regular global domain.


\subsubsection{Figure 2-4. Global Domain file creation}
\label{\detokenize{users_guide/using-clm-tools/creating-domain-files:figure-2-4-global-domain-file-creation}}
Insert figure 2-4

Starting from SCRIP grid files for both your atmosphere and ocean, you use \sphinxstylestrong{tools/mapping/gen\_mapping\_files/gen\_cesm\_maps.sh} to create a mapping file between the atmosphere and ocean. That mapping file is then used as input to \sphinxstylestrong{gen\_domain} to create output domain files for both atmosphere and ocean. The atmosphere domain file is then used by both CLM and DATM for I compsets, while the ocean domain file is ignored. For this process you have to define your SCRIP grid files on your own. For a regional or single-point case that doesn’t include ocean see \sphinxhref{CLM-URL}{Figure 2-5}. (See \sphinxhref{CLM-URL}{Figure 2-2} for the legend for this figure.)

Note, that the SCRIP grid file used to start this process, is also used in \sphinxstylestrong{mkmapdata.sh} (see \sphinxhref{CLM-URL}{the Section called Creating mapping files that mksurfdata\_map will use}). Next we create domain files for a single-point or regional domain.


\subsubsection{Figure 2-5. Domain file creation using mknoocnmap.pl}
\label{\detokenize{users_guide/using-clm-tools/creating-domain-files:figure-2-5-domain-file-creation-using-mknoocnmap-pl}}
Insert figure 2-5

For a regular latitude/longitude grid that can be used for regional or single point simulations \textendash{} you can use \sphinxstylestrong{mknoocnmap.pl}. It creates a SCRIP grid file that can then be used as input to \sphinxstylestrong{mkmapdata.sh} as well as a SCRIP mapping file that is then input to \sphinxstylestrong{gen\_domain}. The output of \sphinxstylestrong{gen\_domain} is a atmosphere domain file used by both CLM and DATM and a ocean domain file that is ignored. (See \sphinxhref{CLM-URL}{Figure 2-2} for the legend for this figure.)

In this case the process creates both SCRIP grid files to be used by \sphinxstylestrong{mkmapdata.sh} as well as the domain files that will be used by both CLM and DATM.


\subsection{Comparing History Files}
\label{\detokenize{users_guide/using-clm-tools/cprnc::doc}}\label{\detokenize{users_guide/using-clm-tools/cprnc:comparing-history-files}}
\sphinxstylestrong{cprnc} is a tool shared by both CAM and CLM to compare two NetCDF history files.
It differences every field that has a time-axis that is also shared on both files, and reports a summary of the difference.
The summary includes the three largest differences, as well as the root mean square (RMS) difference.
It also gives some summary information on the field as well.
You have to enter at least one file, and up to two files.
With one file it gives you summary information on the file, and with two it gives you information on the differences between the two.
At the end it will give you a summary of the fields compared and how many fields were different and how many were identical.

Options:

-m = do NOT align time-stamps before comparing

-v = verbose output

-ipr

-jpr

-kpr

See the \sphinxstylestrong{cprnc} \sphinxhref{CLM-URL}{README} file for more details.

\begin{sphinxadmonition}{note}{Note:}
To compare files with OUT a time axis you can use the \sphinxstylestrong{cprnc.ncl} NCL script in \sphinxcode{models/lnd/clm/tools/shared/ncl\_scripts}. It won’t give you the details on the differences but will report if the files are identical or different.
\end{sphinxadmonition}


\section{Adding New Resolutions}
\label{\detokenize{users_guide/adding-new-resolutions/index:adding-new-resolutions-section}}\label{\detokenize{users_guide/adding-new-resolutions/index::doc}}\label{\detokenize{users_guide/adding-new-resolutions/index:adding-new-resolutions}}

\subsection{Adding New Resolutions}
\label{\detokenize{users_guide/adding-new-resolutions/CLM-3.0-Adding-New-Resolutions-or-New-Files-to-the-build-namelist-Database:adding-resolutions}}\label{\detokenize{users_guide/adding-new-resolutions/CLM-3.0-Adding-New-Resolutions-or-New-Files-to-the-build-namelist-Database::doc}}\label{\detokenize{users_guide/adding-new-resolutions/CLM-3.0-Adding-New-Resolutions-or-New-Files-to-the-build-namelist-Database:adding-new-resolutions}}
In the last chapter we gave the details on how to create new files for input into CLM.
These files could be either global resolutions, regional-grids or even a single grid point.
If you want to easily have these files available for continued use in your development you will then want to include them in the build-namelist database so that build-namelist can easily find them for you.
You can deal with them, just by putting the settings in the \sphinxcode{user\_nl\_clm namelist} file, or by using \sphinxcode{CLM\_USRDAT\_NAME}.
Another way to deal with them is to enter them into the database for build-namelist, so that build-namelist can find them for you.
This keeps one central database for all your files, rather than having multiple locations to keep track of files.
If you have a LOT of files to keep track of it also might be easier than keeping track by hand, especially if you have to periodically update your files.
If you just have a few quick experiments to try, for a short time period you might be best off using the other methods mentioned above.

There are two parts to adding files to the build-namelist database.
The first part is adding new resolution names which is done in the \sphinxcode{models/lnd/clm/bld/namelist\_files/namelist\_definition\_clm4\_5.xml} file (and in the \sphinxcode{models/lnd/clm/bld/config\_files/config\_definition.xml} file when adding supported single-point datasets).
You can then use the new resolution by using \sphinxcode{CLM\_USRDAT\_NAME}.
If you also want to be able to give the resolution into \sphinxstylestrong{create\_newcase} \textendash{} you’ll need to add the grid to the \sphinxcode{scripts/ccsm\_utils/Case.template/config\_grid.xml} file.

The second part is actually adding the new filenames which is done in the \sphinxcode{models/lnd/clm/bld/namelist\_files/namelist\_defaults\_clm4\_5.xml} file (\sphinxcode{models/lnd/clm/bld/namelist\_files/namelist\_defaults\_clm4\_5\_tools.xml} file for CLM tools).
If you aren’t adding any new resolutions, and you are just changing the files for existing resolutions, you don’t need to edit the namelist\_definition file.


\subsection{Managing Your Data Own Files}
\label{\detokenize{users_guide/adding-new-resolutions/CLM-3.1-Managing-Your-Own-Data-files:managing-your-data-own-files}}\label{\detokenize{users_guide/adding-new-resolutions/CLM-3.1-Managing-Your-Own-Data-files::doc}}\label{\detokenize{users_guide/adding-new-resolutions/CLM-3.1-Managing-Your-Own-Data-files:managing-your-data-files}}
If you are running on a supported machine (such as yellowstone or hopper) the standard input datasets will already be available and you won’t have to check them out of the subversion inputdata server. However, you also will NOT be able to add your own datafiles to these standard inputdata directories \textendash{} because most likely you won’t have permissions to do so. In order to add files to the XML database or to use \sphinxcode{CLM\_USRDAT\_NAME} you need to put data in the standard locations so that they can be found. The recommended way to do this is to use the \sphinxstylestrong{link\_dirtree} tool in the CESM scripts. Some information on \sphinxstylestrong{link\_dirtree} is available in the \sphinxhref{CLM-URL}{CESM1.2.0 Scripts User’s Guide}. We also have some examples of it’s use here and in other sections of this User’s Guide.

Using \sphinxstylestrong{link\_dirtree} is quite simple, you give the directory where data exists and then the directory that you want to create where datasets will point to the original source files. In the example below we use “\$HOME/inputdata”, but MYCSMDATA could be any directory you have access to where you want to put your data.

\sphinxcode{{}`
\textgreater{} cd scripts
\# First make sure you have a inputdata location that you can write to
\# You only need to do this step once, so you won't need to do this in the future
\# (except to bring in any updated files in the original \$CSMDATA location).
\textgreater{} setenv MYCSMDATA \$HOME/inputdata    \# Set env var for the directory for input data
\textgreater{} ./link\_dirtree \$CSMDATA \$MYCSMDATA
{}`}

Then when you create a case you will change \sphinxcode{DIN\_LOC\_ROOT\_CSMDATA} to point to the location you linked to rather than the default location.

\sphinxcode{\textgreater{} ./xmlchange DIN\_LOC\_ROOT\_CSMDATA=\$MYCSMDATA}

In order to list the files that you have created you merely need to use the UNIX command find to find the files that are NOT softlinks. So for example executing the following command:

\sphinxcode{\textgreater{} find \$MYCSMDATA -type f -print}

for me gives the following truncated list of CLM\_USRDAT\_NAME files that I have created.

\sphinxcode{{}`
/glade/p/work/erik/inputdata/lnd/clm2/pftdata/pft-physiology.c130503.nc
/glade/p/work/erik/inputdata/atm/datm7/CLM1PT\_data/1x1pt\_BE-Vie/1997-01.nc
/glade/p/work/erik/inputdata/atm/datm7/CLM1PT\_data/1x1pt\_BE-Vie/1997-02.nc
/glade/p/work/erik/inputdata/atm/datm7/CLM1PT\_data/1x1pt\_BE-Vie/1997-03.nc
/glade/p/work/erik/inputdata/atm/datm7/CLM1PT\_data/1x1pt\_BE-Vie/1997-04.nc
{}`}

You can also use \sphinxstylestrong{find} to list files that have a particular pattern in the name as well (using the -name option with wildcards). Also you can always rerun the \sphinxstylestrong{link\_dirtree} command if any new files are added that you need to be linked into your directory tree. Since, the files are soft-links \textendash{} it doesn’t take up much space other than the files that you add there. This way all of the files are kept in one place, they are organized by usage according to CESM standards, and you can easily find your own files, and CLM can find them as well.

If you are running on a supported machine (such as yellowstone or hopper) the standard input datasets will already be available and you won’t have to check them out of the subversion inputdata server. However, you also will NOT be able to add your own datafiles to these standard inputdata directories \textendash{} because most likely you won’t have permissions to do so. In order to add files to the XML database or to use \sphinxcode{CLM\_USRDAT\_NAME} you need to put data in the standard locations so that they can be found. The recommended way to do this is to use the \sphinxstylestrong{link\_dirtree} tool in the CESM scripts. Some information on \sphinxstylestrong{link\_dirtree} is available in the \sphinxhref{CLM-URL}{CESM1.2.0 Scripts User’s Guide}. We also have some examples of it’s use here and in other sections of this User’s Guide.

Using \sphinxstylestrong{link\_dirtree} is quite simple, you give the directory where data exists and then the directory that you want to create where datasets will point to the original source files. In the example below we use “\$HOME/inputdata”, but MYCSMDATA could be any directory you have access to where you want to put your data.

\sphinxcode{{}`
\textgreater{} cd scripts
\# First make sure you have a inputdata location that you can write to
\# You only need to do this step once, so you won't need to do this in the future
\# (except to bring in any updated files in the original \$CSMDATA location).
\textgreater{} setenv MYCSMDATA \$HOME/inputdata    \# Set env var for the directory for input data
\textgreater{} ./link\_dirtree \$CSMDATA \$MYCSMDATA
{}`}

Then when you create a case you will change \sphinxcode{DIN\_LOC\_ROOT\_CSMDATA} to point to the location you linked to rather than the default location.

\sphinxcode{\textgreater{} ./xmlchange DIN\_LOC\_ROOT\_CSMDATA=\$MYCSMDATA}

In order to list the files that you have created you merely need to use the UNIX command find to find the files that are NOT softlinks. So for example executing the following command:

\sphinxcode{\textgreater{} find \$MYCSMDATA -type f -print}

for me gives the following truncated list of CLM\_USRDAT\_NAME files that I have created.

\sphinxcode{{}`
/glade/p/work/erik/inputdata/lnd/clm2/pftdata/pft-physiology.c130503.nc
/glade/p/work/erik/inputdata/atm/datm7/CLM1PT\_data/1x1pt\_BE-Vie/1997-01.nc
/glade/p/work/erik/inputdata/atm/datm7/CLM1PT\_data/1x1pt\_BE-Vie/1997-02.nc
/glade/p/work/erik/inputdata/atm/datm7/CLM1PT\_data/1x1pt\_BE-Vie/1997-03.nc
/glade/p/work/erik/inputdata/atm/datm7/CLM1PT\_data/1x1pt\_BE-Vie/1997-04.nc
{}`}

You can also use \sphinxstylestrong{find} to list files that have a particular pattern in the name as well (using the -name option with wildcards). Also you can always rerun the \sphinxstylestrong{link\_dirtree} command if any new files are added that you need to be linked into your directory tree. Since, the files are soft-links \textendash{} it doesn’t take up much space other than the files that you add there. This way all of the files are kept in one place, they are organized by usage according to CESM standards, and you can easily find your own files, and CLM can find them as well.


\subsection{Adding Resolution Names}
\label{\detokenize{users_guide/adding-new-resolutions/CLM-3.2-Adding-Resolution-Names:adding-resolution-names}}\label{\detokenize{users_guide/adding-new-resolutions/CLM-3.2-Adding-Resolution-Names::doc}}\label{\detokenize{users_guide/adding-new-resolutions/CLM-3.2-Adding-Resolution-Names:id1}}
If you are adding files for new resolutions which aren’t covered in the namelist\_definition file \textendash{} you’ll need to add them in.
The list of valid resolutions is in the id=”res” entry in the \sphinxcode{models/lnd/clm/bld/namelist\_files/namelist\_definition\_clm4\_5.xml} file.
You need to choose a name for your new resolution and simply add it to the comma delimited list of valid\_values for the id=”res” entry.
The convention for global Gaussian grids is number\_of\_latitudes x number\_of\_longitudes.
The convention for global finite volume grids is latitude\_grid\_size x longitude\_grid\_size where latitude and longitude is measured in degrees.
The convention for unstructured HOMME grids is ne\textless{}size\textgreater{}np4, where \textless{}size\textgreater{} corresponds to the resolution.
The higher \textless{}size\textgreater{} is the higher the resolution.
So for example, ne60np4 is roughly half-degree while ne240np4 is roughly a eighth degree.
For regional or single-point datasets the names have a grid size number\_of\_latitudes x number\_of\_longitudes followed by an underscore and then a descriptive name such as a City name followed by an abbreviation for the Country in caps.
The only hard requirement is that names be unique for different grid files. Here’s what the entry for resolutions looks like in the file:

\begin{sphinxVerbatim}[commandchars=\\\{\}]
\PYG{o}{\PYGZlt{}}\PYG{n}{entry} \PYG{n+nb}{id}\PYG{o}{=}\PYG{l+s+s2}{\PYGZdq{}}\PYG{l+s+s2}{res}\PYG{l+s+s2}{\PYGZdq{}} \PYG{n+nb}{type}\PYG{o}{=}\PYG{l+s+s2}{\PYGZdq{}}\PYG{l+s+s2}{char*30}\PYG{l+s+s2}{\PYGZdq{}} \PYG{n}{category}\PYG{o}{=}\PYG{l+s+s2}{\PYGZdq{}}\PYG{l+s+s2}{default\PYGZus{}settings}\PYG{l+s+s2}{\PYGZdq{}}
       \PYG{n}{group}\PYG{o}{=}\PYG{l+s+s2}{\PYGZdq{}}\PYG{l+s+s2}{default\PYGZus{}settings}\PYG{l+s+s2}{\PYGZdq{}}
       \PYG{n}{valid\PYGZus{}values}\PYG{o}{=}
       \PYG{l+s+s2}{\PYGZdq{}}\PYG{l+s+s2}{512x1024,360x720cru,128x256,64x128,48x96,32x64,8x16,94x192,0.23x0.31,0.9x1.25,1.9x2.5,2.5x3.33,}
       \PYG{l+m+mi}{4}\PYG{n}{x5}\PYG{p}{,}\PYG{l+m+mi}{10}\PYG{n}{x15}\PYG{p}{,}\PYG{l+m+mi}{5}\PYG{n}{x5\PYGZus{}amazon}\PYG{p}{,}\PYG{l+m+mi}{1}\PYG{n}{x1\PYGZus{}tropicAtl}\PYG{p}{,}\PYG{l+m+mi}{1}\PYG{n}{x1\PYGZus{}camdenNJ}\PYG{p}{,}\PYG{l+m+mi}{1}\PYG{n}{x1\PYGZus{}vacouverCAN}\PYG{p}{,}\PYG{l+m+mi}{1}\PYG{n}{x1\PYGZus{}mexicocityMEX}\PYG{p}{,}\PYG{l+m+mi}{1}\PYG{n}{x1\PYGZus{}asphaltjungleNJ}\PYG{p}{,}
       \PYG{l+m+mi}{1}\PYG{n}{x1\PYGZus{}brazil}\PYG{p}{,}\PYG{l+m+mi}{1}\PYG{n}{x1\PYGZus{}urbanc\PYGZus{}alpha}\PYG{p}{,}\PYG{l+m+mi}{1}\PYG{n}{x1\PYGZus{}numaIA}\PYG{p}{,}\PYG{l+m+mi}{1}\PYG{n}{x1\PYGZus{}smallvilleIA}\PYG{p}{,}\PYG{l+m+mf}{0.1}\PYG{n}{x0}\PYG{o}{.}\PYG{l+m+mi}{1}\PYG{p}{,}\PYG{l+m+mf}{0.5}\PYG{n}{x0}\PYG{o}{.}\PYG{l+m+mi}{5}\PYG{p}{,}\PYG{l+m+mi}{3}\PYG{n}{x3min}\PYG{p}{,}\PYG{l+m+mi}{5}\PYG{n}{x5min}\PYG{p}{,}\PYG{l+m+mi}{10}\PYG{n}{x10min}\PYG{p}{,}\PYG{l+m+mf}{0.33}\PYG{n}{x0}\PYG{o}{.}\PYG{l+m+mi}{3}\PYG{p}{,}
       \PYG{n}{ne4np4}\PYG{p}{,}\PYG{n}{ne16np4}\PYG{p}{,}\PYG{n}{ne30np4}\PYG{p}{,}\PYG{n}{ne60np4}\PYG{p}{,}\PYG{n}{ne120np4}\PYG{p}{,}\PYG{n}{ne240np4}\PYG{p}{,}\PYG{n}{wus12}\PYG{p}{,}\PYG{n}{us20}\PYG{p}{,}\PYG{l+m+mi}{1}\PYG{n}{km}\PYG{o}{\PYGZhy{}}\PYG{n}{merge}\PYG{o}{\PYGZhy{}}\PYG{l+m+mi}{10}\PYG{n+nb}{min}\PYG{l+s+s2}{\PYGZdq{}}\PYG{l+s+s2}{\PYGZgt{}}
       \PYG{n}{Horizontal} \PYG{n}{resolutions}
       \PYG{n}{Note}\PYG{p}{:} \PYG{l+m+mf}{0.1}\PYG{n}{x0}\PYG{o}{.}\PYG{l+m+mi}{1}\PYG{p}{,} \PYG{l+m+mf}{0.5}\PYG{n}{x0}\PYG{o}{.}\PYG{l+m+mi}{5}\PYG{p}{,} \PYG{l+m+mi}{5}\PYG{n}{x5min}\PYG{p}{,} \PYG{l+m+mi}{10}\PYG{n}{x10min}\PYG{p}{,} \PYG{l+m+mi}{3}\PYG{n}{x3min}\PYG{p}{,} \PYG{l+m+mi}{1}\PYG{n}{km}\PYG{o}{\PYGZhy{}}\PYG{n}{merge}\PYG{o}{\PYGZhy{}}\PYG{l+m+mi}{10}\PYG{n+nb}{min}\PYG{p}{,} \PYG{o+ow}{and} \PYG{l+m+mf}{0.33}\PYG{n}{x0}\PYG{o}{.}\PYG{l+m+mi}{33} \PYG{n}{are} \PYG{n}{only} \PYG{n}{used} \PYG{k}{for} \PYG{n}{CLM} \PYG{n}{tools}
\PYG{o}{\PYGZlt{}}\PYG{o}{/}\PYG{n}{entry}\PYG{o}{\PYGZgt{}}
\end{sphinxVerbatim}

As you can see you just add your new resolution names to the end of the valid\_values list.

When using PTCLM and adding supported single-point resolutions, you’ll also want to add these resolutions to the \sphinxcode{models/lnd/clm/bld/config\_files/config\_definition.xml} under the \sphinxcode{sitespf\_pt} name.
The entry in that file looks like:

\begin{sphinxVerbatim}[commandchars=\\\{\}]
\PYG{o}{\PYGZlt{}}\PYG{n}{entry} \PYG{n+nb}{id}\PYG{o}{=}\PYG{l+s+s2}{\PYGZdq{}}\PYG{l+s+s2}{sitespf\PYGZus{}pt}\PYG{l+s+s2}{\PYGZdq{}}
     \PYG{n}{valid\PYGZus{}values}\PYG{o}{=}\PYG{l+s+s2}{\PYGZdq{}}\PYG{l+s+s2}{none,1x1\PYGZus{}brazil,1x1\PYGZus{}tropicAtl,5x5\PYGZus{}amazon,1x1\PYGZus{}camdenNJ,1x1\PYGZus{}vancouverCAN,}
     \PYG{l+m+mi}{1}\PYG{n}{x1\PYGZus{}mexicocityMEX}\PYG{p}{,}\PYG{l+m+mi}{1}\PYG{n}{x1\PYGZus{}asphaltjungleNJ}\PYG{p}{,}\PYG{l+m+mi}{1}\PYG{n}{x1\PYGZus{}urbanc\PYGZus{}alpha}\PYG{p}{,}\PYG{l+m+mi}{1}\PYG{n}{x1\PYGZus{}numaIA}\PYG{p}{,}\PYG{l+m+mi}{1}\PYG{n}{x1\PYGZus{}smallvilleIA}\PYG{p}{,}\PYG{n}{us20}\PYG{p}{,}\PYG{n}{wus12}\PYG{l+s+s2}{\PYGZdq{}}
     \PYG{n}{value}\PYG{o}{=}\PYG{l+s+s2}{\PYGZdq{}}\PYG{l+s+s2}{none}\PYG{l+s+s2}{\PYGZdq{}} \PYG{n}{category}\PYG{o}{=}\PYG{l+s+s2}{\PYGZdq{}}\PYG{l+s+s2}{physics}\PYG{l+s+s2}{\PYGZdq{}}\PYG{o}{\PYGZgt{}}
     \PYG{n}{Flag} \PYG{n}{to} \PYG{n}{turn} \PYG{n}{on} \PYG{n}{site} \PYG{n}{specific} \PYG{n}{special} \PYG{n}{configuration} \PYG{n}{flags} \PYG{k}{for} \PYG{n}{supported} \PYG{n}{single}
     \PYG{n}{point} \PYG{n}{resolutions}\PYG{o}{.} \PYG{n}{See} \PYG{n}{the} \PYG{n}{specific} \PYG{n}{config\PYGZus{}defaults\PYGZus{}}\PYG{o}{*}\PYG{o}{.}\PYG{n}{xml} \PYG{n}{file} \PYG{k}{for} \PYG{n}{the} \PYG{n}{special}
     \PYG{n}{settings} \PYG{n}{that} \PYG{n}{are} \PYG{n+nb}{set} \PYG{k}{for} \PYG{n}{a} \PYG{n}{particular} \PYG{n}{site}\PYG{o}{.}
\PYG{o}{\PYGZlt{}}\PYG{o}{/}\PYG{n}{entry}\PYG{o}{\PYGZgt{}}
\end{sphinxVerbatim}

PTCLM assumes that any supported single-point resolutions are valid settings for \sphinxcode{sitespf\_pt}.


\subsection{Changing Default Filenames}
\label{\detokenize{users_guide/adding-new-resolutions/CLM-3.3-Adding-or-Changing-Default-Filenames:changing-default-filenames}}\label{\detokenize{users_guide/adding-new-resolutions/CLM-3.3-Adding-or-Changing-Default-Filenames::doc}}\label{\detokenize{users_guide/adding-new-resolutions/CLM-3.3-Adding-or-Changing-Default-Filenames:id1}}
To add or change the default filenames you edit the \sphinxcode{models/lnd/clm/bld/namelist\_files/namelist\_defaults\_clm4\_5.xml} and either change an existing filename or add a new one.
Most entries in the default namelist files, include different attributes that describe the different properties that describe the differences in the datasets.
Attributes include the: resolution, year to simulation, range of years to simulate for transient datafiles, the land-mask, the representative concentration pathway (rcp) for future scenarios, and the type of biogeochemistry (bgc) model used.
For example the fatmgrid for the 1.9x2.5 resolution is as follows:

\sphinxcode{{}`
\textless{}fsurdat hgrid="0.9x1.25" sim\_year="1850" crop="off" \textgreater{}
lnd/clm2/surfdata\_map/surfdata\_0.9x1.25\_simyr1850\_c130415.nc\textless{}/fsurdat\textgreater{}
\textless{}/fsurdat\textgreater{}
{}`}

Other \sphinxcode{fsurdat} files are distinguished from this one by their resolution (hgrid), simulation year (sim\_year) and prognostic crop (crop) attributes.

To add or change the default filenames for CLM tools edit the \sphinxcode{models/lnd/clm/bld/namelist\_files/namelist\_defaults\_clm4\_5\_tools.xml} and either change an existing filename or add a new one.
Editing this file is similar to the \sphinxcode{namelist\_defaults\_clm4\_5.xml} talked about above.


\subsubsection{What are the required files?}
\label{\detokenize{users_guide/adding-new-resolutions/CLM-3.3-Adding-or-Changing-Default-Filenames:what-are-the-required-files}}
Different types of simulations and different types of configurations for CLM require different lists of files.
The CLM4.5-BGC or Carbon Nitrogen (cn) Biogeochemistry model for example requires \sphinxcode{stream\_fldfilename\_ndep} files, which are NOT required by CLMSP.
Transient simulations also require transient datasets, and the names of these datasets are sometimes different from the static versions (sometimes both are required as in the dynamic PFT cases).

In the following table we list the different files used by CLM, they are listed in order of importance, dependencies, and customizing.
So the required files are all near the top, and the files used only under different conditions are listed later, and files with the fewest dependencies are near the top, as are the files that are least likely to be customized.


\paragraph{Table 3-1. Required Files for Different Configurations and Simulation Types}
\label{\detokenize{users_guide/adding-new-resolutions/CLM-3.3-Adding-or-Changing-Default-Filenames:table-3-1-required-files-for-different-configurations-and-simulation-types}}
Insert table 3-1


\section{Running Special Cases}
\label{\detokenize{users_guide/running-special-cases/index:running-special-cases-section}}\label{\detokenize{users_guide/running-special-cases/index:running-special-cases}}\label{\detokenize{users_guide/running-special-cases/index::doc}}

\subsection{What is a special case?}
\label{\detokenize{users_guide/running-special-cases/what-is-a-special-case::doc}}\label{\detokenize{users_guide/running-special-cases/what-is-a-special-case:what-is-a-special-case}}\label{\detokenize{users_guide/running-special-cases/what-is-a-special-case:id1}}
All of the following special cases  cases that take more than one step to do.
The straightforward cases have compsets and/or build-namelist use-cases setup for them or require simple editing of a single-case.
All of the cases here require you to do at least two simulations with different configurations, or require more complex editing of the case (changing the streams files).

\begin{sphinxadmonition}{note}{Note:}
The cases in this chapter are more sophisticated and require more technical knowledge and skill than cases in previous chapters. The user should be very familiar with doing simple cases before moving onto the cases described here.
\end{sphinxadmonition}


\subsection{Running the prognostic crop model}
\label{\detokenize{users_guide/running-special-cases/running-the-prognostic-crop-model:running-the-prognostic-crop-model}}\label{\detokenize{users_guide/running-special-cases/running-the-prognostic-crop-model::doc}}
The prognostic crop model is setup to work with CLM4.5-BGC or CLM4.0-CN (with or without DV) for present day conditions and we have surface and initial condition datasets at f19 resolution.
In order to use the initial condition file, we need to set the \sphinxcode{RUN\_TYPE} to startup rather than \sphinxcode{hybrid} since the compset for f19 sets up to use an initial condition file without crop active.
To activate the crop model you can choose a compset that has “CROP” in the name such as “ICRUCLM45BGCCROP” or simply add “-crop on” to \sphinxcode{CLM\_CONFIG\_OPTS}.


\subsubsection{Example: Crop Simulation}
\label{\detokenize{users_guide/running-special-cases/running-the-prognostic-crop-model:example-crop-simulation}}
\begin{sphinxVerbatim}[commandchars=\\\{\}]
\PYG{o}{\PYGZgt{}} \PYG{n}{cd} \PYG{n}{scripts}
\PYG{o}{\PYGZgt{}} \PYG{o}{.}\PYG{o}{/}\PYG{n}{create\PYGZus{}newcase} \PYG{o}{\PYGZhy{}}\PYG{n}{case} \PYG{n}{CROP} \PYG{o}{\PYGZhy{}}\PYG{n}{res} \PYG{n}{f19\PYGZus{}g16} \PYG{o}{\PYGZhy{}}\PYG{n}{compset} \PYG{n}{I1850CRUCLM45BGC}
\PYG{o}{\PYGZgt{}} \PYG{n}{cd} \PYG{n}{CROP}

\PYG{c+c1}{\PYGZsh{} Append \PYGZdq{}\PYGZhy{}crop on\PYGZdq{} to CLM\PYGZus{}CONFIG\PYGZus{}OPTS in env\PYGZus{}build.xml (you could also use an editor)}
\PYG{o}{\PYGZgt{}} \PYG{o}{.}\PYG{o}{/}\PYG{n}{xmlchange} \PYG{n}{CLM\PYGZus{}CONFIG\PYGZus{}OPTS}\PYG{o}{=}\PYG{l+s+s2}{\PYGZdq{}}\PYG{l+s+s2}{\PYGZhy{}crop on}\PYG{l+s+s2}{\PYGZdq{}} \PYG{o}{\PYGZhy{}}\PYG{n}{append}

\PYG{c+c1}{\PYGZsh{} Change to startup type so uses spunup initial conditions file for crop if it exists}
\PYG{c+c1}{\PYGZsh{} By default the model will do a hybrid startup with an initial condition file}
\PYG{c+c1}{\PYGZsh{} incompatible with the crop surface dataset.}

\PYG{o}{\PYGZgt{}} \PYG{o}{.}\PYG{o}{/}\PYG{n}{xmlchange} \PYG{n}{RUN\PYGZus{}TYPE}\PYG{o}{=}\PYG{n}{startup}
\PYG{o}{\PYGZgt{}} \PYG{o}{.}\PYG{o}{/}\PYG{n}{case}\PYG{o}{.}\PYG{n}{setup}

\PYG{c+c1}{\PYGZsh{} Now build and run normally}
\PYG{o}{\PYGZgt{}} \PYG{o}{.}\PYG{o}{/}\PYG{n}{case}\PYG{o}{.}\PYG{n}{build}
\PYG{o}{\PYGZgt{}} \PYG{o}{.}\PYG{o}{/}\PYG{n}{case}\PYG{o}{.}\PYG{n}{submit}
\end{sphinxVerbatim}


\subsection{Running with irrigation}
\label{\detokenize{users_guide/running-special-cases/running-with-irrigation:running-with-irrigation}}\label{\detokenize{users_guide/running-special-cases/running-with-irrigation::doc}}
In CLM4.5 irrigation can ONLY be used WITH crop.
To turn on irrigation in CLM4.5 we simply add “-irrig on” to \sphinxcode{CLM\_BLDNML\_OPTS}.
Just as in the crop example we also change \sphinxcode{RUN\_TYPE} to \sphinxcode{startup} so that we don’t use an initial condition file that is incompatible with irrigation.


\subsubsection{Example: Irrigation Simulation}
\label{\detokenize{users_guide/running-special-cases/running-with-irrigation:example-irrigation-simulation}}
\begin{sphinxVerbatim}[commandchars=\\\{\}]
\PYG{c+c1}{\PYGZsh{} Note here we do a CLMSP simulation as that is what has been validated}
\PYG{o}{\PYGZgt{}} \PYG{n}{cd} \PYG{n}{scripts}
\PYG{o}{\PYGZgt{}} \PYG{o}{.}\PYG{o}{/}\PYG{n}{create\PYGZus{}newcase} \PYG{o}{\PYGZhy{}}\PYG{n}{case} \PYG{n}{IRRIG} \PYG{o}{\PYGZhy{}}\PYG{n}{res} \PYG{n}{f19\PYGZus{}g16} \PYG{o}{\PYGZhy{}}\PYG{n}{compset} \PYG{n}{I}
\PYG{o}{\PYGZgt{}} \PYG{n}{cd} \PYG{n}{IRRIG}

\PYG{c+c1}{\PYGZsh{} Append \PYGZdq{}\PYGZhy{}irrig\PYGZdq{} to CLM\PYGZus{}BLDNML\PYGZus{}OPTS in env\PYGZus{}run.xml (you could also use an editor)}
\PYG{o}{\PYGZgt{}} \PYG{o}{.}\PYG{o}{/}\PYG{n}{xmlchange} \PYG{n}{CLM\PYGZus{}BLDNML\PYGZus{}OPTS}\PYG{o}{=}\PYG{l+s+s2}{\PYGZdq{}}\PYG{l+s+s2}{\PYGZhy{}irrig}\PYG{l+s+s2}{\PYGZdq{}} \PYG{o}{\PYGZhy{}}\PYG{n}{append}

\PYG{c+c1}{\PYGZsh{} Change to startup type so uses spunup initial conditions file for irrigation if it exists}
\PYG{c+c1}{\PYGZsh{} By default the model will do a hybrid startup with an initial condition file}
\PYG{c+c1}{\PYGZsh{} incompatible with the irrigation surface dataset.}
\PYG{o}{\PYGZgt{}} \PYG{o}{.}\PYG{o}{/}\PYG{n}{xmlchange} \PYG{n}{RUN\PYGZus{}TYPE}\PYG{o}{=}\PYG{n}{startup}
\PYG{o}{\PYGZgt{}} \PYG{o}{.}\PYG{o}{/}\PYG{n}{case}\PYG{o}{.}\PYG{n}{setup}

\PYG{c+c1}{\PYGZsh{} Now build and run normally}
\PYG{o}{\PYGZgt{}} \PYG{o}{.}\PYG{o}{/}\PYG{n}{case}\PYG{o}{.}\PYG{n}{build}
\PYG{o}{\PYGZgt{}} \PYG{o}{.}\PYG{o}{/}\PYG{n}{case}\PYG{o}{.}\PYG{n}{submit}
\end{sphinxVerbatim}


\subsection{Spinning up the Satellite Phenology Model}
\label{\detokenize{users_guide/running-special-cases/CLM-4.3-Spinning-up-the-Satellite-Phenology-Model-(CLMSP-spinup):spinning-up-the-satellite-phenology-model}}\label{\detokenize{users_guide/running-special-cases/CLM-4.3-Spinning-up-the-Satellite-Phenology-Model-(CLMSP-spinup)::doc}}\label{\detokenize{users_guide/running-special-cases/CLM-4.3-Spinning-up-the-Satellite-Phenology-Model-(CLMSP-spinup):spinning-up-sp}}
To spin-up the CLMSP model you merely need to run CLMSP for 50 simulation years starting from arbitrary initial conditions.
You then use the final restart file for initial conditions in other simulations.
Because, this is a straight forward operation we will NOT give the details on how to do that here, but leave it as an exercise for the reader.
See the \sphinxhref{CLM-URL}{Example 4-7} as an example of doing this as the last step for CLMCN.


\subsection{Spinup of CLM4.5-BGC}
\label{\detokenize{users_guide/running-special-cases/CLM-4.4-Spinning-up-the-CLM4.5-biogeochemistry-(CLMBGC-spinup)::doc}}\label{\detokenize{users_guide/running-special-cases/CLM-4.4-Spinning-up-the-CLM4.5-biogeochemistry-(CLMBGC-spinup):spinning-up-clm45-bgc}}\label{\detokenize{users_guide/running-special-cases/CLM-4.4-Spinning-up-the-CLM4.5-biogeochemistry-(CLMBGC-spinup):spinup-of-clm4-5-bgc}}
To get the CLM4.5-BGC model to a steady state, you first run it from arbitrary initial conditions using the “accelerated decomposition spinup” (-bgc\_spinup on in CLM \sphinxstylestrong{configure}) mode for 1000 simulation years.
After this you branch from this mode in the “final spinup” (-bgc\_spinup off in CLM \sphinxstylestrong{configure}), and run for (at least 200+ simulation years).
\begin{description}
\item[{\sphinxstylestrong{1. 45\_AD\_SPINUP}}] \leavevmode
For the first step of running 1000+ years in “-bgc\_spinup on” mode, you will setup a case, and then edit the values in env\_build.xml and env\_run.xml so that the right configuration is turned on and the simulation is setup to run for the required length of simulation time. So do the following:

\end{description}


\subsubsection{Example:: AD\_SPINUP Simulation for CLM4.5-BGC}
\label{\detokenize{users_guide/running-special-cases/CLM-4.4-Spinning-up-the-CLM4.5-biogeochemistry-(CLMBGC-spinup):example-ad-spinup-simulation-for-clm4-5-bgc}}
\begin{sphinxVerbatim}[commandchars=\\\{\}]
\PYG{o}{\PYGZgt{}} \PYG{n}{cd} \PYG{n}{scripts}
\PYG{o}{\PYGZgt{}} \PYG{o}{.}\PYG{o}{/}\PYG{n}{create\PYGZus{}newcase} \PYG{o}{\PYGZhy{}}\PYG{n}{case} \PYG{n}{BGC\PYGZus{}spinup} \PYG{o}{\PYGZhy{}}\PYG{n}{res} \PYG{n}{f19\PYGZus{}g16} \PYG{o}{\PYGZhy{}}\PYG{n}{compset} \PYG{n}{I1850CRUCLM45BGC} \PYG{o}{\PYGZhy{}}\PYG{n}{mach} \PYG{n}{yellowstone\PYGZus{}intel}
\PYG{o}{\PYGZgt{}} \PYG{n}{cd} \PYG{n}{BGC\PYGZus{}spinup}
\PYG{c+c1}{\PYGZsh{} Append \PYGZdq{}\PYGZhy{}spinup on\PYGZdq{} to CLM\PYGZus{}BLDNML\PYGZus{}OPTS}
\PYG{o}{\PYGZgt{}} \PYG{o}{.}\PYG{o}{/}\PYG{n}{xmlchange} \PYG{n}{CLM\PYGZus{}BLDNML\PYGZus{}OPTS}\PYG{o}{=}\PYG{l+s+s2}{\PYGZdq{}}\PYG{l+s+s2}{\PYGZhy{}bgc\PYGZus{}spinup on}\PYG{l+s+s2}{\PYGZdq{}} \PYG{o}{\PYGZhy{}}\PYG{n}{append}
\PYG{c+c1}{\PYGZsh{} The following sets CLM\PYGZus{}FORCE\PYGZus{}COLDSTART to \PYGZdq{}on\PYGZdq{}, and run\PYGZhy{}type to startup (you could also use an editor)}
\PYG{o}{\PYGZgt{}} \PYG{o}{.}\PYG{o}{/}\PYG{n}{xmlchange} \PYG{n}{CLM\PYGZus{}FORCE\PYGZus{}COLDSTART}\PYG{o}{=}\PYG{n}{on}\PYG{p}{,}\PYG{n}{RUN\PYGZus{}TYPE}\PYG{o}{=}\PYG{n}{startup}
\PYG{c+c1}{\PYGZsh{} Make the output history files only annual, by adding the following to the user\PYGZus{}nl\PYGZus{}clm namelist}
\PYG{o}{\PYGZgt{}} \PYG{n}{echo} \PYG{l+s+s1}{\PYGZsq{}}\PYG{l+s+s1}{hist\PYGZus{}nhtfrq = \PYGZhy{}8760}\PYG{l+s+s1}{\PYGZsq{}} \PYG{o}{\PYGZgt{}\PYGZgt{}} \PYG{n}{user\PYGZus{}nl\PYGZus{}clm}
\PYG{c+c1}{\PYGZsh{} Now setup}
\PYG{o}{\PYGZgt{}} \PYG{o}{.}\PYG{o}{/}\PYG{n}{cesm\PYGZus{}setup} \PYG{o}{\PYGZhy{}}\PYG{n}{case}
\PYG{c+c1}{\PYGZsh{} Now build}
\PYG{o}{\PYGZgt{}} \PYG{o}{.}\PYG{o}{/}\PYG{n}{BGC\PYGZus{}spinup}\PYG{o}{.}\PYG{n}{build}
\PYG{c+c1}{\PYGZsh{} The following sets RESUBMIT to 30 times in env\PYGZus{}run.xml (you could also use an editor)}
\PYG{c+c1}{\PYGZsh{} The following sets STOP\PYGZus{}DATE,STOP\PYGZus{}N and STOP\PYGZus{}OPTION to Jan/1/1001, 20, \PYGZdq{}nyears\PYGZdq{} in env\PYGZus{}run.xml (you could also use an       editor)}
\PYG{o}{\PYGZgt{}} \PYG{o}{.}\PYG{o}{/}\PYG{n}{xmlchange} \PYG{n}{RESUBMIT}\PYG{o}{=}\PYG{l+m+mi}{20}\PYG{p}{,}\PYG{n}{STOP\PYGZus{}N}\PYG{o}{=}\PYG{l+m+mi}{50}\PYG{p}{,}\PYG{n}{STOP\PYGZus{}OPTION}\PYG{o}{=}\PYG{n}{nyears}\PYG{p}{,}\PYG{n}{STOP\PYGZus{}DATE}\PYG{o}{=}\PYG{l+m+mi}{10010101}
\PYG{c+c1}{\PYGZsh{} Now run normally}
\PYG{o}{\PYGZgt{}} \PYG{o}{.}\PYG{o}{/}\PYG{n}{BGC\PYGZus{}spinup}\PYG{o}{.}\PYG{n}{submit}
\end{sphinxVerbatim}

\begin{sphinxadmonition}{note}{Note:}
This same procedure works for CLM4.5-CN as well, you can typically shorten the spinup time from 1000 years to 600 though.
\end{sphinxadmonition}

Afterwards save the last restart file from this simulation to use in the next step.
\begin{description}
\item[{\sphinxstylestrong{2. Final spinup for CLM4.5-BGC}}] \leavevmode
Next save the last restart file from this step and use it as the “finidat” file to use for one more spinup for at least 200+ years in normal mode. So do the following:

\end{description}


\subsubsection{Example: Final CLMBGC Spinup Simulation for CLM4.5-BGC}
\label{\detokenize{users_guide/running-special-cases/CLM-4.4-Spinning-up-the-CLM4.5-biogeochemistry-(CLMBGC-spinup):example-final-clmbgc-spinup-simulation-for-clm4-5-bgc}}
\begin{sphinxVerbatim}[commandchars=\\\{\}]
\PYGZgt{} cd scripts
\PYGZgt{} ./create\PYGZus{}newcase \PYGZhy{}case BGC\PYGZus{}finalspinup \PYGZhy{}res f19\PYGZus{}g16 \PYGZhy{}compset I1850CRUCLM45BGC \PYGZhy{}mach yellowstone\PYGZus{}intel
\PYGZgt{} cd BGC\PYGZus{}finalspinup
\PYGZsh{} Now, Copy the last CLM restart file from the earlier case into your run directory
\PYGZgt{} cp /ptmp/\PYGZdl{}LOGIN/archive/BGC\PYGZus{}spinup/rest/BGC\PYGZus{}spinup.clm*.r*.1002\PYGZhy{}01\PYGZhy{}01\PYGZhy{}00000.nc \PYGZbs{}
/glade/scratch/\PYGZdl{}LOGIN/CN\PYGZus{}finalspinup
\PYGZsh{} Set the runtype to startup
\PYGZgt{} ./xmlchange RUN\PYGZus{}TYPE=startup
\PYGZsh{} And copy the rpointer files for datm and drv from the earlier case
\PYGZgt{} cp /glade/scratch/\PYGZdl{}LOGIN/archive/BGC\PYGZus{}spinup/rest/rpointer.atm /ptmp/\PYGZdl{}LOGIN/CN\PYGZus{}finalspinup
\PYGZgt{} cp /glade/scratch/\PYGZdl{}LOGIN/archive/BGC\PYGZus{}spinup/rest/rpointer.drv /ptmp/\PYGZdl{}LOGIN/CN\PYGZus{}finalspinup
\PYGZsh{} Set the finidat file to the last restart file saved in previous step
\PYGZgt{} echo \PYGZsq{} finidat = \PYGZdq{}BGC\PYGZus{}spinup.clm2.r.1002\PYGZhy{}01\PYGZhy{}01\PYGZhy{}00000.nc\PYGZdq{}\PYGZsq{} \PYGZgt{} user\PYGZus{}nl\PYGZus{}clm
\PYGZsh{} Now setup
\PYGZgt{} ./cesm\PYGZus{}setup
\PYGZgt{} Now build
\PYGZgt{} ./BGC\PYGZus{}finalspinup.build
\PYGZsh{} The following sets RESUBMIT to 4 times in env\PYGZus{}run.xml (you could also use an editor)
\PYGZsh{} The following sets STOP\PYGZus{}N and STOP\PYGZus{}OPTION to 50 and \PYGZdq{}nyears\PYGZdq{} in env\PYGZus{}run.xml (you could also use an editor)
\PYGZgt{} ./xmlchange RESUBMIT=4,STOP\PYGZus{}OPTION=nyears,STOP\PYGZus{}N=50
\PYGZgt{} Now run as normal
\PYGZgt{} ./BGC\PYGZus{}finalspinup.submit
\end{sphinxVerbatim}

To assess if the model is spunup plot trends of CLMBGC variables of interest. If you see a trend, you may need to run the simulation longer. Finally save the restart file from the end of this simulation to use as an “finidat” file for future simulations.

\begin{sphinxadmonition}{note}{Note:}
This same final spinup procedure works for CLM4.5-CN as well, you can typically shorten the spinup time from 200 years to 50 though.
\end{sphinxadmonition}


\subsection{Running with MOAR data}
\label{\detokenize{users_guide/running-special-cases/CLM-4.7-Running-with-MOAR-data-as-atmospheric-forcing-to-spinup-the-model:running-with-moar-data}}\label{\detokenize{users_guide/running-special-cases/CLM-4.7-Running-with-MOAR-data-as-atmospheric-forcing-to-spinup-the-model::doc}}\label{\detokenize{users_guide/running-special-cases/CLM-4.7-Running-with-MOAR-data-as-atmospheric-forcing-to-spinup-the-model:id1}}
Because it takes so long to spinup the CN model (as we just saw previously), if you are doing fully coupled simulations with active atmosphere and ocean, you will want to do the spinup portion of this “offline”.
So instead of doing expensive fully coupled simulations for the spinup duration, you run CLM in a very cheap “I” compset using atmospheric forcing from a shorter fully coupled simulation (or a simulation run previously by someone else).

In this example we will use the \sphinxcode{I1850SPINUPCN compset} to setup CLM to run with atmospheric forcing from a previous fully coupled simulation with data that is already stored on disk on yellowstone.
There are several simulations that have high frequency data for which we can do this. You can also do this on a machine other than yellowstone, but would need to download the data from the Earth System Grid and change the datapath similar to \sphinxhref{CLM-URL}{Example 4-11}.


\subsubsection{Example: Simulation with MOAR Data on yellowstone}
\label{\detokenize{users_guide/running-special-cases/CLM-4.7-Running-with-MOAR-data-as-atmospheric-forcing-to-spinup-the-model:example-simulation-with-moar-data-on-yellowstone}}
\begin{sphinxVerbatim}[commandchars=\\\{\}]
\PYG{o}{\PYGZgt{}} \PYG{n}{cd} \PYG{n}{scripts}
\PYG{o}{\PYGZgt{}} \PYG{o}{.}\PYG{o}{/}\PYG{n}{create\PYGZus{}newcase} \PYG{o}{\PYGZhy{}}\PYG{n}{case} \PYG{n}{MOARforce1850} \PYG{o}{\PYGZhy{}}\PYG{n}{res} \PYG{n}{f19\PYGZus{}g16} \PYG{o}{\PYGZhy{}}\PYG{n}{compset} \PYG{n}{I1850SPINUPCN} \PYG{o}{\PYGZhy{}}\PYG{n}{mach} \PYG{n}{yellowstone\PYGZus{}intel}
\PYG{o}{\PYGZgt{}} \PYG{n}{cd} \PYG{n}{MOARforce1850}
\PYG{c+c1}{\PYGZsh{} The following sets the casename to point to for atm forcing (you could also use an editor)}
\PYG{o}{\PYGZgt{}} \PYG{o}{.}\PYG{o}{/}\PYG{n}{xmlchange} \PYG{n}{DATM\PYGZus{}CPL\PYGZus{}CASE}\PYG{o}{=}\PYG{n}{b40}\PYG{o}{.}\PYG{l+m+mf}{1850.}\PYG{n}{track1}\PYG{o}{.}\PYG{l+m+mi}{1}\PYG{n}{deg}\PYG{o}{.}\PYG{l+m+mi}{006}\PYG{n}{a}
\PYG{c+c1}{\PYGZsh{} The following sets the align year and years to run over for atm forcing}
\PYG{c+c1}{\PYGZsh{}  (you could also use an editor)}
\PYG{o}{\PYGZgt{}} \PYG{o}{.}\PYG{o}{/}\PYG{n}{xmlchange} \PYG{n}{DATM\PYGZus{}CPL\PYGZus{}YR\PYGZus{}ALIGN}\PYG{o}{=}\PYG{l+m+mi}{1}\PYG{p}{,}\PYG{n}{DATM\PYGZus{}CPL\PYGZus{}YR\PYGZus{}START}\PYG{o}{=}\PYG{l+m+mi}{960}\PYG{p}{,}\PYG{n}{DATM\PYGZus{}CPL\PYGZus{}YR\PYGZus{}END}\PYG{o}{=}\PYG{l+m+mi}{1030}
\PYG{o}{\PYGZgt{}} \PYG{o}{.}\PYG{o}{/}\PYG{n}{cesm\PYGZus{}setup}
\PYG{c+c1}{\PYGZsh{} Now build and run as normal}
\PYG{o}{\PYGZgt{}} \PYG{o}{.}\PYG{o}{/}\PYG{n}{MOARforce1850}\PYG{o}{.}\PYG{n}{build}
\PYG{o}{\PYGZgt{}} \PYG{o}{.}\PYG{o}{/}\PYG{n}{MOARforce1850}\PYG{o}{.}\PYG{n}{submit}
\end{sphinxVerbatim}


\subsection{Running with atmospheric forcing from a previous simulation}
\label{\detokenize{users_guide/running-special-cases/CLM-4.8-Running-with-your-own-previous-simulation-as-atmospheric-forcing-to-spinup-the-model:running-with-atmospheric-forcing-from-a-previous-simulation}}\label{\detokenize{users_guide/running-special-cases/CLM-4.8-Running-with-your-own-previous-simulation-as-atmospheric-forcing-to-spinup-the-model:running-with-previous-simulation-forcing}}\label{\detokenize{users_guide/running-special-cases/CLM-4.8-Running-with-your-own-previous-simulation-as-atmospheric-forcing-to-spinup-the-model::doc}}
Another way that you might want to spinup the model is to run your own simulation for a relatively short period (either a B, E, or F compset) and then use it as forcing for your “I” case later.
By only running 20 to 50 years for the fully coupled case, you’ll save a substantial amount of computer time rather than running the entire spinup period with a fully coupled model.

The first thing we need to do is to run a fully coupled case and save the atmospheric coupling fields on a three hourly basis. In this example, we will run on yellowstone and archive the data to a local disk that we can then use in the next simulation.


\subsubsection{Example: Fully Coupled Simulation to Create Data to Force Next Example Simulation}
\label{\detokenize{users_guide/running-special-cases/CLM-4.8-Running-with-your-own-previous-simulation-as-atmospheric-forcing-to-spinup-the-model:example-fully-coupled-simulation-to-create-data-to-force-next-example-simulation}}
\begin{sphinxVerbatim}[commandchars=\\\{\}]
\PYGZgt{} cd scripts
\PYGZgt{} ./create\PYGZus{}newcase \PYGZhy{}case myBCN1850 \PYGZhy{}res f09\PYGZus{}g16 \PYGZhy{}compset B1850CN  \PYGZhy{}mach yellowstone\PYGZus{}intel
\PYGZgt{} cd myBCN1850
\PYGZgt{} ./cesm\PYGZus{}setup
\PYGZsh{} Set histaux\PYGZus{}a2x3hr to .true. in your user\PYGZus{}nl\PYGZus{}cpl output from the atmosphere model
\PYGZsh{} will be saved 3 hourly
echo \PYGZdq{}histaux\PYGZus{}a2x3hr=.true.\PYGZdq{} \PYGZgt{}\PYGZgt{} user\PYGZus{}nl\PYGZus{}cpl
\PYGZsh{} edit the driver code in order to save the correct list of fields (see note below)
\PYGZgt{} cp ../../models/drv/driver/ccsm\PYGZus{}comp\PYGZus{}mod.F90 SourceMods/src.cpl
\PYGZgt{} \PYGZdl{}EDITOR SourceMods/src.cpl
\PYGZsh{} Now build
\PYGZgt{} ./myBCN1850.build
\PYGZsh{} The following sets the archival disk space (you could also use an editor)
\PYGZgt{} ./xmlchange DOUT\PYGZus{}S\PYGZus{}ROOT=\PYGZsq{}/glade/home/\PYGZdl{}USER/\PYGZdl{}CASE\PYGZsq{}
\PYGZsh{} Make sure files are archived to disk, but NOT to long term storage
\PYGZsh{} (you could also use an editor)
\PYGZgt{} ./xmlchange DOUT\PYGZus{}S=TRUE,DOUT\PYGZus{}L\PYGZus{}MS=FALSE
\PYGZsh{} Set the run length to run a total of 20 years (you could also use an editor)
\PYGZgt{} ./xmlchange RESUBMIT=9,STOP\PYGZus{}OPTION=nyears,STOP\PYGZus{}N=2
\PYGZsh{} Now run as normal
\PYGZgt{} ./myBCN1850.submit
\end{sphinxVerbatim}

\begin{sphinxadmonition}{warning}{Warning:}
Because of bug 1733 (see the \sphinxhref{CLM-URL}{models/lnd/clm/doc/KnownBugs} file on this) you’ll need to edit the driver code in order for it to produce the correct list of fields needed to run the model later.
\end{sphinxadmonition}

Now we run an I compset forced with the data from the previous simulation using the \sphinxcode{CPLHIST3HrWx} option to DATM\_MODE. See \sphinxhref{CLM-URL}{the Section called CPLHIST3HrWx mode and it’s DATM settings in Chapter 1} for more information on the DATM settings for \sphinxcode{CPLHIST3HrWx} mode.


\subsubsection{Example: Simulation Forced with Data from the Previous Simulation}
\label{\detokenize{users_guide/running-special-cases/CLM-4.8-Running-with-your-own-previous-simulation-as-atmospheric-forcing-to-spinup-the-model:example-simulation-forced-with-data-from-the-previous-simulation}}
\begin{sphinxVerbatim}[commandchars=\\\{\}]
\PYGZgt{} cd scripts
\PYGZgt{} ./create\PYGZus{}newcase \PYGZhy{}case frcwmyBCN1850 \PYGZhy{}res f09\PYGZus{}g16 \PYGZhy{}compset I1850SPINUPCN \PYGZhy{}mach yellowstone\PYGZus{}intel
\PYGZgt{} cd frcWmyBCN1850
\PYGZsh{} The following sets the casename to point to for atm forcing (you could also use an editor)
\PYGZgt{} ./xmlchange DATM\PYGZus{}CPLHIST\PYGZus{}CASE=\PYGZdq{}myBCN1850\PYGZdq{}
\PYGZsh{} The following sets the align year and years to run over for atm forcing
\PYGZsh{}  (you could also use an editor)
\PYGZgt{} ./xmlchange DATM\PYGZus{}CPLHIST\PYGZus{}YR\PYGZus{}ALIGN=\PYGZdq{}1\PYGZdq{},DATM\PYGZus{}CPLHIST\PYGZus{}YR\PYGZus{}START=1,DATM\PYGZus{}CPLHIST\PYGZus{}YR\PYGZus{}END=20
\PYGZsh{} Set the strm\PYGZus{}datdir in the namelist\PYGZus{}defaults\PYGZus{}datm.xml
\PYGZsh{} file to the archival path of the case above in the form of: /glade/home/achive/\PYGZdl{}USER/\PYGZdl{}DATM\PYGZus{}CPLHIST\PYGZus{}CASE/cpl/hist
\PYGZsh{} NOTE: THIS WILL CHANGE THE PATH FOR ALL I1850SPINUPCN COMPSET CASES MADE AFTER THIS!
\PYGZgt{} \PYGZdl{}EDITOR ../../models/atm/datm/bld/namelist\PYGZus{}files/namelist\PYGZus{}defaults\PYGZus{}datm.xml
\PYGZgt{} ./cesm\PYGZus{}setup
\PYGZsh{} Now build and run as normal
\PYGZgt{} ./frcwmyBCN1850.build
\PYGZgt{} ./frcwmyBCN1850.submit
\end{sphinxVerbatim}

\begin{sphinxadmonition}{note}{Note:}
We did this by editing the “namelist\_defaults\_datm.xml” which will change the settings for ALL future \sphinxcode{I1850SPINUPCN} cases you run. You could also do this by editing the path in the resulting streams text files in the CaseDocs directory, and then create a “user\_” streams file with the correct path. This would change the streams file JUST for this case. The steps do it this way are:
\end{sphinxadmonition}

\begin{sphinxVerbatim}[commandchars=\\\{\}]
\PYGZgt{} ./preview\PYGZus{}namelists
\PYGZgt{} cp CaseDocs/datm.streams.txt.CPLHIST3HrWx.Precip            user\PYGZus{}datm.streams.txt.CPLHIST3HrWx.Precip
\PYGZgt{} cp CaseDocs/datm.streams.txt.CPLHIST3HrWx.Solar             user\PYGZus{}datm.streams.txt.CPLHIST3HrWx.Solar
\PYGZgt{} cp CaseDocs/datm.streams.txt.CPLHIST3HrWx.nonSolarNonPrecip user\PYGZus{}datm.streams.txt.CPLHIST3HrWx.nonSolarNonPrecip
\PYGZsh{} Change the \PYGZlt{}fieldInfo\PYGZgt{} field \PYGZlt{}filePath\PYGZgt{} to point to the correct directory i.e.: /glade/home/achive/\PYGZdl{}USER/\PYGZdl{}DATM\PYGZus{}CPLHIST\PYGZus{}CASE/cpl/hist
\PYGZgt{} \PYGZdl{}EDITOR user\PYGZus{}datm.streams.txt.CPLHIST3HrWx.*
\PYGZgt{} ./preview\PYGZus{}namelists
\PYGZsh{} Then make sure the CaseDocs/datm.streams.txt.CPLHIST3HrWx.* files have the correct path
\end{sphinxVerbatim}


\subsection{Running with historical CO2 forcing}
\label{\detokenize{users_guide/running-special-cases/CLM-4.9-Running-stand-alone-CLM-with-transient-historical-CO2-concentration:running-with-historical-co2-forcing}}\label{\detokenize{users_guide/running-special-cases/CLM-4.9-Running-stand-alone-CLM-with-transient-historical-CO2-concentration::doc}}\label{\detokenize{users_guide/running-special-cases/CLM-4.9-Running-stand-alone-CLM-with-transient-historical-CO2-concentration:id1}}
In this case you want to run a simulation with stand-alone CLM responding to changes in CO2 for a historical period.
For this example, we will start with the “I\_1850-2000\_CN” compset that has transient: land-use, Nitrogen and Aerosol deposition already.
You could also use another compset if you didn’t want these other features to be transient.
In order to get CO2 to be transient we need to add a new streams file and add it to the list of streams in the user\_nl\_datm file.
You also need a NetCDF datafile that datm can read that gives the variation. You could supply your own file, but we have a standard file that is used by CAM for this and our example will make use of this file.

\begin{sphinxadmonition}{note}{Note:}
Most everything here has to do with changing datm rather than CLM to allow this to happen. As such the user that wishes to do this should first become more familiar with datm and read the \sphinxhref{CLM-URL}{CESM Data Model User’s Guide} especially as it pertains to the datm.
\end{sphinxadmonition}

\begin{sphinxadmonition}{warning}{Warning:}
This section documents the process for doing something that is non-standard. There may be errors with the documentation and process, and you may have to do some work before all of this works for you. If that is the case, we recommend that you do further research into understanding the process and the files, as well as understanding the datm and how it works. You may have to read documentation found in the code for datm as well as “csm\_share”.
\end{sphinxadmonition}

The datm has “streams” files that have rough XML-like syntax and specify the location and file to get data from, as well as information on the variable names and the data locations of the grid points.
The datm expects specific variable names and the datm “maps” the expected variable names from the file to the names expected by datm.
The file we are working with here is a file with a single-point, that covers the entire globe (so the vertices go from -90 to 90 degrees in latitude and 0 to 360 degrees in longitude).
Since it’s a single point it’s a little easier to work with than datasets that may be at a given horizontal resolution.
The datm also expects that variables will be in certain units, and only expects a limited number of variables so arbitrary fields can NOT be exchanged this way.
However, the process would be similar for datasets that do contain more than one point.

The three things that are needed: a domain file, a data file, and a streams text file.
The domain file is a CF-compliant NetCDF file that has information on the grid points (latitudes and longitudes for cell-centers and vertices, mask , fraction, and areas).
The datafile is a CF-compliant NetCDF file with the data that will be mapped.
The streams text file is the XML-like file that tells datm how to find the files and how to map the variables datm knows about to the variable names on the NetCDF files. Note, that in our case the domain file and the data file are the same file. In other cases, the domain file may be separate from the data file.

First we are going to create a case, and we will edit the \sphinxcode{user\_nl\_datm} so that we add a CO2 data stream in.
There is a streams text file available in \sphinxcode{models/lnd/clm/doc/UsersGuide/co2\_streams.txt}, that includes file with a CO2 time-series from 1765 to 2007.


\subsubsection{Example: Transient Simulation with Historical CO2}
\label{\detokenize{users_guide/running-special-cases/CLM-4.9-Running-stand-alone-CLM-with-transient-historical-CO2-concentration:example-transient-simulation-with-historical-co2}}
\begin{sphinxVerbatim}[commandchars=\\\{\}]
\PYGZgt{} cd scripts
\PYGZgt{} ./create\PYGZus{}newcase \PYGZhy{}case DATM\PYGZus{}CO2\PYGZus{}TSERIES \PYGZhy{}res f19\PYGZus{}g16 \PYGZhy{}compset I20TRCRUCLM45BGC
\PYGZgt{} cd DATM\PYGZus{}CO2\PYGZus{}TSERIES

\PYGZsh{} Set CCSM\PYGZus{}BGC to CO2A so that CO2 will be passed from atmosphere to land
\PYGZsh{} Set CLM\PYGZus{}CO2\PYGZus{}TYPE to diagnostic so that the land will use the value sent from the atmosphere
\PYGZgt{} ./xmlchange CCSM\PYGZus{}BGC=CO2A,CLM\PYGZus{}CO2\PYGZus{}TYPE=diagnostic
\PYGZgt{} ./case.setup

\PYGZsh{} Create the streams file for CO2
\PYGZgt{} cat \PYGZlt{}\PYGZlt{} EOF \PYGZgt{}\PYGZgt{} datm.streams.txt.co2tseries

\PYGZlt{}streamstemplate\PYGZgt{}
   \PYGZlt{}general\PYGZus{}comment\PYGZgt{}
      This is a streams file to pass historical CO2 from datm8 to the other
      surface models. It reads in a historical dataset derived from data used
      by CAM. The getco2\PYGZus{}historical.ncl script in models/lnd/clm2/tools/ncl\PYGZus{}scripts
      was used to convert the CAM file to a streams compatible format (adding domain
      information and making CO2 have latitude/longitude even if only for a single
      point.
   \PYGZlt{}/general\PYGZus{}comment\PYGZgt{}
\PYGZlt{}stream\PYGZgt{}
   \PYGZlt{}comment\PYGZgt{}
     Input stream description file for historical CO2 reconstruction data

     04 March 2010: Converted to form that can be used by datm8 by Erik Kluzek
     18 December 2009: Prepared by B. Eaton using data provided by
     Jean\PYGZhy{}Francois Lamarque. All variables except f11 are directly from
     PRE2005\PYGZus{}MIDYR\PYGZus{}CONC.DAT. Data from 1765 to 2007 with 2006/2007 just
     a repeat of 2005.
   \PYGZlt{}/comment\PYGZgt{}
   \PYGZlt{}dataSource\PYGZgt{}
      CLMNCEP
   \PYGZlt{}/dataSource\PYGZgt{}
   \PYGZlt{}domainInfo\PYGZgt{}
      \PYGZlt{}variableNames\PYGZgt{}
         time    time
         lonc    lon
         latc    lat
         area    area
         mask    mask
      \PYGZlt{}/variableNames\PYGZgt{}
      \PYGZlt{}filePath\PYGZgt{}
         \PYGZdl{}CSMDATA/atm/datm7/CO2
      \PYGZlt{}/filePath\PYGZgt{}
      \PYGZlt{}fileNames\PYGZgt{}
         fco2\PYGZus{}datm\PYGZus{}1765\PYGZhy{}2007\PYGZus{}c100614.nc
      \PYGZlt{}/fileNames\PYGZgt{}
   \PYGZlt{}/domainInfo\PYGZgt{}
   \PYGZlt{}fieldInfo\PYGZgt{}
      \PYGZlt{}variableNames\PYGZgt{}
         CO2        co2diag
      \PYGZlt{}/variableNames\PYGZgt{}
      \PYGZlt{}filePath\PYGZgt{}
         \PYGZdl{}CSMDATA/atm/datm7/CO2
      \PYGZlt{}/filePath\PYGZgt{}
      \PYGZlt{}fileNames\PYGZgt{}
         fco2\PYGZus{}datm\PYGZus{}1765\PYGZhy{}2007\PYGZus{}c100614.nc
      \PYGZlt{}/fileNames\PYGZgt{}
   \PYGZlt{}/fieldInfo\PYGZgt{}
\PYGZlt{}/stream\PYGZgt{}
\PYGZlt{}/streamstemplate\PYGZgt{}

EOF

\PYGZsh{} And copy it to the run directory
\PYGZgt{} cp datm.streams.txt.co2tseries \PYGZdl{}RUNDIR

\PYGZsh{} Run preview namelist so we have the namelist in CaseDocs
\PYGZgt{} ./preview\PYGZus{}namelists
\end{sphinxVerbatim}

The first thing we will do is to edit the \sphinxcode{user\_nl\_datm} file to add a CO2 file stream in.
To do this we will copy a \sphinxcode{user\_nl\_datm} in with the changes needed. The file \sphinxcode{addco2\_user\_nl\_datm.user\_nl} is in \sphinxcode{models/lnd/clm/doc/UsersGuide} and looks like this…

\begin{sphinxVerbatim}[commandchars=\\\{\}]
\PYG{n}{dtlimit} \PYG{o}{=} \PYG{l+m+mf}{1.5}\PYG{p}{,}\PYG{l+m+mf}{1.5}\PYG{p}{,}\PYG{l+m+mf}{1.5}\PYG{p}{,}\PYG{l+m+mf}{1.5}\PYG{p}{,}\PYG{l+m+mf}{1.5}
\PYG{n}{fillalgo} \PYG{o}{=} \PYG{l+s+s1}{\PYGZsq{}}\PYG{l+s+s1}{nn}\PYG{l+s+s1}{\PYGZsq{}}\PYG{p}{,}\PYG{l+s+s1}{\PYGZsq{}}\PYG{l+s+s1}{nn}\PYG{l+s+s1}{\PYGZsq{}}\PYG{p}{,}\PYG{l+s+s1}{\PYGZsq{}}\PYG{l+s+s1}{nn}\PYG{l+s+s1}{\PYGZsq{}}\PYG{p}{,}\PYG{l+s+s1}{\PYGZsq{}}\PYG{l+s+s1}{nn}\PYG{l+s+s1}{\PYGZsq{}}\PYG{p}{,}\PYG{l+s+s1}{\PYGZsq{}}\PYG{l+s+s1}{nn}\PYG{l+s+s1}{\PYGZsq{}}
\PYG{n}{fillmask} \PYG{o}{=} \PYG{l+s+s1}{\PYGZsq{}}\PYG{l+s+s1}{nomask}\PYG{l+s+s1}{\PYGZsq{}}\PYG{p}{,}\PYG{l+s+s1}{\PYGZsq{}}\PYG{l+s+s1}{nomask}\PYG{l+s+s1}{\PYGZsq{}}\PYG{p}{,}\PYG{l+s+s1}{\PYGZsq{}}\PYG{l+s+s1}{nomask}\PYG{l+s+s1}{\PYGZsq{}}\PYG{p}{,}\PYG{l+s+s1}{\PYGZsq{}}\PYG{l+s+s1}{nomask}\PYG{l+s+s1}{\PYGZsq{}}\PYG{p}{,}\PYG{l+s+s1}{\PYGZsq{}}\PYG{l+s+s1}{nomask}\PYG{l+s+s1}{\PYGZsq{}}
\PYG{n}{mapalgo} \PYG{o}{=} \PYG{l+s+s1}{\PYGZsq{}}\PYG{l+s+s1}{bilinear}\PYG{l+s+s1}{\PYGZsq{}}\PYG{p}{,}\PYG{l+s+s1}{\PYGZsq{}}\PYG{l+s+s1}{bilinear}\PYG{l+s+s1}{\PYGZsq{}}\PYG{p}{,}\PYG{l+s+s1}{\PYGZsq{}}\PYG{l+s+s1}{bilinear}\PYG{l+s+s1}{\PYGZsq{}}\PYG{p}{,}\PYG{l+s+s1}{\PYGZsq{}}\PYG{l+s+s1}{bilinear}\PYG{l+s+s1}{\PYGZsq{}}\PYG{p}{,}\PYG{l+s+s1}{\PYGZsq{}}\PYG{l+s+s1}{nn}\PYG{l+s+s1}{\PYGZsq{}}
\PYG{n}{mapmask} \PYG{o}{=} \PYG{l+s+s1}{\PYGZsq{}}\PYG{l+s+s1}{nomask}\PYG{l+s+s1}{\PYGZsq{}}\PYG{p}{,}\PYG{l+s+s1}{\PYGZsq{}}\PYG{l+s+s1}{nomask}\PYG{l+s+s1}{\PYGZsq{}}\PYG{p}{,}\PYG{l+s+s1}{\PYGZsq{}}\PYG{l+s+s1}{nomask}\PYG{l+s+s1}{\PYGZsq{}}\PYG{p}{,}\PYG{l+s+s1}{\PYGZsq{}}\PYG{l+s+s1}{nomask}\PYG{l+s+s1}{\PYGZsq{}}\PYG{p}{,}\PYG{n}{nomask}\PYG{l+s+s1}{\PYGZsq{}}
\PYG{n}{streams} \PYG{o}{=} \PYG{l+s+s2}{\PYGZdq{}}\PYG{l+s+s2}{datm.streams.txt.CLM\PYGZus{}QIAN.Solar 1895 1948 1972  }\PYG{l+s+s2}{\PYGZdq{}}\PYG{p}{,} \PYG{l+s+s2}{\PYGZdq{}}\PYG{l+s+s2}{datm.streams.txt.CLM\PYGZus{}QIAN.Precip 1895 1948 1972  }\PYG{l+s+s2}{\PYGZdq{}}\PYG{p}{,}
          \PYG{l+s+s2}{\PYGZdq{}}\PYG{l+s+s2}{datm.streams.txt.CLM\PYGZus{}QIAN.TPQW 1895 1948 1972  }\PYG{l+s+s2}{\PYGZdq{}}\PYG{p}{,} \PYG{l+s+s2}{\PYGZdq{}}\PYG{l+s+s2}{datm.streams.txt.presaero.trans\PYGZus{}1850\PYGZhy{}2000 1849 1849 2006}\PYG{l+s+s2}{\PYGZdq{}}\PYG{p}{,}
          \PYG{l+s+s2}{\PYGZdq{}}\PYG{l+s+s2}{datm.streams.txt.co2tseries 1766 1766 2005 }\PYG{l+s+s2}{\PYGZdq{}}
\PYG{n}{taxmode} \PYG{o}{=} \PYG{l+s+s1}{\PYGZsq{}}\PYG{l+s+s1}{cycle}\PYG{l+s+s1}{\PYGZsq{}}\PYG{p}{,}\PYG{l+s+s1}{\PYGZsq{}}\PYG{l+s+s1}{cycle}\PYG{l+s+s1}{\PYGZsq{}}\PYG{p}{,}\PYG{l+s+s1}{\PYGZsq{}}\PYG{l+s+s1}{cycle}\PYG{l+s+s1}{\PYGZsq{}}\PYG{p}{,}\PYG{l+s+s1}{\PYGZsq{}}\PYG{l+s+s1}{cycle}\PYG{l+s+s1}{\PYGZsq{}}\PYG{p}{,}\PYG{l+s+s1}{\PYGZsq{}}\PYG{l+s+s1}{extend}\PYG{l+s+s1}{\PYGZsq{}}
\PYG{n}{tintalgo} \PYG{o}{=} \PYG{l+s+s1}{\PYGZsq{}}\PYG{l+s+s1}{coszen}\PYG{l+s+s1}{\PYGZsq{}}\PYG{p}{,}\PYG{l+s+s1}{\PYGZsq{}}\PYG{l+s+s1}{nearest}\PYG{l+s+s1}{\PYGZsq{}}\PYG{p}{,}\PYG{l+s+s1}{\PYGZsq{}}\PYG{l+s+s1}{linear}\PYG{l+s+s1}{\PYGZsq{}}\PYG{p}{,}\PYG{l+s+s1}{\PYGZsq{}}\PYG{l+s+s1}{linear}\PYG{l+s+s1}{\PYGZsq{}}\PYG{p}{,}\PYG{l+s+s1}{\PYGZsq{}}\PYG{l+s+s1}{linear}\PYG{l+s+s1}{\PYGZsq{}}
\end{sphinxVerbatim}

You just copy this into your case directory. But, also compare it to the version in \sphinxcode{CaseDocs} to make sure the changes are just to add in the new CO2 stream. Check to see that filenames, and start, end and align years are correct.

\begin{sphinxVerbatim}[commandchars=\\\{\}]
\PYG{o}{\PYGZgt{}} \PYG{n}{cp} \PYG{o}{.}\PYG{o}{.}\PYG{o}{/}\PYG{o}{.}\PYG{o}{.}\PYG{o}{/}\PYG{n}{models}\PYG{o}{/}\PYG{n}{lnd}\PYG{o}{/}\PYG{n}{clm}\PYG{o}{/}\PYG{n}{doc}\PYG{o}{/}\PYG{n}{UsersGuide}\PYG{o}{/}\PYG{n}{addco2\PYGZus{}user\PYGZus{}nl\PYGZus{}datm}\PYG{o}{.}\PYG{n}{user\PYGZus{}nl} \PYG{n}{user\PYGZus{}nl\PYGZus{}datm}
\PYG{o}{\PYGZgt{}} \PYG{n}{diff} \PYG{n}{user\PYGZus{}nl\PYGZus{}datm} \PYG{n}{CaseDocs}\PYG{o}{/}\PYG{n}{datm\PYGZus{}atm\PYGZus{}in}
\end{sphinxVerbatim}

Once, you’ve done that you can build and run your case normally.

\begin{sphinxadmonition}{warning}{Warning:}
This procedure assumes you are using a \sphinxcode{I20TRCRUCLM45BGC} compset out of the box, with \sphinxcode{DATM\_PRESAERO} equal to trans\_1850-2000. So it assumes standard CLM4.5 CRUNCEP atmosphere forcing, and transient prescribed aerosols from streams files. If your case changes anything here your \sphinxcode{user\_nl\_datm} file will need to be adjusted to work with it.
\end{sphinxadmonition}

\begin{sphinxadmonition}{note}{Note:}
The intent of the \sphinxcode{user\_nl\_datm} is to add an extra streams file for CO2 to the end of the streams variable, and other arrays associated with streams (adding mapalgo as a new array with bilinear for everything, but the CO2 file which should be “nn” for nearest neighbor). Other variables should be the same as the other stream values.
\end{sphinxadmonition}

\begin{sphinxadmonition}{warning}{Warning:}
The streams file above is hard-coded for the path of the file on NCAR computers. To use it on an outside machine you’ll need to edit the filepath in the streams file to point to the location where you have the file.
\end{sphinxadmonition}

After going through these steps, you will have a case where you have datm reading in an extra streams text file that points to a data file with CO2 data on it that will send that data to the CLM.


\section{Running Single Point Regional Cases}
\label{\detokenize{users_guide/running-single-points/index:running-single-point-regional-cases}}\label{\detokenize{users_guide/running-single-points/index:running-single-points}}\label{\detokenize{users_guide/running-single-points/index::doc}}

\subsection{Single and Regional Grid Configurations}
\label{\detokenize{users_guide/running-single-points/single-point-and-regional-grid-configurations:single-point-configurations}}\label{\detokenize{users_guide/running-single-points/single-point-and-regional-grid-configurations::doc}}\label{\detokenize{users_guide/running-single-points/single-point-and-regional-grid-configurations:single-and-regional-grid-configurations}}
CLM allows you to set up and run cases with a single-point or a local region as well as global resolutions.
This is often useful for running quick cases for testing, evaluating specific vegetation types, or land-units, or running with observed data for a specific site.

There are three different ways to do this for normal-supported site
\begin{description}
\item[{\sphinxcode{PTS\_MODE}}] \leavevmode
runs for a single point using global datasets.

\item[{\sphinxcode{CLM\_USRDAT\_NAME}}] \leavevmode
runs using your own datasets (single-point or regional).

\item[{\sphinxcode{PTCLM}}] \leavevmode
easily setup simulations to run for tower sites..

\end{description}

\begin{sphinxadmonition}{note}{Note:}
\sphinxcode{PTS\_MODE} and \sphinxcode{PTCLM} only works for a single point, while the other two options can also work for regional datasets as well.
\end{sphinxadmonition}


\subsubsection{Choosing the right single point options}
\label{\detokenize{users_guide/running-single-points/single-point-and-regional-grid-configurations:choosing-the-right-single-point-options}}\label{\detokenize{users_guide/running-single-points/single-point-and-regional-grid-configurations:options-for-single-points}}
Running for a \sphinxstyleemphasis{normal supported site} is a great solution, if one of the supported single-point/regional datasets, is your region of interest (see \sphinxhref{CLM-URL}{the Section called Running Supported Single-point/Regional Datasets}).
All the datasets are created for you, and you can easily select one and run, out of the box with it using a supported resolution from the top level of the CESM scripts.
The problem is that there is a very limited set of supported datasets.
You can also use this method for your own datasets, but you have to create the datasets, and add them to the XML database in scripts, CLM and to the DATM. This is worthwhile if you want to repeat many multiple cases for a given point or region.

In general \sphinxhref{CLM-URL}{the Section called Running PTS\_MODE configurations} is the quick and dirty method that gets you started without having to create datasets \textendash{} but has limitations.
It’s good for an initial attempt at seeing results for a point of interest, but since you can NOT restart with it, it’s usage is limited.
It is the quickest method as you can create a case for it directly from \sphinxstylestrong{create\_newcase}. Although you can’t restart, running a single point is very fast, and you can run for long simulation times even without restarts.

Next, \sphinxcode{CLM\_USRDAT\_NAME} is the best way to setup cases quickly where you have to create your own datasets (see \sphinxhref{CLM-URL}{the Section called Creating your own single-point/regional surface datasets}).
With this method you don’t have to change DATM or add files to the XML database \textendash{} but you have to follow a strict naming convention for files.
However, once the files are named and in the proper location, you can easily setup new cases that use these datasets.
This is good for treating all the required datasets as a “group” and for a particular model version. For advanced CLM developers who need to track dataset changes with different model versions you would be best off adding these datasets as supported datasets with the “normal supported datasets” method.

Lastly \sphinxstyleemphasis{PTCLM} is a great way to easily create datasets, setup simulations and run simulations for tower sites.
It takes advantage of both normal supported site functionality and CLM\_USRDAT\_NAME internally.
A big advantage to it, is that it’s one-stop shopping, it runs tools to create datasets, and runs \sphinxstylestrong{create\_newcase} and sets the appropriate env variables for you. So you only have to learn how to run one tool, rather than work with many different ones. PTCLM is described in the next chapter \sphinxhref{CLM-URL}{Chapter 6}.

Finally, if you also have meteorology data that you want to force your CLM simulations with you’ll need to setup cases as described in \sphinxhref{CLM-URL}{the Section called Running with your own atmosphere forcing}.
You’ll need to create CLM datasets either according to \sphinxcode{CLM\_USRDAT\_NAME}.
You may also need to modify DATM to use your forcing data.
And you’ll need to change your forcing data to be in a format that DATM can use. In the PTCLM chapter \sphinxhref{CLM-URL}{the Section called Converting AmeriFlux Data for use by PTCLM in Chapter 6} section tells you how to use AmeriFlux data for atmospheric forcing.


\subsection{Running a single point using global data - PTS\_MODE}
\label{\detokenize{users_guide/running-single-points/running-pts_mode-configurations:running-a-single-point-using-global-data-pts-mode}}\label{\detokenize{users_guide/running-single-points/running-pts_mode-configurations:pts-mode}}\label{\detokenize{users_guide/running-single-points/running-pts_mode-configurations::doc}}
\sphinxcode{PTS\_MODE} enables you to run the model using global datasets, but just picking a single point from those datasets and operating on it.
It can be a very quick way to do fast simulations and get a quick turnaround.

To setup a \sphinxcode{PTS\_MODE} simulation you use the “-pts\_lat” and “-pts\_lon” arguments to \sphinxstylestrong{create\_newcase} to give the latitude and longitude of the point you want to simulate for (the code will pick the point on the global grid nearest to the point you give. Here’s an example to setup a simulation for the nearest point at 2-degree resolution to Boulder Colorado.

\begin{sphinxVerbatim}[commandchars=\\\{\}]
\PYG{o}{\PYGZgt{}} \PYG{n}{cd} \PYG{n}{scripts}
\PYG{o}{\PYGZgt{}} \PYG{o}{.}\PYG{o}{/}\PYG{n}{create\PYGZus{}newcase} \PYG{o}{\PYGZhy{}}\PYG{n}{case} \PYG{n}{testPTS\PYGZus{}MODE} \PYG{o}{\PYGZhy{}}\PYG{n}{res} \PYG{n}{f19\PYGZus{}g16} \PYG{o}{\PYGZhy{}}\PYG{n}{compset} \PYG{n}{I1850CRUCLM45BGC} \PYG{o}{\PYGZhy{}}\PYG{n}{pts\PYGZus{}lat} \PYG{l+m+mf}{40.0} \PYG{o}{\PYGZhy{}}\PYG{n}{pts\PYGZus{}lon} \PYG{o}{\PYGZhy{}}\PYG{l+m+mi}{105}
\PYG{o}{\PYGZgt{}} \PYG{n}{cd} \PYG{n}{testPTS\PYGZus{}MODE}

\PYG{c+c1}{\PYGZsh{} We make sure the model will start up cold rather than using initial conditions}
\PYG{o}{\PYGZgt{}} \PYG{o}{.}\PYG{o}{/}\PYG{n}{xmlchange} \PYG{n}{CLM\PYGZus{}FORCE\PYGZus{}COLDSTART}\PYG{o}{=}\PYG{n}{on}\PYG{p}{,}\PYG{n}{RUN\PYGZus{}TYPE}\PYG{o}{=}\PYG{n}{startup}
\end{sphinxVerbatim}

Then setup, build and run as normal. We make sure initial conditions are NOT used since \sphinxcode{PTS\_MODE} currently CAN NOT run with initial conditions.

\begin{sphinxadmonition}{note}{Note:}
By default it sets up to run with MPILIB=mpi-serial (in the env\_build.xml file) turned on, which allows you to run the model interactively. On some machines this mode is NOT supported and you may need to change it to FALSE before you are able to build.
\end{sphinxadmonition}

\begin{sphinxadmonition}{warning}{Warning:}
\sphinxcode{PTS\_MODE} currently does NOT restart nor is it able to startup from global initial condition files. See bugs “1017 and 1025” in the \sphinxhref{CLM-URL}{models/lnd/clm/doc/KnownLimitationss} file.
\end{sphinxadmonition}

\begin{sphinxadmonition}{note}{Note:}
You can change the point you are simulating for at run-time by changing the values of \sphinxcode{PTS\_LAT} and \sphinxcode{PTS\_LON} in the \sphinxcode{env\_run.xml} file.
\end{sphinxadmonition}


\subsubsection{Running on in a single processor}
\label{\detokenize{users_guide/running-single-points/running-pts_mode-configurations:running-on-in-a-single-processor}}
Note, that when running with \sphinxcode{PTS\_MODE} the number of processors is automatically set to one.
When running a single grid point you can only use a single processor.
You might also want to set the \sphinxcode{env\_build.xml} variable: \sphinxcode{MPILIB=mpi-serial} to \sphinxcode{TRUE} so that you can also run interactively without having to use MPI to start up your job.

On many machines, batch queues have a minimum number of nodes or processors that can be used.
On these machines you may have to change the queue and possibly the time-limits of the job, to get it to run in the batch queue.
On the NCAR machine, yellowstone, this is done for you automatically, and the “caldera” queue is used for such single-processor simulations.
Another way to get around this problem is to run the job interactively using \sphinxcode{MPILIB=mpi-serial} so that you don’t submit the job to the batch queue.
For single point mode you also may want to consider using a smaller workstation or cluster, rather than a super-computer, because you can’t take advantage of the multi-processing power of the super-computer anyway.


\subsection{Running Single Point Configurations}
\label{\detokenize{users_guide/running-single-points/running-single-point-configurations:running-single-point-configurations}}\label{\detokenize{users_guide/running-single-points/running-single-point-configurations::doc}}\label{\detokenize{users_guide/running-single-points/running-single-point-configurations:running-single-point-datasets}}
In addition to \sphinxcode{PTS\_MODE}, CLM supports running using single-point or regional datasets that are customized to a particular region.
CLM supports a a small number of out-of-the-box single-point and regional datasets.
However, users can create their own dataset.

To get the list of supported dataset resolutions do this:

\begin{sphinxVerbatim}[commandchars=\\\{\}]
\PYG{o}{\PYGZgt{}} \PYG{n}{cd} \PYG{n}{models}\PYG{o}{/}\PYG{n}{lnd}\PYG{o}{/}\PYG{n}{clm}\PYG{o}{/}\PYG{n}{doc}
\PYG{o}{\PYGZgt{}} \PYG{o}{.}\PYG{o}{.}\PYG{o}{/}\PYG{n}{bld}\PYG{o}{/}\PYG{n}{build}\PYG{o}{\PYGZhy{}}\PYG{n}{namelist} \PYG{o}{\PYGZhy{}}\PYG{n}{res} \PYG{n+nb}{list}
\end{sphinxVerbatim}

Which results in the following:

\begin{sphinxVerbatim}[commandchars=\\\{\}]
\PYG{n}{CLM} \PYG{n}{build}\PYG{o}{\PYGZhy{}}\PYG{n}{namelist} \PYG{o}{\PYGZhy{}} \PYG{n}{valid} \PYG{n}{values} \PYG{k}{for} \PYG{n}{res} \PYG{p}{(}\PYG{n}{Horizontal} \PYG{n}{resolutions}
\PYG{n}{Note}\PYG{p}{:} \PYG{l+m+mf}{0.1}\PYG{n}{x0}\PYG{o}{.}\PYG{l+m+mi}{1}\PYG{p}{,} \PYG{l+m+mf}{0.5}\PYG{n}{x0}\PYG{o}{.}\PYG{l+m+mi}{5}\PYG{p}{,} \PYG{l+m+mi}{5}\PYG{n}{x5min}\PYG{p}{,} \PYG{l+m+mi}{10}\PYG{n}{x10min}\PYG{p}{,} \PYG{l+m+mi}{3}\PYG{n}{x3min} \PYG{o+ow}{and} \PYG{l+m+mf}{0.33}\PYG{n}{x0}\PYG{o}{.}\PYG{l+m+mi}{33} \PYG{n}{are} \PYG{n}{only} \PYG{n}{used} \PYG{k}{for} \PYG{n}{CLM} \PYG{n}{tools}\PYG{p}{)}\PYG{p}{:}
      \PYG{n}{Values}\PYG{p}{:} \PYG{n}{default} \PYG{l+m+mi}{512}\PYG{n}{x1024} \PYG{l+m+mi}{360}\PYG{n}{x720cru} \PYG{l+m+mi}{128}\PYG{n}{x256} \PYG{l+m+mi}{64}\PYG{n}{x128} \PYG{l+m+mi}{48}\PYG{n}{x96} \PYG{l+m+mi}{32}\PYG{n}{x64} \PYG{l+m+mi}{8}\PYG{n}{x16} \PYG{l+m+mi}{94}\PYG{n}{x192}  \PYGZbs{}
              \PYG{l+m+mf}{0.23}\PYG{n}{x0}\PYG{o}{.}\PYG{l+m+mi}{31} \PYG{l+m+mf}{0.47}\PYG{n}{x0}\PYG{o}{.}\PYG{l+m+mi}{63} \PYG{l+m+mf}{0.9}\PYG{n}{x1}\PYG{o}{.}\PYG{l+m+mi}{25} \PYG{l+m+mf}{1.9}\PYG{n}{x2}\PYG{o}{.}\PYG{l+m+mi}{5} \PYG{l+m+mf}{2.5}\PYG{n}{x3}\PYG{o}{.}\PYG{l+m+mi}{33} \PYG{l+m+mi}{4}\PYG{n}{x5} \PYG{l+m+mi}{10}\PYG{n}{x15} \PYG{l+m+mi}{5}\PYG{n}{x5\PYGZus{}amazon} \PYG{l+m+mi}{1}\PYG{n}{x1\PYGZus{}tropicAtl} \PYG{l+m+mi}{1}\PYG{n}{x1\PYGZus{}camdenNJ}  \PYGZbs{}
              \PYG{l+m+mi}{1}\PYG{n}{x1\PYGZus{}vancouverCAN} \PYG{l+m+mi}{1}\PYG{n}{x1\PYGZus{}mexicocityMEX} \PYG{l+m+mi}{1}\PYG{n}{x1\PYGZus{}asphaltjungleNJ} \PYG{l+m+mi}{1}\PYG{n}{x1\PYGZus{}brazil} \PYG{l+m+mi}{1}\PYG{n}{x1\PYGZus{}urbanc\PYGZus{}alpha} \PYG{l+m+mi}{1}\PYG{n}{x1\PYGZus{}numaIA}  \PYGZbs{}
              \PYG{l+m+mi}{1}\PYG{n}{x1\PYGZus{}smallvilleIA} \PYG{l+m+mf}{0.1}\PYG{n}{x0}\PYG{o}{.}\PYG{l+m+mi}{1} \PYG{l+m+mf}{0.5}\PYG{n}{x0}\PYG{o}{.}\PYG{l+m+mi}{5} \PYG{l+m+mi}{3}\PYG{n}{x3min} \PYG{l+m+mi}{5}\PYG{n}{x5min} \PYG{l+m+mi}{10}\PYG{n}{x10min} \PYG{l+m+mf}{0.33}\PYG{n}{x0}\PYG{o}{.}\PYG{l+m+mi}{33} \PYG{n}{ne4np4} \PYG{n}{ne16np4} \PYG{n}{ne30np4} \PYG{n}{ne60np4}  \PYGZbs{}
              \PYG{n}{ne120np4} \PYG{n}{ne240np4} \PYG{n}{wus12} \PYG{n}{us20}
      \PYG{n}{Default} \PYG{o}{=} \PYG{l+m+mf}{1.9}\PYG{n}{x2}\PYG{o}{.}\PYG{l+m+mi}{5}
     \PYG{p}{(}\PYG{n}{NOTE}\PYG{p}{:} \PYG{n}{resolution} \PYG{o+ow}{and} \PYG{n}{mask} \PYG{o+ow}{and} \PYG{n}{other} \PYG{n}{settings} \PYG{n}{may} \PYG{n}{influence} \PYG{n}{what} \PYG{n}{the} \PYG{n}{default} \PYG{o+ow}{is}\PYG{p}{)}
\end{sphinxVerbatim}

The resolution names that have an underscore in them (“\_”) are all single-point or regional resolutions.

To run for the Brazil test site do the following:


\subsubsection{Example: Running CLM over a single-point test site in Brazil with the default Qian atmosphere data forcing.}
\label{\detokenize{users_guide/running-single-points/running-single-point-configurations:example-running-clm-over-a-single-point-test-site-in-brazil-with-the-default-qian-atmosphere-data-forcing}}
\begin{sphinxVerbatim}[commandchars=\\\{\}]
\PYGZgt{} cd scripts
\PYGZgt{} set SITE=1x1\PYGZus{}brazil
\PYGZgt{} ./create\PYGZus{}newcase \PYGZhy{}case testSPDATASET \PYGZhy{}res \PYGZdl{}SITE \PYGZhy{}compset I
\PYGZgt{} cd testSPDATASET
\end{sphinxVerbatim}

Then setup, build and run normally.

Then to run for the urban Mexico City Mexico test site that also has atmosphere forcing data, but to run it with the Qian forcing data, but over the period for which it’s own forcing data is provided do the following:


\subsubsection{Example: Running CLM over the single-point of Mexicocity Mexico with the default Qian atmosphere data forcing.}
\label{\detokenize{users_guide/running-single-points/running-single-point-configurations:example-running-clm-over-the-single-point-of-mexicocity-mexico-with-the-default-qian-atmosphere-data-forcing}}
\begin{sphinxVerbatim}[commandchars=\\\{\}]
\PYGZgt{} cd scripts
\PYGZsh{} Set a variable to the site you want to use (as it\PYGZsq{}s used several times below)
\PYGZgt{} set SITE=1x1\PYGZus{}mexicocityMEX
\PYGZgt{} ./create\PYGZus{}newcase \PYGZhy{}case testSPDATASET \PYGZhy{}res \PYGZdl{}SITE \PYGZhy{}compset I
\PYGZgt{} cd testSPDATASET
\PYGZsh{} Set DATM prescribed aerosols to single\PYGZhy{}point dataset
\PYGZsh{} Will then use the dataset with just the point for this \PYGZdl{}SITE
\PYGZgt{} ./xmlchange DATM\PYGZus{}PRESAERO=pt1\PYGZus{}pt1
\end{sphinxVerbatim}

Then setup, build and run normally.

\sphinxstylestrong{Important:} Just like PTS\_MODE above, By default it sets up to run with \sphinxcode{MPILIB=mpi-serial} (in the \sphinxcode{env\_build.xml} file) turned on, which allows you to run the model interactively. On some machines this mode is NOT supported and you may need to change it to FALSE before you are able to build.

\begin{sphinxadmonition}{warning}{Warning:}
See \sphinxhref{CLM-URL}{the Section called Warning about Running with a Single-Processor on a Batch Machine} for a warning about running single-point jobs on batch machines.
\end{sphinxadmonition}

\begin{sphinxadmonition}{note}{Note:}
When running a \sphinxcode{pt1\_pt1} resolution the number of processors is automatically set to one. When running a single grid point you can only use a single processor. You might also want to set the \sphinxcode{env\_build.xml} variable: \sphinxcode{MPILIB=mpi-serial} to TRUE so that you can also run interactively without having to use mpi to start up your job.
\end{sphinxadmonition}


\paragraph{Using Supported Single-point Datasets that have their own Atmospheric Forcing}
\label{\detokenize{users_guide/running-single-points/running-single-point-configurations:using-supported-single-point-datasets-that-have-their-own-atmospheric-forcing}}
Of the supported single-point datasets we have three that also have atmospheric forcing data that go with them: Mexico City (Mexico), Vancouver, (Canada, British Columbia), and \sphinxcode{urbanc\_alpha} (test data for an Urban inter-comparison project).
Mexico city and Vancouver also have “\#ifdef” in the source code for them to work with modified urban data parameters that are particular to these locations.
They can be turned on by using the \sphinxcode{CLM\_CONFIG\_OPTS env\_build.xml} variable to set the “-sitespf\_pt” option in the CLM \sphinxstylestrong{configure}.
To turn on the atmospheric forcing for these datasets, you set the \sphinxcode{env\_run.xml DATM\_MODE} variable to \sphinxcode{CLM1PT}, and then the atmospheric forcing datasets will be used for the point picked.

When running with datasets that have their own atmospheric forcing you need to be careful to run over the period that data is available.
If you have at least one year of forcing it will cycle over the available data over and over again no matter how long of a simulation you run.
However, if you have less than a years worth of data (or if the start date doesn’t start at the beginning of the year, or the end date doesn’t end at the end of the year) then you won’t be able to run over anything but the data extent.
In this case you will need to carefully set the \sphinxcode{RUN\_STARTDATE}, \sphinxcode{START\_TOD} and \sphinxcode{STOP\_N/STOP\_OPTION} variables for your case to run over the entire time extent of your data.
For the supported data points, these values are in the XML database and you can use the \sphinxstylestrong{queryDefaultNamelist.pl} script to query the values and set them for your case (they are set for the three urban test cases: Mexicocity, Vancouver, and urbanc\_alpha).

In the example below we will show how to do this for the Vancouver, Canada point.


\subsubsection{Example: Running CLM over the single-point of Vancouver Canada with supplied atmospheric forcing data for Vancouver.}
\label{\detokenize{users_guide/running-single-points/running-single-point-configurations:example-running-clm-over-the-single-point-of-vancouver-canada-with-supplied-atmospheric-forcing-data-for-vancouver}}
\begin{sphinxVerbatim}[commandchars=\\\{\}]
\PYGZgt{} cd scripts

\PYGZsh{} Set a variable to the site you want to use (as it\PYGZsq{}s used several times below)
\PYGZgt{} set SITE=1x1\PYGZus{}vancouverCAN

\PYGZsh{} Create a case at the single\PYGZhy{}point resolutions with their forcing
\PYGZgt{} ./create\PYGZus{}newcase \PYGZhy{}case testSPDATASETnAtmForcing \PYGZhy{}res \PYGZdl{}SITE \PYGZhy{}compset I1PTCLM45
\PYGZgt{} cd testSPDATASETnAtmForcing

\PYGZsh{} Set namelist options for urban test site
\PYGZgt{} ./xmlchange CLM\PYGZus{}NML\PYGZus{}USE\PYGZus{}CASE=stdurbpt\PYGZus{}pd

\PYGZsh{} Figure out the start and end date for this dataset
\PYGZsh{} You can do this by examining the datafile.
\PYGZgt{} set STOP\PYGZus{}N=330
\PYGZgt{} set START\PYGZus{}YEAR=1992
\PYGZgt{} set STARTDATE=\PYGZdl{}\PYGZob{}START\PYGZus{}YEAR\PYGZcb{}\PYGZhy{}08\PYGZhy{}12
\PYGZgt{} @ NDAYS = \PYGZdl{}STOP\PYGZus{}N / 24
\PYGZgt{} ./xmlchange RUN\PYGZus{}STARTDATE=\PYGZdl{}STARTDATE,STOP\PYGZus{}N=\PYGZdl{}STOP\PYGZus{}N,STOP\PYGZus{}OPTION=nsteps

\PYGZsh{} Set the User namelist to set the output frequencies of the history files
\PYGZsh{} Setting the stdurbpt use\PYGZhy{}case option create three history file streams
\PYGZsh{} The frequencies and number of time\PYGZhy{}samples needs to be set
\PYGZgt{} cat \PYGZlt{}\PYGZlt{} EOF \PYGZgt{} user\PYGZus{}nl\PYGZus{}clm
hist\PYGZus{}mfilt = \PYGZdl{}NDAYS,\PYGZdl{}STOP\PYGZus{}N,\PYGZdl{}STOP\PYGZus{}N
hist\PYGZus{}nhtfrq = \PYGZhy{}1,1,1
EOF

\PYGZsh{} Set DATM prescribed aerosols to single\PYGZhy{}point dataset
\PYGZsh{} Will then use the dataset with just the point for this site
\PYGZgt{} ./xmlchange DATM\PYGZus{}PRESAERO=pt1\PYGZus{}pt1
\PYGZgt{} ./case.setup
\end{sphinxVerbatim}

\begin{sphinxadmonition}{warning}{Warning:}
If you don’t set the start-year and run-length carefully as shown above the model will abort with a “dtlimit error” in the atmosphere model (see bug 1110 in the \sphinxhref{CLM-URL}{models/lnd/clm/doc/KnownLimitationss} file for documentation on this). Since, the forcing data for this site (and the MexicoCity site) is less than a year, the model won’t be able to run for a full year. The \sphinxcode{1x1\_urbanc\_alpha} site has data for more than a full year, but neither year is complete hence, it has the same problem (see the problem for this site above).
\end{sphinxadmonition}

\begin{sphinxadmonition}{note}{Note:}
Just like \sphinxcode{PTS\_MODE} above, By default it sets up to run with \sphinxcode{MPILIB=mpi-serial} (in the env\_build.xml file) turned on, which allows you to run the model interactively.
\end{sphinxadmonition}

\begin{sphinxadmonition}{note}{Note:}
When running a \sphinxcode{pt1\_pt1} resolution the number of processors is automatically set to one. When running a single grid point you can only use a single processor. You might also want to set the \sphinxcode{env\_build.xml} variable: \sphinxcode{MPILIB=mpi-serial} to \sphinxcode{TRUE} so that you can also run interactively without having to use mpi to start up your job.
\end{sphinxadmonition}


\paragraph{Creating your own single-point dataset}
\label{\detokenize{users_guide/running-single-points/running-single-point-configurations:creating-your-own-single-point-dataset}}
The following provides an example of setting up a case using \sphinxcode{CLM\_USRDAT\_NAME} where you rename the files according to the \sphinxcode{CLM\_USRDAT\_NAME} convention.
We have an example of such datafiles in the repository for a specific region over Alaska (actually just a sub-set of the global f19 grid).


\subsubsection{Example: Using CLM\_USRDAT\_NAME to run a simulation using user datasets for a specific region over Alaska}
\label{\detokenize{users_guide/running-single-points/running-single-point-configurations:example-using-clm-usrdat-name-to-run-a-simulation-using-user-datasets-for-a-specific-region-over-alaska}}
\begin{sphinxVerbatim}[commandchars=\\\{\}]
\PYGZgt{} cd scripts
\PYGZgt{} ./create\PYGZus{}newcase \PYGZhy{}case my\PYGZus{}userdataset\PYGZus{}test \PYGZhy{}res CLM\PYGZus{}USRDAT \PYGZhy{}compset ICRUCLM45
\PYGZgt{} cd my\PYGZus{}userdataset\PYGZus{}test/
\PYGZgt{} set GRIDNAME=13x12pt\PYGZus{}f19\PYGZus{}alaskaUSA
\PYGZgt{} set LMASK=gx1v6
\PYGZgt{} ./xmlchange CLM\PYGZus{}USRDAT\PYGZus{}NAME=\PYGZdl{}GRIDNAME,CLM\PYGZus{}BLDNML\PYGZus{}OPTS=\PYGZdq{}\PYGZhy{}mask \PYGZdl{}LMASK\PYGZdq{}
\PYGZgt{} ./xmlchange ATM\PYGZus{}DOMAIN\PYGZus{}FILE=domain.lnd.\PYGZdl{}\PYGZob{}GRIDNAME\PYGZcb{}\PYGZus{}\PYGZdl{}LMASK.nc
\PYGZgt{} ./xmlchange LND\PYGZus{}DOMAIN\PYGZus{}FILE=domain.lnd.\PYGZdl{}\PYGZob{}GRIDNAME\PYGZcb{}\PYGZus{}\PYGZdl{}LMASK.nc

\PYGZsh{} Make sure the file exists in your \PYGZdl{}CSMDATA or else use svn to download it there
\PYGZgt{} ls \PYGZdl{}CSMDATA/lnd/clm2/surfdata\PYGZus{}map/surfdata\PYGZus{}\PYGZdl{}\PYGZob{}GRIDNAME\PYGZcb{}\PYGZus{}simyr2000.nc

\PYGZsh{} If it doesn\PYGZsq{}t exist, comment out the following...
\PYGZsh{}\PYGZgt{} setenv SVN\PYGZus{}INP\PYGZus{}URL https://svn\PYGZhy{}ccsm\PYGZhy{}inputdata.cgd.ucar.edu/trunk/inputdata/
\PYGZsh{}\PYGZgt{} svn export \PYGZdl{}SVN\PYGZus{}INP\PYGZus{}URL/lnd/clm2/surfdata\PYGZus{}map/surfdata\PYGZus{}\PYGZdl{}\PYGZob{}GRIDNAME\PYGZcb{}\PYGZus{}simyr2000.nc \PYGZdl{}CSMDATA/lnd/clm2/surfdata\PYGZus{}map/surfdata\PYGZus{}\PYGZdl{}\PYGZob{}GRIDNAME\PYGZcb{}\PYGZus{}simyr2000.nc
\PYGZgt{} ./case.setup
\end{sphinxVerbatim}

The first step is to create the domain and surface datasets using the process outlined in \sphinxhref{CLM-URL}{the Section called The File Creation Process in Chapter 2}. Below we show an example of the process.


\subsubsection{Example: Creating a surface dataset for a single point}
\label{\detokenize{users_guide/running-single-points/running-single-point-configurations:example-creating-a-surface-dataset-for-a-single-point}}
\begin{sphinxVerbatim}[commandchars=\\\{\}]
\PYGZsh{} set the GRIDNAME and creation date that will be used later
\PYGZgt{} setenv GRIDNAME 1x1\PYGZus{}boulderCO
\PYGZgt{} setenv CDATE    {}`date +\PYGZpc{}y\PYGZpc{}m\PYGZpc{}d{}`
\PYGZsh{} Create the SCRIP grid file for the location and create a unity mapping file for it.
\PYGZgt{} cd models/lnd/clm/tools/shared/mkmapdata
\PYGZgt{} ./mknoocnmap.pl \PYGZhy{}p 40,255 \PYGZhy{}n \PYGZdl{}GRIDNAME
\PYGZsh{} Set pointer to MAPFILE just created that will be used later
\PYGZgt{} setenv MAPFILE {}`pwd{}`/map\PYGZus{}\PYGZdl{}\PYGZob{}GRIDNAME\PYGZcb{}\PYGZus{}noocean\PYGZus{}to\PYGZus{}\PYGZdl{}\PYGZob{}GRIDNAME\PYGZcb{}\PYGZus{}nomask\PYGZus{}aave\PYGZus{}da\PYGZus{}\PYGZdl{}\PYGZob{}CDATE\PYGZcb{}.nc
\PYGZsh{} create the mapping files needed by mksurfdata\PYGZus{}map.
\PYGZgt{} cd ../../shared/mkmapdata
\PYGZgt{} setenv GRIDFILE ../mkmapgrids/SCRIPgrid\PYGZus{}\PYGZdl{}\PYGZob{}GRIDNAME\PYGZcb{}\PYGZus{}nomask\PYGZus{}\PYGZdl{}\PYGZob{}CDATE\PYGZcb{}.nc
\PYGZgt{} ./mkmapdata.sh \PYGZhy{}r \PYGZdl{}GRIDNAME \PYGZhy{}f \PYGZdl{}GRIDFILE \PYGZhy{}t regional
\PYGZsh{} create the domain file
\PYGZgt{} cd ../../../../tools/mapping/gen\PYGZus{}domain\PYGZus{}files/src
\PYGZgt{} ../../../scripts/ccsm\PYGZus{}utils/Machines/configure \PYGZhy{}mach yellowstone \PYGZhy{}compiler intel
\PYGZgt{} gmake
\PYGZgt{} cd ..
\PYGZgt{} setenv OCNDOM domain.ocn\PYGZus{}noocean.nc
\PYGZgt{} setenv ATMDOM domain.lnd.\PYGZob{}\PYGZdl{}GRIDNAME\PYGZcb{}\PYGZus{}noocean.nc
\PYGZgt{} ./gen\PYGZus{}domain \PYGZhy{}m \PYGZdl{}MAPFILE \PYGZhy{}o \PYGZdl{}OCNDOM \PYGZhy{}l \PYGZdl{}ATMDOM
\PYGZsh{} Save the location where the domain file was created
\PYGZgt{} setenv GENDOM\PYGZus{}PATH {}`pwd{}`
\PYGZsh{} Finally create the surface dataset
\PYGZgt{} cd ../../../../lnd/clm/tools/clm4\PYGZus{}5/mksurfdata\PYGZus{}map/src
\PYGZgt{} gmake
\PYGZgt{} cd ..
\PYGZgt{} ./mksurfdata.pl \PYGZhy{}r usrspec \PYGZhy{}usr\PYGZus{}gname \PYGZdl{}GRIDNAME \PYGZhy{}usr\PYGZus{}gdate \PYGZdl{}CDATE
\end{sphinxVerbatim}

The next step is to create a case that points to the files you created above. We will still use the \sphinxcode{CLM\_USRDAT\_NAME} option as a way to get a case setup without having to add the grid to scripts.


\subsubsection{Example: Setting up a case from the single-point surface dataset just created}
\label{\detokenize{users_guide/running-single-points/running-single-point-configurations:example-setting-up-a-case-from-the-single-point-surface-dataset-just-created}}
\begin{sphinxVerbatim}[commandchars=\\\{\}]
\PYGZsh{} First setup an environment variable that points to the top of the CESM directory.
\PYGZgt{} setenv CESMROOT \PYGZlt{}directory\PYGZhy{}of\PYGZhy{}path\PYGZhy{}to\PYGZhy{}main\PYGZhy{}cesm\PYGZhy{}directory\PYGZgt{}
\PYGZsh{} Next make sure you have a inputdata location that you can write to
\PYGZsh{} You only need to do this step once, so you won\PYGZsq{}t need to do this in the future
\PYGZgt{} setenv MYCSMDATA \PYGZdl{}HOME/inputdata     \PYGZsh{} Set env var for the directory for input data
\PYGZgt{} ./link\PYGZus{}dirtree \PYGZdl{}CSMDATA \PYGZdl{}MYCSMDATA
\PYGZsh{} Copy the file you created above to your new \PYGZdl{}MYCSMDATA location following the CLMUSRDAT
\PYGZsh{} naming convention (leave off the creation date)
\PYGZgt{} cp \PYGZdl{}CESMROOT/models/lnd/clm/tools/clm4\PYGZus{}5/mksurfdata\PYGZus{}map/surfdata\PYGZus{}\PYGZdl{}\PYGZob{}GRIDNAME\PYGZcb{}\PYGZus{}simyr1850\PYGZus{}\PYGZdl{}CDATE.nc \PYGZbs{}
\PYGZdl{}MYCSMDATA/lnd/clm2/surfdata\PYGZus{}map/surfdata\PYGZus{}\PYGZdl{}\PYGZob{}GRIDNAME\PYGZcb{}\PYGZus{}simyr1850.nc
\PYGZgt{} cd \PYGZdl{}CESMROOT/scripts
\PYGZgt{} ./create\PYGZus{}newcase \PYGZhy{}case my\PYGZus{}usernldatasets\PYGZus{}test \PYGZhy{}res CLM\PYGZus{}USRDAT \PYGZhy{}compset I1850CRUCLM45BGC \PYGZbs{}
\PYGZhy{}mach yellowstone\PYGZus{}intel
\PYGZgt{} cd my\PYGZus{}usernldatasets\PYGZus{}test
\PYGZgt{} ./xmlchange DIN\PYGZus{}LOC\PYGZus{}ROOT=\PYGZdl{}MYCSMDATA
\PYGZsh{} Set the path to the location of gen\PYGZus{}domain set in the creation step above
\PYGZgt{} ./xmlchange ATM\PYGZus{}DOMAIN\PYGZus{}PATH=\PYGZdl{}GENDOM\PYGZus{}PATH,LND\PYGZus{}DOMAIN\PYGZus{}PATH=\PYGZdl{}GENDOM\PYGZus{}PATH
\PYGZgt{} ./xmlchange ATM\PYGZus{}DOMAIN\PYGZus{}FILE=\PYGZdl{}ATMDOM,LND\PYGZus{}DOMAIN\PYGZus{}FILE=\PYGZdl{}ATMDOM
\PYGZgt{} ./xmlchange CLM\PYGZus{}USRDAT\PYGZus{}NAME=\PYGZdl{}GRIDNAME
\PYGZgt{} ./case.setup
\end{sphinxVerbatim}

\begin{sphinxadmonition}{note}{Note:}
With this and previous versions of the model we recommended using \sphinxcode{CLM\_USRDAT\_NAME} as a way to identify your own datasets without having to enter them into the XML database. This has two down-sides. First you can’t include creation dates in your filenames, which means you can’t keep track of different versions by date. It also means you HAVE to rename the files after you created them with \sphinxstylestrong{mksurfdata.pl}. And secondly, you have to use \sphinxstylestrong{linkdirtree} in order to place the files in a location outside of the usual \sphinxcode{DIN\_LOC\_ROOT} (assuming you don’t have write access to adding new files to the standard location on the machine you are using). Now, since \sphinxcode{user\_nl} files are supported for ALL model components, and the same domain files are read by both CLM and DATM and set using the envxml variables: \sphinxcode{ATM\_DOMAIN\_PATH}, \sphinxcode{ATM\_DOMAIN\_FILE}, \sphinxcode{LND\_DOMAIN\_PATH}, and \sphinxcode{LND\_DOMAIN\_FILE} \textendash{} you can use this mechanism (\sphinxcode{user\_nl\_clm} and \sphinxcode{user\_nl\_datm} and those envxml variables) to point to your datasets in any location. In the future we will deprecate \sphinxcode{CLM\_USRDAT\_NAME} and recommend \sphinxcode{user\_nl\_clm} and \sphinxcode{user\_nl\_datm} and the \sphinxcode{DOMAIN} envxml variables.
\end{sphinxadmonition}


\section{Running PTCLM}
\label{\detokenize{users_guide/running-PTCLM/index::doc}}\label{\detokenize{users_guide/running-PTCLM/index:running-ptclm}}\label{\detokenize{users_guide/running-PTCLM/index:id1}}\phantomsection\label{\detokenize{users_guide/running-PTCLM/introduction-to-ptclm:introduction-to-ptclm-rst}}

\subsection{What is PTCLM}
\label{\detokenize{users_guide/running-PTCLM/introduction-to-ptclm:introduction-to-ptclm-rst}}\label{\detokenize{users_guide/running-PTCLM/introduction-to-ptclm:what-is-ptclm}}\label{\detokenize{users_guide/running-PTCLM/introduction-to-ptclm::doc}}\label{\detokenize{users_guide/running-PTCLM/introduction-to-ptclm:id1}}
PTCLM (pronounced either as point clime or Pee-Tee clime) is a Python script to help you set up PoinT CLM simulations.

It runs the CLM tools for you to get datasets set up, and copies them to a location you can use them according to the \sphinxcode{CLM\_USRDAT\_NAME} naming convention.

Then it runs \sphinxstylestrong{create\_newcase} for you and modifies the env settings and namelist appropriately.

PTCLM has a simple ASCII text file for storing basic information for your sites.

We also have complete lists for AmeriFlux and Fluxnet-Canada sites, although we only have the meteorology data for one site.

For other sites you will need to obtain the meteorology data and translate it to a format that the CESM datm model can use.

But, even without meteorology data PTCLM is useful to setup datasets to run with standard \sphinxcode{CLM\_QIAN} data.

The original authors of PTCLM are: Daniel M. Ricciuto, Dali Wang, Peter E. Thornton, Wilfred M. Post all at Environmental Sciences Division, Oak Ridge National Laboratory (ORNL) and R. Quinn Thomas at Cornell University. It was then modified fairly extensively by Erik Kluzek at NCAR. We want to thank all of these individuals for this contribution to the CESM effort. We also want to thank the folks at University of Michigan Biological Stations (US-UMB) who allowed us to use their Fluxnet station data and import it into our inputdata repository, especially Gil Bohrer the PI on record for this site.


\subsection{Details of PTCLM}
\label{\detokenize{users_guide/running-PTCLM/introduction-to-ptclm:id2}}\label{\detokenize{users_guide/running-PTCLM/introduction-to-ptclm:details-of-ptclm}}
To get help on PTCLM1.110726 use the “\textendash{}help” option as follows.

\begin{sphinxVerbatim}[commandchars=\\\{\}]
\PYG{o}{\PYGZgt{}} \PYG{n}{cd} \PYG{n}{scripts}\PYG{o}{/}\PYG{n}{ccsm\PYGZus{}utils}\PYG{o}{/}\PYG{n}{Tools}\PYG{o}{/}\PYG{n}{lnd}\PYG{o}{/}\PYG{n}{clm}\PYG{o}{/}\PYG{n}{PTCLM}
\PYG{o}{\PYGZgt{}} \PYG{o}{.}\PYG{o}{/}\PYG{n}{PTCLM}\PYG{o}{.}\PYG{n}{py} \PYG{o}{\PYGZhy{}}\PYG{o}{\PYGZhy{}}\PYG{n}{help}
\end{sphinxVerbatim}

The output to the above command is as follows:

\begin{sphinxVerbatim}[commandchars=\\\{\}]
 Usage: PTCLM.py [options] \PYGZhy{}d inputdatadir \PYGZhy{}m machine \PYGZhy{}s sitename

 Python script to create cases to run single point simulations with tower site data.

 Options:
    \PYGZhy{}\PYGZhy{}version             show program\PYGZsq{}s version number and exit
    \PYGZhy{}h, \PYGZhy{}\PYGZhy{}help            show this help message and exit

Required Options:
  \PYGZhy{}d CCSM\PYGZus{}INPUT, \PYGZhy{}\PYGZhy{}csmdata=CCSM\PYGZus{}INPUT
                      Location of CCSM input data
  \PYGZhy{}m MYMACHINE, \PYGZhy{}\PYGZhy{}machine=MYMACHINE
                      Machine, valid CESM script machine (\PYGZhy{}m list to list valid
                      machines)
  \PYGZhy{}s MYSITE, \PYGZhy{}\PYGZhy{}site=MYSITE
                      Site\PYGZhy{}code to run, FLUXNET code or CLM1PT name (\PYGZhy{}s list to list
                      valid names)

Configure and Run Options:
  \PYGZhy{}c MYCOMPSET, \PYGZhy{}\PYGZhy{}compset=MYCOMPSET
                      Compset for CCSM simulation (Must be a valid \PYGZsq{}I\PYGZsq{} compset [other
                      than IG compsets], use \PYGZhy{}c list to list valid compsets)
  \PYGZhy{}\PYGZhy{}coldstart         Do a coldstart with arbitrary initial conditions
  \PYGZhy{}\PYGZhy{}caseidprefix=MYCASEID
                      Unique identifier to include as a prefix to the case name
  \PYGZhy{}\PYGZhy{}cesm\PYGZus{}root=BASE\PYGZus{}CESM
                      Root CESM directory (top level directory with models and scripts
                      subdirs)
  \PYGZhy{}\PYGZhy{}debug             Flag to turn on debug mode so won\PYGZsq{}t run, but display what would
                      happen
  \PYGZhy{}\PYGZhy{}finidat=FINIDAT   Name of finidat initial conditions file to start CLM from
  \PYGZhy{}\PYGZhy{}list              List all valid: sites, compsets, and machines
  \PYGZhy{}\PYGZhy{}namelist=NAMELIST
                      List of namelist items to add to CLM namelist (example:
                      \PYGZhy{}\PYGZhy{}namelist=\PYGZdq{}hist\PYGZus{}fincl1=\PYGZsq{}TG\PYGZsq{},hist\PYGZus{}nhtfrq=\PYGZhy{}1\PYGZdq{}
  \PYGZhy{}\PYGZhy{}QIAN\PYGZus{}tower\PYGZus{}yrs    Use the QIAN forcing data year that correspond to the tower
                      years
  \PYGZhy{}\PYGZhy{}rmold             Remove the old case directory before starting
  \PYGZhy{}\PYGZhy{}run\PYGZus{}n=MYRUN\PYGZus{}N     Number of time units to run simulation
  \PYGZhy{}\PYGZhy{}run\PYGZus{}units=MYRUN\PYGZus{}UNITS
                      Time units to run simulation (steps,days,years, etc.)
  \PYGZhy{}\PYGZhy{}quiet             Print minimul information on what the script is doing
  \PYGZhy{}\PYGZhy{}sitegroupname=SITEGROUP
                      Name of the group of sites to search for you selected site in
                      (look for prefix group names in the PTCLM\PYGZus{}sitedata directory)
  \PYGZhy{}\PYGZhy{}stdurbpt          If you want to setup for standard urban namelist settings
  \PYGZhy{}\PYGZhy{}useQIAN           use QIAN input forcing data instead of tower site meterology data
  \PYGZhy{}\PYGZhy{}verbose           Print out extra information on what the script is doing

Input data generation options:
  These are options having to do with generation of input datasets.  Note: When
  running for supported CLM1PT single\PYGZhy{}point datasets you can NOT generate new
  datasets.  For supported CLM1PT single\PYGZhy{}point datasets, you MUST run with the
  following settings: \PYGZhy{}\PYGZhy{}nopointdata And you must NOT set any of these: \PYGZhy{}\PYGZhy{}soilgrid
  \PYGZhy{}\PYGZhy{}pftgrid \PYGZhy{}\PYGZhy{}owritesrf

  \PYGZhy{}\PYGZhy{}nopointdata       Do NOT make point data (use data already created)
  \PYGZhy{}\PYGZhy{}owritesrf         Overwrite the existing surface datasets if they exist (normally
                      do NOT recreate them)
  \PYGZhy{}\PYGZhy{}pftgrid           Use pft information from global gridded file (rather than site
                      data)
  \PYGZhy{}\PYGZhy{}soilgrid          Use soil information from global gridded file (rather than site
                      data)

Main Script Version Id: \PYGZdl{}Id: PTCLM.py 47576 2013\PYGZhy{}05\PYGZhy{}29 19:11:16Z erik \PYGZdl{} Scripts URL: \PYGZdl{}HeadURL: https://svn\PYGZhy{}ccsm\PYGZhy{}models.cgd.ucar.edu/PTCLM/trunk\PYGZus{}tags/PTCLM1\PYGZus{}130529/PTCLM.py \PYGZdl{}:
\end{sphinxVerbatim}

Here we give a simple example of using PTCLM1 for a straightforward case of running at the US-UMB Fluxnet site on yellowstone where we already have the meteorology data on the machine.
Note, see \sphinxhref{CLM-URL}{the Section called Converting AmeriFlux Data for use by PTCLM} for permission information to use this data.


\subsubsection{Example 6-1. Example of running PTCLM1 for US-UMB on yellowstone}
\label{\detokenize{users_guide/running-PTCLM/introduction-to-ptclm:example-6-1-example-of-running-ptclm1-for-us-umb-on-yellowstone}}
\begin{sphinxVerbatim}[commandchars=\\\{\}]
\PYGZgt{} setenv CSMDATA   \PYGZdl{}CESMDATAROOT/inputdata
\PYGZgt{} setenv MYCSMDATA \PYGZdl{}HOME/inputdata
\PYGZgt{} setenv SITE      US\PYGZhy{}UMB
\PYGZgt{} setenv MYMACH    yellowstone\PYGZus{}intel
\PYGZgt{} setenv MYCASE    testPTCLM
\PYGZsh{} First link the standard input files to a location you have write access
\PYGZgt{} cd scripts
\PYGZgt{} ./link\PYGZus{}dirtree \PYGZdl{}CSMDATA \PYGZdl{}MYCSMDATA

\PYGZsh{} Next build all of the clm tools you will need
\PYGZgt{} cd ../models/lnd/clm/tools/clm4\PYGZus{}5/mksurfdata\PYGZus{}map
\PYGZgt{} gmake
\PYGZgt{} gmake clean
\PYGZgt{} cd ../../../../../../tools/mapping/gen\PYGZus{}domain\PYGZus{}files/src
\PYGZgt{} ../../../../scripts/ccsm\PYGZus{}utils/Machines/configure \PYGZhy{}mach yellowstone \PYGZhy{}compiler intel
\PYGZgt{} gmake
\PYGZgt{} gmake clean
\PYGZsh{} next run PTCLM (NOTE \PYGZhy{}\PYGZhy{} MAKE SURE python IS IN YOUR PATH)
\PYGZgt{} cd ../../../../../scripts/ccsm\PYGZus{}utils/Tools/lnd/clm/PTCLM
\PYGZgt{} ./PTCLM.py \PYGZhy{}m \PYGZdl{}MYMACH  \PYGZhy{}\PYGZhy{}case=\PYGZdl{}MYCASE \PYGZhy{}\PYGZhy{}site=\PYGZdl{}SITE \PYGZhy{}\PYGZhy{}csmdata=\PYGZdl{}MYCSMDATA \PYGZhy{}\PYGZhy{}aerdepgrid \PYGZhy{}\PYGZhy{}ndepgrid
\PYGZsh{} NOTE: we use \PYGZhy{}\PYGZhy{}aerdepgrid \PYGZhy{}\PYGZhy{}ndepgrid so that you use the global
\PYGZsh{} aerosol and Nitrogen deposition files rather than site\PYGZhy{}specific ones.
\PYGZgt{} cd ../../../../../\PYGZdl{}MYCASE
\PYGZsh{} Finally setup, build, and run the case as normal
\end{sphinxVerbatim}


\subsection{Using PTCLM}
\label{\detokenize{users_guide/running-PTCLM/using-ptclm::doc}}\label{\detokenize{users_guide/running-PTCLM/using-ptclm:using-ptclm}}\label{\detokenize{users_guide/running-PTCLM/using-ptclm:using-ptclm-rst}}
There are three types of options to PTCLM1: required, setup/run-time, and dataset generation options.
The three required options are the three settings that MUST be specified for PTCLM to work at all. The other settings have default values that will default to something useful. The setup/run-time options control how the simulation will be setup and run. The dataset generation options control the generation of datasets needed when PTCLM is run. Most options use a double dash “\textendash{}” “longname” such as “\textendash{}list”, but the most common options also have a short-name with a single dash (such as -m instead of \textendash{}machine).

The required options to PTCLM are: inputdata directory (-d), machine (-m) and site-name (-s).
Inputdata directory is the directory where you have the CESM inputdata files, you need to have write access to this directory, so if you are running on a machine that you do NOT have write access to the standard inputdata location (such as NCAR yellowstone or LBNL hopper) you need to link the standard files to a location you do have control over. We recommend using the \sphinxcode{scripts/link\_dirtree} tool to do that. “machine” is the scripts name for the machine/compiler you will be using for your case. And finally site-name is the name of the site that you want to run for. Site-name can either be a valid supported dataset name or a Fluxnet site name from the list of sites you are running on (see the \textendash{}sitegroupname for more information about the site lists).

After PTCLM is run a case directory where you can then setup, build and run your CESM case as normal.
It also creates a \sphinxcode{README.PTCLM} in that directory that documents the commandline options to PTCLM that were used to create it.

After “help” the “list” option is one of the most useful options for getting help on using PTCLM.
This option gives you information about some of the other options to PTCLM. To get a list of the machine, sites, and compsets that can be used for PTCLM use the “\textendash{}list” option as follows.

\begin{sphinxVerbatim}[commandchars=\\\{\}]
\PYG{o}{\PYGZgt{}} \PYG{n}{cd} \PYG{n}{scripts}\PYG{o}{/}\PYG{n}{ccsm\PYGZus{}utils}\PYG{o}{/}\PYG{n}{Tools}\PYG{o}{/}\PYG{n}{lnd}\PYG{o}{/}\PYG{n}{clm}\PYG{o}{/}\PYG{n}{PTCLM}
\PYG{o}{\PYGZgt{}} \PYG{o}{.}\PYG{o}{/}\PYG{n}{PTCLM}\PYG{o}{.}\PYG{n}{py} \PYG{o}{\PYGZhy{}}\PYG{o}{\PYGZhy{}}\PYG{n+nb}{list}
\end{sphinxVerbatim}

The output to the above command is as follows:

\begin{sphinxVerbatim}[commandchars=\\\{\}]
\PYG{o}{/}\PYG{n+nb}{bin}\PYG{o}{/}\PYG{n}{sh}\PYG{p}{:} \PYG{n}{line} \PYG{l+m+mi}{1}\PYG{p}{:} \PYG{n}{PTCLM}\PYG{o}{.}\PYG{n}{py}\PYG{p}{:} \PYG{n}{command} \PYG{o+ow}{not} \PYG{n}{found}
\end{sphinxVerbatim}


\subsubsection{Steps in running PTCLM}
\label{\detokenize{users_guide/running-PTCLM/using-ptclm:steps-in-running-ptclm}}\begin{enumerate}
\item {} 
Setup Inputdata directory with write access (use link\_dirtree script)

You need to setup an inputdata directory where you have write access to it.
Normally, for NCAR machines the data is on an inputdata where the user does NOT have write access to it.
A way that you can get around this is to use the \sphinxstylestrong{link\_dirtree} script to create softlinks from the normal location to a location you have write access to.
So for example on yellowstone:

\begin{sphinxVerbatim}[commandchars=\\\{\}]
\PYGZgt{} setenv CSMDATA \PYGZdl{}CESMDATAROOT/inputdata
\PYGZgt{} setenv MYCSMDATA \PYGZdl{}HOME/inputdata
\PYGZgt{} mkdir \PYGZdl{}MYCSMDATA
\PYGZgt{} cd scripts
\PYGZgt{} ./link\PYGZus{}dirtree \PYGZdl{}CSMDATA \PYGZdl{}MYCSMDATA
\end{sphinxVerbatim}

See \sphinxhref{CLM-URL}{the Section called Managing Your Own Data-files in Chapter 3} for more information on this.

\item {} 
Build the CLM tools
Next you need to make sure all the CLM FORTRAN tools are built.

\begin{sphinxVerbatim}[commandchars=\\\{\}]
\PYG{o}{\PYGZgt{}} \PYG{n}{cd} \PYG{n}{models}\PYG{o}{/}\PYG{n}{lnd}\PYG{o}{/}\PYG{n}{clm}\PYG{o}{/}\PYG{n}{tools}\PYG{o}{/}\PYG{n}{clm4\PYGZus{}5}\PYG{o}{/}\PYG{n}{mksurfdata\PYGZus{}map}
\PYG{o}{\PYGZgt{}} \PYG{n}{gmake}
\PYG{o}{\PYGZgt{}} \PYG{n}{gmake} \PYG{n}{clean}
\PYG{o}{\PYGZgt{}} \PYG{n}{cd} \PYG{o}{.}\PYG{o}{.}\PYG{o}{/}\PYG{o}{.}\PYG{o}{.}\PYG{o}{/}\PYG{o}{.}\PYG{o}{.}\PYG{o}{/}\PYG{o}{.}\PYG{o}{.}\PYG{o}{/}\PYG{o}{.}\PYG{o}{.}\PYG{o}{/}\PYG{o}{.}\PYG{o}{.}\PYG{o}{/}\PYG{n}{tools}\PYG{o}{/}\PYG{n}{mapping}\PYG{o}{/}\PYG{n}{gen\PYGZus{}domain\PYGZus{}files}\PYG{o}{/}\PYG{n}{src}
\PYG{o}{\PYGZgt{}} \PYG{o}{.}\PYG{o}{.}\PYG{o}{/}\PYG{o}{.}\PYG{o}{.}\PYG{o}{/}\PYG{o}{.}\PYG{o}{.}\PYG{o}{/}\PYG{o}{.}\PYG{o}{.}\PYG{o}{/}\PYG{n}{scripts}\PYG{o}{/}\PYG{n}{ccsm\PYGZus{}utils}\PYG{o}{/}\PYG{n}{Machines}\PYG{o}{/}\PYG{n}{configure} \PYG{o}{\PYGZhy{}}\PYG{n}{mach} \PYG{n}{yellowstone} \PYG{o}{\PYGZhy{}}\PYG{n}{compiler} \PYG{n}{intel}
\PYG{o}{\PYGZgt{}} \PYG{n}{gmake}
\PYG{o}{\PYGZgt{}} \PYG{n}{gmake} \PYG{n}{clean}
\end{sphinxVerbatim}

\item {} 
Run PTCLM
Next you actually run PTCLM1 which does the different things listed below:
\begin{enumerate}
\item {} 
PTCLM names your case based on your input

\begin{sphinxVerbatim}[commandchars=\\\{\}]
\PYG{p}{[}\PYG{n}{Prefix\PYGZus{}}\PYG{p}{]}\PYG{n}{SiteCode\PYGZus{}Compset}\PYG{p}{[}\PYG{n}{\PYGZus{}QIAN}\PYG{p}{]}
\end{sphinxVerbatim}
\begin{description}
\item[{Where:}] \leavevmode
\sphinxcode{Prefix} is from the caseidprefix option (or blank if not used).

\sphinxcode{SiteCode} is the site name you entered with the -s option.

\sphinxcode{Compset} is the compset name you entered with the -c option.

\sphinxcode{\_QIAN} is part of the name only if the useQIAN is used.

\end{description}

For example, the casename for the following will be:

\begin{sphinxVerbatim}[commandchars=\\\{\}]
\PYGZgt{} cd scripts
\PYGZgt{} ./PTCLM.py \PYGZhy{}m yellowstone\PYGZus{}intel \PYGZhy{}s US\PYGZhy{}UMB \PYGZhy{}d \PYGZdl{}MYCSMDATA \PYGZhy{}c ICRUCLM45BGC \PYGZhy{}\PYGZhy{}use QIAN \PYGZdq{}US\PYGZhy{}UMB\PYGZus{}I\PYGZus{}2000\PYGZus{}CN\PYGZus{}QIAN\PYGZdq{}
\end{sphinxVerbatim}

\item {} 
PTCLM creates datasets for you
It will populate \$MYCSMDATA with new datasets it creates using the CLM tools.

\item {} 
If a transient compset and PTCLM1 finds a \_dynpftdata.txt file
If you are running a transient compset (such as the “I\_1850-2000\_CN” compset) AND you there is a file in the PTCLM\_sitedata directory under the PTCLM directory called \$SITE\_dynpftdata.txt it will use this file for the land-use changes.
Otherwise it will leave land-use constant, unless you use the pftgrid option so it uses the global dataset for landuse changes.
See the Section called Dynamic Land-Use Change Files for use by PTCLM for more information on this.
There is a sample transient dataset called US-Ha1\_dynpftdata.txt.
Transient compsets, are compsets that create transient land-use change and forcing conditions such as: ‘I\_1850-2000’, ‘I\_1850-2000\_CN’, ‘I\_RCP8.5\_CN’, ‘I\_RCP6.0\_CN’, ‘I\_RCP4.5\_CN’, or ‘I\_RCP2.6\_CN’.

\item {} 
PTCLM creates a pft-physiology for you
PTCLM1 will create a local copy of the pft-physiology specific for your site that you could then customize with changes specific for that site.

\item {} 
PTCLM creates a README.PTCLM for you
PTCLM1 will create a simple text file with the command line for it in a file called README.PTCLM in the case directory it creates for you.

\end{enumerate}

\item {} 
Customize, setup, build and run case as normal
You then customize your case as you would normally. See the Chapter 1 chapter for more information on doing this.

\end{enumerate}


\subsubsection{PTCLM options}
\label{\detokenize{users_guide/running-PTCLM/using-ptclm:ptclm-options}}
Next we discuss the setup and run-time options, dividing them up into setup, initial condition (IC), and run-time options.

Configure options include:
\begin{itemize}
\item {} 
\textendash{}compset=MYCOMPSET

\item {} 
\textendash{}caseidprefix=MYCASEID

\item {} 
\textendash{}cesm\_root=BASE\_CESM

\item {} 
\textendash{}namelist=NAMELIST

\item {} 
\textendash{}rmold

\item {} 
\textendash{}scratchroot=SCRATCHROOT

\item {} 
\textendash{}sitegroupname=SITEGROUP

\item {} 
\textendash{}QIAN\_tower\_yrs

\item {} 
\textendash{}useQIAN

\end{itemize}
\begin{description}
\item[{\sphinxcode{-{-}compset}}] \leavevmode
The “-c” option is the most commonly used option after the required options, as it specifies the CESM scripts component set to use with PTCLM1.
The default compset is the “ICN” compset with CN on for present day conditions.

\item[{\sphinxcode{-{-}caseidprefix}}] \leavevmode
This option gives a prefix to include in the casename when the case is created, in case you want to customize your casenames a bit.
By default, casenames are figured out based on the other options. The argument to this option can either be a name to prefix casenames with and/or a pathname to include.
Hence, if you want cases to appear in a specific directory you can give the pathname to that directory with this option.

\item[{\sphinxcode{-{-}cesm\_root}}] \leavevmode
This option is for running PTCLM1 with a different root directory to CESM than the version PTCLM exists in. Normally you do NOT need to use this option.

\item[{\sphinxcode{-{-}namelist}}] \leavevmode
This option adds any items given into the CLM user\_nl\_clm namelist. This allows you to add customizations to the namelist before the clm.buildnml.csh file is created for the case.

\item[{\sphinxcode{-{-}rmold}}] \leavevmode
This option will remove an old case directory of the same name if one exists. Otherwise, if an old case directory already exists and you try to run PTCLM it will return with an error.

\item[{\sphinxcode{-{-}scratchroot}}] \leavevmode
This option is ONLY valid when using one of the generic machines (the -m option). This passed onto \sphinxstylestrong{create\_newcase} and gives the location where cases will be built and run.

\item[{\sphinxcode{-{-}sitegroupname}}] \leavevmode
In the PTCLM directory there is a subdirectory “PTCLM\_sitedata” that contains files with the site, PFT and soil data information for groups of sites.
These site groups are all separate ASCII files with the same prefix followed by a “\_*data.txt” name. See \sphinxhref{CLM-URL}{the Section called PTCLM Group Site Lists} for more information on these files. By default we have provided three different valid group names:

\end{description}


\paragraph{EXAMPLE}
\label{\detokenize{users_guide/running-PTCLM/using-ptclm:example}}
AmeriFlux

Fluxnet-Canada

The EXAMPLE is the group used by default and ONLY includes the US-UMB site as that is the only site we have data provided for.
The other two site groups include the site information for all of both the AmeriFlux and Fluxnet-Canada sites.
You can use the “sitegroupname” option to use one of the other lists, or you can create your own lists using the EXAMPLE file as an example.
Your list of sites could be real world locations or could be theoretical “virtual” sites given to exercise CLM on differing biomes for example.
Note, see \sphinxhref{CLM-URL}{the Section called Converting AmeriFlux Data for use by PTCLM} with permission information to use the US-UMB data.
\begin{description}
\item[{\sphinxcode{-{-}useQIAN}}] \leavevmode
This option says to use the standard CLM global Qian T62 atmospheric forcing rather than any tower site forcing data available. Otherwise, PTCLM will try to find tower forcing data for the specific site entered.

\item[{\sphinxcode{-{-}QIAN\_tower\_yrs}}] \leavevmode
This option is used with the “useQIAN” option to set the years to cycle over for the Qian data. In this case Qian atmospheric forcing will be used, but the simulation will run over the same years that tower site is available for this site.

\end{description}

\sphinxstylestrong{IC options include:}
\begin{itemize}
\item {} 
\textendash{}coldstart

\item {} 
\textendash{}finidat=FINIDAT

\end{itemize}

The coldstart option says to startup with OUT an initial condition file, while the finidat option explicitly gives the initial condition file to use. Obviously, the coldstart and finidat options can NOT be used together.
\begin{description}
\item[{\sphinxcode{-{-}coldstart}}] \leavevmode
This option ensures that a cold-start will be done with arbitrary initial conditions.

\item[{\sphinxcode{-{-}finidat}}] \leavevmode
This option sets the initial condition file to startup the simulation from.

\end{description}

\sphinxstylestrong{Run-time options include:}
\begin{itemize}
\item {} 
\textendash{}debug

\item {} 
\textendash{}run\_n=MYRUN\_N

\item {} 
\textendash{}run\_units=MYRUN\_UNITS

\item {} 
\textendash{}stdurbpt

\item {} 
\textendash{}debug

\end{itemize}

This option tells PTCLM to echo what it would do if it were run, but NOT actually run anything. So it will show you the dataset creation commands it would use. It does however, run \sphinxstylestrong{create\_newcase}, but then it only displays the \sphinxstylestrong{xmlchange} commands and changes that it would do. Also note that if you give the “\textendash{}rmold” option it won’t delete the case directory beforehand. Primarily this is intended for debugging the operation of PTCLM.
\begin{description}
\item[{\sphinxcode{-{-}run\_n}}] \leavevmode
This option along with run\_units is used to set the length for the simulation. “run\_n” is the number of units to use. The default run length depends on the site, compset, and configuration.

\item[{\sphinxcode{-{-}run\_units}}] \leavevmode
This option is the units of time to use for the length of the simulation. It is used along with “run\_n” to set the length of the simulation. The default run length depends on the site, compset, and configuration.

\item[{\sphinxcode{-{-}stdurbpt}}] \leavevmode
This option turns on the “stdurbpt\_pd” use-case for CLM\_NML\_USE\_CASE. This option can NOT be used for compsets that set the use-case to something besides present-day.

\end{description}

\sphinxstylestrong{The dataset generation options are:}
\begin{itemize}
\item {} 
\textendash{}pftgrid

\item {} 
\textendash{}soilgrid

\item {} 
\textendash{}nopointdata

\item {} 
\textendash{}owritesrfaer

\end{itemize}

The options that with a “grid” suffix all mean to create datasets using the global gridded information rather than using the site specific point data. By default the site specific point data is used. The “nopointdata” and “owritesrfaer” options have to do with file creation.

Because supported single-point datasets already have the data created for them, you MUST use the “nopointdata” and “ndepgrid” options when you are using a supported single-point site. You must use “ndepgrid” even for a compset without CN. You also can NOT use the options: “soilgrid”, “pftgrid”, “aerdepgrid”, or “owritesrfaer”.
\begin{description}
\item[{\sphinxcode{-{-}pftgrid}}] \leavevmode
This option says to use the PFT values provided on the global dataset rather than using the specific site based values from the PTCLM\_sitedata/*\_pftdata.txt file when creating the surface dataset.
This option must NOT be used when you you are using a site that is a supported single point dataset.

\item[{\sphinxcode{-{-}soilgrid}}] \leavevmode
This option says to use the soil values provided on the global dataset rather than using the specific site based values from the PTCLM\_sitedata/*\_soildata.txt file when creating the surface dataset.
This option must NOT be used when you you are using a site that is a supported single point dataset.

\item[{\sphinxcode{-{-}nopointdata}}] \leavevmode
This option says to NOT create any input datasets \textendash{} assume this step has already been done.
If datasets weren’t already created, your case will fail when you try to run it.
In general the first time you run PTCLM for a new site you want it to generate new datasets, but the next time and future times you want to use this option so that it doesn’t waste a lot of time rebuilding datasets over again.

\begin{sphinxadmonition}{note}{Note:}
This option is required when you you are using a site that is a supported single point dataset.
\end{sphinxadmonition}

\item[{\sphinxcode{-{-}owritesrfaer}}] \leavevmode
This option says to overwrite any surface and/or aerosol deposition datasets that were already created.
Otherwise, the creation of these files will be skipped if a file is already found (but it WILL create files if they don’t exist).
This option must NOT be used when you you are using a site that is a supported single point dataset.

\end{description}


\subsection{Examples of using PTCLM}
\label{\detokenize{users_guide/running-PTCLM/ptclm-examples:ptclm-examples}}\label{\detokenize{users_guide/running-PTCLM/ptclm-examples::doc}}\label{\detokenize{users_guide/running-PTCLM/ptclm-examples:examples-of-using-ptclm}}
Now let’s give a few more complex examples using some of the options we have discussed above.

In this first example, we’ll demonstrate using a supported single point dataset, which then requires using the “nopointdata”. We’ll also demonstrate the compset option, “stdurbpt” and “caseidprefix” options.


\subsubsection{Example: Running PTCLM for the Mexicocity supported single point dataset}
\label{\detokenize{users_guide/running-PTCLM/ptclm-examples:example-running-ptclm-for-the-mexicocity-supported-single-point-dataset}}
\begin{sphinxVerbatim}[commandchars=\\\{\}]
\PYGZgt{} cd scripts/ccsm\PYGZus{}utils/Tools/lnd/clm/PTCLM
\PYGZgt{} ./PTCLM.py \PYGZhy{}m yellowstone\PYGZus{}intel \PYGZhy{}s 1x1\PYGZus{}mexicocityMEX \PYGZhy{}d \PYGZdl{}CSMDATA \PYGZhy{}\PYGZhy{}nopointdata \PYGZbs{}
\PYGZhy{}\PYGZhy{}stdurbpt \PYGZhy{}c ICRUCLM45 \PYGZhy{}\PYGZhy{}caseidprefix {}`pwd{}`/myPTCLMcases/site
\PYGZgt{} cd myPTCLMcases/site\PYGZus{}1x1\PYGZus{}mexicocityMEX\PYGZus{}I
\PYGZgt{} ./cesm\PYGZus{}setup
\PYGZsh{} Now build and run normally
\PYGZgt{} ./site\PYGZus{}1x1\PYGZus{}mexicocityMEX\PYGZus{}I.build
\PYGZsh{} Here we show running interactively
\PYGZgt{} ./site\PYGZus{}1x1\PYGZus{}mexicocityMEX\PYGZus{}I.run
\end{sphinxVerbatim}

Now, let’s demonstrate using a different group list, doing a spinup, running with Qian global forcing data, but using tower years to set the years to run over. This uses the options: sitegroupname, useQIAN, and QIANtower\_years.


\subsubsection{Example: Running PTCLM for a spinup simulation with Qian data for tower years.}
\label{\detokenize{users_guide/running-PTCLM/ptclm-examples:example-running-ptclm-for-a-spinup-simulation-with-qian-data-for-tower-years}}
\begin{sphinxVerbatim}[commandchars=\\\{\}]
\PYGZgt{} cd scripts/ccsm\PYGZus{}utils/Tools/lnd/clm/PTCLM
\PYGZgt{} ./PTCLM.py \PYGZhy{}m yellowstone\PYGZus{}intel \PYGZhy{}s US\PYGZhy{}Ha1 \PYGZhy{}d \PYGZdl{}CSMDATA \PYGZhy{}\PYGZhy{}sitegroupname AmeriFlux \PYGZhy{}\PYGZhy{}useQIAN \PYGZhy{}\PYGZhy{}QIAN\PYGZus{}tower\PYGZus{}yrs
\PYGZgt{} cd ../../../../../US\PYGZhy{}Ha1\PYGZus{}ICRUCLM45BGC\PYGZus{}QIAN
\PYGZgt{} ./cesm\PYGZus{}setup
\PYGZsh{} Now build and run normally
\PYGZgt{} ./US\PYGZhy{}Ha1\PYGZus{}ICRUCLM45BGC\PYGZus{}QIAN.build
\PYGZsh{} Here we show running interactively
\PYGZgt{} ./US\PYGZhy{}Ha1\PYGZus{}ICRUCLM45BGC\PYGZus{}QIAN.run
{}`{}`{}`
\end{sphinxVerbatim}

Finally, let’s demonstrate using a generic machine (which then requires the scratchroot option), using the global grid for PFT and soil types, and setting the run length to two months.


\subsubsection{Example: Running PTCLM on a user-defined machine with global PFT and soil types dataset}
\label{\detokenize{users_guide/running-PTCLM/ptclm-examples:example-running-ptclm-on-a-user-defined-machine-with-global-pft-and-soil-types-dataset}}
\begin{sphinxVerbatim}[commandchars=\\\{\}]
\PYGZgt{} cd scripts/ccsm\PYGZus{}utils/Tools/lnd/clm/PTCLM
\PYGZsh{} Note, see the the Section called Converting AmeriFlux Data for use by PTCLM with permission information
\PYGZsh{} to use the US\PYGZhy{}UMB data.
\PYGZgt{} ./PTCLM.py \PYGZhy{}m userdefined\PYGZus{}intel \PYGZhy{}s US\PYGZhy{}UMB \PYGZhy{}d \PYGZdl{}CSMDATA \PYGZhy{}\PYGZhy{}pftgrid \PYGZhy{}\PYGZhy{}soilgrid \PYGZbs{}
\PYGZhy{}\PYGZhy{}scratchroot \PYGZdl{}HOME \PYGZhy{}\PYGZhy{}run\PYGZus{}n 2 \PYGZhy{}\PYGZhy{}run\PYGZus{}units nmonths
\PYGZgt{} cd ../../../../../US\PYGZhy{}UMB\PYGZus{}ICRUCLM45BGC
\PYGZsh{} If userdefined is NOT set up for you Uncomment the following and set OS, NTASKS\PYGZus{}PER\PYGZus{}NODE, TMPDIR
\PYGZsh{} \PYGZgt{} ./xmlchange OS=\PYGZdl{}OS,MAX\PYGZus{}TASKS\PYGZus{}PER\PYGZus{}NODE=\PYGZdl{}NTASKS\PYGZus{}PER\PYGZus{}NODE,MPILIB=mpi\PYGZhy{}serial
\PYGZsh{} \PYGZgt{} ./xmlchange RUNDIR=\PYGZdl{}TMPDIR/\PYGZdl{}USER/\PYGZbs{}\PYGZdl{}CASE/run,DIN\PYGZus{}LOC\PYGZus{}ROOT=\PYGZdl{}CSMDATA,COMPILER=intel
\PYGZsh{} \PYGZgt{} ./xmlchange EXEROOT=\PYGZdl{}TMPDIR/\PYGZdl{}USER/\PYGZbs{}\PYGZdl{}CASE
\PYGZgt{} ./cesm\PYGZus{}setup
\PYGZsh{} Now build
\PYGZgt{} ./US\PYGZhy{}UMB\PYGZus{}ICRUCLM45BGC.userdefined\PYGZus{}intel.build
\PYGZsh{} To get the files from the svn server...
\PYGZsh{} First list the files from the streams text file
\PYGZgt{} ../ccsm\PYGZus{}utils/Tools/listfilesin\PYGZus{}streams \PYGZbs{}
\PYGZhy{}t \PYGZdl{}HOME/US\PYGZhy{}UMB\PYGZus{}ICRUCLM45BGC/run/clm1PT.1x1pt\PYGZus{}US\PYGZhy{}UMB.stream.txt \PYGZhy{}l \PYGZbs{}
\PYGZgt{} Buildconf/datm.input\PYGZus{}data\PYGZus{}list
\PYGZsh{} And now run the script to export data to your machine
\PYGZgt{} ../ccsm\PYGZus{}utils/Tools/check\PYGZus{}input\PYGZus{}data \PYGZhy{}export
\PYGZsh{} Here we show running interactively
\PYGZgt{} ./US\PYGZhy{}UMB\PYGZus{}ICRUCLM45BGC.userdefined\PYGZus{}intel.run
\end{sphinxVerbatim}


\subsection{Adding PTCLM Site Data}
\label{\detokenize{users_guide/running-PTCLM/adding-ptclm-site-data::doc}}\label{\detokenize{users_guide/running-PTCLM/adding-ptclm-site-data:adding-ptclm-site-data}}\label{\detokenize{users_guide/running-PTCLM/adding-ptclm-site-data:id1}}
The “sitegroupname” option to PTCLM1.110726 looks for groups of sites in the files in the \sphinxcode{PTCLM\_sitedata} directory under the PTCLM directory.
You can add new names available for this option including your own lists of sites, by adding more files in this directory.
There are three files for each “sitegroupname”: \sphinxcode{\$SITEGROUP\_sitedata.txt}, \sphinxcode{\$SITEGROUP\_soildata.txt} and \sphinxcode{\$SITEGROUP\_pftdata.txt} (where \sphinxcode{\$SITEGROUP} is the name that would be entered as “sitegroupname” to PTCLM).
Each file needs to have the same list of sites, but gives different information: site data, PFT data, and soil data respectively.
Although the site codes need to be the same between the three files, the files do NOT have to be in the same order.
Each file has a one-line header that lists the contents of each column which are separated by commas.
The first column for each of the files is the “site\_code” which must be consistent between the three files.
The site code can be any unique character string, but in general we use the AmeriFlux site code.

Site data file:{}`{}` \$SITEGROUP\_sitedata.txt{}`{}`): The header for this file is:

\begin{sphinxVerbatim}[commandchars=\\\{\}]
\PYG{n}{site\PYGZus{}code}\PYG{p}{,}\PYG{n}{name}\PYG{p}{,}\PYG{n}{state}\PYG{p}{,}\PYG{n}{lon}\PYG{p}{,}\PYG{n}{lat}\PYG{p}{,}\PYG{n}{elev}\PYG{p}{,}\PYG{n}{startyear}\PYG{p}{,}\PYG{n}{endyear}\PYG{p}{,}\PYG{n}{alignyear}
\end{sphinxVerbatim}

The columns: name, state, and elevation are informational only. Name is a longer descriptive name of the site, and state is the state for U.S. sites or country for non U.S. sites. The columns: lon and lat are the longitude and latitude of the location in decimal degrees. The last three columns are the start and ending year for the data and the align year for an 1850 case for the data. The align year is currently unused.

Soil data file: \sphinxcode{\$SITEGROUP\_soildata.txt}): The header for this file is:

\begin{sphinxVerbatim}[commandchars=\\\{\}]
\PYG{n}{site\PYGZus{}code}\PYG{p}{,}\PYG{n}{soil\PYGZus{}depth}\PYG{p}{,}\PYG{n}{n\PYGZus{}layers}\PYG{p}{,}\PYG{n}{layer\PYGZus{}depth}\PYG{p}{,}\PYG{n}{layer\PYGZus{}sand}\PYG{o}{\PYGZpc{}}\PYG{p}{,}\PYG{n}{layer\PYGZus{}clay}\PYG{o}{\PYGZpc{}}
\end{sphinxVerbatim}

The first three fields after “site\_code” are currently unused. The only two that are used are the percent sand and clay columns to set the soil texture.

PFT data file: \sphinxcode{\$SITEGROUP\_pftdata.txt{}`}): The header for this file is:

\begin{sphinxVerbatim}[commandchars=\\\{\}]
\PYG{n}{site\PYGZus{}code}\PYG{p}{,}\PYG{n}{pft\PYGZus{}f1}\PYG{p}{,}\PYG{n}{pft\PYGZus{}c1}\PYG{p}{,}\PYG{n}{pft\PYGZus{}f2}\PYG{p}{,}\PYG{n}{pft\PYGZus{}c2}\PYG{p}{,}\PYG{n}{pft\PYGZus{}f3}\PYG{p}{,}\PYG{n}{pft\PYGZus{}c3}\PYG{p}{,}\PYG{n}{pft\PYGZus{}f4}\PYG{p}{,}\PYG{n}{pft\PYGZus{}c4}\PYG{p}{,}\PYG{n}{pft\PYGZus{}f5}\PYG{p}{,}\PYG{n}{pft\PYGZus{}c5}
\end{sphinxVerbatim}

This file gives the vegetation coverage for the different vegetation types for the site. The file only supports up to five PFT’s at the same time. The columns with “pft\_f” are the fractions for each PFT, and the columns with “pft\_c” is the integer index of the given PFT. Look at the pft-physiology file to see what the PFT index for each PFT type is.


\subsubsection{Dynamic Land-Use Change Files for use by PTCLM}
\label{\detokenize{users_guide/running-PTCLM/adding-ptclm-site-data:dynamic-land-use-change-files-for-use-by-ptclm}}
There is a mechanism for giving site-specific land-use change in PTCLM. Adding site specific files to the \sphinxcode{PTCLM\_sitedata} directory under PTCLM allows you to specify the change in vegetation and change in harvesting (for the CN model) for that site. Files are named: \sphinxcode{\$SITE\_dynpftdata.txt}. There is a sample file for the US-Ha1 site called: \sphinxcode{US-Ha1\_dynpftdata.txt}. The file has a one-line header with the information that the file has, and then one-line for each year with a transition. The header line is as follows:

\begin{sphinxVerbatim}[commandchars=\\\{\}]
\PYG{n}{trans\PYGZus{}year}\PYG{p}{,}\PYG{n}{pft\PYGZus{}f1}\PYG{p}{,}\PYG{n}{pft\PYGZus{}c1}\PYG{p}{,}\PYG{n}{pft\PYGZus{}f2}\PYG{p}{,}\PYG{n}{pft\PYGZus{}c2}\PYG{p}{,}\PYG{n}{pft\PYGZus{}f3}\PYG{p}{,}\PYG{n}{pft\PYGZus{}c3}\PYG{p}{,}\PYG{n}{pft\PYGZus{}f4}\PYG{p}{,}\PYG{n}{pft\PYGZus{}c4}\PYG{p}{,}\PYG{n}{pft\PYGZus{}f5}\PYG{p}{,}\PYG{n}{pft\PYGZus{}c5}\PYG{p}{,}\PYG{n}{har\PYGZus{}vh1}\PYG{p}{,}\PYG{n}{har\PYGZus{}vh2}\PYG{p}{,}\PYG{n}{har\PYGZus{}sh1}\PYG{p}{,}\PYG{n}{har\PYGZus{}sh2}\PYG{p}{,}\PYG{n}{har\PYGZus{}sh3}\PYG{p}{,}\PYG{n}{graze}\PYG{p}{,}\PYG{n}{hold\PYGZus{}harv}\PYG{p}{,}\PYG{n}{hold\PYGZus{}graze}
\end{sphinxVerbatim}

This file only requires a line for each year where a transition or harvest happens. As in the “pftdata” file above “pft\_f” refers to the fraction and “pft\_c” refers to the PFT index, and only up to five vegetation types are allowed to co-exist. The last eight columns have to do with harvesting and grazing. The last two columns are whether to hold harvesting and/or grazing constant until the next transition year and will just be either 1 or 0. This file will be converted by the \sphinxstylestrong{PTCLM\_sitedata/cnvrt\_trnsyrs2\_pftdyntxtfile.pl} script in the PTCLM directory to a format that \sphinxstylestrong{mksurfdata\_map} can read that has an entry for each year for the range of years valid for the compset in question.


\subsubsection{Converting AmeriFlux Data for use by PTCLM}
\label{\detokenize{users_guide/running-PTCLM/adding-ptclm-site-data:converting-ameriflux-data-for-use-by-ptclm}}
AmeriFlux data comes in comma separated format and is available from:
\sphinxurl{http://public.ornl.gov/ameriflux/dataproducts.shtml}. Before you download the data you need to agree to the usage terms.

Here is a copy of the usage terms from the web-site on June/13/2011.

“The AmeriFlux data provided on this site are freely available and were furnished by individual AmeriFlux scientists who encourage their use.
Please kindly inform the appropriate AmeriFlux scientist(s) of how you are using the data and of any publication plans.
Please acknowledge the data source as a citation or in the acknowledgments if the data are not yet published.
If the AmeriFlux Principal Investigators (PIs) feel that they should be acknowledged or offered participation as authors, they will let you know and we assume that an agreement on such matters will be reached before publishing and/or use of the data for publication.
If your work directly competes with the PI’s analysis they may ask that they have the opportunity to submit a manuscript before you submit one that uses unpublished data.
In addition, when publishing, please acknowledge the agency that supported the research.
Lastly, we kindly request that those publishing papers using AmeriFlux data provide preprints to the PIs providing the data and to the data archive at the Carbon Dioxide Information Analysis Center (CDIAC).”

The above agreement applies to the “US-UMB” dataset imported into our repository as well, and Gil Bohrer is the PI on record for that dataset.

The CESM can NOT handle missing data, so we recommend using the “Level 4” Gap filled datasets.
The fields will also need to be renamed.
The “WS” column becomes “WIND”, “PREC” becomes “PRECmms”, “RH” stays as “RH”, “TA” becomes “TBOT”, “Rg” becomes “FSDS”, “Rgl” becomes “FLDS”, “PRESS” becomes “PSRF”.
“ZBOT” can just be set to the constant of “30” (m).
The units of Temperature need to be converted from “Celsius” to “Kelvin” (use the value in \sphinxcode{SHR\_CONST\_TKFRZ} in the file \sphinxcode{models/csm\_share/shr/shr\_const.F90} of \sphinxcode{273.15}.
The units of Pressure also need to be converted from “kPa” to “Pa”. LATIXY, and LONGXY should also be set to the latitude and longitude of the site.


\subsubsection{Example: PTCLM transient example over a shorter time period}
\label{\detokenize{users_guide/running-PTCLM/adding-ptclm-site-data:example-ptclm-transient-example-over-a-shorter-time-period}}
This is an example of using PTCLM for Harvard Forest (AmeriFlux site code US-Ha1) for transient land use 1991-2006.
In order to do this we would’ve needed to have converted the AmeriFlux data into NetCDF format as show in the \sphinxhref{CLM-URL}{the Section called Converting AmeriFlux Data for use by PTCLM} section above.
Also note that this site has a site-specific dynamic land-use change file for it \sphinxcode{PTCLM\_sitedata/US-Ha1\_dynpftdata.txt} in the PTCLM directory and this file will be used for land-use change and harvesting rather than the global dataset.

\begin{sphinxVerbatim}[commandchars=\\\{\}]
\PYGZgt{} cd scripts/ccsm\PYGZus{}utils/Tools/lnd/clm/PTCLM
\PYGZsh{} We are going to use forcing data over 1991 to 2006, but we need to start with
\PYGZsh{} a transient compset to do so, so we use the 20th Century transient: 1850\PYGZhy{}2000
\PYGZsh{} Note: When creating the fpftdyn dataset for this site it will use the
\PYGZsh{}     PTCLM\PYGZus{}sitedata/US\PYGZhy{}Ha1\PYGZus{}dynpftdata.txt
\PYGZsh{} file for land\PYGZhy{}use change and harvesting
\PYGZgt{} ./PTCLM.py \PYGZhy{}m yellowstone\PYGZus{}intel \PYGZhy{}s US\PYGZhy{}Ha1 \PYGZhy{}d \PYGZdl{}MYCSMDATA \PYGZhy{}\PYGZhy{}sitegroupname AmeriFlux \PYGZhy{}c I20TRCRUCLM45BGC
\PYGZgt{} mkdir \PYGZdl{}MYCSMDATA/atm/datm7/CLM1PT\PYGZus{}data/1x1pt\PYGZus{}US\PYGZhy{}Ha1
\PYGZgt{} cd \PYGZdl{}MYCSMDATA/atm/datm7/CLM1PT\PYGZus{}data/1x1pt\PYGZus{}US\PYGZhy{}Ha1
\PYGZsh{} Copy data in NetCDF format to this directory, filenames should be YYYY\PYGZhy{}MM.nc
\PYGZsh{} The fieldnames on the file should be:
\PYGZsh{}    FLDS,FSDS,LATIXY,   LONGXY,   PRECTmms,PSRF,RH,TBOT,WIND,ZBOT
\PYGZsh{} With units
\PYGZsh{}    W/m2,W/m2,degrees\PYGZus{}N,degrees\PYGZus{}E,mm/s,    Pa,  \PYGZpc{}, K,   m/s, m
\PYGZsh{} The time coordinate units should be: days since YYYY\PYGZhy{}MM\PYGZhy{}DD 00:00:00
\PYGZgt{} cd ../../../../../US\PYGZhy{}Ha1\PYGZus{}I20TRCRUCLM45BGC
\PYGZsh{} Now we need to set the start date to 1991, and make sure the align year is for 1991
\PYGZgt{} ./xmlchange RUN\PYGZus{}STARTDATE=1991\PYGZhy{}01\PYGZhy{}01,DATM\PYGZus{}CLMNCEP\PYGZus{}YR\PYGZus{}ALIGN=1991
\PYGZsh{} Similarly for Nitrogen deposition data we cycle over: 1991 to 2006
\PYGZgt{} cat \PYGZlt{}\PYGZlt{} EOF \PYGZgt{}\PYGZgt{} user\PYGZus{}nl\PYGZus{}clm
model\PYGZus{}year\PYGZus{}align\PYGZus{}ndep=1991,stream\PYGZus{}year\PYGZus{}first\PYGZus{}ndep=1991,stream\PYGZus{}year\PYGZus{}last\PYGZus{}ndep=2006
EOF
\PYGZsh{} Now setup the case, and we\PYGZsq{}ll edit the datm namelist for prescribed aerosols
\PYGZgt{} ./cesm\PYGZus{}setup
\end{sphinxVerbatim}


\section{Troubleshooting}
\label{\detokenize{users_guide/trouble-shooting/index::doc}}\label{\detokenize{users_guide/trouble-shooting/index:troubleshooting}}\label{\detokenize{users_guide/trouble-shooting/index:id1}}

\subsection{Trouble Shooting}
\label{\detokenize{users_guide/trouble-shooting/trouble-shooting:trouble-shooting}}\label{\detokenize{users_guide/trouble-shooting/trouble-shooting::doc}}\label{\detokenize{users_guide/trouble-shooting/trouble-shooting:id1}}
In this chapter we give some guidance on what to do when you encounter some of the most common problems. We can’t cover all the problems that a user could potentially have, but we will try to help you recognize some of the most common situations. And we’ll give you some suggestions on how to approach the problem to come up with a solution.

In general you will run into one of three type of problems:
\begin{enumerate}
\item {} 
\sphinxstyleemphasis{setup-time}

\item {} 
\sphinxstyleemphasis{build-time}

\item {} 
\sphinxstyleemphasis{run-time}

\end{enumerate}


\subsubsection{Setup Problems}
\label{\detokenize{users_guide/trouble-shooting/trouble-shooting:setup-problems}}
The first type of problem happens when you invoke the \sphinxstylestrong{case.setup} command.
This indicates there is something wrong with your input datasets, or the details of what you are trying to setup the model to do.
There’s also a trouble-shooting chapter in the \sphinxhref{CLM-URL}{CESM1.2.0 Scripts User’s Guide}.
Many of the problems with configuration can be resolved with the guidelines given there.
Here we will restrict ourselves to problems from the input files.


\paragraph{Example: Missing datasets}
\label{\detokenize{users_guide/trouble-shooting/trouble-shooting:example-missing-datasets}}
\begin{sphinxVerbatim}[commandchars=\\\{\}]
\PYG{o}{\PYGZgt{}} \PYG{o}{.}\PYG{o}{/}\PYG{n}{create\PYGZus{}newcase} \PYG{o}{\PYGZhy{}}\PYG{n}{case} \PYG{n}{ne60rcp6} \PYG{o}{\PYGZhy{}}\PYG{n}{res} \PYG{n}{ne60\PYGZus{}g16} \PYG{o}{\PYGZhy{}}\PYG{n}{compset} \PYG{n}{IRCP60CN} \PYGZbs{}
\PYG{o}{\PYGZhy{}}\PYG{n}{mach} \PYG{n}{yellowstone\PYGZus{}intel}
\PYG{o}{\PYGZgt{}} \PYG{o}{.}\PYG{o}{/}\PYG{n}{case}\PYG{o}{.}\PYG{n}{setup}
\end{sphinxVerbatim}

The following is what is displayed to the screen.

\begin{sphinxVerbatim}[commandchars=\\\{\}]
.
.
.
Running preview\PYGZus{}namelist script
CLM configure done.
CLM adding use\PYGZus{}case 1850\PYGZhy{}2100\PYGZus{}rcp6\PYGZus{}transient defaults for var clm\PYGZus{}demand with val fpftdyn
CLM adding use\PYGZus{}case 1850\PYGZhy{}2100\PYGZus{}rcp6\PYGZus{}transient defaults for var clm\PYGZus{}start\PYGZus{}type with val startup
CLM adding use\PYGZus{}case 1850\PYGZhy{}2100\PYGZus{}rcp6\PYGZus{}transient defaults for var model\PYGZus{}year\PYGZus{}align\PYGZus{}ndep with val 1850
CLM adding use\PYGZus{}case 1850\PYGZhy{}2100\PYGZus{}rcp6\PYGZus{}transient defaults for var rcp with val 6
CLM adding use\PYGZus{}case 1850\PYGZhy{}2100\PYGZus{}rcp6\PYGZus{}transient defaults for var sim\PYGZus{}year with val 1850
CLM adding use\PYGZus{}case 1850\PYGZhy{}2100\PYGZus{}rcp6\PYGZus{}transient defaults for var sim\PYGZus{}year\PYGZus{}range with val 1850\PYGZhy{}2100
CLM adding use\PYGZus{}case 1850\PYGZhy{}2100\PYGZus{}rcp6\PYGZus{}transient defaults for var stream\PYGZus{}year\PYGZus{}first\PYGZus{}ndep with val 1850
CLM adding use\PYGZus{}case 1850\PYGZhy{}2100\PYGZus{}rcp6\PYGZus{}transient defaults for var stream\PYGZus{}year\PYGZus{}last\PYGZus{}ndep with val 2100
CLM adding use\PYGZus{}case 1850\PYGZhy{}2100\PYGZus{}rcp6\PYGZus{}transient defaults for var use\PYGZus{}case\PYGZus{}desc with val Simulate transient land\PYGZhy{}use, aerosol and Nitrogen deposition changes
with historical data from 1850 to 2005 and then with the RCP6 scenario from AIM

build\PYGZhy{}namelist \PYGZhy{} No default value found for fpftdyn.
            Are defaults provided for this resolution and land mask?
ERROR: clm.buildnml.csh failed
ERROR: /Users/erik/clm\PYGZus{}cesm1\PYGZus{}1\PYGZus{}1\PYGZus{}rel/scripts/ne60rcp6/preview\PYGZus{}namelists failed: 25344
\end{sphinxVerbatim}

The important thing to note here is the line:

\begin{sphinxVerbatim}[commandchars=\\\{\}]
\PYG{n}{ERROR}\PYG{p}{:} \PYG{n}{clm}\PYG{o}{.}\PYG{n}{buildnml}\PYG{o}{.}\PYG{n}{csh} \PYG{n}{failed}
\end{sphinxVerbatim}

which tells us that the problem is in the land \sphinxstylestrong{clm.buildnml.csh}. It may also indicate problems in one of the other buildnml.csh files (atm, cesm, cpl, glc, ice, or ocn), in which case you should consult the appropriate model user’s guide.

In the example, the error is that the CLM XML database does NOT have a \sphinxcode{finidat} for the given resolution, rcp scenario and ocean mask. That means you will need to create the file and then supply the file into your case. See \sphinxhref{CLM-URL}{Chapter 2} for more information on creating files, and see \sphinxhref{CLM-URL}{Chapter 3} for more information on adding files to the XML database. Alternatively, you can provide the file to your case by creating a user namelist as shown in \sphinxhref{CLM-URL}{the Section called User Namelist in Chapter 1}.

\begin{sphinxadmonition}{note}{Note:}
The two most common problems from your \sphinxstylestrong{clm.buildnml.csh} will be errors from the CLM \sphinxstylestrong{configure} or \sphinxstylestrong{build-namelist}. For more information on these scripts see: \sphinxhref{CLM-URL}{the Section called More information on the CLM configure script in Chapter 1} and \sphinxhref{CLM-URL}{the section on CLM\_BLDNML\_OPTS}.
\end{sphinxadmonition}


\subsubsection{Build problems}
\label{\detokenize{users_guide/trouble-shooting/trouble-shooting:build-problems}}
The following is an example of running the build for a case and having it fail in the land model build.
As you can see it lists which model component is being built and the build log for that component.

\begin{sphinxVerbatim}[commandchars=\\\{\}]
\PYG{n}{CCSM} \PYG{n}{BUILDEXE} \PYG{n}{SCRIPT} \PYG{n}{STARTING}
\PYG{o}{\PYGZhy{}} \PYG{n}{Build} \PYG{n}{Libraries}\PYG{p}{:} \PYG{n}{mct} \PYG{n}{pio} \PYG{n}{csm\PYGZus{}share}
  \PYG{n}{Sat} \PYG{n}{Jun} \PYG{l+m+mi}{19} \PYG{l+m+mi}{21}\PYG{p}{:}\PYG{l+m+mi}{21}\PYG{p}{:}\PYG{l+m+mi}{19} \PYG{n}{MDT} \PYG{l+m+mi}{2010} \PYG{o}{/}\PYG{n}{ptmp}\PYG{o}{/}\PYG{n}{erik}\PYG{o}{/}\PYG{n}{test\PYGZus{}build}\PYG{o}{/}\PYG{n}{mct}\PYG{o}{/}\PYG{n}{mct}\PYG{o}{.}\PYG{n}{bldlog}\PYG{o}{.}\PYG{l+m+mi}{100619}\PYG{o}{\PYGZhy{}}\PYG{l+m+mi}{212107}
  \PYG{n}{Sat} \PYG{n}{Jun} \PYG{l+m+mi}{19} \PYG{l+m+mi}{21}\PYG{p}{:}\PYG{l+m+mi}{22}\PYG{p}{:}\PYG{l+m+mi}{18} \PYG{n}{MDT} \PYG{l+m+mi}{2010} \PYG{o}{/}\PYG{n}{ptmp}\PYG{o}{/}\PYG{n}{erik}\PYG{o}{/}\PYG{n}{test\PYGZus{}build}\PYG{o}{/}\PYG{n}{pio}\PYG{o}{/}\PYG{n}{pio}\PYG{o}{.}\PYG{n}{bldlog}\PYG{o}{.}\PYG{l+m+mi}{100619}\PYG{o}{\PYGZhy{}}\PYG{l+m+mi}{212107}
  \PYG{n}{Sat} \PYG{n}{Jun} \PYG{l+m+mi}{19} \PYG{l+m+mi}{21}\PYG{p}{:}\PYG{l+m+mi}{23}\PYG{p}{:}\PYG{l+m+mi}{18} \PYG{n}{MDT} \PYG{l+m+mi}{2010}
  \PYG{o}{/}\PYG{n}{ptmp}\PYG{o}{/}\PYG{n}{erik}\PYG{o}{/}\PYG{n}{test\PYGZus{}build}\PYG{o}{/}\PYG{n}{csm\PYGZus{}share}\PYG{o}{/}\PYG{n}{csm\PYGZus{}share}\PYG{o}{.}\PYG{n}{bldlog}\PYG{o}{.}\PYG{l+m+mi}{100619}\PYG{o}{\PYGZhy{}}\PYG{l+m+mi}{212107}
  \PYG{n}{Sat} \PYG{n}{Jun} \PYG{l+m+mi}{19} \PYG{l+m+mi}{21}\PYG{p}{:}\PYG{l+m+mi}{24}\PYG{p}{:}\PYG{l+m+mi}{00} \PYG{n}{MDT} \PYG{l+m+mi}{2010} \PYG{o}{/}\PYG{n}{ptmp}\PYG{o}{/}\PYG{n}{erik}\PYG{o}{/}\PYG{n}{test\PYGZus{}build}\PYG{o}{/}\PYG{n}{run}\PYG{o}{/}\PYG{n}{cpl}\PYG{o}{.}\PYG{n}{bldlog}\PYG{o}{.}\PYG{l+m+mi}{100619}\PYG{o}{\PYGZhy{}}\PYG{l+m+mi}{212107}
  \PYG{n}{Sat} \PYG{n}{Jun} \PYG{l+m+mi}{19} \PYG{l+m+mi}{21}\PYG{p}{:}\PYG{l+m+mi}{24}\PYG{p}{:}\PYG{l+m+mi}{00} \PYG{n}{MDT} \PYG{l+m+mi}{2010} \PYG{o}{/}\PYG{n}{ptmp}\PYG{o}{/}\PYG{n}{erik}\PYG{o}{/}\PYG{n}{test\PYGZus{}build}\PYG{o}{/}\PYG{n}{run}\PYG{o}{/}\PYG{n}{atm}\PYG{o}{.}\PYG{n}{bldlog}\PYG{o}{.}\PYG{l+m+mi}{100619}\PYG{o}{\PYGZhy{}}\PYG{l+m+mi}{212107}
  \PYG{n}{Sat} \PYG{n}{Jun} \PYG{l+m+mi}{19} \PYG{l+m+mi}{21}\PYG{p}{:}\PYG{l+m+mi}{24}\PYG{p}{:}\PYG{l+m+mi}{06} \PYG{n}{MDT} \PYG{l+m+mi}{2010} \PYG{o}{/}\PYG{n}{ptmp}\PYG{o}{/}\PYG{n}{erik}\PYG{o}{/}\PYG{n}{test\PYGZus{}build}\PYG{o}{/}\PYG{n}{run}\PYG{o}{/}\PYG{n}{lnd}\PYG{o}{.}\PYG{n}{bldlog}\PYG{o}{.}\PYG{l+m+mi}{100619}\PYG{o}{\PYGZhy{}}\PYG{l+m+mi}{212107}
  \PYG{n}{ERROR}\PYG{p}{:} \PYG{n}{clm}\PYG{o}{.}\PYG{n}{buildexe}\PYG{o}{.}\PYG{n}{csh} \PYG{n}{failed}\PYG{p}{,} \PYG{n}{see} \PYG{o}{/}\PYG{n}{ptmp}\PYG{o}{/}\PYG{n}{erik}\PYG{o}{/}\PYG{n}{test\PYGZus{}build}\PYG{o}{/}\PYG{n}{run}\PYG{o}{/}\PYG{n}{lnd}\PYG{o}{.}\PYG{n}{bldlog}\PYG{o}{.}\PYG{l+m+mi}{100619}\PYG{o}{\PYGZhy{}}\PYG{l+m+mi}{212107}
  \PYG{n}{ERROR}\PYG{p}{:} \PYG{n}{cat} \PYG{o}{/}\PYG{n}{ptmp}\PYG{o}{/}\PYG{n}{erik}\PYG{o}{/}\PYG{n}{test\PYGZus{}build}\PYG{o}{/}\PYG{n}{run}\PYG{o}{/}\PYG{n}{lnd}\PYG{o}{.}\PYG{n}{bldlog}\PYG{o}{.}\PYG{l+m+mi}{100619}\PYG{o}{\PYGZhy{}}\PYG{l+m+mi}{212107}
\end{sphinxVerbatim}

You can then examine the build log that failed and see what went wrong. Most compilers will give the full filepath and line number for the file that filed to compile.


\subsubsection{Run Time Problems}
\label{\detokenize{users_guide/trouble-shooting/trouble-shooting:run-time-problems}}
Tracking down problems while the model is running is much more difficult to do than setup or build problems.
In this section we will give some suggestions on how to find run time problems.
Below we show the log file results of a job that aborted while running.

\begin{sphinxVerbatim}[commandchars=\\\{\}]
\PYG{n}{CCSM} \PYG{n}{PRESTAGE} \PYG{n}{SCRIPT} \PYG{n}{HAS} \PYG{n}{FINISHED} \PYG{n}{SUCCESSFULLY}
\PYG{n}{Sun} \PYG{n}{Jun} \PYG{l+m+mi}{20} \PYG{l+m+mi}{18}\PYG{p}{:}\PYG{l+m+mi}{24}\PYG{p}{:}\PYG{l+m+mi}{06} \PYG{n}{MDT} \PYG{l+m+mi}{2010} \PYG{o}{\PYGZhy{}}\PYG{o}{\PYGZhy{}} \PYG{n}{CSM} \PYG{n}{EXECUTION} \PYG{n}{BEGINS} \PYG{n}{HERE}
\PYG{n}{Sun} \PYG{n}{Jun} \PYG{l+m+mi}{20} \PYG{l+m+mi}{18}\PYG{p}{:}\PYG{l+m+mi}{24}\PYG{p}{:}\PYG{l+m+mi}{35} \PYG{n}{MDT} \PYG{l+m+mi}{2010} \PYG{o}{\PYGZhy{}}\PYG{o}{\PYGZhy{}} \PYG{n}{CSM} \PYG{n}{EXECUTION} \PYG{n}{HAS} \PYG{n}{FINISHED}
\PYG{n}{Model} \PYG{n}{did} \PYG{o+ow}{not} \PYG{n}{complete} \PYG{o}{\PYGZhy{}} \PYG{n}{see} \PYG{o}{/}\PYG{n}{ptmp}\PYG{o}{/}\PYG{n}{erik}\PYG{o}{/}\PYG{n}{test\PYGZus{}run}\PYG{o}{/}\PYG{n}{run}\PYG{o}{/}\PYG{n}{cpl}\PYG{o}{.}\PYG{n}{log}\PYG{o}{.}\PYG{l+m+mi}{100620}\PYG{o}{\PYGZhy{}}\PYG{l+m+mi}{182358}
\end{sphinxVerbatim}

In the next section we will talk about using the different log files to track down problems, and find out where the problem is coming from. In the section after that we give some general advice on debugging problems and some suggestions on ideas that may be helpful to track the problem down. Some of the examples below are from the \sphinxhref{CLM-URL}{models/lnd/clm/doc/KnownBugs} file.


\paragraph{Tracking Problems by Querying Log Files}
\label{\detokenize{users_guide/trouble-shooting/trouble-shooting:tracking-problems-by-querying-log-files}}
The first thing to do when tracking down problems is to query the different log files to see if you can discover where the problem occurs, and any error messages about it.
It’s important to figure out if the problem comes in at initialization or in the run phase of the model, and in which model component the problem happens.
There are different log files for the different major components, and they all end with the date and time in YYMMDD-HHMMSS format (2-digit: year, month, day, hour minute and second).
When the model runs to completion the log files will be copied to the logs directory in the script directory, but when the model fails they will remain in the run directory.
Here’s an example list of log files from an “I” case where the model dies in the land model initialization.
For “I” cases the sea-ice and ocean components are just stubs and don’t create log files (and unless running with the active land-ice model “glc” log files won’t be created either).

\begin{sphinxVerbatim}[commandchars=\\\{\}]
\PYG{n}{atm}\PYG{o}{.}\PYG{n}{log}\PYG{o}{.}\PYG{l+m+mi}{100620}\PYG{o}{\PYGZhy{}}\PYG{l+m+mi}{182358}
\PYG{n}{cesm}\PYG{o}{.}\PYG{n}{log}\PYG{o}{.}\PYG{l+m+mi}{100620}\PYG{o}{\PYGZhy{}}\PYG{l+m+mi}{182358}
\PYG{n}{cpl}\PYG{o}{.}\PYG{n}{log}\PYG{o}{.}\PYG{l+m+mi}{100620}\PYG{o}{\PYGZhy{}}\PYG{l+m+mi}{182358}
\PYG{n}{lnd}\PYG{o}{.}\PYG{n}{log}\PYG{o}{.}\PYG{l+m+mi}{100620}\PYG{o}{\PYGZhy{}}\PYG{l+m+mi}{182358}
\end{sphinxVerbatim}


\paragraph{The coupler log file}
\label{\detokenize{users_guide/trouble-shooting/trouble-shooting:the-coupler-log-file}}
The first log file to check is the coupler log file so that you can see where the model dies and which model component it fails in. When the model dies at initialization the last model component listed is the component that failed.

Example of a case that fails in the CLM land model initialization.

\begin{sphinxVerbatim}[commandchars=\\\{\}]
\PYG{p}{(}\PYG{n}{seq\PYGZus{}timemgr\PYGZus{}clockPrint}\PYG{p}{)}     \PYG{n}{Prev} \PYG{n}{Time}   \PYG{o}{=} \PYG{l+m+mi}{00001201}   \PYG{l+m+mi}{00000}
\PYG{p}{(}\PYG{n}{seq\PYGZus{}timemgr\PYGZus{}clockPrint}\PYG{p}{)}     \PYG{n}{Next} \PYG{n}{Time}   \PYG{o}{=} \PYG{l+m+mi}{99991201}   \PYG{l+m+mi}{00000}
\PYG{p}{(}\PYG{n}{seq\PYGZus{}timemgr\PYGZus{}clockPrint}\PYG{p}{)}     \PYG{n}{Intervl} \PYG{n}{yms} \PYG{o}{=}     \PYG{l+m+mi}{9999}       \PYG{l+m+mi}{0}           \PYG{l+m+mi}{0}

\PYG{p}{(}\PYG{n}{seq\PYGZus{}mct\PYGZus{}drv}\PYG{p}{)} \PYG{p}{:} \PYG{n}{Initialize} \PYG{n}{each} \PYG{n}{component}\PYG{p}{:} \PYG{n}{atm}\PYG{p}{,} \PYG{n}{lnd}\PYG{p}{,} \PYG{n}{ocn}\PYG{p}{,} \PYG{o+ow}{and} \PYG{n}{ice}
\PYG{p}{(}\PYG{n}{seq\PYGZus{}mct\PYGZus{}drv}\PYG{p}{)} \PYG{p}{:} \PYG{n}{Initialize} \PYG{n}{atm} \PYG{n}{component}
\PYG{p}{(}\PYG{n}{seq\PYGZus{}mct\PYGZus{}drv}\PYG{p}{)} \PYG{p}{:} \PYG{n}{Initialize} \PYG{n}{lnd} \PYG{n}{component}
\end{sphinxVerbatim}


\paragraph{The cesm log file}
\label{\detokenize{users_guide/trouble-shooting/trouble-shooting:the-cesm-log-file}}
The cesm log files are to some extent the “garbage collection” of log output.
The CLM sends it’s output from it’s master processor, but sends other output and possibly errors to the cesm log file.
Because, of this, often error messages are somewhere in the cesm log file.
However, since there is so much other output it may be difficult to find.
For example, here is some output from an older version of CESM (CESM1.0.2) where the RTM river routing file (before it was converted to NetCDF) was not provided and the error on the open statement for the file was embedded near the end of the cesm log file.

\begin{sphinxVerbatim}[commandchars=\\\{\}]
\PYG{n}{NODE}\PYG{c+c1}{\PYGZsh{}  NAME}
\PYG{p}{(}    \PYG{l+m+mi}{0}\PYG{p}{)}  \PYG{n}{be1105en}\PYG{o}{.}\PYG{n}{ucar}\PYG{o}{.}\PYG{n}{edu}
\PYG{l+s+s2}{\PYGZdq{}}\PYG{l+s+s2}{/gpfs/proj2/fis/cgd/home/erik/clm\PYGZus{}trunk/models/lnd/clm/src/riverroute/RtmMod.F90}\PYG{l+s+s2}{\PYGZdq{}}\PYG{p}{,} \PYG{n}{line}
\PYG{l+m+mi}{239}\PYG{p}{:} \PYG{l+m+mi}{1525}\PYG{o}{\PYGZhy{}}\PYG{l+m+mi}{155} \PYG{n}{The} \PYG{n}{file} \PYG{n}{name} \PYG{n}{provided} \PYG{o+ow}{in} \PYG{n}{the} \PYG{n}{OPEN} \PYG{n}{statement} \PYG{k}{for} \PYG{n}{unit} \PYG{l+m+mi}{1} \PYG{n}{has} \PYG{n}{zero} \PYG{n}{length} \PYG{o+ow}{or}
\PYG{n}{contains} \PYG{n+nb}{all} \PYG{n}{blanks}\PYG{o}{.}  \PYG{n}{The} \PYG{n}{program} \PYG{n}{will} \PYG{n}{recover} \PYG{n}{by} \PYG{n}{ignoring} \PYG{n}{the} \PYG{n}{OPEN} \PYG{n}{statement}\PYG{o}{.}
\PYG{l+s+s2}{\PYGZdq{}}\PYG{l+s+s2}{/gpfs/proj2/fis/cgd/home/erik/clm\PYGZus{}trunk/models/lnd/clm/src/riverroute/RtmMod.F90}\PYG{l+s+s2}{\PYGZdq{}}\PYG{p}{,} \PYG{n}{line}
\PYG{l+m+mi}{241}\PYG{p}{:} \PYG{l+m+mi}{1525}\PYG{o}{\PYGZhy{}}\PYG{l+m+mi}{001} \PYG{n}{The} \PYG{n}{READ} \PYG{n}{statement} \PYG{n}{on} \PYG{n}{the} \PYG{n}{file} \PYG{n}{fort}\PYG{o}{.}\PYG{l+m+mi}{1} \PYG{n}{cannot} \PYG{n}{be} \PYG{n}{completed} \PYG{n}{because} \PYG{n}{the} \PYG{n}{end}
\PYG{n}{of} \PYG{n}{the} \PYG{n}{file} \PYG{n}{was} \PYG{n}{reached}\PYG{o}{.}  \PYG{n}{The} \PYG{n}{program} \PYG{n}{will} \PYG{n}{stop}\PYG{o}{.}

\PYG{n}{Running}\PYG{p}{:} \PYG{o}{.}\PYG{o}{/}\PYG{n}{cesm}\PYG{o}{.}\PYG{n}{exe}
\PYG{n}{Please} \PYG{n}{wait}\PYG{o}{.}\PYG{o}{.}\PYG{o}{.}

\PYG{n}{Memory} \PYG{n}{usage} \PYG{k}{for}   \PYG{o}{.}\PYG{o}{/}\PYG{n}{cesm}\PYG{o}{.}\PYG{n}{exe} \PYG{p}{(}\PYG{n}{task} \PYG{c+c1}{\PYGZsh{}   0) is:      51696 KB. Exit status: 1. Signal: 0}
\end{sphinxVerbatim}

Although the example is from an earlier version of the model it still serves to illustrate finding problems from the cesm log file.

When working with the cesm log file, for a run-time problem, you will need to be able to separate it’s output into three categories: pre-crash, crash, and post-crash.
The pre-crash section is everything that is normal output for good operation of the model.
The crash section is the section where the model dies and reports on the actual problem.
the post-crash section is the cleanup and finalization after the model dies.
The most important part of this of course is the crash section.
The tricky part is distinguishing it from the other sections.
Also because the cesm log file most likely has duplicated output from multiple processors it is even more difficult to distinguish the different sections and to some extent the sections may be intertwined, as different processors reach the different sections at different times.
Because, of this reducing the number of processors for your simulation may help you sort out the output in the file (see \sphinxhref{CLM-URL}{the Section called Run with a smaller set of processors}).
Also much of the output from the cesm log file are system level information having to do with MPI multiprocessing.
Usually you can ignore this information, but it makes it more difficult to trudge through.

Sometimes the cesm log file is the ONLY file available, because the model terminates early in initialization.
In this case understanding the output in the cesm log file becomes even more important.
This also indicates the model did NOT advance far enough to reach the initialization of the individual model components.
This may mean that the initialization of the multiprocessing for MPI and/or OpenMP failed, or that the reading of the driver namelist file “drv\_in” failed.

Here we show those three sections for a cesm log file where a two task job failed on reading the namelist file.
For a typical job with many tasks similar sections of this will be repeated not just twice but for each task and hence make it harder to read.

\sphinxstyleemphasis{Pre-crash section of the cesm log file}

\begin{sphinxVerbatim}[commandchars=\\\{\}]
\PYG{n}{ATTENTION}\PYG{p}{:} \PYG{l+m+mi}{0031}\PYG{o}{\PYGZhy{}}\PYG{l+m+mi}{386}  \PYG{n}{MP\PYGZus{}INSTANCES} \PYG{n}{setting} \PYG{n}{ignored} \PYG{n}{when} \PYG{n}{LoadLeveler} \PYG{o+ow}{is} \PYG{o+ow}{not} \PYG{n}{being} \PYG{n}{used}\PYG{o}{.}

\PYG{n}{ATTENTION}\PYG{p}{:} \PYG{l+m+mi}{0031}\PYG{o}{\PYGZhy{}}\PYG{l+m+mi}{386}  \PYG{n}{MP\PYGZus{}INSTANCES} \PYG{n}{setting} \PYG{n}{ignored} \PYG{n}{when} \PYG{n}{LoadLeveler} \PYG{o+ow}{is} \PYG{o+ow}{not} \PYG{n}{being} \PYG{n}{used}\PYG{o}{.}
\PYG{n}{ATTENTION}\PYG{p}{:} \PYG{l+m+mi}{0031}\PYG{o}{\PYGZhy{}}\PYG{l+m+mi}{378} \PYG{n}{MP\PYGZus{}EUIDEVICE} \PYG{n}{setting} \PYG{n}{ignored} \PYG{n}{when} \PYG{n}{LoadLeveler} \PYG{o+ow}{is} \PYG{o+ow}{not} \PYG{n}{being} \PYG{n}{used}\PYG{o}{.}
\PYG{n}{ATTENTION}\PYG{p}{:} \PYG{l+m+mi}{0031}\PYG{o}{\PYGZhy{}}\PYG{l+m+mi}{386}  \PYG{n}{MP\PYGZus{}INSTANCES} \PYG{n}{setting} \PYG{n}{ignored} \PYG{n}{when} \PYG{n}{LoadLeveler} \PYG{o+ow}{is} \PYG{o+ow}{not} \PYG{n}{being} \PYG{n}{used}\PYG{o}{.}
   \PYG{l+m+mi}{0}\PYG{p}{:}\PYG{n}{INFO}\PYG{p}{:} \PYG{l+m+mi}{0031}\PYG{o}{\PYGZhy{}}\PYG{l+m+mi}{724}  \PYG{n}{Executing} \PYG{n}{program}\PYG{p}{:} \PYG{o}{\PYGZlt{}}\PYG{o}{/}\PYG{n}{usr}\PYG{o}{/}\PYG{n}{local}\PYG{o}{/}\PYG{n}{lsf}\PYG{o}{/}\PYG{l+m+mf}{7.0}\PYG{o}{/}\PYG{n}{aix5}\PYG{o}{\PYGZhy{}}\PYG{l+m+mi}{64}\PYG{o}{/}\PYG{n+nb}{bin}\PYG{o}{/}\PYG{n}{lsnrt\PYGZus{}run}\PYG{o}{\PYGZgt{}}
   \PYG{l+m+mi}{1}\PYG{p}{:}\PYG{n}{INFO}\PYG{p}{:} \PYG{l+m+mi}{0031}\PYG{o}{\PYGZhy{}}\PYG{l+m+mi}{724}  \PYG{n}{Executing} \PYG{n}{program}\PYG{p}{:} \PYG{o}{\PYGZlt{}}\PYG{o}{/}\PYG{n}{usr}\PYG{o}{/}\PYG{n}{local}\PYG{o}{/}\PYG{n}{lsf}\PYG{o}{/}\PYG{l+m+mf}{7.0}\PYG{o}{/}\PYG{n}{aix5}\PYG{o}{\PYGZhy{}}\PYG{l+m+mi}{64}\PYG{o}{/}\PYG{n+nb}{bin}\PYG{o}{/}\PYG{n}{lsnrt\PYGZus{}run}\PYG{o}{\PYGZgt{}}
   \PYG{l+m+mi}{0}\PYG{p}{:}\PYG{o}{/}\PYG{n}{contrib}\PYG{o}{/}\PYG{n+nb}{bin}\PYG{o}{/}\PYG{n}{cesm\PYGZus{}launch}\PYG{p}{:} \PYG{n}{process} \PYG{l+m+mi}{401894} \PYG{n}{bound} \PYG{n}{to} \PYG{n}{logical} \PYG{n}{CPU} \PYG{l+m+mi}{0} \PYG{n}{on} \PYG{n}{host} \PYG{n}{be0310en}\PYG{o}{.}\PYG{n}{ucar}\PYG{o}{.}\PYG{n}{edu} \PYG{o}{.}\PYG{o}{.}\PYG{o}{.}
   \PYG{l+m+mi}{1}\PYG{p}{:}\PYG{o}{/}\PYG{n}{contrib}\PYG{o}{/}\PYG{n+nb}{bin}\PYG{o}{/}\PYG{n}{cesm\PYGZus{}launch}\PYG{p}{:} \PYG{n}{process} \PYG{l+m+mi}{439264} \PYG{n}{bound} \PYG{n}{to} \PYG{n}{logical} \PYG{n}{CPU} \PYG{l+m+mi}{1} \PYG{n}{on} \PYG{n}{host} \PYG{n}{be0310en}\PYG{o}{.}\PYG{n}{ucar}\PYG{o}{.}\PYG{n}{edu} \PYG{o}{.}\PYG{o}{.}\PYG{o}{.}
   \PYG{l+m+mi}{0}\PYG{p}{:}\PYG{n}{INFO}\PYG{p}{:} \PYG{l+m+mi}{0031}\PYG{o}{\PYGZhy{}}\PYG{l+m+mi}{619}  \PYG{l+m+mi}{64}\PYG{n}{bit}\PYG{p}{(}\PYG{n}{us}\PYG{p}{,} \PYG{n}{Packet} \PYG{n}{striping} \PYG{n}{on}\PYG{p}{)}  \PYG{n}{ppe\PYGZus{}rmas} \PYG{n}{MPCI\PYGZus{}MSG}\PYG{p}{:} \PYG{n}{MPI}\PYG{o}{/}\PYG{n}{MPCI} \PYG{n}{library} \PYG{n}{was} \PYG{n}{compiled} \PYG{n}{on}   \PYG{n}{Wed} \PYG{n}{Aug}  \PYG{l+m+mi}{5} \PYG{l+m+mi}{13}\PYG{p}{:}\PYG{l+m+mi}{36}\PYG{p}{:}\PYG{l+m+mi}{06} \PYG{l+m+mi}{2009}
   \PYG{l+m+mi}{0}\PYG{p}{:}
   \PYG{l+m+mi}{1}\PYG{p}{:}\PYG{n}{LAPI} \PYG{n}{version} \PYG{c+c1}{\PYGZsh{}14.26 2008/11/23 11:02:30 1.296 src/rsct/lapi/lapi.c, lapi, rsct\PYGZus{}rpt53, rpt53s004a 09/04/29 64bit(us)  library compiled on Wed Apr 29 15:30:42 2009}
   \PYG{l+m+mi}{1}\PYG{p}{:}\PYG{o}{.}
   \PYG{l+m+mi}{1}\PYG{p}{:}\PYG{n}{LAPI} \PYG{o+ow}{is} \PYG{n}{using} \PYG{n}{lightweight} \PYG{n}{lock}\PYG{o}{.}
   \PYG{l+m+mi}{0}\PYG{p}{:}\PYG{n}{LAPI} \PYG{n}{version} \PYG{c+c1}{\PYGZsh{}14.26 2008/11/23 11:02:30 1.296 src/rsct/lapi/lapi.c, lapi, rsct\PYGZus{}rpt53, rpt53s004a 09/04/29 64bit(us)  library compiled on Wed Apr 29 15:30:42 2009}
   \PYG{l+m+mi}{0}\PYG{p}{:}\PYG{o}{.}
   \PYG{l+m+mi}{0}\PYG{p}{:}\PYG{n}{LAPI} \PYG{o+ow}{is} \PYG{n}{using} \PYG{n}{lightweight} \PYG{n}{lock}\PYG{o}{.}
   \PYG{l+m+mi}{0}\PYG{p}{:}\PYG{n}{Use} \PYG{n}{health} \PYG{n}{ping} \PYG{k}{for} \PYG{n}{failover}\PYG{o}{/}\PYG{n}{recovery}
   \PYG{l+m+mi}{1}\PYG{p}{:}\PYG{n}{Use} \PYG{n}{health} \PYG{n}{ping} \PYG{k}{for} \PYG{n}{failover}\PYG{o}{/}\PYG{n}{recovery}
   \PYG{l+m+mi}{0}\PYG{p}{:}\PYG{n}{Initial} \PYG{n}{communication} \PYG{n}{over} \PYG{n}{instance} \PYG{l+m+mf}{2.}
   \PYG{l+m+mi}{1}\PYG{p}{:}\PYG{n}{Initial} \PYG{n}{communication} \PYG{n}{over} \PYG{n}{instance} \PYG{l+m+mf}{0.}
   \PYG{l+m+mi}{1}\PYG{p}{:}\PYG{n}{IB} \PYG{n}{RDMA} \PYG{n}{initialization} \PYG{n}{completed} \PYG{n}{successfully}
   \PYG{l+m+mi}{1}\PYG{p}{:}\PYG{n}{The} \PYG{n}{MPI} \PYG{n}{shared} \PYG{n}{memory} \PYG{n}{protocol} \PYG{o+ow}{is} \PYG{n}{used} \PYG{k}{for} \PYG{n}{the} \PYG{n}{job}
   \PYG{l+m+mi}{0}\PYG{p}{:}\PYG{n}{IB} \PYG{n}{RDMA} \PYG{n}{initialization} \PYG{n}{completed} \PYG{n}{successfully}
   \PYG{l+m+mi}{0}\PYG{p}{:}\PYG{n}{LAPI} \PYG{n}{job} \PYG{n}{ID} \PYG{k}{for} \PYG{n}{this} \PYG{n}{job} \PYG{o+ow}{is}\PYG{p}{:} \PYG{l+m+mi}{1684890719}
   \PYG{l+m+mi}{0}\PYG{p}{:}\PYG{n}{The} \PYG{n}{MPI} \PYG{n}{shared} \PYG{n}{memory} \PYG{n}{protocol} \PYG{o+ow}{is} \PYG{n}{used} \PYG{k}{for} \PYG{n}{the} \PYG{n}{job}
   \PYG{l+m+mi}{0}\PYG{p}{:}\PYG{p}{(}\PYG{n}{seq\PYGZus{}comm\PYGZus{}setcomm}\PYG{p}{)}  \PYG{n}{initialize} \PYG{n}{ID} \PYG{p}{(}  \PYG{l+m+mi}{7} \PYG{n}{GLOBAL} \PYG{p}{)} \PYG{n}{pelist}   \PYG{o}{=}     \PYG{l+m+mi}{0}     \PYG{l+m+mi}{1}     \PYG{l+m+mi}{1} \PYG{p}{(} \PYG{n}{npes} \PYG{o}{=}     \PYG{l+m+mi}{2}\PYG{p}{)} \PYG{p}{(} \PYG{n}{nthreads} \PYG{o}{=}  \PYG{l+m+mi}{1}\PYG{p}{)}
   \PYG{l+m+mi}{0}\PYG{p}{:}\PYG{p}{(}\PYG{n}{seq\PYGZus{}comm\PYGZus{}setcomm}\PYG{p}{)}  \PYG{n}{initialize} \PYG{n}{ID} \PYG{p}{(}  \PYG{l+m+mi}{2}   \PYG{n}{ATM}  \PYG{p}{)} \PYG{n}{pelist}   \PYG{o}{=}     \PYG{l+m+mi}{0}     \PYG{l+m+mi}{1}     \PYG{l+m+mi}{1} \PYG{p}{(} \PYG{n}{npes} \PYG{o}{=}     \PYG{l+m+mi}{2}\PYG{p}{)} \PYG{p}{(} \PYG{n}{nthreads} \PYG{o}{=}  \PYG{l+m+mi}{1}\PYG{p}{)}
   \PYG{l+m+mi}{0}\PYG{p}{:}\PYG{p}{(}\PYG{n}{seq\PYGZus{}comm\PYGZus{}setcomm}\PYG{p}{)}  \PYG{n}{initialize} \PYG{n}{ID} \PYG{p}{(}  \PYG{l+m+mi}{1}   \PYG{n}{LND}  \PYG{p}{)} \PYG{n}{pelist}   \PYG{o}{=}     \PYG{l+m+mi}{0}     \PYG{l+m+mi}{1}     \PYG{l+m+mi}{1} \PYG{p}{(} \PYG{n}{npes} \PYG{o}{=}     \PYG{l+m+mi}{2}\PYG{p}{)} \PYG{p}{(} \PYG{n}{nthreads} \PYG{o}{=}  \PYG{l+m+mi}{1}\PYG{p}{)}
   \PYG{l+m+mi}{0}\PYG{p}{:}\PYG{p}{(}\PYG{n}{seq\PYGZus{}comm\PYGZus{}setcomm}\PYG{p}{)}  \PYG{n}{initialize} \PYG{n}{ID} \PYG{p}{(}  \PYG{l+m+mi}{4}   \PYG{n}{ICE}  \PYG{p}{)} \PYG{n}{pelist}   \PYG{o}{=}     \PYG{l+m+mi}{0}     \PYG{l+m+mi}{1}     \PYG{l+m+mi}{1} \PYG{p}{(} \PYG{n}{npes} \PYG{o}{=}     \PYG{l+m+mi}{2}\PYG{p}{)} \PYG{p}{(} \PYG{n}{nthreads} \PYG{o}{=}  \PYG{l+m+mi}{1}\PYG{p}{)}
   \PYG{l+m+mi}{0}\PYG{p}{:}\PYG{p}{(}\PYG{n}{seq\PYGZus{}comm\PYGZus{}setcomm}\PYG{p}{)}  \PYG{n}{initialize} \PYG{n}{ID} \PYG{p}{(}  \PYG{l+m+mi}{5}   \PYG{n}{GLC}  \PYG{p}{)} \PYG{n}{pelist}   \PYG{o}{=}     \PYG{l+m+mi}{0}     \PYG{l+m+mi}{1}     \PYG{l+m+mi}{1} \PYG{p}{(} \PYG{n}{npes} \PYG{o}{=}     \PYG{l+m+mi}{2}\PYG{p}{)} \PYG{p}{(} \PYG{n}{nthreads} \PYG{o}{=}  \PYG{l+m+mi}{1}\PYG{p}{)}
   \PYG{l+m+mi}{0}\PYG{p}{:}\PYG{p}{(}\PYG{n}{seq\PYGZus{}comm\PYGZus{}setcomm}\PYG{p}{)}  \PYG{n}{initialize} \PYG{n}{ID} \PYG{p}{(}  \PYG{l+m+mi}{3}   \PYG{n}{OCN}  \PYG{p}{)} \PYG{n}{pelist}   \PYG{o}{=}     \PYG{l+m+mi}{0}     \PYG{l+m+mi}{1}     \PYG{l+m+mi}{1} \PYG{p}{(} \PYG{n}{npes} \PYG{o}{=}     \PYG{l+m+mi}{2}\PYG{p}{)} \PYG{p}{(} \PYG{n}{nthreads} \PYG{o}{=}  \PYG{l+m+mi}{1}\PYG{p}{)}
   \PYG{l+m+mi}{0}\PYG{p}{:}\PYG{p}{(}\PYG{n}{seq\PYGZus{}comm\PYGZus{}setcomm}\PYG{p}{)}  \PYG{n}{initialize} \PYG{n}{ID} \PYG{p}{(}  \PYG{l+m+mi}{6}   \PYG{n}{CPL}  \PYG{p}{)} \PYG{n}{pelist}   \PYG{o}{=}     \PYG{l+m+mi}{0}     \PYG{l+m+mi}{1}     \PYG{l+m+mi}{1} \PYG{p}{(} \PYG{n}{npes} \PYG{o}{=}     \PYG{l+m+mi}{2}\PYG{p}{)} \PYG{p}{(} \PYG{n}{nthreads} \PYG{o}{=}  \PYG{l+m+mi}{1}\PYG{p}{)}
   \PYG{l+m+mi}{0}\PYG{p}{:}\PYG{p}{(}\PYG{n}{seq\PYGZus{}comm\PYGZus{}joincomm}\PYG{p}{)} \PYG{n}{initialize} \PYG{n}{ID} \PYG{p}{(}  \PYG{l+m+mi}{8} \PYG{n}{CPLATM} \PYG{p}{)} \PYG{n}{join} \PYG{n}{IDs} \PYG{o}{=}     \PYG{l+m+mi}{6}     \PYG{l+m+mi}{2}       \PYG{p}{(} \PYG{n}{npes} \PYG{o}{=}     \PYG{l+m+mi}{2}\PYG{p}{)} \PYG{p}{(} \PYG{n}{nthreads} \PYG{o}{=}  \PYG{l+m+mi}{1}\PYG{p}{)}
   \PYG{l+m+mi}{0}\PYG{p}{:}\PYG{p}{(}\PYG{n}{seq\PYGZus{}comm\PYGZus{}joincomm}\PYG{p}{)} \PYG{n}{initialize} \PYG{n}{ID} \PYG{p}{(}  \PYG{l+m+mi}{9} \PYG{n}{CPLLND} \PYG{p}{)} \PYG{n}{join} \PYG{n}{IDs} \PYG{o}{=}     \PYG{l+m+mi}{6}     \PYG{l+m+mi}{1}       \PYG{p}{(} \PYG{n}{npes} \PYG{o}{=}     \PYG{l+m+mi}{2}\PYG{p}{)} \PYG{p}{(} \PYG{n}{nthreads} \PYG{o}{=}  \PYG{l+m+mi}{1}\PYG{p}{)}
   \PYG{l+m+mi}{0}\PYG{p}{:}\PYG{p}{(}\PYG{n}{seq\PYGZus{}comm\PYGZus{}joincomm}\PYG{p}{)} \PYG{n}{initialize} \PYG{n}{ID} \PYG{p}{(} \PYG{l+m+mi}{10} \PYG{n}{CPLICE} \PYG{p}{)} \PYG{n}{join} \PYG{n}{IDs} \PYG{o}{=}     \PYG{l+m+mi}{6}     \PYG{l+m+mi}{4}       \PYG{p}{(} \PYG{n}{npes} \PYG{o}{=}     \PYG{l+m+mi}{2}\PYG{p}{)} \PYG{p}{(} \PYG{n}{nthreads} \PYG{o}{=}  \PYG{l+m+mi}{1}\PYG{p}{)}
   \PYG{l+m+mi}{0}\PYG{p}{:}\PYG{p}{(}\PYG{n}{seq\PYGZus{}comm\PYGZus{}joincomm}\PYG{p}{)} \PYG{n}{initialize} \PYG{n}{ID} \PYG{p}{(} \PYG{l+m+mi}{11} \PYG{n}{CPLOCN} \PYG{p}{)} \PYG{n}{join} \PYG{n}{IDs} \PYG{o}{=}     \PYG{l+m+mi}{6}     \PYG{l+m+mi}{3}       \PYG{p}{(} \PYG{n}{npes} \PYG{o}{=}     \PYG{l+m+mi}{2}\PYG{p}{)} \PYG{p}{(} \PYG{n}{nthreads} \PYG{o}{=}  \PYG{l+m+mi}{1}\PYG{p}{)}
   \PYG{l+m+mi}{0}\PYG{p}{:}\PYG{p}{(}\PYG{n}{seq\PYGZus{}comm\PYGZus{}joincomm}\PYG{p}{)} \PYG{n}{initialize} \PYG{n}{ID} \PYG{p}{(} \PYG{l+m+mi}{12} \PYG{n}{CPLGLC} \PYG{p}{)} \PYG{n}{join} \PYG{n}{IDs} \PYG{o}{=}     \PYG{l+m+mi}{6}     \PYG{l+m+mi}{5}       \PYG{p}{(} \PYG{n}{npes} \PYG{o}{=}     \PYG{l+m+mi}{2}\PYG{p}{)} \PYG{p}{(} \PYG{n}{nthreads} \PYG{o}{=}  \PYG{l+m+mi}{1}\PYG{p}{)}
   \PYG{l+m+mi}{0}\PYG{p}{:}
   \PYG{l+m+mi}{0}\PYG{p}{:} \PYG{p}{(}\PYG{n}{seq\PYGZus{}comm\PYGZus{}printcomms}\PYG{p}{)} \PYG{n}{ID} \PYG{n}{layout} \PYG{p}{:} \PYG{k}{global} \PYG{n}{pes} \PYG{n}{vs} \PYG{n}{local} \PYG{n}{pe} \PYG{k}{for} \PYG{n}{each} \PYG{n}{ID}
   \PYG{l+m+mi}{0}\PYG{p}{:}     \PYG{n}{gpe}        \PYG{n}{LND}      \PYG{n}{ATM}      \PYG{n}{OCN}      \PYG{n}{ICE}      \PYG{n}{GLC}      \PYG{n}{CPL}    \PYG{n}{GLOBAL}   \PYG{n}{CPLATM}   \PYG{n}{CPLLND}   \PYG{n}{CPLICE}   \PYG{n}{CPLOCN}   \PYG{n}{CPLGLC}    \PYG{n}{nthrds}
   \PYG{l+m+mi}{0}\PYG{p}{:}     \PYG{o}{\PYGZhy{}}\PYG{o}{\PYGZhy{}}\PYG{o}{\PYGZhy{}}      \PYG{o}{\PYGZhy{}}\PYG{o}{\PYGZhy{}}\PYG{o}{\PYGZhy{}}\PYG{o}{\PYGZhy{}}\PYG{o}{\PYGZhy{}}\PYG{o}{\PYGZhy{}}   \PYG{o}{\PYGZhy{}}\PYG{o}{\PYGZhy{}}\PYG{o}{\PYGZhy{}}\PYG{o}{\PYGZhy{}}\PYG{o}{\PYGZhy{}}\PYG{o}{\PYGZhy{}}   \PYG{o}{\PYGZhy{}}\PYG{o}{\PYGZhy{}}\PYG{o}{\PYGZhy{}}\PYG{o}{\PYGZhy{}}\PYG{o}{\PYGZhy{}}\PYG{o}{\PYGZhy{}}   \PYG{o}{\PYGZhy{}}\PYG{o}{\PYGZhy{}}\PYG{o}{\PYGZhy{}}\PYG{o}{\PYGZhy{}}\PYG{o}{\PYGZhy{}}\PYG{o}{\PYGZhy{}}   \PYG{o}{\PYGZhy{}}\PYG{o}{\PYGZhy{}}\PYG{o}{\PYGZhy{}}\PYG{o}{\PYGZhy{}}\PYG{o}{\PYGZhy{}}\PYG{o}{\PYGZhy{}}   \PYG{o}{\PYGZhy{}}\PYG{o}{\PYGZhy{}}\PYG{o}{\PYGZhy{}}\PYG{o}{\PYGZhy{}}\PYG{o}{\PYGZhy{}}\PYG{o}{\PYGZhy{}}   \PYG{o}{\PYGZhy{}}\PYG{o}{\PYGZhy{}}\PYG{o}{\PYGZhy{}}\PYG{o}{\PYGZhy{}}\PYG{o}{\PYGZhy{}}\PYG{o}{\PYGZhy{}}   \PYG{o}{\PYGZhy{}}\PYG{o}{\PYGZhy{}}\PYG{o}{\PYGZhy{}}\PYG{o}{\PYGZhy{}}\PYG{o}{\PYGZhy{}}\PYG{o}{\PYGZhy{}}   \PYG{o}{\PYGZhy{}}\PYG{o}{\PYGZhy{}}\PYG{o}{\PYGZhy{}}\PYG{o}{\PYGZhy{}}\PYG{o}{\PYGZhy{}}\PYG{o}{\PYGZhy{}}   \PYG{o}{\PYGZhy{}}\PYG{o}{\PYGZhy{}}\PYG{o}{\PYGZhy{}}\PYG{o}{\PYGZhy{}}\PYG{o}{\PYGZhy{}}\PYG{o}{\PYGZhy{}}   \PYG{o}{\PYGZhy{}}\PYG{o}{\PYGZhy{}}\PYG{o}{\PYGZhy{}}\PYG{o}{\PYGZhy{}}\PYG{o}{\PYGZhy{}}\PYG{o}{\PYGZhy{}}   \PYG{o}{\PYGZhy{}}\PYG{o}{\PYGZhy{}}\PYG{o}{\PYGZhy{}}\PYG{o}{\PYGZhy{}}\PYG{o}{\PYGZhy{}}\PYG{o}{\PYGZhy{}}    \PYG{o}{\PYGZhy{}}\PYG{o}{\PYGZhy{}}\PYG{o}{\PYGZhy{}}\PYG{o}{\PYGZhy{}}\PYG{o}{\PYGZhy{}}\PYG{o}{\PYGZhy{}}
   \PYG{l+m+mi}{0}\PYG{p}{:}       \PYG{l+m+mi}{0}   \PYG{p}{:}       \PYG{l+m+mi}{0}        \PYG{l+m+mi}{0}        \PYG{l+m+mi}{0}        \PYG{l+m+mi}{0}        \PYG{l+m+mi}{0}        \PYG{l+m+mi}{0}        \PYG{l+m+mi}{0}        \PYG{l+m+mi}{0}        \PYG{l+m+mi}{0}        \PYG{l+m+mi}{0}        \PYG{l+m+mi}{0}        \PYG{l+m+mi}{0}        \PYG{l+m+mi}{1}
   \PYG{l+m+mi}{1}\PYG{p}{:}       \PYG{l+m+mi}{1}   \PYG{p}{:}       \PYG{l+m+mi}{1}        \PYG{l+m+mi}{1}        \PYG{l+m+mi}{1}        \PYG{l+m+mi}{1}        \PYG{l+m+mi}{1}        \PYG{l+m+mi}{1}        \PYG{l+m+mi}{1}        \PYG{l+m+mi}{1}        \PYG{l+m+mi}{1}        \PYG{l+m+mi}{1}        \PYG{l+m+mi}{1}        \PYG{l+m+mi}{1}        \PYG{l+m+mi}{1}
   \PYG{l+m+mi}{1}\PYG{p}{:}
   \PYG{l+m+mi}{0}\PYG{p}{:} \PYG{p}{(}\PYG{n}{t\PYGZus{}initf}\PYG{p}{)} \PYG{n}{Read} \PYG{o+ow}{in} \PYG{n}{prof\PYGZus{}inparm} \PYG{n}{namelist} \PYG{n}{from}\PYG{p}{:} \PYG{n}{drv\PYGZus{}in}
   \PYG{l+m+mi}{1}\PYG{p}{:} \PYG{p}{(}\PYG{n}{seq\PYGZus{}io\PYGZus{}init}\PYG{p}{)} \PYG{n}{cpl\PYGZus{}io\PYGZus{}stride}\PYG{p}{,} \PYG{n}{iotasks} \PYG{o+ow}{or} \PYG{n}{root} \PYG{n}{out} \PYG{n}{of} \PYG{n}{bounds} \PYG{o}{\PYGZhy{}} \PYG{n}{resetting} \PYG{n}{to} \PYG{n}{defaults}  \PYG{l+m+mi}{4} \PYG{l+m+mi}{0} \PYG{l+m+mi}{1}
   \PYG{l+m+mi}{0}\PYG{p}{:} \PYG{n}{piolib\PYGZus{}mod}\PYG{o}{.}\PYG{n}{f90} \PYG{l+m+mi}{1353} \PYG{l+m+mi}{1} \PYG{l+m+mi}{2} \PYG{l+m+mi}{1} \PYG{l+m+mi}{2}
   \PYG{l+m+mi}{1}\PYG{p}{:} \PYG{n}{piolib\PYGZus{}mod}\PYG{o}{.}\PYG{n}{f90} \PYG{l+m+mi}{1353} \PYG{l+m+mi}{1} \PYG{l+m+mi}{2} \PYG{l+m+mi}{1} \PYG{l+m+mi}{2}
   \PYG{l+m+mi}{0}\PYG{p}{:} \PYG{n}{pio\PYGZus{}support}\PYG{p}{:}\PYG{p}{:}\PYG{n}{pio\PYGZus{}die}\PYG{p}{:}\PYG{p}{:} \PYG{n}{myrank}\PYG{o}{=} \PYG{l+m+mi}{0} \PYG{p}{:} \PYG{n}{ERROR}\PYG{p}{:} \PYG{n}{piolib\PYGZus{}mod}\PYG{o}{.}\PYG{n}{f90}\PYG{p}{:} \PYG{l+m+mi}{1354} \PYG{p}{:} \PYG{o+ow}{not} \PYG{n}{enough} \PYG{n}{procs} \PYG{k}{for} \PYG{n}{the} \PYG{n}{stride}
   \PYG{l+m+mi}{1}\PYG{p}{:} \PYG{n}{pio\PYGZus{}support}\PYG{p}{:}\PYG{p}{:}\PYG{n}{pio\PYGZus{}die}\PYG{p}{:}\PYG{p}{:} \PYG{n}{myrank}\PYG{o}{=} \PYG{l+m+mi}{1} \PYG{p}{:} \PYG{n}{ERROR}\PYG{p}{:} \PYG{n}{piolib\PYGZus{}mod}\PYG{o}{.}\PYG{n}{f90}\PYG{p}{:} \PYG{l+m+mi}{1354} \PYG{p}{:} \PYG{o+ow}{not} \PYG{n}{enough} \PYG{n}{procs} \PYG{k}{for} \PYG{n}{the} \PYG{n}{stride}
\end{sphinxVerbatim}

\sphinxstyleemphasis{Crash section of the cesm log file}

\begin{sphinxVerbatim}[commandchars=\\\{\}]
\PYG{l+m+mi}{0}\PYG{p}{:}
\PYG{l+m+mi}{0}\PYG{p}{:}  \PYG{n}{Traceback}\PYG{p}{:}
\PYG{l+m+mi}{1}\PYG{p}{:}
\PYG{l+m+mi}{1}\PYG{p}{:}  \PYG{n}{Traceback}\PYG{p}{:}
\PYG{l+m+mi}{0}\PYG{p}{:}    \PYG{n}{Offset} \PYG{l+m+mh}{0x00000c4c} \PYG{o+ow}{in} \PYG{n}{procedure} \PYG{n}{\PYGZus{}\PYGZus{}pio\PYGZus{}support\PYGZus{}NMOD\PYGZus{}piodie}\PYG{p}{,} \PYG{n}{near} \PYG{n}{line} \PYG{l+m+mi}{88} \PYG{o+ow}{in} \PYG{n}{file} \PYG{n}{pio\PYGZus{}support}\PYG{o}{.}\PYG{n}{F90}\PYG{o}{.}\PYG{o+ow}{in}
\PYG{l+m+mi}{1}\PYG{p}{:}    \PYG{n}{Offset} \PYG{l+m+mh}{0x00000c4c} \PYG{o+ow}{in} \PYG{n}{procedure} \PYG{n}{\PYGZus{}\PYGZus{}pio\PYGZus{}support\PYGZus{}NMOD\PYGZus{}piodie}\PYG{p}{,} \PYG{n}{near} \PYG{n}{line} \PYG{l+m+mi}{88} \PYG{o+ow}{in} \PYG{n}{file} \PYG{n}{pio\PYGZus{}support}\PYG{o}{.}\PYG{n}{F90}\PYG{o}{.}\PYG{o+ow}{in}
\PYG{l+m+mi}{0}\PYG{p}{:}    \PYG{n}{Offset} \PYG{l+m+mh}{0x00000fd0} \PYG{o+ow}{in} \PYG{n}{procedure} \PYG{n}{\PYGZus{}\PYGZus{}piolib\PYGZus{}mod\PYGZus{}NMOD\PYGZus{}init}\PYG{p}{,} \PYG{n}{near} \PYG{n}{line} \PYG{l+m+mi}{1354} \PYG{o+ow}{in} \PYG{n}{file} \PYG{n}{piolib\PYGZus{}mod}\PYG{o}{.}\PYG{n}{F90}
\PYG{l+m+mi}{1}\PYG{p}{:}    \PYG{n}{Offset} \PYG{l+m+mh}{0x00000fd0} \PYG{o+ow}{in} \PYG{n}{procedure} \PYG{n}{\PYGZus{}\PYGZus{}piolib\PYGZus{}mod\PYGZus{}NMOD\PYGZus{}init}\PYG{p}{,} \PYG{n}{near} \PYG{n}{line} \PYG{l+m+mi}{1354} \PYG{o+ow}{in} \PYG{n}{file} \PYG{n}{piolib\PYGZus{}mod}\PYG{o}{.}\PYG{n}{F90}
\PYG{l+m+mi}{1}\PYG{p}{:}    \PYG{n}{Offset} \PYG{l+m+mh}{0x00000398} \PYG{o+ow}{in} \PYG{n}{procedure} \PYG{n}{\PYGZus{}\PYGZus{}seq\PYGZus{}io\PYGZus{}mod\PYGZus{}NMOD\PYGZus{}seq\PYGZus{}io\PYGZus{}init}\PYG{p}{,} \PYG{n}{near} \PYG{n}{line} \PYG{l+m+mi}{247} \PYG{o+ow}{in} \PYG{n}{file} \PYG{o}{/}\PYG{n}{gpfs}\PYG{o}{/}\PYG{n}{proj2}\PYG{o}{/}\PYG{n}{fis}\PYG{o}{/}\PYG{n}{cgd}\PYG{o}{/}\PYG{n}{home}\PYG{o}{/}\PYG{n}{erik}\PYG{o}{/}\PYG{n}{clm\PYGZus{}trunk}\PYG{o}{/}\PYG{n}{models}\PYG{o}{/}\PYG{n}{drv}\PYG{o}{/}\PYG{n}{shr}\PYG{o}{/}\PYG{n}{seq\PYGZus{}io\PYGZus{}mod}\PYG{o}{.}\PYG{n}{F90}
\PYG{l+m+mi}{0}\PYG{p}{:}    \PYG{n}{Offset} \PYG{l+m+mh}{0x00000398} \PYG{o+ow}{in} \PYG{n}{procedure} \PYG{n}{\PYGZus{}\PYGZus{}seq\PYGZus{}io\PYGZus{}mod\PYGZus{}NMOD\PYGZus{}seq\PYGZus{}io\PYGZus{}init}\PYG{p}{,} \PYG{n}{near} \PYG{n}{line} \PYG{l+m+mi}{247} \PYG{o+ow}{in} \PYG{n}{file} \PYG{o}{/}\PYG{n}{gpfs}\PYG{o}{/}\PYG{n}{proj2}\PYG{o}{/}\PYG{n}{fis}\PYG{o}{/}\PYG{n}{cgd}\PYG{o}{/}\PYG{n}{home}\PYG{o}{/}\PYG{n}{erik}\PYG{o}{/}\PYG{n}{clm\PYGZus{}trunk}\PYG{o}{/}\PYG{n}{models}\PYG{o}{/}\PYG{n}{drv}\PYG{o}{/}\PYG{n}{shr}\PYG{o}{/}\PYG{n}{seq\PYGZus{}io\PYGZus{}mod}\PYG{o}{.}\PYG{n}{F90}
\PYG{l+m+mi}{0}\PYG{p}{:}    \PYG{n}{Offset} \PYG{l+m+mh}{0x0001aa88} \PYG{o+ow}{in} \PYG{n}{procedure} \PYG{n}{ccsm\PYGZus{}driver}\PYG{p}{,} \PYG{n}{near} \PYG{n}{line} \PYG{l+m+mi}{465} \PYG{o+ow}{in} \PYG{n}{file} \PYG{o}{/}\PYG{n}{gpfs}\PYG{o}{/}\PYG{n}{proj2}\PYG{o}{/}\PYG{n}{fis}\PYG{o}{/}\PYG{n}{cgd}\PYG{o}{/}\PYG{n}{home}\PYG{o}{/}\PYG{n}{erik}\PYG{o}{/}\PYG{n}{clm\PYGZus{}trunk}\PYG{o}{/}\PYG{n}{models}\PYG{o}{/}\PYG{n}{drv}\PYG{o}{/}\PYG{n}{driver}\PYG{o}{/}\PYG{n}{ccsm\PYGZus{}driver}\PYG{o}{.}\PYG{n}{F90}
\PYG{l+m+mi}{0}\PYG{p}{:}    \PYG{o}{\PYGZhy{}}\PYG{o}{\PYGZhy{}}\PYG{o}{\PYGZhy{}} \PYG{n}{End} \PYG{n}{of} \PYG{n}{call} \PYG{n}{chain} \PYG{o}{\PYGZhy{}}\PYG{o}{\PYGZhy{}}\PYG{o}{\PYGZhy{}}
\PYG{l+m+mi}{1}\PYG{p}{:}    \PYG{n}{Offset} \PYG{l+m+mh}{0x0001aa88} \PYG{o+ow}{in} \PYG{n}{procedure} \PYG{n}{ccsm\PYGZus{}driver}\PYG{p}{,} \PYG{n}{near} \PYG{n}{line} \PYG{l+m+mi}{465} \PYG{o+ow}{in} \PYG{n}{file} \PYG{o}{/}\PYG{n}{gpfs}\PYG{o}{/}\PYG{n}{proj2}\PYG{o}{/}\PYG{n}{fis}\PYG{o}{/}\PYG{n}{cgd}\PYG{o}{/}\PYG{n}{home}\PYG{o}{/}\PYG{n}{erik}\PYG{o}{/}\PYG{n}{clm\PYGZus{}trunk}\PYG{o}{/}\PYG{n}{models}\PYG{o}{/}\PYG{n}{drv}\PYG{o}{/}\PYG{n}{driver}\PYG{o}{/}\PYG{n}{ccsm\PYGZus{}driver}\PYG{o}{.}\PYG{n}{F90}
\PYG{l+m+mi}{1}\PYG{p}{:}    \PYG{o}{\PYGZhy{}}\PYG{o}{\PYGZhy{}}\PYG{o}{\PYGZhy{}} \PYG{n}{End} \PYG{n}{of} \PYG{n}{call} \PYG{n}{chain} \PYG{o}{\PYGZhy{}}\PYG{o}{\PYGZhy{}}\PYG{o}{\PYGZhy{}}
\end{sphinxVerbatim}

\sphinxstyleemphasis{Post-crash section of the cesm log file}

\begin{sphinxVerbatim}[commandchars=\\\{\}]
\PYG{l+m+mi}{1}\PYG{p}{:}\PYG{n}{Communication} \PYG{n}{statistics} \PYG{n}{of} \PYG{n}{task} \PYG{l+m+mi}{1} \PYG{o+ow}{is} \PYG{n}{associated} \PYG{k}{with} \PYG{n}{task} \PYG{n}{key}\PYG{p}{:} \PYG{l+m+mi}{1684890719}\PYG{n}{\PYGZus{}1}
\PYG{l+m+mi}{0}\PYG{p}{:}\PYG{n}{Communication} \PYG{n}{statistics} \PYG{n}{of} \PYG{n}{task} \PYG{l+m+mi}{0} \PYG{o+ow}{is} \PYG{n}{associated} \PYG{k}{with} \PYG{n}{task} \PYG{n}{key}\PYG{p}{:} \PYG{l+m+mi}{1684890719}\PYG{n}{\PYGZus{}0}
\PYG{l+m+mi}{0}\PYG{p}{:}
\PYG{l+m+mi}{0}\PYG{p}{:}\PYG{n}{Running}\PYG{p}{:} \PYG{o}{.}\PYG{o}{/}\PYG{n}{cesm}\PYG{o}{.}\PYG{n}{exe}
\PYG{l+m+mi}{0}\PYG{p}{:}\PYG{n}{Please} \PYG{n}{wait}\PYG{o}{.}\PYG{o}{.}\PYG{o}{.}
\PYG{l+m+mi}{0}\PYG{p}{:}
\PYG{l+m+mi}{0}\PYG{p}{:}\PYG{n}{Memory} \PYG{n}{usage} \PYG{k}{for}   \PYG{o}{.}\PYG{o}{/}\PYG{n}{cesm}\PYG{o}{.}\PYG{n}{exe} \PYG{p}{(}\PYG{n}{task} \PYG{c+c1}{\PYGZsh{}   0) is:     198892 KB. Exit status: 134. Signal: 0}
\PYG{l+m+mi}{1}\PYG{p}{:}
\PYG{l+m+mi}{1}\PYG{p}{:}\PYG{n}{Running}\PYG{p}{:} \PYG{o}{.}\PYG{o}{/}\PYG{n}{cesm}\PYG{o}{.}\PYG{n}{exe}
\PYG{l+m+mi}{1}\PYG{p}{:}\PYG{n}{Please} \PYG{n}{wait}\PYG{o}{.}\PYG{o}{.}\PYG{o}{.}
\PYG{l+m+mi}{1}\PYG{p}{:}
\PYG{l+m+mi}{1}\PYG{p}{:}\PYG{n}{Memory} \PYG{n}{usage} \PYG{k}{for}   \PYG{o}{.}\PYG{o}{/}\PYG{n}{cesm}\PYG{o}{.}\PYG{n}{exe} \PYG{p}{(}\PYG{n}{task} \PYG{c+c1}{\PYGZsh{}   0) is:     198572 KB. Exit status: 134. Signal: 0}
\PYG{n}{INFO}\PYG{p}{:} \PYG{l+m+mi}{0031}\PYG{o}{\PYGZhy{}}\PYG{l+m+mi}{656}  \PYG{n}{I}\PYG{o}{/}\PYG{n}{O} \PYG{n}{file} \PYG{n}{STDOUT} \PYG{n}{closed} \PYG{n}{by} \PYG{n}{task} \PYG{l+m+mi}{0}
\PYG{n}{INFO}\PYG{p}{:} \PYG{l+m+mi}{0031}\PYG{o}{\PYGZhy{}}\PYG{l+m+mi}{656}  \PYG{n}{I}\PYG{o}{/}\PYG{n}{O} \PYG{n}{file} \PYG{n}{STDERR} \PYG{n}{closed} \PYG{n}{by} \PYG{n}{task} \PYG{l+m+mi}{0}
\PYG{n}{ERROR}\PYG{p}{:} \PYG{l+m+mi}{0031}\PYG{o}{\PYGZhy{}}\PYG{l+m+mi}{250}  \PYG{n}{task} \PYG{l+m+mi}{0}\PYG{p}{:} \PYG{n}{IOT}\PYG{o}{/}\PYG{n}{Abort} \PYG{n}{trap}
\PYG{n}{INFO}\PYG{p}{:} \PYG{l+m+mi}{0031}\PYG{o}{\PYGZhy{}}\PYG{l+m+mi}{656}  \PYG{n}{I}\PYG{o}{/}\PYG{n}{O} \PYG{n}{file} \PYG{n}{STDOUT} \PYG{n}{closed} \PYG{n}{by} \PYG{n}{task} \PYG{l+m+mi}{1}
\PYG{n}{INFO}\PYG{p}{:} \PYG{l+m+mi}{0031}\PYG{o}{\PYGZhy{}}\PYG{l+m+mi}{656}  \PYG{n}{I}\PYG{o}{/}\PYG{n}{O} \PYG{n}{file} \PYG{n}{STDERR} \PYG{n}{closed} \PYG{n}{by} \PYG{n}{task} \PYG{l+m+mi}{1}
\PYG{n}{ERROR}\PYG{p}{:} \PYG{l+m+mi}{0031}\PYG{o}{\PYGZhy{}}\PYG{l+m+mi}{250}  \PYG{n}{task} \PYG{l+m+mi}{1}\PYG{p}{:} \PYG{n}{IOT}\PYG{o}{/}\PYG{n}{Abort} \PYG{n}{trap}
\PYG{n}{INFO}\PYG{p}{:} \PYG{l+m+mi}{0031}\PYG{o}{\PYGZhy{}}\PYG{l+m+mi}{639}  \PYG{n}{Exit} \PYG{n}{status} \PYG{k+kn}{from} \PYG{n+nn}{pm\PYGZus{}respond} \PYG{o}{=} \PYG{l+m+mi}{0}
\PYG{n}{ATTENTION}\PYG{p}{:} \PYG{l+m+mi}{0031}\PYG{o}{\PYGZhy{}}\PYG{l+m+mi}{386}  \PYG{n}{MP\PYGZus{}INSTANCES} \PYG{n}{setting} \PYG{n}{ignored} \PYG{n}{when} \PYG{n}{LoadLeveler} \PYG{o+ow}{is} \PYG{o+ow}{not} \PYG{n}{being} \PYG{n}{used}\PYG{o}{.}
\PYG{n}{Job}  \PYG{o}{/}\PYG{n}{usr}\PYG{o}{/}\PYG{n}{local}\PYG{o}{/}\PYG{n}{lsf}\PYG{o}{/}\PYG{l+m+mf}{7.0}\PYG{o}{/}\PYG{n}{aix5}\PYG{o}{\PYGZhy{}}\PYG{l+m+mi}{64}\PYG{o}{/}\PYG{n+nb}{bin}\PYG{o}{/}\PYG{n}{poejob} \PYG{o}{/}\PYG{n}{contrib}\PYG{o}{/}\PYG{n+nb}{bin}\PYG{o}{/}\PYG{n}{ccsm\PYGZus{}launch} \PYG{o}{/}\PYG{n}{contrib}\PYG{o}{/}\PYG{n+nb}{bin}\PYG{o}{/}\PYG{n}{job\PYGZus{}memusage}\PYG{o}{.}\PYG{n}{exe} \PYG{o}{.}\PYG{o}{/}\PYG{n}{cesm}\PYG{o}{.}\PYG{n}{exe}

\PYG{n}{TID}   \PYG{n}{HOST\PYGZus{}NAME}   \PYG{n}{COMMAND\PYGZus{}LINE}            \PYG{n}{STATUS}            \PYG{n}{TERMINATION\PYGZus{}TIME}
\PYG{o}{==}\PYG{o}{==}\PYG{o}{=} \PYG{o}{==}\PYG{o}{==}\PYG{o}{==}\PYG{o}{==}\PYG{o}{==} \PYG{o}{==}\PYG{o}{==}\PYG{o}{==}\PYG{o}{==}\PYG{o}{==}\PYG{o}{==}\PYG{o}{==}\PYG{o}{==}  \PYG{o}{==}\PYG{o}{==}\PYG{o}{==}\PYG{o}{==}\PYG{o}{==}\PYG{o}{==}\PYG{o}{==}\PYG{o}{==}\PYG{o}{==}\PYG{o}{==}\PYG{o}{==}\PYG{o}{=}  \PYG{o}{==}\PYG{o}{==}\PYG{o}{==}\PYG{o}{==}\PYG{o}{==}\PYG{o}{==}\PYG{o}{==}\PYG{o}{==}\PYG{o}{==}\PYG{o}{=}
\PYG{l+m+mi}{00000} \PYG{n}{be0310en}   \PYG{o}{/}\PYG{n}{contrib}\PYG{o}{/}\PYG{n+nb}{bin}\PYG{o}{/}\PYG{n}{ccs}  \PYG{n}{Exit} \PYG{p}{(}\PYG{l+m+mi}{134}\PYG{p}{)}               \PYG{l+m+mi}{08}\PYG{o}{/}\PYG{l+m+mi}{31}\PYG{o}{/}\PYG{l+m+mi}{2010} \PYG{l+m+mi}{12}\PYG{p}{:}\PYG{l+m+mi}{32}\PYG{p}{:}\PYG{l+m+mi}{57}
\PYG{l+m+mi}{00001} \PYG{n}{be0310en}   \PYG{o}{/}\PYG{n}{contrib}\PYG{o}{/}\PYG{n+nb}{bin}\PYG{o}{/}\PYG{n}{ccs}  \PYG{n}{Exit} \PYG{p}{(}\PYG{l+m+mi}{134}\PYG{p}{)}               \PYG{l+m+mi}{08}\PYG{o}{/}\PYG{l+m+mi}{31}\PYG{o}{/}\PYG{l+m+mi}{2010} \PYG{l+m+mi}{12}\PYG{p}{:}\PYG{l+m+mi}{32}\PYG{p}{:}\PYG{l+m+mi}{57}
\end{sphinxVerbatim}


\paragraph{The CLM log file}
\label{\detokenize{users_guide/trouble-shooting/trouble-shooting:the-clm-log-file}}
Of course when you are working with and making changes to CLM, most of your focus will be on the CLM log file and the errors it shows.
As already pointed out if you don’t see errors in the \sphinxcode{lnd.log.*} file you should look in the \sphinxcode{cesm.log.*} to see if any errors showed up there.

Here’s an example of the \sphinxcode{lnd.log.*} file when running \sphinxcode{PTS\_MODE} with initial conditions (this is bug 1025 in the \sphinxhref{CLM-URL}{models/lnd/clm/doc/KnownLimitationss} file).

\begin{sphinxVerbatim}[commandchars=\\\{\}]
\PYG{n}{Successfully} \PYG{n}{initialized} \PYG{n}{variables} \PYG{k}{for} \PYG{n}{accumulation}

\PYG{n}{reading} \PYG{n}{restart} \PYG{n}{file} \PYG{n}{I2000CN\PYGZus{}f09\PYGZus{}g16\PYGZus{}c100503}\PYG{o}{.}\PYG{n}{clm2}\PYG{o}{.}\PYG{n}{r}\PYG{o}{.}\PYG{l+m+mi}{0001}\PYG{o}{\PYGZhy{}}\PYG{l+m+mi}{01}\PYG{o}{\PYGZhy{}}\PYG{l+m+mi}{01}\PYG{o}{\PYGZhy{}}\PYG{l+m+mf}{00000.}\PYG{n}{nc}
\PYG{n}{Reading} \PYG{n}{restart} \PYG{n}{dataset}
\PYG{n}{ERROR} \PYG{o}{\PYGZhy{}} \PYG{n}{setlatlon}\PYG{o}{.}\PYG{n}{F}\PYG{p}{:}\PYG{n}{Cant} \PYG{n}{get} \PYG{n}{variable} \PYG{n}{dim} \PYG{k}{for} \PYG{n}{lat} \PYG{o+ow}{or} \PYG{n}{lsmlat}
\PYG{n}{ENDRUN}\PYG{p}{:} \PYG{n}{called} \PYG{n}{without} \PYG{n}{a} \PYG{n}{message} \PYG{n}{string}
\end{sphinxVerbatim}


\paragraph{The DATM log file}
\label{\detokenize{users_guide/trouble-shooting/trouble-shooting:the-datm-log-file}}
When working with “I cases” the second most common problems after CLM problems are problems with the data atmosphere model. So examining the \sphinxcode{atm.log.*} is important.

Here’s an example of a problem that occurs when the wrong prescribed aerosol file is given to a \sphinxcode{pt1\_pt1} simulation.

\begin{sphinxVerbatim}[commandchars=\\\{\}]
\PYG{p}{(}\PYG{n}{datm\PYGZus{}comp\PYGZus{}init}\PYG{p}{)}  \PYG{n}{atm} \PYG{n}{mode} \PYG{o}{=} \PYG{n}{CLMNCEP}
\PYG{p}{(}\PYG{n}{shr\PYGZus{}strdata\PYGZus{}init}\PYG{p}{)}  \PYG{n}{calling} \PYG{n}{shr\PYGZus{}dmodel\PYGZus{}mapSet} \PYG{k}{for} \PYG{n}{fill}
\PYG{p}{(}\PYG{n}{shr\PYGZus{}strdata\PYGZus{}init}\PYG{p}{)}  \PYG{n}{calling} \PYG{n}{shr\PYGZus{}dmodel\PYGZus{}mapSet} \PYG{k}{for} \PYG{n}{remap}
\PYG{p}{(}\PYG{l+s+s1}{\PYGZsq{}}\PYG{l+s+s1}{shr\PYGZus{}map\PYGZus{}getWts}\PYG{l+s+s1}{\PYGZsq{}}\PYG{p}{)} \PYG{n}{ERROR}\PYG{p}{:} \PYG{n}{yd} \PYG{n}{outside} \PYG{n}{bounds}  \PYG{l+m+mf}{19.5000000000000000}
\PYG{p}{(}\PYG{n}{shr\PYGZus{}sys\PYGZus{}abort}\PYG{p}{)} \PYG{n}{ERROR}\PYG{p}{:} \PYG{p}{(}\PYG{l+s+s1}{\PYGZsq{}}\PYG{l+s+s1}{shr\PYGZus{}map\PYGZus{}getWts}\PYG{l+s+s1}{\PYGZsq{}}\PYG{p}{)}  \PYG{n}{ERROR} \PYG{n}{yd} \PYG{n}{outside} \PYG{l+m+mi}{90} \PYG{n}{degree} \PYG{n}{bounds}
\PYG{p}{(}\PYG{n}{shr\PYGZus{}sys\PYGZus{}abort}\PYG{p}{)} \PYG{n}{WARNING}\PYG{p}{:} \PYG{n}{calling} \PYG{n}{shr\PYGZus{}mpi\PYGZus{}abort}\PYG{p}{(}\PYG{p}{)} \PYG{o+ow}{and} \PYG{n}{stopping}
\end{sphinxVerbatim}


\paragraph{The batch log files}
\label{\detokenize{users_guide/trouble-shooting/trouble-shooting:the-batch-log-files}}
The names of the batch log files will depend on the batch system of the machine that is being used. They will normally be in the script directory. Usually, they don’t contain important information, but they are a last resort place to look for error messages. On the NCAR system “yellowstone” the batch files are called with names that start with the batch submission script and then either “stderr.o” or “stdout.o”, with the job number at the end.


\subsubsection{General Advice on Debugging Run time Problems}
\label{\detokenize{users_guide/trouble-shooting/trouble-shooting:general-advice-on-debugging-run-time-problems}}
Here are some suggestions on how to track down a problem while running. In general if the problem still occurs for a simpler case, it will be easier to track down.
\begin{enumerate}
\item {} 
\sphinxstyleemphasis{Run in DEBUG mode}

\item {} 
\sphinxstyleemphasis{Run with a smaller set of processors}

\item {} 
\sphinxstyleemphasis{Run in serial mode with a single processor}

\item {} 
\sphinxstyleemphasis{Run at a lower resolution}

\item {} 
\sphinxstyleemphasis{Run a simpler case}

\item {} 
\sphinxstyleemphasis{Run with a debugger}

\end{enumerate}


\paragraph{Run in DEBUG mode}
\label{\detokenize{users_guide/trouble-shooting/trouble-shooting:run-in-debug-mode}}
The first thing to try is to run in DEBUG mode so that float point trapping will be triggered as well as array bounds checking and other things the compiler can turn on to help you find problems.
To do this edit the \sphinxcode{env\_build.xml} file and set DEBUG to TRUE as follows:

\begin{sphinxVerbatim}[commandchars=\\\{\}]
\PYG{o}{\PYGZgt{}} \PYG{o}{.}\PYG{o}{/}\PYG{n}{xmlchange} \PYG{n}{DEBUG}\PYG{o}{=}\PYG{n}{TRUE}
\end{sphinxVerbatim}


\paragraph{Run with a smaller set of processors}
\label{\detokenize{users_guide/trouble-shooting/trouble-shooting:run-with-a-smaller-set-of-processors}}
Another way to simplify the system is to run with a smaller set of processors. You will need to clean the setup and edit the \textendash{}env\_mach\_pes.xml\textendash{}. For example, to run with four processors:

\begin{sphinxVerbatim}[commandchars=\\\{\}]
\PYG{o}{\PYGZgt{}} \PYG{o}{.}\PYG{o}{/}\PYG{n}{case}\PYG{o}{.}\PYG{n}{setup} \PYG{o}{\PYGZhy{}}\PYG{n}{clean}
\PYG{o}{\PYGZgt{}} \PYG{o}{.}\PYG{o}{/}\PYG{n}{xmlchange} \PYG{n}{NTASKS\PYGZus{}ATM}\PYG{o}{=}\PYG{l+m+mi}{4}\PYG{p}{,}\PYG{n}{NTASKS\PYGZus{}LND}\PYG{o}{=}\PYG{l+m+mi}{4}\PYG{p}{,}\PYG{n}{NTASKS\PYGZus{}ICE}\PYG{o}{=}\PYG{l+m+mi}{4}\PYG{p}{,}\PYG{n}{NTASKS\PYGZus{}OCN}\PYG{o}{=}\PYG{l+m+mi}{4}\PYG{p}{,}\PYG{n}{NTASKS\PYGZus{}CPL}\PYG{o}{=}\PYG{l+m+mi}{4}\PYG{p}{,}\PYG{n}{NTASKS\PYGZus{}GLC}\PYG{o}{=}\PYG{l+m+mi}{4}
\PYG{o}{\PYGZgt{}} \PYG{o}{.}\PYG{o}{/}\PYG{n}{case}\PYG{o}{.}\PYG{n}{setup}
\end{sphinxVerbatim}

Another recommended simplification is to run without threading, so set the NTHRDS for each component to “1” if it isn’t already. Sometimes, multiprocessing problems require a certain number of processors before they occur so you may not be able to debug the problem without enough processors. But, it’s always good to reduce it to as low a number as possible to make it simpler. For threading problems you may have to have threading enabled to find the problem, but you can run with 1, 2, or 3 threads to see what happens.


\paragraph{Run in serial mode with a single processor}
\label{\detokenize{users_guide/trouble-shooting/trouble-shooting:run-in-serial-mode-with-a-single-processor}}
Simplifying to one processor removes all multi-processing problems and makes the case as simple as possible. If you can enable \sphinxcode{MPILIB=mpi-serial} you will also be able to run interactively rather than having to submit to a job queue, which sometimes makes it easier to run and debug. If you can use \sphinxcode{MPILIB=mpi-serial} you can also use threading, but still run interactively in order to use more processors to make it faster if needed.

\begin{sphinxVerbatim}[commandchars=\\\{\}]
\PYG{o}{\PYGZgt{}} \PYG{o}{.}\PYG{o}{/}\PYG{n}{case}\PYG{o}{.}\PYG{n}{setup} \PYG{o}{\PYGZhy{}}\PYG{n}{clean}
\PYG{c+c1}{\PYGZsh{} Set tasks and threads for each component to 1}
\PYG{c+c1}{\PYGZsh{} You could also set threads to something \PYGZgt{} 1 for speed, but still}
\PYG{c+c1}{\PYGZsh{} run interactively if threading isn\PYGZsq{}t an issue.}

\PYG{o}{\PYGZgt{}} \PYG{o}{.}\PYG{o}{/}\PYG{n}{xmlchange} \PYG{n}{NTASKS\PYGZus{}ATM}\PYG{o}{=}\PYG{l+m+mi}{1}\PYG{p}{,}\PYG{n}{NTHRDS\PYGZus{}ATM}\PYG{o}{=}\PYG{l+m+mi}{1}\PYG{p}{,}\PYG{n}{NTASKS\PYGZus{}LND}\PYG{o}{=}\PYG{l+m+mi}{1}\PYG{p}{,}\PYG{n}{NTHRDS\PYGZus{}LND}\PYG{o}{=}\PYG{l+m+mi}{1}\PYG{p}{,}\PYG{n}{NTASKS\PYGZus{}ICE}\PYG{o}{=}\PYG{l+m+mi}{1}\PYG{p}{,}\PYG{n}{NTHRDS\PYGZus{}ICE}\PYG{o}{=}\PYG{l+m+mi}{1}
\PYG{o}{\PYGZgt{}} \PYG{o}{.}\PYG{o}{/}\PYG{n}{xmlchange} \PYG{n}{NTASKS\PYGZus{}OCN}\PYG{o}{=}\PYG{l+m+mi}{1}\PYG{p}{,}\PYG{n}{NTHRDS\PYGZus{}OCN}\PYG{o}{=}\PYG{l+m+mi}{1}\PYG{p}{,}\PYG{n}{NTASKS\PYGZus{}CPL}\PYG{o}{=}\PYG{l+m+mi}{1}\PYG{p}{,}\PYG{n}{NTHRDS\PYGZus{}CPL}\PYG{o}{=}\PYG{l+m+mi}{1}\PYG{p}{,}\PYG{n}{NTASKS\PYGZus{}GLC}\PYG{o}{=}\PYG{l+m+mi}{1}\PYG{p}{,}\PYG{n}{NTHRDS\PYGZus{}GLC}\PYG{o}{=}\PYG{l+m+mi}{1}
\PYG{c+c1}{\PYGZsh{} set MPILIB to mpi\PYGZhy{}serial so that you can run interactively}
\PYG{o}{\PYGZgt{}} \PYG{o}{.}\PYG{o}{/}\PYG{n}{xmlchange} \PYG{n}{MPILIB}\PYG{o}{=}\PYG{n}{mpi}\PYG{o}{\PYGZhy{}}\PYG{n}{serial}
\PYG{o}{\PYGZgt{}} \PYG{o}{.}\PYG{o}{/}\PYG{n}{case}\PYG{o}{.}\PYG{n}{setup}
\PYG{c+c1}{\PYGZsh{} Then build your case}
\PYG{c+c1}{\PYGZsh{} And finally run, by running the *.run script interactively}
\end{sphinxVerbatim}


\paragraph{Run at a lower resolution}
\label{\detokenize{users_guide/trouble-shooting/trouble-shooting:run-at-a-lower-resolution}}
If you can create a new case running at a lower resolution and replicate the problem it may be easier to solve. This of course requires creating a whole new case, and trying out different lower resolutions.


\paragraph{Run a simpler case}
\label{\detokenize{users_guide/trouble-shooting/trouble-shooting:run-a-simpler-case}}
Along the same lines, you might try running a simpler case, trying another compset with a simpler setup and see if you can replicate the problem and then debug from that simpler case. Again, of course you will need to create new cases to do this.


\paragraph{Run with a debugger}
\label{\detokenize{users_guide/trouble-shooting/trouble-shooting:run-with-a-debugger}}
Another suggestion is to run the model with a debugger such as: \sphinxstylestrong{dbx}, \sphinxstylestrong{gdb}, or \sphinxstylestrong{totalview}.
Often to run with a debugger you will need to reduce the number of processors as outlined above.
Some debuggers such as \sphinxstylestrong{dbx} will only work with one processor, while more advanced debuggers such as \sphinxstylestrong{totalview} can work with both MPI tasks and OMP threads.
Even simple debuggers though can be used to query core files, to see where the code was at when it died (for example using the \sphinxstylestrong{where} in \sphinxstylestrong{dbx} for a core file can be very helpful.
For help in running with a debugger you will need to contact your system administrators for the machine you are running on.


\chapter{CLM Technical Note}
\label{\detokenize{tech_note/index:tech-note}}\label{\detokenize{tech_note/index:clm-technical-note}}\label{\detokenize{tech_note/index::doc}}
\sphinxstylestrong{July 2017}

\sphinxstylestrong{Technical Description of version 5.0 of the Community Land Model
(CLM)}

\sphinxstylestrong{*Coordinating Lead Authors*}

\sphinxstylestrong{David M. Lawrence, Rosie Fisher, Charles D. Koven, Keith W. Oleson, Sean Swenson}

\sphinxstylestrong{*Lead Authors*}

\sphinxstylestrong{Gordon Bonan, Bardan Ghimire, Leo van Kampenhout, Daniel Kennedy, Erik Kluzek, Peter J. Lawrence, Fang Li, Hongyi Li, Danica Lombardozzi, Yaqiong Lu, Justin Perket, William J. Riley, William Sacks, Mingjie Shi, Will Wieder, Chonggang Xu}

\sphinxstylestrong{*Contributing Authors*}

\sphinxstylestrong{Ben Andre, Ali Ashehad, Andrew Badger, Gautam Bisht, Patrick Broxton, Michael Brunke, Jonathon Buzan, Martyn Clark, Tony Craig, Kyla Dahlin, Beth Drewniak, Louisa Emmons, Josh Fisher, Mark Flanner, Pierre Gentine, Jan Lenaerts, Sam Levis,
L. Ruby Leung, William Lipscomb, Jon Pelletier, Daniel M. Ricciuto, Ben Sanderson, Andrew Slater, Zachary M. Subin, Jinyun Tang, Ahmed Tawfik, Quinn Thomas, Simone Tilmes, Mariana Vertenstein, Francis Vitt, Xubin Zeng}

The National Center for Atmospheric Research (NCAR) is operated by the
nonprofit University Corporation for Atmospheric Research (UCAR) under
the sponsorship of the National Science Foundation. Any opinions,
findings, conclusions, or recommendations expressed in this publication
are those of the author(s) and do not necessarily reflect the views of
the National Science Foundation.

National Center for Atmospheric Research
P. O. Box 3000, Boulder, Colorado 80307-300

\sphinxstylestrong{LIST OF FIGURES}
\begin{itemize}
\item {} 
\hyperref[\detokenize{tech_note/Introduction/CLM50_Tech_Note_Introduction:figure-land-processes}]{Figure \ref{\detokenize{tech_note/Introduction/CLM50_Tech_Note_Introduction:figure-land-processes}}} Land biogeophysical, biogeochemical, and landscape processes simulated by CLM (adapted from Lawrence et al. (2011) for CLM4.5).

\item {} 
\hyperref[\detokenize{tech_note/Ecosystem/CLM50_Tech_Note_Ecosystem:figure-clm-subgrid-hierarchy}]{Figure \ref{\detokenize{tech_note/Ecosystem/CLM50_Tech_Note_Ecosystem:figure-clm-subgrid-hierarchy}}} Configuration of the CLM subgrid hierarchy.

\item {} 
\hyperref[\detokenize{tech_note/Radiative_Fluxes/CLM50_Tech_Note_Radiative_Fluxes:figure-radiation-schematic}]{Figure \ref{\detokenize{tech_note/Radiative_Fluxes/CLM50_Tech_Note_Radiative_Fluxes:figure-radiation-schematic}}} Schematic diagram of (a) direct beam radiation, (b) diffuse solar radiation, and (c) longwave radiation absorbed, transmitted, and reflected by vegetation and ground.

\item {} 
\hyperref[\detokenize{tech_note/Fluxes/CLM50_Tech_Note_Fluxes:figure-schematic-diagram-of-sensible-heat-fluxes}]{Figure \ref{\detokenize{tech_note/Fluxes/CLM50_Tech_Note_Fluxes:figure-schematic-diagram-of-sensible-heat-fluxes}}} Schematic diagram of sensible heat fluxes for (a) non-vegetated surfaces and (b) vegetated surfaces.

\item {} 
\hyperref[\detokenize{tech_note/Fluxes/CLM50_Tech_Note_Fluxes:figure-schematic-diagram-of-latent-heat-fluxes}]{Figure \ref{\detokenize{tech_note/Fluxes/CLM50_Tech_Note_Fluxes:figure-schematic-diagram-of-latent-heat-fluxes}}} Schematic diagram of water vapor fluxes for (a) non-vegetated surfaces and (b) vegetated surfaces.

\item {} 
\hyperref[\detokenize{tech_note/Soil_Snow_Temperatures/CLM50_Tech_Note_Soil_Snow_Temperatures:figure-soil-temperature-schematic}]{Figure \ref{\detokenize{tech_note/Soil_Snow_Temperatures/CLM50_Tech_Note_Soil_Snow_Temperatures:figure-soil-temperature-schematic}}}. Schematic diagram of numerical scheme used to solve for soil temperature.

\item {} 
\hyperref[\detokenize{tech_note/Hydrology/CLM50_Tech_Note_Hydrology:figure-hydrologic-processes}]{Figure \ref{\detokenize{tech_note/Hydrology/CLM50_Tech_Note_Hydrology:figure-hydrologic-processes}}} Hydrologic processes represented in CLM.

\item {} 
\hyperref[\detokenize{tech_note/Hydrology/CLM50_Tech_Note_Hydrology:figure-water-flux-schematic}]{Figure \ref{\detokenize{tech_note/Hydrology/CLM50_Tech_Note_Hydrology:figure-water-flux-schematic}}} Schematic diagram of numerical scheme used to solve for soil water fluxes.

\item {} 
\hyperref[\detokenize{tech_note/Snow_Hydrology/CLM50_Tech_Note_Snow_Hydrology:figure-three-layer-snow-pack}]{Figure \ref{\detokenize{tech_note/Snow_Hydrology/CLM50_Tech_Note_Snow_Hydrology:figure-three-layer-snow-pack}}} Example of three layer snow pack (snl=-3).

\item {} 
\hyperref[\detokenize{tech_note/Urban/CLM50_Tech_Note_Urban:figure-schematic-representation-of-the-urban-landunit}]{Figure \ref{\detokenize{tech_note/Urban/CLM50_Tech_Note_Urban:figure-schematic-representation-of-the-urban-landunit}}} Schematic representation of the urban land unit.

\item {} 
\hyperref[\detokenize{tech_note/MOSART/CLM50_Tech_Note_MOSART:figure-mosart-conceptual-diagram}]{Figure \ref{\detokenize{tech_note/MOSART/CLM50_Tech_Note_MOSART:figure-mosart-conceptual-diagram}}} MOSART conceptual diagram.

\item {} 
\hyperref[\detokenize{tech_note/Urban/CLM50_Tech_Note_Urban:figure-schematic-of-urban-and-atmospheric-model-coupling}]{Figure \ref{\detokenize{tech_note/Urban/CLM50_Tech_Note_Urban:figure-schematic-of-urban-and-atmospheric-model-coupling}}} Schematic of urban and atmospheric model coupling.

\item {} 
\hyperref[\detokenize{tech_note/CN_Pools/CLM50_Tech_Note_CN_Pools:figure-vegetation-fluxes-and-pools}]{Figure \ref{\detokenize{tech_note/CN_Pools/CLM50_Tech_Note_CN_Pools:figure-vegetation-fluxes-and-pools}}} Vegetation fluxes and pools.

\item {} 
\sphinxcode{Figure Carbon and nitrogen pools} Carbon and nitrogen pools.

\item {} 
\hyperref[\detokenize{tech_note/Vegetation_Phenology_Turnover/CLM50_Tech_Note_Vegetation_Phenology_Turnover:figure-annual-phenology-cycle}]{Figure \ref{\detokenize{tech_note/Vegetation_Phenology_Turnover/CLM50_Tech_Note_Vegetation_Phenology_Turnover:figure-annual-phenology-cycle}}} Example of annual phenology cycle for seasonal deciduous.

\item {} 
14.2. Example fluxes and pools sizes for an onset growth period of 15 days, with initial transfer pool size of 100 gC m-2 and a timestep of one hour. a) Flux leaving transfer pool (e.g. CFleaf\_xfer,leaf). b) Carbon content of transfer pool and its associated display pool (e.g. CSleaf\_xfer and CSleaf, respectively).

\item {} 
14.3. Example fluxes and pool sizes for an offset (litterfall) period of 15 days, with initial display pool size of 100 gC m-2 and a timestep of one hour. a) Litterfall flux (e.g CFleaf,litter). b) Carbon content of display pool and litter pool through the litterfall period, ignoring the losses from litter pool due to decomposition during this period.

\item {} 
\hyperref[\detokenize{tech_note/Decomposition/CLM50_Tech_Note_Decomposition:figure-schematic-of-decomposition-model-in-clm}]{Figure \ref{\detokenize{tech_note/Decomposition/CLM50_Tech_Note_Decomposition:figure-schematic-of-decomposition-model-in-clm}}} Schematic of decomposition model in CLM.

\item {} 
\hyperref[\detokenize{tech_note/Decomposition/CLM50_Tech_Note_Decomposition:figure-pool-structure}]{Figure \ref{\detokenize{tech_note/Decomposition/CLM50_Tech_Note_Decomposition:figure-pool-structure}}} Pool structure, transitions, respired fractions (numbers at end of arrows), and turnover times (numbers in boxes) for the 2 alternate soil decomposition models included in CLM.

\item {} 
\hyperref[\detokenize{tech_note/External_Nitrogen_Cycle/CLM50_Tech_Note_External_Nitrogen_Cycle:figure-biological-nitrogen-fixation}]{Figure \ref{\detokenize{tech_note/External_Nitrogen_Cycle/CLM50_Tech_Note_External_Nitrogen_Cycle:figure-biological-nitrogen-fixation}}} Biological nitrogen fixation as a function of annual net primary production.

\item {} 
\hyperref[\detokenize{tech_note/Methane/CLM50_Tech_Note_Methane:figure-methane-schematic}]{Figure \ref{\detokenize{tech_note/Methane/CLM50_Tech_Note_Methane:figure-methane-schematic}}} Schematic representation of biological and physical processes integrated in CLM that affect the net CH4 surface flux. (left) Fully inundated portion of a CLM gridcell and (right) variably saturated portion of a gridcell.

\item {} 
\hyperref[\detokenize{tech_note/Transient_Landcover/CLM50_Tech_Note_Transient_Landcover:figure-schematic-of-land-cover-change}]{Figure \ref{\detokenize{tech_note/Transient_Landcover/CLM50_Tech_Note_Transient_Landcover:figure-schematic-of-land-cover-change}}} Schematic of land cover change impacts on CLM carbon pools and fluxes.

\item {} 
\hyperref[\detokenize{tech_note/Transient_Landcover/CLM50_Tech_Note_Transient_Landcover:figure-schematic-of-translation-of-annual-luh2-land-units}]{Figure \ref{\detokenize{tech_note/Transient_Landcover/CLM50_Tech_Note_Transient_Landcover:figure-schematic-of-translation-of-annual-luh2-land-units}}} Schematic of translation of annual UNH land units to CLM4 plant functional types.

\end{itemize}

\sphinxstylestrong{LIST OF TABLES}
\begin{itemize}
\item {} 
\hyperref[\detokenize{tech_note/Ecosystem/CLM50_Tech_Note_Ecosystem:table-plant-functional-types}]{Table \ref{\detokenize{tech_note/Ecosystem/CLM50_Tech_Note_Ecosystem:table-plant-functional-types}}} Plant functional types

\item {} 
\hyperref[\detokenize{tech_note/Ecosystem/CLM50_Tech_Note_Ecosystem:table-plant-functional-type-canopy-top-and-bottom-heights}]{Table \ref{\detokenize{tech_note/Ecosystem/CLM50_Tech_Note_Ecosystem:table-plant-functional-type-canopy-top-and-bottom-heights}}} Plant functional type canopy top and bottom heights

\item {} 
\hyperref[\detokenize{tech_note/Ecosystem/CLM50_Tech_Note_Ecosystem:table-soil-layer-structure}]{Table \ref{\detokenize{tech_note/Ecosystem/CLM50_Tech_Note_Ecosystem:table-soil-layer-structure}}} Soil layer structure

\item {} 
\hyperref[\detokenize{tech_note/Ecosystem/CLM50_Tech_Note_Ecosystem:table-atmospheric-input-to-land-model}]{Table \ref{\detokenize{tech_note/Ecosystem/CLM50_Tech_Note_Ecosystem:table-atmospheric-input-to-land-model}}} Atmospheric input to land model

\item {} 
\hyperref[\detokenize{tech_note/Ecosystem/CLM50_Tech_Note_Ecosystem:table-land-model-output-to-atmospheric-model}]{Table \ref{\detokenize{tech_note/Ecosystem/CLM50_Tech_Note_Ecosystem:table-land-model-output-to-atmospheric-model}}} Land model output to atmospheric model

\item {} 
\hyperref[\detokenize{tech_note/Ecosystem/CLM50_Tech_Note_Ecosystem:table-surface-data-required-for-clm-and-their-base-spatial-resolution}]{Table \ref{\detokenize{tech_note/Ecosystem/CLM50_Tech_Note_Ecosystem:table-surface-data-required-for-clm-and-their-base-spatial-resolution}}} Surface data required for CLM4.5 and their base spatial resolution

\item {} 
\hyperref[\detokenize{tech_note/Ecosystem/CLM50_Tech_Note_Ecosystem:table-physical-constants}]{Table \ref{\detokenize{tech_note/Ecosystem/CLM50_Tech_Note_Ecosystem:table-physical-constants}}} Physical constants

\item {} 
\hyperref[\detokenize{tech_note/Surface_Albedos/CLM50_Tech_Note_Surface_Albedos:table-plant-functional-type-optical-properties}]{Table \ref{\detokenize{tech_note/Surface_Albedos/CLM50_Tech_Note_Surface_Albedos:table-plant-functional-type-optical-properties}}} Plant functional type optical properties

\item {} 
\hyperref[\detokenize{tech_note/Surface_Albedos/CLM50_Tech_Note_Surface_Albedos:table-intercepted-snow-optical-properties}]{Table \ref{\detokenize{tech_note/Surface_Albedos/CLM50_Tech_Note_Surface_Albedos:table-intercepted-snow-optical-properties}}} Intercepted snow optical properties

\item {} 
\hyperref[\detokenize{tech_note/Surface_Albedos/CLM50_Tech_Note_Surface_Albedos:table-dry-and-saturated-soil-albedos}]{Table \ref{\detokenize{tech_note/Surface_Albedos/CLM50_Tech_Note_Surface_Albedos:table-dry-and-saturated-soil-albedos}}} Dry and saturated soil albedos

\item {} 
\hyperref[\detokenize{tech_note/Surface_Albedos/CLM50_Tech_Note_Surface_Albedos:table-spectral-bands-and-weights-used-for-snow-radiative-transfer}]{Table \ref{\detokenize{tech_note/Surface_Albedos/CLM50_Tech_Note_Surface_Albedos:table-spectral-bands-and-weights-used-for-snow-radiative-transfer}}} Spectral bands and weights used for snow radiative transfer

\item {} 
\hyperref[\detokenize{tech_note/Surface_Albedos/CLM50_Tech_Note_Surface_Albedos:table-single-scatter-albedo-values-used-for-snowpack-impurities-and-ice}]{Table \ref{\detokenize{tech_note/Surface_Albedos/CLM50_Tech_Note_Surface_Albedos:table-single-scatter-albedo-values-used-for-snowpack-impurities-and-ice}}} Single-scatter albedo values used for snowpack impurities and ice

\item {} 
\hyperref[\detokenize{tech_note/Surface_Albedos/CLM50_Tech_Note_Surface_Albedos:table-mass-extinction-values}]{Table \ref{\detokenize{tech_note/Surface_Albedos/CLM50_Tech_Note_Surface_Albedos:table-mass-extinction-values}}} Mass extinction values (m2 kg-1) used for snowpack impurities and ice.

\item {} 
\hyperref[\detokenize{tech_note/Surface_Albedos/CLM50_Tech_Note_Surface_Albedos:table-asymmetry-scattering-parameters-used-for-snowpack-impurities-and-ice}]{Table \ref{\detokenize{tech_note/Surface_Albedos/CLM50_Tech_Note_Surface_Albedos:table-asymmetry-scattering-parameters-used-for-snowpack-impurities-and-ice}}} Asymmetry scattering parameters used for snowpack impurities and ice.

\item {} 
\hyperref[\detokenize{tech_note/Surface_Albedos/CLM50_Tech_Note_Surface_Albedos:table-orbital-parameters}]{Table \ref{\detokenize{tech_note/Surface_Albedos/CLM50_Tech_Note_Surface_Albedos:table-orbital-parameters}}} Orbital parameters

\item {} 
\hyperref[\detokenize{tech_note/Fluxes/CLM50_Tech_Note_Fluxes:table-plant-functional-type-aerodynamic-parameters}]{Table \ref{\detokenize{tech_note/Fluxes/CLM50_Tech_Note_Fluxes:table-plant-functional-type-aerodynamic-parameters}}} Plant functional type aerodynamic parameters

\item {} 
\hyperref[\detokenize{tech_note/Fluxes/CLM50_Tech_Note_Fluxes:table-coefficients-for-saturation-vapor-pressure}]{Table \ref{\detokenize{tech_note/Fluxes/CLM50_Tech_Note_Fluxes:table-coefficients-for-saturation-vapor-pressure}}} Coefficients for e$_{\text{sat}}$$^{\text{T}}$

\item {} 
\hyperref[\detokenize{tech_note/Fluxes/CLM50_Tech_Note_Fluxes:table-coefficients-for-derivative-of-esat}]{Table \ref{\detokenize{tech_note/Fluxes/CLM50_Tech_Note_Fluxes:table-coefficients-for-derivative-of-esat}}} Coefficients for 112:numref:{}`{}` 6.1. Soil layer structure.

\item {} 
\hyperref[\detokenize{tech_note/Snow_Hydrology/CLM50_Tech_Note_Snow_Hydrology:table-meltwater-scavenging}]{Table \ref{\detokenize{tech_note/Snow_Hydrology/CLM50_Tech_Note_Snow_Hydrology:table-meltwater-scavenging}}} Meltwater scavenging efficiency for particles within snow

\item {} 
\hyperref[\detokenize{tech_note/Snow_Hydrology/CLM50_Tech_Note_Snow_Hydrology:table-snow-layer-thickness}]{Table \ref{\detokenize{tech_note/Snow_Hydrology/CLM50_Tech_Note_Snow_Hydrology:table-snow-layer-thickness}}} Minimum and maximum thickness of snow layers (m)

\item {} 
\hyperref[\detokenize{tech_note/Photosynthesis/CLM50_Tech_Note_Photosynthesis:table-plant-functional-type-pft-stomatal-conductance-parameters}]{Table \ref{\detokenize{tech_note/Photosynthesis/CLM50_Tech_Note_Photosynthesis:table-plant-functional-type-pft-stomatal-conductance-parameters}}} Plant functional type (PFT) stomatal conductance parameters.

\item {} 
\hyperref[\detokenize{tech_note/Photosynthesis/CLM50_Tech_Note_Photosynthesis:table-temperature-dependence-parameters-for-c3-photosynthesis}]{Table \ref{\detokenize{tech_note/Photosynthesis/CLM50_Tech_Note_Photosynthesis:table-temperature-dependence-parameters-for-c3-photosynthesis}}} Temperature dependence parameters for C3 photosynthesis.

\item {} 
\hyperref[\detokenize{tech_note/Plant_Hydraulics/CLM50_Tech_Note_Plant_Hydraulics:table-plant-functional-type-root-distribution-parameters}]{Table \ref{\detokenize{tech_note/Plant_Hydraulics/CLM50_Tech_Note_Plant_Hydraulics:table-plant-functional-type-root-distribution-parameters}}} Plant functional type root distribution parameters.

\item {} 
\hyperref[\detokenize{tech_note/MOSART/CLM50_Tech_Note_MOSART:table-mosart-parameters}]{Table \ref{\detokenize{tech_note/MOSART/CLM50_Tech_Note_MOSART:table-mosart-parameters}}} List of parameters in the global hydrography dataset.

\item {} 
\hyperref[\detokenize{tech_note/CN_Allocation/CLM50_Tech_Note_CN_Allocation:table-allocation-and-cn-ratio-parameters}]{Table \ref{\detokenize{tech_note/CN_Allocation/CLM50_Tech_Note_CN_Allocation:table-allocation-and-cn-ratio-parameters}}} Allocation and carbon:nitrogen ratio parameters

\item {} 
\hyperref[\detokenize{tech_note/Decomposition/CLM50_Tech_Note_Decomposition:table-decomposition-rate-constants}]{Table \ref{\detokenize{tech_note/Decomposition/CLM50_Tech_Note_Decomposition:table-decomposition-rate-constants}}} Decomposition rate constants for litter and SOM pools, C:N ratios, and acceleration parameters (see section 15.8 for explanation) for the CLM-CN decomposition pool structure.

\item {} 
\hyperref[\detokenize{tech_note/Decomposition/CLM50_Tech_Note_Decomposition:table-respiration-fractions-for-litter-and-som-pools}]{Table \ref{\detokenize{tech_note/Decomposition/CLM50_Tech_Note_Decomposition:table-respiration-fractions-for-litter-and-som-pools}}} Respiration fractions for litter and SOM pools

\item {} 
\hyperref[\detokenize{tech_note/Decomposition/CLM50_Tech_Note_Decomposition:table-turnover-times}]{Table \ref{\detokenize{tech_note/Decomposition/CLM50_Tech_Note_Decomposition:table-turnover-times}}} Turnover times, C:N ratios, and acceleration parameters (see section 15.8 for explanation) for the Century-based decomposition cascade.

\item {} 
\hyperref[\detokenize{tech_note/Decomposition/CLM50_Tech_Note_Decomposition:table-respiration-fractions-for-century-based-structure}]{Table \ref{\detokenize{tech_note/Decomposition/CLM50_Tech_Note_Decomposition:table-respiration-fractions-for-century-based-structure}}} Respiration fractions for litter and SOM pools for Century-based structure

\item {} 
\hyperref[\detokenize{tech_note/Fire/CLM50_Tech_Note_Fire:table-pft-specific-combustion-completeness-and-fire-mortality-factors}]{Table \ref{\detokenize{tech_note/Fire/CLM50_Tech_Note_Fire:table-pft-specific-combustion-completeness-and-fire-mortality-factors}}} PFT-specific combustion completeness and fire mortality factors.

\item {} 
\hyperref[\detokenize{tech_note/Methane/CLM50_Tech_Note_Methane:table-methane-parameter-descriptions}]{Table \ref{\detokenize{tech_note/Methane/CLM50_Tech_Note_Methane:table-methane-parameter-descriptions}}}  Parameter descriptions and sensitivity analysis ranges applied in the methane model.

\item {} 
\hyperref[\detokenize{tech_note/Methane/CLM50_Tech_Note_Methane:table-temperature-dependence-of-aqueous-and-gaseous-diffusion}]{Table \ref{\detokenize{tech_note/Methane/CLM50_Tech_Note_Methane:table-temperature-dependence-of-aqueous-and-gaseous-diffusion}}} Temperature dependence of aqueous and gaseous diffusion coefficients for CH4 and O2.

\item {} 
\hyperref[\detokenize{tech_note/Crop_Irrigation/CLM50_Tech_Note_Crop_Irrigation:table-crop-plant-functional-types}]{Table \ref{\detokenize{tech_note/Crop_Irrigation/CLM50_Tech_Note_Crop_Irrigation:table-crop-plant-functional-types}}} Crop plant functional types (pfts) in CLM4.5CNcrop and their parameters relating to phenology and morphology. Numbers in the first column correspond to the list of pfts in \hyperref[\detokenize{tech_note/Ecosystem/CLM50_Tech_Note_Ecosystem:table-plant-functional-types}]{Table \ref{\detokenize{tech_note/Ecosystem/CLM50_Tech_Note_Ecosystem:table-plant-functional-types}}}.

\item {} 
\hyperref[\detokenize{tech_note/Crop_Irrigation/CLM50_Tech_Note_Crop_Irrigation:table-crop-plant-functional-types}]{Table \ref{\detokenize{tech_note/Crop_Irrigation/CLM50_Tech_Note_Crop_Irrigation:table-crop-plant-functional-types}}} Crop pfts in CLM4.5CNcrop and their parameters relating to allocation. Numbers in the first column correspond to the list of pfts in \hyperref[\detokenize{tech_note/Ecosystem/CLM50_Tech_Note_Ecosystem:table-plant-functional-types}]{Table \ref{\detokenize{tech_note/Ecosystem/CLM50_Tech_Note_Ecosystem:table-plant-functional-types}}}.

\item {} 
\hyperref[\detokenize{tech_note/Crop_Irrigation/CLM50_Tech_Note_Crop_Irrigation:table-crop-allocation-parameters}]{Table \ref{\detokenize{tech_note/Crop_Irrigation/CLM50_Tech_Note_Crop_Irrigation:table-crop-allocation-parameters}}} Crop allocation parameters for the active crop plant functional types (pfts) in CLM5BGCCROP. Numbers in the first row correspond to the list of pfts in \hyperref[\detokenize{tech_note/Crop_Irrigation/CLM50_Tech_Note_Crop_Irrigation:table-crop-plant-functional-types}]{Table \ref{\detokenize{tech_note/Crop_Irrigation/CLM50_Tech_Note_Crop_Irrigation:table-crop-plant-functional-types}}}.

\item {} 
\sphinxcode{Table Plant functional type (PFT) biogeography rules} Plant functional type (PFT) biogeography rules with respect to climate.

\item {} 
\hyperref[\detokenize{tech_note/Dust/CLM50_Tech_Note_Dust:table-dust-mass-fraction}]{Table \ref{\detokenize{tech_note/Dust/CLM50_Tech_Note_Dust:table-dust-mass-fraction}}} Mass fraction m$_{\text{i}}$ , mass median diameter $_{\text{v, i}}$ , and geometric standard deviation $_{\text{g, i}}$ , per dust source mode i

\item {} 
\hyperref[\detokenize{tech_note/Dust/CLM50_Tech_Note_Dust:table-dust-minimum-and-maximum-particle-diameters}]{Table \ref{\detokenize{tech_note/Dust/CLM50_Tech_Note_Dust:table-dust-minimum-and-maximum-particle-diameters}}} Minimum and maximum particle diameters in each dust transport bin j

\end{itemize}

\sphinxstylestrong{ACKNOWLEDGEMENTS}

The authors would like to acknowledge the substantial contributions of
the following members of the Land Model and Biogeochemistry Working
Groups to the development of the Community Land Model since its
inception in 1996: Benjamin Andre, Ian Baker, Michael Barlage, Mike
Bosilovich, Marcia Branstetter, Tony Craig, Aiguo Dai, Yongjiu Dai, Mark
Decker, Scott Denning, Robert Dickinson, Paul Dirmeyer, Jared Entin, Jay
Famiglietti, Johannes Feddema, Mark Flanner, Jon Foley, Andrew Fox, Inez
Fung, David Gochis, Alex Guenther, Tim Hoar, Forrest Hoffman, Paul
Houser, Trish Jackson, Brian Kauffman, Silvia Kloster, Natalie Mahowald,
Jiafu Mao, Lei Meng, Sheri Michelson, Guo-Yue Niu, Adam Phillips, Taotao
Qian, Jon Radakovich, James Randerson, Nan Rosenbloom, Steve Running,
Koichi Sakaguchi, Adam Schlosser, Andrew Slater, Reto Stöckli, Ying Sun, Quinn
Thomas, Peter Thornton, Mariana Vertenstein, Nicholas Viovy, Aihui Wang, Guiling Wang,
Zong-Liang Yang, Charlie Zender, Xiaodong Zeng, and Xubin Zeng.


\section{Introduction}
\label{\detokenize{tech_note/Introduction/CLM50_Tech_Note_Introduction:rst-introduction}}\label{\detokenize{tech_note/Introduction/CLM50_Tech_Note_Introduction:introduction}}\label{\detokenize{tech_note/Introduction/CLM50_Tech_Note_Introduction::doc}}
The purpose of this document is to fully describe the biogeophysical and
biogeochemical parameterizations and numerical implementation of version
5.0 of the Community Land Model (CLM5.0). Scientific justification and
evaluation of these parameterizations can be found in the referenced
scientific papers (\DUrole{xref,std,std-ref}{rst\_References}). This document and the CLM5.0
User’s Guide together provide the user with the scientific description
and operating instructions for CLM.


\subsection{Model History}
\label{\detokenize{tech_note/Introduction/CLM50_Tech_Note_Introduction:model-history}}

\subsubsection{Inception of CLM}
\label{\detokenize{tech_note/Introduction/CLM50_Tech_Note_Introduction:inception-of-clm}}
The early development of the Community Land Model can be described as
the merging of a community-developed land model focusing on
biogeophysics and a concurrent effort at NCAR to expand the NCAR Land
Surface Model (NCAR LSM, {\hyperref[\detokenize{tech_note/References/CLM50_Tech_Note_References:bonan1996}]{\sphinxcrossref{\DUrole{std,std-ref}{Bonan 1996}}}}) to include the carbon cycle,
vegetation dynamics, and river routing. The concept of a
community-developed land component of the Community Climate System Model
(CCSM) was initially proposed at the CCSM Land Model Working Group
(LMWG) meeting in February 1996. Initial software specifications and
development focused on evaluating the best features of three existing
land models: the NCAR LSM ({\hyperref[\detokenize{tech_note/References/CLM50_Tech_Note_References:bonan1996}]{\sphinxcrossref{\DUrole{std,std-ref}{Bonan 1996, 1998}}}}) used in the Community
Climate Model (CCM3) and the initial version of CCSM; the Institute of
Atmospheric Physics, Chinese Academy of Sciences land model (IAP94) ({\hyperref[\detokenize{tech_note/References/CLM50_Tech_Note_References:daizeng1997}]{\sphinxcrossref{\DUrole{std,std-ref}{Dai
and Zeng 1997}}}}); and the Biosphere-Atmosphere Transfer Scheme (BATS)
({\hyperref[\detokenize{tech_note/References/CLM50_Tech_Note_References:dickinsonetal1993}]{\sphinxcrossref{\DUrole{std,std-ref}{Dickinson et al. 1993}}}}) used with CCM2. A scientific steering committee
was formed to review the initial specifications of the design provided
by Robert Dickinson, Gordon Bonan, Xubin Zeng, and Yongjiu Dai and to
facilitate further development. Steering committee members were selected
so as to provide guidance and expertise in disciplines not generally
well-represented in land surface models (e.g., carbon cycling,
ecological modeling, hydrology, and river routing) and included
scientists from NCAR, the university community, and government
laboratories (R. Dickinson, G. Bonan, X. Zeng, Paul Dirmeyer, Jay
Famiglietti, Jon Foley, and Paul Houser).

The specifications for the new model, designated the Common Land Model,
were discussed and agreed upon at the June 1998 CCSM Workshop LMWG
meeting. An initial code was developed by Y. Dai and was examined in
March 1999 by Mike Bosilovich, P. Dirmeyer, and P. Houser. At this point
an extensive period of code testing was initiated. Keith Oleson, Y. Dai,
Adam Schlosser, and P. Houser presented preliminary results of offline
1-dimensional testing at the June 1999 CCSM Workshop LMWG meeting.
Results from more extensive offline testing at plot, catchment, and
large scale (up to global) were presented by Y. Dai, A. Schlosser, K.
Oleson, M. Bosilovich, Zong-Liang Yang, Ian Baker, P. Houser, and P.
Dirmeyer at the LMWG meeting hosted by COLA (Center for
Ocean-Land-Atmosphere Studies) in November 1999. Field data used for
validation included sites adopted by the Project for Intercomparison of
Land-surface Parameterization Schemes ({\hyperref[\detokenize{tech_note/References/CLM50_Tech_Note_References:henderson-sellersetal1993}]{\sphinxcrossref{\DUrole{std,std-ref}{Henderson-Sellers et al. 1993}}}})
(Cabauw, Valdai, Red-Arkansas river basin) and others {[}FIFE ({\hyperref[\detokenize{tech_note/References/CLM50_Tech_Note_References:sellersetal1988}]{\sphinxcrossref{\DUrole{std,std-ref}{Sellers et
al. 1988}}}}), BOREAS {\hyperref[\detokenize{tech_note/References/CLM50_Tech_Note_References:sellersetal1995}]{\sphinxcrossref{\DUrole{std,std-ref}{(Sellers et al. 1995}}}}), HAPEX-MOBILHY ({\hyperref[\detokenize{tech_note/References/CLM50_Tech_Note_References:andreetal1986}]{\sphinxcrossref{\DUrole{std,std-ref}{André et al.
1986}}}}), ABRACOS ({\hyperref[\detokenize{tech_note/References/CLM50_Tech_Note_References:gashetal1996}]{\sphinxcrossref{\DUrole{std,std-ref}{Gash et al. 1996}}}}), Sonoran Desert ({\hyperref[\detokenize{tech_note/References/CLM50_Tech_Note_References:unlandetal1996}]{\sphinxcrossref{\DUrole{std,std-ref}{Unland et al. 1996}}}}),
GSWP ({\hyperref[\detokenize{tech_note/References/CLM50_Tech_Note_References:dirmeyeretal1999}]{\sphinxcrossref{\DUrole{std,std-ref}{Dirmeyer et al. 1999}}}}){]}. Y. Dai also presented results from a
preliminary coupling of the Common Land Model to CCM3, indicating that
the land model could be successfully coupled to a climate model.

Results of coupled simulations using CCM3 and the Common Land Model were
presented by X. Zeng at the June 2000 CCSM Workshop LMWG meeting.
Comparisons with the NCAR LSM and observations indicated major
improvements to the seasonality of runoff, substantial reduction of a
summer cold bias, and snow depth. Some deficiencies related to runoff
and albedo were noted, however, that were subsequently addressed. Z.-L.
Yang and I. Baker demonstrated improvements in the simulation of snow
and soil temperatures. Sam Levis reported on efforts to incorporate a
river routing model to deliver runoff to the ocean model in CCSM. Soon
after the workshop, the code was delivered to NCAR for implementation
into the CCSM framework. Documentation for the Common Land Model is
provided by {\hyperref[\detokenize{tech_note/References/CLM50_Tech_Note_References:daietal2001}]{\sphinxcrossref{\DUrole{std,std-ref}{Dai et al. (2001)}}}} while the coupling with CCM3 is described
in {\hyperref[\detokenize{tech_note/References/CLM50_Tech_Note_References:zengetal2002}]{\sphinxcrossref{\DUrole{std,std-ref}{Zeng et al. (2002)}}}}. The model was introduced to the modeling
community in {\hyperref[\detokenize{tech_note/References/CLM50_Tech_Note_References:daietal2003}]{\sphinxcrossref{\DUrole{std,std-ref}{Dai et al. (2003)}}}}.


\subsubsection{CLM2}
\label{\detokenize{tech_note/Introduction/CLM50_Tech_Note_Introduction:clm2}}
Concurrent with the development of the Common Land Model, the NCAR LSM
was undergoing further development at NCAR in the areas of carbon
cycling, vegetation dynamics, and river routing. The preservation of
these advancements necessitated several modifications to the Common Land
Model. The biome-type land cover classification scheme was replaced with
a plant functional type (PFT) representation with the specification of
PFTs and leaf area index from satellite data ({\hyperref[\detokenize{tech_note/References/CLM50_Tech_Note_References:olesonbonan2000}]{\sphinxcrossref{\DUrole{std,std-ref}{Oleson and Bonan 2000}}}};
{\hyperref[\detokenize{tech_note/References/CLM50_Tech_Note_References:bonanetal2002a}]{\sphinxcrossref{\DUrole{std,std-ref}{Bonan et al. 2002a, b}}}}). This also required modifications to
parameterizations for vegetation albedo and vertical burying of
vegetation by snow. Changes were made to canopy scaling, leaf
physiology, and soil water limitations on photosynthesis to resolve
deficiencies indicated by the coupling to a dynamic vegetation model.
Vertical heterogeneity in soil texture was implemented to improve
coupling with a dust emission model. A river routing model was
incorporated to improve the fresh water balance over oceans. Numerous
modest changes were made to the parameterizations to conform to the
strict energy and water balance requirements of CCSM. Further
substantial software development was also required to meet coding
standards. The resulting model was adopted in May 2002 as the Community
Land Model (CLM2) for use with the Community Atmosphere Model (CAM2, the
successor to CCM3) and version 2 of the Community Climate System Model
(CCSM2).

K. Oleson reported on initial results from a coupling of CCM3 with CLM2
at the June 2001 CCSM Workshop LMWG meeting. Generally, the CLM2
preserved most of the improvements seen in the Common Land Model,
particularly with respect to surface air temperature, runoff, and snow.
These simulations are documented in {\hyperref[\detokenize{tech_note/References/CLM50_Tech_Note_References:bonanetal2002a}]{\sphinxcrossref{\DUrole{std,std-ref}{Bonan et al. (2002a)}}}}. Further small
improvements to the biogeophysical parameterizations, ongoing software
development, and extensive analysis and validation within CAM2 and CCSM2
culminated in the release of CLM2 to the community in May 2002.

Following this release, Peter Thornton implemented changes to the model
structure required to represent carbon and nitrogen cycling in the
model. This involved changing data structures from a single vector of
spatially independent sub-grid patches to one that recognizes three
hierarchical scales within a model grid cell: land unit, snow/soil
column, and PFT. Furthermore, as an option, the model can be configured
so that PFTs can share a single soil column and thus “compete” for
water. This version of the model (CLM2.1) was released to the community
in February 2003. CLM2.1, without the compete option turned on, produced
only round off level changes when compared to CLM2.


\subsubsection{CLM3}
\label{\detokenize{tech_note/Introduction/CLM50_Tech_Note_Introduction:clm3}}
CLM3 implemented further software improvements related to performance
and model output, a re-writing of the code to support vector-based
computational platforms, and improvements in biogeophysical
parameterizations to correct deficiencies in the coupled model climate.
Of these parameterization improvements, two were shown to have a
noticeable impact on simulated climate. A variable aerodynamic
resistance for heat/moisture transfer from ground to canopy air that
depends on canopy density was implemented. This reduced unrealistically
high surface temperatures in semi-arid regions. The second improvement
added stability corrections to the diagnostic 2-m air temperature
calculation which reduced biases in this temperature. Competition
between PFTs for water, in which PFTs share a single soil column, is the
default mode of operation in this model version. CLM3 was released to
the community in June 2004. {\hyperref[\detokenize{tech_note/References/CLM50_Tech_Note_References:dickinsonetal2006}]{\sphinxcrossref{\DUrole{std,std-ref}{Dickinson et al. (2006)}}}}
describe the climate statistics of CLM3 when coupled to CCSM3.0.
{\hyperref[\detokenize{tech_note/References/CLM50_Tech_Note_References:hacketal2006}]{\sphinxcrossref{\DUrole{std,std-ref}{Hack et al. (2006)}}}} provide an analysis of selected
features of the land hydrological cycle.
{\hyperref[\detokenize{tech_note/References/CLM50_Tech_Note_References:lawrenceetal2007}]{\sphinxcrossref{\DUrole{std,std-ref}{Lawrence et al. (2007)}}}} examine the impact of
changes in CLM3
hydrological parameterizations on partitioning of evapotranspiration
(ET) and its effect on the timescales of ET response to precipitation
events, interseasonal soil moisture storage, soil moisture memory, and
land-atmosphere coupling. {\hyperref[\detokenize{tech_note/References/CLM50_Tech_Note_References:qianetal2006}]{\sphinxcrossref{\DUrole{std,std-ref}{Qian et al. (2006)}}}} evaluate CLM3’s performance
in simulating soil moisture content, runoff, and river discharge when
forced by observed precipitation, temperature and other atmospheric
data.


\subsubsection{CLM3.5}
\label{\detokenize{tech_note/Introduction/CLM50_Tech_Note_Introduction:clm3-5}}
Although the simulation of land surface climate by CLM3 was in many ways
adequate, most of the unsatisfactory aspects of the simulated climate
noted by the above studies could be traced directly to deficiencies in
simulation of the hydrological cycle. In 2004, a project was initiated
to improve the hydrology in CLM3 as part of the development of CLM
version 3.5. A selected set of promising approaches to alleviating the
hydrologic biases in CLM3 were tested and implemented. These included
new surface datasets based on Moderate Resolution Imaging
Spectroradiometer (MODIS) products, new parameterizations for canopy
integration, canopy interception, frozen soil, soil water availability,
and soil evaporation, a TOPMODEL-based model for surface and subsurface
runoff, a groundwater model for determining water table depth, and the
introduction of a factor to simulate nitrogen limitation on plant
productivity. {\hyperref[\detokenize{tech_note/References/CLM50_Tech_Note_References:olesonetal2008a}]{\sphinxcrossref{\DUrole{std,std-ref}{Oleson et al. (2008a)}}}} show that CLM3.5 exhibits
significant improvements over CLM3 in its partitioning of global ET
which result in wetter soils, less plant water stress, increased
transpiration and photosynthesis, and an improved annual cycle of total
water storage. Phase and amplitude of the runoff annual cycle is
generally improved. Dramatic improvements in vegetation biogeography
result when CLM3.5 is coupled to a dynamic global vegetation model.
{\hyperref[\detokenize{tech_note/References/CLM50_Tech_Note_References:stocklietal2008}]{\sphinxcrossref{\DUrole{std,std-ref}{Stöckli et al. (2008)}}}} examine the performance of CLM3.5 at local scales
by making use of a network of long-term ground-based ecosystem
observations {[}FLUXNET ({\hyperref[\detokenize{tech_note/References/CLM50_Tech_Note_References:baldocchietal2001}]{\sphinxcrossref{\DUrole{std,std-ref}{Baldocchi et al. 2001}}}}){]}. Data from 15 FLUXNET
sites were used to demonstrate significantly improved soil hydrology and
energy partitioning in CLM3.5. CLM3.5 was released to the community in
May, 2007.


\subsubsection{CLM4}
\label{\detokenize{tech_note/Introduction/CLM50_Tech_Note_Introduction:clm4}}
The motivation for the next version of the model, CLM4, was to
incorporate several recent scientific advances in the understanding and
representation of land surface processes, expand model capabilities, and
improve surface and atmospheric forcing datasets ({\hyperref[\detokenize{tech_note/References/CLM50_Tech_Note_References:lawrenceetal2011}]{\sphinxcrossref{\DUrole{std,std-ref}{Lawrence et al. 2011}}}}).
Included in the first category are more sophisticated representations of
soil hydrology and snow processes. In particular, new treatments of soil
column-groundwater interactions, soil evaporation, aerodynamic
parameters for sparse/dense canopies, vertical burial of vegetation by
snow, snow cover fraction and aging, black carbon and dust deposition,
and vertical distribution of solar energy for snow were implemented.
Major new capabilities in the model include a representation of the
carbon-nitrogen cycle (CLM4CN, see next paragraph for additional
information), the ability to model land cover change in a transient
mode, inclusion of organic soil and deep soil into the existing mineral
soil treatment to enable more realistic modeling of permafrost, an urban
canyon model to contrast rural and urban energy balance and climate
(CLMU), and an updated biogenic volatile organic compounds (BVOC) model.
Other modifications of note include refinement of the global PFT,
wetland, and lake distributions, more realistic optical properties for
grasslands and croplands, and an improved diurnal cycle and spectral
distribution of incoming solar radiation to force the model in land-only
mode.

Many of the ideas incorporated into the carbon and nitrogen cycle
component of CLM4 derive from the earlier development of the land-only
ecosystem process model Biome-BGC (Biome BioGeochemical Cycles),
originating at the Numerical Terradynamic Simulation Group (NTSG) at the
University of Montana, under the guidance of Prof. Steven Running.
Biome-BGC itself is an extension of an earlier model, Forest-BGC
({\hyperref[\detokenize{tech_note/References/CLM50_Tech_Note_References:runningcoughlan1988}]{\sphinxcrossref{\DUrole{std,std-ref}{Running and Coughlan, 1988}}}}; {\hyperref[\detokenize{tech_note/References/CLM50_Tech_Note_References:runninggower1991}]{\sphinxcrossref{\DUrole{std,std-ref}{Running and Gower, 1991}}}}), which
simulates water, carbon, and, to a limited extent, nitrogen fluxes for
forest ecosystems. Forest-BGC was designed to be driven by remote
sensing inputs of vegetation structure, and so used a diagnostic
(prescribed) leaf area index, or, in the case of the dynamic allocation
version of the model ({\hyperref[\detokenize{tech_note/References/CLM50_Tech_Note_References:runninggower1991}]{\sphinxcrossref{\DUrole{std,std-ref}{Running and Gower, 1991}}}}), prescribed maximum
leaf area index.

Biome-BGC expanded on the Forest-BGC logic by introducing a more
mechanistic calculation of leaf and canopy scale photosynthesis ({\hyperref[\detokenize{tech_note/References/CLM50_Tech_Note_References:huntrunning1992}]{\sphinxcrossref{\DUrole{std,std-ref}{Hunt
and Running, 1992}}}}), and extending the physiological parameterizations
to include multiple woody and non-woody vegetation types ({\hyperref[\detokenize{tech_note/References/CLM50_Tech_Note_References:huntetal1996}]{\sphinxcrossref{\DUrole{std,std-ref}{Hunt et al.
1996}}}}; {\hyperref[\detokenize{tech_note/References/CLM50_Tech_Note_References:runninghunt1993}]{\sphinxcrossref{\DUrole{std,std-ref}{Running and Hunt, 1993}}}}). Later versions of Biome-BGC introduced
more mechanistic descriptions of belowground carbon and nitrogen cycles,
nitrogen controls on photosynthesis and decomposition, sunlit and shaded
canopies, vertical gradient in leaf morphology, and explicit treatment
of fire and harvest disturbance and regrowth dynamics ({\hyperref[\detokenize{tech_note/References/CLM50_Tech_Note_References:kimballetal1997}]{\sphinxcrossref{\DUrole{std,std-ref}{Kimball et al.
1997}}}}; {\hyperref[\detokenize{tech_note/References/CLM50_Tech_Note_References:thornton1998}]{\sphinxcrossref{\DUrole{std,std-ref}{Thornton, 1998}}}}; {\hyperref[\detokenize{tech_note/References/CLM50_Tech_Note_References:thorntonetal2002}]{\sphinxcrossref{\DUrole{std,std-ref}{Thornton et al. 2002}}}}; {\hyperref[\detokenize{tech_note/References/CLM50_Tech_Note_References:whiteetal2000}]{\sphinxcrossref{\DUrole{std,std-ref}{White et al. 2000}}}}).
Biome-BGC version 4.1.2 ({\hyperref[\detokenize{tech_note/References/CLM50_Tech_Note_References:thorntonetal2002}]{\sphinxcrossref{\DUrole{std,std-ref}{Thornton et al. 2002}}}}) provided a point of
departure for integrating new biogeochemistry components into CLM4.

CLM4 was released to the community in June, 2010 along with the
Community Climate System Model version 4 (CCSM4). CLM4 is used in CCSM4,
CESM1, CESM1.1, and remains available as the default land component
model option for coupled simulations in CESM1.2.


\subsubsection{CLM4.5}
\label{\detokenize{tech_note/Introduction/CLM50_Tech_Note_Introduction:clm4-5}}
The motivations for the development of CLM4.5 were similar to those for CLM4:
incorporate several recent scientific advances in the understanding and
representation of land surface processes, expand model capabilities, and
improve surface and atmospheric forcing datasets.

Specifically, several parameterizations were revised to reflect new
scientific understanding and in an attempt to reduce biases identified
in CLM4 simulations including low soil carbon stocks especially in the
Arctic, excessive tropical GPP and unrealistically low Arctic GPP, a dry
soil bias in Arctic soils, unrealistically high LAI in the tropics, a
transient 20\({}^{th}\) century carbon response that was
inconsistent with observational estimates, and several other more minor
problems or biases.

The main modifications include updates to canopy processes including a
revised canopy radiation scheme and canopy scaling of leaf processes,
co-limitations on photosynthesis, revisions to photosynthetic parameters
({\hyperref[\detokenize{tech_note/References/CLM50_Tech_Note_References:bonanetal2011}]{\sphinxcrossref{\DUrole{std,std-ref}{Bonan et al. 2011}}}}), temperature acclimation of photosynthesis, and
improved stability of the iterative solution in the photosynthesis and
stomatal conductance model ({\hyperref[\detokenize{tech_note/References/CLM50_Tech_Note_References:sunetal2012}]{\sphinxcrossref{\DUrole{std,std-ref}{Sun et al. 2012}}}}). Hydrology updates included
modifications such that hydraulic properties of frozen soils are
determined by liquid water content only rather than total water content
and the introduction of an ice impedance function, and other corrections
that increase the consistency between soil water state and water table
position and allow for a perched water table above icy permafrost ground
({\hyperref[\detokenize{tech_note/References/CLM50_Tech_Note_References:swensonetal2012}]{\sphinxcrossref{\DUrole{std,std-ref}{Swenson et al. 2012}}}}). A new snow cover fraction parameterization is
incorporated that reflects the hysteresis in fractional snow cover for a
given snow depth between accumulation and melt phases ({\hyperref[\detokenize{tech_note/References/CLM50_Tech_Note_References:swensonlawrence2012}]{\sphinxcrossref{\DUrole{std,std-ref}{Swenson and
Lawrence, 2012}}}}). The lake model in CLM4 was replaced with a completely
revised and more realistic lake model ({\hyperref[\detokenize{tech_note/References/CLM50_Tech_Note_References:subinetal2012a}]{\sphinxcrossref{\DUrole{std,std-ref}{Subin et al. 2012a}}}}). A surface
water store was introduced, replacing the wetland land unit and
permitting prognostic wetland distribution modeling. The surface
energy fluxes are calculated separately ({\hyperref[\detokenize{tech_note/References/CLM50_Tech_Note_References:swensonlawrence2012}]{\sphinxcrossref{\DUrole{std,std-ref}{Swenson and Lawrence, 2012}}}}) for
snow-covered, water-covered, and snow/water-free portions of vegetated
and crop land units, and snow-covered and snow-free portions of glacier
land units. Globally constant river flow velocity is replaced with
variable flow velocity based on mean grid cell slope. A vertically
resolved soil biogeochemistry scheme is introduced with base
decomposition rates modified by soil temperature, water, and oxygen
limitations and also including vertical mixing of soil carbon and
nitrogen due to bioturbation, cryoturbation, and diffusion ({\hyperref[\detokenize{tech_note/References/CLM50_Tech_Note_References:kovenetal2013}]{\sphinxcrossref{\DUrole{std,std-ref}{Koven et al.
2013}}}}). The litter and soil carbon and nitrogen pool structure as well as
nitrification and denitrification that were modified based on the Century
model. Biological fixation was revised to distribute fixation more
realistically over the year ({\hyperref[\detokenize{tech_note/References/CLM50_Tech_Note_References:kovenetal2013}]{\sphinxcrossref{\DUrole{std,std-ref}{Koven et al. 2013}}}}). The fire model was
replaced with a model that includes representations of natural and
anthropogenic triggers and suppression as well as agricultural,
deforestation, and peat fires ({\hyperref[\detokenize{tech_note/References/CLM50_Tech_Note_References:lietal2012a}]{\sphinxcrossref{\DUrole{std,std-ref}{Li et al. 2012a,b}}}}; {\hyperref[\detokenize{tech_note/References/CLM50_Tech_Note_References:lietal2013a}]{\sphinxcrossref{\DUrole{std,std-ref}{Li et al. 2013a}}}}). The
biogenic volatile organic compounds model is updated to MEGAN2.1
({\hyperref[\detokenize{tech_note/References/CLM50_Tech_Note_References:guentheretal2012}]{\sphinxcrossref{\DUrole{std,std-ref}{Guenther et al. 2012}}}}).

Additions to the model include a methane production, oxidation, and
emissions model ({\hyperref[\detokenize{tech_note/References/CLM50_Tech_Note_References:rileyetal2011a}]{\sphinxcrossref{\DUrole{std,std-ref}{Riley et al. 2011a}}}}) and an extension of the crop model
to include interactive fertilization, organ pools ({\hyperref[\detokenize{tech_note/References/CLM50_Tech_Note_References:drewniaketal2013}]{\sphinxcrossref{\DUrole{std,std-ref}{Drewniak et al.
2013}}}}), and irrigation ({\hyperref[\detokenize{tech_note/References/CLM50_Tech_Note_References:sacksetal2009}]{\sphinxcrossref{\DUrole{std,std-ref}{Sacks et al. 2009}}}}). Elements of the Variable
Infiltration Capacity (VIC) model are included as an alternative
optional runoff generation scheme ({\hyperref[\detokenize{tech_note/References/CLM50_Tech_Note_References:lietal2011}]{\sphinxcrossref{\DUrole{std,std-ref}{Li et al. 2011}}}}). There is also an
option to run with a multilayer canopy ({\hyperref[\detokenize{tech_note/References/CLM50_Tech_Note_References:bonanetal2012}]{\sphinxcrossref{\DUrole{std,std-ref}{Bonan et al. 2012}}}}). Multiple
urban density classes, rather than the single dominant urban density
class used in CLM4, are modeled in the urban land unit. Carbon
(\({}^{13}\)C and \({}^{14}\)C) isotopes are enabled ({\hyperref[\detokenize{tech_note/References/CLM50_Tech_Note_References:kovenetal2013}]{\sphinxcrossref{\DUrole{std,std-ref}{Koven
et al. 2013}}}}). Minor changes include a switch of the C3 Arctic grass and
shrub phenology from stress deciduous to seasonal deciduous and a change
in the glacier bare ice albedo to better reflect recent estimates.
Finally, the carbon and nitrogen cycle spinup is accelerated and
streamlined with a revised spinup method, though the spinup timescale
remains long.

Finally, the predominantly low resolution input data for provided with
CLM4 to create CLM4 surface datasets is replaced with newer and higher
resolution input datasets where possible (see section \hyperref[\detokenize{tech_note/Ecosystem/CLM50_Tech_Note_Ecosystem:surface-data}]{\ref{\detokenize{tech_note/Ecosystem/CLM50_Tech_Note_Ecosystem:surface-data}}}
for details). The default meteorological forcing dataset provided with CLM4
({\hyperref[\detokenize{tech_note/References/CLM50_Tech_Note_References:qianetal2006}]{\sphinxcrossref{\DUrole{std,std-ref}{Qian et al. 2006)}}}} is replaced with the 1901-2010
CRUNCEP forcing dataset (see Chapter \hyperref[\detokenize{tech_note/Land-Only_Mode/CLM50_Tech_Note_Land-Only_Mode:rst-land-only-mode}]{\ref{\detokenize{tech_note/Land-Only_Mode/CLM50_Tech_Note_Land-Only_Mode:rst-land-only-mode}}}) for CLM4.5,
though users can also still use the {\hyperref[\detokenize{tech_note/References/CLM50_Tech_Note_References:qianetal2006}]{\sphinxcrossref{\DUrole{std,std-ref}{Qian et al. (2006)}}}}
dataset or other alternative forcing datasets.

CLM4.5 was released to the community in June 2013 along with the
Community Earth System Model version 1.2 (CESM1.2).


\subsubsection{CLM5.0}
\label{\detokenize{tech_note/Introduction/CLM50_Tech_Note_Introduction:clm5-0}}
Developments for CLM5.0 build on the progress made in CLM4.5.  Most major components of the model have been updated with particularly
notable changes made to soil and plant hydrology, snow density, river modeling, carbon and nitrogen cycling and coupling,
and crop modeling.
Much of the focus of development centered on a
push towards more mechanistic treatment of key processes, in addition to more comprehensive and explicit representation
of land use and land-cover change. Prior versions of CLM included relatively few options for physics parameterizations or structure.
In CLM5, where new parameterizations or model decisions were made, in most cases, the CLM4.5 parameterization was maintained so that users could switch back and forth between different parameterizations via namelist control where appropriate or desirable.  Throughout the CLM5 Technical Descpription, in general only the default parameterization for any given process is described.  Readers are referred to the CLM4.5 or CLM4 Technical Descriptions for detailed descriptions of non-default parameterizations.

The hydrology updates include the introduction of a dry surface layer-based soil evaporation resistance parameterization {\hyperref[\detokenize{tech_note/References/CLM50_Tech_Note_References:swensonlawrence2014}]{\sphinxcrossref{\DUrole{std,std-ref}{(Swenson and Lawrence, 2014)}}}} and a revised canopy interception parameterization.  Canopy interception is now divided into liquid and solid phases, with the intercepted snow subject to unloading events due to wind or above-freezing temperatures.  The snow-covered fraction of the canopy is used within the canopy radiation and surface albedo calculation.  Instead of applying a spatially uniform soil thickness, soil thickness can vary in space {\hyperref[\detokenize{tech_note/References/CLM50_Tech_Note_References:brunkeetal2016}]{\sphinxcrossref{\DUrole{std,std-ref}{(Brunke et al. 2016}}}} and {\hyperref[\detokenize{tech_note/References/CLM50_Tech_Note_References:swensonlawrence2015}]{\sphinxcrossref{\DUrole{std,std-ref}{Swenson and Lawrence, 2015)}}}} and is set to values within a range of 0.4m to 8.5m depth, derived from a spatially explicit soil thickness data product {\hyperref[\detokenize{tech_note/References/CLM50_Tech_Note_References:pelletieretal2016}]{\sphinxcrossref{\DUrole{std,std-ref}{(Pelletier et al., 2016)}}}}.  The explicit treatment of soil thickness allows for the deprecation of the unconfined aquifer parameterization used in CLM4.5, which is replaced with a zero flux boundary condition and explicit modeling of both the saturated and unsaturated zones.  The default model soil layer resolution is increased, especially within the top 3m, to more explicitly represent active layer thickness within the permafrost zone.  Rooting profiles were used inconsistently in CLM4.5 with {\hyperref[\detokenize{tech_note/References/CLM50_Tech_Note_References:zeng2001}]{\sphinxcrossref{\DUrole{std,std-ref}{Zeng (2001)}}}} profiles used for water and {\hyperref[\detokenize{tech_note/References/CLM50_Tech_Note_References:jacksonetal1996}]{\sphinxcrossref{\DUrole{std,std-ref}{Jackson et al. (1996)}}}} profiles used for carbon inputs.  For CLM5, the Jackson et al. (1996) rooting profiles are used for both water and carbon.  Roots are deepened for the broadleaf evergreen tropical tree and broadleaf deciduous tropical tree types.  Finally, an adaptive time-stepping solution to the Richard’s equation is introduced, which improves the accuracy and stability of the numerical soil water solution.  The River Transport Model (RTM) is replaced with the Model for Scale Adaptive River Transport (MOSART, {\hyperref[\detokenize{tech_note/References/CLM50_Tech_Note_References:lietal2013b}]{\sphinxcrossref{\DUrole{std,std-ref}{Li et al., 2013b)}}}} in which surface runoff is routed across hillslopes and then discharged along with subsurface runoff into a tributary subnetwork before entering the main channel.

Several changes are included that are mainly targeted at improving the simulation of surface mass balance over ice
sheets.  The fresh snow density parameterization is updated to more realistically capture the temperature effects and to additionally account for wind effects on new snow density {\hyperref[\detokenize{tech_note/References/CLM50_Tech_Note_References:vankampenhoutetal2017}]{\sphinxcrossref{\DUrole{std,std-ref}{(van Kampenhout et al., 2017)}}}}.  The maximum number of snow layers and snow amount is increased from 5 layers and 1m snow water equivalent to 12 layers and 10m snow water equivalent to allow for the formation of firn in regions of persistent snow-cover (e.g., glaciers and ice sheets) {\hyperref[\detokenize{tech_note/References/CLM50_Tech_Note_References:vankampenhoutetal2017}]{\sphinxcrossref{\DUrole{std,std-ref}{(van Kampenhout et al., 2017)}}}}.  The CISM2 ice sheet model is active for Greenland by default with one-way coupling (surface mass balance impacts ice sheet dynamics, but ice sheet dynamics do not feedback onto surface elevation).  Two-way coupling can be activated through a namelist switch.  The introduction in CLM5 of the capability to
dynamically adjust landunit weights means that a glacier can initiate, grow, shrink, or disappear during
a simulation when two-way coupling is active.  Multiple elevation classes (10 elevation classes by default) and associated temperature, rain/snow partitioning, and downwelling longwave downscaling are used for glacier landunits to account for the strong topographic elevation heterogeneity over glaciers and ice sheets.

A plant hydraulic stress routine is introduced which explicitly models water transport through the vegetation according to a simple hydraulic framework (Kennedy et al., to be submitted).  The water supply equations are used to solve for vegetation water potential forced by transpiration demand and a set of layer-by-layer soil water potentials.  Stomatal conductance, therefore, is a function of prognostic leaf water potential.  Water stress is calculated as the ratio of attenuated stomatal conductance to maximum stomatal conductance.  An emergent feature of the plant hydraulics is soil hydraulic redistribution.  In CLM5, maximum stomatal conductance is obtained from the Medlyn conductance model {\hyperref[\detokenize{tech_note/References/CLM50_Tech_Note_References:medlynetal2011}]{\sphinxcrossref{\DUrole{std,std-ref}{(Medlyn et al., 2011)}}}}, rather than the Ball-Berry stomatal conductance model that was utilized in CLM4.5 and prior versions of the model. The Medlyn stomatal conductance model is preferred mainly for it’s more realistic behavior at low humidity levels {\hyperref[\detokenize{tech_note/References/CLM50_Tech_Note_References:rogersetal2017}]{\sphinxcrossref{\DUrole{std,std-ref}{(Rogers et al., 2017)}}}}. The stress deciduous vegetation phenology trigger is augmented with a antecedent precipitation requirement {\hyperref[\detokenize{tech_note/References/CLM50_Tech_Note_References:dahlinetal2015}]{\sphinxcrossref{\DUrole{std,std-ref}{(Dahlin et al. 2015)}}}}.

Plant nutrient dynamics are substantially updated to resolve several deficiencies with the CLM4 and CLM4.5 nutrient cycling representation.  The Fixation and Update of Nitrogen (FUN) model based on the work of {\hyperref[\detokenize{tech_note/References/CLM50_Tech_Note_References:fisheretal2010}]{\sphinxcrossref{\DUrole{std,std-ref}{Fisher et al. (2010)}}}}, {\hyperref[\detokenize{tech_note/References/CLM50_Tech_Note_References:brzosteketal2014}]{\sphinxcrossref{\DUrole{std,std-ref}{Brzostek et al. (2014)}}}}, and {\hyperref[\detokenize{tech_note/References/CLM50_Tech_Note_References:shietal2016}]{\sphinxcrossref{\DUrole{std,std-ref}{Shi et al. (2016)}}}} is incorporated.  The concept of FUN is that in most cases, N uptake requires the expenditure of energy in the form of carbon, and further, that there are numerous potential sources of N in the environment which a plant may exchange for carbon. The ratio of carbon expended to N acquired is therefore the cost, or exchange rate, of N acquisition.  FUN calculates the rate of symbiotic N fixation, with this N passed straight to the plant, not the mineral N pool.  Separately, CLM5 also calculates rates of symbiotic (or free living) N fixation as a function of evapotranspiration ({\hyperref[\detokenize{tech_note/References/CLM50_Tech_Note_References:clevelandetal1999}]{\sphinxcrossref{\DUrole{std,std-ref}{Cleveland et al. 1999}}}}), which
is added to the soil inorganic ammonium (NH$_{\text{4}}$$^{\text{+}}$) pool.  The static plant carbon:nitrogen (C:N) ratios utilized in CLM4 and CLM4.5 are replaced with variable plant C:N ratios which
allows plants to adjust their C:N ratio, and therefore their leaf nitrogen content, with the cost of N uptake {\hyperref[\detokenize{tech_note/References/CLM50_Tech_Note_References:ghimireetal2016}]{\sphinxcrossref{\DUrole{std,std-ref}{(Ghimire et al. 2016)}}}}.
The implementation of a flexible C:N ratio means that the model no longer relies on instantaneous downregulation
of potential photosynthesis rates based on soil mineral nitrogen availability to represent nutrient limitation.  Furthermore, stomatal conductance
is now based on the N-limited photosynthesis rather than on potential photosynthesis. Finally, the Leaf Use of
Nitrogen for Assimilation (LUNA, {\hyperref[\detokenize{tech_note/References/CLM50_Tech_Note_References:xuetal2012}]{\sphinxcrossref{\DUrole{std,std-ref}{Xu et al., 2012}}}} and {\hyperref[\detokenize{tech_note/References/CLM50_Tech_Note_References:alietal2016}]{\sphinxcrossref{\DUrole{std,std-ref}{Ali et al., 2016)}}}} model is incorporated.  The LUNA model calculates
photosynthetic capacity based on optimization of the use of leaf nitrogen under different environmental conditions such that
light capture, carboxylation, and respiration are co-limiting.

CLM5 applies a fixed allocation scheme for woody vegetation.  The decision to use a fixed allocation scheme in CLM5, rather than a dynamic NPP-based allocation scheme, as was used in CLM4 and CLM4.5, was driven by the fact that observations indicate that biomass saturates with increasing productivity, in contrast to the behavior in CLM4 and CLM4.5 where biomass continuously increases with increasing productivity ({\hyperref[\detokenize{tech_note/References/CLM50_Tech_Note_References:negronjuarezetal2015}]{\sphinxcrossref{\DUrole{std,std-ref}{Negron-Juarez et al., 2015}}}}).  Soil carbon decomposition processes are unchanged in CLM5, but a new metric for apparent soil carbon turnover times ({\hyperref[\detokenize{tech_note/References/CLM50_Tech_Note_References:kovenetal2017}]{\sphinxcrossref{\DUrole{std,std-ref}{Koven et al., 2017}}}}) suggested parameter changes that produce a weak intrinsic depth limitation on soil carbon turnover rates (rather than the strong depth limitaiton in CLM4.5) and that the thresholds for soil moisture limitation on soil carbon turnover rates in dry soils should be set at a wetter soil moisture level than that used in CLM4.5.

Representation of human management of the land (agriculture, wood harvest) is augmented in several ways. The CLM4.5 crop model is extended to operate globally through the addition of rice and sugarcane as well as tropical varieties of corn and soybean {\hyperref[\detokenize{tech_note/References/CLM50_Tech_Note_References:badgeranddirmeyer2015}]{\sphinxcrossref{\DUrole{std,std-ref}{(Badger and Dirmeyer, 2015}}}} and {\hyperref[\detokenize{tech_note/References/CLM50_Tech_Note_References:levisetal2016}]{\sphinxcrossref{\DUrole{std,std-ref}{Levis et al., 2016)}}}}.  These crop types are added to the existing temperate corn, temperature soybean, spring wheat, and cotton crop types.
Fertilization rates and irrigation equipped area updated annually based on crop type and geographic region through an input dataset.  The irrigation trigger is updated.  Additional minor changes include crop phenological triggers that
vary by latitude for selected crop types, grain C and N is now removed at harvest to a 1-year product pool with
the carbon for the next season’s crop seed removed from the grain carbon at harvest.  Through the introduction of
the capability to dynamically adjust landunit weights during a simulation, the crop model can now be run coincidentally
with prescribed land use, which significantly expands the capabilities of the model.  Mass-based rather than area-based wood harvest is applied. Several heat stress indices for both urban and rural areas are calculated and output by default {\hyperref[\detokenize{tech_note/References/CLM50_Tech_Note_References:buzanetal2015}]{\sphinxcrossref{\DUrole{std,std-ref}{(Buzan et al., 2015)}}}}.   A more sophisticated and realistic building space heating and air conditioning submodel that prognoses interior building air temperature and includes more realistic space heating and air conditioning wasteheat factors
is incorporated.

The fire model is the same as utilized in CLM4.5 except that a modified scheme is used to estimate the dependence of fire occurrence and spread on fuel wetness for non-peat fires outside cropland and tropical closed forests {\hyperref[\detokenize{tech_note/References/CLM50_Tech_Note_References:lilawrence2017}]{\sphinxcrossref{\DUrole{std,std-ref}{(Li and Lawrence, 2017)}}}} and the dependence of agricultural fires on fuel load is removed.

Included with the release of CLM5.0 is a functionally supported version of the Functionally-Assembled Terrestrial Ecosystem Simulator (FATES, {\hyperref[\detokenize{tech_note/References/CLM50_Tech_Note_References:fisheretal2015}]{\sphinxcrossref{\DUrole{std,std-ref}{Fisher et al., 2015)}}}}. A major motivation of FATES is to allow the prediction of biome boundaries directly from plant physiological traits via their competitive interactions.  FATES is a cohort model of vegetation competition and co-existence, allowing a representation of the biosphere which accounts for the division of the land surface into successional stages, and for competition for light between height structured cohorts of representative trees of various plant functional types.  FATES is not active by default in CLM5.0.

Note that the classical dynamic global vegetation model (CLM-DGVM) that has been available within CLM4 and CLM4.5 remains available, though it is largely untested.  The technical description of the CLM-DGVM can be found within the CLM4.5 Technical Description ({\hyperref[\detokenize{tech_note/References/CLM50_Tech_Note_References:olesonetal2013}]{\sphinxcrossref{\DUrole{std,std-ref}{Oleson et al. 2013)}}}}.

During the course of the development of CLM5.0, it became clear that the increasing complexity of the model combined with the increasing number and range
of model development projects required updates to the underlying CLM infrastructure.  Many such software improvements
are included in CLM5 including a partial transition to an object-oriented modular software structure. Many hard coded model
parameters have been extracted into either the parameter file or the CLM namelist, which allows users to more readily calibrate the model for use at
specific locations or to conduct parameter sensitivity studies.  As part of the effort to increase
the scientific utility of the code, in most instances older generation parameterizations (i.e., the parameterizations
available in CLM4 or CLM4.5) are retained under namelist switches, allowing the user to revert to CLM4.5
from the same code base or to revert individual parameterizations where the old parameterizations are compatible with the new code.  Finally, multiple vertical soil layer structures
are defined and it is relatively easy to add additional structures.


\subsection{Biogeophysical and Biogeochemical Processes}
\label{\detokenize{tech_note/Introduction/CLM50_Tech_Note_Introduction:biogeophysical-and-biogeochemical-processes}}
Biogeophysical and biogeochemical processes are simulated for each
subgrid land unit, column, and plant functional type (PFT) independently
and each subgrid unit maintains its own prognostic variables (see
section \hyperref[\detokenize{tech_note/Ecosystem/CLM50_Tech_Note_Ecosystem:surface-heterogeneity-and-data-structure}]{\ref{\detokenize{tech_note/Ecosystem/CLM50_Tech_Note_Ecosystem:surface-heterogeneity-and-data-structure}}} for definitions
of subgrid units). The same atmospheric
forcing is used to force all subgrid units within a grid cell. The
surface variables and fluxes required by the atmosphere are obtained by
averaging the subgrid quantities weighted by their fractional areas. The
processes simulated include (\hyperref[\detokenize{tech_note/Introduction/CLM50_Tech_Note_Introduction:figure-land-processes}]{Figure \ref{\detokenize{tech_note/Introduction/CLM50_Tech_Note_Introduction:figure-land-processes}}}):
\begin{enumerate}
\item {} 
Surface characterization including land type heterogeneity and
ecosystem structure (Chapter \hyperref[\detokenize{tech_note/Ecosystem/CLM50_Tech_Note_Ecosystem:rst-surface-characterization-vertical-discretization-and-model-input-requirements}]{\ref{\detokenize{tech_note/Ecosystem/CLM50_Tech_Note_Ecosystem:rst-surface-characterization-vertical-discretization-and-model-input-requirements}}})

\item {} 
Absorption, reflection, and transmittance of solar radiation (Chapter
\hyperref[\detokenize{tech_note/Surface_Albedos/CLM50_Tech_Note_Surface_Albedos:rst-surface-albedos}]{\ref{\detokenize{tech_note/Surface_Albedos/CLM50_Tech_Note_Surface_Albedos:rst-surface-albedos}}}, \hyperref[\detokenize{tech_note/Radiative_Fluxes/CLM50_Tech_Note_Radiative_Fluxes:rst-radiative-fluxes}]{\ref{\detokenize{tech_note/Radiative_Fluxes/CLM50_Tech_Note_Radiative_Fluxes:rst-radiative-fluxes}}})

\item {} 
Absorption and emission of longwave radiation (Chapter \hyperref[\detokenize{tech_note/Radiative_Fluxes/CLM50_Tech_Note_Radiative_Fluxes:rst-radiative-fluxes}]{\ref{\detokenize{tech_note/Radiative_Fluxes/CLM50_Tech_Note_Radiative_Fluxes:rst-radiative-fluxes}}})

\item {} 
Momentum, sensible heat (ground and canopy), and latent heat (ground
evaporation, canopy evaporation, transpiration) fluxes (Chapter \hyperref[\detokenize{tech_note/Fluxes/CLM50_Tech_Note_Fluxes:rst-momentum-sensible-heat-and-latent-heat-fluxes}]{\ref{\detokenize{tech_note/Fluxes/CLM50_Tech_Note_Fluxes:rst-momentum-sensible-heat-and-latent-heat-fluxes}}})

\item {} 
Heat transfer in soil and snow including phase change (Chapter \hyperref[\detokenize{tech_note/Soil_Snow_Temperatures/CLM50_Tech_Note_Soil_Snow_Temperatures:rst-soil-and-snow-temperatures}]{\ref{\detokenize{tech_note/Soil_Snow_Temperatures/CLM50_Tech_Note_Soil_Snow_Temperatures:rst-soil-and-snow-temperatures}}})

\item {} 
Canopy hydrology (interception, throughfall, and drip) (Chapter \hyperref[\detokenize{tech_note/Hydrology/CLM50_Tech_Note_Hydrology:rst-hydrology}]{\ref{\detokenize{tech_note/Hydrology/CLM50_Tech_Note_Hydrology:rst-hydrology}}})

\item {} 
Soil hydrology (surface runoff, infiltration, redistribution of water
within the column, sub-surface drainage, groundwater) (Chapter \hyperref[\detokenize{tech_note/Hydrology/CLM50_Tech_Note_Hydrology:rst-hydrology}]{\ref{\detokenize{tech_note/Hydrology/CLM50_Tech_Note_Hydrology:rst-hydrology}}})

\item {} 
Snow hydrology (snow accumulation and melt, compaction, water
transfer between snow layers) (Chapter \hyperref[\detokenize{tech_note/Snow_Hydrology/CLM50_Tech_Note_Snow_Hydrology:rst-snow-hydrology}]{\ref{\detokenize{tech_note/Snow_Hydrology/CLM50_Tech_Note_Snow_Hydrology:rst-snow-hydrology}}})

\item {} 
Stomatal physiology, photosythetic capacity, and photosynthesis (Chapters \hyperref[\detokenize{tech_note/Photosynthesis/CLM50_Tech_Note_Photosynthesis:rst-stomatal-resistance-and-photosynthesis}]{\ref{\detokenize{tech_note/Photosynthesis/CLM50_Tech_Note_Photosynthesis:rst-stomatal-resistance-and-photosynthesis}}} and
\hyperref[\detokenize{tech_note/Photosynthetic_Capacity/CLM50_Tech_Note_Photosynthetic_Capacity:rst-photosynthetic-capacity}]{\ref{\detokenize{tech_note/Photosynthetic_Capacity/CLM50_Tech_Note_Photosynthetic_Capacity:rst-photosynthetic-capacity}}})

\item {} 
Plant hydraulics (Chapter \hyperref[\detokenize{tech_note/Plant_Hydraulics/CLM50_Tech_Note_Plant_Hydraulics:rst-plant-hydraulics}]{\ref{\detokenize{tech_note/Plant_Hydraulics/CLM50_Tech_Note_Plant_Hydraulics:rst-plant-hydraulics}}})

\item {} 
Lake temperatures and fluxes (Chapter \hyperref[\detokenize{tech_note/Lake/CLM50_Tech_Note_Lake:rst-lake-model}]{\ref{\detokenize{tech_note/Lake/CLM50_Tech_Note_Lake:rst-lake-model}}})

\item {} 
Glacier processes (Chapter \hyperref[\detokenize{tech_note/Glacier/CLM50_Tech_Note_Glacier:rst-glaciers}]{\ref{\detokenize{tech_note/Glacier/CLM50_Tech_Note_Glacier:rst-glaciers}}})

\item {} 
River routing and river flow (Chapter \hyperref[\detokenize{tech_note/MOSART/CLM50_Tech_Note_MOSART:rst-river-transport-model-rtm}]{\ref{\detokenize{tech_note/MOSART/CLM50_Tech_Note_MOSART:rst-river-transport-model-rtm}}})

\item {} 
Urban energy balance and climate (Chapter \hyperref[\detokenize{tech_note/Urban/CLM50_Tech_Note_Urban:rst-urban-model-clmu}]{\ref{\detokenize{tech_note/Urban/CLM50_Tech_Note_Urban:rst-urban-model-clmu}}})

\item {} 
Vegetation carbon and nitrogen allocation (Chapter \hyperref[\detokenize{tech_note/CN_Allocation/CLM50_Tech_Note_CN_Allocation:rst-cn-allocation}]{\ref{\detokenize{tech_note/CN_Allocation/CLM50_Tech_Note_CN_Allocation:rst-cn-allocation}}})

\item {} 
Vegetation phenology (Chapter \hyperref[\detokenize{tech_note/Vegetation_Phenology_Turnover/CLM50_Tech_Note_Vegetation_Phenology_Turnover:rst-vegetation-phenology-and-turnover}]{\ref{\detokenize{tech_note/Vegetation_Phenology_Turnover/CLM50_Tech_Note_Vegetation_Phenology_Turnover:rst-vegetation-phenology-and-turnover}}})

\item {} 
Plant respiration (Chapter \hyperref[\detokenize{tech_note/Plant_Respiration/CLM50_Tech_Note_Plant_Respiration:rst-plant-respiration}]{\ref{\detokenize{tech_note/Plant_Respiration/CLM50_Tech_Note_Plant_Respiration:rst-plant-respiration}}})

\item {} 
Soil and litter carbon decomposition (Chapter \hyperref[\detokenize{tech_note/Decomposition/CLM50_Tech_Note_Decomposition:rst-decomposition}]{\ref{\detokenize{tech_note/Decomposition/CLM50_Tech_Note_Decomposition:rst-decomposition}}})

\item {} 
Fixation and uptake of nitrogen (Chapter \hyperref[\detokenize{tech_note/FUN/CLM50_Tech_Note_FUN:rst-fun}]{\ref{\detokenize{tech_note/FUN/CLM50_Tech_Note_FUN:rst-fun}}})

\item {} 
External nitrogen cycling including deposition,
denitrification, leaching, and losses due to fire (Chapter \hyperref[\detokenize{tech_note/External_Nitrogen_Cycle/CLM50_Tech_Note_External_Nitrogen_Cycle:rst-external-nitrogen-cycle}]{\ref{\detokenize{tech_note/External_Nitrogen_Cycle/CLM50_Tech_Note_External_Nitrogen_Cycle:rst-external-nitrogen-cycle}}})

\item {} 
Plant mortality (Chapter \sphinxcode{rst\_Plant\_Mortality})

\item {} 
Fire ignition, suppression, spread, and emissions, including natural, deforestation, and
agricultural fire (Chapter \hyperref[\detokenize{tech_note/Fire/CLM50_Tech_Note_Fire:rst-fire}]{\ref{\detokenize{tech_note/Fire/CLM50_Tech_Note_Fire:rst-fire}}})

\item {} 
Methane production, oxidation, and emissions (Chapter \hyperref[\detokenize{tech_note/Methane/CLM50_Tech_Note_Methane:rst-methane-model}]{\ref{\detokenize{tech_note/Methane/CLM50_Tech_Note_Methane:rst-methane-model}}})

\item {} 
Crop dynamics, irrigation, and fertilization (Chapter \hyperref[\detokenize{tech_note/Crop_Irrigation/CLM50_Tech_Note_Crop_Irrigation:rst-crops-and-irrigation}]{\ref{\detokenize{tech_note/Crop_Irrigation/CLM50_Tech_Note_Crop_Irrigation:rst-crops-and-irrigation}}})

\item {} 
Land cover and land use change including wood harvest (Chapter \hyperref[\detokenize{tech_note/Transient_Landcover/CLM50_Tech_Note_Transient_Landcover:rst-transient-landcover-change}]{\ref{\detokenize{tech_note/Transient_Landcover/CLM50_Tech_Note_Transient_Landcover:rst-transient-landcover-change}}})

\item {} 
Biogenic volatile organic compound emissions (Chapter \hyperref[\detokenize{tech_note/BVOCs/CLM50_Tech_Note_BVOCs:rst-biogenic-volatile-organic-compounds-bvocs}]{\ref{\detokenize{tech_note/BVOCs/CLM50_Tech_Note_BVOCs:rst-biogenic-volatile-organic-compounds-bvocs}}})

\item {} 
Dust mobilization and deposition (Chapter \hyperref[\detokenize{tech_note/Dust/CLM50_Tech_Note_Dust:rst-dust-model}]{\ref{\detokenize{tech_note/Dust/CLM50_Tech_Note_Dust:rst-dust-model}}})

\item {} 
Carbon isotope fractionation (Chapter \hyperref[\detokenize{tech_note/Isotopes/CLM50_Tech_Note_Isotopes:rst-carbon-isotopes}]{\ref{\detokenize{tech_note/Isotopes/CLM50_Tech_Note_Isotopes:rst-carbon-isotopes}}})

\end{enumerate}

\begin{figure}[htbp]
\centering
\capstart

\noindent\sphinxincludegraphics{{image13}.png}
\caption{Land biogeophysical, biogeochemical, and landscape processes simulated by CLM (adapted from {\hyperref[\detokenize{tech_note/References/CLM50_Tech_Note_References:lawrenceetal2011}]{\sphinxcrossref{\DUrole{std,std-ref}{Lawrence et al. (2011)}}}} for CLM4.5).}\label{\detokenize{tech_note/Introduction/CLM50_Tech_Note_Introduction:figure-land-processes}}\label{\detokenize{tech_note/Introduction/CLM50_Tech_Note_Introduction:id1}}\end{figure}


\section{Surface Characterization, Vertical Discretization, and Model Input Requirements}
\label{\detokenize{tech_note/Ecosystem/CLM50_Tech_Note_Ecosystem::doc}}\label{\detokenize{tech_note/Ecosystem/CLM50_Tech_Note_Ecosystem:surface-characterization-vertical-discretization-and-model-input-requirements}}\label{\detokenize{tech_note/Ecosystem/CLM50_Tech_Note_Ecosystem:rst-surface-characterization-vertical-discretization-and-model-input-requirements}}

\subsection{Surface Characterization}
\label{\detokenize{tech_note/Ecosystem/CLM50_Tech_Note_Ecosystem:surface-characterization}}\label{\detokenize{tech_note/Ecosystem/CLM50_Tech_Note_Ecosystem:id1}}

\subsubsection{Surface Heterogeneity and Data Structure}
\label{\detokenize{tech_note/Ecosystem/CLM50_Tech_Note_Ecosystem:id2}}\label{\detokenize{tech_note/Ecosystem/CLM50_Tech_Note_Ecosystem:surface-heterogeneity-and-data-structure}}
Spatial land surface heterogeneity in CLM is represented as a nested
subgrid hierarchy in which grid cells are composed of multiple land
units, snow/soil columns, and PFTs (\hyperref[\detokenize{tech_note/Ecosystem/CLM50_Tech_Note_Ecosystem:figure-clm-subgrid-hierarchy}]{Figure \ref{\detokenize{tech_note/Ecosystem/CLM50_Tech_Note_Ecosystem:figure-clm-subgrid-hierarchy}}}).
Each grid cell can have
a different number of land units, each land unit can have a different
number of columns, and each column can have multiple PFTs. The first
subgrid level, the land unit, is intended to capture the broadest
spatial patterns of subgrid heterogeneity. The current land units are
glacier, lake, urban, vegetated, and crop (when the crop model option is
turned on). The land unit level can be used to further delineate these
patterns. For example, the urban land unit is divided into density
classes representing the tall building district, high density, and
medium density urban areas.

The second subgrid level, the column, is intended to capture potential
variability in the soil and snow state variables within a single land
unit. For example, the vegetated land unit could contain several columns
with independently evolving vertical profiles of soil water and
temperature. Similarly, the managed vegetation land unit can be
divided into two columns, irrigated and non-irrigated. The default snow/soil
column is represented by 25 layers for ground (with up to 20 of these layers classified as soil layers and the remaining layers classified as bedrock layers)  and up to 10 layers
for snow, depending on snow depth. The central characteristic of the
column subgrid level is that this is where the state variables for water
and energy in the soil and snow are defined, as well as the fluxes of
these components within the soil and snow. Regardless of the number and
type of PFTs occupying space on the column, the column physics operates
with a single set of upper boundary fluxes, as well as a single set of
transpiration fluxes from multiple soil levels. These boundary fluxes
are weighted averages over all PFTs. Currently, for glacier, lake, and
vegetated land units, a single column is assigned to each land unit. The
crop land unit is split into irrigated and unirrigated columns with a
single crop occupying each column. The urban land units have five
columns (roof, sunlit walls and shaded walls, and pervious and
impervious canyon floor) (Oleson et al. 2010b).

\begin{figure}[htbp]
\centering
\capstart

\noindent\sphinxincludegraphics{{image1}.png}
\caption{Configuration of the CLM subgrid hierarchy.  Box in upper right shows hypothetical subgrid distribution for a single grid cell.  Note that the Crop land unit is only used when the model is run with the crop model active. Abbreviations: TBD \textendash{} Tall Building District; HD \textendash{} High Density; MD \textendash{} Medium Density, G \textendash{} Glacier, L \textendash{} Lake, U \textendash{} Urban, C \textendash{} Crop, V \textendash{} Vegetated, PFT \textendash{} Plant Functional Type, Irr \textendash{} Irrigated, UIrr \textendash{} Unirrigated.  Red arrows indicate allowed land unit transitions.  Purple arrows indicate allowed patch-level transitions.}\label{\detokenize{tech_note/Ecosystem/CLM50_Tech_Note_Ecosystem:figure-clm-subgrid-hierarchy}}\label{\detokenize{tech_note/Ecosystem/CLM50_Tech_Note_Ecosystem:id14}}\end{figure}

The third subgrid level is referred to as the patch level. Patches can be PFTs or bare ground on the vegetated land unit
and crop functional types (CFTs) on the crop land unit.
The patch level is intended to capture the
biogeophysical and biogeochemical differences between broad categories
of plants in terms of their functional characteristics. On the vegetated
land unit, up to 16 possible PFTs that differ in physiology and
structure may coexist on a single column. All fluxes to and from the
surface are defined at the PFT level, as are the vegetation state
variables (e.g. vegetation temperature and canopy water storage). On the
crop land unit, typically, different crop types can be represented on each

crop land unit column (see Chapter \hyperref[\detokenize{tech_note/Crop_Irrigation/CLM50_Tech_Note_Crop_Irrigation:rst-crops-and-irrigation}]{\ref{\detokenize{tech_note/Crop_Irrigation/CLM50_Tech_Note_Crop_Irrigation:rst-crops-and-irrigation}}} for details).

In addition to state and flux variable data structures for conserved
components at each subgrid level (e.g., energy, water, carbon), each
subgrid level also has a physical state data structure for handling
quantities that are not involved in conservation checks (diagnostic
variables). For example, the urban canopy air temperature and humidity
are defined through physical state variables at the land unit level, the
number of snow layers and the soil roughness lengths are defined as
physical state variables at the column level, and the leaf area index
and the fraction of canopy that is wet are defined as physical state
variables at the PFT level.

The standard configuration of the model subgrid hierarchy is illustrated
in \hyperref[\detokenize{tech_note/Ecosystem/CLM50_Tech_Note_Ecosystem:figure-clm-subgrid-hierarchy}]{Figure \ref{\detokenize{tech_note/Ecosystem/CLM50_Tech_Note_Ecosystem:figure-clm-subgrid-hierarchy}}}. Here, only four PFTs are shown
associated with the single
column beneath the vegetated land unit but up to sixteen are possible.
The crop land unit is present only when the crop model is active.

Note that the biogeophysical processes related to soil and snow require
PFT level properties to be aggregated to the column level. For example,
the net heat flux into the ground is required as a boundary condition
for the solution of snow/soil temperatures (Chapter \hyperref[\detokenize{tech_note/Soil_Snow_Temperatures/CLM50_Tech_Note_Soil_Snow_Temperatures:rst-soil-and-snow-temperatures}]{\ref{\detokenize{tech_note/Soil_Snow_Temperatures/CLM50_Tech_Note_Soil_Snow_Temperatures:rst-soil-and-snow-temperatures}}}). This column
level property must be determined by aggregating the net heat flux from
all PFTs sharing the column. This is generally accomplished in the model
by computing a weighted sum of the desired quantity over all PFTs whose
weighting depends on the PFT area relative to all PFTs, unless otherwise
noted in the text.


\subsubsection{Vegetation Composition}
\label{\detokenize{tech_note/Ecosystem/CLM50_Tech_Note_Ecosystem:id3}}\label{\detokenize{tech_note/Ecosystem/CLM50_Tech_Note_Ecosystem:vegetation-composition}}
Vegetated surfaces are comprised of up to 15 possible plant functional
types (PFTs) plus bare ground (\hyperref[\detokenize{tech_note/Ecosystem/CLM50_Tech_Note_Ecosystem:table-plant-functional-types}]{Table \ref{\detokenize{tech_note/Ecosystem/CLM50_Tech_Note_Ecosystem:table-plant-functional-types}}}). An
additional PFT is added if
the irrigation model is active and six additional PFTs are added if the
crop model is active (Chapter \hyperref[\detokenize{tech_note/Crop_Irrigation/CLM50_Tech_Note_Crop_Irrigation:rst-crops-and-irrigation}]{\ref{\detokenize{tech_note/Crop_Irrigation/CLM50_Tech_Note_Crop_Irrigation:rst-crops-and-irrigation}}}). These
plant types differ in leaf and stem optical properties that determine reflection,
transmittance, and absorption of solar radiation (\hyperref[\detokenize{tech_note/Surface_Albedos/CLM50_Tech_Note_Surface_Albedos:table-plant-functional-type-optical-properties}]{Table \ref{\detokenize{tech_note/Surface_Albedos/CLM50_Tech_Note_Surface_Albedos:table-plant-functional-type-optical-properties}}}), root
distribution parameters that control the uptake of water from the soil
(\hyperref[\detokenize{tech_note/Plant_Hydraulics/CLM50_Tech_Note_Plant_Hydraulics:table-plant-functional-type-root-distribution-parameters}]{Table \ref{\detokenize{tech_note/Plant_Hydraulics/CLM50_Tech_Note_Plant_Hydraulics:table-plant-functional-type-root-distribution-parameters}}}), aerodynamic parameters that determine resistance to heat,
moisture, and momentum transfer (\hyperref[\detokenize{tech_note/Fluxes/CLM50_Tech_Note_Fluxes:table-plant-functional-type-aerodynamic-parameters}]{Table \ref{\detokenize{tech_note/Fluxes/CLM50_Tech_Note_Fluxes:table-plant-functional-type-aerodynamic-parameters}}}), and photosynthetic
parameters that determine stomatal resistance, photosynthesis, and
transpiration (\hyperref[\detokenize{tech_note/Photosynthesis/CLM50_Tech_Note_Photosynthesis:table-plant-functional-type-pft-stomatal-conductance-parameters}]{Table \ref{\detokenize{tech_note/Photosynthesis/CLM50_Tech_Note_Photosynthesis:table-plant-functional-type-pft-stomatal-conductance-parameters}}},
\hyperref[\detokenize{tech_note/Photosynthesis/CLM50_Tech_Note_Photosynthesis:table-temperature-dependence-parameters-for-c3-photosynthesis}]{Table \ref{\detokenize{tech_note/Photosynthesis/CLM50_Tech_Note_Photosynthesis:table-temperature-dependence-parameters-for-c3-photosynthesis}}}). The composition and abundance of PFTs
within a grid cell can either be prescribed as time-invariant fields
(e.g., using the present day dataset described in section 21.3.3) or can
evolve with time if the model is run in transient landcover mode
(Chapter \hyperref[\detokenize{tech_note/Transient_Landcover/CLM50_Tech_Note_Transient_Landcover:rst-transient-landcover-change}]{\ref{\detokenize{tech_note/Transient_Landcover/CLM50_Tech_Note_Transient_Landcover:rst-transient-landcover-change}}}).


\begin{savenotes}\sphinxattablestart
\centering
\sphinxcapstartof{table}
\sphinxcaption{Plant functional types}\label{\detokenize{tech_note/Ecosystem/CLM50_Tech_Note_Ecosystem:table-plant-functional-types}}\label{\detokenize{tech_note/Ecosystem/CLM50_Tech_Note_Ecosystem:id15}}
\sphinxaftercaption
\begin{tabular}[t]{|*{2}{\X{1}{2}|}}
\hline
\sphinxstylethead{\sphinxstyletheadfamily 
Plant functional type
\unskip}\relax &\sphinxstylethead{\sphinxstyletheadfamily 
Acronym
\unskip}\relax \\
\hline
Needleleaf evergreen tree \textendash{} temperate
&
NET Temperate
\\
\hline
Needleleaf evergreen tree - boreal
&
NET Boreal
\\
\hline
Needleleaf deciduous tree \textendash{} boreal
&
NDT Boreal
\\
\hline
Broadleaf evergreen tree \textendash{} tropical
&
BET Tropical
\\
\hline
Broadleaf evergreen tree \textendash{} temperate
&
BET Temperate
\\
\hline
Broadleaf deciduous tree \textendash{} tropical
&
BDT Tropical
\\
\hline
Broadleaf deciduous tree \textendash{} temperate
&
BDT Temperate
\\
\hline
Broadleaf deciduous tree \textendash{} boreal
&
BDT Boreal
\\
\hline
Broadleaf evergreen shrub - temperate
&
BES Temperate
\\
\hline
Broadleaf deciduous shrub \textendash{} temperate
&
BDS Temperate
\\
\hline
Broadleaf deciduous shrub \textendash{} boreal
&
BDS Boreal
\\
\hline
C$_{\text{3}}$ arctic grass
&\begin{itemize}
\item {} 
\end{itemize}
\\
\hline
C$_{\text{3}}$ grass
&\begin{itemize}
\item {} 
\end{itemize}
\\
\hline
C$_{\text{4}}$ grass
&\begin{itemize}
\item {} 
\end{itemize}
\\
\hline
C$_{\text{3}}$ Unmanaged Rainfed Crop
&
UCrop UIrr
\\
\hline
$^{\text{1}}$C$_{\text{3}}$ Unmanaged Irrigated Crop
&
UCrop Irr
\\
\hline
$^{\text{2}}$Managed Rainfed Unirrigated Crops
&
Crop UIrr
\\
\hline
$^{\text{2}}$Managed Irrigated Crops
&
Crop Irr
\\
\hline
\end{tabular}
\par
\sphinxattableend\end{savenotes}

$^{\text{1}}$Only used if irrigation is active (Chapter \hyperref[\detokenize{tech_note/Crop_Irrigation/CLM50_Tech_Note_Crop_Irrigation:rst-crops-and-irrigation}]{\ref{\detokenize{tech_note/Crop_Irrigation/CLM50_Tech_Note_Crop_Irrigation:rst-crops-and-irrigation}}}).
$^{\text{2}}$Only used if crop model is active (see Chapter \hyperref[\detokenize{tech_note/Crop_Irrigation/CLM50_Tech_Note_Crop_Irrigation:rst-crops-and-irrigation}]{\ref{\detokenize{tech_note/Crop_Irrigation/CLM50_Tech_Note_Crop_Irrigation:rst-crops-and-irrigation}}} for list of represented crops).


\subsubsection{Vegetation Structure}
\label{\detokenize{tech_note/Ecosystem/CLM50_Tech_Note_Ecosystem:id4}}\label{\detokenize{tech_note/Ecosystem/CLM50_Tech_Note_Ecosystem:vegetation-structure}}
Vegetation structure is defined by leaf and stem area indices
(\(L,\, S\)) and canopy top and bottom heights (\(z_{top}\),\(z_{bot}\) ).
Separate leaf and
stem area indices and canopy heights are prescribed or calculated for each PFT. Daily leaf
and stem area indices are obtained from griddeddatasets of monthly values (section
\hyperref[\detokenize{tech_note/Ecosystem/CLM50_Tech_Note_Ecosystem:surface-data}]{\ref{\detokenize{tech_note/Ecosystem/CLM50_Tech_Note_Ecosystem:surface-data}}}). Canopy top and bottom heights for trees are from ICESat ({\hyperref[\detokenize{tech_note/References/CLM50_Tech_Note_References:simardetal2011}]{\sphinxcrossref{\DUrole{std,std-ref}{Simard et al. (2011)}}}}).
Canopy top and bottom heights for short vegetation are obtained from gridded datasets but are invariant in space
and time and were obtained from PFT-specific values ({\hyperref[\detokenize{tech_note/References/CLM50_Tech_Note_References:bonanetal2002a}]{\sphinxcrossref{\DUrole{std,std-ref}{Bonan et al. (2002a)}}}}) (\hyperref[\detokenize{tech_note/Ecosystem/CLM50_Tech_Note_Ecosystem:table-plant-functional-type-canopy-top-and-bottom-heights}]{Table \ref{\detokenize{tech_note/Ecosystem/CLM50_Tech_Note_Ecosystem:table-plant-functional-type-canopy-top-and-bottom-heights}}}).
When the biogeochemistry model is active,
vegetation state (LAI, SAI, canopy top and bottom heights) are calculated prognostically
(see Chapter \hyperref[\detokenize{tech_note/Vegetation_Phenology_Turnover/CLM50_Tech_Note_Vegetation_Phenology_Turnover:rst-vegetation-phenology-and-turnover}]{\ref{\detokenize{tech_note/Vegetation_Phenology_Turnover/CLM50_Tech_Note_Vegetation_Phenology_Turnover:rst-vegetation-phenology-and-turnover}}}).


\begin{savenotes}\sphinxattablestart
\centering
\sphinxcapstartof{table}
\sphinxcaption{Plant functional type canopy top and bottom heights}\label{\detokenize{tech_note/Ecosystem/CLM50_Tech_Note_Ecosystem:table-plant-functional-type-canopy-top-and-bottom-heights}}\label{\detokenize{tech_note/Ecosystem/CLM50_Tech_Note_Ecosystem:id16}}
\sphinxaftercaption
\begin{tabulary}{\linewidth}[t]{|T|T|T|}
\hline
\sphinxstylethead{\sphinxstyletheadfamily 
Plant functional type
\unskip}\relax &
\(z_{top}\)
&
\(z_{bot}\)
\\
\hline
BES Temperate
&
0.5
&
0.1
\\
\hline
BDS Temperate
&
0.5
&
0.1
\\
\hline
BDS Boreal
&
0.5
&
0.1
\\
\hline
C$_{\text{3}}$ arctic grass
&
0.5
&
0.01
\\
\hline
C$_{\text{3}}$ grass
&
0.5
&
0.01
\\
\hline
C$_{\text{4}}$ grass
&
0.5
&
0.01
\\
\hline
UCrop UIrr
&
0.5
&
0.01
\\
\hline
UCrop Irr
&
0.5
&
0.01
\\
\hline
Crop UIrr
&
0.5
&
0.01
\\
\hline
Crop Irr
&
0.5
&
0.01
\\
\hline
\end{tabulary}
\par
\sphinxattableend\end{savenotes}


\subsubsection{Phenology and vegetation burial by snow}
\label{\detokenize{tech_note/Ecosystem/CLM50_Tech_Note_Ecosystem:id5}}\label{\detokenize{tech_note/Ecosystem/CLM50_Tech_Note_Ecosystem:phenology-and-vegetation-burial-by-snow}}
When the biogeochemistry model is inactive, leaf and stem area indices
(m$^{\text{2}}$ leaf area m$^{\text{-2}}$ ground area) are updated
daily by linearly interpolating between monthly values. Monthly PFT leaf
area index values are developed from the 1-km MODIS-derived monthly grid
cell average leaf area index of {\hyperref[\detokenize{tech_note/References/CLM50_Tech_Note_References:mynenietal2002}]{\sphinxcrossref{\DUrole{std,std-ref}{Myneni et al. (2002)}}}},
as described in {\hyperref[\detokenize{tech_note/References/CLM50_Tech_Note_References:lawrencechase2007}]{\sphinxcrossref{\DUrole{std,std-ref}{Lawrence and Chase (2007)}}}}. Stem area
ndex is calculated from the monthly PFT leaf area index using the methods of
{\hyperref[\detokenize{tech_note/References/CLM50_Tech_Note_References:zengetal2002}]{\sphinxcrossref{\DUrole{std,std-ref}{Zeng et al. (2002)}}}}. The leaf and stem area indices are
adjusted for vertical burying by snow ({\hyperref[\detokenize{tech_note/References/CLM50_Tech_Note_References:wangzeng2009}]{\sphinxcrossref{\DUrole{std,std-ref}{Wang and Zeng 2009}}}})
as
\phantomsection\label{\detokenize{tech_note/Ecosystem/CLM50_Tech_Note_Ecosystem:equation-2.1}}\begin{equation}\label{equation:tech_note/Ecosystem/CLM50_Tech_Note_Ecosystem:2.1}
\begin{split}A=A^{*} ( 1-f_{veg}^{sno} )\end{split}
\end{equation}
where \(A^{\*}\) is the leaf or stem area before adjustment for
snow, \(A\) is the remaining exposed leaf or stem area,
\(f_{veg}^{sno}\) is the vertical fraction of vegetation covered by snow
\phantomsection\label{\detokenize{tech_note/Ecosystem/CLM50_Tech_Note_Ecosystem:equation-2.2}}\begin{equation}\label{equation:tech_note/Ecosystem/CLM50_Tech_Note_Ecosystem:2.2}
\begin{split}{f_{veg}^{sno} = \frac{z_{sno} -z_{bot} }{z_{top} -z_{bot} }         \qquad {\rm for\; tree\; and\; shrub}} \\
{f_{veg}^{sno} = \frac{\min \left(z_{sno} ,\, z_{c} \right)}{z_{c} } \qquad {\rm for\; grass\; and\; crop}}\end{split}
\end{equation}
where \(z_{sno} -z_{bot} \ge 0,{\rm \; }0\le f_{veg}^{sno} \le 1\), \(z_{sno}\)  is the depth of snow (m)
(Chapter \hyperref[\detokenize{tech_note/Snow_Hydrology/CLM50_Tech_Note_Snow_Hydrology:rst-snow-hydrology}]{\ref{\detokenize{tech_note/Snow_Hydrology/CLM50_Tech_Note_Snow_Hydrology:rst-snow-hydrology}}}), and \(z_{c} = 0.2\) is the snow depth when short vegetation is assumed to
be completely buried by snow (m). For numerical reasons, exposed leaf and stem area are set to zero if less than
0.05. If the sum of exposed leaf and stem area is zero, then the surface is treated as snow-covered ground.


\subsection{Vertical Discretization}
\label{\detokenize{tech_note/Ecosystem/CLM50_Tech_Note_Ecosystem:id6}}\label{\detokenize{tech_note/Ecosystem/CLM50_Tech_Note_Ecosystem:vertical-discretization}}

\subsubsection{Soil Layers}
\label{\detokenize{tech_note/Ecosystem/CLM50_Tech_Note_Ecosystem:id7}}\label{\detokenize{tech_note/Ecosystem/CLM50_Tech_Note_Ecosystem:soil-layers}}
The soil column can be discretized into an arbitrary number of layers.  The default
vertical discretization (\hyperref[\detokenize{tech_note/Ecosystem/CLM50_Tech_Note_Ecosystem:table-soil-layer-structure}]{Table \ref{\detokenize{tech_note/Ecosystem/CLM50_Tech_Note_Ecosystem:table-soil-layer-structure}}}) uses
\(N_{levgrnd} = 25\) layers, of which \(N_{levsoi} = 20\) are hydrologically and
biogeochemically active.  The deepest 5 layers are only included in the thermodynamical
calculations ({\hyperref[\detokenize{tech_note/References/CLM50_Tech_Note_References:lawrenceetal2008}]{\sphinxcrossref{\DUrole{std,std-ref}{Lawrence et al. 2008}}}}) described in Chapter
\hyperref[\detokenize{tech_note/Soil_Snow_Temperatures/CLM50_Tech_Note_Soil_Snow_Temperatures:rst-soil-and-snow-temperatures}]{\ref{\detokenize{tech_note/Soil_Snow_Temperatures/CLM50_Tech_Note_Soil_Snow_Temperatures:rst-soil-and-snow-temperatures}}}.

The layer structure of the soil is described by the node depth, \(z_{i}\)
(m), the thickness of each layer, \(\Delta z_{i}\)  (m), and the depths
at the layer interfaces \(z_{h,\, i}\)  (m).


\begin{savenotes}\sphinxattablestart
\centering
\sphinxcapstartof{table}
\sphinxcaption{Soil layer structure}\label{\detokenize{tech_note/Ecosystem/CLM50_Tech_Note_Ecosystem:table-soil-layer-structure}}\label{\detokenize{tech_note/Ecosystem/CLM50_Tech_Note_Ecosystem:id17}}
\sphinxaftercaption
\begin{tabulary}{\linewidth}[t]{|T|T|T|T|}
\hline
\sphinxstylethead{\sphinxstyletheadfamily 
Layer
\unskip}\relax &
\(z_{i}\)
&
\(\Delta z_{i}\)
&
\(z_{h,\, i}\)
\\
\hline
1
&
0.010
&
0.020
&
0.020
\\
\hline
2
&
0.040
&
0.040
&
0.060
\\
\hline
3
&
0.090
&
0.060
&
0.120
\\
\hline
4
&
0.160
&
0.080
&
0.200
\\
\hline
5
&
0.260
&
0.120
&
0.320
\\
\hline
6
&
0.400
&
0.160
&
0.480
\\
\hline
7
&
0.580
&
0.200
&
0.680
\\
\hline
8
&
0.800
&
0.240
&
0.920
\\
\hline
9
&
1.060
&
0.280
&
1.200
\\
\hline
10
&
1.360
&
0.320
&
1.520
\\
\hline
11
&
1.700
&
0.360
&
1.880
\\
\hline
12
&
2.080
&
0.400
&
2.280
\\
\hline
13
&
2.500
&
0.440
&
2.720
\\
\hline
14
&
2.990
&
0.540
&
3.260
\\
\hline
15
&
3.580
&
0.640
&
3.900
\\
\hline
16
&
4.270
&
0.740
&
4.640
\\
\hline
17
&
5.060
&
0.840
&
5.480
\\
\hline
18
&
5.950
&
0.940
&
6.420
\\
\hline
19
&
6.940
&
1.040
&
7.460
\\
\hline
20
&
8.030
&
1.140
&
8.600
\\
\hline
21
&
9.795
&
2.390
&
10.990
\\
\hline
22
&
13.328
&
4.676
&
15.666
\\
\hline
23
&
19.483
&
7.635
&
23.301
\\
\hline
24
&
28.871
&
11.140
&
34.441
\\
\hline
25
&
41.998
&
15.115
&
49.556
\\
\hline
\end{tabulary}
\par
\sphinxattableend\end{savenotes}

Layer node depth (\(z_{i}\) ), thickness (\(\Delta z_{i}\) ), and depth at
layer interface (\(z_{h,\, i}\) ) for default soil column. All in meters.


\subsubsection{Depth to Bedrock}
\label{\detokenize{tech_note/Ecosystem/CLM50_Tech_Note_Ecosystem:id8}}\label{\detokenize{tech_note/Ecosystem/CLM50_Tech_Note_Ecosystem:depth-to-bedrock}}
The hydrologically and biogeochemically active portion of the soil column can be
restricted to a thickness less than that of the maximum soil depth.  By providing
a depth-to-bedrock dataset, which may vary spatially, the number of layers used
in the hydrologic and biogeochemical calculations, \(N_{bedrock}\), may be
specified, subject to the constraint \(N_{bedrock} \le N_{levsoi}\)


\subsection{Model Input Requirements}
\label{\detokenize{tech_note/Ecosystem/CLM50_Tech_Note_Ecosystem:model-input-requirements}}\label{\detokenize{tech_note/Ecosystem/CLM50_Tech_Note_Ecosystem:id9}}

\subsubsection{Atmospheric Coupling}
\label{\detokenize{tech_note/Ecosystem/CLM50_Tech_Note_Ecosystem:id10}}\label{\detokenize{tech_note/Ecosystem/CLM50_Tech_Note_Ecosystem:atmospheric-coupling}}
The current state of the atmosphere (\hyperref[\detokenize{tech_note/Ecosystem/CLM50_Tech_Note_Ecosystem:table-atmospheric-input-to-land-model}]{Table \ref{\detokenize{tech_note/Ecosystem/CLM50_Tech_Note_Ecosystem:table-atmospheric-input-to-land-model}}})
at a given time step is
used to force the land model. This atmospheric state is provided by an
atmospheric model in coupled mode or from an observed dataset in land-only
mode (Chapter \hyperref[\detokenize{tech_note/Land-Only_Mode/CLM50_Tech_Note_Land-Only_Mode:rst-land-only-mode}]{\ref{\detokenize{tech_note/Land-Only_Mode/CLM50_Tech_Note_Land-Only_Mode:rst-land-only-mode}}}). The land model then initiates a full set of
calculations for surface energy, constituent, momentum, and radiative
fluxes. The land model calculations are implemented in two steps. The
land model proceeds with the calculation of surface energy, constituent,
momentum, and radiative fluxes using the snow and soil hydrologic states
from the previous time step. The land model then updates the soil and
snow hydrology calculations based on these fluxes. These fields are
passed to the atmosphere (\hyperref[\detokenize{tech_note/Ecosystem/CLM50_Tech_Note_Ecosystem:table-land-model-output-to-atmospheric-model}]{Table \ref{\detokenize{tech_note/Ecosystem/CLM50_Tech_Note_Ecosystem:table-land-model-output-to-atmospheric-model}}}). The albedos sent to the atmosphere
are for the solar zenith angle at the next time step but with surface
conditions from the current time step.


\begin{savenotes}\sphinxattablestart
\centering
\sphinxcapstartof{table}
\sphinxcaption{Atmospheric input to land model}\label{\detokenize{tech_note/Ecosystem/CLM50_Tech_Note_Ecosystem:table-atmospheric-input-to-land-model}}\label{\detokenize{tech_note/Ecosystem/CLM50_Tech_Note_Ecosystem:id18}}
\sphinxaftercaption
\begin{tabulary}{\linewidth}[t]{|T|T|T|}
\hline
\sphinxstylethead{\sphinxstyletheadfamily 
Field
\unskip}\relax &\sphinxstylethead{\sphinxstyletheadfamily 
variable name
\unskip}\relax &\sphinxstylethead{\sphinxstyletheadfamily 
units
\unskip}\relax \\
\hline
$^{\text{1}}$Reference height
&
\(z'_{atm}\)
&
m
\\
\hline
Atmosphere model’s surface height
&
\(z_{surf,atm}\)
&
m
\\
\hline
Zonal wind at \(z_{atm}\)
&
\(u_{atm}\)
&
m s$^{\text{-1}}$
\\
\hline
Meridional wind at \(z_{atm}\)
&
\(v_{atm}\)
&
m s$^{\text{-1}}$
\\
\hline
Potential temperature
&
\(\overline{\theta _{atm} }\)
&
K
\\
\hline
Specific humidity at \(z_{atm}\)
&
\(q_{atm}\)
&
kg kg$^{\text{-1}}$
\\
\hline
Pressure at \(z_{atm}\)
&
\(P_{atm}\)
&
Pa
\\
\hline
Temperature at \(z_{atm}\)
&
\(T_{atm}\)
&
K
\\
\hline
Incident longwave radiation
&
\(L_{atm} \, \downarrow\)
&
W m$^{\text{-2}}$
\\
\hline
$^{\text{2}}$Liquid precipitation
&
\(q_{rain}\)
&
mm s$^{\text{-1}}$
\\
\hline
$^{\text{2}}$Solid precipitation
&
\(q_{sno}\)
&
mm s$^{\text{-1}}$
\\
\hline
Incident direct beam visible solar radiation
&
\(S_{atm} \, \downarrow _{vis}^{\mu }\)
&
W m$^{\text{-2}}$
\\
\hline
Incident direct beam near-infrared solar radiation
&
\(S_{atm} \, \downarrow _{nir}^{\mu }\)
&
W m$^{\text{-2}}$
\\
\hline
Incident diffuse visible solar radiation
&
\(S_{atm} \, \downarrow _{vis}\)
&
W m$^{\text{-2}}$
\\
\hline
Incident diffuse near-infrared solar radiation
&
\(S_{atm} \, \downarrow _{nir}\)
&
W m$^{\text{-2}}$
\\
\hline
Carbon dioxide (CO$_{\text{2}}$) concentration
&
\(c_{a}\)
&
ppmv
\\
\hline
$^{\text{3}}$Aerosol deposition rate
&
\(D_{sp}\)
&
kg m$^{\text{-2}}$ s$^{\text{-1}}$
\\
\hline
$^{\text{4}}$Nitrogen deposition rate
&
\(NF_{ndep\_ s{\it min}n}\)
&
g (N) m$^{\text{-2}}$ yr$^{\text{-1}}$
\\
\hline
$^{\text{5}}$Lightning frequency
&
\(I_{l}\)
&
flash km$^{\text{-2}}$ hr$^{\text{-1}}$
\\
\hline
\end{tabulary}
\par
\sphinxattableend\end{savenotes}

$^{\text{1}}$The atmospheric reference height received from the
atmospheric model \(z'_{atm}\)  is assumed to be the height above
the surface as defined by the roughness length \(z_{0}\)  plus
displacement height \(d\). Thus, the reference height used for flux
computations (Chapter \hyperref[\detokenize{tech_note/Fluxes/CLM50_Tech_Note_Fluxes:rst-momentum-sensible-heat-and-latent-heat-fluxes}]{\ref{\detokenize{tech_note/Fluxes/CLM50_Tech_Note_Fluxes:rst-momentum-sensible-heat-and-latent-heat-fluxes}}})
is \(z_{atm} =z'_{atm} +z_{0} +d\). The
reference heights for temperature, wind, and specific humidity
(\(z_{atm,\, h}\) , \(z_{atm,\, {\it m}}\) ,
\(z_{atm,\, w}\) ) are required. These are set equal
to\(z_{atm}\) .

$^{\text{2}}$CAM provides convective and large-scale liquid
and solid precipitation, which are added to yield total liquid
precipitation \(q_{rain}\)  and solid precipitation
\(q_{sno}\) .

$^{\text{3}}$There are 14 aerosol deposition rates required depending
on species and affinity for bonding with water; 8 of these are dust
deposition rates (dry and wet rates for 4 dust size bins,
\(D_{dst,\, dry1} ,\, D_{dst,\, dry2} ,\, D_{dst,\, dry3} ,\, D_{dst,\, dry4}\) ,
\(D_{dst,\, \, wet1} ,D_{dst,\, wet2} ,\, D_{dst,wet3} ,\, D_{dst,\, wet4}\) ),
3 are black carbon deposition rates (dry and wet hydrophilic and dry
hydrophobic rates,
\(D_{bc,\, dryhphil} ,\, D_{bc,\, wethphil} ,\, D_{bc,\, dryhphob}\) ),
and 3 are organic carbon deposition rates (dry and wet hydrophilic and
dry hydrophobic rates,
\(D_{oc,\, dryhphil} ,\, D_{oc,\, wethphil} ,\, D_{oc,\, dryhphob}\) ).
These fluxes are computed interactively by the atmospheric model (when
prognostic aerosol representation is active) or are prescribed from a
time-varying (annual cycle or transient), globally-gridded deposition
file defined in the namelist (see the CLM4.5 User’s Guide). Aerosol
deposition rates were calculated in a transient 1850-2009 CAM simulation
(at a resolution of 1.9x2.5x26L) with interactive chemistry (troposphere
and stratosphere) driven by CCSM3 20$^{\text{th}}$ century
sea-surface temperatures and emissions (Lamarque et al. 2010) for
short-lived gases and aerosols; observed concentrations were specified
for methane, N$_{\text{2}}$O, the ozone-depleting substances (CFCs)
,and CO$_{\text{2}}$. The fluxes are used by the snow-related
parameterizations (Chapters \hyperref[\detokenize{tech_note/Surface_Albedos/CLM50_Tech_Note_Surface_Albedos:rst-surface-albedos}]{\ref{\detokenize{tech_note/Surface_Albedos/CLM50_Tech_Note_Surface_Albedos:rst-surface-albedos}}} and numref:\sphinxtitleref{rst\_Snow Hydrology}).

$^{\text{4}}$The nitrogen deposition rate is required by the
biogeochemistry model when active and represents the total deposition of
mineral nitrogen onto the land surface, combining deposition of
NO$_{\text{y}}$ and NH$_{\text{x}}$. The rate is supplied either
as a time-invariant spatially-varying annual mean rate or time-varying
for a transient simulation. Nitrogen deposition rates were calculated
from the same CAM chemistry simulation that generated the aerosol
deposition rates.

$^{\text{5}}$Climatological 3-hourly lightning frequency at
\(\sim\)1.8$^{\text{o}}$ resolution is provided, which was
calculated via bilinear interpolation from 1995-2011 NASA LIS/OTD grid
product v2.2 (\sphinxurl{http://ghrc.msfc.nasa.gov}) 2-hourly, 2.5$^{\text{o}}$
lightning frequency data. In future versions of the model, lightning
data may be obtained directly from the atmosphere model.

Density of air (\(\rho _{atm}\) ) (kg m$^{\text{-3}}$) is also
required but is calculated directly from
\(\rho _{atm} =\frac{P_{atm} -0.378e_{atm} }{R_{da} T_{atm} }\)
where \(P_{atm}\)  is atmospheric pressure (Pa), \(e_{atm}\)  is
atmospheric vapor pressure (Pa), \(R_{da}\)  is the gas constant for
dry air (J kg$^{\text{-1}}$ K$^{\text{-1}}$) (\hyperref[\detokenize{tech_note/Ecosystem/CLM50_Tech_Note_Ecosystem:table-physical-constants}]{Table \ref{\detokenize{tech_note/Ecosystem/CLM50_Tech_Note_Ecosystem:table-physical-constants}}}), and
\(T_{atm}\)  is the atmospheric temperature (K). The atmospheric
vapor pressure \(e_{atm}\)  is derived from atmospheric specific
humidity \(q_{atm}\)  (kg kg$^{\text{-1}}$) as
\(e_{atm} =\frac{q_{atm} P_{atm} }{0.622+0.378q_{atm} }\) .

The O$_{\text{2}}$ partial pressure (Pa) is required but is
calculated from molar ratio and the atmospheric pressure
\(P_{atm}\)  as \(o_{i} =0.209P_{atm}\) .


\begin{savenotes}\sphinxattablestart
\centering
\sphinxcapstartof{table}
\sphinxcaption{Land model output to atmospheric model}\label{\detokenize{tech_note/Ecosystem/CLM50_Tech_Note_Ecosystem:table-land-model-output-to-atmospheric-model}}\label{\detokenize{tech_note/Ecosystem/CLM50_Tech_Note_Ecosystem:id19}}
\sphinxaftercaption
\begin{tabular}[t]{|*{3}{\X{1}{3}|}}
\hline
\sphinxstylethead{\sphinxstyletheadfamily 
Field
\unskip}\relax &\sphinxstylethead{\sphinxstyletheadfamily 
Variable name
\unskip}\relax &\sphinxstylethead{\sphinxstyletheadfamily 
units
\unskip}\relax \\
\hline
$^{\text{1}}$Latent heat flux
&
\(\lambda _{vap} E_{v} +\lambda E_{g}\)
&
W m$^{\text{-2}}$
\\
\hline
Sensible heat flux
&
\(H_{v} +H_{g}\)
&
W m$^{\text{-2}}$
\\
\hline
Water vapor flux
&
\(E_{v} +E_{g}\)
&
mm s$^{\text{-1}}$
\\
\hline
Zonal momentum flux
&
\(\tau _{x}\)
&
kg m$^{\text{-1}}$ s$^{\text{-2}}$
\\
\hline
Meridional momentum flux
&
\(\tau _{y}\)
&
kg m$^{\text{-1}}$ s$^{\text{-2}}$
\\
\hline
Emitted longwave radiation
&
\(L\, \uparrow\)
&
W m$^{\text{-2}}$
\\
\hline
Direct beam visible albedo
&
\(I\, \uparrow _{vis}^{\mu }\)
&\begin{itemize}
\item {} 
\end{itemize}
\\
\hline
Direct beam near-infrared albedo
&
\(I\, \uparrow _{nir}^{\mu }\)
&\begin{itemize}
\item {} 
\end{itemize}
\\
\hline
Diffuse visible albedo
&
\(I\, \uparrow _{vis}\)
&\begin{itemize}
\item {} 
\end{itemize}
\\
\hline
Diffuse near-infrared albedo
&
\(I\, \uparrow _{nir}\)
&\begin{itemize}
\item {} 
\end{itemize}
\\
\hline
Absorbed solar radiation
&
\(\vec{S}\)
&
W m$^{\text{-2}}$
\\
\hline
Radiative temperature
&
\(T_{rad}\)
&
K
\\
\hline
Temperature at 2 meter height
&
\(T_{2m}\)
&
K
\\
\hline
Specific humidity at 2 meter height
&
\(q_{2m}\)
&
kg kg$^{\text{-1}}$
\\
\hline
Wind speed at 10 meter height
&
\(u_{10m}\)
&
m s$^{\text{-1}}$
\\
\hline
Snow water equivalent
&
\(W_{sno}\)
&
m
\\
\hline
Aerodynamic resistance
&
\(r_{am}\)
&
s m$^{\text{-1}}$
\\
\hline
Friction velocity
&
\(u_{*}\)
&
m s$^{\text{-1}}$
\\
\hline
$^{\text{2}}$Dust flux
&
\(F_{j}\)
&
kg m$^{\text{-2}}$ s$^{\text{-1}}$
\\
\hline
Net ecosystem exchange
&
NEE
&
kgCO$_{\text{2}}$ m$^{\text{-2}}$ s$^{\text{-1}}$
\\
\hline
\end{tabular}
\par
\sphinxattableend\end{savenotes}

$^{\text{1}}$\(\lambda _{vap}\)  is the latent heat of
vaporization (J kg$^{\text{-1}}$) (\hyperref[\detokenize{tech_note/Ecosystem/CLM50_Tech_Note_Ecosystem:table-physical-constants}]{Table \ref{\detokenize{tech_note/Ecosystem/CLM50_Tech_Note_Ecosystem:table-physical-constants}}}) and \(\lambda\)  is
either the latent heat of vaporization \(\lambda _{vap}\)  or latent
heat of sublimation \(\lambda _{sub}\)  (J kg$^{\text{-1}}$)
(\hyperref[\detokenize{tech_note/Ecosystem/CLM50_Tech_Note_Ecosystem:table-physical-constants}]{Table \ref{\detokenize{tech_note/Ecosystem/CLM50_Tech_Note_Ecosystem:table-physical-constants}}}) depending on the liquid water and ice content of the top
snow/soil layer (section 5.4).

$^{\text{2}}$There are \(j=1,\ldots ,4\) dust transport bins.


\subsubsection{Initialization}
\label{\detokenize{tech_note/Ecosystem/CLM50_Tech_Note_Ecosystem:id11}}\label{\detokenize{tech_note/Ecosystem/CLM50_Tech_Note_Ecosystem:initialization}}
Initialization of the land model (i.e., providing the model with initial
temperature and moisture states) depends on the type of run (startup or
restart) (see the CLM4.5 User’s Guide). A startup run starts the model
from either initial conditions that are set internally in the Fortran
code (referred to as arbitrary initial conditions) or from an initial
conditions dataset that enables the model to start from a spun up state
(i.e., where the land is in equilibrium with the simulated climate). In
restart runs, the model is continued from a previous simulation and
initialized from a restart file that ensures that the output is
bit-for-bit the same as if the previous simulation had not stopped. The
fields that are required from the restart or initial conditions files
can be obtained by examining the code. Arbitrary initial conditions are
specified as follows.

Soil points are initialized with
surface ground temperature \(T_{g}\)  and soil layer temperature
\(T_{i}\) , for \(i=1,\ldots ,N_{levgrnd}\) , of 274 K,
vegetation temperature \(T_{v}\)  of 283 K, no snow or canopy water
(\(W_{sno} =0\), \(W_{can} =0\)), and volumetric soil water
content \(\theta _{i} =0.15\) mm$^{\text{3}}$ mm$^{\text{-3}}$
for layers \(i=1,\ldots ,N_{levsoi}\)  and \(\theta _{i} =0.0\)
mm$^{\text{3}}$ mm$^{\text{-3}}$ for layers
\(i=N_{levsoi} +1,\ldots ,N_{levgrnd}\) . placeLake temperatures
(\(T_{g}\)  and \(T_{i}\) ) are initialized at 277 K and
\(W_{sno} =0\).

Glacier temperatures (\(T_{g} =T_{snl+1}\)  and \(T_{i}\)  for
\(i=snl+1,\ldots ,N_{levgrnd}\)  where \(snl\) is the negative
of the number of snow layers, i.e., \(snl\) ranges from \textendash{}5 to 0) are
initialized to 250 K with a snow water equivalent \(W_{sno} =1000\)
mm, snow depth \(z_{sno} =\frac{W_{sno} }{\rho _{sno} }\)  (m) where
\(\rho _{sno} =250\) kg m$^{\text{-3}}$ is an initial estimate
for the bulk density of snow, and \(\theta _{i}\) =1.0 for
\(i=1,\ldots ,N_{levgrnd}\) . The snow layer structure (e.g., number
of snow layers \(snl\) and layer thickness) is initialized based on
the snow depth (section 6.1). The snow liquid water and ice contents (kg
m$^{\text{-2}}$) are initialized as \(w_{liq,\, i} =0\) and
\(w_{ice,\, i} =\Delta z_{i} \rho _{sno}\) , respectively, where
\(i=snl+1,\ldots ,0\) are the snow layers, and \(\Delta z_{i}\)
is the thickness of snow layer \(i\) (m). The soil liquid water and
ice contents are initialized as \(w_{liq,\, i} =0\) and
\(w_{ice,\, i} =\Delta z_{i} \rho _{ice} \theta _{i}\)  for
\(T_{i} \le T_{f}\) , and
\(w_{liq,\, i} =\Delta z_{i} \rho _{liq} \theta _{i}\)  and
\(w_{ice,\, i} =0\) for \(T_{i} >T_{f}\) , where
\(\rho _{ice}\)  and \(\rho _{liq}\)  are the densities of ice
and liquid water (kg m$^{\text{-3}}$) (\hyperref[\detokenize{tech_note/Ecosystem/CLM50_Tech_Note_Ecosystem:table-physical-constants}]{Table \ref{\detokenize{tech_note/Ecosystem/CLM50_Tech_Note_Ecosystem:table-physical-constants}}}), and \(T_{f}\)
is the freezing temperature of water (K) (\hyperref[\detokenize{tech_note/Ecosystem/CLM50_Tech_Note_Ecosystem:table-physical-constants}]{Table \ref{\detokenize{tech_note/Ecosystem/CLM50_Tech_Note_Ecosystem:table-physical-constants}}}). All vegetated and
glacier land units are initialized with water stored in the unconfined
aquifer and unsaturated soil \(W_{a} =4000\) mm and water table
depth \(z_{\nabla }\)  at five meters below the soil column.


\subsubsection{Surface Data}
\label{\detokenize{tech_note/Ecosystem/CLM50_Tech_Note_Ecosystem:surface-data}}\label{\detokenize{tech_note/Ecosystem/CLM50_Tech_Note_Ecosystem:id12}}
Required surface data for each land grid cell are listed in
\hyperref[\detokenize{tech_note/Ecosystem/CLM50_Tech_Note_Ecosystem:table-surface-data-required-for-clm-and-their-base-spatial-resolution}]{Table \ref{\detokenize{tech_note/Ecosystem/CLM50_Tech_Note_Ecosystem:table-surface-data-required-for-clm-and-their-base-spatial-resolution}}}
and include the glacier, lake, and urban fractions of the grid cell
(vegetated and crop occupy the remainder), the fractional cover of each
plant functional type (PFT), monthly leaf and stem area index and canopy
top and bottom heights for each PFT, soil color, soil texture, soil
organic matter density, maximum fractional saturated area, slope,
elevation, biogenic volatile organic compounds (BVOCs) emissions
factors, population density, gross domestic production, peat area
fraction, and peak month of agricultural burning. Optional surface data
include crop irrigation and managed crops. All fields are aggregated to
the model’s grid from high-resolution input datasets (
\hyperref[\detokenize{tech_note/Ecosystem/CLM50_Tech_Note_Ecosystem:table-surface-data-required-for-clm-and-their-base-spatial-resolution}]{Table \ref{\detokenize{tech_note/Ecosystem/CLM50_Tech_Note_Ecosystem:table-surface-data-required-for-clm-and-their-base-spatial-resolution}}}) that
are obtained from a variety of sources described below.


\begin{savenotes}\sphinxattablestart
\centering
\sphinxcapstartof{table}
\sphinxcaption{Surface data required for CLM and their base spatial resolution}\label{\detokenize{tech_note/Ecosystem/CLM50_Tech_Note_Ecosystem:table-surface-data-required-for-clm-and-their-base-spatial-resolution}}\label{\detokenize{tech_note/Ecosystem/CLM50_Tech_Note_Ecosystem:id20}}
\sphinxaftercaption
\begin{tabulary}{\linewidth}[t]{|T|T|}
\hline
\sphinxstylethead{\sphinxstyletheadfamily 
Surface Field
\unskip}\relax &\sphinxstylethead{\sphinxstyletheadfamily 
Resolution
\unskip}\relax \\
\hline
Percent glacier
&
0.05$^{\text{o}}$
\\
\hline
Percent lake and lake depth
&
0.05$^{\text{o}}$
\\
\hline
Percent urban
&
0.05$^{\text{o}}$
\\
\hline
Percent plant functional types (PFTs)
&
0.05$^{\text{o}}$
\\
\hline
Monthly leaf and stem area index
&
0.5$^{\text{o}}$
\\
\hline
Canopy height (top, bottom)
&
0.5$^{\text{o}}$
\\
\hline
Soil color
&
0.5$^{\text{o}}$
\\
\hline
Percent sand, percent clay
&
0.083$^{\text{o}}$
\\
\hline
Soil organic matter density
&
0.083$^{\text{o}}$
\\
\hline
Maximum fractional saturated area
&
0.125$^{\text{o}}$
\\
\hline
Elevation
&
1km
\\
\hline
Slope
&
1km
\\
\hline
Biogenic Volatile Organic Compounds
&
0.5$^{\text{o}}$
\\
\hline
Crop Irrigation
&
0.083$^{\text{o}}$
\\
\hline
Managed crops
&
0.5$^{\text{o}}$
\\
\hline
Population density
&
0.5$^{\text{o}}$
\\
\hline
Gross domestic production
&
0.5$^{\text{o}}$
\\
\hline
Peat area fraction
&
0.5$^{\text{o}}$
\\
\hline
Peak month of agricultural waste burning
&
0.5$^{\text{o}}$
\\
\hline
\end{tabulary}
\par
\sphinxattableend\end{savenotes}

At the base spatial resolution of 0.05$^{\text{o}}$, the percentage of
each PFT is defined with respect to the vegetated portion of the grid
cell and the sum of the PFTs is 100\%. The percent lake, wetland,
glacier, and urban at their base resolution are specified with respect
to the entire grid cell. The surface dataset creation routines re-adjust
the PFT percentages to ensure that the sum of all land cover types in
the grid cell sum to 100\%. A minimum threshold of 0.1\% of the grid cell
by area is required for urban areas.

The percentage glacier mask was derived from vector data of global
glacier and ice sheet spatial coverage. Vector data for glaciers (ice
caps, icefields and mountain glaciers) were taken from the first
globally complete glacier inventory, the Randolph Glacier Inventory
version 1.0 (RGIv1.0: {\hyperref[\detokenize{tech_note/References/CLM50_Tech_Note_References:arendtetal2012}]{\sphinxcrossref{\DUrole{std,std-ref}{Arendt et al. 2012}}}}).
Vector data for the Greenland Ice Sheet were provided by Frank Paul and
Tobias Bolch (University of Zurich: {\hyperref[\detokenize{tech_note/References/CLM50_Tech_Note_References:rastneretal2012}]{\sphinxcrossref{\DUrole{std,std-ref}{Rastner et al. 2012}}}}).
Antarctic Ice Sheet data were provided by
Andrew Bliss (University of Alaska) and were extracted from the
Scientific Committee on Antarctic Research (SCAR) Antarctic Digital
Database version 5.0. Floating ice is only provided for the Antarctic
and does not include the small area of Arctic ice shelves. High spatial
resolution vector data were then processed to determine the area of
glacier, ice sheet and floating ice within 30-second grid cells
globally. The 30-second glacier, ice sheet and Antarctic ice shelf masks
were subsequently draped over equivalent-resolution GLOBE topography
(Global Land One-km Base Elevation Project, Hastings et al. 1999) to
extract approximate ice-covered elevations of ice-covered regions. Grid
cells flagged as land-ice in the mask but ocean in GLOBE (typically,
around ice sheets at high latitudes) were designated land-ice with an
elevation of 0 meters. Finally, the high-resolution mask/topography
datasets were aggregated and processed into three 3-minute datasets:
3-minute fractional areal land ice coverage (including both glaciers and
ice sheets); 3-minute distributions of areal glacier fractional coverage
by elevation and areal ice sheet fractional coverage by elevation. Ice
fractions were binned at 100 meter intervals, with bin edges defined
from 0 to 6000 meters (plus one top bin encompassing all remaining
high-elevation ice, primarily in the Himalaya). These distributions by
elevation are needed when running CLM4 with multiple glacier elevation
classes.

Percent lake and lake depth are area-averaged from the 90-second
resolution data of {\hyperref[\detokenize{tech_note/References/CLM50_Tech_Note_References:kourzeneva2009}]{\sphinxcrossref{\DUrole{std,std-ref}{Kourzeneva (2009, 2010)}}}} to the 0.05$^{\text{o}}$
resolution using the MODIS land-mask. Percent urban is derived from
LandScan 2004, a population density dataset derived from census data,
nighttime lights satellite observations, road proximity and slope
({\hyperref[\detokenize{tech_note/References/CLM50_Tech_Note_References:dobsonetal2000}]{\sphinxcrossref{\DUrole{std,std-ref}{Dobson et al. 2000}}}}) as described by
{\hyperref[\detokenize{tech_note/References/CLM50_Tech_Note_References:jacksonetal2010}]{\sphinxcrossref{\DUrole{std,std-ref}{Jackson et al. (2010)}}}} at 1km
resolution and aggregated to 0.05$^{\text{o}}$. A number of urban
radiative, thermal, and morphological fields are also required and are
obtained from {\hyperref[\detokenize{tech_note/References/CLM50_Tech_Note_References:jacksonetal2010}]{\sphinxcrossref{\DUrole{std,std-ref}{Jackson et al. (2010)}}}}. Their description can be found in
Table 3 of the Community Land Model Urban (CLMU) technical note ({\hyperref[\detokenize{tech_note/References/CLM50_Tech_Note_References:olesonetal2010b}]{\sphinxcrossref{\DUrole{std,std-ref}{Oleson
et al. 2010b}}}}).

Percent PFTs are derived from MODIS satellite data as described in
{\hyperref[\detokenize{tech_note/References/CLM50_Tech_Note_References:lawrencechase2007}]{\sphinxcrossref{\DUrole{std,std-ref}{Lawrence and Chase (2007)}}}} (section 21.3.3).
Prescribed PFT leaf area index is derived from the MODIS satellite data of
{\hyperref[\detokenize{tech_note/References/CLM50_Tech_Note_References:mynenietal2002}]{\sphinxcrossref{\DUrole{std,std-ref}{Myneni et al. (2002)}}}} using the de-aggregation methods
described in {\hyperref[\detokenize{tech_note/References/CLM50_Tech_Note_References:lawrencechase2007}]{\sphinxcrossref{\DUrole{std,std-ref}{Lawrence and Chase (2007)}}}}
(section 2.2.3). Prescribed PFT stem area index is derived from PFT leaf
area index phenology combined with the methods of {\hyperref[\detokenize{tech_note/References/CLM50_Tech_Note_References:zengetal2002}]{\sphinxcrossref{\DUrole{std,std-ref}{Zeng et al. (2002)}}}}.
Prescribed canopy top and bottom heights are from {\hyperref[\detokenize{tech_note/References/CLM50_Tech_Note_References:bonan1996}]{\sphinxcrossref{\DUrole{std,std-ref}{Bonan (1996)}}}} as
described in {\hyperref[\detokenize{tech_note/References/CLM50_Tech_Note_References:bonanetal2002b}]{\sphinxcrossref{\DUrole{std,std-ref}{Bonan et al. (2002b)}}}}. If the biogeochemistry model is
active, it supplies the leaf and stem area index and canopy top and
bottom heights dynamically, and the prescribed values are ignored.

Soil color determines dry and saturated soil albedo (section \hyperref[\detokenize{tech_note/Surface_Albedos/CLM50_Tech_Note_Surface_Albedos:ground-albedos}]{\ref{\detokenize{tech_note/Surface_Albedos/CLM50_Tech_Note_Surface_Albedos:ground-albedos}}}).
Soil colors are from {\hyperref[\detokenize{tech_note/References/CLM50_Tech_Note_References:lawrencechase2007}]{\sphinxcrossref{\DUrole{std,std-ref}{Lawrence and Chase (2007)}}}}.

The soil texture and organic matter content determine soil thermal and
hydrologic properties (sections 6.3 and 7.4.1). The International
Geosphere-Biosphere Programme (IGBP) soil dataset (Global Soil Data Task
2000) of 4931 soil mapping units and their sand and clay content for
each soil layer were used to create a mineral soil texture dataset
{\hyperref[\detokenize{tech_note/References/CLM50_Tech_Note_References:bonanetal2002b}]{\sphinxcrossref{\DUrole{std,std-ref}{(Bonan et al. 2002b)}}}}. Soil organic matter data is merged from two
sources. The majority of the globe is from ISRIC-WISE ({\hyperref[\detokenize{tech_note/References/CLM50_Tech_Note_References:batjes2006}]{\sphinxcrossref{\DUrole{std,std-ref}{Batjes, 2006}}}}).
The high latitudes come from the 0.25$^{\text{o}}$ version of the
Northern Circumpolar Soil Carbon Database ({\hyperref[\detokenize{tech_note/References/CLM50_Tech_Note_References:hugeliusetal2012}]{\sphinxcrossref{\DUrole{std,std-ref}{Hugelius et al. 2012}}}}). Both
datasets report carbon down to 1m depth. Carbon is partitioned across
the top seven CLM4 layers (\(\sim\)1m depth) as in
{\hyperref[\detokenize{tech_note/References/CLM50_Tech_Note_References:lawrenceslater2008}]{\sphinxcrossref{\DUrole{std,std-ref}{Lawrence and Slater (2008)}}}}.

The maximum fractional saturated area (\(f_{\max }\) ) is used in
determining surface runoff and infiltration (section 7.3). Maximum
fractional saturated area at 0.125$^{\text{o}}$ resolution is
calculated from 1-km compound topographic indices (CTIs) based on the
USGS HYDRO1K dataset ({\hyperref[\detokenize{tech_note/References/CLM50_Tech_Note_References:verdingreenlee1996}]{\sphinxcrossref{\DUrole{std,std-ref}{Verdin and Greenlee 1996}}}})
following the algorithm in {\hyperref[\detokenize{tech_note/References/CLM50_Tech_Note_References:niuetal2005}]{\sphinxcrossref{\DUrole{std,std-ref}{Niu et al. (2005)}}}}.
\(f_{\max }\)  is the ratio between the number
of 1-km pixels with CTIs equal to or larger than the mean CTI and the
total number of pixels in a 0.125$^{\text{o}}$ grid cell. See
section 7.3.1 and {\hyperref[\detokenize{tech_note/References/CLM50_Tech_Note_References:lietal2013b}]{\sphinxcrossref{\DUrole{std,std-ref}{Li et al. (2013b)}}}} for further details. Slope and
elevation are also obtained from the USGS HYDRO1K 1-km dataset
({\hyperref[\detokenize{tech_note/References/CLM50_Tech_Note_References:verdingreenlee1996}]{\sphinxcrossref{\DUrole{std,std-ref}{Verdin and Greenlee 1996}}}}).  Slope is used in the
surface water parameterization (section \hyperref[\detokenize{tech_note/Hydrology/CLM50_Tech_Note_Hydrology:surface-water-storage}]{\ref{\detokenize{tech_note/Hydrology/CLM50_Tech_Note_Hydrology:surface-water-storage}}}), and
elevation is used to calculate the grid cell standard deviation of
topography for the snow cover fraction parameterization (section \hyperref[\detokenize{tech_note/Snow_Hydrology/CLM50_Tech_Note_Snow_Hydrology:snow-covered-area-fraction}]{\ref{\detokenize{tech_note/Snow_Hydrology/CLM50_Tech_Note_Snow_Hydrology:snow-covered-area-fraction}}}).

Biogenic Volatile Organic Compounds emissions factors are from the Model
of Emissions of Gases and Aerosols from Nature version 2.1 (MEGAN2.1;
{\hyperref[\detokenize{tech_note/References/CLM50_Tech_Note_References:guentheretal2012}]{\sphinxcrossref{\DUrole{std,std-ref}{Guenther et al. 2012}}}}).

The default list of PFTs includes an unmanaged crop treated as a second
C3 grass (\hyperref[\detokenize{tech_note/Ecosystem/CLM50_Tech_Note_Ecosystem:table-plant-functional-types}]{Table \ref{\detokenize{tech_note/Ecosystem/CLM50_Tech_Note_Ecosystem:table-plant-functional-types}}}). The unmanaged crop has grid cell fractional cover
assigned from MODIS satellite data ({\hyperref[\detokenize{tech_note/References/CLM50_Tech_Note_References:lawrencechase2007}]{\sphinxcrossref{\DUrole{std,std-ref}{Lawrence and Chase (2007)}}}}). A managed
crop option uses grid cell fractional cover from the present-day crop
dataset of {\hyperref[\detokenize{tech_note/References/CLM50_Tech_Note_References:ramankuttyfoley1998}]{\sphinxcrossref{\DUrole{std,std-ref}{Ramankutty and Foley (1998)}}}}
(CLM4CNcrop). Managed crops are assigned in the proportions given by
{\hyperref[\detokenize{tech_note/References/CLM50_Tech_Note_References:ramankuttyfoley1998}]{\sphinxcrossref{\DUrole{std,std-ref}{Ramankutty and Foley (1998)}}}} without
exceeding the area previously assigned to the unmanaged crop. The
unmanaged crop continues to occupy any of its original area that remains
and continues to be handled just by the CN part of CLM4CNcrop. The
managed crop types (corn, soybean, and temperate cereals) were chosen
based on the availability of corresponding algorithms in AgroIBIS
({\hyperref[\detokenize{tech_note/References/CLM50_Tech_Note_References:kuchariketal2000}]{\sphinxcrossref{\DUrole{std,std-ref}{Kucharik et al. 2000}}}};
{\hyperref[\detokenize{tech_note/References/CLM50_Tech_Note_References:kucharikbrye2003}]{\sphinxcrossref{\DUrole{std,std-ref}{Kucharik and Brye 2003}}}}). Temperate cereals
include wheat, barley, and rye here. All temperate cereals are treated
as summer crops (like spring wheat, for example) at this time. Winter
cereals (such as winter wheat) may be introduced in a future version of
the model.

To allow crops to coexist with natural vegetation in a grid cell and be
treated by separate models (i.e., CLM4.5BGCcrop versus the Dynamic
Vegetation version (CLM4.5BGCDV)), we separate the vegetated land unit
into a naturally vegetated land unit and a human managed land unit. PFTs
in the naturally vegetated land unit share one soil column and compete
for water (default CLM setting). PFTs in the human managed land unit do
not share soil columns and thus permit for differences in land
management between crops.

CLM includes the option to irrigate cropland areas that are equipped for
irrigation. The application of irrigation responds dynamically to climate
(see Chapter \hyperref[\detokenize{tech_note/Crop_Irrigation/CLM50_Tech_Note_Crop_Irrigation:rst-crops-and-irrigation}]{\ref{\detokenize{tech_note/Crop_Irrigation/CLM50_Tech_Note_Crop_Irrigation:rst-crops-and-irrigation}}}). In CLM, irrigation is
implemented for the C3 generic crop only. When irrigation is enabled, the
cropland area of each grid cell is divided into an irrigated and unirrigated
fraction according to a dataset of areas equipped for irrigation
({\hyperref[\detokenize{tech_note/References/CLM50_Tech_Note_References:siebertetal2005}]{\sphinxcrossref{\DUrole{std,std-ref}{Siebert et al. (2005)}}}}). The area of irrigated
cropland in each grid cell is given by the
smaller of the grid cell’s total cropland area, according to the default
CLM4 dataset, and the grid cell’s area equipped for irrigation. The
remainder of the grid cell’s cropland area (if any) is then assigned to
unirrigated cropland. Irrigated and unirrigated crops are placed on
separate soil columns, so that irrigation is only applied to the soil
beneath irrigated crops.

Several input datasets are required for the fire model ({\hyperref[\detokenize{tech_note/References/CLM50_Tech_Note_References:lietal2013a}]{\sphinxcrossref{\DUrole{std,std-ref}{Li et al. 2013a}}}})
including population density, gross domestic production, peat area
fraction, and peak month of agricultural waste burning. Population
density at 0.5$^{\text{o}}$ resolution for 1850-2100 combines 5-min
resolution decadal population density data for 1850\textendash{}1980 from the
Database of the Global Environment version 3.1 (HYDEv3.1) with
0.5$^{\text{o}}$ resolution population density data for 1990, 1995,
2000, and 2005 from the Gridded Population of the World version 3
dataset (GPWv3) (CIESIN, 2005). Gross Domestic Production (GDP) per
capita in 2000 at 0.5$^{\text{o}}$ is from {\hyperref[\detokenize{tech_note/References/CLM50_Tech_Note_References:vanvuurenetal2006}]{\sphinxcrossref{\DUrole{std,std-ref}{Van Vuuren et al. (2006)}}}},
which is the base-year GDP data for IPCC-SRES and derived from
country-level World Bank’s World Development Indicators (WDI) measured
in constant 1995 US\$ ({\hyperref[\detokenize{tech_note/References/CLM50_Tech_Note_References:worldbank2004}]{\sphinxcrossref{\DUrole{std,std-ref}{World Bank, 2004}}}}) and the UN Statistics Database
({\hyperref[\detokenize{tech_note/References/CLM50_Tech_Note_References:unstat2005}]{\sphinxcrossref{\DUrole{std,std-ref}{UNSTAT, 2005}}}}). The peatland area fraction at 0.5$^{\text{o}}$
resolution is derived from three vector datasets: peatland data in
Indonesia and Malaysian Borneo ({\hyperref[\detokenize{tech_note/References/CLM50_Tech_Note_References:olsonetal2001}]{\sphinxcrossref{\DUrole{std,std-ref}{Olson et al. 2001}}}}); peatland data in
Canada ({\hyperref[\detokenize{tech_note/References/CLM50_Tech_Note_References:tarnocaietal2011}]{\sphinxcrossref{\DUrole{std,std-ref}{Tarnocai et al. 2011}}}}); and bog, fen and mire data in boreal
regions (north of 45$^{\text{o}}$N) outside Canada provided by the
Global Lakes and Wetlands Database (GLWD) ({\hyperref[\detokenize{tech_note/References/CLM50_Tech_Note_References:lehnerdoll2004}]{\sphinxcrossref{\DUrole{std,std-ref}{Lehner and Döll, 2004}}}}). The
climatological peak month for agricultural waste burning is from {\hyperref[\detokenize{tech_note/References/CLM50_Tech_Note_References:vanderwerfetal2010}]{\sphinxcrossref{\DUrole{std,std-ref}{van der
Werf et al. (2010)}}}}.


\subsubsection{Adjustable Parameters and Physical Constants}
\label{\detokenize{tech_note/Ecosystem/CLM50_Tech_Note_Ecosystem:adjustable-parameters-and-physical-constants}}\label{\detokenize{tech_note/Ecosystem/CLM50_Tech_Note_Ecosystem:id13}}
Values of certain adjustable parameters inherent in the biogeophysical
or biogeochemical parameterizations have either been obtained from the
literature or calibrated based on comparisons with observations. These
are described in the text. Physical constants, generally shared by all
of the components in the coupled modeling system, are presented in
\hyperref[\detokenize{tech_note/Ecosystem/CLM50_Tech_Note_Ecosystem:table-physical-constants}]{Table \ref{\detokenize{tech_note/Ecosystem/CLM50_Tech_Note_Ecosystem:table-physical-constants}}}.


\begin{savenotes}\sphinxattablestart
\centering
\sphinxcapstartof{table}
\sphinxcaption{Physical constants}\label{\detokenize{tech_note/Ecosystem/CLM50_Tech_Note_Ecosystem:table-physical-constants}}\label{\detokenize{tech_note/Ecosystem/CLM50_Tech_Note_Ecosystem:id21}}
\sphinxaftercaption
\begin{tabular}[t]{|\X{40}{100}|\X{20}{100}|\X{20}{100}|\X{20}{100}|}
\hline
\sphinxstylethead{\sphinxstyletheadfamily 
description
\unskip}\relax &\sphinxstylethead{\sphinxstyletheadfamily 
name
\unskip}\relax &\sphinxstylethead{\sphinxstyletheadfamily 
value
\unskip}\relax &\sphinxstylethead{\sphinxstyletheadfamily 
units
\unskip}\relax \\
\hline
Pi
&
\(\pi\)
&
3.14159265358979323846
&
-
\\
\hline
Acceleration of gravity
&
\(g\)
&
9.80616
&
m s$^{\text{-2}}$
\\
\hline
Standard pressure
&
\(P_{std}\)
&
101325
&
Pa
\\
\hline
Stefan-Boltzmann constant
&
\(\sigma\)
&
5.67 \(\times 10^{-8}\)
&
W m $^{\text{-2}}$ K \({}^{-4}\)
\\
\hline
Boltzmann constant
&
\(\kappa\)
&
1.38065 \(\times 10^{-23}\)
&
J K $^{\text{-1}}$ molecule $^{\text{-1}}$
\\
\hline
Avogadro’s number
&
\(N_{A}\)
&
6.02214 \(\times 10^{26}\)
&
molecule kmol$^{\text{-1}}$
\\
\hline
Universal gas constant
&
\(R_{gas}\)
&
\(N_{A} \kappa\)
&
J K $^{\text{-1}}$ kmol $^{\text{-1}}$
\\
\hline
Molecular weight of dry air
&
\(MW_{da}\)
&
28.966
&
kg kmol $^{\text{-1}}$
\\
\hline
Dry air gas constant
&
\(R_{da}\)
&
\({R_{gas} \mathord{\left/ {\vphantom {R_{gas}  MW_{da} }} \right. \kern-\nulldelimiterspace} MW_{da} }\)
&
J K $^{\text{-1}}$ kg $^{\text{-1}}$
\\
\hline
Molecular weight of water vapor
&
\(MW_{wv}\)
&
18.016
&
kg kmol $^{\text{-1}}$
\\
\hline
Water vapor gas constant
&
\(R_{wv}\)
&
\({R_{gas} \mathord{\left/ {\vphantom {R_{gas}  MW_{wv} }} \right. \kern-\nulldelimiterspace} MW_{wv} }\)
&
J K $^{\text{-1}}$ kg $^{\text{-1}}$
\\
\hline
Von Karman constant
&
\(k\)
&
0.4
&
-
\\
\hline
Freezing temperature of fresh water
&
\(T_{f}\)
&
273.15
&
K
\\
\hline
Density of liquid water
&
\(\rho _{liq}\)
&
1000
&
kg m $^{\text{-3}}$
\\
\hline
Density of ice
&
\(\rho _{ice}\)
&
917
&
kg m $^{\text{-3}}$
\\
\hline
Specific heat capacity of dry air
&
\(C_{p}\)
&
1.00464 \(\times 10^{3}\)
&
J kg $^{\text{-1}}$ K $^{\text{-1}}$
\\
\hline
Specific heat capacity of water
&
\(C_{liq}\)
&
4.188 \(\times 10^{3}\)
&
J kg $^{\text{-1}}$ K $^{\text{-1}}$
\\
\hline
Specific heat capacity of ice
&
\(C_{ice}\)
&
2.11727 \(\times 10^{3}\)
&
J kg $^{\text{-1}}$ K $^{\text{-1}}$
\\
\hline
Latent heat of vaporization
&
\(\lambda _{vap}\)
&
2.501 \(\times 10^{6}\)
&
J kg $^{\text{-1}}$
\\
\hline
Latent heat of fusion
&
\(L_{f}\)
&
3.337 \(\times 10^{5}\)
&
J kg $^{\text{-1}}$
\\
\hline
Latent heat of sublimation
&
\(\lambda _{sub}\)
&
\(\lambda _{vap} +L_{f}\)
&
J kg $^{\text{-1}}$
\\
\hline
$^{\text{1}}$ “Thermal conductivity of water”
&
\(\lambda _{liq}\)
&
0.57
&
W m $^{\text{-1}}$ K $^{\text{-1}}$
\\
\hline
$^{\text{1}}$ “Thermal conductivity of ice”
&
\(\lambda _{ice}\)
&
2.29
&
W m $^{\text{-1}}$ K $^{\text{-1}}$
\\
\hline
$^{\text{1}}$ “Thermal conductivity of air”
&
\(\lambda _{air}\)
&
0.023 W m $^{\text{-1}}$ K $^{\text{-1}}$
&\\
\hline
Radius of the earth
&
\(R_{e}\)
&
6.37122
&
\(\times 10^{6}\) m
\\
\hline
\end{tabular}
\par
\sphinxattableend\end{savenotes}

$^{\text{1}}$Not shared by other components of the coupled modeling system.


\section{Surface Albedos}
\label{\detokenize{tech_note/Surface_Albedos/CLM50_Tech_Note_Surface_Albedos:surface-albedos}}\label{\detokenize{tech_note/Surface_Albedos/CLM50_Tech_Note_Surface_Albedos::doc}}\label{\detokenize{tech_note/Surface_Albedos/CLM50_Tech_Note_Surface_Albedos:rst-surface-albedos}}

\subsection{Canopy Radiative Transfer}
\label{\detokenize{tech_note/Surface_Albedos/CLM50_Tech_Note_Surface_Albedos:id1}}\label{\detokenize{tech_note/Surface_Albedos/CLM50_Tech_Note_Surface_Albedos:canopy-radiative-transfer}}
Radiative transfer within vegetative canopies is calculated from the
two-stream approximation of {\hyperref[\detokenize{tech_note/References/CLM50_Tech_Note_References:dickinson1983}]{\sphinxcrossref{\DUrole{std,std-ref}{Dickinson (1983)}}}} and
{\hyperref[\detokenize{tech_note/References/CLM50_Tech_Note_References:sellers1985}]{\sphinxcrossref{\DUrole{std,std-ref}{Sellers (1985)}}}} as described by {\hyperref[\detokenize{tech_note/References/CLM50_Tech_Note_References:bonan1996}]{\sphinxcrossref{\DUrole{std,std-ref}{Bonan (1996)}}}}
\phantomsection\label{\detokenize{tech_note/Surface_Albedos/CLM50_Tech_Note_Surface_Albedos:equation-3.1}}\begin{equation}\label{equation:tech_note/Surface_Albedos/CLM50_Tech_Note_Surface_Albedos:3.1}
\begin{split}-\bar{\mu }\frac{dI\, \uparrow }{d\left(L+S\right)} +\left[1-\left(1-\beta \right)\omega \right]I\, \uparrow -\omega \beta I\, \downarrow =\omega \bar{\mu }K\beta _{0} e^{-K\left(L+S\right)}\end{split}
\end{equation}\phantomsection\label{\detokenize{tech_note/Surface_Albedos/CLM50_Tech_Note_Surface_Albedos:equation-3.2}}\begin{equation}\label{equation:tech_note/Surface_Albedos/CLM50_Tech_Note_Surface_Albedos:3.2}
\begin{split}\bar{\mu }\frac{dI\, \downarrow }{d\left(L+S\right)} +\left[1-\left(1-\beta \right)\omega \right]I\, \downarrow -\omega \beta I\, \uparrow =\omega \bar{\mu }K\left(1-\beta _{0} \right)e^{-K\left(L+S\right)}\end{split}
\end{equation}
where \(I\, \uparrow\)  and \(I\, \downarrow\)  are the upward
and downward diffuse radiative fluxes per unit incident flux,
\(K={G\left(\mu \right)\mathord{\left/ {\vphantom {G\left(\mu \right) \mu }} \right. \kern-\nulldelimiterspace} \mu }\)
is the optical depth of direct beam per unit leaf and stem area,
\(\mu\)  is the cosine of the zenith angle of the incident beam,
\(G\left(\mu \right)\) is the relative projected area of leaf and
stem elements in the direction \(\cos ^{-1} \mu\) ,
\(\bar{\mu }\) is the average inverse diffuse optical depth per unit
leaf and stem area, \(\omega\)  is a scattering coefficient,
\(\beta\)  and \(\beta _{0}\)  are upscatter parameters for
diffuse and direct beam radiation, respectively, \(L\) is the
exposed leaf area index , and \(S\) is the exposed stem area index
(section \hyperref[\detokenize{tech_note/Ecosystem/CLM50_Tech_Note_Ecosystem:phenology-and-vegetation-burial-by-snow}]{\ref{\detokenize{tech_note/Ecosystem/CLM50_Tech_Note_Ecosystem:phenology-and-vegetation-burial-by-snow}}}). Given the
direct beam albedo \(\alpha _{g,\, \Lambda }^{\mu }\)  and diffuse albedo
\(\alpha _{g,\, \Lambda }\)  of the ground (section \hyperref[\detokenize{tech_note/Surface_Albedos/CLM50_Tech_Note_Surface_Albedos:ground-albedos}]{\ref{\detokenize{tech_note/Surface_Albedos/CLM50_Tech_Note_Surface_Albedos:ground-albedos}}}), these
equations are solved to calculate the fluxes, per unit incident flux,
absorbed by the vegetation, reflected by the vegetation, and transmitted
through the vegetation for direct and diffuse radiation and for visible
(\(<\) 0.7\(\mu {\rm m}\)) and near-infrared
(\(\geq\) 0.7\(\mu {\rm m}\)) wavebands. The absorbed
radiation is partitioned to sunlit and shaded fractions of the canopy.
The optical parameters \(G\left(\mu \right)\), \(\bar{\mu }\),
\(\omega\), \(\beta\), and \(\beta _{0}\)  are calculated
based on work in {\hyperref[\detokenize{tech_note/References/CLM50_Tech_Note_References:sellers1985}]{\sphinxcrossref{\DUrole{std,std-ref}{Sellers (1985)}}}} as follows.

The relative projected area of leaves and stems in the direction
\(\cos ^{-1} \mu\)  is
\phantomsection\label{\detokenize{tech_note/Surface_Albedos/CLM50_Tech_Note_Surface_Albedos:equation-3.3}}\begin{equation}\label{equation:tech_note/Surface_Albedos/CLM50_Tech_Note_Surface_Albedos:3.3}
\begin{split}G\left(\mu \right)=\phi _{1} +\phi _{2} \mu\end{split}
\end{equation}
where \(\phi _{1} ={\rm 0.5}-0.633\chi _{L} -0.33\chi _{L}^{2}\)
and \(\phi _{2} =0.877\left(1-2\phi _{1} \right)\) for
\(-0.4\le \chi _{L} \le 0.6\). \(\chi _{L}\)  is the departure
of leaf angles from a random distribution and equals +1 for horizontal
leaves, 0 for random leaves, and \textendash{}1 for vertical leaves.

The average inverse diffuse optical depth per unit leaf and stem area is
\phantomsection\label{\detokenize{tech_note/Surface_Albedos/CLM50_Tech_Note_Surface_Albedos:equation-3.4}}\begin{equation}\label{equation:tech_note/Surface_Albedos/CLM50_Tech_Note_Surface_Albedos:3.4}
\begin{split}\bar{\mu }=\int _{0}^{1}\frac{\mu '}{G\left(\mu '\right)}  d\mu '=\frac{1}{\phi _{2} } \left[1-\frac{\phi _{1} }{\phi _{2} } \ln \left(\frac{\phi _{1} +\phi _{2} }{\phi _{1} } \right)\right]\end{split}
\end{equation}
where \(\mu '\) is the direction of the scattered flux.

The optical parameters \(\omega\), \(\beta\), and \(\beta _{0}\),
which vary with wavelength (\(\Lambda\) ), are weighted combinations of values
for vegetation and snow, using the canopy snow-covered fraction \(f_{can,\, sno}\)
(Chapter \hyperref[\detokenize{tech_note/Hydrology/CLM50_Tech_Note_Hydrology:rst-hydrology}]{\ref{\detokenize{tech_note/Hydrology/CLM50_Tech_Note_Hydrology:rst-hydrology}}}). The optical parameters are
\phantomsection\label{\detokenize{tech_note/Surface_Albedos/CLM50_Tech_Note_Surface_Albedos:equation-3.5}}\begin{equation}\label{equation:tech_note/Surface_Albedos/CLM50_Tech_Note_Surface_Albedos:3.5}
\begin{split}\omega _{\Lambda } =\omega _{\Lambda }^{veg} \left(1-f_{can,\, sno} \right)+\omega _{\Lambda }^{sno} f_{can,\, sno}\end{split}
\end{equation}\phantomsection\label{\detokenize{tech_note/Surface_Albedos/CLM50_Tech_Note_Surface_Albedos:equation-3.6}}\begin{equation}\label{equation:tech_note/Surface_Albedos/CLM50_Tech_Note_Surface_Albedos:3.6}
\begin{split}\omega _{\Lambda } \beta _{\Lambda } =\omega _{\Lambda }^{veg} \beta _{\Lambda }^{veg} \left(1-f_{can,\, sno} \right)+\omega _{\Lambda }^{sno} \beta _{\Lambda }^{sno} f_{can,\, sno}\end{split}
\end{equation}\phantomsection\label{\detokenize{tech_note/Surface_Albedos/CLM50_Tech_Note_Surface_Albedos:equation-3.7}}\begin{equation}\label{equation:tech_note/Surface_Albedos/CLM50_Tech_Note_Surface_Albedos:3.7}
\begin{split}\omega _{\Lambda } \beta _{0,\, \Lambda } =\omega _{\Lambda }^{veg} \beta _{0,\, \Lambda }^{veg} \left(1-f_{can,\, sno} \right)+\omega _{\Lambda }^{sno} \beta _{0,\, \Lambda }^{sno} f_{can,\, sno}\end{split}
\end{equation}
The snow and vegetation weights are applied to the products
\(\omega _{\Lambda } \beta _{\Lambda }\)  and \(\omega _{\Lambda } \beta _{0,\, \Lambda }\)  because these
products are used in the two-stream equations. If there is no snow on the canopy, this reduces to
\phantomsection\label{\detokenize{tech_note/Surface_Albedos/CLM50_Tech_Note_Surface_Albedos:equation-3.8}}\begin{equation}\label{equation:tech_note/Surface_Albedos/CLM50_Tech_Note_Surface_Albedos:3.8}
\begin{split}\omega _{\Lambda } =\omega _{\Lambda }^{veg}\end{split}
\end{equation}\phantomsection\label{\detokenize{tech_note/Surface_Albedos/CLM50_Tech_Note_Surface_Albedos:equation-3.9}}\begin{equation}\label{equation:tech_note/Surface_Albedos/CLM50_Tech_Note_Surface_Albedos:3.9}
\begin{split}\omega _{\Lambda } \beta _{\Lambda } =\omega _{\Lambda }^{veg} \beta _{\Lambda }^{veg}\end{split}
\end{equation}\phantomsection\label{\detokenize{tech_note/Surface_Albedos/CLM50_Tech_Note_Surface_Albedos:equation-3.10}}\begin{equation}\label{equation:tech_note/Surface_Albedos/CLM50_Tech_Note_Surface_Albedos:3.10}
\begin{split}\omega _{\Lambda } \beta _{0,\, \Lambda } =\omega _{\Lambda }^{veg} \beta _{0,\, \Lambda }^{veg} .\end{split}
\end{equation}
For vegetation,
\(\omega _{\Lambda }^{veg} =\alpha _{\Lambda } +\tau _{\Lambda }\) .
\(\alpha _{\Lambda }\)  is a weighted combination of the leaf and
stem reflectances
(\(\alpha _{\Lambda }^{leaf} ,\alpha _{\Lambda }^{stem}\) )
\phantomsection\label{\detokenize{tech_note/Surface_Albedos/CLM50_Tech_Note_Surface_Albedos:equation-3.11}}\begin{equation}\label{equation:tech_note/Surface_Albedos/CLM50_Tech_Note_Surface_Albedos:3.11}
\begin{split}\alpha _{\Lambda } =\alpha _{\Lambda }^{leaf} w_{leaf} +\alpha _{\Lambda }^{stem} w_{stem}\end{split}
\end{equation}
where
\(w_{leaf} ={L\mathord{\left/ {\vphantom {L \left(L+S\right)}} \right. \kern-\nulldelimiterspace} \left(L+S\right)}\)
and
\(w_{stem} ={S\mathord{\left/ {\vphantom {S \left(L+S\right)}} \right. \kern-\nulldelimiterspace} \left(L+S\right)}\) .
\(\tau _{\Lambda }\)  is a weighted combination of the leaf and stem transmittances (\(\tau _{\Lambda }^{leaf}, \tau _{\Lambda }^{stem}\))
\phantomsection\label{\detokenize{tech_note/Surface_Albedos/CLM50_Tech_Note_Surface_Albedos:equation-3.12}}\begin{equation}\label{equation:tech_note/Surface_Albedos/CLM50_Tech_Note_Surface_Albedos:3.12}
\begin{split}\tau _{\Lambda } =\tau _{\Lambda }^{leaf} w_{leaf} +\tau _{\Lambda }^{stem} w_{stem} .\end{split}
\end{equation}
The upscatter for diffuse radiation is
\phantomsection\label{\detokenize{tech_note/Surface_Albedos/CLM50_Tech_Note_Surface_Albedos:equation-3.13}}\begin{equation}\label{equation:tech_note/Surface_Albedos/CLM50_Tech_Note_Surface_Albedos:3.13}
\begin{split}\omega _{\Lambda }^{veg} \beta _{\Lambda }^{veg} =\frac{1}{2} \left[\alpha _{\Lambda } +\tau _{\Lambda } +\left(\alpha _{\Lambda } -\tau _{\Lambda } \right)\cos ^{2} \bar{\theta }\right]\end{split}
\end{equation}
where \(\bar{\theta }\) is the mean leaf inclination angle relative
to the horizontal plane (i.e., the angle between leaf normal and local
vertical) ({\hyperref[\detokenize{tech_note/References/CLM50_Tech_Note_References:sellers1985}]{\sphinxcrossref{\DUrole{std,std-ref}{Sellers (1985)}}}}). Here, \(\cos \bar{\theta }\) is
approximated by
\phantomsection\label{\detokenize{tech_note/Surface_Albedos/CLM50_Tech_Note_Surface_Albedos:equation-3.14}}\begin{equation}\label{equation:tech_note/Surface_Albedos/CLM50_Tech_Note_Surface_Albedos:3.14}
\begin{split}\cos \bar{\theta }=\frac{1+\chi _{L} }{2}\end{split}
\end{equation}
Using this approximation, for vertical leaves (\(\chi _{L} =-1\),
\(\bar{\theta }=90^{{\rm o}}\) ),
\(\omega _{\Lambda }^{veg} \beta _{\Lambda }^{veg} =0.5\left(\alpha _{\Lambda } +\tau _{\Lambda } \right)\),
and for horizontal leaves (\(\chi _{L} =1\),
\(\bar{\theta }=0^{{\rm o}}\) ) ,
\(\omega _{\Lambda }^{veg} \beta _{\Lambda }^{veg} =\alpha _{\Lambda }\) ,
which agree with both {\hyperref[\detokenize{tech_note/References/CLM50_Tech_Note_References:dickinson1983}]{\sphinxcrossref{\DUrole{std,std-ref}{Dickinson (1983)}}}} and {\hyperref[\detokenize{tech_note/References/CLM50_Tech_Note_References:sellers1985}]{\sphinxcrossref{\DUrole{std,std-ref}{Sellers (1985)}}}}. For random
(spherically distributed) leaves (\(\chi _{L} =0\),
\(\bar{\theta }=60^{{\rm o}}\) ), the approximation yields
\(\omega _{\Lambda }^{veg} \beta _{\Lambda }^{veg} ={5\mathord{\left/ {\vphantom {5 8}} \right. \kern-\nulldelimiterspace} 8} \alpha _{\Lambda } +{3\mathord{\left/ {\vphantom {3 8}} \right. \kern-\nulldelimiterspace} 8} \tau _{\Lambda }\)
whereas the approximate solution of {\hyperref[\detokenize{tech_note/References/CLM50_Tech_Note_References:dickinson1983}]{\sphinxcrossref{\DUrole{std,std-ref}{Dickinson (1983)}}}} is
\(\omega _{\Lambda }^{veg} \beta _{\Lambda }^{veg} ={2\mathord{\left/ {\vphantom {2 3}} \right. \kern-\nulldelimiterspace} 3} \alpha _{\Lambda } +{1\mathord{\left/ {\vphantom {1 3}} \right. \kern-\nulldelimiterspace} 3} \tau _{\Lambda }\) .
This discrepancy arises from the fact that a spherical leaf angle
distribution has a true mean leaf inclination
\(\bar{\theta }\approx 57\) {\hyperref[\detokenize{tech_note/References/CLM50_Tech_Note_References:campbellnorman1998}]{\sphinxcrossref{\DUrole{std,std-ref}{(Campbell and Norman 1998)}}}} in equation ,
while \(\bar{\theta }=60\) in equation . The upscatter for direct
beam radiation is
\phantomsection\label{\detokenize{tech_note/Surface_Albedos/CLM50_Tech_Note_Surface_Albedos:equation-3.15}}\begin{equation}\label{equation:tech_note/Surface_Albedos/CLM50_Tech_Note_Surface_Albedos:3.15}
\begin{split}\omega _{\Lambda }^{veg} \beta _{0,\, \Lambda }^{veg} =\frac{1+\bar{\mu }K}{\bar{\mu }K} a_{s} \left(\mu \right)_{\Lambda }\end{split}
\end{equation}
where the single scattering albedo is
\phantomsection\label{\detokenize{tech_note/Surface_Albedos/CLM50_Tech_Note_Surface_Albedos:equation-3.16}}\begin{equation}\label{equation:tech_note/Surface_Albedos/CLM50_Tech_Note_Surface_Albedos:3.16}
\begin{split}\begin{array}{rcl} {a_{s} \left(\mu \right)_{\Lambda } } & {=} & {\frac{\omega _{\Lambda }^{veg} }{2} \int _{0}^{1}\frac{\mu 'G\left(\mu \right)}{\mu G\left(\mu '\right)+\mu 'G\left(\mu \right)}  d\mu '} \\ {} & {=} & {\frac{\omega _{\Lambda }^{veg} }{2} \frac{G\left(\mu \right)}{\min (\mu \phi _{2} +G\left(\mu \right),1e-6)} \left[1-\frac{\mu \phi _{1} }{\min (\mu \phi _{2} +G\left(\mu \right),1e-6)} \ln \left(\frac{\mu \phi _{1} +\min (\mu \phi _{2} +G\left(\mu \right),1e-6)}{\mu \phi _{1} } \right)\right].} \end{array}\end{split}
\end{equation}
Note here the restriction on \(\mu \phi _{2} +G\left(\mu \right)\).  We have seen cases where small values
can cause unrealistic single scattering albedo associated with the log calculation,
thereby eventually causing a negative soil albedo.

The upward diffuse fluxes per unit incident direct beam and diffuse flux
(i.e., the surface albedos) are
\phantomsection\label{\detokenize{tech_note/Surface_Albedos/CLM50_Tech_Note_Surface_Albedos:equation-3.17}}\begin{equation}\label{equation:tech_note/Surface_Albedos/CLM50_Tech_Note_Surface_Albedos:3.17}
\begin{split}I\, \uparrow _{\Lambda }^{\mu } =\frac{h_{1} }{\sigma } +h_{2} +h_{3}\end{split}
\end{equation}\phantomsection\label{\detokenize{tech_note/Surface_Albedos/CLM50_Tech_Note_Surface_Albedos:equation-3.18}}\begin{equation}\label{equation:tech_note/Surface_Albedos/CLM50_Tech_Note_Surface_Albedos:3.18}
\begin{split}I\, \uparrow _{\Lambda } =h_{7} +h_{8} .\end{split}
\end{equation}
The downward diffuse fluxes per unit incident direct beam and diffuse
radiation, respectively, are
\phantomsection\label{\detokenize{tech_note/Surface_Albedos/CLM50_Tech_Note_Surface_Albedos:equation-3.19}}\begin{equation}\label{equation:tech_note/Surface_Albedos/CLM50_Tech_Note_Surface_Albedos:3.19}
\begin{split}I\, \downarrow _{\Lambda }^{\mu } =\frac{h_{4} }{\sigma } e^{-K\left(L+S\right)} +h_{5} s_{1} +\frac{h_{6} }{s_{1} }\end{split}
\end{equation}\phantomsection\label{\detokenize{tech_note/Surface_Albedos/CLM50_Tech_Note_Surface_Albedos:equation-3.20}}\begin{equation}\label{equation:tech_note/Surface_Albedos/CLM50_Tech_Note_Surface_Albedos:3.20}
\begin{split}I\, \downarrow _{\Lambda } =h_{9} s_{1} +\frac{h_{10} }{s_{1} } .\end{split}
\end{equation}
With reference to \hyperref[\detokenize{tech_note/Radiative_Fluxes/CLM50_Tech_Note_Radiative_Fluxes:figure-radiation-schematic}]{Figure \ref{\detokenize{tech_note/Radiative_Fluxes/CLM50_Tech_Note_Radiative_Fluxes:figure-radiation-schematic}}}, the direct beam flux transmitted through
the canopy, per unit incident flux, is \(e^{-K\left(L+S\right)}\) ,
and the direct beam and diffuse fluxes absorbed by the vegetation, per
unit incident flux, are
\phantomsection\label{\detokenize{tech_note/Surface_Albedos/CLM50_Tech_Note_Surface_Albedos:equation-3.21}}\begin{equation}\label{equation:tech_note/Surface_Albedos/CLM50_Tech_Note_Surface_Albedos:3.21}
\begin{split}\vec{I}_{\Lambda }^{\mu } =1-I\, \uparrow _{\Lambda }^{\mu } -\left(1-\alpha _{g,\, \Lambda } \right)I\, \downarrow _{\Lambda }^{\mu } -\left(1-\alpha _{g,\, \Lambda }^{\mu } \right)e^{-K\left(L+S\right)}\end{split}
\end{equation}\phantomsection\label{\detokenize{tech_note/Surface_Albedos/CLM50_Tech_Note_Surface_Albedos:equation-3.22}}\begin{equation}\label{equation:tech_note/Surface_Albedos/CLM50_Tech_Note_Surface_Albedos:3.22}
\begin{split}\vec{I}_{\Lambda } =1-I\, \uparrow _{\Lambda } -\left(1-\alpha _{g,\, \Lambda } \right)I\, \downarrow _{\Lambda } .\end{split}
\end{equation}
These fluxes are partitioned to the sunlit and shaded canopy using an
analytical solution to the two-stream approximation for sunlit and
shaded leaves {\hyperref[\detokenize{tech_note/References/CLM50_Tech_Note_References:daietal2004}]{\sphinxcrossref{\DUrole{std,std-ref}{(Dai et al. 2004)}}}}, as described by {\hyperref[\detokenize{tech_note/References/CLM50_Tech_Note_References:bonanetal2011}]{\sphinxcrossref{\DUrole{std,std-ref}{Bonan et al. (2011)}}}}.
The absorption of direct beam radiation by sunlit leaves is
\phantomsection\label{\detokenize{tech_note/Surface_Albedos/CLM50_Tech_Note_Surface_Albedos:equation-3.23}}\begin{equation}\label{equation:tech_note/Surface_Albedos/CLM50_Tech_Note_Surface_Albedos:3.23}
\begin{split}\vec{I}_{sun,\Lambda }^{\mu } =\left(1-\omega _{\Lambda } \right)\left[1-s_{2} +\frac{1}{\bar{\mu }} \left(a_{1} +a_{2} \right)\right]\end{split}
\end{equation}
and for shaded leaves is
\phantomsection\label{\detokenize{tech_note/Surface_Albedos/CLM50_Tech_Note_Surface_Albedos:equation-3.24}}\begin{equation}\label{equation:tech_note/Surface_Albedos/CLM50_Tech_Note_Surface_Albedos:3.24}
\begin{split}\vec{I}_{sha,\Lambda }^{\mu } =\vec{I}_{\Lambda }^{\mu } -\vec{I}_{sun,\Lambda }^{\mu }\end{split}
\end{equation}
with
\phantomsection\label{\detokenize{tech_note/Surface_Albedos/CLM50_Tech_Note_Surface_Albedos:equation-3.25}}\begin{equation}\label{equation:tech_note/Surface_Albedos/CLM50_Tech_Note_Surface_Albedos:3.25}
\begin{split}a_{1} =\frac{h_{1} }{\sigma } \left[\frac{1-s_{2}^{2} }{2K} \right]+h_{2} \left[\frac{1-s_{2} s_{1} }{K+h} \right]+h_{3} \left[\frac{1-{s_{2} \mathord{\left/ {\vphantom {s_{2}  s_{1} }} \right. \kern-\nulldelimiterspace} s_{1} } }{K-h} \right]\end{split}
\end{equation}\phantomsection\label{\detokenize{tech_note/Surface_Albedos/CLM50_Tech_Note_Surface_Albedos:equation-3.26}}\begin{equation}\label{equation:tech_note/Surface_Albedos/CLM50_Tech_Note_Surface_Albedos:3.26}
\begin{split}a_{2} =\frac{h_{4} }{\sigma } \left[\frac{1-s_{2}^{2} }{2K} \right]+h_{5} \left[\frac{1-s_{2} s_{1} }{K+h} \right]+h_{6} \left[\frac{1-{s_{2} \mathord{\left/ {\vphantom {s_{2}  s_{1} }} \right. \kern-\nulldelimiterspace} s_{1} } }{K-h} \right].\end{split}
\end{equation}
For diffuse radiation, the absorbed radiation for sunlit leaves is
\phantomsection\label{\detokenize{tech_note/Surface_Albedos/CLM50_Tech_Note_Surface_Albedos:equation-3.27}}\begin{equation}\label{equation:tech_note/Surface_Albedos/CLM50_Tech_Note_Surface_Albedos:3.27}
\begin{split}\vec{I}_{sun,\Lambda }^{} =\left[\frac{1-\omega _{\Lambda } }{\bar{\mu }} \right]\left(a_{1} +a_{2} \right)\end{split}
\end{equation}
and for shaded leaves is
\phantomsection\label{\detokenize{tech_note/Surface_Albedos/CLM50_Tech_Note_Surface_Albedos:equation-3.28}}\begin{equation}\label{equation:tech_note/Surface_Albedos/CLM50_Tech_Note_Surface_Albedos:3.28}
\begin{split}\vec{I}_{sha,\Lambda }^{} =\vec{I}_{\Lambda }^{} -\vec{I}_{sun,\Lambda }^{}\end{split}
\end{equation}
with
\phantomsection\label{\detokenize{tech_note/Surface_Albedos/CLM50_Tech_Note_Surface_Albedos:equation-3.29}}\begin{equation}\label{equation:tech_note/Surface_Albedos/CLM50_Tech_Note_Surface_Albedos:3.29}
\begin{split}a_{1} =h_{7} \left[\frac{1-s_{2} s_{1} }{K+h} \right]+h_{8} \left[\frac{1-{s_{2} \mathord{\left/ {\vphantom {s_{2}  s_{1} }} \right. \kern-\nulldelimiterspace} s_{1} } }{K-h} \right]\end{split}
\end{equation}\phantomsection\label{\detokenize{tech_note/Surface_Albedos/CLM50_Tech_Note_Surface_Albedos:equation-3.30}}\begin{equation}\label{equation:tech_note/Surface_Albedos/CLM50_Tech_Note_Surface_Albedos:3.30}
\begin{split}a_{2} =h_{9} \left[\frac{1-s_{2} s_{1} }{K+h} \right]+h_{10} \left[\frac{1-{s_{2} \mathord{\left/ {\vphantom {s_{2}  s_{1} }} \right. \kern-\nulldelimiterspace} s_{1} } }{K-h} \right].\end{split}
\end{equation}
The parameters \(h_{1}\) \textendash{}\(h_{10}\) , \(\sigma\) ,
\(h\), \(s_{1}\) , and \(s_{2}\)  are from {\hyperref[\detokenize{tech_note/References/CLM50_Tech_Note_References:sellers1985}]{\sphinxcrossref{\DUrole{std,std-ref}{Sellers (1985)}}}}
{[}note the error in \(h_{4}\)  in {\hyperref[\detokenize{tech_note/References/CLM50_Tech_Note_References:sellers1985}]{\sphinxcrossref{\DUrole{std,std-ref}{Sellers (1985)}}}}{]}:
\phantomsection\label{\detokenize{tech_note/Surface_Albedos/CLM50_Tech_Note_Surface_Albedos:equation-3.31}}\begin{equation}\label{equation:tech_note/Surface_Albedos/CLM50_Tech_Note_Surface_Albedos:3.31}
\begin{split}b=1-\omega _{\Lambda } +\omega _{\Lambda } \beta _{\Lambda }\end{split}
\end{equation}\phantomsection\label{\detokenize{tech_note/Surface_Albedos/CLM50_Tech_Note_Surface_Albedos:equation-3.32}}\begin{equation}\label{equation:tech_note/Surface_Albedos/CLM50_Tech_Note_Surface_Albedos:3.32}
\begin{split}c=\omega _{\Lambda } \beta _{\Lambda }\end{split}
\end{equation}\phantomsection\label{\detokenize{tech_note/Surface_Albedos/CLM50_Tech_Note_Surface_Albedos:equation-3.33}}\begin{equation}\label{equation:tech_note/Surface_Albedos/CLM50_Tech_Note_Surface_Albedos:3.33}
\begin{split}d=\omega _{\Lambda } \bar{\mu }K\beta _{0,\, \Lambda }\end{split}
\end{equation}\phantomsection\label{\detokenize{tech_note/Surface_Albedos/CLM50_Tech_Note_Surface_Albedos:equation-3.34}}\begin{equation}\label{equation:tech_note/Surface_Albedos/CLM50_Tech_Note_Surface_Albedos:3.34}
\begin{split}f=\omega _{\Lambda } \bar{\mu }K\left(1-\beta _{0,\, \Lambda } \right)\end{split}
\end{equation}\phantomsection\label{\detokenize{tech_note/Surface_Albedos/CLM50_Tech_Note_Surface_Albedos:equation-3.35}}\begin{equation}\label{equation:tech_note/Surface_Albedos/CLM50_Tech_Note_Surface_Albedos:3.35}
\begin{split}h=\frac{\sqrt{b^{2} -c^{2} } }{\bar{\mu }}\end{split}
\end{equation}\phantomsection\label{\detokenize{tech_note/Surface_Albedos/CLM50_Tech_Note_Surface_Albedos:equation-3.36}}\begin{equation}\label{equation:tech_note/Surface_Albedos/CLM50_Tech_Note_Surface_Albedos:3.36}
\begin{split}\sigma =\left(\bar{\mu }K\right)^{2} +c^{2} -b^{2}\end{split}
\end{equation}\phantomsection\label{\detokenize{tech_note/Surface_Albedos/CLM50_Tech_Note_Surface_Albedos:equation-3.37}}\begin{equation}\label{equation:tech_note/Surface_Albedos/CLM50_Tech_Note_Surface_Albedos:3.37}
\begin{split}u_{1} =b-{c\mathord{\left/ {\vphantom {c \alpha _{g,\, \Lambda }^{\mu } }} \right. \kern-\nulldelimiterspace} \alpha _{g,\, \Lambda }^{\mu } } {\rm \; or\; }u_{1} =b-{c\mathord{\left/ {\vphantom {c \alpha _{g,\, \Lambda } }} \right. \kern-\nulldelimiterspace} \alpha _{g,\, \Lambda } }\end{split}
\end{equation}\phantomsection\label{\detokenize{tech_note/Surface_Albedos/CLM50_Tech_Note_Surface_Albedos:equation-3.38}}\begin{equation}\label{equation:tech_note/Surface_Albedos/CLM50_Tech_Note_Surface_Albedos:3.38}
\begin{split}u_{2} =b-c\alpha _{g,\, \Lambda }^{\mu } {\rm \; or\; }u_{2} =b-c\alpha _{g,\, \Lambda }\end{split}
\end{equation}\phantomsection\label{\detokenize{tech_note/Surface_Albedos/CLM50_Tech_Note_Surface_Albedos:equation-3.39}}\begin{equation}\label{equation:tech_note/Surface_Albedos/CLM50_Tech_Note_Surface_Albedos:3.39}
\begin{split}u_{3} =f+c\alpha _{g,\, \Lambda }^{\mu } {\rm \; or\; }u_{3} =f+c\alpha _{g,\, \Lambda }\end{split}
\end{equation}\phantomsection\label{\detokenize{tech_note/Surface_Albedos/CLM50_Tech_Note_Surface_Albedos:equation-3.40}}\begin{equation}\label{equation:tech_note/Surface_Albedos/CLM50_Tech_Note_Surface_Albedos:3.40}
\begin{split}s_{1} =\exp \left\{-\min \left[h\left(L+S\right),40\right]\right\}\end{split}
\end{equation}\phantomsection\label{\detokenize{tech_note/Surface_Albedos/CLM50_Tech_Note_Surface_Albedos:equation-3.41}}\begin{equation}\label{equation:tech_note/Surface_Albedos/CLM50_Tech_Note_Surface_Albedos:3.41}
\begin{split}s_{2} =\exp \left\{-\min \left[K\left(L+S\right),40\right]\right\}\end{split}
\end{equation}\phantomsection\label{\detokenize{tech_note/Surface_Albedos/CLM50_Tech_Note_Surface_Albedos:equation-3.42}}\begin{equation}\label{equation:tech_note/Surface_Albedos/CLM50_Tech_Note_Surface_Albedos:3.42}
\begin{split}p_{1} =b+\bar{\mu }h\end{split}
\end{equation}\phantomsection\label{\detokenize{tech_note/Surface_Albedos/CLM50_Tech_Note_Surface_Albedos:equation-3.43}}\begin{equation}\label{equation:tech_note/Surface_Albedos/CLM50_Tech_Note_Surface_Albedos:3.43}
\begin{split}p_{2} =b-\bar{\mu }h\end{split}
\end{equation}\phantomsection\label{\detokenize{tech_note/Surface_Albedos/CLM50_Tech_Note_Surface_Albedos:equation-3.44}}\begin{equation}\label{equation:tech_note/Surface_Albedos/CLM50_Tech_Note_Surface_Albedos:3.44}
\begin{split}p_{3} =b+\bar{\mu }K\end{split}
\end{equation}\phantomsection\label{\detokenize{tech_note/Surface_Albedos/CLM50_Tech_Note_Surface_Albedos:equation-3.45}}\begin{equation}\label{equation:tech_note/Surface_Albedos/CLM50_Tech_Note_Surface_Albedos:3.45}
\begin{split}p_{4} =b-\bar{\mu }K\end{split}
\end{equation}\phantomsection\label{\detokenize{tech_note/Surface_Albedos/CLM50_Tech_Note_Surface_Albedos:equation-3.46}}\begin{equation}\label{equation:tech_note/Surface_Albedos/CLM50_Tech_Note_Surface_Albedos:3.46}
\begin{split}d_{1} =\frac{p_{1} \left(u_{1} -\bar{\mu }h\right)}{s_{1} } -p_{2} \left(u_{1} +\bar{\mu }h\right)s_{1}\end{split}
\end{equation}\phantomsection\label{\detokenize{tech_note/Surface_Albedos/CLM50_Tech_Note_Surface_Albedos:equation-3.47}}\begin{equation}\label{equation:tech_note/Surface_Albedos/CLM50_Tech_Note_Surface_Albedos:3.47}
\begin{split}d_{2} =\frac{u_{2} +\bar{\mu }h}{s_{1} } -\left(u_{2} -\bar{\mu }h\right)s_{1}\end{split}
\end{equation}\phantomsection\label{\detokenize{tech_note/Surface_Albedos/CLM50_Tech_Note_Surface_Albedos:equation-3.48}}\begin{equation}\label{equation:tech_note/Surface_Albedos/CLM50_Tech_Note_Surface_Albedos:3.48}
\begin{split}h_{1} =-dp_{4} -cf\end{split}
\end{equation}\phantomsection\label{\detokenize{tech_note/Surface_Albedos/CLM50_Tech_Note_Surface_Albedos:equation-3.49}}\begin{equation}\label{equation:tech_note/Surface_Albedos/CLM50_Tech_Note_Surface_Albedos:3.49}
\begin{split}h_{2} =\frac{1}{d_{1} } \left[\left(d-\frac{h_{1} }{\sigma } p_{3} \right)\frac{\left(u_{1} -\bar{\mu }h\right)}{s_{1} } -p_{2} \left(d-c-\frac{h_{1} }{\sigma } \left(u_{1} +\bar{\mu }K\right)\right)s_{2} \right]\end{split}
\end{equation}\phantomsection\label{\detokenize{tech_note/Surface_Albedos/CLM50_Tech_Note_Surface_Albedos:equation-3.50}}\begin{equation}\label{equation:tech_note/Surface_Albedos/CLM50_Tech_Note_Surface_Albedos:3.50}
\begin{split}h_{3} =\frac{-1}{d_{1} } \left[\left(d-\frac{h_{1} }{\sigma } p_{3} \right)\left(u_{1} +\bar{\mu }h\right)s_{1} -p_{1} \left(d-c-\frac{h_{1} }{\sigma } \left(u_{1} +\bar{\mu }K\right)\right)s_{2} \right]\end{split}
\end{equation}\phantomsection\label{\detokenize{tech_note/Surface_Albedos/CLM50_Tech_Note_Surface_Albedos:equation-3.51}}\begin{equation}\label{equation:tech_note/Surface_Albedos/CLM50_Tech_Note_Surface_Albedos:3.51}
\begin{split}h_{4} =-fp_{3} -cd\end{split}
\end{equation}\phantomsection\label{\detokenize{tech_note/Surface_Albedos/CLM50_Tech_Note_Surface_Albedos:equation-3.52}}\begin{equation}\label{equation:tech_note/Surface_Albedos/CLM50_Tech_Note_Surface_Albedos:3.52}
\begin{split}h_{5} =\frac{-1}{d_{2} } \left[\left(\frac{h_{4} \left(u_{2} +\bar{\mu }h\right)}{\sigma s_{1} } \right)+\left(u_{3} -\frac{h_{4} }{\sigma } \left(u_{2} -\bar{\mu }K\right)\right)s_{2} \right]\end{split}
\end{equation}\phantomsection\label{\detokenize{tech_note/Surface_Albedos/CLM50_Tech_Note_Surface_Albedos:equation-3.53}}\begin{equation}\label{equation:tech_note/Surface_Albedos/CLM50_Tech_Note_Surface_Albedos:3.53}
\begin{split}h_{6} =\frac{1}{d_{2} } \left[\frac{h_{4} }{\sigma } \left(u_{2} -\bar{\mu }h\right)s_{1} +\left(u_{3} -\frac{h_{4} }{\sigma } \left(u_{2} -\bar{\mu }K\right)\right)s_{2} \right]\end{split}
\end{equation}\phantomsection\label{\detokenize{tech_note/Surface_Albedos/CLM50_Tech_Note_Surface_Albedos:equation-3.54}}\begin{equation}\label{equation:tech_note/Surface_Albedos/CLM50_Tech_Note_Surface_Albedos:3.54}
\begin{split}h_{7} =\frac{c\left(u_{1} -\bar{\mu }h\right)}{d_{1} s_{1} }\end{split}
\end{equation}\phantomsection\label{\detokenize{tech_note/Surface_Albedos/CLM50_Tech_Note_Surface_Albedos:equation-3.55}}\begin{equation}\label{equation:tech_note/Surface_Albedos/CLM50_Tech_Note_Surface_Albedos:3.55}
\begin{split}h_{8} =\frac{-c\left(u_{1} +\bar{\mu }h\right)s_{1} }{d_{1} }\end{split}
\end{equation}\phantomsection\label{\detokenize{tech_note/Surface_Albedos/CLM50_Tech_Note_Surface_Albedos:equation-3.56}}\begin{equation}\label{equation:tech_note/Surface_Albedos/CLM50_Tech_Note_Surface_Albedos:3.56}
\begin{split}h_{9} =\frac{u_{2} +\bar{\mu }h}{d_{2} s_{1} }\end{split}
\end{equation}\phantomsection\label{\detokenize{tech_note/Surface_Albedos/CLM50_Tech_Note_Surface_Albedos:equation-3.57}}\begin{equation}\label{equation:tech_note/Surface_Albedos/CLM50_Tech_Note_Surface_Albedos:3.57}
\begin{split}h_{10} =\frac{-s_{1} \left(u_{2} -\bar{\mu }h\right)}{d_{2} } .\end{split}
\end{equation}
Plant functional type optical properties (\hyperref[\detokenize{tech_note/Surface_Albedos/CLM50_Tech_Note_Surface_Albedos:table-plant-functional-type-optical-properties}]{Table \ref{\detokenize{tech_note/Surface_Albedos/CLM50_Tech_Note_Surface_Albedos:table-plant-functional-type-optical-properties}}})
for trees and shrubs are from {\hyperref[\detokenize{tech_note/References/CLM50_Tech_Note_References:dormansellers1989}]{\sphinxcrossref{\DUrole{std,std-ref}{Dorman and Sellers (1989)}}}}. Leaf and stem optical
properties (VIS and NIR reflectance and transmittance) were derived
for grasslands and crops from full optical range spectra of measured
optical properties ({\hyperref[\detokenize{tech_note/References/CLM50_Tech_Note_References:asneretal1998}]{\sphinxcrossref{\DUrole{std,std-ref}{Asner et al. 1998}}}}). Optical properties for
intercepted snow (\hyperref[\detokenize{tech_note/Surface_Albedos/CLM50_Tech_Note_Surface_Albedos:table-intercepted-snow-optical-properties}]{Table \ref{\detokenize{tech_note/Surface_Albedos/CLM50_Tech_Note_Surface_Albedos:table-intercepted-snow-optical-properties}}}) are from
{\hyperref[\detokenize{tech_note/References/CLM50_Tech_Note_References:sellersetal1986}]{\sphinxcrossref{\DUrole{std,std-ref}{Sellers et al. (1986)}}}}.


\begin{savenotes}\sphinxattablestart
\centering
\sphinxcapstartof{table}
\sphinxcaption{Plant functional type optical properties}\label{\detokenize{tech_note/Surface_Albedos/CLM50_Tech_Note_Surface_Albedos:table-plant-functional-type-optical-properties}}\label{\detokenize{tech_note/Surface_Albedos/CLM50_Tech_Note_Surface_Albedos:id7}}
\sphinxaftercaption
\begin{tabulary}{\linewidth}[t]{|T|T|T|T|T|T|T|T|T|T|}
\hline
\sphinxstylethead{\sphinxstyletheadfamily 
Plant Functional Type
\unskip}\relax &
\(\chi _{L}\)
&
\(\alpha _{vis}^{leaf}\)
&
\(\alpha _{nir}^{leaf}\)
&
\(\alpha _{vis}^{stem}\)
&
\(\alpha _{nir}^{stem}\)
&
\(\tau _{vis}^{leaf}\)
&
\(\tau _{nir}^{leaf}\)
&
\(\tau _{vis}^{stem}\)
&
\(\tau _{nir}^{stem}\)
\\
\hline
NET Temperate
&
0.01
&
0.07
&
0.35
&
0.16
&
0.39
&
0.05
&
0.10
&
0.001
&
0.001
\\
\hline
NET Boreal
&
0.01
&
0.07
&
0.35
&
0.16
&
0.39
&
0.05
&
0.10
&
0.001
&
0.001
\\
\hline
NDT Boreal
&
0.01
&
0.07
&
0.35
&
0.16
&
0.39
&
0.05
&
0.10
&
0.001
&
0.001
\\
\hline
BET Tropical
&
0.10
&
0.10
&
0.45
&
0.16
&
0.39
&
0.05
&
0.25
&
0.001
&
0.001
\\
\hline
BET temperate
&
0.10
&
0.10
&
0.45
&
0.16
&
0.39
&
0.05
&
0.25
&
0.001
&
0.001
\\
\hline
BDT tropical
&
0.01
&
0.10
&
0.45
&
0.16
&
0.39
&
0.05
&
0.25
&
0.001
&
0.001
\\
\hline
BDT temperate
&
0.25
&
0.10
&
0.45
&
0.16
&
0.39
&
0.05
&
0.25
&
0.001
&
0.001
\\
\hline
BDT boreal
&
0.25
&
0.10
&
0.45
&
0.16
&
0.39
&
0.05
&
0.25
&
0.001
&
0.001
\\
\hline
BES temperate
&
0.01
&
0.07
&
0.35
&
0.16
&
0.39
&
0.05
&
0.10
&
0.001
&
0.001
\\
\hline
BDS temperate
&
0.25
&
0.10
&
0.45
&
0.16
&
0.39
&
0.05
&
0.25
&
0.001
&
0.001
\\
\hline
BDS boreal
&
0.25
&
0.10
&
0.45
&
0.16
&
0.39
&
0.05
&
0.25
&
0.001
&
0.001
\\
\hline
C$_{\text{3}}$ arctic grass
&
-0.30
&
0.11
&
0.35
&
0.31
&
0.53
&
0.05
&
0.34
&
0.120
&
0.250
\\
\hline
C$_{\text{3}}$ grass
&
-0.30
&
0.11
&
0.35
&
0.31
&
0.53
&
0.05
&
0.34
&
0.120
&
0.250
\\
\hline
C$_{\text{4}}$ grass
&
-0.30
&
0.11
&
0.35
&
0.31
&
0.53
&
0.05
&
0.34
&
0.120
&
0.250
\\
\hline
C$_{\text{3}}$ Crop
&
-0.30
&
0.11
&
0.35
&
0.31
&
0.53
&
0.05
&
0.34
&
0.120
&
0.250
\\
\hline
Temp Corn
&
-0.50
&
0.11
&
0.35
&
0.31
&
0.53
&
0.05
&
0.34
&
0.120
&
0.250
\\
\hline
Spring Wheat
&
-0.50
&
0.11
&
0.35
&
0.31
&
0.53
&
0.05
&
0.34
&
0.120
&
0.250
\\
\hline
Temp Soybean
&
-0.50
&
0.11
&
0.35
&
0.31
&
0.53
&
0.05
&
0.34
&
0.120
&
0.250
\\
\hline
Cotton
&
-0.50
&
0.11
&
0.35
&
0.31
&
0.53
&
0.05
&
0.34
&
0.120
&
0.250
\\
\hline
Rice
&
-0.50
&
0.11
&
0.35
&
0.31
&
0.53
&
0.05
&
0.34
&
0.120
&
0.250
\\
\hline
Sugarcane
&
-0.50
&
0.11
&
0.35
&
0.31
&
0.53
&
0.05
&
0.34
&
0.120
&
0.250
\\
\hline
Tropical Corn
&
-0.50
&
0.11
&
0.35
&
0.31
&
0.53
&
0.05
&
0.34
&
0.120
&
0.250
\\
\hline
Tropical Soybean
&
-0.50
&
0.11
&
0.35
&
0.31
&
0.53
&
0.05
&
0.34
&
0.120
&
0.250
\\
\hline
\end{tabulary}
\par
\sphinxattableend\end{savenotes}


\begin{savenotes}\sphinxattablestart
\centering
\sphinxcapstartof{table}
\sphinxcaption{Intercepted snow optical properties}\label{\detokenize{tech_note/Surface_Albedos/CLM50_Tech_Note_Surface_Albedos:table-intercepted-snow-optical-properties}}\label{\detokenize{tech_note/Surface_Albedos/CLM50_Tech_Note_Surface_Albedos:id8}}
\sphinxaftercaption
\begin{tabulary}{\linewidth}[t]{|T|T|T|}
\hline
\sphinxstylethead{\sphinxstyletheadfamily 
Parameter
\unskip}\relax &\sphinxstylethead{\sphinxstyletheadfamily 
vis
\unskip}\relax &\sphinxstylethead{\sphinxstyletheadfamily 
nir
\unskip}\relax \\
\hline
\(\omega ^{sno}\)
&
0.8
&
0.4
\\
\hline
\(\beta ^{sno}\)
&
0.5
&
0.5
\\
\hline
\(\beta _{0}^{sno}\)
&
0.5
&
0.5
\\
\hline
\end{tabulary}
\par
\sphinxattableend\end{savenotes}


\subsection{Ground Albedos}
\label{\detokenize{tech_note/Surface_Albedos/CLM50_Tech_Note_Surface_Albedos:id2}}\label{\detokenize{tech_note/Surface_Albedos/CLM50_Tech_Note_Surface_Albedos:ground-albedos}}
The overall direct beam \(\alpha _{g,\, \Lambda }^{\mu }\)  and diffuse
\(\alpha _{g,\, \Lambda }\)  ground albedos are weighted
combinations of “soil” and snow albedos
\phantomsection\label{\detokenize{tech_note/Surface_Albedos/CLM50_Tech_Note_Surface_Albedos:equation-3.58}}\begin{equation}\label{equation:tech_note/Surface_Albedos/CLM50_Tech_Note_Surface_Albedos:3.58}
\begin{split}\alpha _{g,\, \Lambda }^{\mu } =\alpha _{soi,\, \Lambda }^{\mu } \left(1-f_{sno} \right)+\alpha _{sno,\, \Lambda }^{\mu } f_{sno}\end{split}
\end{equation}\phantomsection\label{\detokenize{tech_note/Surface_Albedos/CLM50_Tech_Note_Surface_Albedos:equation-3.59}}\begin{equation}\label{equation:tech_note/Surface_Albedos/CLM50_Tech_Note_Surface_Albedos:3.59}
\begin{split}\alpha _{g,\, \Lambda } =\alpha _{soi,\, \Lambda } \left(1-f_{sno} \right)+\alpha _{sno,\, \Lambda } f_{sno}\end{split}
\end{equation}
where \(f_{sno}\)  is the fraction of the ground covered with snow
(section \hyperref[\detokenize{tech_note/Snow_Hydrology/CLM50_Tech_Note_Snow_Hydrology:snow-covered-area-fraction}]{\ref{\detokenize{tech_note/Snow_Hydrology/CLM50_Tech_Note_Snow_Hydrology:snow-covered-area-fraction}}}).

\(\alpha _{soi,\, \Lambda }^{\mu }\)  and
\(\alpha _{soi,\, \Lambda }\)  vary with glacier, lake, wetland, and
soil surfaces. Glacier albedos are from {\hyperref[\detokenize{tech_note/References/CLM50_Tech_Note_References:paterson1994}]{\sphinxcrossref{\DUrole{std,std-ref}{Paterson (1994)}}}}
\begin{equation*}
\begin{split}\alpha _{soi,\, vis}^{\mu } =\alpha _{soi,\, vis} =0.6\end{split}
\end{equation*}\begin{equation*}
\begin{split}\alpha _{soi,\, nir}^{\mu } =\alpha _{soi,\, nir} =0.4.\end{split}
\end{equation*}
Unfrozen lake and wetland albedos depend on the cosine of the solar
zenith angle \(\mu\)
\phantomsection\label{\detokenize{tech_note/Surface_Albedos/CLM50_Tech_Note_Surface_Albedos:equation-3.60}}\begin{equation}\label{equation:tech_note/Surface_Albedos/CLM50_Tech_Note_Surface_Albedos:3.60}
\begin{split}\alpha _{soi,\, \Lambda }^{\mu } =\alpha _{soi,\, \Lambda } =0.05\left(\mu +0.15\right)^{-1} .\end{split}
\end{equation}
Frozen lake and wetland albedos are from NCAR LSM ({\hyperref[\detokenize{tech_note/References/CLM50_Tech_Note_References:bonan1996}]{\sphinxcrossref{\DUrole{std,std-ref}{Bonan 1996}}}})
\begin{equation*}
\begin{split}\alpha _{soi,\, vis}^{\mu } =\alpha _{soi,\, vis} =0.60\end{split}
\end{equation*}\begin{equation*}
\begin{split}\alpha _{soi,\, nir}^{\mu } =\alpha _{soi,\, nir} =0.40.\end{split}
\end{equation*}
As in NCAR LSM ({\hyperref[\detokenize{tech_note/References/CLM50_Tech_Note_References:bonan1996}]{\sphinxcrossref{\DUrole{std,std-ref}{Bonan 1996}}}}), soil albedos vary with color class
\phantomsection\label{\detokenize{tech_note/Surface_Albedos/CLM50_Tech_Note_Surface_Albedos:equation-3.61}}\begin{equation}\label{equation:tech_note/Surface_Albedos/CLM50_Tech_Note_Surface_Albedos:3.61}
\begin{split}\alpha _{soi,\, \Lambda }^{\mu } =\alpha _{soi,\, \Lambda } =\left(\alpha _{sat,\, \Lambda } +\Delta \right)\le \alpha _{dry,\, \Lambda }\end{split}
\end{equation}
where \(\Delta\)  depends on the volumetric water content of the
first soil layer \(\theta _{1}\)  (section \hyperref[\detokenize{tech_note/Hydrology/CLM50_Tech_Note_Hydrology:soil-water}]{\ref{\detokenize{tech_note/Hydrology/CLM50_Tech_Note_Hydrology:soil-water}}}) as
\(\Delta =0.11-0.40\theta _{1} >0\), and
\(\alpha _{sat,\, \Lambda }\)  and
\(\alpha _{dry,\, \Lambda }\)  are albedos for saturated and dry
soil color classes (\hyperref[\detokenize{tech_note/Surface_Albedos/CLM50_Tech_Note_Surface_Albedos:table-dry-and-saturated-soil-albedos}]{Table \ref{\detokenize{tech_note/Surface_Albedos/CLM50_Tech_Note_Surface_Albedos:table-dry-and-saturated-soil-albedos}}}).

CLM soil colors are prescribed so that they best reproduce observed
MODIS local solar noon surface albedo values at the CLM grid cell
following the methods of {\hyperref[\detokenize{tech_note/References/CLM50_Tech_Note_References:lawrencechase2007}]{\sphinxcrossref{\DUrole{std,std-ref}{Lawrence and Chase (2007)}}}}.
The soil colors are fitted over the range of 20 soil classes shown in
\hyperref[\detokenize{tech_note/Surface_Albedos/CLM50_Tech_Note_Surface_Albedos:table-dry-and-saturated-soil-albedos}]{Table \ref{\detokenize{tech_note/Surface_Albedos/CLM50_Tech_Note_Surface_Albedos:table-dry-and-saturated-soil-albedos}}} and compared
to the MODIS monthly local solar noon all-sky surface albedo as
described in {\hyperref[\detokenize{tech_note/References/CLM50_Tech_Note_References:strahleretal1999}]{\sphinxcrossref{\DUrole{std,std-ref}{Strahler et al. (1999)}}}} and
{\hyperref[\detokenize{tech_note/References/CLM50_Tech_Note_References:schaafetal2002}]{\sphinxcrossref{\DUrole{std,std-ref}{Schaaf et al. (2002)}}}}. The CLM
two-stream radiation model was used to calculate the model equivalent
surface albedo using climatological monthly soil moisture along with the
vegetation parameters of PFT fraction, LAI, and SAI. The soil color that
produced the closest all-sky albedo in the two-stream radiation model
was selected as the best fit for the month. The fitted monthly soil
colors were averaged over all snow-free months to specify a
representative soil color for the grid cell. In cases where there was no
snow-free surface albedo for the year, the soil color derived from
snow-affected albedo was used to give a representative soil color that
included the effects of the minimum permanent snow cover.


\begin{savenotes}\sphinxattablestart
\centering
\sphinxcapstartof{table}
\sphinxcaption{Dry and saturated soil albedos}\label{\detokenize{tech_note/Surface_Albedos/CLM50_Tech_Note_Surface_Albedos:table-dry-and-saturated-soil-albedos}}\label{\detokenize{tech_note/Surface_Albedos/CLM50_Tech_Note_Surface_Albedos:id9}}
\sphinxaftercaption
\begin{tabulary}{\linewidth}[t]{|T|T|T|T|T|T|T|T|T|T|}
\hline
&\sphinxstartmulticolumn{2}%
\begin{varwidth}[t]{\sphinxcolwidth{2}{10}}
Dry
\par
\vskip-\baselineskip\strut\end{varwidth}%
\sphinxstopmulticolumn
&\sphinxstartmulticolumn{2}%
\begin{varwidth}[t]{\sphinxcolwidth{2}{10}}
Saturated
\par
\vskip-\baselineskip\strut\end{varwidth}%
\sphinxstopmulticolumn
&&\sphinxstartmulticolumn{2}%
\begin{varwidth}[t]{\sphinxcolwidth{2}{10}}
Dry
\par
\vskip-\baselineskip\strut\end{varwidth}%
\sphinxstopmulticolumn
&\sphinxstartmulticolumn{2}%
\begin{varwidth}[t]{\sphinxcolwidth{2}{10}}
Saturated
\par
\vskip-\baselineskip\strut\end{varwidth}%
\sphinxstopmulticolumn
\\
\hline
Color Class
&
vis
&
nir
&
vis
&
nir
&
Color Class
&
vis
&
nir
&
vis
&
nir
\\
\hline
1
&
0.36
&
0.61
&
0.25
&
0.50
&
11
&
0.24
&
0.37
&
0.13
&
0.26
\\
\hline
2
&
0.34
&
0.57
&
0.23
&
0.46
&
12
&
0.23
&
0.35
&
0.12
&
0.24
\\
\hline
3
&
0.32
&
0.53
&
0.21
&
0.42
&
13
&
0.22
&
0.33
&
0.11
&
0.22
\\
\hline
4
&
0.31
&
0.51
&
0.20
&
0.40
&
14
&
0.20
&
0.31
&
0.10
&
0.20
\\
\hline
5
&
0.30
&
0.49
&
0.19
&
0.38
&
15
&
0.18
&
0.29
&
0.09
&
0.18
\\
\hline
6
&
0.29
&
0.48
&
0.18
&
0.36
&
16
&
0.16
&
0.27
&
0.08
&
0.16
\\
\hline
7
&
0.28
&
0.45
&
0.17
&
0.34
&
17
&
0.14
&
0.25
&
0.07
&
0.14
\\
\hline
8
&
0.27
&
0.43
&
0.16
&
0.32
&
18
&
0.12
&
0.23
&
0.06
&
0.12
\\
\hline
9
&
0.26
&
0.41
&
0.15
&
0.30
&
19
&
0.10
&
0.21
&
0.05
&
0.10
\\
\hline
10
&
0.25
&
0.39
&
0.14
&
0.28
&
20
&
0.08
&
0.16
&
0.04
&
0.08
\\
\hline
\end{tabulary}
\par
\sphinxattableend\end{savenotes}


\subsubsection{Snow Albedo}
\label{\detokenize{tech_note/Surface_Albedos/CLM50_Tech_Note_Surface_Albedos:snow-albedo}}\label{\detokenize{tech_note/Surface_Albedos/CLM50_Tech_Note_Surface_Albedos:id3}}
Snow albedo and solar absorption within each snow layer are simulated
with the Snow, Ice, and Aerosol Radiative Model (SNICAR), which
incorporates a two-stream radiative transfer solution from
{\hyperref[\detokenize{tech_note/References/CLM50_Tech_Note_References:toonetal1989}]{\sphinxcrossref{\DUrole{std,std-ref}{Toon et al. (1989)}}}}. Albedo and the vertical absorption
profile depend on solar zenith angle, albedo of the substrate underlying snow, mass
concentrations of atmospheric-deposited aerosols (black carbon, mineral
dust, and organic carbon), and ice effective grain size
(\(r_{e}\)), which is simulated with a snow aging routine
described in section \hyperref[\detokenize{tech_note/Surface_Albedos/CLM50_Tech_Note_Surface_Albedos:snow-aging}]{\ref{\detokenize{tech_note/Surface_Albedos/CLM50_Tech_Note_Surface_Albedos:snow-aging}}}. Representation of impurity mass
concentrations within the snowpack is described in section
\hyperref[\detokenize{tech_note/Snow_Hydrology/CLM50_Tech_Note_Snow_Hydrology:black-and-organic-carbon-and-mineral-dust-within-snow}]{\ref{\detokenize{tech_note/Snow_Hydrology/CLM50_Tech_Note_Snow_Hydrology:black-and-organic-carbon-and-mineral-dust-within-snow}}}.
Implementation of SNICAR in CLM is also described somewhat by
{\hyperref[\detokenize{tech_note/References/CLM50_Tech_Note_References:flannerzender2005}]{\sphinxcrossref{\DUrole{std,std-ref}{Flanner and Zender (2005)}}}} and
{\hyperref[\detokenize{tech_note/References/CLM50_Tech_Note_References:flanneretal2007}]{\sphinxcrossref{\DUrole{std,std-ref}{Flanner et al. (2007)}}}}.

The two-stream solution requires the following bulk optical properties
for each snow layer and spectral band: extinction optical depth
(\(\tau\)), single-scatter albedo (\(\omega\)), and
scattering asymmetry parameter (\sphinxstyleemphasis{g}). The snow layers used for radiative
calculations are identical to snow layers applied elsewhere in CLM,
except for the case when snow mass is greater than zero but no snow
layers exist. When this occurs, a single radiative layer is specified to
have the column snow mass and an effective grain size of freshly-fallen
snow (section \hyperref[\detokenize{tech_note/Surface_Albedos/CLM50_Tech_Note_Surface_Albedos:snow-aging}]{\ref{\detokenize{tech_note/Surface_Albedos/CLM50_Tech_Note_Surface_Albedos:snow-aging}}}). The bulk optical properties are weighted functions
of each constituent \sphinxstyleemphasis{k}, computed for each snow layer and spectral band
as
\phantomsection\label{\detokenize{tech_note/Surface_Albedos/CLM50_Tech_Note_Surface_Albedos:equation-3.62}}\begin{equation}\label{equation:tech_note/Surface_Albedos/CLM50_Tech_Note_Surface_Albedos:3.62}
\begin{split}\tau =\sum _{1}^{k}\tau _{k}\end{split}
\end{equation}\phantomsection\label{\detokenize{tech_note/Surface_Albedos/CLM50_Tech_Note_Surface_Albedos:equation-3.63}}\begin{equation}\label{equation:tech_note/Surface_Albedos/CLM50_Tech_Note_Surface_Albedos:3.63}
\begin{split}\omega =\frac{\sum _{1}^{k}\omega _{k} \tau _{k}  }{\sum _{1}^{k}\tau _{k}  }\end{split}
\end{equation}\phantomsection\label{\detokenize{tech_note/Surface_Albedos/CLM50_Tech_Note_Surface_Albedos:equation-3.64}}\begin{equation}\label{equation:tech_note/Surface_Albedos/CLM50_Tech_Note_Surface_Albedos:3.64}
\begin{split}g=\frac{\sum _{1}^{k}g_{k} \omega _{k} \tau _{k}  }{\sum _{1}^{k}\omega _{k} \tau _{k}  }\end{split}
\end{equation}
For each constituent (ice, two black carbon species, two organic carbon species, and
four dust species), \(\omega\), \sphinxstyleemphasis{g}, and the mass extinction cross-section
\(\psi\) (m$^{\text{2}}$ kg$_{\text{-1}}$) are computed offline with Mie Theory, e.g.,
applying the computational technique from {\hyperref[\detokenize{tech_note/References/CLM50_Tech_Note_References:bohrenhuffman1983}]{\sphinxcrossref{\DUrole{std,std-ref}{Bohren and Huffman (1983)}}}}.
The extinction optical depth for each constituent depends on its mass  extinction
cross-section and layer mass, \(w _{k}\) (kgm$^{\text{-1}}$) as
\phantomsection\label{\detokenize{tech_note/Surface_Albedos/CLM50_Tech_Note_Surface_Albedos:equation-3.65}}\begin{equation}\label{equation:tech_note/Surface_Albedos/CLM50_Tech_Note_Surface_Albedos:3.65}
\begin{split}\tau _{k} =\psi _{k} w_{k}\end{split}
\end{equation}
The two-stream solution ({\hyperref[\detokenize{tech_note/References/CLM50_Tech_Note_References:toonetal1989}]{\sphinxcrossref{\DUrole{std,std-ref}{Toon et al. (1989)}}}}) applies a tri-diagonal matrix
solution to produce upward and downward radiative fluxes at each layer
interface, from which net radiation, layer absorption, and surface
albedo are easily derived. Solar fluxes are computed in five spectral
bands, listed in \hyperref[\detokenize{tech_note/Surface_Albedos/CLM50_Tech_Note_Surface_Albedos:table-spectral-bands-and-weights-used-for-snow-radiative-transfer}]{Table \ref{\detokenize{tech_note/Surface_Albedos/CLM50_Tech_Note_Surface_Albedos:table-spectral-bands-and-weights-used-for-snow-radiative-transfer}}}.
Because snow albedo varies strongly across
the solar spectrum, it was determined that four bands were needed to
accurately represent the near-infrared (NIR) characteristics of snow,
whereas only one band was needed for the visible spectrum. Boundaries of
the NIR bands were selected to capture broad radiative features and
maximize accuracy and computational efficiency. We partition NIR (0.7-5.0
\(\mu\) m) surface downwelling flux from CLM according to the weights listed
in \hyperref[\detokenize{tech_note/Surface_Albedos/CLM50_Tech_Note_Surface_Albedos:table-spectral-bands-and-weights-used-for-snow-radiative-transfer}]{Table \ref{\detokenize{tech_note/Surface_Albedos/CLM50_Tech_Note_Surface_Albedos:table-spectral-bands-and-weights-used-for-snow-radiative-transfer}}},
which are unique for diffuse and direct incident flux. These fixed weights were
determined with offline hyperspectral radiative transfer calculations for an
atmosphere typical of mid-latitude winter ({\hyperref[\detokenize{tech_note/References/CLM50_Tech_Note_References:flanneretal2007}]{\sphinxcrossref{\DUrole{std,std-ref}{Flanner et al. (2007)}}}}).
The tri-diagonal solution includes intermediate terms that allow for easy
interchange of two-stream techniques. We apply the Eddington solution
for the visible band (following {\hyperref[\detokenize{tech_note/References/CLM50_Tech_Note_References:wiscombewarren1980}]{\sphinxcrossref{\DUrole{std,std-ref}{Wiscombe and Warren 1980}}}}) and the
hemispheric mean solution (({\hyperref[\detokenize{tech_note/References/CLM50_Tech_Note_References:toonetal1989}]{\sphinxcrossref{\DUrole{std,std-ref}{Toon et al. (1989)}}}}) for NIR bands. These
choices were made because the Eddington scheme works well for highly
scattering media, but can produce negative albedo for absorptive NIR
bands with diffuse incident flux. Delta scalings are applied to
\(\tau\), \(\omega\), and \(g\) ({\hyperref[\detokenize{tech_note/References/CLM50_Tech_Note_References:wiscombewarren1980}]{\sphinxcrossref{\DUrole{std,std-ref}{Wiscombe and Warren 1980}}}}) in
all spectral bands, producing effective values (denoted with *) that
are applied in the two-stream solution
\phantomsection\label{\detokenize{tech_note/Surface_Albedos/CLM50_Tech_Note_Surface_Albedos:equation-3.66}}\begin{equation}\label{equation:tech_note/Surface_Albedos/CLM50_Tech_Note_Surface_Albedos:3.66}
\begin{split}\tau ^{*} =\left(1-\omega g^{2} \right)\tau\end{split}
\end{equation}\phantomsection\label{\detokenize{tech_note/Surface_Albedos/CLM50_Tech_Note_Surface_Albedos:equation-3.67}}\begin{equation}\label{equation:tech_note/Surface_Albedos/CLM50_Tech_Note_Surface_Albedos:3.67}
\begin{split}\omega ^{*} =\frac{\left(1-g^{2} \right)\omega }{1-g^{2} \omega }\end{split}
\end{equation}\phantomsection\label{\detokenize{tech_note/Surface_Albedos/CLM50_Tech_Note_Surface_Albedos:equation-3.68}}\begin{equation}\label{equation:tech_note/Surface_Albedos/CLM50_Tech_Note_Surface_Albedos:3.68}
\begin{split}g^{*} =\frac{g}{1+g}\end{split}
\end{equation}

\begin{savenotes}\sphinxattablestart
\centering
\sphinxcapstartof{table}
\sphinxcaption{Spectral bands and weights used for snow radiative transfer}\label{\detokenize{tech_note/Surface_Albedos/CLM50_Tech_Note_Surface_Albedos:table-spectral-bands-and-weights-used-for-snow-radiative-transfer}}\label{\detokenize{tech_note/Surface_Albedos/CLM50_Tech_Note_Surface_Albedos:id10}}
\sphinxaftercaption
\begin{tabulary}{\linewidth}[t]{|T|T|T|}
\hline
\sphinxstylethead{\sphinxstyletheadfamily 
Spectral band
\unskip}\relax &\sphinxstylethead{\sphinxstyletheadfamily 
Direct-beam weight
\unskip}\relax &\sphinxstylethead{\sphinxstyletheadfamily 
Diffuse weight
\unskip}\relax \\
\hline
Band 1: 0.3-0.7\(\mu\)m (visible)
&
(1.0)
&
(1.0)
\\
\hline
Band 2: 0.7-1.0\(\mu\)m (near-IR)
&
0.494
&
0.586
\\
\hline
Band 3: 1.0-1.2\(\mu\)m (near-IR)
&
0.181
&
0.202
\\
\hline
Band 4: 1.2-1.5\(\mu\)m (near-IR)
&
0.121
&
0.109
\\
\hline
Band 5: 1.5-5.0\(\mu\)m (near-IR)
&
0.204
&
0.103
\\
\hline
\end{tabulary}
\par
\sphinxattableend\end{savenotes}

Under direct-beam conditions, singularities in the radiative
approximation are occasionally approached in spectral bands 4 and 5 that
produce unrealistic conditions (negative energy absorption in a layer,
negative albedo, or total absorbed flux greater than incident flux).
When any of these three conditions occur, the Eddington approximation is
attempted instead, and if both approximations fail, the cosine of the
solar zenith angle is adjusted by 0.02 (conserving incident flux) and a
warning message is produced. This situation occurs in only about 1 in
10 $^{\text{6}}$ computations of snow albedo. After looping over the
five spectral bands, absorption fluxes and albedo are averaged back into
the bulk NIR band used by the rest of CLM.

Soil albedo (or underlying substrate albedo), which is defined for
visible and NIR bands, is a required boundary condition for the snow
radiative transfer calculation. Currently, the bulk NIR soil albedo is
applied to all four NIR snow bands. With ground albedo as a lower
boundary condition, SNICAR simulates solar absorption in all snow layers
as well as the underlying soil or ground. With a thin snowpack,
penetrating solar radiation to the underlying soil can be quite large
and heat cannot be released from the soil to the atmosphere in this
situation. Thus, if the snowpack has total snow depth less than 0.1 m
(\(z_{sno} < 0.1\)) and there are no explicit snow layers, the solar
radiation is absorbed by the top soil layer.  If there is a single snow layer,
the solar radiation is absorbed in that layer.  If there is more than a single
snow layer, 75\% of the solar radiation is absorbed in the top snow layer,
and 25\% is absorbed in the next lowest snow layer. This prevents unrealistic
soil warming within a single timestep.

The radiative transfer calculation is performed twice for each column
containing a mass of snow greater than
\(1 \times 10^{-30}\) kgm$^{\text{-2}}$ (excluding lake and urban columns); once each for
direct-beam and diffuse incident flux. Absorption in each layer
\(i\) of pure snow is initially recorded as absorbed flux per unit
incident flux on the ground (\(S_{sno,\, i}\) ), as albedos must be
calculated for the next timestep with unknown incident flux. The snow
absorption fluxes that are used for column temperature calculations are
\phantomsection\label{\detokenize{tech_note/Surface_Albedos/CLM50_Tech_Note_Surface_Albedos:equation-ZEqnNum275338}}\begin{equation}\label{equation:tech_note/Surface_Albedos/CLM50_Tech_Note_Surface_Albedos:ZEqnNum275338}
\begin{split}S_{g,\, i} =S_{sno,\, i} \left(1-\alpha _{sno} \right)\end{split}
\end{equation}
This weighting is performed for direct-beam and diffuse, visible and NIR
fluxes. After the ground-incident fluxes (transmitted through the
vegetation canopy) have been calculated for the current time step
(sections \hyperref[\detokenize{tech_note/Surface_Albedos/CLM50_Tech_Note_Surface_Albedos:canopy-radiative-transfer}]{\ref{\detokenize{tech_note/Surface_Albedos/CLM50_Tech_Note_Surface_Albedos:canopy-radiative-transfer}}} and \hyperref[\detokenize{tech_note/Radiative_Fluxes/CLM50_Tech_Note_Radiative_Fluxes:solar-fluxes}]{\ref{\detokenize{tech_note/Radiative_Fluxes/CLM50_Tech_Note_Radiative_Fluxes:solar-fluxes}}}),
the layer absorption factors

(\(S_{g,\, i}\)) are multiplied by the ground-incident fluxes to
produce solar absorption (W m$^{\text{-2}}$) in each snow layer and
the underlying ground.


\subsubsection{Snowpack Optical Properties}
\label{\detokenize{tech_note/Surface_Albedos/CLM50_Tech_Note_Surface_Albedos:snowpack-optical-properties}}\label{\detokenize{tech_note/Surface_Albedos/CLM50_Tech_Note_Surface_Albedos:id4}}
Ice optical properties for the five spectral bands are derived offline
and stored in a namelist-defined lookup table for online retrieval (see
CLM5.0 User’s Guide). Mie properties are first computed at fine spectral
resolution (470 bands), and are then weighted into the five bands
applied by CLM according to incident solar flux,
\(I^{\downarrow } (\lambda )\). For example, the broadband
mass-extinction cross section (\(\bar{\psi }\)) over wavelength
interval \(\lambda _{1}\) to \(\lambda _{2}\) is
\phantomsection\label{\detokenize{tech_note/Surface_Albedos/CLM50_Tech_Note_Surface_Albedos:equation-3.70}}\begin{equation}\label{equation:tech_note/Surface_Albedos/CLM50_Tech_Note_Surface_Albedos:3.70}
\begin{split}\bar{\psi }=\frac{\int _{\lambda _{1} }^{\lambda _{2} }\psi \left(\lambda \right) I^{\downarrow } \left(\lambda \right){\rm d}\lambda }{\int _{\lambda _{1} }^{\lambda _{2} }I^{\downarrow } \left(\lambda \right){\rm d}\lambda  }\end{split}
\end{equation}
Broadband single-scatter albedo (\(\bar{\omega }\)) is additionally
weighted by the diffuse albedo for a semi-infinite snowpack (\(\alpha _{sno}\))
\phantomsection\label{\detokenize{tech_note/Surface_Albedos/CLM50_Tech_Note_Surface_Albedos:equation-3.71}}\begin{equation}\label{equation:tech_note/Surface_Albedos/CLM50_Tech_Note_Surface_Albedos:3.71}
\begin{split}\bar{\omega }=\frac{\int _{\lambda _{1} }^{\lambda _{2} }\omega (\lambda )I^{\downarrow } ( \lambda )\alpha _{sno} (\lambda ){\rm d}\lambda }{\int _{\lambda _{1} }^{\lambda _{2} }I^{\downarrow } ( \lambda )\alpha _{sno} (\lambda ){\rm d}\lambda }\end{split}
\end{equation}
Inclusion of this additional albedo weight was found to improve accuracy
of the five-band albedo solutions (relative to 470-band solutions)
because of the strong dependence of optically-thick snowpack albedo on
ice grain single-scatter albedo ({\hyperref[\detokenize{tech_note/References/CLM50_Tech_Note_References:flanneretal2007}]{\sphinxcrossref{\DUrole{std,std-ref}{Flanner et al. (2007)}}}}).
The lookup tables contain optical properties for lognormal distributions of ice
particles over the range of effective radii: 30\(\mu\)m
\(< r _{e} < \text{1500} \mu \text{m}\), at 1 \(\mu\) m resolution.
Single-scatter albedos for the end-members of this size range are listed in
\hyperref[\detokenize{tech_note/Surface_Albedos/CLM50_Tech_Note_Surface_Albedos:table-single-scatter-albedo-values-used-for-snowpack-impurities-and-ice}]{Table \ref{\detokenize{tech_note/Surface_Albedos/CLM50_Tech_Note_Surface_Albedos:table-single-scatter-albedo-values-used-for-snowpack-impurities-and-ice}}}.

Optical properties for black carbon are described in {\hyperref[\detokenize{tech_note/References/CLM50_Tech_Note_References:flanneretal2007}]{\sphinxcrossref{\DUrole{std,std-ref}{Flanner et al. (2007)}}}}.
Single-scatter albedo, mass extinction cross-section, and
asymmetry parameter values for all snowpack species, in the five
spectral bands used, are listed in \hyperref[\detokenize{tech_note/Surface_Albedos/CLM50_Tech_Note_Surface_Albedos:table-single-scatter-albedo-values-used-for-snowpack-impurities-and-ice}]{Table \ref{\detokenize{tech_note/Surface_Albedos/CLM50_Tech_Note_Surface_Albedos:table-single-scatter-albedo-values-used-for-snowpack-impurities-and-ice}}},
\hyperref[\detokenize{tech_note/Surface_Albedos/CLM50_Tech_Note_Surface_Albedos:table-mass-extinction-values}]{Table \ref{\detokenize{tech_note/Surface_Albedos/CLM50_Tech_Note_Surface_Albedos:table-mass-extinction-values}}}, and
\hyperref[\detokenize{tech_note/Surface_Albedos/CLM50_Tech_Note_Surface_Albedos:table-asymmetry-scattering-parameters-used-for-snowpack-impurities-and-ice}]{Table \ref{\detokenize{tech_note/Surface_Albedos/CLM50_Tech_Note_Surface_Albedos:table-asymmetry-scattering-parameters-used-for-snowpack-impurities-and-ice}}}.
These properties were also derived with Mie Theory, using various published
sources of indices of refraction and assumptions about particle size
distribution. Weighting into the five CLM spectral bands was determined
only with incident solar flux, as in equation .


\begin{savenotes}\sphinxattablestart
\centering
\sphinxcapstartof{table}
\sphinxcaption{Single-scatter albedo values used for snowpack impurities and ice}\label{\detokenize{tech_note/Surface_Albedos/CLM50_Tech_Note_Surface_Albedos:table-single-scatter-albedo-values-used-for-snowpack-impurities-and-ice}}\label{\detokenize{tech_note/Surface_Albedos/CLM50_Tech_Note_Surface_Albedos:id11}}
\sphinxaftercaption
\begin{tabulary}{\linewidth}[t]{|T|T|T|T|T|T|}
\hline
\sphinxstylethead{\sphinxstyletheadfamily 
Species
\unskip}\relax &\sphinxstylethead{\sphinxstyletheadfamily 
Band 1
\unskip}\relax &\sphinxstylethead{\sphinxstyletheadfamily 
Band 2
\unskip}\relax &\sphinxstylethead{\sphinxstyletheadfamily 
Band 3
\unskip}\relax &\sphinxstylethead{\sphinxstyletheadfamily 
Band 4
\unskip}\relax &\sphinxstylethead{\sphinxstyletheadfamily 
Band 5
\unskip}\relax \\
\hline
Hydrophilic black carbon
&
0.516
&
0.434
&
0.346
&
0.276
&
0.139
\\
\hline
Hydrophobic black carbon
&
0.288
&
0.187
&
0.123
&
0.089
&
0.040
\\
\hline
Hydrophilic organic carbon
&
0.997
&
0.994
&
0.990
&
0.987
&
0.951
\\
\hline
Hydrophobic organic carbon
&
0.963
&
0.921
&
0.860
&
0.814
&
0.744
\\
\hline
Dust 1
&
0.979
&
0.994
&
0.993
&
0.993
&
0.953
\\
\hline
Dust 2
&
0.944
&
0.984
&
0.989
&
0.992
&
0.983
\\
\hline
Dust 3
&
0.904
&
0.965
&
0.969
&
0.973
&
0.978
\\
\hline
Dust 4
&
0.850
&
0.940
&
0.948
&
0.953
&
0.955
\\
\hline
Ice (\(r _{e}\) = 30 \(\mu\) m)
&
0.9999
&
0.9999
&
0.9992
&
0.9938
&
0.9413
\\
\hline
Ice (\(r _{e}\) = 1500 \(\mu\) m)
&
0.9998
&
0.9960
&
0.9680
&
0.8730
&
0.5500
\\
\hline
\end{tabulary}
\par
\sphinxattableend\end{savenotes}


\begin{savenotes}\sphinxattablestart
\centering
\sphinxcapstartof{table}
\sphinxcaption{Mass extinction values (m$^{\text{2}}$ kg$^{\text{-1}}$) used for snowpack impurities and ice}\label{\detokenize{tech_note/Surface_Albedos/CLM50_Tech_Note_Surface_Albedos:table-mass-extinction-values}}\label{\detokenize{tech_note/Surface_Albedos/CLM50_Tech_Note_Surface_Albedos:id12}}
\sphinxaftercaption
\begin{tabulary}{\linewidth}[t]{|T|T|T|T|T|T|}
\hline
\sphinxstylethead{\sphinxstyletheadfamily 
Species
\unskip}\relax &\sphinxstylethead{\sphinxstyletheadfamily 
Band 1
\unskip}\relax &\sphinxstylethead{\sphinxstyletheadfamily 
Band 2
\unskip}\relax &\sphinxstylethead{\sphinxstyletheadfamily 
Band 3
\unskip}\relax &\sphinxstylethead{\sphinxstyletheadfamily 
Band 4
\unskip}\relax &\sphinxstylethead{\sphinxstyletheadfamily 
Band 5
\unskip}\relax \\
\hline
Hydrophilic black carbon
&
25369
&
12520
&
7739
&
5744
&
3527
\\
\hline
Hydrophobic black carbon
&
11398
&
5923
&
4040
&
3262
&
2224
\\
\hline
Hydrophilic organic carbon
&
37774
&
22112
&
14719
&
10940
&
5441
\\
\hline
Hydrophobic organic carbon
&
3289
&
1486
&
872
&
606
&
248
\\
\hline
Dust 1
&
2687
&
2420
&
1628
&
1138
&
466
\\
\hline
Dust 2
&
841
&
987
&
1184
&
1267
&
993
\\
\hline
Dust 3
&
388
&
419
&
400
&
397
&
503
\\
\hline
Dust 4
&
197
&
203
&
208
&
205
&
229
\\
\hline
Ice (\(r _{e}\) = 30 \(\mu\) m)
&
55.7
&
56.1
&
56.3
&
56.6
&
57.3
\\
\hline
Ice (\(r _{e}\) = 1500 \(\mu\) m)
&
1.09
&
1.09
&
1.09
&
1.09
&
1.1
\\
\hline
\end{tabulary}
\par
\sphinxattableend\end{savenotes}


\begin{savenotes}\sphinxattablestart
\centering
\sphinxcapstartof{table}
\sphinxcaption{Asymmetry scattering parameters used for snowpack impurities and ice.}\label{\detokenize{tech_note/Surface_Albedos/CLM50_Tech_Note_Surface_Albedos:table-asymmetry-scattering-parameters-used-for-snowpack-impurities-and-ice}}\label{\detokenize{tech_note/Surface_Albedos/CLM50_Tech_Note_Surface_Albedos:id13}}
\sphinxaftercaption
\begin{tabulary}{\linewidth}[t]{|T|T|T|T|T|T|}
\hline
\sphinxstylethead{\sphinxstyletheadfamily 
Species
\unskip}\relax &\sphinxstylethead{\sphinxstyletheadfamily 
Band 1
\unskip}\relax &\sphinxstylethead{\sphinxstyletheadfamily 
Band 2
\unskip}\relax &\sphinxstylethead{\sphinxstyletheadfamily 
Band 3
\unskip}\relax &\sphinxstylethead{\sphinxstyletheadfamily 
Band 4
\unskip}\relax &\sphinxstylethead{\sphinxstyletheadfamily 
Band 5
\unskip}\relax \\
\hline
Hydrophilic black carbon
&
0.52
&
0.34
&
0.24
&
0.19
&
0.10
\\
\hline
Hydrophobic black carbon
&
0.35
&
0.21
&
0.15
&
0.11
&
0.06
\\
\hline
Hydrophilic organic carbon
&
0.77
&
0.75
&
0.72
&
0.70
&
0.64
\\
\hline
Hydrophobic organic carbon
&
0.62
&
0.57
&
0.54
&
0.51
&
0.44
\\
\hline
Dust 1
&
0.69
&
0.72
&
0.67
&
0.61
&
0.44
\\
\hline
Dust 2
&
0.70
&
0.65
&
0.70
&
0.72
&
0.70
\\
\hline
Dust 3
&
0.79
&
0.75
&
0.68
&
0.63
&
0.67
\\
\hline
Dust 4
&
0.83
&
0.79
&
0.77
&
0.76
&
0.73
\\
\hline
Ice (\(r _{e}\) = 30\(\mu\)m)
&
0.88
&
0.88
&
0.88
&
0.88
&
0.90
\\
\hline
Ice (\(r _{e}\) = 1500\(\mu\)m)
&
0.89
&
0.90
&
0.90
&
0.92
&
0.97
\\
\hline
\end{tabulary}
\par
\sphinxattableend\end{savenotes}


\subsubsection{Snow Aging}
\label{\detokenize{tech_note/Surface_Albedos/CLM50_Tech_Note_Surface_Albedos:snow-aging}}\label{\detokenize{tech_note/Surface_Albedos/CLM50_Tech_Note_Surface_Albedos:id5}}
Snow aging is represented as evolution of the ice effective grain size
(\(r_{e}\)). Previous studies have shown that use of spheres
which conserve the surface area-to-volume ratio (or specific surface
area) of ice media composed of more complex shapes produces relatively
small errors in simulated hemispheric fluxes
(e.g., {\hyperref[\detokenize{tech_note/References/CLM50_Tech_Note_References:grenfellwarren1999}]{\sphinxcrossref{\DUrole{std,std-ref}{Grenfell and Warren 1999}}}}).
Effective radius is the surface area-weighted mean
radius of an ensemble of spherical particles and is directly related to specific
surface area (\sphinxstyleemphasis{SSA}) as
\(r_{e} ={3\mathord{\left/ {\vphantom {3 \left(\rho _{ice} SSA\right)}} \right. \kern-\nulldelimiterspace} \left(\rho _{ice} SSA\right)}\) ,
where \(\rho_{ice}\) is the density of ice. Hence,
\(r_{e}\) is a simple and practical metric for relating the
snowpack microphysical state to dry snow radiative characteristics.

Wet snow processes can also drive rapid changes in albedo. The presence
of liquid water induces rapid coarsening of the surrounding ice grains
(e.g., {\hyperref[\detokenize{tech_note/References/CLM50_Tech_Note_References:brun1989}]{\sphinxcrossref{\DUrole{std,std-ref}{Brun 1989}}}}), and liquid water tends to refreeze into large ice
clumps that darken the bulk snowpack. The presence of small liquid
drops, by itself, does not significantly darken snowpack, as ice and
water have very similar indices of refraction throughout the solar
spectrum. Pooled or ponded water, however, can significantly darken
snowpack by greatly reducing the number of refraction events per unit
mass. This influence is not currently accounted for.

The net change in effective grain size occurring each time step is
represented in each snow layer as a summation of changes caused by dry
snow metamorphism (\(dr_{e,dry}\)), liquid water-induced
metamorphism (\(dr_{e,wet}\)), refreezing of liquid water, and
addition of freshly-fallen snow. The mass of each snow layer is
partitioned into fractions of snow carrying over from the previous time
step (\(f_{old}\)), freshly-fallen snow
(\(f_{new}\)), and refrozen liquid water
(\(f_{rfz}\)), such that snow \(r_{e}\) is updated
each time step \sphinxstyleemphasis{t} as
\phantomsection\label{\detokenize{tech_note/Surface_Albedos/CLM50_Tech_Note_Surface_Albedos:equation-3.72}}\begin{equation}\label{equation:tech_note/Surface_Albedos/CLM50_Tech_Note_Surface_Albedos:3.72}
\begin{split}r_{e} \left(t\right)=\left[r_{e} \left(t-1\right)+dr_{e,\, dry} +dr_{e,\, wet} \right]f_{old} +r_{e,\, 0} f_{new} +r_{e,\, rfz} f_{rfrz}\end{split}
\end{equation}
Here, the effective radius of freshly-fallen snow
(\(r_{e,0}\)) is fixed globally at 54.5 \(\mu\) m (corresponding to a specific surface area of 60 m$^{\text{2}}$ kg$^{\text{-1}}$), and the effective
radius of refrozen liquid water (\(r_{e,rfz}\)) is set to 1000\(\mu\) m.

Dry snow aging is based on a microphysical model described by {\hyperref[\detokenize{tech_note/References/CLM50_Tech_Note_References:flannerzender2006}]{\sphinxcrossref{\DUrole{std,std-ref}{Flanner
and Zender (2006)}}}}. This model simulates diffusive vapor flux
amongst collections of ice crystals with various size and inter-particle
spacing. Specific surface area and effective radius are prognosed for
any combination of snow temperature, temperature gradient, density, and
initial size distribution. The combination of warm snow, large
temperature gradient, and low density produces the most rapid snow
aging, whereas aging proceeds slowly in cold snow, regardless of
temperature gradient and density. Because this model is currently too
computationally expensive for inclusion in climate models, we fit
parametric curves to model output over a wide range of snow conditions
and apply these parameters in CLM. The functional form of the parametric
equation is
\phantomsection\label{\detokenize{tech_note/Surface_Albedos/CLM50_Tech_Note_Surface_Albedos:equation-3.73}}\begin{equation}\label{equation:tech_note/Surface_Albedos/CLM50_Tech_Note_Surface_Albedos:3.73}
\begin{split}\frac{dr_{e,\, dry} }{dt} =\left(\frac{dr_{e} }{dt} \right)_{0} \left(\frac{\eta }{\left(r_{e} -r_{e,\, 0} \right)+\eta } \right)^{{1\mathord{\left/ {\vphantom {1 \kappa }} \right. \kern-\nulldelimiterspace} \kappa } }\end{split}
\end{equation}
The parameters \({(\frac{dr_{e}}{dt}})_{0}\),
\(\eta\), and \(\kappa\) are retrieved interactively from a
lookup table with dimensions corresponding to snow temperature,
temperature gradient, and density. The domain covered by this lookup
table includes temperature ranging from 223 to 273 K, temperature
gradient ranging from 0 to 300 K m$^{\text{-1}}$, and density ranging
from 50 to 400 kg m$^{\text{-3}}$. Temperature gradient is calculated
at the midpoint of each snow layer \sphinxstyleemphasis{n}, using mid-layer temperatures
(\(T_{n}\)) and snow layer thicknesses (\(dz_{n}\)), as
\phantomsection\label{\detokenize{tech_note/Surface_Albedos/CLM50_Tech_Note_Surface_Albedos:equation-3.74}}\begin{equation}\label{equation:tech_note/Surface_Albedos/CLM50_Tech_Note_Surface_Albedos:3.74}
\begin{split}\left(\frac{dT}{dz} \right)_{n} =\frac{1}{dz_{n} } abs\left[\frac{T_{n-1} dz_{n} +T_{n} dz_{n-1} }{dz_{n} +dz_{n-1} } +\frac{T_{n+1} dz_{n} +T_{n} dz_{n+1} }{dz_{n} +dz_{n+1} } \right]\end{split}
\end{equation}
For the bottom snow layer (\(n=0\)),
\(T_{n+1}\) is taken as the temperature of the
top soil layer, and for the top snow layer it is assumed that
\(T_{n-1}\) = \(T_{n}\).

The contribution of liquid water to enhanced metamorphism is based on
parametric equations published by {\hyperref[\detokenize{tech_note/References/CLM50_Tech_Note_References:brun1989}]{\sphinxcrossref{\DUrole{std,std-ref}{Brun (1989)}}}}, who measured grain
growth rates under different liquid water contents. This relationship,
expressed in terms of \(r_{e} (\mu \text{m})\) and
subtracting an offset due to dry aging, depends on the mass liquid water
fraction \(f_{liq}\) as
\phantomsection\label{\detokenize{tech_note/Surface_Albedos/CLM50_Tech_Note_Surface_Albedos:equation-3.75}}\begin{equation}\label{equation:tech_note/Surface_Albedos/CLM50_Tech_Note_Surface_Albedos:3.75}
\begin{split}\frac{dr_{e} }{dt} =\frac{10^{18} C_{1} f_{liq} ^{3} }{4\pi r_{e} ^{2} }\end{split}
\end{equation}
The constant \sphinxstyleemphasis{C}$_{\text{1}}$ is 4.22\(\times\)10$^{\text{-13}}$, and:
\(f_{liq} =w_{liq} /(w_{liq} +w_{ice} )\)(Chapter \hyperref[\detokenize{tech_note/Snow_Hydrology/CLM50_Tech_Note_Snow_Hydrology:rst-snow-hydrology}]{\ref{\detokenize{tech_note/Snow_Hydrology/CLM50_Tech_Note_Snow_Hydrology:rst-snow-hydrology}}}).

In cases where snow mass is greater than zero, but a snow layer has not
yet been defined, \(r_{e}\) is set to \(r_{e,0}\). When snow layers are combined or
divided, \(r_{e}\) is calculated as a mass-weighted mean of
the two layers, following computations of other state variables (section
\hyperref[\detokenize{tech_note/Snow_Hydrology/CLM50_Tech_Note_Snow_Hydrology:snow-layer-combination-and-subdivision}]{\ref{\detokenize{tech_note/Snow_Hydrology/CLM50_Tech_Note_Snow_Hydrology:snow-layer-combination-and-subdivision}}}). Finally, the allowable range of \(r_{e}\),
corresponding to the range over which Mie optical properties have been
defined, is 30-1500\(\mu\) m.


\subsection{Solar Zenith Angle}
\label{\detokenize{tech_note/Surface_Albedos/CLM50_Tech_Note_Surface_Albedos:id6}}\label{\detokenize{tech_note/Surface_Albedos/CLM50_Tech_Note_Surface_Albedos:solar-zenith-angle}}
The CLM uses the same formulation for solar zenith angle as the
Community Atmosphere Model. The cosine of the solar zenith angle
\(\mu\)  is
\phantomsection\label{\detokenize{tech_note/Surface_Albedos/CLM50_Tech_Note_Surface_Albedos:equation-3.76}}\begin{equation}\label{equation:tech_note/Surface_Albedos/CLM50_Tech_Note_Surface_Albedos:3.76}
\begin{split}\mu =\sin \phi \sin \delta -\cos \phi \cos \delta \cos h\end{split}
\end{equation}
where \(h\) is the solar hour angle (radians) (24 hour periodicity),
\(\delta\)  is the solar declination angle (radians), and
\(\phi\)  is latitude (radians) (positive in Northern Hemisphere).
The solar hour angle \(h\) (radians) is
\phantomsection\label{\detokenize{tech_note/Surface_Albedos/CLM50_Tech_Note_Surface_Albedos:equation-3.77}}\begin{equation}\label{equation:tech_note/Surface_Albedos/CLM50_Tech_Note_Surface_Albedos:3.77}
\begin{split}h=2\pi d+\theta\end{split}
\end{equation}
where \(d\) is calendar day (\(d=0.0\) at 0Z on January 1), and
\(\theta\)  is longitude (radians) (positive east of the
Greenwich meridian).

The solar declination angle \(\delta\)  is calculated as in {\hyperref[\detokenize{tech_note/References/CLM50_Tech_Note_References:berger1978a}]{\sphinxcrossref{\DUrole{std,std-ref}{Berger
(1978a,b)}}}} and is valid for one million years past or hence, relative to
1950 A.D. The orbital parameters may be specified directly or the
orbital parameters are calculated for the desired year. The required
orbital parameters to be input by the user are the obliquity of the
Earth \(\varepsilon\)  (degrees,
\(-90^{\circ } <\varepsilon <90^{\circ }\) ), Earth’s eccentricity
\(e\) (\(0.0<e<0.1\)), and the longitude of the perihelion
relative to the moving vernal equinox \(\tilde{\omega }\)
(\(0^{\circ } <\tilde{\omega }<360^{\circ }\) ) (unadjusted for the
apparent orbit of the Sun around the Earth ({\hyperref[\detokenize{tech_note/References/CLM50_Tech_Note_References:bergeretal1993}]{\sphinxcrossref{\DUrole{std,std-ref}{Berger et al. 1993}}}})). The
solar declination \(\delta\)  (radians) is
\phantomsection\label{\detokenize{tech_note/Surface_Albedos/CLM50_Tech_Note_Surface_Albedos:equation-3.78}}\begin{equation}\label{equation:tech_note/Surface_Albedos/CLM50_Tech_Note_Surface_Albedos:3.78}
\begin{split}\delta =\sin ^{-1} \left[\sin \left(\varepsilon \right)\sin \left(\lambda \right)\right]\end{split}
\end{equation}
where \(\varepsilon\)  is Earth’s obliquity and \(\lambda\)  is
the true longitude of the Earth.

The obliquity of the Earth \(\varepsilon\)  (degrees) is
\phantomsection\label{\detokenize{tech_note/Surface_Albedos/CLM50_Tech_Note_Surface_Albedos:equation-3.79}}\begin{equation}\label{equation:tech_note/Surface_Albedos/CLM50_Tech_Note_Surface_Albedos:3.79}
\begin{split}\varepsilon =\varepsilon *+\sum _{i=1}^{i=47}A_{i}  \cos \left(f_{i} t+\delta _{i} \right)\end{split}
\end{equation}
where \(\varepsilon *\) is a constant of integration (\hyperref[\detokenize{tech_note/Surface_Albedos/CLM50_Tech_Note_Surface_Albedos:table-orbital-parameters}]{Table \ref{\detokenize{tech_note/Surface_Albedos/CLM50_Tech_Note_Surface_Albedos:table-orbital-parameters}}}),
\(A_{i}\) , \(f_{i}\) , and \(\delta _{i}\)  are amplitude,
mean rate, and phase terms in the cosine series expansion ({\hyperref[\detokenize{tech_note/References/CLM50_Tech_Note_References:berger1978a}]{\sphinxcrossref{\DUrole{std,std-ref}{Berger
(1978a,b)}}}}, and \(t=t_{0} -1950\) where \(t_{0}\)  is the year.
The series expansion terms are not shown here but can be found in the
source code file shr\_orb\_mod.F90.

The true longitude of the Earth \(\lambda\)  (radians) is counted
counterclockwise from the vernal equinox (\(\lambda =0\) at the
vernal equinox)
\phantomsection\label{\detokenize{tech_note/Surface_Albedos/CLM50_Tech_Note_Surface_Albedos:equation-3.80}}\begin{equation}\label{equation:tech_note/Surface_Albedos/CLM50_Tech_Note_Surface_Albedos:3.80}
\begin{split}\lambda =\lambda _{m} +\left(2e-\frac{1}{4} e^{3} \right)\sin \left(\lambda _{m} -\tilde{\omega }\right)+\frac{5}{4} e^{2} \sin 2\left(\lambda _{m} -\tilde{\omega }\right)+\frac{13}{12} e^{3} \sin 3\left(\lambda _{m} -\tilde{\omega }\right)\end{split}
\end{equation}
where \(\lambda _{m}\)  is the mean longitude of the Earth at the
vernal equinox, \(e\) is Earth’s eccentricity, and
\(\tilde{\omega }\) is the longitude of the perihelion relative to
the moving vernal equinox. The mean longitude \(\lambda _{m}\)  is
\phantomsection\label{\detokenize{tech_note/Surface_Albedos/CLM50_Tech_Note_Surface_Albedos:equation-3.81}}\begin{equation}\label{equation:tech_note/Surface_Albedos/CLM50_Tech_Note_Surface_Albedos:3.81}
\begin{split}\lambda _{m} =\lambda _{m0} +\frac{2\pi \left(d-d_{ve} \right)}{365}\end{split}
\end{equation}
where \(d_{ve} =80.5\) is the calendar day at vernal equinox (March
21 at noon), and
\phantomsection\label{\detokenize{tech_note/Surface_Albedos/CLM50_Tech_Note_Surface_Albedos:equation-3.82}}\begin{equation}\label{equation:tech_note/Surface_Albedos/CLM50_Tech_Note_Surface_Albedos:3.82}
\begin{split}\lambda _{m0} =2\left[\left(\frac{1}{2} e+\frac{1}{8} e^{3} \right)\left(1+\beta \right)\sin \tilde{\omega }-\frac{1}{4} e^{2} \left(\frac{1}{2} +\beta \right)\sin 2\tilde{\omega }+\frac{1}{8} e^{3} \left(\frac{1}{3} +\beta \right)\sin 3\tilde{\omega }\right]\end{split}
\end{equation}
where \(\beta =\sqrt{1-e^{2} }\) . Earth’s eccentricity \(e\)
is
\phantomsection\label{\detokenize{tech_note/Surface_Albedos/CLM50_Tech_Note_Surface_Albedos:equation-3.83}}\begin{equation}\label{equation:tech_note/Surface_Albedos/CLM50_Tech_Note_Surface_Albedos:3.83}
\begin{split}e=\sqrt{\left(e^{\cos } \right)^{2} +\left(e^{\sin } \right)^{2} }\end{split}
\end{equation}
where
\phantomsection\label{\detokenize{tech_note/Surface_Albedos/CLM50_Tech_Note_Surface_Albedos:equation-3.84}}\begin{equation}\label{equation:tech_note/Surface_Albedos/CLM50_Tech_Note_Surface_Albedos:3.84}
\begin{split}\begin{array}{l} {e^{\cos } =\sum _{j=1}^{19}M_{j} \cos \left(g_{j} t+B_{j} \right) ,} \\ {e^{\sin } =\sum _{j=1}^{19}M_{j} \sin \left(g_{j} t+B_{j} \right) } \end{array}\end{split}
\end{equation}
are the cosine and sine series expansions for \(e\), and
\(M_{j}\) , \(g_{j}\) , and \(B_{j}\)  are amplitude, mean
rate, and phase terms in the series expansions ({\hyperref[\detokenize{tech_note/References/CLM50_Tech_Note_References:berger1978a}]{\sphinxcrossref{\DUrole{std,std-ref}{Berger (1978a,b)}}}}). The
longitude of the perihelion relative to the moving vernal equinox
\(\tilde{\omega }\) (degrees) is
\phantomsection\label{\detokenize{tech_note/Surface_Albedos/CLM50_Tech_Note_Surface_Albedos:equation-3.85}}\begin{equation}\label{equation:tech_note/Surface_Albedos/CLM50_Tech_Note_Surface_Albedos:3.85}
\begin{split}\tilde{\omega }=\Pi \frac{180}{\pi } +\psi\end{split}
\end{equation}
where \(\Pi\)  is the longitude of the perihelion measured from the
reference vernal equinox (i.e., the vernal equinox at 1950 A.D.) and
describes the absolute motion of the perihelion relative to the fixed
stars, and \(\psi\)  is the annual general precession in longitude
and describes the absolute motion of the vernal equinox along Earth’s
orbit relative to the fixed stars. The general precession \(\psi\)
(degrees) is
\phantomsection\label{\detokenize{tech_note/Surface_Albedos/CLM50_Tech_Note_Surface_Albedos:equation-3.86}}\begin{equation}\label{equation:tech_note/Surface_Albedos/CLM50_Tech_Note_Surface_Albedos:3.86}
\begin{split}\psi =\frac{\tilde{\psi }t}{3600} +\zeta +\sum _{i=1}^{78}F_{i}  \sin \left(f_{i} ^{{'} } t+\delta _{i} ^{{'} } \right)\end{split}
\end{equation}
where \(\tilde{\psi }\) (arcseconds) and \(\zeta\)  (degrees)
are constants (\hyperref[\detokenize{tech_note/Surface_Albedos/CLM50_Tech_Note_Surface_Albedos:table-orbital-parameters}]{Table \ref{\detokenize{tech_note/Surface_Albedos/CLM50_Tech_Note_Surface_Albedos:table-orbital-parameters}}}), and \(F_{i}\) , \(f_{i} ^{{'} }\) ,
and \(\delta _{i} ^{{'} }\)  are amplitude, mean rate, and phase
terms in the sine series expansion ({\hyperref[\detokenize{tech_note/References/CLM50_Tech_Note_References:berger1978a}]{\sphinxcrossref{\DUrole{std,std-ref}{Berger (1978a,b)}}}})). The longitude of
the perihelion \(\Pi\)  (radians) depends on the sine and cosine
series expansions for the eccentricity \(e\)as follows:
\phantomsection\label{\detokenize{tech_note/Surface_Albedos/CLM50_Tech_Note_Surface_Albedos:equation-3.87}}\begin{equation}\label{equation:tech_note/Surface_Albedos/CLM50_Tech_Note_Surface_Albedos:3.87}
\begin{split}\Pi =\left\{\begin{array}{lr}
0 & \qquad {\rm for\; -1}\times {\rm 10}^{{\rm -8}} \le e^{\cos } \le 1\times 10^{-8} {\rm \; and\; }e^{\sin } = 0 \\
1.5\pi & \qquad {\rm for\; -1}\times {\rm 10}^{{\rm -8}} \le e^{\cos } \le 1\times 10^{-8} {\rm \; and\; }e^{\sin } < 0 \\
0.5\pi & \qquad {\rm for\; -1}\times {\rm 10}^{{\rm -8}} \le e^{\cos } \le 1\times 10^{-8} {\rm \; and\; }e^{\sin } > 0 \\
\tan ^{-1} \left[\frac{e^{\sin } }{e^{\cos } } \right]+\pi & \qquad {\rm for\; }e^{\cos } <{\rm -1}\times {\rm 10}^{{\rm -8}}  \\
\tan ^{-1} \left[\frac{e^{\sin } }{e^{\cos } } \right]+2\pi & \qquad {\rm for\; }e^{\cos } >{\rm 1}\times {\rm 10}^{{\rm -8}} {\rm \; and\; }e^{\sin } <0 \\
\tan ^{-1} \left[\frac{e^{\sin } }{e^{\cos } } \right] & \qquad {\rm for\; }e^{\cos } >{\rm 1}\times {\rm 10}^{{\rm -8}} {\rm \; and\; }e^{\sin } \ge 0
\end{array}\right\}.\end{split}
\end{equation}
The numerical solution for the longitude of the perihelion
\(\tilde{\omega }\) is constrained to be between 0 and 360 degrees
(measured from the autumn equinox). A constant 180 degrees is then added
to \(\tilde{\omega }\) because the Sun is considered as revolving
around the Earth (geocentric coordinate system) ({\hyperref[\detokenize{tech_note/References/CLM50_Tech_Note_References:bergeretal1993}]{\sphinxcrossref{\DUrole{std,std-ref}{Berger et al. 1993}}}})).


\begin{savenotes}\sphinxattablestart
\centering
\sphinxcapstartof{table}
\sphinxcaption{Orbital parameters}\label{\detokenize{tech_note/Surface_Albedos/CLM50_Tech_Note_Surface_Albedos:table-orbital-parameters}}\label{\detokenize{tech_note/Surface_Albedos/CLM50_Tech_Note_Surface_Albedos:id14}}
\sphinxaftercaption
\begin{tabulary}{\linewidth}[t]{|T|T|}
\hline
\sphinxstylethead{\sphinxstyletheadfamily 
Parameter
\unskip}\relax &\sphinxstylethead{\sphinxstyletheadfamily \unskip}\relax \\
\hline
\(\varepsilon *\)
&
23.320556
\\
\hline
\(\tilde{\psi }\) (arcseconds)
&
50.439273
\\
\hline
\(\zeta\)  (degrees)
&
3.392506
\\
\hline
\end{tabulary}
\par
\sphinxattableend\end{savenotes}


\section{Radiative Fluxes}
\label{\detokenize{tech_note/Radiative_Fluxes/CLM50_Tech_Note_Radiative_Fluxes:radiative-fluxes}}\label{\detokenize{tech_note/Radiative_Fluxes/CLM50_Tech_Note_Radiative_Fluxes::doc}}\label{\detokenize{tech_note/Radiative_Fluxes/CLM50_Tech_Note_Radiative_Fluxes:rst-radiative-fluxes}}
The net radiation at the surface is
\(\left(\vec{S}_{v} +\vec{S}_{g} \right)-\left(\vec{L}_{v} +\vec{L}_{g} \right)\),
where \(\vec{S}\) is the net solar flux absorbed by the vegetation
(“v”) and the ground (“g”) and \(\vec{L}\) is the net longwave flux
(positive toward the atmosphere) (W m$^{\text{-2}}$).


\subsection{Solar Fluxes}
\label{\detokenize{tech_note/Radiative_Fluxes/CLM50_Tech_Note_Radiative_Fluxes:solar-fluxes}}\label{\detokenize{tech_note/Radiative_Fluxes/CLM50_Tech_Note_Radiative_Fluxes:id1}}
\hyperref[\detokenize{tech_note/Radiative_Fluxes/CLM50_Tech_Note_Radiative_Fluxes:figure-radiation-schematic}]{Figure \ref{\detokenize{tech_note/Radiative_Fluxes/CLM50_Tech_Note_Radiative_Fluxes:figure-radiation-schematic}}} illustrates the direct beam and diffuse fluxes in the canopy.

\(I\, \uparrow _{\Lambda }^{\mu }\)  and
\(I\, \uparrow _{\Lambda }\)  are the upward diffuse fluxes, per
unit incident direct beam and diffuse flux (section \hyperref[\detokenize{tech_note/Surface_Albedos/CLM50_Tech_Note_Surface_Albedos:canopy-radiative-transfer}]{\ref{\detokenize{tech_note/Surface_Albedos/CLM50_Tech_Note_Surface_Albedos:canopy-radiative-transfer}}}).
\(I\, \downarrow _{\Lambda }^{\mu }\)  and
\(I\, \downarrow _{\Lambda }\) are the downward diffuse fluxes
below the vegetation per unit incident direct beam and diffuse radiation
(section \hyperref[\detokenize{tech_note/Surface_Albedos/CLM50_Tech_Note_Surface_Albedos:canopy-radiative-transfer}]{\ref{\detokenize{tech_note/Surface_Albedos/CLM50_Tech_Note_Surface_Albedos:canopy-radiative-transfer}}}). The direct beam flux
transmitted through the canopy, per
unit incident flux, is \(e^{-K\left(L+S\right)}\) .
\(\vec{I}_{\Lambda }^{\mu }\)  and \(\vec{I}_{\Lambda }^{}\)
are the fluxes absorbed by the vegetation, per unit incident direct beam
and diffuse radiation (section \hyperref[\detokenize{tech_note/Surface_Albedos/CLM50_Tech_Note_Surface_Albedos:canopy-radiative-transfer}]{\ref{\detokenize{tech_note/Surface_Albedos/CLM50_Tech_Note_Surface_Albedos:canopy-radiative-transfer}}}).
\(\alpha _{g,\, \Lambda }^{\mu }\)  and
\(\alpha _{g,\, \Lambda }\)  are the direct beam and diffuse ground
albedos (section \hyperref[\detokenize{tech_note/Surface_Albedos/CLM50_Tech_Note_Surface_Albedos:ground-albedos}]{\ref{\detokenize{tech_note/Surface_Albedos/CLM50_Tech_Note_Surface_Albedos:ground-albedos}}}). \(L\) and \(S\) are the exposed leaf area
index and stem area index (section \hyperref[\detokenize{tech_note/Ecosystem/CLM50_Tech_Note_Ecosystem:phenology-and-vegetation-burial-by-snow}]{\ref{\detokenize{tech_note/Ecosystem/CLM50_Tech_Note_Ecosystem:phenology-and-vegetation-burial-by-snow}}}).
\(K\) is the optical
depth of direct beam per unit leaf and stem area (section \hyperref[\detokenize{tech_note/Surface_Albedos/CLM50_Tech_Note_Surface_Albedos:canopy-radiative-transfer}]{\ref{\detokenize{tech_note/Surface_Albedos/CLM50_Tech_Note_Surface_Albedos:canopy-radiative-transfer}}}).

\begin{figure}[htbp]
\centering
\capstart

\noindent\sphinxincludegraphics{{image15}.png}
\caption{Schematic diagram of (a) direct beam radiation, (b) diffuse
solar radiation, and (c) longwave radiation absorbed, transmitted, and
reflected by vegetation and ground.}\label{\detokenize{tech_note/Radiative_Fluxes/CLM50_Tech_Note_Radiative_Fluxes:figure-radiation-schematic}}\label{\detokenize{tech_note/Radiative_Fluxes/CLM50_Tech_Note_Radiative_Fluxes:id3}}\end{figure}

For clarity, terms involving \(T^{n+1} -T^{n}\)  are not shown in
(c).

The total solar radiation absorbed by the vegetation and ground is
\phantomsection\label{\detokenize{tech_note/Radiative_Fluxes/CLM50_Tech_Note_Radiative_Fluxes:equation-4.1}}\begin{equation}\label{equation:tech_note/Radiative_Fluxes/CLM50_Tech_Note_Radiative_Fluxes:4.1}
\begin{split}\vec{S}_{v} =\sum _{\Lambda }S_{atm} \, \downarrow _{\Lambda }^{\mu } \overrightarrow{I}_{\Lambda }^{\mu } +S_{atm} \, \downarrow _{\Lambda } \overrightarrow{I}_{\Lambda }\end{split}
\end{equation}\phantomsection\label{\detokenize{tech_note/Radiative_Fluxes/CLM50_Tech_Note_Radiative_Fluxes:equation-4.2}}\begin{equation}\label{equation:tech_note/Radiative_Fluxes/CLM50_Tech_Note_Radiative_Fluxes:4.2}
\begin{split}\begin{array}{l} {\vec{S}_{g} =\sum _{\Lambda }S_{atm} \, \downarrow _{\Lambda }^{\mu } e^{-K\left(L+S\right)} \left(1-\alpha _{g,\, \Lambda }^{\mu } \right) +} \\ {\qquad \left(S_{atm} \, \downarrow _{\Lambda }^{\mu } I\downarrow _{\Lambda }^{\mu } +S_{atm} \downarrow _{\Lambda } I\downarrow _{\Lambda } \right)\left(1-\alpha _{g,\, \Lambda } \right)} \end{array}\end{split}
\end{equation}
where \(S_{atm} \, \downarrow _{\Lambda }^{\mu }\)  and
\(S_{atm} \, \downarrow _{\Lambda }\)  are the incident direct beam
and diffuse solar fluxes (W m$^{\text{-2}}$). For non-vegetated
surfaces, \(e^{-K\left(L+S\right)} =1\),
\(\overrightarrow{I}_{\Lambda }^{\mu } =\overrightarrow{I}_{\Lambda } =0\),
\(I\, \downarrow _{\Lambda }^{\mu } =0\), and
\(I\, \downarrow _{\Lambda } =1\), so that
\phantomsection\label{\detokenize{tech_note/Radiative_Fluxes/CLM50_Tech_Note_Radiative_Fluxes:equation-4.3}}\begin{equation}\label{equation:tech_note/Radiative_Fluxes/CLM50_Tech_Note_Radiative_Fluxes:4.3}
\begin{split}\begin{array}{l} {\vec{S}_{g} =\sum _{\Lambda }S_{atm} \, \downarrow _{\Lambda }^{\mu } \left(1-\alpha _{g,\, \Lambda }^{\mu } \right) +S_{atm} \, \downarrow _{\Lambda } \left(1-\alpha _{g,\, \Lambda } \right)} \\ {\vec{S}_{v} =0} \end{array}.\end{split}
\end{equation}
Solar radiation is conserved as
\phantomsection\label{\detokenize{tech_note/Radiative_Fluxes/CLM50_Tech_Note_Radiative_Fluxes:equation-4.4}}\begin{equation}\label{equation:tech_note/Radiative_Fluxes/CLM50_Tech_Note_Radiative_Fluxes:4.4}
\begin{split}\sum _{\Lambda }\left(S_{atm} \, \downarrow _{\Lambda }^{\mu } +S_{atm} \, \downarrow _{\Lambda } \right)=\left(\vec{S}_{v} +\vec{S}_{g} \right) +\sum _{\Lambda }\left(S_{atm} \, \downarrow _{\Lambda }^{\mu } I\uparrow _{\Lambda }^{\mu } +S_{atm} \, \downarrow _{\Lambda } I\uparrow _{\Lambda } \right)\end{split}
\end{equation}
where the latter term in parentheses is reflected solar radiation.

Photosynthesis and transpiration depend non-linearly on solar radiation,
via the light response of stomata. The canopy is treated as two leaves
(sunlit and shaded) and the solar radiation in the visible waveband
(\(<\) 0.7 µm) absorbed by the vegetation is apportioned to the
sunlit and shaded leaves (section \hyperref[\detokenize{tech_note/Surface_Albedos/CLM50_Tech_Note_Surface_Albedos:canopy-radiative-transfer}]{\ref{\detokenize{tech_note/Surface_Albedos/CLM50_Tech_Note_Surface_Albedos:canopy-radiative-transfer}}}).
The absorbed photosynthetically
active (visible waveband) radiation averaged over the sunlit canopy (per
unit plant area) is
\phantomsection\label{\detokenize{tech_note/Radiative_Fluxes/CLM50_Tech_Note_Radiative_Fluxes:equation-4.5}}\begin{equation}\label{equation:tech_note/Radiative_Fluxes/CLM50_Tech_Note_Radiative_Fluxes:4.5}
\begin{split}\phi ^{sun} ={\left(\vec{I}_{sun,vis}^{\mu } S_{atm} \downarrow _{vis}^{\mu } +\vec{I}_{sun,vis}^{} S_{atm} \downarrow _{vis}^{} \right)\mathord{\left/ {\vphantom {\left(\vec{I}_{sun,vis}^{\mu } S_{atm} \downarrow _{vis}^{\mu } +\vec{I}_{sun,vis}^{} S_{atm} \downarrow _{vis}^{} \right) L^{sun} }} \right. \kern-\nulldelimiterspace} L^{sun} }\end{split}
\end{equation}
and the absorbed radiation for the average shaded leaf (per unit plant
area) is
\phantomsection\label{\detokenize{tech_note/Radiative_Fluxes/CLM50_Tech_Note_Radiative_Fluxes:equation-4.6}}\begin{equation}\label{equation:tech_note/Radiative_Fluxes/CLM50_Tech_Note_Radiative_Fluxes:4.6}
\begin{split}\phi ^{sha} ={\left(\vec{I}_{sha,vis}^{\mu } S_{atm} \downarrow _{vis}^{\mu } +\vec{I}_{sha,vis}^{} S_{atm} \downarrow _{vis}^{} \right)\mathord{\left/ {\vphantom {\left(\vec{I}_{sha,vis}^{\mu } S_{atm} \downarrow _{vis}^{\mu } +\vec{I}_{sha,vis}^{} S_{atm} \downarrow _{vis}^{} \right) L^{sha} }} \right. \kern-\nulldelimiterspace} L^{sha} }\end{split}
\end{equation}
with \(L^{sun}\)  and \(L^{sha}\)  the sunlit and shaded plant
area index, respectively. The sunlit plant area index is
\phantomsection\label{\detokenize{tech_note/Radiative_Fluxes/CLM50_Tech_Note_Radiative_Fluxes:equation-4.7}}\begin{equation}\label{equation:tech_note/Radiative_Fluxes/CLM50_Tech_Note_Radiative_Fluxes:4.7}
\begin{split}L^{sun} =\frac{1-e^{-K(L+S)} }{K}\end{split}
\end{equation}
and the shaded leaf area index is \(L^{sha} =(L+S)-L^{sun}\) . In
calculating \(L^{sun}\) ,
\phantomsection\label{\detokenize{tech_note/Radiative_Fluxes/CLM50_Tech_Note_Radiative_Fluxes:equation-4.8}}\begin{equation}\label{equation:tech_note/Radiative_Fluxes/CLM50_Tech_Note_Radiative_Fluxes:4.8}
\begin{split}K=\frac{G\left(\mu \right)}{\mu }\end{split}
\end{equation}
where \(G\left(\mu \right)\) and \(\mu\)  are parameters in the
two-stream approximation (section \hyperref[\detokenize{tech_note/Surface_Albedos/CLM50_Tech_Note_Surface_Albedos:canopy-radiative-transfer}]{\ref{\detokenize{tech_note/Surface_Albedos/CLM50_Tech_Note_Surface_Albedos:canopy-radiative-transfer}}}).

The model uses the two-stream approximation to calculate radiative
transfer of direct and diffuse radiation through a canopy that is
differentiated into leaves that are sunlit and those that are shaded
(section \hyperref[\detokenize{tech_note/Surface_Albedos/CLM50_Tech_Note_Surface_Albedos:canopy-radiative-transfer}]{\ref{\detokenize{tech_note/Surface_Albedos/CLM50_Tech_Note_Surface_Albedos:canopy-radiative-transfer}}}). The two-stream equations
are integrated over all plant
area (leaf and stem area) in the canopy. The model has an optional
(though not supported) multi-layer canopy, as described by
{\hyperref[\detokenize{tech_note/References/CLM50_Tech_Note_References:bonanetal2012}]{\sphinxcrossref{\DUrole{std,std-ref}{Bonan et al. (2012)}}}}.
The multi-layer model is only intended to address the
non-linearity of light profiles, photosynthesis, and stomatal
conductance in the plant canopy.

In the multi-layer canopy, canopy-integrated radiative fluxes are
calculated from the two-stream approximation. The model additionally
derives the light profile with depth in the canopy by taking the
derivatives of the absorbed radiative fluxes with respect to plant area
index (\(L'=L+S\)) and evaluating them incrementally through the
canopy with cumulative plant area index (\(x\)). The terms
\({d\vec{I}_{sun,\Lambda }^{\mu } (x)\mathord{\left/ {\vphantom {d\vec{I}_{sun,\Lambda }^{\mu } (x) dL'}} \right. \kern-\nulldelimiterspace} dL'}\)
and
\({d\vec{I}_{sun,\Lambda }^{} (x)\mathord{\left/ {\vphantom {d\vec{I}_{sun,\Lambda }^{} (x) dL'}} \right. \kern-\nulldelimiterspace} dL'}\)
are the direct beam and diffuse solar radiation, respectively, absorbed
by the sunlit fraction of the canopy (per unit plant area) at a depth
defined by the cumulative plant area index \(x\);
\({d\vec{I}_{sha,\Lambda }^{\mu } (x)\mathord{\left/ {\vphantom {d\vec{I}_{sha,\Lambda }^{\mu } (x) dL'}} \right. \kern-\nulldelimiterspace} dL'}\) and
\({d\vec{I}_{sha,\Lambda }^{} (x)\mathord{\left/ {\vphantom {d\vec{I}_{sha,\Lambda }^{} (x) dL'}} \right. \kern-\nulldelimiterspace} dL'}\)
are the corresponding fluxes for the shaded fraction of the canopy at
depth \(x\). These fluxes are normalized by the sunlit or shaded
fraction at depth \(x\), defined by
\(f_{sun} =\exp \left(-Kx\right)\), to give fluxes per unit sunlit
or shaded plant area at depth \(x\).


\subsection{Longwave Fluxes}
\label{\detokenize{tech_note/Radiative_Fluxes/CLM50_Tech_Note_Radiative_Fluxes:longwave-fluxes}}\label{\detokenize{tech_note/Radiative_Fluxes/CLM50_Tech_Note_Radiative_Fluxes:id2}}
The net longwave radiation (W m$^{\text{-2}}$) (positive toward the
atmosphere) at the surface is
\phantomsection\label{\detokenize{tech_note/Radiative_Fluxes/CLM50_Tech_Note_Radiative_Fluxes:equation-4.9}}\begin{equation}\label{equation:tech_note/Radiative_Fluxes/CLM50_Tech_Note_Radiative_Fluxes:4.9}
\begin{split}\vec{L}=L\, \uparrow -L_{atm} \, \downarrow\end{split}
\end{equation}
where \(L\, \uparrow\)  is the upward longwave radiation from the
surface and \(L_{atm} \, \downarrow\)  is the downward atmospheric
longwave radiation (W m$^{\text{-2}}$). The radiative temperature
\(T_{rad}\)  (K) is defined from the upward longwave radiation as
\phantomsection\label{\detokenize{tech_note/Radiative_Fluxes/CLM50_Tech_Note_Radiative_Fluxes:equation-4.10}}\begin{equation}\label{equation:tech_note/Radiative_Fluxes/CLM50_Tech_Note_Radiative_Fluxes:4.10}
\begin{split}T_{rad} =\left(\frac{L\, \uparrow }{\sigma } \right)^{{1\mathord{\left/ {\vphantom {1 4}} \right. \kern-\nulldelimiterspace} 4} }\end{split}
\end{equation}
where \(\sigma\)  is the Stefan-Boltzmann constant (Wm$^{\text{-2}}$ K$^{\text{-4}}$) (\hyperref[\detokenize{tech_note/Ecosystem/CLM50_Tech_Note_Ecosystem:table-physical-constants}]{Table \ref{\detokenize{tech_note/Ecosystem/CLM50_Tech_Note_Ecosystem:table-physical-constants}}}). With reference to
\hyperref[\detokenize{tech_note/Radiative_Fluxes/CLM50_Tech_Note_Radiative_Fluxes:figure-radiation-schematic}]{Figure \ref{\detokenize{tech_note/Radiative_Fluxes/CLM50_Tech_Note_Radiative_Fluxes:figure-radiation-schematic}}}, the upward longwave radiation from the surface to the atmosphere is
\phantomsection\label{\detokenize{tech_note/Radiative_Fluxes/CLM50_Tech_Note_Radiative_Fluxes:equation-4.11}}\begin{equation}\label{equation:tech_note/Radiative_Fluxes/CLM50_Tech_Note_Radiative_Fluxes:4.11}
\begin{split}\begin{array}{l} {L\, \uparrow =\delta _{veg} L_{vg} \, \uparrow +\left(1-\delta _{veg} \right)\left(1-\varepsilon _{g} \right)L_{atm} \, \downarrow +} \\ {\qquad \left(1-\delta _{veg} \right)\varepsilon _{g} \sigma \left(T_{g}^{n} \right)^{4} +4\varepsilon _{g} \sigma \left(T_{g}^{n} \right)^{3} \left(T_{g}^{n+1} -T_{g}^{n} \right)} \end{array}\end{split}
\end{equation}
where \(L_{vg} \, \uparrow\)  is the upward longwave radiation from
the vegetation/soil system for exposed leaf and stem area
\(L+S\ge 0.05\), \(\delta _{veg}\)  is a step function and is
zero for \(L+S<0.05\) and one otherwise, \(\varepsilon _{g}\)
is the ground emissivity, and \(T_{g}^{n+1}\)  and
\(T_{g}^{n}\)  are the snow/soil surface temperatures at the current
and previous time steps, respectively ({\hyperref[\detokenize{tech_note/Soil_Snow_Temperatures/CLM50_Tech_Note_Soil_Snow_Temperatures:rst-soil-and-snow-temperatures}]{\sphinxcrossref{\DUrole{std,std-ref}{Soil and Snow Temperatures}}}}).

For non-vegetated surfaces, the above equation reduces to
\phantomsection\label{\detokenize{tech_note/Radiative_Fluxes/CLM50_Tech_Note_Radiative_Fluxes:equation-4.12}}\begin{equation}\label{equation:tech_note/Radiative_Fluxes/CLM50_Tech_Note_Radiative_Fluxes:4.12}
\begin{split}L\, \uparrow =\left(1-\varepsilon _{g} \right)L_{atm} \, \downarrow +\varepsilon _{g} \sigma \left(T_{g}^{n} \right)^{4} +4\varepsilon _{g} \sigma \left(T_{g}^{n} \right)^{3} \left(T_{g}^{n+1} -T_{g}^{n} \right)\end{split}
\end{equation}
where the first term is the atmospheric longwave radiation reflected by
the ground, the second term is the longwave radiation emitted by the
ground, and the last term is the increase (decrease) in longwave
radiation emitted by the ground due to an increase (decrease) in ground
temperature.

For vegetated surfaces, the upward longwave radiation from the surface
reduces to
\phantomsection\label{\detokenize{tech_note/Radiative_Fluxes/CLM50_Tech_Note_Radiative_Fluxes:equation-4.13}}\begin{equation}\label{equation:tech_note/Radiative_Fluxes/CLM50_Tech_Note_Radiative_Fluxes:4.13}
\begin{split}L\, \uparrow =L_{vg} \, \uparrow +4\varepsilon _{g} \sigma \left(T_{g}^{n} \right)^{3} \left(T_{g}^{n+1} -T_{g}^{n} \right)\end{split}
\end{equation}
where
\phantomsection\label{\detokenize{tech_note/Radiative_Fluxes/CLM50_Tech_Note_Radiative_Fluxes:equation-4.14}}\begin{equation}\label{equation:tech_note/Radiative_Fluxes/CLM50_Tech_Note_Radiative_Fluxes:4.14}
\begin{split}\begin{array}{l} {L_{vg} \, \uparrow =\left(1-\varepsilon _{g} \right)\left(1-\varepsilon _{v} \right)\left(1-\varepsilon _{v} \right)L_{atm} \, \downarrow } \\ {\qquad \qquad +\varepsilon _{v} \left[1+\left(1-\varepsilon _{g} \right)\left(1-\varepsilon _{v} \right)\right]\sigma \left(T_{v}^{n} \right)^{3} \left[T_{v}^{n} +4\left(T_{v}^{n+1} -T_{v}^{n} \right)\right]} \\ {\qquad \qquad +\varepsilon _{g} \left(1-\varepsilon _{v} \right)\sigma \left(T_{g}^{n} \right)^{4} } \\ {\qquad =\left(1-\varepsilon _{g} \right)\left(1-\varepsilon _{v} \right)\left(1-\varepsilon _{v} \right)L_{atm} \, \downarrow } \\ {\qquad \qquad +\varepsilon _{v} \sigma \left(T_{v}^{n} \right)^{4} } \\ {\qquad \qquad +\varepsilon _{v} \left(1-\varepsilon _{g} \right)\left(1-\varepsilon _{v} \right)\sigma \left(T_{v}^{n} \right)^{4} } \\ {\qquad \qquad +4\varepsilon _{v} \sigma \left(T_{v}^{n} \right)^{3} \left(T_{v}^{n+1} -T_{v}^{n} \right)} \\ {\qquad \qquad +4\varepsilon _{v} \left(1-\varepsilon _{g} \right)\left(1-\varepsilon _{v} \right)\sigma \left(T_{v}^{n} \right)^{3} \left(T_{v}^{n+1} -T_{v}^{n} \right)} \\ {\qquad \qquad +\varepsilon _{g} \left(1-\varepsilon _{v} \right)\sigma \left(T_{g}^{n} \right)^{4} } \end{array}\end{split}
\end{equation}
where \(\varepsilon _{v}\)  is the vegetation emissivity and
\(T_{v}^{n+1}\)  and \(T_{v}^{n}\)  are the vegetation
temperatures at the current and previous time steps, respectively
({\hyperref[\detokenize{tech_note/Fluxes/CLM50_Tech_Note_Fluxes:rst-momentum-sensible-heat-and-latent-heat-fluxes}]{\sphinxcrossref{\DUrole{std,std-ref}{Momentum, Sensible Heat, and Latent Heat Fluxes}}}}).
The first term in the equation above is the atmospheric
longwave radiation that is transmitted through the canopy, reflected by
the ground, and transmitted through the canopy to the atmosphere. The
second term is the longwave radiation emitted by the canopy directly to
the atmosphere. The third term is the longwave radiation emitted
downward from the canopy, reflected by the ground, and transmitted
through the canopy to the atmosphere. The fourth term is the increase
(decrease) in longwave radiation due to an increase (decrease) in canopy
temperature that is emitted by the canopy directly to the atmosphere.
The fifth term is the increase (decrease) in longwave radiation due to
an increase (decrease) in canopy temperature that is emitted downward
from the canopy, reflected from the ground, and transmitted through the
canopy to the atmosphere. The last term is the longwave radiation
emitted by the ground and transmitted through the canopy to the
atmosphere.

The upward longwave radiation from the ground is
\phantomsection\label{\detokenize{tech_note/Radiative_Fluxes/CLM50_Tech_Note_Radiative_Fluxes:equation-4.15}}\begin{equation}\label{equation:tech_note/Radiative_Fluxes/CLM50_Tech_Note_Radiative_Fluxes:4.15}
\begin{split}L_{g} \, \uparrow =\left(1-\varepsilon _{g} \right)L_{v} \, \downarrow +\varepsilon _{g} \sigma \left(T_{g}^{n} \right)^{4}\end{split}
\end{equation}
where \(L_{v} \, \downarrow\)  is the downward longwave radiation
below the vegetation
\phantomsection\label{\detokenize{tech_note/Radiative_Fluxes/CLM50_Tech_Note_Radiative_Fluxes:equation-4.16}}\begin{equation}\label{equation:tech_note/Radiative_Fluxes/CLM50_Tech_Note_Radiative_Fluxes:4.16}
\begin{split}L_{v} \, \downarrow =\left(1-\varepsilon _{v} \right)L_{atm} \, \downarrow +\varepsilon _{v} \sigma \left(T_{v}^{n} \right)^{4} +4\varepsilon _{v} \sigma \left(T_{v}^{n} \right)^{3} \left(T_{v}^{n+1} -T_{v}^{n} \right).\end{split}
\end{equation}
The net longwave radiation flux for the ground is (positive toward the
atmosphere)
\phantomsection\label{\detokenize{tech_note/Radiative_Fluxes/CLM50_Tech_Note_Radiative_Fluxes:equation-4.17}}\begin{equation}\label{equation:tech_note/Radiative_Fluxes/CLM50_Tech_Note_Radiative_Fluxes:4.17}
\begin{split}\vec{L}_{g} =\varepsilon _{g} \sigma \left(T_{g}^{n} \right)^{4} -\delta _{veg} \varepsilon _{g} L_{v} \, \downarrow -\left(1-\delta _{veg} \right)\varepsilon _{g} L_{atm} \, \downarrow .\end{split}
\end{equation}
The above expression for \(\vec{L}_{g}\)  is the net longwave
radiation forcing that is used in the soil temperature calculation
({\hyperref[\detokenize{tech_note/Soil_Snow_Temperatures/CLM50_Tech_Note_Soil_Snow_Temperatures:rst-soil-and-snow-temperatures}]{\sphinxcrossref{\DUrole{std,std-ref}{Soil and Snow Temperatures}}}}). Once updated soil
temperatures have been obtained, the term
\(4\varepsilon _{g} \sigma \left(T_{g}^{n} \right)^{3} \left(T_{g}^{n+1} -T_{g}^{n} \right)\)
is added to \(\vec{L}_{g}\)  to calculate the ground heat flux
(section \hyperref[\detokenize{tech_note/Fluxes/CLM50_Tech_Note_Fluxes:update-of-ground-sensible-and-latent-heat-fluxes}]{\ref{\detokenize{tech_note/Fluxes/CLM50_Tech_Note_Fluxes:update-of-ground-sensible-and-latent-heat-fluxes}}})

The net longwave radiation flux for vegetation is (positive toward the
atmosphere)
\phantomsection\label{\detokenize{tech_note/Radiative_Fluxes/CLM50_Tech_Note_Radiative_Fluxes:equation-4.18}}\begin{equation}\label{equation:tech_note/Radiative_Fluxes/CLM50_Tech_Note_Radiative_Fluxes:4.18}
\begin{split}\vec{L}_{v} =\left[2-\varepsilon _{v} \left(1-\varepsilon _{g} \right)\right]\varepsilon _{v} \sigma \left(T_{v} \right)^{4} -\varepsilon _{v} \varepsilon _{g} \sigma \left(T_{g}^{n} \right)^{4} -\varepsilon _{v} \left[1+\left(1-\varepsilon _{g} \right)\left(1-\varepsilon _{v} \right)\right]L_{atm} \, \downarrow .\end{split}
\end{equation}
These equations assume that absorptivity equals emissivity. The
emissivity of the ground is
\phantomsection\label{\detokenize{tech_note/Radiative_Fluxes/CLM50_Tech_Note_Radiative_Fluxes:equation-4.19}}\begin{equation}\label{equation:tech_note/Radiative_Fluxes/CLM50_Tech_Note_Radiative_Fluxes:4.19}
\begin{split}\varepsilon _{g} =\varepsilon _{soi} \left(1-f_{sno} \right)+\varepsilon _{sno} f_{sno}\end{split}
\end{equation}
where \(\varepsilon _{soi} =0.96\) for soil, 0.97 for glacier, and
0.96 for wetland, \(\varepsilon _{sno} =0.97\), and \(f_{sno}\)
is the fraction of ground covered by snow
(section \hyperref[\detokenize{tech_note/Snow_Hydrology/CLM50_Tech_Note_Snow_Hydrology:snow-covered-area-fraction}]{\ref{\detokenize{tech_note/Snow_Hydrology/CLM50_Tech_Note_Snow_Hydrology:snow-covered-area-fraction}}}). The
vegetation emissivity is
\phantomsection\label{\detokenize{tech_note/Radiative_Fluxes/CLM50_Tech_Note_Radiative_Fluxes:equation-4.20}}\begin{equation}\label{equation:tech_note/Radiative_Fluxes/CLM50_Tech_Note_Radiative_Fluxes:4.20}
\begin{split}\varepsilon _{v} =1-e^{-{\left(L+S\right)\mathord{\left/ {\vphantom {\left(L+S\right) \bar{\mu }}} \right. \kern-\nulldelimiterspace} \bar{\mu }} }\end{split}
\end{equation}
where \(L\) and \(S\) are the leaf and stem area indices
(section \hyperref[\detokenize{tech_note/Ecosystem/CLM50_Tech_Note_Ecosystem:phenology-and-vegetation-burial-by-snow}]{\ref{\detokenize{tech_note/Ecosystem/CLM50_Tech_Note_Ecosystem:phenology-and-vegetation-burial-by-snow}}}) and
\(\bar{\mu }=1\) is the average inverse optical
depth for longwave radiation.


\section{Momentum, Sensible Heat, and Latent Heat Fluxes}
\label{\detokenize{tech_note/Fluxes/CLM50_Tech_Note_Fluxes:rst-momentum-sensible-heat-and-latent-heat-fluxes}}\label{\detokenize{tech_note/Fluxes/CLM50_Tech_Note_Fluxes:momentum-sensible-heat-and-latent-heat-fluxes}}\label{\detokenize{tech_note/Fluxes/CLM50_Tech_Note_Fluxes::doc}}
The zonal \(\tau _{x}\)  and meridional \(\tau _{y}\)  momentum
fluxes (kg m$^{\text{-1}}$ s$^{\text{-2}}$), sensible heat flux
\(H\) (W m$^{\text{-2}}$), and water vapor flux \(E\) (kg m$^{\text{-2}}$ s$^{\text{-1}}$) between the atmosphere at
reference height \(z_{atm,\, x}\)  (m) {[}where \(x\) is height
for wind (momentum) (\(m\)), temperature (sensible heat)
(\(h\)), and humidity (water vapor) (\(w\)); with zonal and
meridional winds \(u_{atm}\)  and \(v_{atm}\)  (m
s$^{\text{-1}}$), potential temperature \(\theta _{atm}\)  (K),
and specific humidity \(q_{atm}\)  (kg kg$^{\text{-1}}$){]} and the
surface {[}with \(u_{s}\) , \(v_{s}\) , \(\theta _{s}\) , and
\(q_{s}\) {]} are
\phantomsection\label{\detokenize{tech_note/Fluxes/CLM50_Tech_Note_Fluxes:equation-5.1}}\begin{equation}\label{equation:tech_note/Fluxes/CLM50_Tech_Note_Fluxes:5.1}
\begin{split}\tau _{x} =-\rho _{atm} \frac{\left(u_{atm} -u_{s} \right)}{r_{am} }\end{split}
\end{equation}\phantomsection\label{\detokenize{tech_note/Fluxes/CLM50_Tech_Note_Fluxes:equation-5.2}}\begin{equation}\label{equation:tech_note/Fluxes/CLM50_Tech_Note_Fluxes:5.2}
\begin{split}\tau _{y} =-\rho _{atm} \frac{\left(v_{atm} -v_{s} \right)}{r_{am} }\end{split}
\end{equation}\phantomsection\label{\detokenize{tech_note/Fluxes/CLM50_Tech_Note_Fluxes:equation-5.3}}\begin{equation}\label{equation:tech_note/Fluxes/CLM50_Tech_Note_Fluxes:5.3}
\begin{split}H=-\rho _{atm} C_{p} \frac{\left(\theta _{atm} -\theta _{s} \right)}{r_{ah} }\end{split}
\end{equation}\phantomsection\label{\detokenize{tech_note/Fluxes/CLM50_Tech_Note_Fluxes:equation-5.4}}\begin{equation}\label{equation:tech_note/Fluxes/CLM50_Tech_Note_Fluxes:5.4}
\begin{split}E=-\rho _{atm} \frac{\left(q_{atm} -q_{s} \right)}{r_{aw} } .\end{split}
\end{equation}
These fluxes are derived in the next section from Monin-Obukhov
similarity theory developed for the surface layer (i.e., the nearly
constant flux layer above the surface sublayer). In this derivation,
\(u_{s}\)  and \(v_{s}\)  are defined to equal zero at height
\(z_{0m} +d\) (the apparent sink for momentum) so that
\(r_{am}\)  is the aerodynamic resistance (s m$^{\text{-1}}$) for
momentum between the atmosphere at height \(z_{atm,\, m}\)  and the
surface at height \(z_{0m} +d\). Thus, the momentum fluxes become
\phantomsection\label{\detokenize{tech_note/Fluxes/CLM50_Tech_Note_Fluxes:equation-5.5}}\begin{equation}\label{equation:tech_note/Fluxes/CLM50_Tech_Note_Fluxes:5.5}
\begin{split}\tau _{x} =-\rho _{atm} \frac{u_{atm} }{r_{am} }\end{split}
\end{equation}\phantomsection\label{\detokenize{tech_note/Fluxes/CLM50_Tech_Note_Fluxes:equation-5.6}}\begin{equation}\label{equation:tech_note/Fluxes/CLM50_Tech_Note_Fluxes:5.6}
\begin{split}\tau _{y} =-\rho _{atm} \frac{v_{atm} }{r_{am} } .\end{split}
\end{equation}
Likewise, \(\theta _{s}\)  and \(q_{s}\)  are defined at
heights \(z_{0h} +d\) and \(z_{0w} +d\) (the apparent sinks for
heat and water vapor, respectively

\(r_{aw}\)  are the aerodynamic resistances (s m$^{\text{-1}}$)
to sensible heat and water vapor transfer between the atmosphere at
heights \(z_{atm,\, h}\)  and \(z_{atm,\, w}\)  and the surface
at heights \(z_{0h} +d\) and \(z_{0w} +d\), respectively. The
specific heat capacity of air \(C_{p}\)  (J kg$^{\text{-1}}$
K$^{\text{-1}}$) is a constant (\hyperref[\detokenize{tech_note/Ecosystem/CLM50_Tech_Note_Ecosystem:table-physical-constants}]{Table \ref{\detokenize{tech_note/Ecosystem/CLM50_Tech_Note_Ecosystem:table-physical-constants}}}). The atmospheric potential
temperature used here is
\phantomsection\label{\detokenize{tech_note/Fluxes/CLM50_Tech_Note_Fluxes:equation-5.7}}\begin{equation}\label{equation:tech_note/Fluxes/CLM50_Tech_Note_Fluxes:5.7}
\begin{split}\theta _{atm} =T_{atm} +\Gamma _{d} z_{atm,\, h}\end{split}
\end{equation}
where \(T_{atm}\)  is the air temperature (K) at height
\(z_{atm,\, h}\)  and \(\Gamma _{d} =0.0098\) K
m$^{\text{-1}}$ is the negative of the dry adiabatic lapse rate {[}this
expression is first-order equivalent to
\(\theta _{atm} =T_{atm} \left({P_{srf} \mathord{\left/ {\vphantom {P_{srf}  P_{atm} }} \right. \kern-\nulldelimiterspace} P_{atm} } \right)^{{R_{da} \mathord{\left/ {\vphantom {R_{da}  C_{p} }} \right. \kern-\nulldelimiterspace} C_{p} } }\)
({\hyperref[\detokenize{tech_note/References/CLM50_Tech_Note_References:stull1988}]{\sphinxcrossref{\DUrole{std,std-ref}{Stull 1988}}}}), where \(P_{srf}\)  is the surface pressure (Pa),
\(P_{atm}\)  is the atmospheric pressure (Pa), and \(R_{da}\)
is the gas constant for dry air (J kg$^{\text{-1}}$ K$^{\text{-1}}$) (\hyperref[\detokenize{tech_note/Ecosystem/CLM50_Tech_Note_Ecosystem:table-physical-constants}]{Table \ref{\detokenize{tech_note/Ecosystem/CLM50_Tech_Note_Ecosystem:table-physical-constants}}}){]}. By definition,
\(\theta _{s} =T_{s}\) . The density of moist air (kg m$^{\text{-3}}$) is
\phantomsection\label{\detokenize{tech_note/Fluxes/CLM50_Tech_Note_Fluxes:equation-5.8}}\begin{equation}\label{equation:tech_note/Fluxes/CLM50_Tech_Note_Fluxes:5.8}
\begin{split}\rho _{atm} =\frac{P_{atm} -0.378e_{atm} }{R_{da} T_{atm} }\end{split}
\end{equation}
where the atmospheric vapor pressure \(e_{atm}\)  (Pa) is derived
from the atmospheric specific humidity \(q_{atm}\)
\phantomsection\label{\detokenize{tech_note/Fluxes/CLM50_Tech_Note_Fluxes:equation-5.9}}\begin{equation}\label{equation:tech_note/Fluxes/CLM50_Tech_Note_Fluxes:5.9}
\begin{split}e_{atm} =\frac{q_{atm} P_{atm} }{0.622+0.378q_{atm} } .\end{split}
\end{equation}

\subsection{Monin-Obukhov Similarity Theory}
\label{\detokenize{tech_note/Fluxes/CLM50_Tech_Note_Fluxes:id1}}\label{\detokenize{tech_note/Fluxes/CLM50_Tech_Note_Fluxes:monin-obukhov-similarity-theory}}
The surface vertical kinematic fluxes of momentum
\(\overline{u'w'}\) and \(\overline{v'w'}\) (m$^{\text{2}}$ s$_{\text{-2}}$), sensible heat \(\overline{\theta 'w'}\)
(K m s $^{\text{-1}}$), and latent heat \(\overline{q'w'}\) (kg kg$^{\text{-1}}$ m s$^{\text{-1}}$), where \(u'\), \(v'\),
\(w'\), \(\theta '\), and \(q'\) are zonal horizontal wind,
meridional horizontal wind, vertical velocity, potential temperature,
and specific humidity turbulent fluctuations about the mean, are defined
from Monin-Obukhov similarity applied to the surface layer. This theory
states that when scaled appropriately, the dimensionless mean horizontal
wind speed, mean potential temperature, and mean specific humidity
profile gradients depend on unique functions of
\(\zeta =\frac{z-d}{L}\)  ({\hyperref[\detokenize{tech_note/References/CLM50_Tech_Note_References:zengetal1998}]{\sphinxcrossref{\DUrole{std,std-ref}{Zeng et al. 1998}}}}) as
\phantomsection\label{\detokenize{tech_note/Fluxes/CLM50_Tech_Note_Fluxes:equation-5.10}}\begin{equation}\label{equation:tech_note/Fluxes/CLM50_Tech_Note_Fluxes:5.10}
\begin{split}\frac{k\left(z-d\right)}{u_{*} } \frac{\partial \left|{\it u}\right|}{\partial z} =\phi _{m} \left(\zeta \right)\end{split}
\end{equation}\phantomsection\label{\detokenize{tech_note/Fluxes/CLM50_Tech_Note_Fluxes:equation-5.11}}\begin{equation}\label{equation:tech_note/Fluxes/CLM50_Tech_Note_Fluxes:5.11}
\begin{split}\frac{k\left(z-d\right)}{\theta _{*} } \frac{\partial \theta }{\partial z} =\phi _{h} \left(\zeta \right)\end{split}
\end{equation}\phantomsection\label{\detokenize{tech_note/Fluxes/CLM50_Tech_Note_Fluxes:equation-5.12}}\begin{equation}\label{equation:tech_note/Fluxes/CLM50_Tech_Note_Fluxes:5.12}
\begin{split}\frac{k\left(z-d\right)}{q_{*} } \frac{\partial q}{\partial z} =\phi _{w} \left(\zeta \right)\end{split}
\end{equation}
where \(z\) is height in the surface layer (m), \(d\) is the
displacement height (m), \(L\) is the Monin-Obukhov length scale (m)
that accounts for buoyancy effects resulting from vertical density
gradients (i.e., the atmospheric stability), k is the von Karman
constant (\hyperref[\detokenize{tech_note/Ecosystem/CLM50_Tech_Note_Ecosystem:table-physical-constants}]{Table \ref{\detokenize{tech_note/Ecosystem/CLM50_Tech_Note_Ecosystem:table-physical-constants}}}), and \(\left|{\it u}\right|\) is the
atmospheric wind speed (m s$^{\text{-1}}$). \(\phi _{m}\) ,
\(\phi _{h}\) , and \(\phi _{w}\)  are universal (over any
surface) similarity functions of \(\zeta\)  that relate the constant
fluxes of momentum, sensible heat, and latent heat to the mean profile
gradients of \(\left|{\it u}\right|\), \(\theta\) , and
\(q\) in the surface layer. In neutral conditions,
\(\phi _{m} =\phi _{h} =\phi _{w} =1\). The velocity (i.e., friction
velocity) \(u_{\*}\)  (m s$^{\text{-1}}$), temperature
\(\theta _{\*}\)  (K), and moisture \(q_{\*}\)  (kg kg$^{\text{-1}}$) scales are
\phantomsection\label{\detokenize{tech_note/Fluxes/CLM50_Tech_Note_Fluxes:equation-5.13}}\begin{equation}\label{equation:tech_note/Fluxes/CLM50_Tech_Note_Fluxes:5.13}
\begin{split}u_{*}^{2} =\sqrt{\left(\overline{u'w'}\right)^{2} +\left(\overline{v'w'}\right)^{2} } =\frac{\left|{\it \tau }\right|}{\rho _{atm} }\end{split}
\end{equation}\phantomsection\label{\detokenize{tech_note/Fluxes/CLM50_Tech_Note_Fluxes:equation-5.14}}\begin{equation}\label{equation:tech_note/Fluxes/CLM50_Tech_Note_Fluxes:5.14}
\begin{split}\theta _{*} u_{*} =-\overline{\theta 'w'}=-\frac{H}{\rho _{atm} C_{p} }\end{split}
\end{equation}\phantomsection\label{\detokenize{tech_note/Fluxes/CLM50_Tech_Note_Fluxes:equation-5.15}}\begin{equation}\label{equation:tech_note/Fluxes/CLM50_Tech_Note_Fluxes:5.15}
\begin{split}q_{*} u_{*} =-\overline{q'w'}=-\frac{E}{\rho _{atm} }\end{split}
\end{equation}
where \(\left|{\it \tau }\right|\) is the shearing stress (kg m$^{\text{-1}}$ s$^{\text{-2}}$), with zonal and meridional
components \(\overline{u'w'}=-\frac{\tau _{x} }{\rho _{atm} }\)  and
\(\overline{v'w'}=-\frac{\tau _{y} }{\rho _{atm} }\) , respectively,
\(H\) is the sensible heat flux (W m$^{\text{-2}}$) and \(E\)
is the water vapor flux (kg m$^{\text{-2}}$ s$^{\text{-1}}$).

The length scale \(L\) is the Monin-Obukhov length defined as
\phantomsection\label{\detokenize{tech_note/Fluxes/CLM50_Tech_Note_Fluxes:equation-5.16}}\begin{equation}\label{equation:tech_note/Fluxes/CLM50_Tech_Note_Fluxes:5.16}
\begin{split}L=-\frac{u_{*}^{3} }{k\left(\frac{g}{\overline{\theta _{v,\, atm} }} \right)\theta '_{v} w'} =\frac{u_{*}^{2} \overline{\theta _{v,\, atm} }}{kg\theta _{v*} }\end{split}
\end{equation}
where \(g\) is the acceleration of gravity (m s$^{\text{-2}}$)
(\hyperref[\detokenize{tech_note/Ecosystem/CLM50_Tech_Note_Ecosystem:table-physical-constants}]{Table \ref{\detokenize{tech_note/Ecosystem/CLM50_Tech_Note_Ecosystem:table-physical-constants}}}), and
\(\overline{\theta _{v,\, atm} }=\overline{\theta _{atm} }\left(1+0.61q_{atm} \right)\)
is the reference virtual potential temperature. \(L>0\) indicates
stable conditions. \(L<0\) indicates unstable conditions.
\(L=\infty\)  for neutral conditions. The temperature scale
\(\theta _{v*}\)  is defined as
\phantomsection\label{\detokenize{tech_note/Fluxes/CLM50_Tech_Note_Fluxes:equation-5.17}}\begin{equation}\label{equation:tech_note/Fluxes/CLM50_Tech_Note_Fluxes:5.17}
\begin{split}\theta _{v*} u_{*} =\left[\theta _{*} \left(1+0.61q_{atm} \right)+0.61\overline{\theta _{atm} }q_{*} \right]u_{*}\end{split}
\end{equation}
where \(\overline{\theta _{atm} }\) is the atmospheric potential
temperature.

Following {\hyperref[\detokenize{tech_note/References/CLM50_Tech_Note_References:panofskydutton1984}]{\sphinxcrossref{\DUrole{std,std-ref}{Panofsky and Dutton (1984)}}}}, the differential equations for
\(\phi _{m} \left(\zeta \right)\),
\(\phi _{h} \left(\zeta \right)\), and
\(\phi _{w} \left(\zeta \right)\) can be integrated formally without
commitment to their exact forms. Integration between two arbitrary
heights in the surface layer \(z_{2}\)  and \(z_{1}\)
(\(z_{2} >z_{1}\) ) with horizontal winds
\(\left|{\it u}\right|_{1}\)  and \(\left|{\it u}\right|_{2}\) ,
potential temperatures \(\theta _{1}\)  and \(\theta _{2}\) ,
and specific humidities \(q_{1}\)  and \(q_{2}\)  results in
\phantomsection\label{\detokenize{tech_note/Fluxes/CLM50_Tech_Note_Fluxes:equation-5.18}}\begin{equation}\label{equation:tech_note/Fluxes/CLM50_Tech_Note_Fluxes:5.18}
\begin{split}\left|{\it u}\right|_{2} -\left|{\it u}\right|_{1} =\frac{u_{*} }{k} \left[\ln \left(\frac{z_{2} -d}{z_{1} -d} \right)-\psi _{m} \left(\frac{z_{2} -d}{L} \right)+\psi _{m} \left(\frac{z_{1} -d}{L} \right)\right]\end{split}
\end{equation}\phantomsection\label{\detokenize{tech_note/Fluxes/CLM50_Tech_Note_Fluxes:equation-5.19}}\begin{equation}\label{equation:tech_note/Fluxes/CLM50_Tech_Note_Fluxes:5.19}
\begin{split}\theta _{2} -\theta _{1} =\frac{\theta _{*} }{k} \left[\ln \left(\frac{z_{2} -d}{z_{1} -d} \right)-\psi _{h} \left(\frac{z_{2} -d}{L} \right)+\psi _{h} \left(\frac{z_{1} -d}{L} \right)\right]\end{split}
\end{equation}\phantomsection\label{\detokenize{tech_note/Fluxes/CLM50_Tech_Note_Fluxes:equation-5.20}}\begin{equation}\label{equation:tech_note/Fluxes/CLM50_Tech_Note_Fluxes:5.20}
\begin{split}q_{2} -q_{1} =\frac{q_{*} }{k} \left[\ln \left(\frac{z_{2} -d}{z_{1} -d} \right)-\psi _{w} \left(\frac{z_{2} -d}{L} \right)+\psi _{w} \left(\frac{z_{1} -d}{L} \right)\right].\end{split}
\end{equation}
The functions \(\psi _{m} \left(\zeta \right)\),
\(\psi _{h} \left(\zeta \right)\), and
\(\psi _{w} \left(\zeta \right)\) are defined as
\phantomsection\label{\detokenize{tech_note/Fluxes/CLM50_Tech_Note_Fluxes:equation-5.21}}\begin{equation}\label{equation:tech_note/Fluxes/CLM50_Tech_Note_Fluxes:5.21}
\begin{split}\psi _{m} \left(\zeta \right)=\int _{{z_{0m} \mathord{\left/ {\vphantom {z_{0m}  L}} \right. \kern-\nulldelimiterspace} L} }^{\zeta }\frac{\left[1-\phi _{m} \left(x\right)\right]}{x} \, dx\end{split}
\end{equation}\phantomsection\label{\detokenize{tech_note/Fluxes/CLM50_Tech_Note_Fluxes:equation-5.22}}\begin{equation}\label{equation:tech_note/Fluxes/CLM50_Tech_Note_Fluxes:5.22}
\begin{split}\psi _{h} \left(\zeta \right)=\int _{{z_{0h} \mathord{\left/ {\vphantom {z_{0h}  L}} \right. \kern-\nulldelimiterspace} L} }^{\zeta }\frac{\left[1-\phi _{h} \left(x\right)\right]}{x} \, dx\end{split}
\end{equation}\phantomsection\label{\detokenize{tech_note/Fluxes/CLM50_Tech_Note_Fluxes:equation-5.23}}\begin{equation}\label{equation:tech_note/Fluxes/CLM50_Tech_Note_Fluxes:5.23}
\begin{split}\psi _{w} \left(\zeta \right)=\int _{{z_{0w} \mathord{\left/ {\vphantom {z_{0w}  L}} \right. \kern-\nulldelimiterspace} L} }^{\zeta }\frac{\left[1-\phi _{w} \left(x\right)\right]}{x} \, dx\end{split}
\end{equation}
where \(z_{0m}\) , \(z_{0h}\) , and \(z_{0w}\)  are the
roughness lengths (m) for momentum, sensible heat, and water vapor,
respectively.

Defining the surface values
\begin{equation*}
\begin{split}\left|{\it u}\right|_{1} =0{\rm \; at\; }z_{1} =z_{0m} +d,\end{split}
\end{equation*}\begin{equation*}
\begin{split}\theta _{1} =\theta _{s} {\rm \; at\; }z_{1} =z_{0h} +d,{\rm \; and}\end{split}
\end{equation*}\begin{equation*}
\begin{split}q_{1} =q_{s} {\rm \; at\; }z_{1} =z_{0w} +d,\end{split}
\end{equation*}
and the atmospheric values at \(z_{2} =z_{atm,\, x}\)
\phantomsection\label{\detokenize{tech_note/Fluxes/CLM50_Tech_Note_Fluxes:equation-5.24}}\begin{equation}\label{equation:tech_note/Fluxes/CLM50_Tech_Note_Fluxes:5.24}
\begin{split}\left|{\it u}\right|_{2} =V_{a} {\rm =\; }\sqrt{u_{atm}^{2} +v_{atm}^{2} +U_{c}^{2} } \ge 1,\end{split}
\end{equation}\begin{equation*}
\begin{split}\theta _{2} =\theta _{atm} {\rm ,\; and}\end{split}
\end{equation*}\begin{equation*}
\begin{split}q_{2} =q_{atm} {\rm ,\; }\end{split}
\end{equation*}
the integral forms of the flux-gradient relations are
\phantomsection\label{\detokenize{tech_note/Fluxes/CLM50_Tech_Note_Fluxes:equation-5.25}}\begin{equation}\label{equation:tech_note/Fluxes/CLM50_Tech_Note_Fluxes:5.25}
\begin{split}V_{a} =\frac{u_{*} }{k} \left[\ln \left(\frac{z_{atm,\, m} -d}{z_{0m} } \right)-\psi _{m} \left(\frac{z_{atm,\, m} -d}{L} \right)+\psi _{m} \left(\frac{z_{0m} }{L} \right)\right]\end{split}
\end{equation}\phantomsection\label{\detokenize{tech_note/Fluxes/CLM50_Tech_Note_Fluxes:equation-5.26}}\begin{equation}\label{equation:tech_note/Fluxes/CLM50_Tech_Note_Fluxes:5.26}
\begin{split}\theta _{atm} -\theta _{s} =\frac{\theta _{*} }{k} \left[\ln \left(\frac{z_{atm,\, h} -d}{z_{0h} } \right)-\psi _{h} \left(\frac{z_{atm,\, h} -d}{L} \right)+\psi _{h} \left(\frac{z_{0h} }{L} \right)\right]\end{split}
\end{equation}\phantomsection\label{\detokenize{tech_note/Fluxes/CLM50_Tech_Note_Fluxes:equation-5.27}}\begin{equation}\label{equation:tech_note/Fluxes/CLM50_Tech_Note_Fluxes:5.27}
\begin{split}q_{atm} -q_{s} =\frac{q_{*} }{k} \left[\ln \left(\frac{z_{atm,\, w} -d}{z_{0w} } \right)-\psi _{w} \left(\frac{z_{atm,\, w} -d}{L} \right)+\psi _{w} \left(\frac{z_{0w} }{L} \right)\right].\end{split}
\end{equation}
The constraint \(V_{a} \ge 1\) is required simply for numerical
reasons to prevent \(H\) and \(E\) from becoming small with
small wind speeds. The convective velocity \(U_{c}\)  accounts for
the contribution of large eddies in the convective boundary layer to
surface fluxes as follows
\phantomsection\label{\detokenize{tech_note/Fluxes/CLM50_Tech_Note_Fluxes:equation-5.28}}\begin{equation}\label{equation:tech_note/Fluxes/CLM50_Tech_Note_Fluxes:5.28}
\begin{split}U_{c} = \left\{
\begin{array}{ll}
0 & \qquad \zeta \ge {\rm 0} \quad {\rm (stable)} \\
\beta w_{*} & \qquad \zeta < 0 \quad {\rm (unstable)}
\end{array} \right\}\end{split}
\end{equation}
where \(w_{*}\)  is the convective velocity scale
\phantomsection\label{\detokenize{tech_note/Fluxes/CLM50_Tech_Note_Fluxes:equation-5.29}}\begin{equation}\label{equation:tech_note/Fluxes/CLM50_Tech_Note_Fluxes:5.29}
\begin{split}w_{*} =\left(\frac{-gu_{\*} \theta _{v*} z_{i} }{\overline{\theta _{v,\, atm} }} \right)^{{1\mathord{\left/ {\vphantom {1 3}} \right. \kern-\nulldelimiterspace} 3} } ,\end{split}
\end{equation}
\(z_{i} =1000\) is the convective boundary layer height (m), and \(\beta =1\).

The momentum flux gradient relations are ({\hyperref[\detokenize{tech_note/References/CLM50_Tech_Note_References:zengetal1998}]{\sphinxcrossref{\DUrole{std,std-ref}{Zeng et al. 1998}}}})
\phantomsection\label{\detokenize{tech_note/Fluxes/CLM50_Tech_Note_Fluxes:equation-5.30}}\begin{equation}\label{equation:tech_note/Fluxes/CLM50_Tech_Note_Fluxes:5.30}
\begin{split}\begin{array}{llr}
\phi _{m} \left(\zeta \right)=0.7k^{{2\mathord{\left/ {\vphantom {2 3}} \right. \kern-\nulldelimiterspace} 3} } \left(-\zeta \right)^{{1\mathord{\left/ {\vphantom {1 3}} \right. \kern-\nulldelimiterspace} 3} } & \qquad {\rm for\; }\zeta <-1.574 & \ {\rm \; (very\; unstable)} \\
\phi _{m} \left(\zeta \right)=\left(1-16\zeta \right)^{-{1\mathord{\left/ {\vphantom {1 4}} \right. \kern-\nulldelimiterspace} 4} } & \qquad {\rm for\; -1.574}\le \zeta <0 & \ {\rm \; (unstable)} \\
\phi _{m} \left(\zeta \right)=1+5\zeta & \qquad {\rm for\; }0\le \zeta \le 1& \ {\rm \; (stable)} \\
\phi _{m} \left(\zeta \right)=5+\zeta & \qquad {\rm for\; }\zeta  >1 & \ {\rm\; (very\; stable).}
\end{array}\end{split}
\end{equation}
The sensible and latent heat flux gradient relations are ({\hyperref[\detokenize{tech_note/References/CLM50_Tech_Note_References:zengetal1998}]{\sphinxcrossref{\DUrole{std,std-ref}{Zeng et al. 1998}}}})
\phantomsection\label{\detokenize{tech_note/Fluxes/CLM50_Tech_Note_Fluxes:equation-5.31}}\begin{equation}\label{equation:tech_note/Fluxes/CLM50_Tech_Note_Fluxes:5.31}
\begin{split}\begin{array}{llr}
\phi _{h} \left(\zeta \right)=\phi _{w} \left(\zeta \right)=0.9k^{{4\mathord{\left/ {\vphantom {4 3}} \right. \kern-\nulldelimiterspace} 3} } \left(-\zeta \right)^{{-1\mathord{\left/ {\vphantom {-1 3}} \right. \kern-\nulldelimiterspace} 3} } & \qquad {\rm for\; }\zeta <-0.465 & \ {\rm \; (very\; unstable)} \\
\phi _{h} \left(\zeta \right)=\phi _{w} \left(\zeta \right)=\left(1-16\zeta \right)^{-{1\mathord{\left/ {\vphantom {1 2}} \right. \kern-\nulldelimiterspace} 2} } & \qquad {\rm for\; -0.465}\le \zeta <0 & \ {\rm \; (unstable)} \\
\phi _{h} \left(\zeta \right)=\phi _{w} \left(\zeta \right)=1+5\zeta & \qquad {\rm for\; }0\le \zeta \le 1 & \ {\rm \; (stable)} \\
\phi _{h} \left(\zeta \right)=\phi _{w} \left(\zeta \right)=5+\zeta & \qquad {\rm for\; }\zeta  >1 & \ {\rm \; (very\; stable).}
\end{array}\end{split}
\end{equation}
To ensure continuous functions of
\(\phi _{m} \left(\zeta \right)\),
\(\phi _{h} \left(\zeta \right)\), and
\(\phi _{w} \left(\zeta \right)\), the simplest approach (i.e.,
without considering any transition regimes) is to match the relations
for very unstable and unstable conditions at \(\zeta _{m} =-1.574\)
for \(\phi _{m} \left(\zeta \right)\) and
\(\zeta _{h} =\zeta _{w} =-0.465\) for
\(\phi _{h} \left(\zeta \right)=\phi _{w} \left(\zeta \right)\)
({\hyperref[\detokenize{tech_note/References/CLM50_Tech_Note_References:zengetal1998}]{\sphinxcrossref{\DUrole{std,std-ref}{Zeng et al. 1998}}}}). The flux gradient relations can be integrated to
yield wind profiles for the following conditions:

Very unstable \(\left(\zeta <-1.574\right)\)
\phantomsection\label{\detokenize{tech_note/Fluxes/CLM50_Tech_Note_Fluxes:equation-5.32}}\begin{equation}\label{equation:tech_note/Fluxes/CLM50_Tech_Note_Fluxes:5.32}
\begin{split}V_{a} =\frac{u_{*} }{k} \left\{\left[\ln \frac{\zeta _{m} L}{z_{0m} } -\psi _{m} \left(\zeta _{m} \right)\right]+1.14\left[\left(-\zeta \right)^{{1\mathord{\left/ {\vphantom {1 3}} \right. \kern-\nulldelimiterspace} 3} } -\left(-\zeta _{m} \right)^{{1\mathord{\left/ {\vphantom {1 3}} \right. \kern-\nulldelimiterspace} 3} } \right]+\psi _{m} \left(\frac{z_{0m} }{L} \right)\right\}\end{split}
\end{equation}
Unstable \(\left(-1.574\le \zeta <0\right)\)
\phantomsection\label{\detokenize{tech_note/Fluxes/CLM50_Tech_Note_Fluxes:equation-5.33}}\begin{equation}\label{equation:tech_note/Fluxes/CLM50_Tech_Note_Fluxes:5.33}
\begin{split}V_{a} =\frac{u_{*} }{k} \left\{\left[\ln \frac{z_{atm,\, m} -d}{z_{0m} } -\psi _{m} \left(\zeta \right)\right]+\psi _{m} \left(\frac{z_{0m} }{L} \right)\right\}\end{split}
\end{equation}
Stable \(\left(0\le \zeta \le 1\right)\)
\phantomsection\label{\detokenize{tech_note/Fluxes/CLM50_Tech_Note_Fluxes:equation-5.34}}\begin{equation}\label{equation:tech_note/Fluxes/CLM50_Tech_Note_Fluxes:5.34}
\begin{split}V_{a} =\frac{u_{*} }{k} \left\{\left[\ln \frac{z_{atm,\, m} -d}{z_{0m} } +5\zeta \right]-5\frac{z_{0m} }{L} \right\}\end{split}
\end{equation}
Very stable \(\left(\zeta >1\right)\)
\phantomsection\label{\detokenize{tech_note/Fluxes/CLM50_Tech_Note_Fluxes:equation-5.35}}\begin{equation}\label{equation:tech_note/Fluxes/CLM50_Tech_Note_Fluxes:5.35}
\begin{split}V_{a} =\frac{u_{*} }{k} \left\{\left[\ln \frac{L}{z_{0m} } +5\right]+\left[5\ln \zeta +\zeta -1\right]-5\frac{z_{0m} }{L} \right\}\end{split}
\end{equation}
where
\phantomsection\label{\detokenize{tech_note/Fluxes/CLM50_Tech_Note_Fluxes:equation-5.36}}\begin{equation}\label{equation:tech_note/Fluxes/CLM50_Tech_Note_Fluxes:5.36}
\begin{split}\psi _{m} \left(\zeta \right)=2\ln \left(\frac{1+x}{2} \right)+\ln \left(\frac{1+x^{2} }{2} \right)-2\tan ^{-1} x+\frac{\pi }{2}\end{split}
\end{equation}
and

\(x=\left(1-16\zeta \right)^{{1\mathord{\left/ {\vphantom {1 4}} \right. \kern-\nulldelimiterspace} 4} }\) .

The potential temperature profiles are:

Very unstable \(\left(\zeta <-0.465\right)\)
\phantomsection\label{\detokenize{tech_note/Fluxes/CLM50_Tech_Note_Fluxes:equation-5.37}}\begin{equation}\label{equation:tech_note/Fluxes/CLM50_Tech_Note_Fluxes:5.37}
\begin{split}\theta _{atm} -\theta _{s} =\frac{\theta _{*} }{k} \left\{\left[\ln \frac{\zeta _{h} L}{z_{0h} } -\psi _{h} \left(\zeta _{h} \right)\right]+0.8\left[\left(-\zeta _{h} \right)^{{-1\mathord{\left/ {\vphantom {-1 3}} \right. \kern-\nulldelimiterspace} 3} } -\left(-\zeta \right)^{{-1\mathord{\left/ {\vphantom {-1 3}} \right. \kern-\nulldelimiterspace} 3} } \right]+\psi _{h} \left(\frac{z_{0h} }{L} \right)\right\}\end{split}
\end{equation}
Unstable \(\left(-0.465\le \zeta <0\right)\)
\phantomsection\label{\detokenize{tech_note/Fluxes/CLM50_Tech_Note_Fluxes:equation-5.38}}\begin{equation}\label{equation:tech_note/Fluxes/CLM50_Tech_Note_Fluxes:5.38}
\begin{split}\theta _{atm} -\theta _{s} =\frac{\theta _{*} }{k} \left\{\left[\ln \frac{z_{atm,\, h} -d}{z_{0h} } -\psi _{h} \left(\zeta \right)\right]+\psi _{h} \left(\frac{z_{0h} }{L} \right)\right\}\end{split}
\end{equation}
Stable \(\left(0\le \zeta \le 1\right)\)
\phantomsection\label{\detokenize{tech_note/Fluxes/CLM50_Tech_Note_Fluxes:equation-5.39}}\begin{equation}\label{equation:tech_note/Fluxes/CLM50_Tech_Note_Fluxes:5.39}
\begin{split}\theta _{atm} -\theta _{s} =\frac{\theta _{*} }{k} \left\{\left[\ln \frac{z_{atm,\, h} -d}{z_{0h} } +5\zeta \right]-5\frac{z_{0h} }{L} \right\}\end{split}
\end{equation}
Very stable \(\left(\zeta >1\right)\)
\phantomsection\label{\detokenize{tech_note/Fluxes/CLM50_Tech_Note_Fluxes:equation-5.40}}\begin{equation}\label{equation:tech_note/Fluxes/CLM50_Tech_Note_Fluxes:5.40}
\begin{split}\theta _{atm} -\theta _{s} =\frac{\theta _{*} }{k} \left\{\left[\ln \frac{L}{z_{0h} } +5\right]+\left[5\ln \zeta +\zeta -1\right]-5\frac{z_{0h} }{L} \right\}.\end{split}
\end{equation}
The specific humidity profiles are:

Very unstable \(\left(\zeta <-0.465\right)\)
\phantomsection\label{\detokenize{tech_note/Fluxes/CLM50_Tech_Note_Fluxes:equation-5.41}}\begin{equation}\label{equation:tech_note/Fluxes/CLM50_Tech_Note_Fluxes:5.41}
\begin{split}q_{atm} -q_{s} =\frac{q_{*} }{k} \left\{\left[\ln \frac{\zeta _{w} L}{z_{0w} } -\psi _{w} \left(\zeta _{w} \right)\right]+0.8\left[\left(-\zeta _{w} \right)^{{-1\mathord{\left/ {\vphantom {-1 3}} \right. \kern-\nulldelimiterspace} 3} } -\left(-\zeta \right)^{{-1\mathord{\left/ {\vphantom {-1 3}} \right. \kern-\nulldelimiterspace} 3} } \right]+\psi _{w} \left(\frac{z_{0w} }{L} \right)\right\}\end{split}
\end{equation}
Unstable \(\left(-0.465\le \zeta <0\right)\)
\phantomsection\label{\detokenize{tech_note/Fluxes/CLM50_Tech_Note_Fluxes:equation-5.42}}\begin{equation}\label{equation:tech_note/Fluxes/CLM50_Tech_Note_Fluxes:5.42}
\begin{split}q_{atm} -q_{s} =\frac{q_{*} }{k} \left\{\left[\ln \frac{z_{atm,\, w} -d}{z_{0w} } -\psi _{w} \left(\zeta \right)\right]+\psi _{w} \left(\frac{z_{0w} }{L} \right)\right\}\end{split}
\end{equation}
Stable \(\left(0\le \zeta \le 1\right)\)
\phantomsection\label{\detokenize{tech_note/Fluxes/CLM50_Tech_Note_Fluxes:equation-5.43}}\begin{equation}\label{equation:tech_note/Fluxes/CLM50_Tech_Note_Fluxes:5.43}
\begin{split}q_{atm} -q_{s} =\frac{q_{*} }{k} \left\{\left[\ln \frac{z_{atm,\, w} -d}{z_{0w} } +5\zeta \right]-5\frac{z_{0w} }{L} \right\}\end{split}
\end{equation}
Very stable \(\left(\zeta >1\right)\)
\phantomsection\label{\detokenize{tech_note/Fluxes/CLM50_Tech_Note_Fluxes:equation-5.44}}\begin{equation}\label{equation:tech_note/Fluxes/CLM50_Tech_Note_Fluxes:5.44}
\begin{split}q_{atm} -q_{s} =\frac{q_{*} }{k} \left\{\left[\ln \frac{L}{z_{0w} } +5\right]+\left[5\ln \zeta +\zeta -1\right]-5\frac{z_{0w} }{L} \right\}\end{split}
\end{equation}
where
\phantomsection\label{\detokenize{tech_note/Fluxes/CLM50_Tech_Note_Fluxes:equation-5.45}}\begin{equation}\label{equation:tech_note/Fluxes/CLM50_Tech_Note_Fluxes:5.45}
\begin{split}\psi _{h} \left(\zeta \right)=\psi _{w} \left(\zeta \right)=2\ln \left(\frac{1+x^{2} }{2} \right).\end{split}
\end{equation}
Using the definitions of \(u_{*}\) , \(\theta _{*}\) , and
\(q_{*}\) , an iterative solution of these equations can be used to
calculate the surface momentum, sensible heat, and water vapor flux
using atmospheric and surface values for \(\left|{\it u}\right|\),
\(\theta\) , and \(q\) except that \(L\) depends on
\(u_{*}\) , \(\theta _{*}\) , and \(q_{*}\) . However, the
bulk Richardson number
\phantomsection\label{\detokenize{tech_note/Fluxes/CLM50_Tech_Note_Fluxes:equation-5.46}}\begin{equation}\label{equation:tech_note/Fluxes/CLM50_Tech_Note_Fluxes:5.46}
\begin{split}R_{iB} =\frac{\theta _{v,\, atm} -\theta _{v,\, s} }{\overline{\theta _{v,\, atm} }} \frac{g\left(z_{atm,\, m} -d\right)}{V_{a}^{2} }\end{split}
\end{equation}
is related to \(\zeta\)  ({\hyperref[\detokenize{tech_note/References/CLM50_Tech_Note_References:arya2001}]{\sphinxcrossref{\DUrole{std,std-ref}{Arya 2001}}}}) as
\phantomsection\label{\detokenize{tech_note/Fluxes/CLM50_Tech_Note_Fluxes:equation-5.47}}\begin{equation}\label{equation:tech_note/Fluxes/CLM50_Tech_Note_Fluxes:5.47}
\begin{split}R_{iB} =\zeta \left[\ln \left(\frac{z_{atm,\, h} -d}{z_{0h} } \right)-\psi _{h} \left(\zeta \right)\right]\left[\ln \left(\frac{z_{atm,\, m} -d}{z_{0m} } \right)-\psi _{m} \left(\zeta \right)\right]^{-2} .\end{split}
\end{equation}
Using
\(\phi _{h} =\phi _{m}^{2} =\left(1-16\zeta \right)^{-{1\mathord{\left/ {\vphantom {1 2}} \right. \kern-\nulldelimiterspace} 2} }\)
for unstable conditions and \(\phi _{h} =\phi _{m} =1+5\zeta\)  for
stable conditions to determine \(\psi _{m} \left(\zeta \right)\) and
\(\psi _{h} \left(\zeta \right)\), the inverse relationship
\(\zeta =f\left(R_{iB} \right)\) can be solved to obtain a first
guess for \(\zeta\)  and thus \(L\) from
\phantomsection\label{\detokenize{tech_note/Fluxes/CLM50_Tech_Note_Fluxes:equation-5.48}}\begin{equation}\label{equation:tech_note/Fluxes/CLM50_Tech_Note_Fluxes:5.48}
\begin{split}\begin{array}{lcr}
\zeta =\frac{R_{iB} \ln \left(\frac{z_{atm,\, m} -d}{z_{0m} } \right)}{1-5\min \left(R_{iB} ,0.19\right)} & \qquad 0.01\le \zeta \le 2 & \qquad {\rm for\; }R_{iB} \ge 0 {\rm \; (neutral\; or\; stable)} \\
\zeta =R_{iB} \ln \left(\frac{z_{atm,\, m} -d}{z_{0m} } \right) & \qquad -100\le \zeta \le -0.01 & \qquad {\rm for\; }R_{iB} <0 \ {\rm \; (unstable)}
\end{array}.\end{split}
\end{equation}
Upon iteration (section \hyperref[\detokenize{tech_note/Fluxes/CLM50_Tech_Note_Fluxes:numerical-implementation}]{\ref{\detokenize{tech_note/Fluxes/CLM50_Tech_Note_Fluxes:numerical-implementation}}}), the following is used to determine
\(\zeta\)  and thus \(L\)
\phantomsection\label{\detokenize{tech_note/Fluxes/CLM50_Tech_Note_Fluxes:equation-5.49}}\begin{equation}\label{equation:tech_note/Fluxes/CLM50_Tech_Note_Fluxes:5.49}
\begin{split}\zeta =\frac{\left(z_{atm,\, m} -d\right)kg\theta _{v*} }{u_{*}^{2} \overline{\theta _{v,\, atm} }}\end{split}
\end{equation}
where
\begin{equation*}
\begin{split}\begin{array}{cr}
0.01\le \zeta \le 2 & \qquad {\rm for\; }\zeta \ge 0{\rm \; (neutral\; or\; stable)} \\
{\rm -100}\le \zeta \le {\rm -0.01} & \qquad {\rm for\; }\zeta <0{\rm \; (unstable)}
\end{array}.\end{split}
\end{equation*}
The difference in virtual potential air temperature between the
reference height and the surface is
\phantomsection\label{\detokenize{tech_note/Fluxes/CLM50_Tech_Note_Fluxes:equation-5.50}}\begin{equation}\label{equation:tech_note/Fluxes/CLM50_Tech_Note_Fluxes:5.50}
\begin{split}\theta _{v,\, atm} -\theta _{v,\, s} =\left(\theta _{atm} -\theta _{s} \right)\left(1+0.61q_{atm} \right)+0.61\overline{\theta _{atm} }\left(q_{atm} -q_{s} \right).\end{split}
\end{equation}
The momentum, sensible heat, and water vapor fluxes between the surface
and the atmosphere can also be written in the form
\phantomsection\label{\detokenize{tech_note/Fluxes/CLM50_Tech_Note_Fluxes:equation-5.51}}\begin{equation}\label{equation:tech_note/Fluxes/CLM50_Tech_Note_Fluxes:5.51}
\begin{split}\tau _{x} =-\rho _{atm} \frac{\left(u_{atm} -u_{s} \right)}{r_{am} }\end{split}
\end{equation}\phantomsection\label{\detokenize{tech_note/Fluxes/CLM50_Tech_Note_Fluxes:equation-5.52}}\begin{equation}\label{equation:tech_note/Fluxes/CLM50_Tech_Note_Fluxes:5.52}
\begin{split}\tau _{y} =-\rho _{atm} \frac{\left(v_{atm} -v_{s} \right)}{r_{am} }\end{split}
\end{equation}\phantomsection\label{\detokenize{tech_note/Fluxes/CLM50_Tech_Note_Fluxes:equation-5.53}}\begin{equation}\label{equation:tech_note/Fluxes/CLM50_Tech_Note_Fluxes:5.53}
\begin{split}H=-\rho _{atm} C_{p} \frac{\left(\theta _{atm} -\theta _{s} \right)}{r_{ah} }\end{split}
\end{equation}\phantomsection\label{\detokenize{tech_note/Fluxes/CLM50_Tech_Note_Fluxes:equation-5.54}}\begin{equation}\label{equation:tech_note/Fluxes/CLM50_Tech_Note_Fluxes:5.54}
\begin{split}E=-\rho _{atm} \frac{\left(q_{atm} -q_{s} \right)}{r_{aw} }\end{split}
\end{equation}
where the aerodynamic resistances (s m$^{\text{-1}}$) are
\phantomsection\label{\detokenize{tech_note/Fluxes/CLM50_Tech_Note_Fluxes:equation-5.55}}\begin{equation}\label{equation:tech_note/Fluxes/CLM50_Tech_Note_Fluxes:5.55}
\begin{split}r_{am} =\frac{V_{a} }{u_{*}^{2} } =\frac{1}{k^{2} V_{a} } \left[\ln \left(\frac{z_{atm,\, m} -d}{z_{0m} } \right)-\psi _{m} \left(\frac{z_{atm,\, m} -d}{L} \right)+\psi _{m} \left(\frac{z_{0m} }{L} \right)\right]^{2}\end{split}
\end{equation}\phantomsection\label{\detokenize{tech_note/Fluxes/CLM50_Tech_Note_Fluxes:equation-5.56}}\begin{equation}\label{equation:tech_note/Fluxes/CLM50_Tech_Note_Fluxes:5.56}
\begin{split}\begin{array}{l} {r_{ah} =\frac{\theta _{atm} -\theta _{s} }{\theta _{*} u_{*} } =\frac{1}{k^{2} V_{a} } \left[\ln \left(\frac{z_{atm,\, m} -d}{z_{0m} } \right)-\psi _{m} \left(\frac{z_{atm,\, m} -d}{L} \right)+\psi _{m} \left(\frac{z_{0m} }{L} \right)\right]} \\ {\qquad \left[\ln \left(\frac{z_{atm,\, h} -d}{z_{0h} } \right)-\psi _{h} \left(\frac{z_{atm,\, h} -d}{L} \right)+\psi _{h} \left(\frac{z_{0h} }{L} \right)\right]} \end{array}\end{split}
\end{equation}\phantomsection\label{\detokenize{tech_note/Fluxes/CLM50_Tech_Note_Fluxes:equation-5.57}}\begin{equation}\label{equation:tech_note/Fluxes/CLM50_Tech_Note_Fluxes:5.57}
\begin{split}\begin{array}{l} {r_{aw} =\frac{q_{atm} -q_{s} }{q_{*} u_{*} } =\frac{1}{k^{2} V_{a} } \left[\ln \left(\frac{z_{atm,\, m} -d}{z_{0m} } \right)-\psi _{m} \left(\frac{z_{atm,\, m} -d}{L} \right)+\psi _{m} \left(\frac{z_{0m} }{L} \right)\right]} \\ {\qquad \left[\ln \left(\frac{z_{atm,\, {\it w}} -d}{z_{0w} } \right)-\psi _{w} \left(\frac{z_{atm,\, w} -d}{L} \right)+\psi _{w} \left(\frac{z_{0w} }{L} \right)\right]} \end{array}.\end{split}
\end{equation}
A 2-m height “screen” temperature is useful for comparison with
observations
\phantomsection\label{\detokenize{tech_note/Fluxes/CLM50_Tech_Note_Fluxes:equation-5.58}}\begin{equation}\label{equation:tech_note/Fluxes/CLM50_Tech_Note_Fluxes:5.58}
\begin{split}T_{2m} =\theta _{s} +\frac{\theta _{*} }{k} \left[\ln \left(\frac{2+z_{0h} }{z_{0h} } \right)-\psi _{h} \left(\frac{2+z_{0h} }{L} \right)+\psi _{h} \left(\frac{z_{0h} }{L} \right)\right]\end{split}
\end{equation}
where for convenience, “2-m” is defined as 2 m above the apparent sink
for sensible heat (\(z_{0h} +d\)). Similarly, a 2-m height specific
humidity is defined as
\phantomsection\label{\detokenize{tech_note/Fluxes/CLM50_Tech_Note_Fluxes:equation-5.59}}\begin{equation}\label{equation:tech_note/Fluxes/CLM50_Tech_Note_Fluxes:5.59}
\begin{split}q_{2m} =q_{s} +\frac{q_{*} }{k} \left[\ln \left(\frac{2+z_{0w} }{z_{0w} } \right)-\psi _{w} \left(\frac{2+z_{0w} }{L} \right)+\psi _{w} \left(\frac{z_{0w} }{L} \right)\right].\end{split}
\end{equation}
Relative humidity is
\phantomsection\label{\detokenize{tech_note/Fluxes/CLM50_Tech_Note_Fluxes:equation-5.60}}\begin{equation}\label{equation:tech_note/Fluxes/CLM50_Tech_Note_Fluxes:5.60}
\begin{split}RH_{2m} =\min \left(100,\, \frac{q_{2m} }{q_{sat}^{T_{2m} } } \times 100\right)\end{split}
\end{equation}
where \(q_{sat}^{T_{2m} }\)  is the saturated specific humidity at
the 2-m temperature \(T_{2m}\)  (section \hyperref[\detokenize{tech_note/Fluxes/CLM50_Tech_Note_Fluxes:saturation-vapor-pressure}]{\ref{\detokenize{tech_note/Fluxes/CLM50_Tech_Note_Fluxes:saturation-vapor-pressure}}}).

A 10-m wind speed is calculated as (note that this is not consistent
with the 10-m wind speed calculated for the dust model as described in
Chapter \hyperref[\detokenize{tech_note/Dust/CLM50_Tech_Note_Dust:rst-dust-model}]{\ref{\detokenize{tech_note/Dust/CLM50_Tech_Note_Dust:rst-dust-model}}})
\phantomsection\label{\detokenize{tech_note/Fluxes/CLM50_Tech_Note_Fluxes:equation-5.61}}\begin{equation}\label{equation:tech_note/Fluxes/CLM50_Tech_Note_Fluxes:5.61}
\begin{split}u_{10m} =\left\{\begin{array}{l} {V_{a} \qquad z_{atm,\, m} \le 10} \\ {V_{a} -\frac{u_{*} }{k} \left[\ln \left(\frac{z_{atm,\, m} -d}{10+z_{0m} } \right)-\psi _{m} \left(\frac{z_{atm,\, m} -d}{L} \right)+\psi _{m} \left(\frac{10+z_{0m} }{L} \right)\right]\qquad z_{atm,\, m} >10} \end{array}\right\}\end{split}
\end{equation}

\subsection{Sensible and Latent Heat Fluxes for Non-Vegetated Surfaces}
\label{\detokenize{tech_note/Fluxes/CLM50_Tech_Note_Fluxes:sensible-and-latent-heat-fluxes-for-non-vegetated-surfaces}}\label{\detokenize{tech_note/Fluxes/CLM50_Tech_Note_Fluxes:id2}}
Surfaces are considered non-vegetated for the surface flux calculations
if leaf plus stem area index \(L+S<0.05\) (section
\hyperref[\detokenize{tech_note/Ecosystem/CLM50_Tech_Note_Ecosystem:phenology-and-vegetation-burial-by-snow}]{\ref{\detokenize{tech_note/Ecosystem/CLM50_Tech_Note_Ecosystem:phenology-and-vegetation-burial-by-snow}}}). By
definition, this includes bare soil, wetlands, and glaciers. The
solution for lakes is described in Chapter \hyperref[\detokenize{tech_note/Lake/CLM50_Tech_Note_Lake:rst-lake-model}]{\ref{\detokenize{tech_note/Lake/CLM50_Tech_Note_Lake:rst-lake-model}}}. For these surfaces, the
surface may be exposed to the atmosphere, snow covered, and/or surface
water covered, so that the sensible heat flux \(H_{g}\)  (W
m$^{\text{-2}}$) is, with reference to \hyperref[\detokenize{tech_note/Fluxes/CLM50_Tech_Note_Fluxes:figure-schematic-diagram-of-sensible-heat-fluxes}]{Figure \ref{\detokenize{tech_note/Fluxes/CLM50_Tech_Note_Fluxes:figure-schematic-diagram-of-sensible-heat-fluxes}}},
\phantomsection\label{\detokenize{tech_note/Fluxes/CLM50_Tech_Note_Fluxes:equation-5.62}}\begin{equation}\label{equation:tech_note/Fluxes/CLM50_Tech_Note_Fluxes:5.62}
\begin{split}H_{g} =\left(1-f_{sno} -f_{h2osfc} \right)H_{soil} +f_{sno} H_{snow} +f_{h2osfc} H_{h2osfc}\end{split}
\end{equation}
where \(\left(1-f_{sno} -f_{h2osfc} \right)\), \(f_{sno}\) , and
\(f_{h2osfc}\)  are the exposed, snow covered, and surface water
covered fractions of the grid cell. The individual fluxes based on the
temperatures of the soil \(T_{1}\) , snow \(T_{snl+1}\) , and
surface water \(T_{h2osfc}\)  are
\phantomsection\label{\detokenize{tech_note/Fluxes/CLM50_Tech_Note_Fluxes:equation-5.63}}\begin{equation}\label{equation:tech_note/Fluxes/CLM50_Tech_Note_Fluxes:5.63}
\begin{split}H_{soil} =-\rho _{atm} C_{p} \frac{\left(\theta _{atm} -T_{1} \right)}{r_{ah} }\end{split}
\end{equation}\phantomsection\label{\detokenize{tech_note/Fluxes/CLM50_Tech_Note_Fluxes:equation-5.64}}\begin{equation}\label{equation:tech_note/Fluxes/CLM50_Tech_Note_Fluxes:5.64}
\begin{split}H_{sno} =-\rho _{atm} C_{p} \frac{\left(\theta _{atm} -T_{snl+1} \right)}{r_{ah} }\end{split}
\end{equation}\phantomsection\label{\detokenize{tech_note/Fluxes/CLM50_Tech_Note_Fluxes:equation-5.65}}\begin{equation}\label{equation:tech_note/Fluxes/CLM50_Tech_Note_Fluxes:5.65}
\begin{split}H_{h2osfc} =-\rho _{atm} C_{p} \frac{\left(\theta _{atm} -T_{h2osfc} \right)}{r_{ah} }\end{split}
\end{equation}
where \(\rho _{atm}\)  is the density of atmospheric air (kg m$^{\text{-3}}$), \(C_{p}\)  is the specific heat capacity of air
(J kg$^{\text{-1}}$ K$^{\text{-1}}$) (\hyperref[\detokenize{tech_note/Ecosystem/CLM50_Tech_Note_Ecosystem:table-physical-constants}]{Table \ref{\detokenize{tech_note/Ecosystem/CLM50_Tech_Note_Ecosystem:table-physical-constants}}}),
\(\theta _{atm}\)  is the atmospheric potential temperature (K), and
\(r_{ah}\)  is the aerodynamic resistance to sensible heat transfer
(s m$^{\text{-1}}$).

The water vapor flux \(E_{g}\)  (kg m$^{\text{-2}}$ s$^{\text{-1}}$) is, with reference to
\hyperref[\detokenize{tech_note/Fluxes/CLM50_Tech_Note_Fluxes:figure-schematic-diagram-of-latent-heat-fluxes}]{Figure \ref{\detokenize{tech_note/Fluxes/CLM50_Tech_Note_Fluxes:figure-schematic-diagram-of-latent-heat-fluxes}}},
\phantomsection\label{\detokenize{tech_note/Fluxes/CLM50_Tech_Note_Fluxes:equation-5.66}}\begin{equation}\label{equation:tech_note/Fluxes/CLM50_Tech_Note_Fluxes:5.66}
\begin{split}E_{g} =\left(1-f_{sno} -f_{h2osfc} \right)E_{soil} +f_{sno} E_{snow} +f_{h2osfc} E_{h2osfc}\end{split}
\end{equation}\phantomsection\label{\detokenize{tech_note/Fluxes/CLM50_Tech_Note_Fluxes:equation-5.67}}\begin{equation}\label{equation:tech_note/Fluxes/CLM50_Tech_Note_Fluxes:5.67}
\begin{split}E_{soil} =-\frac{\rho _{atm} \left(q_{atm} -q_{soil} \right)}{r_{aw} + r_{soil}}\end{split}
\end{equation}\phantomsection\label{\detokenize{tech_note/Fluxes/CLM50_Tech_Note_Fluxes:equation-5.68}}\begin{equation}\label{equation:tech_note/Fluxes/CLM50_Tech_Note_Fluxes:5.68}
\begin{split}E_{sno} =-\frac{\rho _{atm} \left(q_{atm} -q_{sno} \right)}{r_{aw} }\end{split}
\end{equation}\phantomsection\label{\detokenize{tech_note/Fluxes/CLM50_Tech_Note_Fluxes:equation-5.69}}\begin{equation}\label{equation:tech_note/Fluxes/CLM50_Tech_Note_Fluxes:5.69}
\begin{split}E_{h2osfc} =-\frac{\rho _{atm} \left(q_{atm} -q_{h2osfc} \right)}{r_{aw} }\end{split}
\end{equation}
where \(q_{atm}\)  is the atmospheric specific humidity (kg kg$^{\text{-1}}$), \(q_{soil}\) , \(q_{sno}\) ,
and \(q_{h2osfc}\)  are the specific humidities (kg kg$^{\text{-1}}$) of the soil, snow, and surface water, respectively,
\(r_{aw}\)  is the aerodynamic resistance to water vapor transfer (s m$^{\text{-1}}$), and  \(r _{soi}\)  is the soil
resistance to water vapor transfer (s m$^{\text{-1}}$). The specific humidities of the snow \(q_{sno}\)  and surface water
\(q_{h2osfc}\)  are assumed to be at the saturation specific humidity of their respective temperatures
\phantomsection\label{\detokenize{tech_note/Fluxes/CLM50_Tech_Note_Fluxes:equation-5.70}}\begin{equation}\label{equation:tech_note/Fluxes/CLM50_Tech_Note_Fluxes:5.70}
\begin{split}q_{sno} =q_{sat}^{T_{snl+1} }\end{split}
\end{equation}\phantomsection\label{\detokenize{tech_note/Fluxes/CLM50_Tech_Note_Fluxes:equation-5.71}}\begin{equation}\label{equation:tech_note/Fluxes/CLM50_Tech_Note_Fluxes:5.71}
\begin{split}q_{h2osfc} =q_{sat}^{T_{h2osfc} }\end{split}
\end{equation}
The specific humidity of the soil surface \(q_{soil}\)  is assumed
to be proportional to the saturation specific humidity
\phantomsection\label{\detokenize{tech_note/Fluxes/CLM50_Tech_Note_Fluxes:equation-5.72}}\begin{equation}\label{equation:tech_note/Fluxes/CLM50_Tech_Note_Fluxes:5.72}
\begin{split}q_{soil} =\alpha _{soil} q_{sat}^{T_{1} }\end{split}
\end{equation}
where \(q_{sat}^{T_{1} }\)  is the saturated specific humidity at
the soil surface temperature \(T_{1}\)  (section \hyperref[\detokenize{tech_note/Fluxes/CLM50_Tech_Note_Fluxes:saturation-vapor-pressure}]{\ref{\detokenize{tech_note/Fluxes/CLM50_Tech_Note_Fluxes:saturation-vapor-pressure}}}). The factor
\(\alpha _{soil}\)  is a function of the surface soil water matric
potential \(\psi\)  as in {\hyperref[\detokenize{tech_note/References/CLM50_Tech_Note_References:philip1957}]{\sphinxcrossref{\DUrole{std,std-ref}{Philip (1957)}}}}
\phantomsection\label{\detokenize{tech_note/Fluxes/CLM50_Tech_Note_Fluxes:equation-5.73}}\begin{equation}\label{equation:tech_note/Fluxes/CLM50_Tech_Note_Fluxes:5.73}
\begin{split}\alpha _{soil} =\exp \left(\frac{\psi _{1} g}{1\times 10^{3} R_{wv} T_{1} } \right)\end{split}
\end{equation}
where \(R_{wv}\)  is the gas constant for water vapor (J kg$^{\text{-1}}$ K$^{\text{-1}}$) (\hyperref[\detokenize{tech_note/Ecosystem/CLM50_Tech_Note_Ecosystem:table-physical-constants}]{Table \ref{\detokenize{tech_note/Ecosystem/CLM50_Tech_Note_Ecosystem:table-physical-constants}}}), \(g\) is the
gravitational acceleration (m s$^{\text{-2}}$) (\hyperref[\detokenize{tech_note/Ecosystem/CLM50_Tech_Note_Ecosystem:table-physical-constants}]{Table \ref{\detokenize{tech_note/Ecosystem/CLM50_Tech_Note_Ecosystem:table-physical-constants}}}), and
\(\psi _{1}\)  is the soil water matric potential of the top soil
layer (mm). The soil water matric potential \(\psi _{1}\)  is
\phantomsection\label{\detokenize{tech_note/Fluxes/CLM50_Tech_Note_Fluxes:equation-5.74}}\begin{equation}\label{equation:tech_note/Fluxes/CLM50_Tech_Note_Fluxes:5.74}
\begin{split}\psi _{1} =\psi _{sat,\, 1} s_{1}^{-B_{1} } \ge -1\times 10^{8}\end{split}
\end{equation}
where \(\psi _{sat,\, 1}\)  is the saturated matric potential (mm)
(section \hyperref[\detokenize{tech_note/Hydrology/CLM50_Tech_Note_Hydrology:hydraulic-properties}]{\ref{\detokenize{tech_note/Hydrology/CLM50_Tech_Note_Hydrology:hydraulic-properties}}}),
\(B_{1}\)  is the {\hyperref[\detokenize{tech_note/References/CLM50_Tech_Note_References:clapphornberger1978}]{\sphinxcrossref{\DUrole{std,std-ref}{Clapp and Hornberger (1978)}}}}
parameter (section \hyperref[\detokenize{tech_note/Hydrology/CLM50_Tech_Note_Hydrology:hydraulic-properties}]{\ref{\detokenize{tech_note/Hydrology/CLM50_Tech_Note_Hydrology:hydraulic-properties}}}),
and \(s_{1}\)  is the wetness of the top soil layer with respect to saturation.
The surface wetness \(s_{1}\)  is a function of the liquid water and ice content
\phantomsection\label{\detokenize{tech_note/Fluxes/CLM50_Tech_Note_Fluxes:equation-5.75}}\begin{equation}\label{equation:tech_note/Fluxes/CLM50_Tech_Note_Fluxes:5.75}
\begin{split}s_{1} =\frac{1}{\Delta z_{1} \theta _{sat,\, 1} } \left[\frac{w_{liq,\, 1} }{\rho _{liq} } +\frac{w_{ice,\, 1} }{\rho _{ice} } \right]\qquad 0.01\le s_{1} \le 1.0\end{split}
\end{equation}
where \(\Delta z_{1}\)  is the thickness of the top soil layer (m),
\(\rho _{liq}\)  and \(\rho _{ice}\)  are the density of liquid
water and ice (kg m$^{\text{-3}}$) (\hyperref[\detokenize{tech_note/Ecosystem/CLM50_Tech_Note_Ecosystem:table-physical-constants}]{Table \ref{\detokenize{tech_note/Ecosystem/CLM50_Tech_Note_Ecosystem:table-physical-constants}}}), \(w_{liq,\, 1}\)
and \(w_{ice,\, 1}\)  are the mass of liquid water and ice of the
top soil layer (kg m$^{\text{-2}}$) (Chapter \hyperref[\detokenize{tech_note/Hydrology/CLM50_Tech_Note_Hydrology:rst-hydrology}]{\ref{\detokenize{tech_note/Hydrology/CLM50_Tech_Note_Hydrology:rst-hydrology}}}), and
\(\theta _{sat,\, 1}\)  is the saturated volumetric water content
(i.e., porosity) of the top soil layer (mm$^{\text{3}}$ mm$^{\text{-3}}$) (section \hyperref[\detokenize{tech_note/Hydrology/CLM50_Tech_Note_Hydrology:hydraulic-properties}]{\ref{\detokenize{tech_note/Hydrology/CLM50_Tech_Note_Hydrology:hydraulic-properties}}}). If
\(q_{sat}^{T_{1} } >q_{atm}\)  and \(q_{atm} >q_{soil}\) , then
\(q_{soil} =q_{atm}\)  and \(\frac{dq_{soil} }{dT} =0\). This
prevents large increases (decreases) in \(q_{soil}\)  for small
increases (decreases) in soil moisture in very dry soils.

The resistance to water vapor transfer occurring within the soil matrix
\(r_{soil}\) (s m$^{\text{-1}}$) is
\phantomsection\label{\detokenize{tech_note/Fluxes/CLM50_Tech_Note_Fluxes:equation-5.76}}\begin{equation}\label{equation:tech_note/Fluxes/CLM50_Tech_Note_Fluxes:5.76}
\begin{split}r_{soil} = \frac{DSL}{D_{v} \tau}\end{split}
\end{equation}
where \(DSL\) is the thickness of the dry surface layer (m), \(D_{v}\)
is the molecular diffusivity of water vapor in air (m$^{\text{2}}$ s$^{\text{-2}}$)
and \(\tau\) (\sphinxstyleemphasis{unitless}) describes the tortuosity of the vapor flow paths through
the soil matrix ({\hyperref[\detokenize{tech_note/References/CLM50_Tech_Note_References:swensonlawrence2014}]{\sphinxcrossref{\DUrole{std,std-ref}{Swenson and Lawrence 2014}}}}).

The thickness of the dry surface layer is given by
\phantomsection\label{\detokenize{tech_note/Fluxes/CLM50_Tech_Note_Fluxes:equation-5.77}}\begin{equation}\label{equation:tech_note/Fluxes/CLM50_Tech_Note_Fluxes:5.77}
\begin{split}DSL =
\begin{array}{lr}
D_{max} \ \frac{\left( \theta_{init} - \theta_{1}\right)}
{\left(\theta_{init} - \theta_{air}\right)} & \qquad \theta_{1} < \theta_{init} \\
0 &  \qquad \theta_{1} \ge \theta_{init}
\end{array}\end{split}
\end{equation}
where \(D_{max}\) is a parameter specifying the length scale
of the maximum DSL thickness (default value = 15 mm),
\(\theta_{init}\) (mm$^{\text{3}}$ mm$^{\text{-3}}$) is the moisture value
at which the DSL initiates, \(\theta_{1}\) (mm$^{\text{3}}$ mm$^{\text{-3}}$)
is the moisture value of the top model soil layer, and
\(\theta_{air}\) (mm$^{\text{3}}$ mm$^{\text{-3}}$) is the ‘air dry’ soil
moisture value ({\hyperref[\detokenize{tech_note/References/CLM50_Tech_Note_References:dingman2002}]{\sphinxcrossref{\DUrole{std,std-ref}{Dingman 2002}}}}):
\phantomsection\label{\detokenize{tech_note/Fluxes/CLM50_Tech_Note_Fluxes:equation-5.78}}\begin{equation}\label{equation:tech_note/Fluxes/CLM50_Tech_Note_Fluxes:5.78}
\begin{split}\theta_{air} = \Phi \left( \frac{\Psi_{sat}}{\Psi_{air}} \right)^{\frac{1}{B_{1}}} \ .\end{split}
\end{equation}
where \(\Phi\) is the porosity (mm$^{\text{3}}$ mm$^{\text{-3}}$),
\(\Psi_{sat}\) is the saturated soil matric potential (mm),
\(\Psi_{air} = 10^{7}\) mm is the air dry matric potential, and
\(B_{1}\) is a function of soil texture (section
\hyperref[\detokenize{tech_note/Hydrology/CLM50_Tech_Note_Hydrology:hydraulic-properties}]{\ref{\detokenize{tech_note/Hydrology/CLM50_Tech_Note_Hydrology:hydraulic-properties}}}).

The soil tortuosity is
\phantomsection\label{\detokenize{tech_note/Fluxes/CLM50_Tech_Note_Fluxes:equation-5.79}}\begin{equation}\label{equation:tech_note/Fluxes/CLM50_Tech_Note_Fluxes:5.79}
\begin{split}\tau = \Phi^{2}_{air}\left(\frac{\Phi_{air}}{\Phi}\right)^{\frac{3}{B_{1}}}\end{split}
\end{equation}
where \(\Phi_{air}\) (mm$^{\text{3}}$ mm$^{\text{-3}}$) is the air filled pore space
\phantomsection\label{\detokenize{tech_note/Fluxes/CLM50_Tech_Note_Fluxes:equation-5.80}}\begin{equation}\label{equation:tech_note/Fluxes/CLM50_Tech_Note_Fluxes:5.80}
\begin{split}\Phi_{air} = \Phi - \theta_{air} \ .\end{split}
\end{equation}
\(D_{v}\) depends on temperature
\phantomsection\label{\detokenize{tech_note/Fluxes/CLM50_Tech_Note_Fluxes:equation-5.81}}\begin{equation}\label{equation:tech_note/Fluxes/CLM50_Tech_Note_Fluxes:5.81}
\begin{split}D_{v} = 2.12 \times 10^{-5} \left(\frac{T_{1}}{T_{f}}\right)^{1.75} \ .\end{split}
\end{equation}
where \(T_{1}\) (K) is the temperature of the top soil layer and
\(T_{f}\) (K) is the freezing temperature of water
(\hyperref[\detokenize{tech_note/Ecosystem/CLM50_Tech_Note_Ecosystem:table-physical-constants}]{Table \ref{\detokenize{tech_note/Ecosystem/CLM50_Tech_Note_Ecosystem:table-physical-constants}}}).

The roughness lengths used to calculate \(r_{am}\) ,
\(r_{ah}\) , and \(r_{aw}\)  are \(z_{0m} =z_{0m,\, g}\) ,
\(z_{0h} =z_{0h,\, g}\) , and \(z_{0w} =z_{0w,\, g}\) . The
displacement height \(d=0\). The momentum roughness length is
\(z_{0m,\, g} =0.01\) for soil, glaciers, and wetland, and
\(z_{0m,\, g} =0.0024\) for snow-covered surfaces
(\(f_{sno} >0\)). In general, \(z_{0m}\)  is different from
\(z_{0h}\)  because the transfer of momentum is affected by pressure
fluctuations in the turbulent waves behind the roughness elements, while
for heat and water vapor transfer no such dynamical mechanism exists.
Rather, heat and water vapor must be transferred by molecular diffusion
across the interfacial sublayer. The following relation from
{\hyperref[\detokenize{tech_note/References/CLM50_Tech_Note_References:zilitinkevich1970}]{\sphinxcrossref{\DUrole{std,std-ref}{Zilitinkevich (1970)}}}} is adopted by
{\hyperref[\detokenize{tech_note/References/CLM50_Tech_Note_References:zengdickinson1998}]{\sphinxcrossref{\DUrole{std,std-ref}{Zeng and Dickinson 1998}}}}
\phantomsection\label{\detokenize{tech_note/Fluxes/CLM50_Tech_Note_Fluxes:equation-5.82}}\begin{equation}\label{equation:tech_note/Fluxes/CLM50_Tech_Note_Fluxes:5.82}
\begin{split}z_{0h,\, g} =z_{0w,\, g} =z_{0m,\, g} e^{-a\left({u_{*} z_{0m,\, g} \mathord{\left/ {\vphantom {u_{*} z_{0m,\, g}  \upsilon }} \right. \kern-\nulldelimiterspace} \upsilon } \right)^{0.45} }\end{split}
\end{equation}
where the quantity
\({u_{\*} z_{0m,\, g} \mathord{\left/ {\vphantom {u_{*} z_{0m,\, g}  \upsilon }} \right. \kern-\nulldelimiterspace} \upsilon }\)
is the roughness Reynolds number (and may be interpreted as the Reynolds number of the smallest turbulent eddy in the flow) with the kinematic
viscosity of air \(\upsilon =1.5\times 10^{-5}\)  m$^{\text{2}}$ s$^{\text{-1}}$ and \(a=0.13\).

The numerical solution for the fluxes of momentum, sensible heat, and
water vapor flux from non-vegetated surfaces proceeds as follows:
\begin{enumerate}
\item {} 
An initial guess for the wind speed \(V_{a}\)  is obtained from
\eqref{equation:tech_note/Fluxes/CLM50_Tech_Note_Fluxes:5.24} assuming an initial convective velocity \(U_{c} =0\) m
s$^{\text{-1}}$ for stable conditions
(\(\theta _{v,\, atm} -\theta _{v,\, s} \ge 0\) as evaluated from
\eqref{equation:tech_note/Fluxes/CLM50_Tech_Note_Fluxes:5.50} ) and \(U_{c} =0.5\) for unstable conditions
(\(\theta _{v,\, atm} -\theta _{v,\, s} <0\)).

\item {} 
An initial guess for the Monin-Obukhov length \(L\) is obtained
from the bulk Richardson number using \eqref{equation:tech_note/Fluxes/CLM50_Tech_Note_Fluxes:5.46} and \eqref{equation:tech_note/Fluxes/CLM50_Tech_Note_Fluxes:5.48}.

\item {} 
The following system of equations is iterated three times:

\item {} 
Friction velocity \(u_{*}\)  (\eqref{equation:tech_note/Fluxes/CLM50_Tech_Note_Fluxes:5.32}, \eqref{equation:tech_note/Fluxes/CLM50_Tech_Note_Fluxes:5.33}, \eqref{equation:tech_note/Fluxes/CLM50_Tech_Note_Fluxes:5.34}, \eqref{equation:tech_note/Fluxes/CLM50_Tech_Note_Fluxes:5.35})

\item {} 
Potential temperature scale \(\theta _{*}\)  (\eqref{equation:tech_note/Fluxes/CLM50_Tech_Note_Fluxes:5.37} , \eqref{equation:tech_note/Fluxes/CLM50_Tech_Note_Fluxes:5.38}, \eqref{equation:tech_note/Fluxes/CLM50_Tech_Note_Fluxes:5.39}, \eqref{equation:tech_note/Fluxes/CLM50_Tech_Note_Fluxes:5.40})

\item {} 
Humidity scale \(q_{*}\)  (\eqref{equation:tech_note/Fluxes/CLM50_Tech_Note_Fluxes:5.41}, \eqref{equation:tech_note/Fluxes/CLM50_Tech_Note_Fluxes:5.42}, \eqref{equation:tech_note/Fluxes/CLM50_Tech_Note_Fluxes:5.43}, \eqref{equation:tech_note/Fluxes/CLM50_Tech_Note_Fluxes:5.44})

\item {} 
Roughness lengths for sensible \(z_{0h,\, g}\)  and latent heat
\(z_{0w,\, g}\)  (\eqref{equation:tech_note/Fluxes/CLM50_Tech_Note_Fluxes:5.82} )

\item {} 
Virtual potential temperature scale \(\theta _{v*}\)  ( \eqref{equation:tech_note/Fluxes/CLM50_Tech_Note_Fluxes:5.17})

\item {} 
Wind speed including the convective velocity, \(V_{a}\)  ( \eqref{equation:tech_note/Fluxes/CLM50_Tech_Note_Fluxes:5.24})

\item {} 
Monin-Obukhov length \(L\) (\eqref{equation:tech_note/Fluxes/CLM50_Tech_Note_Fluxes:5.49})

\item {} 
Aerodynamic resistances \(r_{am}\) , \(r_{ah}\) , and
\(r_{aw}\)  (\eqref{equation:tech_note/Fluxes/CLM50_Tech_Note_Fluxes:5.55}, \eqref{equation:tech_note/Fluxes/CLM50_Tech_Note_Fluxes:5.56}, \eqref{equation:tech_note/Fluxes/CLM50_Tech_Note_Fluxes:5.57})

\item {} 
Momentum fluxes \(\tau _{x}\) , \(\tau _{y}\)  (\eqref{equation:tech_note/Fluxes/CLM50_Tech_Note_Fluxes:5.5}, \eqref{equation:tech_note/Fluxes/CLM50_Tech_Note_Fluxes:5.6})

\item {} 
Sensible heat flux \(H_{g}\)  (\eqref{equation:tech_note/Fluxes/CLM50_Tech_Note_Fluxes:5.62})

\item {} 
Water vapor flux \(E_{g}\)  (\eqref{equation:tech_note/Fluxes/CLM50_Tech_Note_Fluxes:5.66})

\item {} 
2-m height air temperature \(T_{2m}\)  and specific humidity
\(q_{2m}\)  (\eqref{equation:tech_note/Fluxes/CLM50_Tech_Note_Fluxes:5.58} , \eqref{equation:tech_note/Fluxes/CLM50_Tech_Note_Fluxes:5.59})

\end{enumerate}

The partial derivatives of the soil surface fluxes with respect to
ground temperature, which are needed for the soil temperature calculations (section
\hyperref[\detokenize{tech_note/Soil_Snow_Temperatures/CLM50_Tech_Note_Soil_Snow_Temperatures:numerical-solution-temperature}]{\ref{\detokenize{tech_note/Soil_Snow_Temperatures/CLM50_Tech_Note_Soil_Snow_Temperatures:numerical-solution-temperature}}}) and to update the soil surface fluxes
(section \hyperref[\detokenize{tech_note/Fluxes/CLM50_Tech_Note_Fluxes:update-of-ground-sensible-and-latent-heat-fluxes}]{\ref{\detokenize{tech_note/Fluxes/CLM50_Tech_Note_Fluxes:update-of-ground-sensible-and-latent-heat-fluxes}}}), are
\phantomsection\label{\detokenize{tech_note/Fluxes/CLM50_Tech_Note_Fluxes:equation-5.83}}\begin{equation}\label{equation:tech_note/Fluxes/CLM50_Tech_Note_Fluxes:5.83}
\begin{split}\frac{\partial H_{g} }{\partial T_{g} } =\frac{\rho _{atm} C_{p} }{r_{ah} }\end{split}
\end{equation}\phantomsection\label{\detokenize{tech_note/Fluxes/CLM50_Tech_Note_Fluxes:equation-5.84}}\begin{equation}\label{equation:tech_note/Fluxes/CLM50_Tech_Note_Fluxes:5.84}
\begin{split}\frac{\partial E_{g} }{\partial T_{g} } =\frac{\beta _{soi} \rho _{atm} }{r_{aw} } \frac{dq_{g} }{dT_{g} }\end{split}
\end{equation}
where
\phantomsection\label{\detokenize{tech_note/Fluxes/CLM50_Tech_Note_Fluxes:equation-5.85}}\begin{equation}\label{equation:tech_note/Fluxes/CLM50_Tech_Note_Fluxes:5.85}
\begin{split}\frac{dq_{g} }{dT_{g} } =\left(1-f_{sno} -f_{h2osfc} \right)\alpha _{soil} \frac{dq_{sat}^{T_{soil} } }{dT_{soil} } +f_{sno} \frac{dq_{sat}^{T_{sno} } }{dT_{sno} } +f_{h2osfc} \frac{dq_{sat}^{T_{h2osfc} } }{dT_{h2osfc} } .\end{split}
\end{equation}
The partial derivatives
\(\frac{\partial r_{ah} }{\partial T_{g} }\)  and
\(\frac{\partial r_{aw} }{\partial T_{g} }\) , which cannot be
determined analytically, are ignored for
\(\frac{\partial H_{g} }{\partial T_{g} }\)  and
\(\frac{\partial E_{g} }{\partial T_{g} }\) .


\subsection{Sensible and Latent Heat Fluxes and Temperature for Vegetated Surfaces}
\label{\detokenize{tech_note/Fluxes/CLM50_Tech_Note_Fluxes:sensible-and-latent-heat-fluxes-and-temperature-for-vegetated-surfaces}}\label{\detokenize{tech_note/Fluxes/CLM50_Tech_Note_Fluxes:id3}}
In the case of a vegetated surface, the sensible heat \(H\) and
water vapor flux \(E\) are partitioned into vegetation and ground
fluxes that depend on vegetation \(T_{v}\)  and ground
\(T_{g}\)  temperatures in addition to surface temperature
\(T_{s}\)  and specific humidity \(q_{s}\) . Because of the
coupling between vegetation temperature and fluxes, Newton-Raphson
iteration is used to solve for the vegetation temperature and the
sensible heat and water vapor fluxes from vegetation simultaneously
using the ground temperature from the previous time step. In section
\hyperref[\detokenize{tech_note/Fluxes/CLM50_Tech_Note_Fluxes:theory}]{\ref{\detokenize{tech_note/Fluxes/CLM50_Tech_Note_Fluxes:theory}}}, the equations used in the iteration scheme are derived. Details
on the numerical scheme are provided in section \hyperref[\detokenize{tech_note/Fluxes/CLM50_Tech_Note_Fluxes:numerical-implementation}]{\ref{\detokenize{tech_note/Fluxes/CLM50_Tech_Note_Fluxes:numerical-implementation}}}.


\subsubsection{Theory}
\label{\detokenize{tech_note/Fluxes/CLM50_Tech_Note_Fluxes:id4}}\label{\detokenize{tech_note/Fluxes/CLM50_Tech_Note_Fluxes:theory}}
The air within the canopy is assumed to have negligible capacity to
store heat so that the sensible heat flux \(H\) between the surface
at height \(z_{0h} +d\) and the atmosphere at height
\(z_{atm,\, h}\)  must be balanced by the sum of the sensible heat
from the vegetation \(H_{v}\)  and the ground \(H_{g}\)
\phantomsection\label{\detokenize{tech_note/Fluxes/CLM50_Tech_Note_Fluxes:equation-5.86}}\begin{equation}\label{equation:tech_note/Fluxes/CLM50_Tech_Note_Fluxes:5.86}
\begin{split}H=H_{v} +H_{g}\end{split}
\end{equation}
where, with reference to \hyperref[\detokenize{tech_note/Fluxes/CLM50_Tech_Note_Fluxes:figure-schematic-diagram-of-sensible-heat-fluxes}]{Figure \ref{\detokenize{tech_note/Fluxes/CLM50_Tech_Note_Fluxes:figure-schematic-diagram-of-sensible-heat-fluxes}}},
\phantomsection\label{\detokenize{tech_note/Fluxes/CLM50_Tech_Note_Fluxes:equation-5.87}}\begin{equation}\label{equation:tech_note/Fluxes/CLM50_Tech_Note_Fluxes:5.87}
\begin{split}H=-\rho _{atm} C_{p} \frac{\left(\theta _{atm} -T_{s} \right)}{r_{ah} }\end{split}
\end{equation}\phantomsection\label{\detokenize{tech_note/Fluxes/CLM50_Tech_Note_Fluxes:equation-5.88}}\begin{equation}\label{equation:tech_note/Fluxes/CLM50_Tech_Note_Fluxes:5.88}
\begin{split}H_{v} =-\rho _{atm} C_{p} \left(T_{s} -T_{v} \right)\frac{\left(L+S\right)}{r_{b} }\end{split}
\end{equation}\phantomsection\label{\detokenize{tech_note/Fluxes/CLM50_Tech_Note_Fluxes:equation-5.89}}\begin{equation}\label{equation:tech_note/Fluxes/CLM50_Tech_Note_Fluxes:5.89}
\begin{split}H_{g} =\left(1-f_{sno} -f_{h2osfc} \right)H_{soil} +f_{sno} H_{snow} +f_{h2osfc} H_{h2osfc} \ ,\end{split}
\end{equation}
where
\phantomsection\label{\detokenize{tech_note/Fluxes/CLM50_Tech_Note_Fluxes:equation-5.90}}\begin{equation}\label{equation:tech_note/Fluxes/CLM50_Tech_Note_Fluxes:5.90}
\begin{split}H_{soil} =-\rho _{atm} C_{p} \frac{\left(T_{s} -T_{1} \right)}{r_{ah} ^{{'} } }\end{split}
\end{equation}\phantomsection\label{\detokenize{tech_note/Fluxes/CLM50_Tech_Note_Fluxes:equation-5.91}}\begin{equation}\label{equation:tech_note/Fluxes/CLM50_Tech_Note_Fluxes:5.91}
\begin{split}H_{sno} =-\rho _{atm} C_{p} \frac{\left(T_{s} -T_{snl+1} \right)}{r_{ah} ^{{'} } }\end{split}
\end{equation}\phantomsection\label{\detokenize{tech_note/Fluxes/CLM50_Tech_Note_Fluxes:equation-5.92}}\begin{equation}\label{equation:tech_note/Fluxes/CLM50_Tech_Note_Fluxes:5.92}
\begin{split}H_{h2osfc} =-\rho _{atm} C_{p} \frac{\left(T_{s} -T_{h2osfc} \right)}{r_{ah} ^{{'} } }\end{split}
\end{equation}
where \(\rho _{atm}\)  is the density of atmospheric air (kg m$^{\text{-3}}$), \(C_{p}\)  is the specific heat capacity of air
(J kg$^{\text{-1}}$ K$^{\text{-1}}$) (\hyperref[\detokenize{tech_note/Ecosystem/CLM50_Tech_Note_Ecosystem:table-physical-constants}]{Table \ref{\detokenize{tech_note/Ecosystem/CLM50_Tech_Note_Ecosystem:table-physical-constants}}}),
\(\theta _{atm}\)  is the atmospheric potential temperature (K), and
\(r_{ah}\)  is the aerodynamic resistance to sensible heat transfer
(s m$^{\text{-1}}$).

Here, \(T_{s}\)  is the surface temperature at height
\(z_{0h} +d\), also referred to as the canopy air temperature.
\(L\) and \(S\) are the exposed leaf and stem area indices
(section \hyperref[\detokenize{tech_note/Ecosystem/CLM50_Tech_Note_Ecosystem:phenology-and-vegetation-burial-by-snow}]{\ref{\detokenize{tech_note/Ecosystem/CLM50_Tech_Note_Ecosystem:phenology-and-vegetation-burial-by-snow}}}), \(r_{b}\)  is the leaf boundary layer resistance (s
m$^{\text{-1}}$), and \(r_{ah} ^{{'} }\)  is the aerodynamic
resistance (s m$^{\text{-1}}$) to heat transfer between the ground at
height \(z_{0h} ^{{'} }\)  and the canopy air at height
\(z_{0h} +d\).

\begin{figure}[htbp]
\centering
\capstart

\noindent\sphinxincludegraphics{{image12}.png}
\caption{Figure Schematic diagram of sensible heat fluxes for (a)
non-vegetated surfaces and (b) vegetated surfaces.}\label{\detokenize{tech_note/Fluxes/CLM50_Tech_Note_Fluxes:figure-schematic-diagram-of-sensible-heat-fluxes}}\label{\detokenize{tech_note/Fluxes/CLM50_Tech_Note_Fluxes:id8}}\end{figure}

\begin{figure}[htbp]
\centering
\capstart

\noindent\sphinxincludegraphics{{image2}.png}
\caption{Figure Schematic diagram of water vapor fluxes for (a)
non-vegetated surfaces and (b) vegetated surfaces.}\label{\detokenize{tech_note/Fluxes/CLM50_Tech_Note_Fluxes:figure-schematic-diagram-of-latent-heat-fluxes}}\label{\detokenize{tech_note/Fluxes/CLM50_Tech_Note_Fluxes:id9}}\end{figure}

Equations \eqref{equation:tech_note/Fluxes/CLM50_Tech_Note_Fluxes:5.86} - \eqref{equation:tech_note/Fluxes/CLM50_Tech_Note_Fluxes:5.89} can be solved for the canopy air
temperature \(T_{s}\)
\phantomsection\label{\detokenize{tech_note/Fluxes/CLM50_Tech_Note_Fluxes:equation-5.93}}\begin{equation}\label{equation:tech_note/Fluxes/CLM50_Tech_Note_Fluxes:5.93}
\begin{split}T_{s} =\frac{c_{a}^{h} \theta _{atm} +c_{g}^{h} T_{g} +c_{v}^{h} T_{v} }{c_{a}^{h} +c_{g}^{h} +c_{v}^{h} }\end{split}
\end{equation}
where
\phantomsection\label{\detokenize{tech_note/Fluxes/CLM50_Tech_Note_Fluxes:equation-5.94}}\begin{equation}\label{equation:tech_note/Fluxes/CLM50_Tech_Note_Fluxes:5.94}
\begin{split}c_{a}^{h} =\frac{1}{r_{ah} }\end{split}
\end{equation}\phantomsection\label{\detokenize{tech_note/Fluxes/CLM50_Tech_Note_Fluxes:equation-5.95}}\begin{equation}\label{equation:tech_note/Fluxes/CLM50_Tech_Note_Fluxes:5.95}
\begin{split}c_{g}^{h} =\frac{1}{r_{ah} ^{{'} } }\end{split}
\end{equation}\phantomsection\label{\detokenize{tech_note/Fluxes/CLM50_Tech_Note_Fluxes:equation-5.96}}\begin{equation}\label{equation:tech_note/Fluxes/CLM50_Tech_Note_Fluxes:5.96}
\begin{split}c_{v}^{h} =\frac{\left(L+S\right)}{r_{b} }\end{split}
\end{equation}
are the sensible heat conductances from the canopy air to the
atmosphere, the ground to canopy air, and leaf surface to canopy air,
respectively (m s$^{\text{-1}}$).

When the expression for \(T_{s}\)  is substituted into equation \eqref{equation:tech_note/Fluxes/CLM50_Tech_Note_Fluxes:5.88},
the sensible heat flux from vegetation \(H_{v}\)  is a function of
\(\theta _{atm}\) , \(T_{g}\) , and \(T_{v}\)
\phantomsection\label{\detokenize{tech_note/Fluxes/CLM50_Tech_Note_Fluxes:equation-5.97}}\begin{equation}\label{equation:tech_note/Fluxes/CLM50_Tech_Note_Fluxes:5.97}
\begin{split}H_{v} = -\rho _{atm} C_{p} \left[c_{a}^{h} \theta _{atm} +c_{g}^{h} T_{g} -\left(c_{a}^{h} +c_{g}^{h} \right)T_{v} \right]\frac{c_{v}^{h} }{c_{a}^{h} +c_{v}^{h} +c_{g}^{h} } .\end{split}
\end{equation}
Similarly, the expression for \(T_{s}\)  can be substituted into
equation to obtain the sensible heat flux from ground \(H_{g}\)
\phantomsection\label{\detokenize{tech_note/Fluxes/CLM50_Tech_Note_Fluxes:equation-5.98}}\begin{equation}\label{equation:tech_note/Fluxes/CLM50_Tech_Note_Fluxes:5.98}
\begin{split}H_{g} = -\rho _{atm} C_{p} \left[c_{a}^{h} \theta _{atm} +c_{v}^{h} T_{v} -\left(c_{a}^{h} +c_{v}^{h} \right)T_{g} \right]\frac{c_{g}^{h} }{c_{a}^{h} +c_{v}^{h} +c_{g}^{h} } .\end{split}
\end{equation}
The air within the canopy is assumed to have negligible capacity to
store water vapor so that the water vapor flux \(E\) between the
surface at height \(z_{0w} +d\) and the atmosphere at height
\(z_{atm,\, w}\)  must be balanced by the sum of the water vapor
flux from the vegetation \(E_{v}\)  and the ground \(E_{g}\)
\phantomsection\label{\detokenize{tech_note/Fluxes/CLM50_Tech_Note_Fluxes:equation-5.99}}\begin{equation}\label{equation:tech_note/Fluxes/CLM50_Tech_Note_Fluxes:5.99}
\begin{split}E = E_{v} +E_{g}\end{split}
\end{equation}
where, with reference to \hyperref[\detokenize{tech_note/Fluxes/CLM50_Tech_Note_Fluxes:figure-schematic-diagram-of-latent-heat-fluxes}]{Figure \ref{\detokenize{tech_note/Fluxes/CLM50_Tech_Note_Fluxes:figure-schematic-diagram-of-latent-heat-fluxes}}},
\phantomsection\label{\detokenize{tech_note/Fluxes/CLM50_Tech_Note_Fluxes:equation-5.100}}\begin{equation}\label{equation:tech_note/Fluxes/CLM50_Tech_Note_Fluxes:5.100}
\begin{split}E = -\rho _{atm} \frac{\left(q_{atm} -q_{s} \right)}{r_{aw} }\end{split}
\end{equation}\phantomsection\label{\detokenize{tech_note/Fluxes/CLM50_Tech_Note_Fluxes:equation-5.101}}\begin{equation}\label{equation:tech_note/Fluxes/CLM50_Tech_Note_Fluxes:5.101}
\begin{split}E_{v} = -\rho _{atm} \frac{\left(q_{s} -q_{sat}^{T_{v} } \right)}{r_{total} }\end{split}
\end{equation}\phantomsection\label{\detokenize{tech_note/Fluxes/CLM50_Tech_Note_Fluxes:equation-5.102}}\begin{equation}\label{equation:tech_note/Fluxes/CLM50_Tech_Note_Fluxes:5.102}
\begin{split}E_{g} = \left(1-f_{sno} -f_{h2osfc} \right)E_{soil} +f_{sno} E_{snow} +f_{h2osfc} E_{h2osfc} \ ,\end{split}
\end{equation}
where
\phantomsection\label{\detokenize{tech_note/Fluxes/CLM50_Tech_Note_Fluxes:equation-5.103}}\begin{equation}\label{equation:tech_note/Fluxes/CLM50_Tech_Note_Fluxes:5.103}
\begin{split}E_{soil} = -\rho _{atm} \frac{\left(q_{s} -q_{soil} \right)}{r_{aw} ^{{'} } +r_{soil} }\end{split}
\end{equation}\phantomsection\label{\detokenize{tech_note/Fluxes/CLM50_Tech_Note_Fluxes:equation-5.104}}\begin{equation}\label{equation:tech_note/Fluxes/CLM50_Tech_Note_Fluxes:5.104}
\begin{split}E_{sno} = -\rho _{atm} \frac{\left(q_{s} -q_{sno} \right)}{r_{aw} ^{{'} } +r_{soil} }\end{split}
\end{equation}\phantomsection\label{\detokenize{tech_note/Fluxes/CLM50_Tech_Note_Fluxes:equation-5.105}}\begin{equation}\label{equation:tech_note/Fluxes/CLM50_Tech_Note_Fluxes:5.105}
\begin{split}E_{h2osfc} = -\rho _{atm} \frac{\left(q_{s} -q_{h2osfc} \right)}{r_{aw} ^{{'} } +r_{soil} }\end{split}
\end{equation}
where \(q_{atm}\)  is the atmospheric specific humidity (kg kg$^{\text{-1}}$), \(r_{aw}\)  is the aerodynamic resistance to
water vapor transfer (s m$^{\text{-1}}$), \(q_{sat}^{T_{v} }\)
(kg kg$^{\text{-1}}$) is the saturation water vapor specific humidity
at the vegetation temperature (section \hyperref[\detokenize{tech_note/Fluxes/CLM50_Tech_Note_Fluxes:saturation-vapor-pressure}]{\ref{\detokenize{tech_note/Fluxes/CLM50_Tech_Note_Fluxes:saturation-vapor-pressure}}}), \(q_{g}\)  ,
\(q_{sno}\)  , and \(q_{h2osfc}\)  are the specific humidities
of the soil, snow, and surface water (section \hyperref[\detokenize{tech_note/Fluxes/CLM50_Tech_Note_Fluxes:sensible-and-latent-heat-fluxes-for-non-vegetated-surfaces}]{\ref{\detokenize{tech_note/Fluxes/CLM50_Tech_Note_Fluxes:sensible-and-latent-heat-fluxes-for-non-vegetated-surfaces}}}),
\(r_{aw} ^{{'} }\)  is the aerodynamic resistance (s
m$^{\text{-1}}$) to water vapor transfer between the ground at height
\(z_{0w} ^{{'} }\)  and the canopy air at height \(z_{0w} +d\),
and \(r_{soil}\)  (\eqref{equation:tech_note/Fluxes/CLM50_Tech_Note_Fluxes:5.76}) is a resistance to diffusion through the soil
(s m$^{\text{-1}}$). \(r_{total}\)  is the total resistance to
water vapor transfer from the canopy to the canopy air and includes
contributions from leaf boundary layer and sunlit and shaded stomatal
resistances \(r_{b}\) , \(r_{s}^{sun}\) , and
\(r_{s}^{sha}\)  (\hyperref[\detokenize{tech_note/Fluxes/CLM50_Tech_Note_Fluxes:figure-schematic-diagram-of-latent-heat-fluxes}]{Figure \ref{\detokenize{tech_note/Fluxes/CLM50_Tech_Note_Fluxes:figure-schematic-diagram-of-latent-heat-fluxes}}}).
The water vapor flux from vegetation
is the sum of water vapor flux from wetted leaf and stem area
\(E_{v}^{w}\)  (evaporation of water intercepted by the canopy) and
transpiration from dry leaf surfaces \(E_{v}^{t}\)
\phantomsection\label{\detokenize{tech_note/Fluxes/CLM50_Tech_Note_Fluxes:equation-5.106}}\begin{equation}\label{equation:tech_note/Fluxes/CLM50_Tech_Note_Fluxes:5.106}
\begin{split}E_{v} =E_{v}^{w} +E_{v}^{t} .\end{split}
\end{equation}
Equations \eqref{equation:tech_note/Fluxes/CLM50_Tech_Note_Fluxes:5.99} - \eqref{equation:tech_note/Fluxes/CLM50_Tech_Note_Fluxes:5.102} can be solved for the canopy specific humidity
\(q_{s}\)
\phantomsection\label{\detokenize{tech_note/Fluxes/CLM50_Tech_Note_Fluxes:equation-5.107}}\begin{equation}\label{equation:tech_note/Fluxes/CLM50_Tech_Note_Fluxes:5.107}
\begin{split}q_{s} =\frac{c_{a}^{w} q_{atm} +c_{g}^{w} q_{g} +c_{v}^{w} q_{sat}^{T_{v} } }{c_{a}^{w} +c_{v}^{w} +c_{g}^{w} }\end{split}
\end{equation}
where
\phantomsection\label{\detokenize{tech_note/Fluxes/CLM50_Tech_Note_Fluxes:equation-5.108}}\begin{equation}\label{equation:tech_note/Fluxes/CLM50_Tech_Note_Fluxes:5.108}
\begin{split}c_{a}^{w} =\frac{1}{r_{aw} }\end{split}
\end{equation}\phantomsection\label{\detokenize{tech_note/Fluxes/CLM50_Tech_Note_Fluxes:equation-5.109}}\begin{equation}\label{equation:tech_note/Fluxes/CLM50_Tech_Note_Fluxes:5.109}
\begin{split}c_{v}^{w} =\frac{\left(L+S\right)}{r_{b} } r''\end{split}
\end{equation}\phantomsection\label{\detokenize{tech_note/Fluxes/CLM50_Tech_Note_Fluxes:equation-5.110}}\begin{equation}\label{equation:tech_note/Fluxes/CLM50_Tech_Note_Fluxes:5.110}
\begin{split}c_{g}^{w} =\frac{1}{r_{aw} ^{{'} } +r_{soil} }\end{split}
\end{equation}
are the water vapor conductances from the canopy air to the atmosphere,
the leaf to canopy air, and ground to canopy air, respectively. The term
\(r''\) is determined from contributions by wet leaves and
transpiration and limited by available water and potential evaporation
as
\phantomsection\label{\detokenize{tech_note/Fluxes/CLM50_Tech_Note_Fluxes:equation-5.111}}\begin{equation}\label{equation:tech_note/Fluxes/CLM50_Tech_Note_Fluxes:5.111}
\begin{split}r'' = \left\{
\begin{array}{lr}
\min \left(f_{wet} +r_{dry} ^{{'} {'} } ,\, \frac{E_{v}^{w,\, pot} r_{dry} ^{{'} {'} } +\frac{W_{can} }{\Delta t} }{E_{v}^{w,\, pot} } \right) & \qquad E_{v}^{w,\, pot} >0,\, \beta _{t} >0 \\
\min \left(f_{wet} ,\, \frac{E_{v}^{w,\, pot} r_{dry} ^{{'} {'} } +\frac{W_{can} }{\Delta t} }{E_{v}^{w,\, pot} } \right) & \qquad E_{v}^{w,\, pot} >0,\, \beta _{t} \le 0 \\
1 & \qquad E_{v}^{w,\, pot} \le 0
\end{array}\right\}\end{split}
\end{equation}
where \(f_{wet}\)  is the fraction of leaves and stems that are wet
(section \hyperref[\detokenize{tech_note/Hydrology/CLM50_Tech_Note_Hydrology:canopy-water}]{\ref{\detokenize{tech_note/Hydrology/CLM50_Tech_Note_Hydrology:canopy-water}}}), \(W_{can}\)  is canopy water (kg m$^{\text{-2}}$)
(section \hyperref[\detokenize{tech_note/Hydrology/CLM50_Tech_Note_Hydrology:canopy-water}]{\ref{\detokenize{tech_note/Hydrology/CLM50_Tech_Note_Hydrology:canopy-water}}}), \(\Delta t\) is the time step (s), and
\(\beta _{t}\)  is a soil moisture function limiting transpiration
(Chapter \hyperref[\detokenize{tech_note/Photosynthesis/CLM50_Tech_Note_Photosynthesis:rst-stomatal-resistance-and-photosynthesis}]{\ref{\detokenize{tech_note/Photosynthesis/CLM50_Tech_Note_Photosynthesis:rst-stomatal-resistance-and-photosynthesis}}}). The potential
evaporation from wet foliage per unit wetted area is
\phantomsection\label{\detokenize{tech_note/Fluxes/CLM50_Tech_Note_Fluxes:equation-5.112}}\begin{equation}\label{equation:tech_note/Fluxes/CLM50_Tech_Note_Fluxes:5.112}
\begin{split}E_{v}^{w,\, pot} =-\frac{\rho _{atm} \left(q_{s} -q_{sat}^{T_{v} } \right)}{r_{b} } .\end{split}
\end{equation}
The term \(r_{dry} ^{{'} {'} }\)  is
\phantomsection\label{\detokenize{tech_note/Fluxes/CLM50_Tech_Note_Fluxes:equation-5.113}}\begin{equation}\label{equation:tech_note/Fluxes/CLM50_Tech_Note_Fluxes:5.113}
\begin{split}r_{dry} ^{{'} {'} } =\frac{f_{dry} r_{b} }{L} \left(\frac{L^{sun} }{r_{b} +r_{s}^{sun} } +\frac{L^{sha} }{r_{b} +r_{s}^{sha} } \right)\end{split}
\end{equation}
where \(f_{dry}\)  is the fraction of leaves that are dry (section
\hyperref[\detokenize{tech_note/Hydrology/CLM50_Tech_Note_Hydrology:canopy-water}]{\ref{\detokenize{tech_note/Hydrology/CLM50_Tech_Note_Hydrology:canopy-water}}}), \(L^{sun}\)  and \(L^{sha}\)  are the sunlit and shaded
leaf area indices (section \hyperref[\detokenize{tech_note/Radiative_Fluxes/CLM50_Tech_Note_Radiative_Fluxes:solar-fluxes}]{\ref{\detokenize{tech_note/Radiative_Fluxes/CLM50_Tech_Note_Radiative_Fluxes:solar-fluxes}}}), and \(r_{s}^{sun}\)  and
\(r_{s}^{sha}\)  are the sunlit and shaded stomatal resistances (s
m$^{\text{-1}}$) (Chapter \hyperref[\detokenize{tech_note/Photosynthesis/CLM50_Tech_Note_Photosynthesis:rst-stomatal-resistance-and-photosynthesis}]{\ref{\detokenize{tech_note/Photosynthesis/CLM50_Tech_Note_Photosynthesis:rst-stomatal-resistance-and-photosynthesis}}}).

When the expression for \(q_{s}\)  is substituted into equation \eqref{equation:tech_note/Fluxes/CLM50_Tech_Note_Fluxes:5.101},
the water vapor flux from vegetation \(E_{v}\)  is a function of
\(q_{atm}\) , \(q_{g}\) , and \(q_{sat}^{T_{v} }\)
\phantomsection\label{\detokenize{tech_note/Fluxes/CLM50_Tech_Note_Fluxes:equation-5.114}}\begin{equation}\label{equation:tech_note/Fluxes/CLM50_Tech_Note_Fluxes:5.114}
\begin{split}E_{v} =-\rho _{atm} \left[c_{a}^{w} q_{atm} +c_{g}^{w} q_{g} -\left(c_{a}^{w} +c_{g}^{w} \right)q_{sat}^{T_{v} } \right]\frac{c_{v}^{w} }{c_{a}^{w} +c_{v}^{w} +c_{g}^{w} } .\end{split}
\end{equation}
Similarly, the expression for \(q_{s}\)  can be substituted into
\eqref{equation:tech_note/Fluxes/CLM50_Tech_Note_Fluxes:5.84} to obtain the water vapor flux from the ground beneath the
canopy \(E_{g}\)
\phantomsection\label{\detokenize{tech_note/Fluxes/CLM50_Tech_Note_Fluxes:equation-5.115}}\begin{equation}\label{equation:tech_note/Fluxes/CLM50_Tech_Note_Fluxes:5.115}
\begin{split}E_{g} =-\rho _{atm} \left[c_{a}^{w} q_{atm} +c_{v}^{w} q_{sat}^{T_{v} } -\left(c_{a}^{w} +c_{v}^{w} \right)q_{g} \right]\frac{c_{g}^{w} }{c_{a}^{w} +c_{v}^{w} +c_{g}^{w} } .\end{split}
\end{equation}
The aerodynamic resistances to heat (moisture) transfer between the
ground at height \(z_{0h} ^{{'} }\)  (\(z_{0w} ^{{'} }\) ) and
the canopy air at height \(z_{0h} +d\) (\(z_{0w} +d\)) are
\phantomsection\label{\detokenize{tech_note/Fluxes/CLM50_Tech_Note_Fluxes:equation-5.116}}\begin{equation}\label{equation:tech_note/Fluxes/CLM50_Tech_Note_Fluxes:5.116}
\begin{split}r_{ah} ^{{'} } =r_{aw} ^{{'} } =\frac{1}{C_{s} U_{av} }\end{split}
\end{equation}
where
\phantomsection\label{\detokenize{tech_note/Fluxes/CLM50_Tech_Note_Fluxes:equation-5.117}}\begin{equation}\label{equation:tech_note/Fluxes/CLM50_Tech_Note_Fluxes:5.117}
\begin{split}U_{av} =V_{a} \sqrt{\frac{1}{r_{am} V_{a} } } =u_{*}\end{split}
\end{equation}
is the magnitude of the wind velocity incident on the leaves
(equivalent here to friction velocity) (m s$^{\text{-1}}$) and
\(C_{s}\)  is the turbulent transfer coefficient between the
underlying soil and the canopy air. \(C_{s}\)  is obtained by
interpolation between values for dense canopy and bare soil
({\hyperref[\detokenize{tech_note/References/CLM50_Tech_Note_References:zengetal2005}]{\sphinxcrossref{\DUrole{std,std-ref}{Zeng et al. 2005}}}})
\phantomsection\label{\detokenize{tech_note/Fluxes/CLM50_Tech_Note_Fluxes:equation-5.118}}\begin{equation}\label{equation:tech_note/Fluxes/CLM50_Tech_Note_Fluxes:5.118}
\begin{split}C_{s} =C_{s,\, bare} W+C_{s,\, dense} (1-W)\end{split}
\end{equation}
where the weight \(W\) is
\phantomsection\label{\detokenize{tech_note/Fluxes/CLM50_Tech_Note_Fluxes:equation-5.119}}\begin{equation}\label{equation:tech_note/Fluxes/CLM50_Tech_Note_Fluxes:5.119}
\begin{split}W=e^{-\left(L+S\right)} .\end{split}
\end{equation}
The dense canopy turbulent transfer coefficient
({\hyperref[\detokenize{tech_note/References/CLM50_Tech_Note_References:dickinsonetal1993}]{\sphinxcrossref{\DUrole{std,std-ref}{Dickinson et al. 1993}}}}) is
\phantomsection\label{\detokenize{tech_note/Fluxes/CLM50_Tech_Note_Fluxes:equation-5.120)}}\begin{equation}\label{equation:tech_note/Fluxes/CLM50_Tech_Note_Fluxes:5.120)}
\begin{split}C_{s,\, dense} =0.004 \ .\end{split}
\end{equation}
The bare soil turbulent transfer coefficient is
\phantomsection\label{\detokenize{tech_note/Fluxes/CLM50_Tech_Note_Fluxes:equation-5.121}}\begin{equation}\label{equation:tech_note/Fluxes/CLM50_Tech_Note_Fluxes:5.121}
\begin{split}C_{s,\, bare} =\frac{k}{a} \left(\frac{z_{0m,\, g} U_{av} }{\upsilon } \right)^{-0.45}\end{split}
\end{equation}
where the kinematic viscosity of air
\(\upsilon =1.5\times 10^{-5}\)  m$^{\text{2}}$ s$^{\text{-1}}$ and \(a=0.13\).

The leaf boundary layer resistance \(r_{b}\)  is
\phantomsection\label{\detokenize{tech_note/Fluxes/CLM50_Tech_Note_Fluxes:equation-5.122}}\begin{equation}\label{equation:tech_note/Fluxes/CLM50_Tech_Note_Fluxes:5.122}
\begin{split}r_{b} =\frac{1}{C_{v} } \left({U_{av} \mathord{\left/ {\vphantom {U_{av}  d_{leaf} }} \right. \kern-\nulldelimiterspace} d_{leaf} } \right)^{{-1\mathord{\left/ {\vphantom {-1 2}} \right. \kern-\nulldelimiterspace} 2} }\end{split}
\end{equation}
where \(C_{v} =0.01\) ms$^{\text{-1/2}}$ is the turbulent
transfer coefficient between the canopy surface and canopy air, and
\(d_{leaf}\)  is the characteristic dimension of the leaves in the
direction of wind flow (\hyperref[\detokenize{tech_note/Fluxes/CLM50_Tech_Note_Fluxes:table-coefficients-for-saturation-vapor-pressure}]{Table \ref{\detokenize{tech_note/Fluxes/CLM50_Tech_Note_Fluxes:table-coefficients-for-saturation-vapor-pressure}}}).

The partial derivatives of the fluxes from the soil beneath the canopy
with respect to ground temperature, which are needed for the soil
temperature calculations (section \hyperref[\detokenize{tech_note/Soil_Snow_Temperatures/CLM50_Tech_Note_Soil_Snow_Temperatures:numerical-solution-temperature}]{\ref{\detokenize{tech_note/Soil_Snow_Temperatures/CLM50_Tech_Note_Soil_Snow_Temperatures:numerical-solution-temperature}}})
and to update the soil surface fluxes (section
\hyperref[\detokenize{tech_note/Fluxes/CLM50_Tech_Note_Fluxes:update-of-ground-sensible-and-latent-heat-fluxes}]{\ref{\detokenize{tech_note/Fluxes/CLM50_Tech_Note_Fluxes:update-of-ground-sensible-and-latent-heat-fluxes}}}), are
\phantomsection\label{\detokenize{tech_note/Fluxes/CLM50_Tech_Note_Fluxes:equation-5.123}}\begin{equation}\label{equation:tech_note/Fluxes/CLM50_Tech_Note_Fluxes:5.123}
\begin{split}\frac{\partial H_{g} }{\partial T_{g} } = \frac{\rho _{atm} C_{p} }{r'_{ah} } \frac{c_{a}^{h} +c_{v}^{h} }{c_{a}^{h} +c_{v}^{h} +c_{g}^{h} }\end{split}
\end{equation}\phantomsection\label{\detokenize{tech_note/Fluxes/CLM50_Tech_Note_Fluxes:equation-5.124}}\begin{equation}\label{equation:tech_note/Fluxes/CLM50_Tech_Note_Fluxes:5.124}
\begin{split}\frac{\partial E_{g} }{\partial T_{g} } = \frac{\rho _{atm} }{r'_{aw} +r_{soil} } \frac{c_{a}^{w} +c_{v}^{w} }{c_{a}^{w} +c_{v}^{w} +c_{g}^{w} } \frac{dq_{g} }{dT_{g} } .\end{split}
\end{equation}
The partial derivatives
\(\frac{\partial r'_{ah} }{\partial T_{g} }\)  and
\(\frac{\partial r'_{aw} }{\partial T_{g} }\) , which cannot be
determined analytically, are ignored for
\(\frac{\partial H_{g} }{\partial T_{g} }\)  and
\(\frac{\partial E_{g} }{\partial T_{g} }\) .

The roughness lengths used to calculate \(r_{am}\) ,
\(r_{ah}\) , and \(r_{aw}\)  from \eqref{equation:tech_note/Fluxes/CLM50_Tech_Note_Fluxes:5.55}, \eqref{equation:tech_note/Fluxes/CLM50_Tech_Note_Fluxes:5.56}, and \eqref{equation:tech_note/Fluxes/CLM50_Tech_Note_Fluxes:5.57} are
\(z_{0m} =z_{0m,\, v}\) , \(z_{0h} =z_{0h,\, v}\) , and
\(z_{0w} =z_{0w,\, v}\) . The vegetation displacement height
\(d\) and the roughness lengths are a function of plant height and
adjusted for canopy density following {\hyperref[\detokenize{tech_note/References/CLM50_Tech_Note_References:zengwang2007}]{\sphinxcrossref{\DUrole{std,std-ref}{Zeng and Wang (2007)}}}}
\phantomsection\label{\detokenize{tech_note/Fluxes/CLM50_Tech_Note_Fluxes:equation-5.125}}\begin{equation}\label{equation:tech_note/Fluxes/CLM50_Tech_Note_Fluxes:5.125}
\begin{split}z_{0m,\, v} = z_{0h,\, v} =z_{0w,\, v} =\exp \left[V\ln \left(z_{top} R_{z0m} \right)+\left(1-V\right)\ln \left(z_{0m,\, g} \right)\right]\end{split}
\end{equation}\phantomsection\label{\detokenize{tech_note/Fluxes/CLM50_Tech_Note_Fluxes:equation-5.126}}\begin{equation}\label{equation:tech_note/Fluxes/CLM50_Tech_Note_Fluxes:5.126}
\begin{split}d = z_{top} R_{d} V\end{split}
\end{equation}
where \(z_{top}\)  is canopy top height (m)
(\sphinxcode{Table Prescribed plant functional type heights}),
\(R_{z0m}\)  and \(R_{d}\)  are the ratio of momentum roughness
length and displacement height to canopy top height, respectively
(\hyperref[\detokenize{tech_note/Fluxes/CLM50_Tech_Note_Fluxes:table-plant-functional-type-aerodynamic-parameters}]{Table \ref{\detokenize{tech_note/Fluxes/CLM50_Tech_Note_Fluxes:table-plant-functional-type-aerodynamic-parameters}}}), and \(z_{0m,\, g}\)
is the ground momentum roughness length (m) (section
\hyperref[\detokenize{tech_note/Fluxes/CLM50_Tech_Note_Fluxes:sensible-and-latent-heat-fluxes-for-non-vegetated-surfaces}]{\ref{\detokenize{tech_note/Fluxes/CLM50_Tech_Note_Fluxes:sensible-and-latent-heat-fluxes-for-non-vegetated-surfaces}}}). The
fractional weight \(V\) is determined from
\phantomsection\label{\detokenize{tech_note/Fluxes/CLM50_Tech_Note_Fluxes:equation-5.127}}\begin{equation}\label{equation:tech_note/Fluxes/CLM50_Tech_Note_Fluxes:5.127}
\begin{split}V = \frac{1-\exp \left\{-\beta \min \left[L+S,\, \left(L+S\right)_{cr} \right]\right\}}{1-\exp \left[-\beta \left(L+S\right)_{cr} \right]}\end{split}
\end{equation}
where \(\beta =1\) and \(\left(L+S\right)_{cr} = 2\)
(m$^{\text{2}}$ m$^{\text{-2}}$) is a critical value of exposed leaf
plus stem area for which \(z_{0m}\)  reaches its maximum.


\begin{savenotes}\sphinxattablestart
\centering
\sphinxcapstartof{table}
\sphinxcaption{Plant functional type aerodynamic parameters}\label{\detokenize{tech_note/Fluxes/CLM50_Tech_Note_Fluxes:table-plant-functional-type-aerodynamic-parameters}}\label{\detokenize{tech_note/Fluxes/CLM50_Tech_Note_Fluxes:id10}}
\sphinxaftercaption
\begin{tabulary}{\linewidth}[t]{|T|T|T|T|}
\hline
\sphinxstylethead{\sphinxstyletheadfamily 
Plant functional type
\unskip}\relax &
\(R_{z0m}\)
&
\(R_{d}\)
&\sphinxstylethead{\sphinxstyletheadfamily 
\(d_{leaf}\)  (m)
\unskip}\relax \\
\hline
NET Temperate
&
0.055
&
0.67
&
0.04
\\
\hline
NET Boreal
&
0.055
&
0.67
&
0.04
\\
\hline
NDT Boreal
&
0.055
&
0.67
&
0.04
\\
\hline
BET Tropical
&
0.075
&
0.67
&
0.04
\\
\hline
BET temperate
&
0.075
&
0.67
&
0.04
\\
\hline
BDT tropical
&
0.055
&
0.67
&
0.04
\\
\hline
BDT temperate
&
0.055
&
0.67
&
0.04
\\
\hline
BDT boreal
&
0.055
&
0.67
&
0.04
\\
\hline
BES temperate
&
0.120
&
0.68
&
0.04
\\
\hline
BDS temperate
&
0.120
&
0.68
&
0.04
\\
\hline
BDS boreal
&
0.120
&
0.68
&
0.04
\\
\hline
C$_{\text{3}}$ arctic grass
&
0.120
&
0.68
&
0.04
\\
\hline
C$_{\text{3}}$ grass
&
0.120
&
0.68
&
0.04
\\
\hline
C$_{\text{4}}$ grass
&
0.120
&
0.68
&
0.04
\\
\hline
Crop R
&
0.120
&
0.68
&
0.04
\\
\hline
Crop I
&
0.120
&
0.68
&
0.04
\\
\hline
Corn R
&
0.120
&
0.68
&
0.04
\\
\hline
Corn I
&
0.120
&
0.68
&
0.04
\\
\hline
Temp Cereal R
&
0.120
&
0.68
&
0.04
\\
\hline
Temp Cereal I
&
0.120
&
0.68
&
0.04
\\
\hline
Winter Cereal R
&
0.120
&
0.68
&
0.04
\\
\hline
Winter Cereal I
&
0.120
&
0.68
&
0.04
\\
\hline
Soybean R
&
0.120
&
0.68
&
0.04
\\
\hline
Soybean I
&
0.120
&
0.68
&
0.04
\\
\hline
\end{tabulary}
\par
\sphinxattableend\end{savenotes}


\subsubsection{Numerical Implementation}
\label{\detokenize{tech_note/Fluxes/CLM50_Tech_Note_Fluxes:numerical-implementation}}\label{\detokenize{tech_note/Fluxes/CLM50_Tech_Note_Fluxes:id5}}
Canopy energy conservation gives
\phantomsection\label{\detokenize{tech_note/Fluxes/CLM50_Tech_Note_Fluxes:equation-5.128}}\begin{equation}\label{equation:tech_note/Fluxes/CLM50_Tech_Note_Fluxes:5.128}
\begin{split}-\overrightarrow{S}_{v} +\overrightarrow{L}_{v} \left(T_{v} \right)+H_{v} \left(T_{v} \right)+\lambda E_{v} \left(T_{v} \right)=0\end{split}
\end{equation}
where \(\overrightarrow{S}_{v}\)  is the solar radiation absorbed by
the vegetation (section \hyperref[\detokenize{tech_note/Radiative_Fluxes/CLM50_Tech_Note_Radiative_Fluxes:solar-fluxes}]{\ref{\detokenize{tech_note/Radiative_Fluxes/CLM50_Tech_Note_Radiative_Fluxes:solar-fluxes}}}), \(\overrightarrow{L}_{v}\)  is the net
longwave radiation absorbed by vegetation (section \hyperref[\detokenize{tech_note/Radiative_Fluxes/CLM50_Tech_Note_Radiative_Fluxes:longwave-fluxes}]{\ref{\detokenize{tech_note/Radiative_Fluxes/CLM50_Tech_Note_Radiative_Fluxes:longwave-fluxes}}}), and
\(H_{v}\)  and \(\lambda E_{v}\)  are the sensible and latent
heat fluxes from vegetation, respectively. The term \(\lambda\)  is
taken to be the latent heat of vaporization \(\lambda _{vap}\)
(\hyperref[\detokenize{tech_note/Ecosystem/CLM50_Tech_Note_Ecosystem:table-physical-constants}]{Table \ref{\detokenize{tech_note/Ecosystem/CLM50_Tech_Note_Ecosystem:table-physical-constants}}}).

\(\overrightarrow{L}_{v}\) , \(H_{v}\) , and
\(\lambda E_{v}\)  depend on the vegetation temperature
\(T_{v}\) . The Newton-Raphson method for finding roots of
non-linear systems of equations can be applied to iteratively solve for
\(T_{v}\)  as
\phantomsection\label{\detokenize{tech_note/Fluxes/CLM50_Tech_Note_Fluxes:equation-5.129}}\begin{equation}\label{equation:tech_note/Fluxes/CLM50_Tech_Note_Fluxes:5.129}
\begin{split}\Delta T_{v} =\frac{\overrightarrow{S}_{v} -\overrightarrow{L}_{v} -H_{v} -\lambda E_{v} }{\frac{\partial \overrightarrow{L}_{v} }{\partial T_{v} } +\frac{\partial H_{v} }{\partial T_{v} } +\frac{\partial \lambda E_{v} }{\partial T_{v} } }\end{split}
\end{equation}
where \(\Delta T_{v} =T_{v}^{n+1} -T_{v}^{n}\)  and the subscript
“n” indicates the iteration.

The partial derivatives are
\phantomsection\label{\detokenize{tech_note/Fluxes/CLM50_Tech_Note_Fluxes:equation-5.130}}\begin{equation}\label{equation:tech_note/Fluxes/CLM50_Tech_Note_Fluxes:5.130}
\begin{split}\frac{\partial \overrightarrow{L}_{v} }{\partial T_{v} } =4\varepsilon _{v} \sigma \left[2-\varepsilon _{v} \left(1-\varepsilon _{g} \right)\right]T_{v}^{3}\end{split}
\end{equation}\phantomsection\label{\detokenize{tech_note/Fluxes/CLM50_Tech_Note_Fluxes:equation-5.131}}\begin{equation}\label{equation:tech_note/Fluxes/CLM50_Tech_Note_Fluxes:5.131}
\begin{split}\frac{\partial H_{v} }{\partial T_{v} } =\rho _{atm} C_{p} \left(c_{a}^{h} +c_{g}^{h} \right)\frac{c_{v}^{h} }{c_{a}^{h} +c_{v}^{h} +c_{g}^{h} }\end{split}
\end{equation}\phantomsection\label{\detokenize{tech_note/Fluxes/CLM50_Tech_Note_Fluxes:equation-5.132}}\begin{equation}\label{equation:tech_note/Fluxes/CLM50_Tech_Note_Fluxes:5.132}
\begin{split}\frac{\partial \lambda E_{v} }{\partial T_{v} } =\lambda \rho _{atm} \left(c_{a}^{w} +c_{g}^{w} \right)\frac{c_{v}^{w} }{c_{a}^{w} +c_{v}^{w} +c_{g}^{w} } \frac{dq_{sat}^{T_{v} } }{dT_{v} } .\end{split}
\end{equation}
The partial derivatives
\(\frac{\partial r_{ah} }{\partial T_{v} }\)  and
\(\frac{\partial r_{aw} }{\partial T_{v} }\) , which cannot be
determined analytically, are ignored for
\(\frac{\partial H_{v} }{\partial T_{v} }\)  and
\(\frac{\partial \lambda E_{v} }{\partial T_{v} }\) . However, if
\(\zeta\)  changes sign more than four times during the temperature
iteration, \(\zeta =-0.01\). This helps prevent “flip-flopping”
between stable and unstable conditions. The total water vapor flux
\(E_{v}\) , transpiration flux \(E_{v}^{t}\) , and sensible heat
flux \(H_{v}\)  are updated for changes in leaf temperature as
\phantomsection\label{\detokenize{tech_note/Fluxes/CLM50_Tech_Note_Fluxes:equation-5.133}}\begin{equation}\label{equation:tech_note/Fluxes/CLM50_Tech_Note_Fluxes:5.133}
\begin{split}E_{v} =-\rho _{atm} \left[c_{a}^{w} q_{atm} +c_{g}^{w} q_{g} -\left(c_{a}^{w} +c_{g}^{w} \right)\left(q_{sat}^{T_{v} } +\frac{dq_{sat}^{T_{v} } }{dT_{v} } \Delta T_{v} \right)\right]\frac{c_{v}^{w} }{c_{a}^{w} +c_{v}^{w} +c_{g}^{w} }\end{split}
\end{equation}\phantomsection\label{\detokenize{tech_note/Fluxes/CLM50_Tech_Note_Fluxes:equation-5.134}}\begin{equation}\label{equation:tech_note/Fluxes/CLM50_Tech_Note_Fluxes:5.134}
\begin{split}E_{v}^{t} =-r_{dry} ^{{'} {'} } \rho _{atm} \left[c_{a}^{w} q_{atm} +c_{g}^{w} q_{g} -\left(c_{a}^{w} +c_{g}^{w} \right)\left(q_{sat}^{T_{v} } +\frac{dq_{sat}^{T_{v} } }{dT_{v} } \Delta T_{v} \right)\right]\frac{c_{v}^{h} }{c_{a}^{w} +c_{v}^{w} +c_{g}^{w} }\end{split}
\end{equation}\phantomsection\label{\detokenize{tech_note/Fluxes/CLM50_Tech_Note_Fluxes:equation-5.135}}\begin{equation}\label{equation:tech_note/Fluxes/CLM50_Tech_Note_Fluxes:5.135}
\begin{split}H_{v} =-\rho _{atm} C_{p} \left[c_{a}^{h} \theta _{atm} +c_{g}^{h} T_{g} -\left(c_{a}^{h} +c_{g}^{h} \right)\left(T_{v} +\Delta T_{v} \right)\right]\frac{c_{v}^{h} }{c_{a}^{h} +c_{v}^{h} +c_{g}^{h} } .\end{split}
\end{equation}
The numerical solution for vegetation temperature and the fluxes of
momentum, sensible heat, and water vapor flux from vegetated surfaces
proceeds as follows:
\begin{enumerate}
\item {} 
Initial values for canopy air temperature and specific humidity are
obtained from
\phantomsection\label{\detokenize{tech_note/Fluxes/CLM50_Tech_Note_Fluxes:equation-5.136}}\begin{equation}\label{equation:tech_note/Fluxes/CLM50_Tech_Note_Fluxes:5.136}
\begin{split}T_{s} =\frac{T_{g} +\theta _{atm} }{2}\end{split}
\end{equation}\phantomsection\label{\detokenize{tech_note/Fluxes/CLM50_Tech_Note_Fluxes:equation-5.137}}\begin{equation}\label{equation:tech_note/Fluxes/CLM50_Tech_Note_Fluxes:5.137}
\begin{split}q_{s} =\frac{q_{g} +q_{atm} }{2} .\end{split}
\end{equation}
\item {} 
An initial guess for the wind speed \(V_{a}\)  is obtained from
\eqref{equation:tech_note/Fluxes/CLM50_Tech_Note_Fluxes:5.24} assuming an initial convective velocity \(U_{c} =0\) m
s$^{\text{-1}}$ for stable conditions
(\(\theta _{v,\, atm} -\theta _{v,\, s} \ge 0\) as evaluated from
\eqref{equation:tech_note/Fluxes/CLM50_Tech_Note_Fluxes:5.50} ) and \(U_{c} =0.5\) for unstable conditions
(\(\theta _{v,\, atm} -\theta _{v,\, s} <0\)).

\item {} 
An initial guess for the Monin-Obukhov length \(L\) is obtained
from the bulk Richardson number using equation and  \eqref{equation:tech_note/Fluxes/CLM50_Tech_Note_Fluxes:5.46} and \eqref{equation:tech_note/Fluxes/CLM50_Tech_Note_Fluxes:5.48}.

\item {} 
Iteration proceeds on the following system of equations:

\item {} 
Friction velocity \(u_{*}\)  (\eqref{equation:tech_note/Fluxes/CLM50_Tech_Note_Fluxes:5.32}, \eqref{equation:tech_note/Fluxes/CLM50_Tech_Note_Fluxes:5.33}, \eqref{equation:tech_note/Fluxes/CLM50_Tech_Note_Fluxes:5.34}, \eqref{equation:tech_note/Fluxes/CLM50_Tech_Note_Fluxes:5.35})

\item {} 
Ratio \(\frac{\theta _{*} }{\theta _{atm} -\theta _{s} }\)
(\eqref{equation:tech_note/Fluxes/CLM50_Tech_Note_Fluxes:5.37} , \eqref{equation:tech_note/Fluxes/CLM50_Tech_Note_Fluxes:5.38}, \eqref{equation:tech_note/Fluxes/CLM50_Tech_Note_Fluxes:5.39}, \eqref{equation:tech_note/Fluxes/CLM50_Tech_Note_Fluxes:5.40})

\item {} 
Ratio \(\frac{q_{*} }{q_{atm} -q_{s} }\)  (\eqref{equation:tech_note/Fluxes/CLM50_Tech_Note_Fluxes:5.41}, \eqref{equation:tech_note/Fluxes/CLM50_Tech_Note_Fluxes:5.42}, \eqref{equation:tech_note/Fluxes/CLM50_Tech_Note_Fluxes:5.43}, \eqref{equation:tech_note/Fluxes/CLM50_Tech_Note_Fluxes:5.44})

\item {} 
Aerodynamic resistances \(r_{am}\) , \(r_{ah}\) , and
\(r_{aw}\)  (\eqref{equation:tech_note/Fluxes/CLM50_Tech_Note_Fluxes:5.55}, \eqref{equation:tech_note/Fluxes/CLM50_Tech_Note_Fluxes:5.56}, \eqref{equation:tech_note/Fluxes/CLM50_Tech_Note_Fluxes:5.57})

\item {} 
Magnitude of the wind velocity incident on the leaves \(U_{av}\)
(\eqref{equation:tech_note/Fluxes/CLM50_Tech_Note_Fluxes:5.117} )

\item {} 
Leaf boundary layer resistance \(r_{b}\)  (\eqref{equation:tech_note/Fluxes/CLM50_Tech_Note_Fluxes:5.136} )

\item {} 
Aerodynamic resistances \(r_{ah} ^{{'} }\)  and
\(r_{aw} ^{{'} }\)  (\eqref{equation:tech_note/Fluxes/CLM50_Tech_Note_Fluxes:5.116} )

\item {} 
Sunlit and shaded stomatal resistances \(r_{s}^{sun}\)  and
\(r_{s}^{sha}\)  (Chapter \hyperref[\detokenize{tech_note/Photosynthesis/CLM50_Tech_Note_Photosynthesis:rst-stomatal-resistance-and-photosynthesis}]{\ref{\detokenize{tech_note/Photosynthesis/CLM50_Tech_Note_Photosynthesis:rst-stomatal-resistance-and-photosynthesis}}})

\item {} 
Sensible heat conductances \(c_{a}^{h}\) , \(c_{g}^{h}\) ,
and \(c_{v}^{h}\)  (\eqref{equation:tech_note/Fluxes/CLM50_Tech_Note_Fluxes:5.94}, \eqref{equation:tech_note/Fluxes/CLM50_Tech_Note_Fluxes:5.95}, \eqref{equation:tech_note/Fluxes/CLM50_Tech_Note_Fluxes:5.96})

\item {} 
Latent heat conductances \(c_{a}^{w}\) , \(c_{v}^{w}\) , and
\(c_{g}^{w}\)  (\eqref{equation:tech_note/Fluxes/CLM50_Tech_Note_Fluxes:5.108}, \eqref{equation:tech_note/Fluxes/CLM50_Tech_Note_Fluxes:5.109}, \eqref{equation:tech_note/Fluxes/CLM50_Tech_Note_Fluxes:5.110})

\item {} 
Sensible heat flux from vegetation \(H_{v}\)  (\eqref{equation:tech_note/Fluxes/CLM50_Tech_Note_Fluxes:5.97} )

\item {} 
Latent heat flux from vegetation \(\lambda E_{v}\)  (\eqref{equation:tech_note/Fluxes/CLM50_Tech_Note_Fluxes:5.101} )

\item {} 
If the latent heat flux has changed sign from the latent heat flux
computed at the previous iteration
(\(\lambda E_{v} ^{n+1} \times \lambda E_{v} ^{n} <0\)), the
latent heat flux is constrained to be 10\% of the computed value. The
difference between the constrained and computed value
(\(\Delta _{1} =0.1\lambda E_{v} ^{n+1} -\lambda E_{v} ^{n+1}\) )
is added to the sensible heat flux later.

\item {} 
Change in vegetation temperature \(\Delta T_{v}\)  (\eqref{equation:tech_note/Fluxes/CLM50_Tech_Note_Fluxes:5.129} ) and
update the vegetation temperature as
\(T_{v}^{n+1} =T_{v}^{n} +\Delta T_{v}\) . \(T_{v}\)  is
constrained to change by no more than 1ºK in one iteration. If this
limit is exceeded, the energy error is
\phantomsection\label{\detokenize{tech_note/Fluxes/CLM50_Tech_Note_Fluxes:equation-5.138}}\begin{equation}\label{equation:tech_note/Fluxes/CLM50_Tech_Note_Fluxes:5.138}
\begin{split}\Delta _{2} =\overrightarrow{S}_{v} -\overrightarrow{L}_{v} -\frac{\partial \overrightarrow{L}_{v} }{\partial T_{v} } \Delta T_{v} -H_{v} -\frac{\partial H_{v} }{\partial T_{v} } \Delta T_{v} -\lambda E_{v} -\frac{\partial \lambda E_{v} }{\partial T_{v} } \Delta T_{v}\end{split}
\end{equation}
\end{enumerate}

where \(\Delta T_{v} =1{\rm \; or\; }-1\). The error
\(\Delta _{2}\)  is added to the sensible heat flux later.
\begin{enumerate}
\item {} 
Water vapor flux \(E_{v}\)  (\eqref{equation:tech_note/Fluxes/CLM50_Tech_Note_Fluxes:5.133} )

\item {} 
Transpiration \(E_{v}^{t}\)  (\eqref{equation:tech_note/Fluxes/CLM50_Tech_Note_Fluxes:5.134} if \(\beta_{t} >0\),
otherwise \(E_{v}^{t} =0\))

\item {} 
The water vapor flux \(E_{v}\)  is constrained to be less than or
equal to the sum of transpiration \(E_{v}^{t}\)  and the water
available from wetted leaves and stems
\({W_{can} \mathord{\left/ {\vphantom {W_{can}  \Delta t}} \right. \kern-\nulldelimiterspace} \Delta t}\) .
The energy error due to this constraint is
\phantomsection\label{\detokenize{tech_note/Fluxes/CLM50_Tech_Note_Fluxes:equation-5.139}}\begin{equation}\label{equation:tech_note/Fluxes/CLM50_Tech_Note_Fluxes:5.139}
\begin{split}\Delta _{3} =\max \left(0,\, E_{v} -E_{v}^{t} -\frac{W_{can} }{\Delta t} \right).\end{split}
\end{equation}
\end{enumerate}

The error \(\lambda \Delta _{3}\)  is added to the sensible heat
flux later.
\begin{enumerate}
\item {} 
Sensible heat flux \(H_{v}\)  (:eq:{}`{}` ). The three energy error
terms, \(\Delta _{1}\) , \(\Delta _{2}\) , and
\(\lambda \Delta _{3}\)  are also added to the sensible heat
flux.

\item {} 
The saturated vapor pressure \(e_{i}\)  (Chapter
\hyperref[\detokenize{tech_note/Photosynthesis/CLM50_Tech_Note_Photosynthesis:rst-stomatal-resistance-and-photosynthesis}]{\ref{\detokenize{tech_note/Photosynthesis/CLM50_Tech_Note_Photosynthesis:rst-stomatal-resistance-and-photosynthesis}}}), saturated
specific humidity \(q_{sat}^{T_{v} }\)  and its derivative
\(\frac{dq_{sat}^{T_{v} } }{dT_{v} }\)  at the leaf surface
(section \hyperref[\detokenize{tech_note/Fluxes/CLM50_Tech_Note_Fluxes:saturation-vapor-pressure}]{\ref{\detokenize{tech_note/Fluxes/CLM50_Tech_Note_Fluxes:saturation-vapor-pressure}}}), are re-evaluated based on
the new \(T_{v}\) .

\item {} 
Canopy air temperature \(T_{s}\)  (\eqref{equation:tech_note/Fluxes/CLM50_Tech_Note_Fluxes:5.93} )

\item {} 
Canopy air specific humidity \(q_{s}\)  (\eqref{equation:tech_note/Fluxes/CLM50_Tech_Note_Fluxes:5.107} )

\item {} 
Temperature difference \(\theta _{atm} -\theta _{s}\)

\item {} 
Specific humidity difference \(q_{atm} -q_{s}\)

\item {} 
Potential temperature scale
\(\theta _{*} =\frac{\theta _{*} }{\theta _{atm} -\theta _{s} } \left(\theta _{atm} -\theta _{s} \right)\)
where \(\frac{\theta _{*} }{\theta _{atm} -\theta _{s} }\)  was
calculated earlier in the iteration

\item {} 
Humidity scale
\(q_{*} =\frac{q_{*} }{q_{atm} -q_{s} } \left(q_{atm} -q_{s} \right)\)
where \(\frac{q_{*} }{q_{atm} -q_{s} }\)  was calculated earlier
in the iteration

\item {} 
Virtual potential temperature scale \(\theta _{v*}\)  (\eqref{equation:tech_note/Fluxes/CLM50_Tech_Note_Fluxes:5.17} )

\item {} 
Wind speed including the convective velocity, \(V_{a}\)  (\eqref{equation:tech_note/Fluxes/CLM50_Tech_Note_Fluxes:5.24} )

\item {} 
Monin-Obukhov length \(L\) (\eqref{equation:tech_note/Fluxes/CLM50_Tech_Note_Fluxes:5.49} )

\item {} 
The iteration is stopped after two or more steps if
\(\tilde{\Delta }T_{v} <0.01\) and
\(\left|\lambda E_{v}^{n+1} -\lambda E_{v}^{n} \right|<0.1\)
where
\(\tilde{\Delta }T_{v} =\max \left(\left|T_{v}^{n+1} -T_{v}^{n} \right|,\, \left|T_{v}^{n} -T_{v}^{n-1} \right|\right)\),
or after forty iterations have been carried out.

\item {} 
Momentum fluxes \(\tau _{x}\) , \(\tau _{y}\)  (\eqref{equation:tech_note/Fluxes/CLM50_Tech_Note_Fluxes:5.5}, \eqref{equation:tech_note/Fluxes/CLM50_Tech_Note_Fluxes:5.6})

\item {} 
Sensible heat flux from ground \(H_{g}\)  (\eqref{equation:tech_note/Fluxes/CLM50_Tech_Note_Fluxes:5.89} )

\item {} 
Water vapor flux from ground \(E_{g}\)  (\eqref{equation:tech_note/Fluxes/CLM50_Tech_Note_Fluxes:5.102} )

\item {} 
2-m height air temperature \(T_{2m}\) , specific humidity
\(q_{2m}\) , relative humidity \(RH_{2m}\) (\eqref{equation:tech_note/Fluxes/CLM50_Tech_Note_Fluxes:5.58} , \eqref{equation:tech_note/Fluxes/CLM50_Tech_Note_Fluxes:5.59}, \eqref{equation:tech_note/Fluxes/CLM50_Tech_Note_Fluxes:5.60})

\end{enumerate}


\subsection{Update of Ground Sensible and Latent Heat Fluxes}
\label{\detokenize{tech_note/Fluxes/CLM50_Tech_Note_Fluxes:id6}}\label{\detokenize{tech_note/Fluxes/CLM50_Tech_Note_Fluxes:update-of-ground-sensible-and-latent-heat-fluxes}}
The sensible and water vapor heat fluxes derived above for bare soil and
soil beneath canopy are based on the ground surface temperature from the
previous time step \(T_{g}^{n}\)  and are used as the surface
forcing for the solution of the soil temperature equations (section
\hyperref[\detokenize{tech_note/Soil_Snow_Temperatures/CLM50_Tech_Note_Soil_Snow_Temperatures:numerical-solution-temperature}]{\ref{\detokenize{tech_note/Soil_Snow_Temperatures/CLM50_Tech_Note_Soil_Snow_Temperatures:numerical-solution-temperature}}}). This solution yields a new ground
surface temperature \(T_{g}^{n+1}\) . The ground sensible and water
vapor fluxes are then updated for \(T_{g}^{n+1}\)  as
\phantomsection\label{\detokenize{tech_note/Fluxes/CLM50_Tech_Note_Fluxes:equation-5.140}}\begin{equation}\label{equation:tech_note/Fluxes/CLM50_Tech_Note_Fluxes:5.140}
\begin{split}H'_{g} =H_{g} +\left(T_{g}^{n+1} -T_{g}^{n} \right)\frac{\partial H_{g} }{\partial T_{g} }\end{split}
\end{equation}\phantomsection\label{\detokenize{tech_note/Fluxes/CLM50_Tech_Note_Fluxes:equation-5.141}}\begin{equation}\label{equation:tech_note/Fluxes/CLM50_Tech_Note_Fluxes:5.141}
\begin{split}E'_{g} =E_{g} +\left(T_{g}^{n+1} -T_{g}^{n} \right)\frac{\partial E_{g} }{\partial T_{g} }\end{split}
\end{equation}
where \(H_{g}\)  and \(E_{g}\)  are the sensible heat and water
vapor fluxes derived from equations and for non-vegetated surfaces and
equations and for vegetated surfaces using \(T_{g}^{n}\) . One
further adjustment is made to \(H'_{g}\)  and \(E'_{g}\) . If
the soil moisture in the top snow/soil layer is not sufficient to
support the updated ground evaporation, i.e., if \(E'_{g} > 0\) and
\(f_{evap} < 1\) where
\phantomsection\label{\detokenize{tech_note/Fluxes/CLM50_Tech_Note_Fluxes:equation-5.142}}\begin{equation}\label{equation:tech_note/Fluxes/CLM50_Tech_Note_Fluxes:5.142}
\begin{split}f_{evap} =\frac{{\left(w_{ice,\; snl+1} +w_{liq,\, snl+1} \right)\mathord{\left/ {\vphantom {\left(w_{ice,\; snl+1} +w_{liq,\, snl+1} \right) \Delta t}} \right. \kern-\nulldelimiterspace} \Delta t} }{\sum _{j=1}^{npft}\left(E'_{g} \right)_{j} \left(wt\right)_{j}  } \le 1,\end{split}
\end{equation}
an adjustment is made to reduce the ground evaporation accordingly as
\phantomsection\label{\detokenize{tech_note/Fluxes/CLM50_Tech_Note_Fluxes:equation-5.143}}\begin{equation}\label{equation:tech_note/Fluxes/CLM50_Tech_Note_Fluxes:5.143}
\begin{split}E''_{g} =f_{evap} E'_{g} .\end{split}
\end{equation}
The term
\(\sum _{j=1}^{npft}\left(E'_{g} \right)_{j} \left(wt\right)_{j}\)
is the sum of \(E'_{g}\)  over all evaporating PFTs where
\(\left(E'_{g} \right)_{j}\)  is the ground evaporation from the
\(j^{th}\)  PFT on the column, \(\left(wt\right)_{j}\)  is the
relative area of the \(j^{th}\)  PFT with respect to the column, and
\(npft\) is the number of PFTs on the column.
\(w_{ice,\, snl+1}\)  and \(w_{liq,\, snl+1}\)  are the ice and
liquid water contents (kg m$^{\text{-2}}$) of the top snow/soil layer
(Chapter \hyperref[\detokenize{tech_note/Hydrology/CLM50_Tech_Note_Hydrology:rst-hydrology}]{\ref{\detokenize{tech_note/Hydrology/CLM50_Tech_Note_Hydrology:rst-hydrology}}}). Any resulting energy deficit is assigned
to sensible heat
as
\phantomsection\label{\detokenize{tech_note/Fluxes/CLM50_Tech_Note_Fluxes:equation-5.144}}\begin{equation}\label{equation:tech_note/Fluxes/CLM50_Tech_Note_Fluxes:5.144}
\begin{split}H''_{g} =H_{g} +\lambda \left(E'_{g} -E''_{g} \right).\end{split}
\end{equation}
The ground water vapor flux \(E''_{g}\)  is partitioned into evaporation
of liquid water from snow/soil \(q_{seva}\) (kgm$^{\text{-2}}$ s$^{\text{-1}}$),
sublimation from snow/soil ice \(q_{subl}\)  (kg m$^{\text{-2}}$ s$^{\text{-1}}$),
liquid dew on snow/soil \(q_{sdew}\)  (kg m$^{\text{-2}}$ s$^{\text{-1}}$), or
frost on snow/soil \(q_{frost}\)  (kg m$^{\text{-2}}$ s$^{\text{-1}}$) as
\phantomsection\label{\detokenize{tech_note/Fluxes/CLM50_Tech_Note_Fluxes:equation-5.145}}\begin{equation}\label{equation:tech_note/Fluxes/CLM50_Tech_Note_Fluxes:5.145}
\begin{split}q_{seva} =\max \left(E''_{sno} \frac{w_{liq,\, snl+1} }{w_{ice,\; snl+1} +w_{liq,\, snl+1} } ,0\right)\qquad E''_{sno} \ge 0,\, w_{ice,\; snl+1} +w_{liq,\, snl+1} >0\end{split}
\end{equation}\phantomsection\label{\detokenize{tech_note/Fluxes/CLM50_Tech_Note_Fluxes:equation-5.146}}\begin{equation}\label{equation:tech_note/Fluxes/CLM50_Tech_Note_Fluxes:5.146}
\begin{split}q_{subl} =E''_{sno} -q_{seva} \qquad E''_{sno} \ge 0\end{split}
\end{equation}\phantomsection\label{\detokenize{tech_note/Fluxes/CLM50_Tech_Note_Fluxes:equation-5.147}}\begin{equation}\label{equation:tech_note/Fluxes/CLM50_Tech_Note_Fluxes:5.147}
\begin{split}q_{sdew} =\left|E''_{sno} \right|\qquad E''_{sno} <0{\rm \; and\; }T_{g} \ge T_{f}\end{split}
\end{equation}\phantomsection\label{\detokenize{tech_note/Fluxes/CLM50_Tech_Note_Fluxes:equation-5.148}}\begin{equation}\label{equation:tech_note/Fluxes/CLM50_Tech_Note_Fluxes:5.148}
\begin{split}q_{frost} =\left|E''_{sno} \right|\qquad E''_{sno} <0{\rm \; and\; }T_{g} <T_{f} .\end{split}
\end{equation}
The loss or gain in snow mass due to \(q_{seva}\) ,
\(q_{subl}\) , \(q_{sdew}\) , and \(q_{frost}\)  on a snow
surface are accounted for during the snow hydrology calculations
(Chapter \hyperref[\detokenize{tech_note/Snow_Hydrology/CLM50_Tech_Note_Snow_Hydrology:rst-snow-hydrology}]{\ref{\detokenize{tech_note/Snow_Hydrology/CLM50_Tech_Note_Snow_Hydrology:rst-snow-hydrology}}}). The loss of soil and surface water due to
\(q_{seva}\)  is accounted for in the calculation of infiltration
(section \hyperref[\detokenize{tech_note/Hydrology/CLM50_Tech_Note_Hydrology:infiltration}]{\ref{\detokenize{tech_note/Hydrology/CLM50_Tech_Note_Hydrology:infiltration}}}), while losses or gains due to \(q_{subl}\) ,
\(q_{sdew}\) , and \(q_{frost}\)  on a soil surface are
accounted for following the sub-surface drainage calculations (section
\hyperref[\detokenize{tech_note/Hydrology/CLM50_Tech_Note_Hydrology:lateral-sub-surface-runoff}]{\ref{\detokenize{tech_note/Hydrology/CLM50_Tech_Note_Hydrology:lateral-sub-surface-runoff}}}).

The ground heat flux \(G\) is calculated as
\phantomsection\label{\detokenize{tech_note/Fluxes/CLM50_Tech_Note_Fluxes:equation-5.149}}\begin{equation}\label{equation:tech_note/Fluxes/CLM50_Tech_Note_Fluxes:5.149}
\begin{split}G=\overrightarrow{S}_{g} -\overrightarrow{L}_{g} -H_{g} -\lambda E_{g}\end{split}
\end{equation}
where \(\overrightarrow{S}_{g}\)  is the solar radiation absorbed by
the ground (section \hyperref[\detokenize{tech_note/Radiative_Fluxes/CLM50_Tech_Note_Radiative_Fluxes:solar-fluxes}]{\ref{\detokenize{tech_note/Radiative_Fluxes/CLM50_Tech_Note_Radiative_Fluxes:solar-fluxes}}}), \(\overrightarrow{L}_{g}\)  is the net
longwave radiation absorbed by the ground (section \hyperref[\detokenize{tech_note/Radiative_Fluxes/CLM50_Tech_Note_Radiative_Fluxes:longwave-fluxes}]{\ref{\detokenize{tech_note/Radiative_Fluxes/CLM50_Tech_Note_Radiative_Fluxes:longwave-fluxes}}})
\phantomsection\label{\detokenize{tech_note/Fluxes/CLM50_Tech_Note_Fluxes:equation-5.150}}\begin{equation}\label{equation:tech_note/Fluxes/CLM50_Tech_Note_Fluxes:5.150}
\begin{split}\vec{L}_{g} =L_{g} \uparrow -\delta _{veg} \varepsilon _{g} L_{v} \, \downarrow -\left(1-\delta _{veg} \right)\varepsilon _{g} L_{atm} \, \downarrow +4\varepsilon _{g} \sigma \left(T_{g}^{n} \right)^{3} \left(T_{g}^{n+1} -T_{g}^{n} \right),\end{split}
\end{equation}
where
\phantomsection\label{\detokenize{tech_note/Fluxes/CLM50_Tech_Note_Fluxes:equation-5.151}}\begin{equation}\label{equation:tech_note/Fluxes/CLM50_Tech_Note_Fluxes:5.151}
\begin{split}L_{g} \uparrow =\varepsilon _{g} \sigma \left[\left(1-f_{sno} -f_{h2osfc} \right)\left(T_{1}^{n} \right)^{4} +f_{sno} \left(T_{sno}^{n} \right)^{4} +f_{h2osfc} \left(T_{h2osfc}^{n} \right)^{4} \right]\end{split}
\end{equation}
and \(H_{g}\)  and \(\lambda E_{g}\)  are the sensible and
latent heat fluxes after the adjustments described above.

When converting ground water vapor flux to an energy flux, the term
\(\lambda\)  is arbitrarily assumed to be
\phantomsection\label{\detokenize{tech_note/Fluxes/CLM50_Tech_Note_Fluxes:equation-5.152}}\begin{equation}\label{equation:tech_note/Fluxes/CLM50_Tech_Note_Fluxes:5.152}
\begin{split}\lambda =\left\{\begin{array}{l} {\lambda _{sub} \qquad {\rm if\; }w_{liq,\, snl+1} =0{\rm \; and\; }w_{ice,\, snl+1} >0} \\ {\lambda _{vap} \qquad {\rm otherwise}} \end{array}\right\}\end{split}
\end{equation}
where \(\lambda _{sub}\)  and \(\lambda _{vap}\)  are the latent
heat of sublimation and vaporization, respectively (J
(kg$^{\text{-1}}$) (\hyperref[\detokenize{tech_note/Ecosystem/CLM50_Tech_Note_Ecosystem:table-physical-constants}]{Table \ref{\detokenize{tech_note/Ecosystem/CLM50_Tech_Note_Ecosystem:table-physical-constants}}}). When converting vegetation water vapor
flux to an energy flux, \(\lambda _{vap}\)  is used.

The system balances energy as
\phantomsection\label{\detokenize{tech_note/Fluxes/CLM50_Tech_Note_Fluxes:equation-5.153}}\begin{equation}\label{equation:tech_note/Fluxes/CLM50_Tech_Note_Fluxes:5.153}
\begin{split}\overrightarrow{S}_{g} +\overrightarrow{S}_{v} +L_{atm} \, \downarrow -L\, \uparrow -H_{v} -H_{g} -\lambda _{vap} E_{v} -\lambda E_{g} -G=0.\end{split}
\end{equation}

\subsection{Saturation Vapor Pressure}
\label{\detokenize{tech_note/Fluxes/CLM50_Tech_Note_Fluxes:id7}}\label{\detokenize{tech_note/Fluxes/CLM50_Tech_Note_Fluxes:saturation-vapor-pressure}}
Saturation vapor pressure \(e_{sat}^{T}\)  (Pa) and its derivative
\(\frac{de_{sat}^{T} }{dT}\) , as a function of temperature
\(T\) (ºC), are calculated from the eighth-order polynomial fits of
{\hyperref[\detokenize{tech_note/References/CLM50_Tech_Note_References:flatauetal1992}]{\sphinxcrossref{\DUrole{std,std-ref}{Flatau et al. (1992)}}}}
\phantomsection\label{\detokenize{tech_note/Fluxes/CLM50_Tech_Note_Fluxes:equation-5.154}}\begin{equation}\label{equation:tech_note/Fluxes/CLM50_Tech_Note_Fluxes:5.154}
\begin{split}e_{sat}^{T} =100\left[a_{0} +a_{1} T+\cdots +a_{n} T^{n} \right]\end{split}
\end{equation}\phantomsection\label{\detokenize{tech_note/Fluxes/CLM50_Tech_Note_Fluxes:equation-5.155}}\begin{equation}\label{equation:tech_note/Fluxes/CLM50_Tech_Note_Fluxes:5.155}
\begin{split}\frac{de_{sat}^{T} }{dT} =100\left[b_{0} +b_{1} T+\cdots +b_{n} T^{n} \right]\end{split}
\end{equation}
where the coefficients for ice are valid for
\(-75\, ^{\circ } {\rm C}\le T<0\, ^{\circ } {\rm C}\) and the
coefficients for water are valid for
\(0\, ^{\circ } {\rm C}\le T\le 100\, ^{\circ } {\rm C}\)
(\hyperref[\detokenize{tech_note/Fluxes/CLM50_Tech_Note_Fluxes:table-coefficients-for-saturation-vapor-pressure}]{Table \ref{\detokenize{tech_note/Fluxes/CLM50_Tech_Note_Fluxes:table-coefficients-for-saturation-vapor-pressure}}} and
\hyperref[\detokenize{tech_note/Fluxes/CLM50_Tech_Note_Fluxes:table-coefficients-for-derivative-of-esat}]{Table \ref{\detokenize{tech_note/Fluxes/CLM50_Tech_Note_Fluxes:table-coefficients-for-derivative-of-esat}}}).
The saturated water vapor specific humidity \(q_{sat}^{T}\) and its derivative
\(\frac{dq_{sat}^{T} }{dT}\) are
\phantomsection\label{\detokenize{tech_note/Fluxes/CLM50_Tech_Note_Fluxes:equation-5.156}}\begin{equation}\label{equation:tech_note/Fluxes/CLM50_Tech_Note_Fluxes:5.156}
\begin{split}q_{sat}^{T} =\frac{0.622e_{sat}^{T} }{P_{atm} -0.378e_{sat}^{T} }\end{split}
\end{equation}\phantomsection\label{\detokenize{tech_note/Fluxes/CLM50_Tech_Note_Fluxes:equation-5.157}}\begin{equation}\label{equation:tech_note/Fluxes/CLM50_Tech_Note_Fluxes:5.157}
\begin{split}\frac{dq_{sat}^{T} }{dT} =\frac{0.622P_{atm} }{\left(P_{atm} -0.378e_{sat}^{T} \right)^{2} } \frac{de_{sat}^{T} }{dT} .\end{split}
\end{equation}

\begin{savenotes}\sphinxattablestart
\centering
\sphinxcapstartof{table}
\sphinxcaption{Coefficients for \(e_{sat}^{T}\)}\label{\detokenize{tech_note/Fluxes/CLM50_Tech_Note_Fluxes:table-coefficients-for-saturation-vapor-pressure}}\label{\detokenize{tech_note/Fluxes/CLM50_Tech_Note_Fluxes:id11}}
\sphinxaftercaption
\begin{tabulary}{\linewidth}[t]{|T|T|T|T|}
\hline
\sphinxstylethead{\sphinxstyletheadfamily \unskip}\relax &\sphinxstylethead{\sphinxstyletheadfamily 
water
\unskip}\relax &\sphinxstartmulticolumn{2}%
\begin{varwidth}[t]{\sphinxcolwidth{2}{4}}
\sphinxstylethead{\sphinxstyletheadfamily ice
\unskip}\relax \par
\vskip-\baselineskip\strut\end{varwidth}%
\sphinxstopmulticolumn
\\
\hline
\(a_{0}\)
&
6.11213476
&\sphinxstartmulticolumn{2}%
\begin{varwidth}[t]{\sphinxcolwidth{2}{4}}
6.11123516
\par
\vskip-\baselineskip\strut\end{varwidth}%
\sphinxstopmulticolumn
\\
\hline
\(a_{1}\)
&\sphinxstartmulticolumn{3}%
\begin{varwidth}[t]{\sphinxcolwidth{3}{4}}
4.44007856 \(\times 10^{-1}\)        \textbar{} 5.03109514\(\times 10^{-1}\)
\par
\vskip-\baselineskip\strut\end{varwidth}%
\sphinxstopmulticolumn
\\
\hline
\(a_{2}\)
&\sphinxstartmulticolumn{3}%
\begin{varwidth}[t]{\sphinxcolwidth{3}{4}}
1.43064234 \(\times 10^{-2}\)        \textbar{} 1.88369801\(\times 10^{-2}\)
\par
\vskip-\baselineskip\strut\end{varwidth}%
\sphinxstopmulticolumn
\\
\hline
\(a_{3}\)
&\sphinxstartmulticolumn{3}%
\begin{varwidth}[t]{\sphinxcolwidth{3}{4}}
2.64461437 \(\times 10^{-4}\)        \textbar{} 4.20547422\(\times 10^{-4}\)
\par
\vskip-\baselineskip\strut\end{varwidth}%
\sphinxstopmulticolumn
\\
\hline
\(a_{4}\)
&\sphinxstartmulticolumn{3}%
\begin{varwidth}[t]{\sphinxcolwidth{3}{4}}
3.05903558 \(\times 10^{-6}\)        \textbar{} 6.14396778\(\times 10^{-6}\)
\par
\vskip-\baselineskip\strut\end{varwidth}%
\sphinxstopmulticolumn
\\
\hline
\(a_{5}\)
&\sphinxstartmulticolumn{3}%
\begin{varwidth}[t]{\sphinxcolwidth{3}{4}}
1.96237241 \(\times 10^{-8}\)        \textbar{} 6.02780717\(\times 10^{-8}\)
\par
\vskip-\baselineskip\strut\end{varwidth}%
\sphinxstopmulticolumn
\\
\hline
\(a_{6}\)
&\sphinxstartmulticolumn{3}%
\begin{varwidth}[t]{\sphinxcolwidth{3}{4}}
8.92344772 \(\times 10^{-11}\)       \textbar{} 3.87940929\(\times 10^{-10}\)
\par
\vskip-\baselineskip\strut\end{varwidth}%
\sphinxstopmulticolumn
\\
\hline
\(a_{7}\)
&\sphinxstartmulticolumn{3}%
\begin{varwidth}[t]{\sphinxcolwidth{3}{4}}
-3.73208410 \(\times 10^{-13}\)      \textbar{} 1.49436277\(\times 10^{-12}\)
\par
\vskip-\baselineskip\strut\end{varwidth}%
\sphinxstopmulticolumn
\\
\hline
\(a_{8}\)
&\sphinxstartmulticolumn{3}%
\begin{varwidth}[t]{\sphinxcolwidth{3}{4}}
2.09339997 \(\times 10^{-16}\)       \textbar{} 2.62655803\(\times 10^{-15}\)
\par
\vskip-\baselineskip\strut\end{varwidth}%
\sphinxstopmulticolumn
\\
\hline
\end{tabulary}
\par
\sphinxattableend\end{savenotes}


\begin{savenotes}\sphinxattablestart
\centering
\sphinxcapstartof{table}
\sphinxcaption{Coefficients for \(\frac{de_{sat}^{T} }{dT}\)}\label{\detokenize{tech_note/Fluxes/CLM50_Tech_Note_Fluxes:table-coefficients-for-derivative-of-esat}}\label{\detokenize{tech_note/Fluxes/CLM50_Tech_Note_Fluxes:id12}}
\sphinxaftercaption
\begin{tabulary}{\linewidth}[t]{|T|T|T|}
\hline
\sphinxstylethead{\sphinxstyletheadfamily \unskip}\relax &\sphinxstylethead{\sphinxstyletheadfamily 
water
\unskip}\relax &\sphinxstylethead{\sphinxstyletheadfamily 
ice
\unskip}\relax \\
\hline
\(b_{0}\)
&
4.44017302\(\times 10^{-1}\)
&
5.03277922\(\times 10^{-1}\)
\\
\hline
\(b_{1}\)
&
2.86064092\(\times 10^{-2}\)
&
3.77289173\(\times 10^{-2}\)
\\
\hline
\(b_{2}\)
&
7.94683137\(\times 10^{-4}\)
&
1.26801703\(\times 10^{-3}\)
\\
\hline
\(b_{3}\)
&
1.21211669\(\times 10^{-5}\)
&
2.49468427\(\times 10^{-5}\)
\\
\hline
\(b_{4}\)
&
1.03354611\(\times 10^{-7}\)
&
3.13703411\(\times 10^{-7}\)
\\
\hline
\(b_{5}\)
&
4.04125005\(\times 10^{-10}\)
&
2.57180651\(\times 10^{-9}\)
\\
\hline
\(b_{6}\)
&
-7.88037859 \(\times 10^{-13}\)
&
1.33268878\(\times 10^{-11}\)
\\
\hline
\(b_{7}\)
&
-1.14596802 \(\times 10^{-14}\)
&
3.94116744\(\times 10^{-14}\)
\\
\hline
\(b_{8}\)
&
3.81294516\(\times 10^{-17}\)
&
4.98070196\(\times 10^{-17}\)
\\
\hline
\end{tabulary}
\par
\sphinxattableend\end{savenotes}


\section{Soil and Snow Temperatures}
\label{\detokenize{tech_note/Soil_Snow_Temperatures/CLM50_Tech_Note_Soil_Snow_Temperatures:soil-and-snow-temperatures}}\label{\detokenize{tech_note/Soil_Snow_Temperatures/CLM50_Tech_Note_Soil_Snow_Temperatures::doc}}\label{\detokenize{tech_note/Soil_Snow_Temperatures/CLM50_Tech_Note_Soil_Snow_Temperatures:rst-soil-and-snow-temperatures}}
The first law of heat conduction is
\phantomsection\label{\detokenize{tech_note/Soil_Snow_Temperatures/CLM50_Tech_Note_Soil_Snow_Temperatures:equation-6.1}}\begin{equation}\label{equation:tech_note/Soil_Snow_Temperatures/CLM50_Tech_Note_Soil_Snow_Temperatures:6.1}
\begin{split}F=-\lambda \nabla T\end{split}
\end{equation}
where \(F\) is the amount of heat conducted across a unit
cross-sectional area in unit time (W m$^{\text{-2}}$),
\(\lambda\)  is thermal conductivity (W m$^{\text{-1}}$
K$^{\text{-1}}$), and \(\nabla T\) is the spatial gradient of
temperature (K m$^{\text{-1}}$). In one-dimensional form
\phantomsection\label{\detokenize{tech_note/Soil_Snow_Temperatures/CLM50_Tech_Note_Soil_Snow_Temperatures:equation-6.2}}\begin{equation}\label{equation:tech_note/Soil_Snow_Temperatures/CLM50_Tech_Note_Soil_Snow_Temperatures:6.2}
\begin{split}F_{z} =-\lambda \frac{\partial T}{\partial z}\end{split}
\end{equation}
where \(z\) is in the vertical direction (m) and is positive
downward and \(F_{z}\)  is positive upward. To account for
non-steady or transient conditions, the principle of energy conservation
in the form of the continuity equation is invoked as
\phantomsection\label{\detokenize{tech_note/Soil_Snow_Temperatures/CLM50_Tech_Note_Soil_Snow_Temperatures:equation-6.3}}\begin{equation}\label{equation:tech_note/Soil_Snow_Temperatures/CLM50_Tech_Note_Soil_Snow_Temperatures:6.3}
\begin{split}c\frac{\partial T}{\partial t} =-\frac{\partial F_{z} }{\partial z}\end{split}
\end{equation}
where \(c\) is the volumetric snow/soil heat capacity (J
m$^{\text{-3}}$ K$^{\text{-1}}$) and \(t\) is time (s).
Combining equations and yields the second law of heat conduction in
one-dimensional form
\phantomsection\label{\detokenize{tech_note/Soil_Snow_Temperatures/CLM50_Tech_Note_Soil_Snow_Temperatures:equation-6.4}}\begin{equation}\label{equation:tech_note/Soil_Snow_Temperatures/CLM50_Tech_Note_Soil_Snow_Temperatures:6.4}
\begin{split}c\frac{\partial T}{\partial t} =\frac{\partial }{\partial z} \left[\lambda \frac{\partial T}{\partial z} \right].\end{split}
\end{equation}
This equation is solved numerically to calculate the soil, snow, and
surface water temperatures for a fifteen-layer soil column with up to
five overlying layers of snow and a single surface water layer with the
boundary conditions of \(h\) as the heat flux into the top soil,
snow, and surface water layers from the overlying atmosphere (section
\hyperref[\detokenize{tech_note/Soil_Snow_Temperatures/CLM50_Tech_Note_Soil_Snow_Temperatures:numerical-solution-temperature}]{\ref{\detokenize{tech_note/Soil_Snow_Temperatures/CLM50_Tech_Note_Soil_Snow_Temperatures:numerical-solution-temperature}}}) and zero heat flux at the bottom
of the soil column. The temperature profile is calculated first without
phase change and then readjusted for phase change (section \hyperref[\detokenize{tech_note/Soil_Snow_Temperatures/CLM50_Tech_Note_Soil_Snow_Temperatures:phase-change}]{\ref{\detokenize{tech_note/Soil_Snow_Temperatures/CLM50_Tech_Note_Soil_Snow_Temperatures:phase-change}}}).


\subsection{Numerical Solution}
\label{\detokenize{tech_note/Soil_Snow_Temperatures/CLM50_Tech_Note_Soil_Snow_Temperatures:numerical-solution-temperature}}\label{\detokenize{tech_note/Soil_Snow_Temperatures/CLM50_Tech_Note_Soil_Snow_Temperatures:numerical-solution}}
The soil column is discretized into 25 layers (section
\hyperref[\detokenize{tech_note/Ecosystem/CLM50_Tech_Note_Ecosystem:vertical-discretization}]{\ref{\detokenize{tech_note/Ecosystem/CLM50_Tech_Note_Ecosystem:vertical-discretization}}}) where \(N_{levgrnd} = 25\) is the
number of soil layers (\hyperref[\detokenize{tech_note/Ecosystem/CLM50_Tech_Note_Ecosystem:table-soil-layer-structure}]{Table \ref{\detokenize{tech_note/Ecosystem/CLM50_Tech_Note_Ecosystem:table-soil-layer-structure}}}).

The overlying snow pack is modeled with up to five layers depending on
the total snow depth. The layers from top to bottom are indexed in the
Fortran code as \(i=-4,-3,-2,-1,0\), which permits the accumulation
or ablation of snow at the top of the snow pack without renumbering the
layers. Layer \(i=0\) is the snow layer next to the soil surface and
layer \(i=snl+1\) is the top layer, where the variable \(snl\)
is the negative of the number of snow layers. The number of snow layers
and the thickness of each layer is a function of snow depth
\(z_{sno}\)  (m) as follows.
\begin{align*}\!\begin{aligned}
\begin{array}{lr}\\
\left\{ \begin{array}{l}
snl=-1 \\
\Delta z_{0} = z_{sno}
\end{array} \right\} & \qquad {\rm for\; 0.01}\le {\rm z}_{{\rm sno}} \le 0.03 \\\\
\left\{ \begin{array}{l}
snl=-2  \\
\Delta z_{-1} ={z_{sno} \mathord{\left/ {\vphantom {z_{sno}  2}} \right. \kern-\nulldelimiterspace} 2} \\
\Delta z_{0} = \Delta z_{-1}
\end{array} \right\} & \qquad {\rm for\; 0.03}\, {\rm <}\, {\rm z}_{{\rm sno}} \le 0.04 \\\\
\left\{ \begin{array}{l}
snl=-2 \\
\Delta z_{-1} = 0.02 \\
\Delta z_{0} = z_{sno} -\Delta z_{-1}
\end{array} \right\} & \qquad {\rm for\; 0.04}\, {\rm <}\, {\rm z}_{{\rm sno}} \le 0.07 \\
\left\{ \begin{array}{l}
snl=-3 \\
\Delta z_{-2} = 0.02 \\
\Delta z_{-1} = {\left(z_{sno} -0.02\right)\mathord{\left/ {\vphantom {\left(z_{sno} -0.02\right) 2}} \right. \kern-\nulldelimiterspace} 2} \\
\Delta z_{0} = \Delta z_{-1}
\end{array} \right\} & \qquad {\rm for\; 0.07}\, {\rm <}\, {\rm z}_{{\rm sno}} \le 0.12 \\\\
\left\{ \begin{array}{l}
snl=-3 \\
\Delta z_{-2} = 0.02 \\
\Delta z_{-1} = 0.05 \\
\Delta z_{0} = z_{sno} -\Delta z_{-2} -\Delta z_{-1}
\end{array} \right\} & \qquad {\rm for\; 0.12}\, {\rm <}\, {\rm z}_{{\rm sno}} \le 0.18 \\\\
\left\{ \begin{array}{l}
snl=-4  \\
\Delta z_{-3} = 0.02 \\
\Delta z_{-2} = 0.05 \\
\Delta z_{-1} = {\left(z_{sno} -\Delta z_{-3} -\Delta z_{-2} \right)\mathord{\left/ {\vphantom {\left(z_{sno} -\Delta z_{-3} -\Delta z_{-2} \right) 2}} \right. \kern-\nulldelimiterspace} 2}  \\
\Delta z_{0} =\Delta z_{-1}
\end{array} \right\} & \qquad {\rm for\; 0.18}\, {\rm <}\, {\rm z}_{{\rm sno}} \le 0.29 \\\\
\left\{ \begin{array}{l}
snl=-4 \\
\Delta z_{-3} = 0.02  \\
\Delta z_{-2} = 0.05  \\
\Delta z_{-1} = 0.11  \\
\Delta z_{0} = z_{sno} -\Delta z_{-3} -\Delta z_{-2} -\Delta z_{-1}
\end{array} \right\} & \qquad {\rm for\; 0.29}\, {\rm <}\, {\rm z}_{{\rm sno}} \le 0.41 \\\\
\left\{ \begin{array}{l}
snl=-5  \\
\Delta z_{-4} = 0.02  \\
\Delta z_{-3} = 0.05  \\
\Delta z_{-2} = 0.11  \\
\Delta z_{-1} = {\left(z_{sno} -\Delta z_{-4} -\Delta z_{-3} -\Delta z_{-2} \right)\mathord{\left/ {\vphantom {\left(z_{sno} -\Delta z_{-4} -\Delta z_{-3} -\Delta z_{-2} \right) 2}} \right. \kern-\nulldelimiterspace} 2}  \\
\Delta z_{0} = \Delta z_{-1}\\
\end{array} \right\} & \qquad {\rm for\; 0.41}\, {\rm <}\, {\rm z}_{{\rm sno}} \le 0.64 \\\\
\left\{ \begin{array}{l}
snl=-5 \\
\Delta z_{-4} = 0.02  \\
\Delta z_{-3} = 0.05  \\
\Delta z_{-2} = 0.11  \\
\Delta z_{-1} = 0.23  \\
\Delta z_{0} = z_{sno} -\Delta z_{-4} -\Delta z_{-3} -\Delta z_{-2} -\Delta z_{-1}
\end{array} \right\} & \qquad {\rm for\; 0.64}\, {\rm <}\, {\rm z}_{{\rm sno}}\\
\end{array}\\
\end{aligned}\end{align*}
The node depths, which are located at the midpoint of the snow layers,
and the layer interfaces are both referenced from the soil surface and
are defined as negative values
\phantomsection\label{\detokenize{tech_note/Soil_Snow_Temperatures/CLM50_Tech_Note_Soil_Snow_Temperatures:equation-6.8}}\begin{equation}\label{equation:tech_note/Soil_Snow_Temperatures/CLM50_Tech_Note_Soil_Snow_Temperatures:6.8}
\begin{split}z_{i} =z_{h,\, i} -0.5\Delta z_{i} \qquad i=snl+1,\ldots ,0\end{split}
\end{equation}\phantomsection\label{\detokenize{tech_note/Soil_Snow_Temperatures/CLM50_Tech_Note_Soil_Snow_Temperatures:equation-6.9}}\begin{equation}\label{equation:tech_note/Soil_Snow_Temperatures/CLM50_Tech_Note_Soil_Snow_Temperatures:6.9}
\begin{split}z_{h,\, i} =z_{h,\, i+1} -\Delta z_{i+1} \qquad i=snl,\ldots ,-1.\end{split}
\end{equation}
Note that \(z_{h,\, 0}\) , the interface between the bottom snow
layer and the top soil layer, is zero. Thermal properties (i.e.,
temperature \(T_{i}\)  {[}K{]}; thermal conductivity
\(\lambda _{i}\)  {[}W m$^{\text{-1}}$ K$^{\text{-1}}${]};
volumetric heat capacity \(c_{i}\)  {[}J m$^{\text{-3}}$
K$^{\text{-1}}${]}) are defined for soil layers at the node depths
(\hyperref[\detokenize{tech_note/Soil_Snow_Temperatures/CLM50_Tech_Note_Soil_Snow_Temperatures:figure-soil-temperature-schematic}]{Figure \ref{\detokenize{tech_note/Soil_Snow_Temperatures/CLM50_Tech_Note_Soil_Snow_Temperatures:figure-soil-temperature-schematic}}}) and for snow layers at the layer midpoints. When present,
snow occupies a fraction of a grid cell’s area, therefore snow depth
represents the thickness of the snowpack averaged over only the snow
covered area. The grid cell average snow depth is related to the depth
of the snow covered area as \(\bar{z}_{sno} =f_{sno} z_{sno}\) . By
default, the grid cell average snow depth is written to the history
file.

The heat flux \(F_{i}\)  (W m$^{\text{-2}}$) from layer \(i\)
to layer \(i+1\) is
\phantomsection\label{\detokenize{tech_note/Soil_Snow_Temperatures/CLM50_Tech_Note_Soil_Snow_Temperatures:equation-6.10}}\begin{equation}\label{equation:tech_note/Soil_Snow_Temperatures/CLM50_Tech_Note_Soil_Snow_Temperatures:6.10}
\begin{split}F_{i} =-\lambda \left[z_{h,\, i} \right]\left(\frac{T_{i} -T_{i+1} }{z_{i+1} -z_{i} } \right)\end{split}
\end{equation}
where the thermal conductivity at the interface
\(\lambda \left[z_{h,\, i} \right]\) is
\phantomsection\label{\detokenize{tech_note/Soil_Snow_Temperatures/CLM50_Tech_Note_Soil_Snow_Temperatures:equation-6.11}}\begin{equation}\label{equation:tech_note/Soil_Snow_Temperatures/CLM50_Tech_Note_Soil_Snow_Temperatures:6.11}
\begin{split}\lambda \left[z_{h,\, i} \right]=\left\{\begin{array}{l} {\frac{\lambda _{i} \lambda _{i+1} \left(z_{i+1} -z_{i} \right)}{\lambda _{i} \left(z_{i+1} -z_{h,\, i} \right)+\lambda _{i+1} \left(z_{h,\, i} -z_{i} \right)} \qquad i=snl+1,\ldots ,N_{levgrnd} -1} \\ {0\qquad i=N_{levgrnd} } \end{array}\right\}.\end{split}
\end{equation}
These equations are derived, with reference to
\hyperref[\detokenize{tech_note/Soil_Snow_Temperatures/CLM50_Tech_Note_Soil_Snow_Temperatures:figure-soil-temperature-schematic}]{Figure \ref{\detokenize{tech_note/Soil_Snow_Temperatures/CLM50_Tech_Note_Soil_Snow_Temperatures:figure-soil-temperature-schematic}}}, assuming
that the heat flux from \(i\) (depth \(z_{i}\) ) to the
interface between \(i\) and \(i+1\) (depth \(z_{h,\, i}\) )
equals the heat flux from the interface to \(i+1\) (depth
\(z_{i+1}\) ), i.e.,
\phantomsection\label{\detokenize{tech_note/Soil_Snow_Temperatures/CLM50_Tech_Note_Soil_Snow_Temperatures:equation-6.12}}\begin{equation}\label{equation:tech_note/Soil_Snow_Temperatures/CLM50_Tech_Note_Soil_Snow_Temperatures:6.12}
\begin{split}-\lambda _{i} \frac{T_{i} -T_{m} }{z_{h,\, i} -z_{i} } =-\lambda _{i+1} \frac{T_{m} -T_{i+1} }{z_{i+1} -z_{h,\, i} }\end{split}
\end{equation}
where \(T_{m}\)  is the temperature at the interface of layers
\(i\) and \(i+1\).

Shown are three soil layers, \(i-1\), \(i\), and \(i+1\).
The thermal conductivity \(\lambda\) , specific heat capacity
\(c\), and temperature \(T\) are defined at the layer node depth
\(z\). \(T_{m}\)  is the interface temperature. The thermal
conductivity \(\lambda \left[z_{h} \right]\) is defined at the
interface of two layers

\(z_{h}\) . The layer thickness is \(\Delta z\). The heat fluxes
\(F_{i-1}\)  and \(F_{i}\)  are defined as positive upwards.

\begin{figure}[htbp]
\centering
\capstart

\noindent\sphinxincludegraphics{{image17}.png}
\caption{Schematic diagram of numerical scheme used to solve for soil temperature.}\label{\detokenize{tech_note/Soil_Snow_Temperatures/CLM50_Tech_Note_Soil_Snow_Temperatures:figure-soil-temperature-schematic}}\label{\detokenize{tech_note/Soil_Snow_Temperatures/CLM50_Tech_Note_Soil_Snow_Temperatures:id5}}\end{figure}

The energy balance for the \(i^{th}\)  layer is
\phantomsection\label{\detokenize{tech_note/Soil_Snow_Temperatures/CLM50_Tech_Note_Soil_Snow_Temperatures:equation-6.13}}\begin{equation}\label{equation:tech_note/Soil_Snow_Temperatures/CLM50_Tech_Note_Soil_Snow_Temperatures:6.13}
\begin{split}\frac{c_{i} \Delta z_{i} }{\Delta t} \left(T_{i}^{n+1} -T_{i}^{n} \right)=-F_{i-1} +F_{i}\end{split}
\end{equation}
where the superscripts \(n\) and \(n+1\) indicate values at the
beginning and end of the time step, respectively, and \(\Delta t\)
is the time step (s). This equation is solved using the Crank-Nicholson
method, which combines the explicit method with fluxes evaluated at
\(n\) (\(F_{i-1}^{n} ,F_{i}^{n}\) ) and the implicit method with
fluxes evaluated at \(n+1\) (\(F_{i-1}^{n+1} ,F_{i}^{n+1}\) )
\phantomsection\label{\detokenize{tech_note/Soil_Snow_Temperatures/CLM50_Tech_Note_Soil_Snow_Temperatures:equation-6.14}}\begin{equation}\label{equation:tech_note/Soil_Snow_Temperatures/CLM50_Tech_Note_Soil_Snow_Temperatures:6.14}
\begin{split}\frac{c_{i} \Delta z_{i} }{\Delta t} \left(T_{i}^{n+1} -T_{i}^{n} \right)=\alpha \left(-F_{i-1}^{n} +F_{i}^{n} \right)+\left(1-\alpha \right)\left(-F_{i-1}^{n+1} +F_{i}^{n+1} \right)\end{split}
\end{equation}
where \(\alpha =0.5\), resulting in a tridiagonal system of
equations
\phantomsection\label{\detokenize{tech_note/Soil_Snow_Temperatures/CLM50_Tech_Note_Soil_Snow_Temperatures:equation-6.15}}\begin{equation}\label{equation:tech_note/Soil_Snow_Temperatures/CLM50_Tech_Note_Soil_Snow_Temperatures:6.15}
\begin{split}r_{i} =a_{i} T_{i-1}^{n+1} +b_{i} T_{i}^{n+1} +c_{i} T_{i+1}^{n+1}\end{split}
\end{equation}
where \(a_{i}\) , \(b_{i}\) , and \(c_{i}\)  are the
subdiagonal, diagonal, and superdiagonal elements in the tridiagonal
matrix and \(r_{i}\)  is a column vector of constants. When surface
water is present, the equation for the top soil layer has an additional
term representing the surface water temperature; this results in a four
element band-diagonal system of equations.

For the top soil layer \(i=1\) , top snow layer \(i=snl+1\), or
surface water layer, the heat flux from the overlying atmosphere
\(h\) (W m$^{\text{-2}}$, defined as positive into the surface)
is
\phantomsection\label{\detokenize{tech_note/Soil_Snow_Temperatures/CLM50_Tech_Note_Soil_Snow_Temperatures:equation-6.16}}\begin{equation}\label{equation:tech_note/Soil_Snow_Temperatures/CLM50_Tech_Note_Soil_Snow_Temperatures:6.16}
\begin{split}h^{n+1} =-\alpha F_{i-1}^{n} -\left(1-\alpha \right)F_{i-1}^{n+1} .\end{split}
\end{equation}
The energy balance for these layers is then
\phantomsection\label{\detokenize{tech_note/Soil_Snow_Temperatures/CLM50_Tech_Note_Soil_Snow_Temperatures:equation-6.17}}\begin{equation}\label{equation:tech_note/Soil_Snow_Temperatures/CLM50_Tech_Note_Soil_Snow_Temperatures:6.17}
\begin{split}\frac{c_{i} \Delta z_{i} }{\Delta t} \left(T_{i}^{n+1} -T_{i}^{n} \right)=h^{n+1} +\alpha F_{i}^{n} +\left(1-\alpha \right)F_{i}^{n+1} .\end{split}
\end{equation}
The heat flux \(h\) at \(n+1\) may be approximated as follows
\phantomsection\label{\detokenize{tech_note/Soil_Snow_Temperatures/CLM50_Tech_Note_Soil_Snow_Temperatures:equation-6.18}}\begin{equation}\label{equation:tech_note/Soil_Snow_Temperatures/CLM50_Tech_Note_Soil_Snow_Temperatures:6.18}
\begin{split}h^{n+1} =h^{n} +\frac{\partial h}{\partial T_{i} } \left(T_{i}^{n+1} -T_{i}^{n} \right).\end{split}
\end{equation}
The resulting equations are then
\phantomsection\label{\detokenize{tech_note/Soil_Snow_Temperatures/CLM50_Tech_Note_Soil_Snow_Temperatures:equation-6.19}}\begin{equation}\label{equation:tech_note/Soil_Snow_Temperatures/CLM50_Tech_Note_Soil_Snow_Temperatures:6.19}
\begin{split}\begin{array}{rcl} {\frac{c_{i} \Delta z_{i} }{\Delta t} \left(T_{i}^{n+1} -T_{i}^{n} \right)} & {=} & {h^{n} +\frac{\partial h}{\partial T_{i} } \left(T_{i}^{n+1} -T_{i} \right)} \\ {} & {} & {-\alpha \frac{\lambda \left[z_{h,\, i} \right]\left(T_{i}^{n} -T_{i+1}^{n} \right)}{z_{i+1} -z_{i} } -\left(1-\alpha \right)\frac{\lambda \left[z_{h,\, i} \right]\left(T_{i}^{n+1} -T_{i+1}^{n+1} \right)}{z_{i+1} -z_{i} } } \end{array}\end{split}
\end{equation}
For the top snow layer, \(i=snl+1\), the coefficients are
\phantomsection\label{\detokenize{tech_note/Soil_Snow_Temperatures/CLM50_Tech_Note_Soil_Snow_Temperatures:equation-6.20}}\begin{equation}\label{equation:tech_note/Soil_Snow_Temperatures/CLM50_Tech_Note_Soil_Snow_Temperatures:6.20}
\begin{split}a_{i} =0\end{split}
\end{equation}\phantomsection\label{\detokenize{tech_note/Soil_Snow_Temperatures/CLM50_Tech_Note_Soil_Snow_Temperatures:equation-6.21}}\begin{equation}\label{equation:tech_note/Soil_Snow_Temperatures/CLM50_Tech_Note_Soil_Snow_Temperatures:6.21}
\begin{split}b_{i} =1+\frac{\Delta t}{c_{i} \Delta z_{i} } \left[\left(1-\alpha \right)\frac{\lambda \left[z_{h,\, i} \right]}{z_{i+1} -z_{i} } -\frac{\partial h}{\partial T_{i} } \right]\end{split}
\end{equation}\phantomsection\label{\detokenize{tech_note/Soil_Snow_Temperatures/CLM50_Tech_Note_Soil_Snow_Temperatures:equation-6.22}}\begin{equation}\label{equation:tech_note/Soil_Snow_Temperatures/CLM50_Tech_Note_Soil_Snow_Temperatures:6.22}
\begin{split}c_{i} =-\left(1-\alpha \right)\frac{\Delta t}{c_{i} \Delta z_{i} } \frac{\lambda \left[z_{h,\, i} \right]}{z_{i+1} -z_{i} }\end{split}
\end{equation}\phantomsection\label{\detokenize{tech_note/Soil_Snow_Temperatures/CLM50_Tech_Note_Soil_Snow_Temperatures:equation-6.23}}\begin{equation}\label{equation:tech_note/Soil_Snow_Temperatures/CLM50_Tech_Note_Soil_Snow_Temperatures:6.23}
\begin{split}r_{i} =T_{i}^{n} +\frac{\Delta t}{c_{i} \Delta z_{i} } \left[h_{sno} ^{n} -\frac{\partial h}{\partial T_{i} } T_{i}^{n} +\alpha F_{i} \right]\end{split}
\end{equation}
where
\phantomsection\label{\detokenize{tech_note/Soil_Snow_Temperatures/CLM50_Tech_Note_Soil_Snow_Temperatures:equation-6.24}}\begin{equation}\label{equation:tech_note/Soil_Snow_Temperatures/CLM50_Tech_Note_Soil_Snow_Temperatures:6.24}
\begin{split}F_{i} =-\lambda \left[z_{h,\, i} \right]\left(\frac{T_{i}^{n} -T_{i+1}^{n} }{z_{i+1} -z_{i} } \right).\end{split}
\end{equation}
The heat flux into the snow surface from the overlying atmosphere
\(h\) is
\phantomsection\label{\detokenize{tech_note/Soil_Snow_Temperatures/CLM50_Tech_Note_Soil_Snow_Temperatures:equation-6.25}}\begin{equation}\label{equation:tech_note/Soil_Snow_Temperatures/CLM50_Tech_Note_Soil_Snow_Temperatures:6.25}
\begin{split}h=\overrightarrow{S}_{sno} -\overrightarrow{L}_{sno} -H_{sno} -\lambda E_{sno}\end{split}
\end{equation}
where \(\overrightarrow{S}_{sno}\)  is the solar radiation absorbed
by the top snow layer (section \hyperref[\detokenize{tech_note/Surface_Albedos/CLM50_Tech_Note_Surface_Albedos:snow-albedo}]{\ref{\detokenize{tech_note/Surface_Albedos/CLM50_Tech_Note_Surface_Albedos:snow-albedo}}}), \(\overrightarrow{L}_{sno}\)
is the longwave radiation absorbed by the snow (positive toward the
atmosphere) (section \hyperref[\detokenize{tech_note/Radiative_Fluxes/CLM50_Tech_Note_Radiative_Fluxes:longwave-fluxes}]{\ref{\detokenize{tech_note/Radiative_Fluxes/CLM50_Tech_Note_Radiative_Fluxes:longwave-fluxes}}}), \(H_{sno}\)  is the
sensible heat flux from the snow (Chapter
\hyperref[\detokenize{tech_note/Fluxes/CLM50_Tech_Note_Fluxes:rst-momentum-sensible-heat-and-latent-heat-fluxes}]{\ref{\detokenize{tech_note/Fluxes/CLM50_Tech_Note_Fluxes:rst-momentum-sensible-heat-and-latent-heat-fluxes}}}), and
\(\lambda E_{sno}\)  is the latent heat flux from the snow (Chapter
\hyperref[\detokenize{tech_note/Fluxes/CLM50_Tech_Note_Fluxes:rst-momentum-sensible-heat-and-latent-heat-fluxes}]{\ref{\detokenize{tech_note/Fluxes/CLM50_Tech_Note_Fluxes:rst-momentum-sensible-heat-and-latent-heat-fluxes}}}). The partial
derivative of the heat flux \(h\) with respect to temperature is
\phantomsection\label{\detokenize{tech_note/Soil_Snow_Temperatures/CLM50_Tech_Note_Soil_Snow_Temperatures:equation-6.26}}\begin{equation}\label{equation:tech_note/Soil_Snow_Temperatures/CLM50_Tech_Note_Soil_Snow_Temperatures:6.26}
\begin{split}\frac{\partial h}{\partial T_{} } =-\frac{\partial \overrightarrow{L}_{} }{\partial T_{} } -\frac{\partial H_{} }{\partial T_{} } -\frac{\partial \lambda E_{} }{\partial T_{} }\end{split}
\end{equation}
where the partial derivative of the net longwave radiation is
\phantomsection\label{\detokenize{tech_note/Soil_Snow_Temperatures/CLM50_Tech_Note_Soil_Snow_Temperatures:equation-6.27}}\begin{equation}\label{equation:tech_note/Soil_Snow_Temperatures/CLM50_Tech_Note_Soil_Snow_Temperatures:6.27}
\begin{split}\frac{\partial \overrightarrow{L}_{} }{\partial T_{} } =4\varepsilon _{g} \sigma \left(T_{}^{n} \right)^{3}\end{split}
\end{equation}
and the partial derivatives of the sensible and latent heat fluxes are
given by equations and for non-vegetated surfaces, and by equations and
for vegetated surfaces. \(\sigma\)  is the Stefan-Boltzmann constant
(W m$^{\text{-2}}$ K$^{\text{-4}}$) (\hyperref[\detokenize{tech_note/Ecosystem/CLM50_Tech_Note_Ecosystem:table-physical-constants}]{Table \ref{\detokenize{tech_note/Ecosystem/CLM50_Tech_Note_Ecosystem:table-physical-constants}}}) and
\(\varepsilon _{g}\)  is the ground emissivity (section
\hyperref[\detokenize{tech_note/Radiative_Fluxes/CLM50_Tech_Note_Radiative_Fluxes:longwave-fluxes}]{\ref{\detokenize{tech_note/Radiative_Fluxes/CLM50_Tech_Note_Radiative_Fluxes:longwave-fluxes}}}). For purposes of computing \(h\) and
\(\frac{\partial h}{\partial T_{g} }\) , the term \(\lambda\)
is arbitrarily assumed to be
\phantomsection\label{\detokenize{tech_note/Soil_Snow_Temperatures/CLM50_Tech_Note_Soil_Snow_Temperatures:equation-6.28}}\begin{equation}\label{equation:tech_note/Soil_Snow_Temperatures/CLM50_Tech_Note_Soil_Snow_Temperatures:6.28}
\begin{split}\lambda =\left\{\begin{array}{l} {\lambda _{sub} \qquad {\rm if\; }w_{liq,\, snl+1} =0{\rm \; and\; }w_{ice,\, snl+1} >0} \\ {\lambda _{vap} \qquad {\rm otherwise}} \end{array}\right\}\end{split}
\end{equation}
where \(\lambda _{sub}\)  and \(\lambda _{vap}\)  are the
latent heat of sublimation and vaporization, respectively (J
kg$^{\text{-1}}$) (\hyperref[\detokenize{tech_note/Ecosystem/CLM50_Tech_Note_Ecosystem:table-physical-constants}]{Table \ref{\detokenize{tech_note/Ecosystem/CLM50_Tech_Note_Ecosystem:table-physical-constants}}}), and \(w_{liq,\, snl+1}\)
and \(w_{ice,\, snl+1}\)  are the liquid water and ice contents of the
top snow/soil layer, respectively (kg m$^{\text{-2}}$)
(Chapter \hyperref[\detokenize{tech_note/Hydrology/CLM50_Tech_Note_Hydrology:rst-hydrology}]{\ref{\detokenize{tech_note/Hydrology/CLM50_Tech_Note_Hydrology:rst-hydrology}}}).

For the top soil layer, \(i=1\), the coefficients are
\phantomsection\label{\detokenize{tech_note/Soil_Snow_Temperatures/CLM50_Tech_Note_Soil_Snow_Temperatures:equation-6.29}}\begin{equation}\label{equation:tech_note/Soil_Snow_Temperatures/CLM50_Tech_Note_Soil_Snow_Temperatures:6.29}
\begin{split}a_{i} =-f_{sno} \left(1-\alpha \right)\frac{\Delta t}{c_{i} \Delta z_{i} } \frac{\lambda \left[z_{h,\, i-1} \right]}{z_{i} -z_{i-1} }\end{split}
\end{equation}\phantomsection\label{\detokenize{tech_note/Soil_Snow_Temperatures/CLM50_Tech_Note_Soil_Snow_Temperatures:equation-6.30}}\begin{equation}\label{equation:tech_note/Soil_Snow_Temperatures/CLM50_Tech_Note_Soil_Snow_Temperatures:6.30}
\begin{split}b_{i} =1+\left(1-\alpha \right)\frac{\Delta t}{c_{i} \Delta z_{i} } \left[f_{sno} \frac{\lambda \left[z_{h,\, i-1} \right]}{z_{i} -z_{i-1} } +\frac{\lambda \left[z_{h,\, i} \right]}{z_{i+1} -z_{i} } \right]-\left(1-f_{sno} \right)\frac{\Delta t}{c_{i} \Delta z_{i} } \frac{\partial h}{\partial T}\end{split}
\end{equation}\phantomsection\label{\detokenize{tech_note/Soil_Snow_Temperatures/CLM50_Tech_Note_Soil_Snow_Temperatures:equation-6.31}}\begin{equation}\label{equation:tech_note/Soil_Snow_Temperatures/CLM50_Tech_Note_Soil_Snow_Temperatures:6.31}
\begin{split}c_{i} =-\left(1-\alpha \right)\frac{\Delta t}{c_{i} \Delta z_{i} } \frac{\lambda \left[z_{h,\, i} \right]}{z_{i+1} -z_{i} }\end{split}
\end{equation}\phantomsection\label{\detokenize{tech_note/Soil_Snow_Temperatures/CLM50_Tech_Note_Soil_Snow_Temperatures:equation-6.32}}\begin{equation}\label{equation:tech_note/Soil_Snow_Temperatures/CLM50_Tech_Note_Soil_Snow_Temperatures:6.32}
\begin{split}r_{i} =T_{i}^{n} +\frac{\Delta t}{c_{i} \Delta z_{i} } \left[\left(1-f_{sno} \right)\left(h_{soil} ^{n} -\frac{\partial h}{\partial T_{} } T_{i}^{n} \right)+\alpha \left(F_{i} -f_{sno} F_{i-1} \right)\right]\end{split}
\end{equation}
The heat flux into the soil surface from the overlying atmosphere
\(h\) is
\phantomsection\label{\detokenize{tech_note/Soil_Snow_Temperatures/CLM50_Tech_Note_Soil_Snow_Temperatures:equation-6.33}}\begin{equation}\label{equation:tech_note/Soil_Snow_Temperatures/CLM50_Tech_Note_Soil_Snow_Temperatures:6.33}
\begin{split}h=\overrightarrow{S}_{soil} -\overrightarrow{L}_{soil} -H_{soil} -\lambda E_{soil}\end{split}
\end{equation}
It can be seen that when no snow is present (\(f_{sno} =0\)), the
expressions for the coefficients of the top soil layer have the same
form as those for the top snow layer.

The surface snow/soil layer temperature computed in this way is the
layer-averaged temperature and hence has somewhat reduced diurnal
amplitude compared with surface temperature. An accurate surface
temperature is provided that compensates for this effect and numerical
error by tuning the heat capacity of the top layer (through adjustment
of the layer thickness) to give an exact match to the analytic solution
for diurnal heating. The top layer thickness for \(i=snl+1\) is
given by
\phantomsection\label{\detokenize{tech_note/Soil_Snow_Temperatures/CLM50_Tech_Note_Soil_Snow_Temperatures:equation-6.34}}\begin{equation}\label{equation:tech_note/Soil_Snow_Temperatures/CLM50_Tech_Note_Soil_Snow_Temperatures:6.34}
\begin{split}\Delta z_{i*} =0.5\left[z_{i} -z_{h,\, i-1} +c_{a} \left(z_{i+1} -z_{h,\, i-1} \right)\right]\end{split}
\end{equation}
where \(c_{a}\)  is a tunable parameter, varying from 0 to 1, and is
taken as 0.34 by comparing the numerical solution with the analytic
solution ({\hyperref[\detokenize{tech_note/References/CLM50_Tech_Note_References:yang1998}]{\sphinxcrossref{\DUrole{std,std-ref}{Z.-L. Yang 1998, unpublished manuscript}}}}).
\(\Delta z_{i*}\)  is used in place of \(\Delta z_{i}\)  for
\(i=snl+1\) in equations -. The top snow/soil layer temperature
computed in this way is the ground surface temperature
\(T_{g}^{n+1}\) .

The boundary condition at the bottom of the snow/soil column is zero
heat flux, \(F_{i} =0\), resulting in, for \(i=N_{levgrnd}\) ,
\phantomsection\label{\detokenize{tech_note/Soil_Snow_Temperatures/CLM50_Tech_Note_Soil_Snow_Temperatures:equation-6.35}}\begin{equation}\label{equation:tech_note/Soil_Snow_Temperatures/CLM50_Tech_Note_Soil_Snow_Temperatures:6.35}
\begin{split}\frac{c_{i} \Delta z_{i} }{\Delta t} \left(T_{i}^{n+1} -T_{i}^{n} \right)=\alpha \frac{\lambda \left[z_{h,\, i-1} \right]\left(T_{i-1}^{n} -T_{i}^{n} \right)}{z_{i} -z_{i-1} } +\left(1-\alpha \right)\frac{\lambda \left[z_{h,\, i-1} \right]\left(T_{i-1}^{n+1} -T_{i}^{n+1} \right)}{z_{i} -z_{i-1} }\end{split}
\end{equation}\phantomsection\label{\detokenize{tech_note/Soil_Snow_Temperatures/CLM50_Tech_Note_Soil_Snow_Temperatures:equation-6.36}}\begin{equation}\label{equation:tech_note/Soil_Snow_Temperatures/CLM50_Tech_Note_Soil_Snow_Temperatures:6.36}
\begin{split}a_{i} =-\left(1-\alpha \right)\frac{\Delta t}{c_{i} \Delta z_{i} } \frac{\lambda \left[z_{h,\, i-1} \right]}{z_{i} -z_{i-1} }\end{split}
\end{equation}\phantomsection\label{\detokenize{tech_note/Soil_Snow_Temperatures/CLM50_Tech_Note_Soil_Snow_Temperatures:equation-6.37}}\begin{equation}\label{equation:tech_note/Soil_Snow_Temperatures/CLM50_Tech_Note_Soil_Snow_Temperatures:6.37}
\begin{split}b_{i} =1+\left(1-\alpha \right)\frac{\Delta t}{c_{i} \Delta z_{i} } \frac{\lambda \left[z_{h,\, i-1} \right]}{z_{i} -z_{i-1} }\end{split}
\end{equation}\phantomsection\label{\detokenize{tech_note/Soil_Snow_Temperatures/CLM50_Tech_Note_Soil_Snow_Temperatures:equation-6.38}}\begin{equation}\label{equation:tech_note/Soil_Snow_Temperatures/CLM50_Tech_Note_Soil_Snow_Temperatures:6.38}
\begin{split}c_{i} =0\end{split}
\end{equation}\phantomsection\label{\detokenize{tech_note/Soil_Snow_Temperatures/CLM50_Tech_Note_Soil_Snow_Temperatures:equation-6.39}}\begin{equation}\label{equation:tech_note/Soil_Snow_Temperatures/CLM50_Tech_Note_Soil_Snow_Temperatures:6.39}
\begin{split}r_{i} =T_{i}^{n} -\alpha \frac{\Delta t}{c_{i} \Delta z_{i} } F_{i-1}\end{split}
\end{equation}
where
\phantomsection\label{\detokenize{tech_note/Soil_Snow_Temperatures/CLM50_Tech_Note_Soil_Snow_Temperatures:equation-6.40}}\begin{equation}\label{equation:tech_note/Soil_Snow_Temperatures/CLM50_Tech_Note_Soil_Snow_Temperatures:6.40}
\begin{split}F_{i-1} =-\frac{\lambda \left[z_{h,\, i-1} \right]}{z_{i} -z_{i-1} } \left(T_{i-1}^{n} -T_{i}^{n} \right).\end{split}
\end{equation}
For the interior snow/soil layers, \(snl+1<i<N_{levgrnd}\) ,
excluding the top soil layer,
\phantomsection\label{\detokenize{tech_note/Soil_Snow_Temperatures/CLM50_Tech_Note_Soil_Snow_Temperatures:equation-6.41}}\begin{equation}\label{equation:tech_note/Soil_Snow_Temperatures/CLM50_Tech_Note_Soil_Snow_Temperatures:6.41}
\begin{split}\begin{array}{rcl} {\frac{c_{i} \Delta z_{i} }{\Delta t} \left(T_{i}^{n+1} -T_{i}^{n} \right)} & {=} & {-\alpha \frac{\lambda \left[z_{h,\, i} \right]\left(T_{i}^{n} -T_{i+1}^{n} \right)}{z_{i+1} -z_{i} } +\alpha \frac{\lambda \left[z_{h,\, i-1} \right]\left(T_{i-1}^{n} -T_{i}^{n} \right)}{z_{i} -z_{i-1} } } \\ {} \end{array}\end{split}
\end{equation}\phantomsection\label{\detokenize{tech_note/Soil_Snow_Temperatures/CLM50_Tech_Note_Soil_Snow_Temperatures:equation-6.42}}\begin{equation}\label{equation:tech_note/Soil_Snow_Temperatures/CLM50_Tech_Note_Soil_Snow_Temperatures:6.42}
\begin{split}a_{i} =-\left(1-\alpha \right)\frac{\Delta t}{c_{i} \Delta z_{i} } \frac{\lambda \left[z_{h,\, i-1} \right]}{z_{i} -z_{i-1} }\end{split}
\end{equation}\phantomsection\label{\detokenize{tech_note/Soil_Snow_Temperatures/CLM50_Tech_Note_Soil_Snow_Temperatures:equation-6.43}}\begin{equation}\label{equation:tech_note/Soil_Snow_Temperatures/CLM50_Tech_Note_Soil_Snow_Temperatures:6.43}
\begin{split}b_{i} =1+\left(1-\alpha \right)\frac{\Delta t}{c_{i} \Delta z_{i} } \left[\frac{\lambda \left[z_{h,\, i-1} \right]}{z_{i} -z_{i-1} } +\frac{\lambda \left[z_{h,\, i} \right]}{z_{i+1} -z_{i} } \right]\end{split}
\end{equation}\phantomsection\label{\detokenize{tech_note/Soil_Snow_Temperatures/CLM50_Tech_Note_Soil_Snow_Temperatures:equation-6.44}}\begin{equation}\label{equation:tech_note/Soil_Snow_Temperatures/CLM50_Tech_Note_Soil_Snow_Temperatures:6.44}
\begin{split}c_{i} =-\left(1-\alpha \right)\frac{\Delta t}{c_{i} \Delta z_{i} } \frac{\lambda \left[z_{h,\, i} \right]}{z_{i+1} -z_{i} }\end{split}
\end{equation}\phantomsection\label{\detokenize{tech_note/Soil_Snow_Temperatures/CLM50_Tech_Note_Soil_Snow_Temperatures:equation-6.45}}\begin{equation}\label{equation:tech_note/Soil_Snow_Temperatures/CLM50_Tech_Note_Soil_Snow_Temperatures:6.45}
\begin{split}r_{i} =T_{i}^{n} +\alpha \frac{\Delta t}{c_{i} \Delta z_{i} } \left(F_{i} -F_{i-1} \right)+\frac{\Delta t}{c_{i} \Delta z_{i} } \vec{S}_{g,i} .\end{split}
\end{equation}
where \(\vec{S}_{g,i}\)  is the absorbed solar flux in layer
\(i\) (section \hyperref[\detokenize{tech_note/Surface_Albedos/CLM50_Tech_Note_Surface_Albedos:snow-albedo}]{\ref{\detokenize{tech_note/Surface_Albedos/CLM50_Tech_Note_Surface_Albedos:snow-albedo}}}).

When surface water exists, the following top soil layer coefficients are
modified
\phantomsection\label{\detokenize{tech_note/Soil_Snow_Temperatures/CLM50_Tech_Note_Soil_Snow_Temperatures:equation-6.46}}\begin{equation}\label{equation:tech_note/Soil_Snow_Temperatures/CLM50_Tech_Note_Soil_Snow_Temperatures:6.46}
\begin{split}\begin{array}{l} {b_{i} =1+\left(1-\alpha \right)\frac{\Delta t}{c_{i} \Delta z_{i} } \left[f_{h2osfc} \frac{\lambda _{h2osfc} }{z_{i} -z_{h2osfc} } +f_{sno} \frac{\lambda \left[z_{h,\, i-1} \right]}{z_{i} -z_{i-1} } +\frac{\lambda \left[z_{h,\, i} \right]}{z_{i+1} -z_{i} } \right]} \\ {\quad \quad -\left(1-f_{sno} -f_{h2osfc} \right)\frac{\Delta t}{c_{i} \Delta z_{i} } \frac{\partial h}{\partial T} } \end{array}\end{split}
\end{equation}\phantomsection\label{\detokenize{tech_note/Soil_Snow_Temperatures/CLM50_Tech_Note_Soil_Snow_Temperatures:equation-6.47}}\begin{equation}\label{equation:tech_note/Soil_Snow_Temperatures/CLM50_Tech_Note_Soil_Snow_Temperatures:6.47}
\begin{split}r_{i} =T_{i}^{n} +\frac{\Delta t}{c_{i} \Delta z_{i} } \left[\begin{array}{l} {\left(1-f_{sno} -f_{h2osfc} \right)\left(h_{soil} ^{n} -\frac{\partial h}{\partial T_{} } T_{i}^{n} \right)} \\ {+\alpha \left(F_{i} -f_{sno} F_{i-1} +f_{h2osfc} \frac{\lambda _{h2osfc} }{z_{1} -z_{h2osfc} } \left(T_{1} -T_{h2osfc} \right)\right)} \end{array}\right]\end{split}
\end{equation}\phantomsection\label{\detokenize{tech_note/Soil_Snow_Temperatures/CLM50_Tech_Note_Soil_Snow_Temperatures:equation-6.48}}\begin{equation}\label{equation:tech_note/Soil_Snow_Temperatures/CLM50_Tech_Note_Soil_Snow_Temperatures:6.48}
\begin{split}d_{i} =-f_{h2osfc} \left(1-\alpha \right)\frac{\Delta t}{c_{i} \Delta z_{i} } \left[\frac{\lambda _{h2osfc} }{z_{i} -z_{h2osfc} } \right]\end{split}
\end{equation}
where \(d_{i}\)  is an additional coefficient representing the heat
flux from the surface water layer. The surface water layer coefficients
are
\phantomsection\label{\detokenize{tech_note/Soil_Snow_Temperatures/CLM50_Tech_Note_Soil_Snow_Temperatures:equation-6.49}}\begin{equation}\label{equation:tech_note/Soil_Snow_Temperatures/CLM50_Tech_Note_Soil_Snow_Temperatures:6.49}
\begin{split}a_{h2osfc} =0\end{split}
\end{equation}\phantomsection\label{\detokenize{tech_note/Soil_Snow_Temperatures/CLM50_Tech_Note_Soil_Snow_Temperatures:equation-6.50}}\begin{equation}\label{equation:tech_note/Soil_Snow_Temperatures/CLM50_Tech_Note_Soil_Snow_Temperatures:6.50}
\begin{split}b_{h2osfc} =1+\frac{\Delta t}{c_{h2osfc} \Delta z_{h2osfc} } \left[\left(1-\alpha \right)\frac{\lambda _{h2osfc} }{z_{1} -z_{h2osfc} } -\frac{\partial h}{\partial T} \right]\end{split}
\end{equation}\phantomsection\label{\detokenize{tech_note/Soil_Snow_Temperatures/CLM50_Tech_Note_Soil_Snow_Temperatures:equation-6.51}}\begin{equation}\label{equation:tech_note/Soil_Snow_Temperatures/CLM50_Tech_Note_Soil_Snow_Temperatures:6.51}
\begin{split}c_{h2osfc} =-\left(1-\alpha \right)\frac{\Delta t}{c_{h2osfc} \Delta z_{h2osfc} } \frac{\lambda _{h2osfc} }{z_{1} -z_{h2osfc} }\end{split}
\end{equation}\phantomsection\label{\detokenize{tech_note/Soil_Snow_Temperatures/CLM50_Tech_Note_Soil_Snow_Temperatures:equation-6.52}}\begin{equation}\label{equation:tech_note/Soil_Snow_Temperatures/CLM50_Tech_Note_Soil_Snow_Temperatures:6.52}
\begin{split}r_{h2osfc} =T_{h2osfc}^{n} +\frac{\Delta t}{c_{i} \Delta z_{i} } \left[h_{h2osfc} ^{n} -\frac{\partial h}{\partial T_{} } T_{h2osfc}^{n} +\alpha \frac{\lambda _{h2osfc} }{z_{1} -z_{h2osfc} } \left(T_{1} -T_{h2osfc} \right)\right]_{}\end{split}
\end{equation}

\subsection{Phase Change}
\label{\detokenize{tech_note/Soil_Snow_Temperatures/CLM50_Tech_Note_Soil_Snow_Temperatures:phase-change}}\label{\detokenize{tech_note/Soil_Snow_Temperatures/CLM50_Tech_Note_Soil_Snow_Temperatures:id1}}

\subsubsection{Soil and Snow Layers}
\label{\detokenize{tech_note/Soil_Snow_Temperatures/CLM50_Tech_Note_Soil_Snow_Temperatures:id2}}\label{\detokenize{tech_note/Soil_Snow_Temperatures/CLM50_Tech_Note_Soil_Snow_Temperatures:soil-and-snow-layers}}
Upon update, the snow/soil temperatures are evaluated to determine if
phase change will take place as
\phantomsection\label{\detokenize{tech_note/Soil_Snow_Temperatures/CLM50_Tech_Note_Soil_Snow_Temperatures:equation-6.53a}}\begin{equation}\label{equation:tech_note/Soil_Snow_Temperatures/CLM50_Tech_Note_Soil_Snow_Temperatures:6.53a}
\begin{split}\begin{array}{lr}
T_{i}^{n+1} >T_{f} {\rm \; and\; }w_{ice,\, i} >0 & \qquad i=snl+1,\ldots ,N_{levgrnd} \qquad {\rm melting}   \end{array}\end{split}
\end{equation}\phantomsection\label{\detokenize{tech_note/Soil_Snow_Temperatures/CLM50_Tech_Note_Soil_Snow_Temperatures:equation-6.53b}}\begin{equation}\label{equation:tech_note/Soil_Snow_Temperatures/CLM50_Tech_Note_Soil_Snow_Temperatures:6.53b}
\begin{split}\begin{array}{lr}
\begin{array}{lr}
T_{i}^{n+1} <T_{f} {\rm \; and\; }w_{liq,\, i} >0 & \qquad i=snl+1,\ldots ,0 \\
T_{i}^{n+1} <T_{f} {\rm \; and\; }w_{liq,\, i} >w_{liq,\, \max ,\, i} & \quad i=1,\ldots ,N_{levgrnd}
\end{array} & \quad {\rm freezing}
\end{array}\end{split}
\end{equation}
where \(T_{i}^{n+1}\)  is the soil layer temperature after solution
of the tridiagonal equation set, \(w_{ice,\, i}\)  and
\(w_{liq,\, i}\)  are the mass of ice and liquid water (kg
m$^{\text{-2}}$) in each snow/soil layer, respectively, and \(T_{f}\)
is the freezing temperature of water (K) (\hyperref[\detokenize{tech_note/Ecosystem/CLM50_Tech_Note_Ecosystem:table-physical-constants}]{Table \ref{\detokenize{tech_note/Ecosystem/CLM50_Tech_Note_Ecosystem:table-physical-constants}}}).
For the freezing process in soil layers, the concept of supercooled soil
water from {\hyperref[\detokenize{tech_note/References/CLM50_Tech_Note_References:niuyang2006}]{\sphinxcrossref{\DUrole{std,std-ref}{Niu and Yang (2006)}}}} is adopted. The supercooled
soil water is the liquid water that coexists with ice over a wide range of
temperatures below freezing and is implemented through a freezing point
depression equation
\phantomsection\label{\detokenize{tech_note/Soil_Snow_Temperatures/CLM50_Tech_Note_Soil_Snow_Temperatures:equation-6.54}}\begin{equation}\label{equation:tech_note/Soil_Snow_Temperatures/CLM50_Tech_Note_Soil_Snow_Temperatures:6.54}
\begin{split}w_{liq,\, \max ,\, i} =\Delta z_{i} \theta _{sat,\, i} \left[\frac{10^{3} L_{f} \left(T_{f} -T_{i} \right)}{gT_{i} \psi _{sat,\, i} } \right]^{{-1\mathord{\left/ {\vphantom {-1 B_{i} }} \right. \kern-\nulldelimiterspace} B_{i} } } \qquad T_{i} <T_{f}\end{split}
\end{equation}
where \(w_{liq,\, \max ,\, i}\)  is the maximum liquid water in
layer \(i\) (kg m$^{\text{-2}}$) when the soil temperature
\(T_{i}\)  is below the freezing temperature \(T_{f}\) ,
\(L_{f}\)  is the latent heat of fusion (J kg$^{\text{-1}}$)
(\hyperref[\detokenize{tech_note/Ecosystem/CLM50_Tech_Note_Ecosystem:table-physical-constants}]{Table \ref{\detokenize{tech_note/Ecosystem/CLM50_Tech_Note_Ecosystem:table-physical-constants}}}), \(g\) is the gravitational acceleration (m
s$^{\text{-2}}$) (\hyperref[\detokenize{tech_note/Ecosystem/CLM50_Tech_Note_Ecosystem:table-physical-constants}]{Table \ref{\detokenize{tech_note/Ecosystem/CLM50_Tech_Note_Ecosystem:table-physical-constants}}}), and \(\psi _{sat,\, i}\)  and
\(B_{i}\)  are the soil texture-dependent saturated matric potential
(mm) and {\hyperref[\detokenize{tech_note/References/CLM50_Tech_Note_References:clapphornberger1978}]{\sphinxcrossref{\DUrole{std,std-ref}{Clapp and Hornberger (1978)}}}} exponent
(section \hyperref[\detokenize{tech_note/Hydrology/CLM50_Tech_Note_Hydrology:soil-water}]{\ref{\detokenize{tech_note/Hydrology/CLM50_Tech_Note_Hydrology:soil-water}}}).

For the special case when snow is present (snow mass \(W_{sno} >0\))
but there are no explicit snow layers (\(snl=0\)) (i.e., there is
not enough snow present to meet the minimum snow depth requirement of
0.01 m), snow melt will take place for soil layer \(i=1\) if the
soil layer temperature is greater than the freezing temperature
(\(T_{1}^{n+1} >T_{f}\) ).

The rate of phase change is assessed from the energy excess (or deficit)
needed to change \(T_{i}\)  to freezing temperature, \(T_{f}\) .
The excess or deficit of energy \(H_{i}\)  (W m$^{\text{-2}}$) is
determined as follows
\phantomsection\label{\detokenize{tech_note/Soil_Snow_Temperatures/CLM50_Tech_Note_Soil_Snow_Temperatures:equation-6.55}}\begin{equation}\label{equation:tech_note/Soil_Snow_Temperatures/CLM50_Tech_Note_Soil_Snow_Temperatures:6.55}
\begin{split}H_{i} =\left\{\begin{array}{lr}
\frac{\partial h}{\partial T} \left(T_{f} -T_{i}^{n} \right)-\frac{c_{i} \Delta z_{i} }{\Delta t} \left(T_{f} -T_{i}^{n} \right) & \quad \quad i=snl+1 \\
\left(1-f_{sno} -f_{h2osfc} \right)\frac{\partial h}{\partial T} \left(T_{f} -T_{i}^{n} \right)-\frac{c_{i} \Delta z_{i} }{\Delta t} \left(T_{f} -T_{i}^{n} \right)\quad {\kern 1pt} {\kern 1pt} {\kern 1pt} {\kern 1pt} & i=1 \\
-\frac{c_{i} \Delta z_{i} }{\Delta t} \left(T_{f} -T_{i}^{n} \right) & \quad \quad i\ne \left\{1,snl+1\right\}
\end{array}\right\}.\end{split}
\end{equation}
If the melting criteria is met \eqref{equation:tech_note/Soil_Snow_Temperatures/CLM50_Tech_Note_Soil_Snow_Temperatures:6.53a} and
\(H_{m} =\frac{H_{i} \Delta t}{L_{f} } >0\), then the ice mass is
readjusted as
\phantomsection\label{\detokenize{tech_note/Soil_Snow_Temperatures/CLM50_Tech_Note_Soil_Snow_Temperatures:equation-6.56}}\begin{equation}\label{equation:tech_note/Soil_Snow_Temperatures/CLM50_Tech_Note_Soil_Snow_Temperatures:6.56}
\begin{split}w_{ice,\, i}^{n+1} =w_{ice,\, i}^{n} -H_{m} \ge 0\qquad i=snl+1,\ldots ,N_{levgrnd} .\end{split}
\end{equation}
If the freezing criteria is met \eqref{equation:tech_note/Soil_Snow_Temperatures/CLM50_Tech_Note_Soil_Snow_Temperatures:6.53b} and \(H_{m} <0\), then
the ice mass is readjusted for \(i=snl+1,\ldots ,0\) as
\phantomsection\label{\detokenize{tech_note/Soil_Snow_Temperatures/CLM50_Tech_Note_Soil_Snow_Temperatures:equation-6.57}}\begin{equation}\label{equation:tech_note/Soil_Snow_Temperatures/CLM50_Tech_Note_Soil_Snow_Temperatures:6.57}
\begin{split}w_{ice,\, i}^{n+1} =\min \left(w_{liq,\, i}^{n} +w_{ice,\, i}^{n} ,w_{ice,\, i}^{n} -H_{m} \right)\end{split}
\end{equation}
and for \(i=1,\ldots ,N_{levgrnd}\)  as
\phantomsection\label{\detokenize{tech_note/Soil_Snow_Temperatures/CLM50_Tech_Note_Soil_Snow_Temperatures:equation-6.58}}\begin{equation}\label{equation:tech_note/Soil_Snow_Temperatures/CLM50_Tech_Note_Soil_Snow_Temperatures:6.58}
\begin{split}w_{ice,\, i}^{n+1} =
\left\{\begin{array}{lr}
\min \left(w_{liq,\, i}^{n} +w_{ice,\, i}^{n} -w_{liq,\, \max ,\, i}^{n} ,\, w_{ice,\, i}^{n} -H_{m} \right) & \qquad w_{liq,\, i}^{n} +w_{ice,\, i}^{n} \ge w_{liq,\, \max ,\, i}^{n} {\rm \; } \\
{\rm 0} & \qquad w_{liq,\, i}^{n} +w_{ice,\, i}^{n} <w_{liq,\, \max ,\, i}^{n} {\rm \; \; }\,
\end{array}\right\}.\end{split}
\end{equation}
Liquid water mass is readjusted as
\phantomsection\label{\detokenize{tech_note/Soil_Snow_Temperatures/CLM50_Tech_Note_Soil_Snow_Temperatures:equation-6.59}}\begin{equation}\label{equation:tech_note/Soil_Snow_Temperatures/CLM50_Tech_Note_Soil_Snow_Temperatures:6.59}
\begin{split}w_{liq,\, i}^{n+1} =w_{liq,\, i}^{n} +w_{ice,\, i}^{n} -w_{ice,\, i}^{n+1} \ge 0.\end{split}
\end{equation}
Because part of the energy \(H_{i}\)  may not be consumed in
melting or released in freezing, the energy is recalculated as
\phantomsection\label{\detokenize{tech_note/Soil_Snow_Temperatures/CLM50_Tech_Note_Soil_Snow_Temperatures:equation-6.60}}\begin{equation}\label{equation:tech_note/Soil_Snow_Temperatures/CLM50_Tech_Note_Soil_Snow_Temperatures:6.60}
\begin{split}H_{i*} =H_{i} -\frac{L_{f} \left(w_{ice,\, i}^{n} -w_{ice,\, i}^{n+1} \right)}{\Delta t}\end{split}
\end{equation}
and this energy is used to cool or warm the snow/soil layer (if
\(\left|H_{i*} \right|>0\)) as
\phantomsection\label{\detokenize{tech_note/Soil_Snow_Temperatures/CLM50_Tech_Note_Soil_Snow_Temperatures:equation-6.61}}\begin{equation}\label{equation:tech_note/Soil_Snow_Temperatures/CLM50_Tech_Note_Soil_Snow_Temperatures:6.61}
\begin{split}T_{i}^{n+1} =
\left\{\begin{array}{lr}
T_{f} +{\frac{\Delta t}{c_{i} \Delta z_{i} } H_{i*} \mathord{\left/ {\vphantom {\frac{\Delta t}{c_{i} \Delta z_{i} } H_{i*}  \left(1-\frac{\Delta t}{c_{i} \Delta z_{i} } \frac{\partial h}{\partial T} \right)}} \right. \kern-\nulldelimiterspace} \left(1-\frac{\Delta t}{c_{i} \Delta z_{i} } \frac{\partial h}{\partial T} \right)} & \quad \quad \quad \quad \, i=snl+1 \\
T_{f} +{\frac{\Delta t}{c_{i} \Delta z_{i} } H_{i*} \mathord{\left/ {\vphantom {\frac{\Delta t}{c_{i} \Delta z_{i} } H_{i*}  \left(1-\left(1-f_{sno} -f_{h2osfc} \right)\frac{\Delta t}{c_{i} \Delta z_{i} } \frac{\partial h}{\partial T} \right)}} \right. \kern-\nulldelimiterspace} \left(1-\left(1-f_{sno} -f_{h2osfc} \right)\frac{\Delta t}{c_{i} \Delta z_{i} } \frac{\partial h}{\partial T} \right)} & \qquad i=1 \\
T_{f} +\frac{\Delta t}{c_{i} \Delta z_{i} } H_{i*} & \quad \quad \quad \quad \, i\ne \left\{1,snl+1\right\}
\end{array}\right\}.\end{split}
\end{equation}
For the special case when snow is present (\(W_{sno} >0\)), there
are no explicit snow layers (\(snl=0\)), and
\(\frac{H_{1} \Delta t}{L_{f} } >0\) (melting), the snow mass
\(W_{sno}\)  (kg m$^{\text{-2}}$) is reduced according to
\phantomsection\label{\detokenize{tech_note/Soil_Snow_Temperatures/CLM50_Tech_Note_Soil_Snow_Temperatures:equation-6.62}}\begin{equation}\label{equation:tech_note/Soil_Snow_Temperatures/CLM50_Tech_Note_Soil_Snow_Temperatures:6.62}
\begin{split}W_{sno}^{n+1} =W_{sno}^{n} -\frac{H_{1} \Delta t}{L_{f} } \ge 0.\end{split}
\end{equation}
The snow depth is reduced proportionally
\phantomsection\label{\detokenize{tech_note/Soil_Snow_Temperatures/CLM50_Tech_Note_Soil_Snow_Temperatures:equation-6.63}}\begin{equation}\label{equation:tech_note/Soil_Snow_Temperatures/CLM50_Tech_Note_Soil_Snow_Temperatures:6.63}
\begin{split}z_{sno}^{n+1} =\frac{W_{sno}^{n+1} }{W_{sno}^{n} } z_{sno}^{n} .\end{split}
\end{equation}
Again, because part of the energy may not be consumed in melting, the
energy for the surface soil layer \(i=1\) is recalculated as
\phantomsection\label{\detokenize{tech_note/Soil_Snow_Temperatures/CLM50_Tech_Note_Soil_Snow_Temperatures:equation-6.64}}\begin{equation}\label{equation:tech_note/Soil_Snow_Temperatures/CLM50_Tech_Note_Soil_Snow_Temperatures:6.64}
\begin{split}H_{1*} =H_{1} -\frac{L_{f} \left(W_{sno}^{n} -W_{sno}^{n+1} \right)}{\Delta t} .\end{split}
\end{equation}
If there is excess energy (\(H_{1*} >0\)), this energy becomes
available to the top soil layer as
\phantomsection\label{\detokenize{tech_note/Soil_Snow_Temperatures/CLM50_Tech_Note_Soil_Snow_Temperatures:equation-6.65}}\begin{equation}\label{equation:tech_note/Soil_Snow_Temperatures/CLM50_Tech_Note_Soil_Snow_Temperatures:6.65}
\begin{split}H_{1} =H_{1*} .\end{split}
\end{equation}
The ice mass, liquid water content, and temperature of the top soil
layer are then determined from \eqref{equation:tech_note/Soil_Snow_Temperatures/CLM50_Tech_Note_Soil_Snow_Temperatures:6.56}, \eqref{equation:tech_note/Soil_Snow_Temperatures/CLM50_Tech_Note_Soil_Snow_Temperatures:6.59}, and \eqref{equation:tech_note/Soil_Snow_Temperatures/CLM50_Tech_Note_Soil_Snow_Temperatures:6.61}
using the recalculated energy from \eqref{equation:tech_note/Soil_Snow_Temperatures/CLM50_Tech_Note_Soil_Snow_Temperatures:6.65}. Snow melt \(M_{1S}\)
(kg m$^{\text{-2}}$ s$^{\text{-1}}$) and phase change energy \(E_{p,\, 1S}\)
(W m$^{\text{-2}}$) for this special case are
\phantomsection\label{\detokenize{tech_note/Soil_Snow_Temperatures/CLM50_Tech_Note_Soil_Snow_Temperatures:equation-6.66}}\begin{equation}\label{equation:tech_note/Soil_Snow_Temperatures/CLM50_Tech_Note_Soil_Snow_Temperatures:6.66}
\begin{split}M_{1S} =\frac{W_{sno}^{n} -W_{sno}^{n+1} }{\Delta t} \ge 0\end{split}
\end{equation}\phantomsection\label{\detokenize{tech_note/Soil_Snow_Temperatures/CLM50_Tech_Note_Soil_Snow_Temperatures:equation-6.67}}\begin{equation}\label{equation:tech_note/Soil_Snow_Temperatures/CLM50_Tech_Note_Soil_Snow_Temperatures:6.67}
\begin{split}E_{p,\, 1S} =L_{f} M_{1S} .\end{split}
\end{equation}
The total energy of phase change \(E_{p}\)  (W m$^{\text{-2}}$)
for the snow/soil column is
\phantomsection\label{\detokenize{tech_note/Soil_Snow_Temperatures/CLM50_Tech_Note_Soil_Snow_Temperatures:equation-6.68}}\begin{equation}\label{equation:tech_note/Soil_Snow_Temperatures/CLM50_Tech_Note_Soil_Snow_Temperatures:6.68}
\begin{split}E_{p} =E_{p,\, 1S} +\sum _{i=snl+1}^{N_{levgrnd} }E_{p,i}\end{split}
\end{equation}
where
\phantomsection\label{\detokenize{tech_note/Soil_Snow_Temperatures/CLM50_Tech_Note_Soil_Snow_Temperatures:equation-6.69}}\begin{equation}\label{equation:tech_note/Soil_Snow_Temperatures/CLM50_Tech_Note_Soil_Snow_Temperatures:6.69}
\begin{split}E_{p,\, i} =L_{f} \frac{\left(w_{ice,\, i}^{n} -w_{ice,\, i}^{n+1} \right)}{\Delta t} .\end{split}
\end{equation}
The total snow melt \(M\) (kg m$^{\text{-2}}$
s$^{\text{-1}}$) is
\phantomsection\label{\detokenize{tech_note/Soil_Snow_Temperatures/CLM50_Tech_Note_Soil_Snow_Temperatures:equation-6.70}}\begin{equation}\label{equation:tech_note/Soil_Snow_Temperatures/CLM50_Tech_Note_Soil_Snow_Temperatures:6.70}
\begin{split}M=M_{1S} +\sum _{i=snl+1}^{i=0}M_{i}\end{split}
\end{equation}
where
\phantomsection\label{\detokenize{tech_note/Soil_Snow_Temperatures/CLM50_Tech_Note_Soil_Snow_Temperatures:equation-6.71}}\begin{equation}\label{equation:tech_note/Soil_Snow_Temperatures/CLM50_Tech_Note_Soil_Snow_Temperatures:6.71}
\begin{split}M_{i} =\frac{\left(w_{ice,\, i}^{n} -w_{ice,\, i}^{n+1} \right)}{\Delta t} \ge 0.\end{split}
\end{equation}
The solution for snow/soil temperatures conserves energy as
\phantomsection\label{\detokenize{tech_note/Soil_Snow_Temperatures/CLM50_Tech_Note_Soil_Snow_Temperatures:equation-6.72}}\begin{equation}\label{equation:tech_note/Soil_Snow_Temperatures/CLM50_Tech_Note_Soil_Snow_Temperatures:6.72}
\begin{split}G-E_{p} -\sum _{i=snl+1}^{i=N_{levgrnd} }\frac{c_{i} \Delta z_{i} }{\Delta t}  \left(T_{i}^{n+1} -T_{i}^{n} \right)=0\end{split}
\end{equation}
where \(G\) is the ground heat flux (section
\hyperref[\detokenize{tech_note/Fluxes/CLM50_Tech_Note_Fluxes:update-of-ground-sensible-and-latent-heat-fluxes}]{\ref{\detokenize{tech_note/Fluxes/CLM50_Tech_Note_Fluxes:update-of-ground-sensible-and-latent-heat-fluxes}}}).


\subsubsection{Surface Water}
\label{\detokenize{tech_note/Soil_Snow_Temperatures/CLM50_Tech_Note_Soil_Snow_Temperatures:surface-water}}\label{\detokenize{tech_note/Soil_Snow_Temperatures/CLM50_Tech_Note_Soil_Snow_Temperatures:id3}}
Phase change of surface water takes place when the surface water
temperature, \(T_{h2osfc}\) , becomes less than \(T_{f}\)  . The
energy available for freezing is
\phantomsection\label{\detokenize{tech_note/Soil_Snow_Temperatures/CLM50_Tech_Note_Soil_Snow_Temperatures:equation-6.73}}\begin{equation}\label{equation:tech_note/Soil_Snow_Temperatures/CLM50_Tech_Note_Soil_Snow_Temperatures:6.73}
\begin{split}H_{h2osfc} =\frac{\partial h}{\partial T} \left(T_{f} -T_{h2osfc}^{n} \right)-\frac{c_{h2osfc} \Delta z_{h2osfc} }{\Delta t} \left(T_{f} -T_{h2osfc}^{n} \right)\end{split}
\end{equation}
where \(c_{h2osfc}\)  is the volumetric heat capacity of water, and
\(\Delta z_{h2osfc}\)  is the depth of the surface water layer. If
\(H_{m} =\frac{H_{h2osfc} \Delta t}{L_{f} } >0\) then \(H_{m}\)
is removed from surface water and added to the snow column as ice
\phantomsection\label{\detokenize{tech_note/Soil_Snow_Temperatures/CLM50_Tech_Note_Soil_Snow_Temperatures:equation-6.74}}\begin{equation}\label{equation:tech_note/Soil_Snow_Temperatures/CLM50_Tech_Note_Soil_Snow_Temperatures:6.74}
\begin{split}H^{n+1} _{h2osfc} =H^{n} _{h2osfc} -H_{m}\end{split}
\end{equation}\phantomsection\label{\detokenize{tech_note/Soil_Snow_Temperatures/CLM50_Tech_Note_Soil_Snow_Temperatures:equation-6.75}}\begin{equation}\label{equation:tech_note/Soil_Snow_Temperatures/CLM50_Tech_Note_Soil_Snow_Temperatures:6.75}
\begin{split}w_{ice,\, 0}^{n+1} =w_{ice,\, 0}^{n} +H_{m}\end{split}
\end{equation}
The snow depth is adjusted to account for the additional ice mass
\phantomsection\label{\detokenize{tech_note/Soil_Snow_Temperatures/CLM50_Tech_Note_Soil_Snow_Temperatures:equation-6.76}}\begin{equation}\label{equation:tech_note/Soil_Snow_Temperatures/CLM50_Tech_Note_Soil_Snow_Temperatures:6.76}
\begin{split}\Delta z_{sno} =\frac{H_{m} }{\rho _{ice} }\end{split}
\end{equation}
If \(H_{m}\) is greater than \(W_{sfc}\) , the excess heat
\(\frac{L_{f} \left(H_{m} -W_{sfc} \right)}{\Delta t}\)  is used to
cool the snow layer.


\subsection{Soil and Snow Thermal Properties}
\label{\detokenize{tech_note/Soil_Snow_Temperatures/CLM50_Tech_Note_Soil_Snow_Temperatures:id4}}\label{\detokenize{tech_note/Soil_Snow_Temperatures/CLM50_Tech_Note_Soil_Snow_Temperatures:soil-and-snow-thermal-properties}}
The thermal properties of the soil are assumed to be a weighted combination of
the mineral and organic properties of the soil
({\hyperref[\detokenize{tech_note/References/CLM50_Tech_Note_References:lawrenceslater2008}]{\sphinxcrossref{\DUrole{std,std-ref}{Lawrence and Slater 2008}}}}).
The soil layer organic matter fraction \(f_{om,i}\)  is
\phantomsection\label{\detokenize{tech_note/Soil_Snow_Temperatures/CLM50_Tech_Note_Soil_Snow_Temperatures:equation-6.77}}\begin{equation}\label{equation:tech_note/Soil_Snow_Temperatures/CLM50_Tech_Note_Soil_Snow_Temperatures:6.77}
\begin{split}f_{om,i} =\rho _{om,i} /\rho _{om,\max } .\end{split}
\end{equation}
Soil thermal conductivity \(\lambda _{i}\)  (W m$^{\text{-1}}$ K$^{\text{-1}}$)
is from {\hyperref[\detokenize{tech_note/References/CLM50_Tech_Note_References:farouki1981}]{\sphinxcrossref{\DUrole{std,std-ref}{Farouki (1981)}}}}
\phantomsection\label{\detokenize{tech_note/Soil_Snow_Temperatures/CLM50_Tech_Note_Soil_Snow_Temperatures:equation-6.78}}\begin{align}\label{equation:tech_note/Soil_Snow_Temperatures/CLM50_Tech_Note_Soil_Snow_Temperatures:6.78}\!\begin{aligned}
\begin{array}{lr}
\lambda _{i} = \left\{
\begin{array}{lr}
K_{e,\, i} \lambda _{sat,\, i} +\left(1-K_{e,\, i} \right)\lambda _{dry,\, i} &\qquad S_{r,\, i} > 1\times 10^{-7}  \\
\lambda _{dry,\, i} &\qquad S_{r,\, i} \le 1\times 10^{-7}
\end{array}\right\} &\qquad i=1,\ldots ,N_{levsoi}  \\\\
\lambda _{i} =\lambda _{bedrock} &\qquad i=N_{levsoi} +1,\ldots N_{levgrnd}
\end{array}\\
\end{aligned}\end{align}
where \(\lambda _{sat,\, i}\)  is the saturated thermal
conductivity, \(\lambda _{dry,\, i}\)  is the dry thermal
conductivity, \(K_{e,\, i}\)  is the Kersten number,
\(S_{r,\, i}\)  is the wetness of the soil with respect to
saturation, and \(\lambda _{bedrock} =3\) W m$^{\text{-1}}$
K$^{\text{-1}}$ is the thermal conductivity assumed for the deep
ground layers (typical of saturated granitic rock;
{\hyperref[\detokenize{tech_note/References/CLM50_Tech_Note_References:clauserhuenges1995}]{\sphinxcrossref{\DUrole{std,std-ref}{Clauser and Huenges 1995}}}}). For glaciers and wetlands,
\phantomsection\label{\detokenize{tech_note/Soil_Snow_Temperatures/CLM50_Tech_Note_Soil_Snow_Temperatures:equation-6.79}}\begin{equation}\label{equation:tech_note/Soil_Snow_Temperatures/CLM50_Tech_Note_Soil_Snow_Temperatures:6.79}
\begin{split}\lambda _{i} =\left\{\begin{array}{l} {\lambda _{liq,\, i} \qquad T_{i} \ge T_{f} } \\ {\lambda _{ice,\, i} \qquad T_{i} <T_{f} } \end{array}\right\}\end{split}
\end{equation}
where \(\lambda _{liq}\)  and \(\lambda _{ice}\)  are the
thermal conductivities of liquid water and ice, respectively (\hyperref[\detokenize{tech_note/Ecosystem/CLM50_Tech_Note_Ecosystem:table-physical-constants}]{Table \ref{\detokenize{tech_note/Ecosystem/CLM50_Tech_Note_Ecosystem:table-physical-constants}}}). The saturated thermal conductivity \(\lambda _{sat,\, i}\)  (W
m$^{\text{-1}}$ K$^{\text{-1}}$) depends on the thermal
conductivities of the soil solid, liquid water, and ice constituents
\phantomsection\label{\detokenize{tech_note/Soil_Snow_Temperatures/CLM50_Tech_Note_Soil_Snow_Temperatures:equation-6.80}}\begin{equation}\label{equation:tech_note/Soil_Snow_Temperatures/CLM50_Tech_Note_Soil_Snow_Temperatures:6.80}
\begin{split}\lambda _{sat} =\lambda _{s}^{1-\theta _{sat} } \lambda _{liq}^{\frac{\theta _{liq} }{\theta _{liq} +\theta _{ice} } \theta _{sat} } \lambda _{ice}^{\theta _{sat} \left(1-\frac{\theta _{liq} }{\theta _{liq} +\theta _{ice} } \right)}\end{split}
\end{equation}
where the thermal conductivity of soil solids
\(\lambda _{s,\, i}\)  varies with the sand, clay, and organic
matter content
\phantomsection\label{\detokenize{tech_note/Soil_Snow_Temperatures/CLM50_Tech_Note_Soil_Snow_Temperatures:equation-6.81}}\begin{equation}\label{equation:tech_note/Soil_Snow_Temperatures/CLM50_Tech_Note_Soil_Snow_Temperatures:6.81}
\begin{split}\lambda _{s,i} =(1-f_{om,i} )\lambda _{s,\min ,i} +f_{om,i} \lambda _{s,om}\end{split}
\end{equation}
where the mineral soil solid thermal conductivity
\(\lambda _{s,\min ,i}\) is
\phantomsection\label{\detokenize{tech_note/Soil_Snow_Temperatures/CLM50_Tech_Note_Soil_Snow_Temperatures:equation-6.82}}\begin{equation}\label{equation:tech_note/Soil_Snow_Temperatures/CLM50_Tech_Note_Soil_Snow_Temperatures:6.82}
\begin{split}\lambda _{s,\, \min ,i} =\frac{8.80{\rm \; }\left(\% sand\right)_{i} +{\rm 2.92\; }\left(\% clay\right)_{i} }{\left(\% sand\right)_{i} +\left(\% clay\right)_{i} } ,\end{split}
\end{equation}
and \(\lambda _{s,om} =0.25\)W m$^{\text{-1}}$
K$^{\text{-1}}$ ({\hyperref[\detokenize{tech_note/References/CLM50_Tech_Note_References:farouki1981}]{\sphinxcrossref{\DUrole{std,std-ref}{Farouki 1981}}}}). \(\theta _{sat,\, i}\)  is the
volumetric water content at saturation (porosity) (section \hyperref[\detokenize{tech_note/Hydrology/CLM50_Tech_Note_Hydrology:hydraulic-properties}]{\ref{\detokenize{tech_note/Hydrology/CLM50_Tech_Note_Hydrology:hydraulic-properties}}}).

The thermal conductivity of dry soil is
\phantomsection\label{\detokenize{tech_note/Soil_Snow_Temperatures/CLM50_Tech_Note_Soil_Snow_Temperatures:equation-6.83}}\begin{equation}\label{equation:tech_note/Soil_Snow_Temperatures/CLM50_Tech_Note_Soil_Snow_Temperatures:6.83}
\begin{split}\lambda _{dry,i} =(1-f_{om,i} )\lambda _{dry,\min ,i} +f_{om,i} \lambda _{dry,om}\end{split}
\end{equation}
where the thermal conductivity of dry mineral soil
\(\lambda _{dry,\min ,i}\) (W m$^{\text{-1}}$
K$^{\text{-1}}$) depends on the bulk density
\(\rho _{d,\, i} =2700\left(1-\theta _{sat,\, i} \right)\) (kg
m$^{\text{-3}}$) as
\phantomsection\label{\detokenize{tech_note/Soil_Snow_Temperatures/CLM50_Tech_Note_Soil_Snow_Temperatures:equation-6.84}}\begin{equation}\label{equation:tech_note/Soil_Snow_Temperatures/CLM50_Tech_Note_Soil_Snow_Temperatures:6.84}
\begin{split}\lambda _{dry,\, \min ,i} =\frac{0.135\rho _{d,\, i} +64.7}{2700-0.947\rho _{d,\, i} }\end{split}
\end{equation}
and \(\lambda _{dry,om} =0.05\) W m$^{\text{-1}}$
K$^{\text{-1}}$ ({\hyperref[\detokenize{tech_note/References/CLM50_Tech_Note_References:farouki1981}]{\sphinxcrossref{\DUrole{std,std-ref}{Farouki 1981}}}}) is the dry thermal conductivity of
organic matter. The Kersten number \(K_{e,\, i}\)  is a function of
the degree of saturation \(S_{r}\)  and phase of water
\phantomsection\label{\detokenize{tech_note/Soil_Snow_Temperatures/CLM50_Tech_Note_Soil_Snow_Temperatures:equation-6.85}}\begin{equation}\label{equation:tech_note/Soil_Snow_Temperatures/CLM50_Tech_Note_Soil_Snow_Temperatures:6.85}
\begin{split}K_{e,\, i} = \left\{
\begin{array}{lr}
\log \left(S_{r,\, i} \right)+1\ge 0 &\qquad T_{i} \ge T_{f}  \\
S_{r,\, i} &\qquad T_{i} <T_{f}
\end{array}\right\}\end{split}
\end{equation}
where
\phantomsection\label{\detokenize{tech_note/Soil_Snow_Temperatures/CLM50_Tech_Note_Soil_Snow_Temperatures:equation-6.86}}\begin{equation}\label{equation:tech_note/Soil_Snow_Temperatures/CLM50_Tech_Note_Soil_Snow_Temperatures:6.86}
\begin{split}S_{r,\, i} =\left(\frac{w_{liq,\, i} }{\rho _{liq} \Delta z_{i} } +\frac{w_{ice,\, i} }{\rho _{ice} \Delta z_{i} } \right)\frac{1}{\theta _{sat,\, i} } =\frac{\theta _{liq,\, i} +\theta _{ice,\, i} }{\theta _{sat,\, i} } \le 1.\end{split}
\end{equation}
Thermal conductivity \(\lambda _{i}\)  (W m$^{\text{-1}}$
K$^{\text{-1}}$) for snow is from {\hyperref[\detokenize{tech_note/References/CLM50_Tech_Note_References:jordan1991}]{\sphinxcrossref{\DUrole{std,std-ref}{Jordan (1991)}}}}
\phantomsection\label{\detokenize{tech_note/Soil_Snow_Temperatures/CLM50_Tech_Note_Soil_Snow_Temperatures:equation-6.87}}\begin{equation}\label{equation:tech_note/Soil_Snow_Temperatures/CLM50_Tech_Note_Soil_Snow_Temperatures:6.87}
\begin{split}\lambda _{i} =\lambda _{air} +\left(7.75\times 10^{-5} \rho _{sno,\, i} +1.105\times 10^{-6} \rho _{sno,\, i}^{2} \right)\left(\lambda _{ice} -\lambda _{air} \right)\end{split}
\end{equation}
where \(\lambda _{air}\)  is the thermal conductivity of air (\hyperref[\detokenize{tech_note/Ecosystem/CLM50_Tech_Note_Ecosystem:table-physical-constants}]{Table \ref{\detokenize{tech_note/Ecosystem/CLM50_Tech_Note_Ecosystem:table-physical-constants}}})
and \(\rho _{sno,\, i}\)  is the bulk density of snow (kg m$^{\text{-3}}$)
\phantomsection\label{\detokenize{tech_note/Soil_Snow_Temperatures/CLM50_Tech_Note_Soil_Snow_Temperatures:equation-6.88}}\begin{equation}\label{equation:tech_note/Soil_Snow_Temperatures/CLM50_Tech_Note_Soil_Snow_Temperatures:6.88}
\begin{split}\rho _{sno,\, i} =\frac{w_{ice,\, i} +w_{liq,\, i} }{\Delta z_{i} } .\end{split}
\end{equation}
The volumetric heat capacity \(c_{i}\)  (J m$^{\text{-3}}$ K$^{\text{-1}}$) for
soil is from {\hyperref[\detokenize{tech_note/References/CLM50_Tech_Note_References:devries1963}]{\sphinxcrossref{\DUrole{std,std-ref}{de Vries (1963)}}}} and depends on the
heat capacities of the soil solid, liquid water, and ice constituents
\phantomsection\label{\detokenize{tech_note/Soil_Snow_Temperatures/CLM50_Tech_Note_Soil_Snow_Temperatures:equation-6.89}}\begin{equation}\label{equation:tech_note/Soil_Snow_Temperatures/CLM50_Tech_Note_Soil_Snow_Temperatures:6.89}
\begin{split}c_{i} =c_{s,\, i} \left(1-\theta _{sat,\, i} \right)+\frac{w_{ice,\, i} }{\Delta z_{i} } C_{ice} +\frac{w_{liq,\, i} }{\Delta z_{i} } C_{liq}\end{split}
\end{equation}
where \(C_{liq}\)  and \(C_{ice}\)  are the specific heat
capacities (J kg$^{\text{-1}}$ K$^{\text{-1}}$) of liquid water
and ice, respectively (\hyperref[\detokenize{tech_note/Ecosystem/CLM50_Tech_Note_Ecosystem:table-physical-constants}]{Table \ref{\detokenize{tech_note/Ecosystem/CLM50_Tech_Note_Ecosystem:table-physical-constants}}}). The heat capacity of soil solids
\(c_{s,i}\) (J m$^{\text{-3}}$ K$^{\text{-1}}$) is
\phantomsection\label{\detokenize{tech_note/Soil_Snow_Temperatures/CLM50_Tech_Note_Soil_Snow_Temperatures:equation-6.90}}\begin{equation}\label{equation:tech_note/Soil_Snow_Temperatures/CLM50_Tech_Note_Soil_Snow_Temperatures:6.90}
\begin{split}c_{s,i} =(1-f_{om,i} )c_{s,\min ,i} +f_{om,i} c_{s,om}\end{split}
\end{equation}
where the heat capacity of mineral soil solids
\(c_{s,\min ,\, i}\)  (J m$^{\text{-3}}$ K$^{\text{-1}}$) is
\phantomsection\label{\detokenize{tech_note/Soil_Snow_Temperatures/CLM50_Tech_Note_Soil_Snow_Temperatures:equation-6.91}}\begin{equation}\label{equation:tech_note/Soil_Snow_Temperatures/CLM50_Tech_Note_Soil_Snow_Temperatures:6.91}
\begin{split}\begin{array}{lr}
c_{s,\min ,\, i} =\left(\frac{2.128{\rm \; }\left(\% sand\right)_{i} +{\rm 2.385\; }\left(\% clay\right)_{i} }{\left(\% sand\right)_{i} +\left(\% clay\right)_{i} } \right)\times 10^{6} &\qquad i=1,\ldots ,N_{levsoi}  \\
c_{s,\, \min ,i} =c_{s,\, bedrock} &\qquad i=N_{levsoi} +1,\ldots ,N_{levgrnd}
\end{array}\end{split}
\end{equation}
where \(c_{s,bedrock} =2\times 10^{6}\)  J m$^{\text{-3}}$
K$^{\text{-1}}$ is the heat capacity of bedrock and
\(c_{s,om} =2.5\times 10^{6}\) J m$^{\text{-3}}$
K$^{\text{-1}}$ ({\hyperref[\detokenize{tech_note/References/CLM50_Tech_Note_References:farouki1981}]{\sphinxcrossref{\DUrole{std,std-ref}{Farouki 1981}}}}) is the heat capacity of organic
matter. For glaciers, wetlands, and snow
\phantomsection\label{\detokenize{tech_note/Soil_Snow_Temperatures/CLM50_Tech_Note_Soil_Snow_Temperatures:equation-6.92}}\begin{equation}\label{equation:tech_note/Soil_Snow_Temperatures/CLM50_Tech_Note_Soil_Snow_Temperatures:6.92}
\begin{split}c_{i} =\frac{w_{ice,\, i} }{\Delta z_{i} } C_{ice} +\frac{w_{liq,\, i} }{\Delta z_{i} } C_{liq} .\end{split}
\end{equation}
For the special case when snow is present (\(W_{sno} >0\)) but
there are no explicit snow layers (\(snl=0\)), the heat capacity of
the top layer is a blend of ice and soil heat capacity
\phantomsection\label{\detokenize{tech_note/Soil_Snow_Temperatures/CLM50_Tech_Note_Soil_Snow_Temperatures:equation-6.93}}\begin{equation}\label{equation:tech_note/Soil_Snow_Temperatures/CLM50_Tech_Note_Soil_Snow_Temperatures:6.93}
\begin{split}c_{1} =c_{1}^{*} +\frac{C_{ice} W_{sno} }{\Delta z_{1} }\end{split}
\end{equation}
where \(c_{1}^{*}\)  is calculated from \eqref{equation:tech_note/Soil_Snow_Temperatures/CLM50_Tech_Note_Soil_Snow_Temperatures:6.89} or \eqref{equation:tech_note/Soil_Snow_Temperatures/CLM50_Tech_Note_Soil_Snow_Temperatures:6.92}.


\section{Hydrology}
\label{\detokenize{tech_note/Hydrology/CLM50_Tech_Note_Hydrology:rst-hydrology}}\label{\detokenize{tech_note/Hydrology/CLM50_Tech_Note_Hydrology::doc}}\label{\detokenize{tech_note/Hydrology/CLM50_Tech_Note_Hydrology:hydrology}}
The model parameterizes interception, throughfall, canopy drip, snow
accumulation and melt, water transfer between snow layers, infiltration,
evaporation, surface runoff, sub-surface drainage, redistribution within
the soil column, and groundwater discharge and recharge to simulate
changes in canopy water \(\Delta W_{can,\,liq}\) , canopy snow water
\(\Delta W_{can,\,sno}\) surface water \(\Delta W_{sfc}\) ,
snow water \(\Delta W_{sno}\) , soil water
\(\Delta w_{liq,\, i}\) , and soil ice \(\Delta w_{ice,\, i}\) ,
and water in the unconfined aquifer \(\Delta W_{a}\)  (all in kg
m$^{\text{-2}}$ or mm of H$_{\text{2}}$O) (\hyperref[\detokenize{tech_note/Hydrology/CLM50_Tech_Note_Hydrology:figure-hydrologic-processes}]{Figure \ref{\detokenize{tech_note/Hydrology/CLM50_Tech_Note_Hydrology:figure-hydrologic-processes}}}).

The total water balance of the system is
\phantomsection\label{\detokenize{tech_note/Hydrology/CLM50_Tech_Note_Hydrology:equation-7.1}}\begin{equation}\label{equation:tech_note/Hydrology/CLM50_Tech_Note_Hydrology:7.1}
\begin{split}\begin{array}{l} {\Delta W_{can,\,liq} +\Delta W_{can,\,sno} +\Delta W_{sfc} +\Delta W_{sno} +} \\ {\sum _{i=1}^{N_{levsoi} }\left(\Delta w_{liq,\, i} +\Delta w_{ice,\, i} \right)+\Delta W_{a} =\left(\begin{array}{l} {q_{rain} +q_{sno} -E_{v} -E_{g} -q_{over} } \\ {-q_{h2osfc} -q_{drai} -q_{rgwl} -q_{snwcp,\, ice} } \end{array}\right) \Delta t} \end{array}\end{split}
\end{equation}
where \(q_{rain}\)  is the liquid part of precipitation,
\(q_{sno}\)  is the solid part of precipitation, \(E_{v}\)  is
ET from vegetation (Chapter \hyperref[\detokenize{tech_note/Fluxes/CLM50_Tech_Note_Fluxes:rst-momentum-sensible-heat-and-latent-heat-fluxes}]{\ref{\detokenize{tech_note/Fluxes/CLM50_Tech_Note_Fluxes:rst-momentum-sensible-heat-and-latent-heat-fluxes}}}), \(E_{g}\)  is ground evaporation
(Chapter \hyperref[\detokenize{tech_note/Fluxes/CLM50_Tech_Note_Fluxes:rst-momentum-sensible-heat-and-latent-heat-fluxes}]{\ref{\detokenize{tech_note/Fluxes/CLM50_Tech_Note_Fluxes:rst-momentum-sensible-heat-and-latent-heat-fluxes}}}), \(q_{over}\)  is surface runoff (section \hyperref[\detokenize{tech_note/Hydrology/CLM50_Tech_Note_Hydrology:surface-runoff}]{\ref{\detokenize{tech_note/Hydrology/CLM50_Tech_Note_Hydrology:surface-runoff}}}),
\(q_{h2osfc}\)  is runoff from surface water storage (section \hyperref[\detokenize{tech_note/Hydrology/CLM50_Tech_Note_Hydrology:surface-runoff}]{\ref{\detokenize{tech_note/Hydrology/CLM50_Tech_Note_Hydrology:surface-runoff}}}),
\(q_{drai}\)  is sub-surface drainage (section \hyperref[\detokenize{tech_note/Hydrology/CLM50_Tech_Note_Hydrology:lateral-sub-surface-runoff}]{\ref{\detokenize{tech_note/Hydrology/CLM50_Tech_Note_Hydrology:lateral-sub-surface-runoff}}}),
\(q_{rgwl}\)  and \(q_{snwcp,ice}\)  are liquid and solid runoff
from glaciers, wetlands, and lakes, and runoff from other surface types
due to snow capping (section \hyperref[\detokenize{tech_note/Hydrology/CLM50_Tech_Note_Hydrology:runoff-from-glaciers-and-snow-capped-surfaces}]{\ref{\detokenize{tech_note/Hydrology/CLM50_Tech_Note_Hydrology:runoff-from-glaciers-and-snow-capped-surfaces}}}) (all in kg m$^{\text{-2}}$
s$^{\text{-1}}$), \(N_{levsoi}\)  is the number of soil layers
(note that hydrology calculations are only done over soil layers 1 to
\(N_{levsoi}\) ; ground levels \(N_{levsoi} +1\) to
\(N_{levgrnd}\) are currently hydrologically inactive; {\hyperref[\detokenize{tech_note/References/CLM50_Tech_Note_References:lawrenceetal2008}]{\sphinxcrossref{\DUrole{std,std-ref}{(Lawrence et
al. 2008)}}}} and \(\Delta t\) is the time step (s).

\begin{figure}[htbp]
\centering
\capstart

\noindent\sphinxincludegraphics{{hydrologic.processes}.png}
\caption{Hydrologic processes represented in CLM.}\label{\detokenize{tech_note/Hydrology/CLM50_Tech_Note_Hydrology:figure-hydrologic-processes}}\label{\detokenize{tech_note/Hydrology/CLM50_Tech_Note_Hydrology:id12}}\end{figure}


\subsection{Canopy Water}
\label{\detokenize{tech_note/Hydrology/CLM50_Tech_Note_Hydrology:canopy-water}}\label{\detokenize{tech_note/Hydrology/CLM50_Tech_Note_Hydrology:id1}}
Liquid precipitation is either intercepted by the canopy, falls
directly to the snow/soil surface (throughfall), or drips off the
vegetation (canopy drip). Solid precipitation is treated similarly,
with the addition of unloading of previously intercepted snow.
Interception by vegetation is divided between liquid and solid phases
\(q_{intr,\,liq}\) and \(q_{intr,\,ice}\)
(kg m$^{\text{-2}}$ s$^{\text{-1}}$)
\phantomsection\label{\detokenize{tech_note/Hydrology/CLM50_Tech_Note_Hydrology:equation-7.2}}\begin{equation}\label{equation:tech_note/Hydrology/CLM50_Tech_Note_Hydrology:7.2}
\begin{split}q_{intr,\,liq} = f_{pi,\,liq} \ q_{rain}\end{split}
\end{equation}\phantomsection\label{\detokenize{tech_note/Hydrology/CLM50_Tech_Note_Hydrology:equation-7.3}}\begin{equation}\label{equation:tech_note/Hydrology/CLM50_Tech_Note_Hydrology:7.3}
\begin{split}q_{intr,\,ice} = f_{pi,\,ice} \ q_{sno}\end{split}
\end{equation}
where \(f_{pi,\,liq}\) and \(f_{pi,\,ice}\) are the
fractions of intercepted precipitation of rain and snow,
respectively
\phantomsection\label{\detokenize{tech_note/Hydrology/CLM50_Tech_Note_Hydrology:equation-7.2b}}\begin{equation}\label{equation:tech_note/Hydrology/CLM50_Tech_Note_Hydrology:7.2b}
\begin{split}f_{pi,\,liq} = \alpha_{liq} \ tanh \left(L+S\right)\end{split}
\end{equation}\phantomsection\label{\detokenize{tech_note/Hydrology/CLM50_Tech_Note_Hydrology:equation-7.3b}}\begin{equation}\label{equation:tech_note/Hydrology/CLM50_Tech_Note_Hydrology:7.3b}
\begin{split}f_{pi,\,ice} =\alpha_{sno} \ \left\{1-\exp \left[-0.5\left(L+S\right)\right]\right\} \ ,\end{split}
\end{equation}
and \(L\) and \(S\) are the exposed leaf and stem area index,
respectively (section \hyperref[\detokenize{tech_note/Ecosystem/CLM50_Tech_Note_Ecosystem:phenology-and-vegetation-burial-by-snow}]{\ref{\detokenize{tech_note/Ecosystem/CLM50_Tech_Note_Ecosystem:phenology-and-vegetation-burial-by-snow}}}), and
the \(\alpha\)‘s scale the fractional area of a leaf that collects water
({\hyperref[\detokenize{tech_note/References/CLM50_Tech_Note_References:lawrenceetal2007}]{\sphinxcrossref{\DUrole{std,std-ref}{Lawrence et al. 2007}}}}).  Default values of
\(\alpha_{liq}\) and \(\alpha_{sno}\) are set to 1.
Throughfall (kg m$^{\text{-2}}$ s$^{\text{-1}}$) is also divided into
liquid and solid phases, reaching the ground (soil or snow surface) as
\phantomsection\label{\detokenize{tech_note/Hydrology/CLM50_Tech_Note_Hydrology:equation-7.4}}\begin{equation}\label{equation:tech_note/Hydrology/CLM50_Tech_Note_Hydrology:7.4}
\begin{split}q_{thru,\, liq} = q_{rain} \left(1 - f_{pi,\,liq}\right)\end{split}
\end{equation}\phantomsection\label{\detokenize{tech_note/Hydrology/CLM50_Tech_Note_Hydrology:equation-7.5}}\begin{equation}\label{equation:tech_note/Hydrology/CLM50_Tech_Note_Hydrology:7.5}
\begin{split}q_{thru,\, ice} = q_{sno} \left(1 - f_{pi,\,ice}\right)\end{split}
\end{equation}
Similarly, the liquid and solid canopy drip fluxes are
\phantomsection\label{\detokenize{tech_note/Hydrology/CLM50_Tech_Note_Hydrology:equation-7.6}}\begin{equation}\label{equation:tech_note/Hydrology/CLM50_Tech_Note_Hydrology:7.6}
\begin{split}q_{drip,\, liq} =\frac{W_{can,\,liq}^{intr} -W_{can,\,liq}^{max } }{\Delta t} \ge 0\end{split}
\end{equation}\phantomsection\label{\detokenize{tech_note/Hydrology/CLM50_Tech_Note_Hydrology:equation-7.7}}\begin{equation}\label{equation:tech_note/Hydrology/CLM50_Tech_Note_Hydrology:7.7}
\begin{split}q_{drip,\, ice} =\frac{W_{can,\,sno}^{intr} -W_{can,\,sno}^{max } }{\Delta t} \ge 0\end{split}
\end{equation}
where
\phantomsection\label{\detokenize{tech_note/Hydrology/CLM50_Tech_Note_Hydrology:equation-7.8}}\begin{equation}\label{equation:tech_note/Hydrology/CLM50_Tech_Note_Hydrology:7.8}
\begin{split}W_{can,liq}^{intr} =W_{can,liq}^{n} +q_{intr,\, liq} \Delta t\ge 0\end{split}
\end{equation}
and
\phantomsection\label{\detokenize{tech_note/Hydrology/CLM50_Tech_Note_Hydrology:equation-7.9}}\begin{equation}\label{equation:tech_note/Hydrology/CLM50_Tech_Note_Hydrology:7.9}
\begin{split}W_{can,sno}^{intr} =W_{can,sno}^{n} +q_{intr,\, ice} \Delta t\ge 0\end{split}
\end{equation}
are the the canopy liquid water and snow water equivalent after accounting for interception,
\(W_{can,\,liq}^{n}\) and \(W_{can,\,sno}^{n}\) are the canopy liquid and snow water
from the previous time step, and \(W_{can,\,liq}^{max }\) and \(W_{can,\,snow}^{max }\)
(kg m$^{\text{-2}}$ or mm of H$_{\text{2}}$O) are the maximum amounts of liquid water and snow the canopy can hold.
They are defined by
\phantomsection\label{\detokenize{tech_note/Hydrology/CLM50_Tech_Note_Hydrology:equation-7.10}}\begin{equation}\label{equation:tech_note/Hydrology/CLM50_Tech_Note_Hydrology:7.10}
\begin{split}W_{can,\,liq}^{max } =p_{liq}\left(L+S\right)\end{split}
\end{equation}\phantomsection\label{\detokenize{tech_note/Hydrology/CLM50_Tech_Note_Hydrology:equation-7.11}}\begin{equation}\label{equation:tech_note/Hydrology/CLM50_Tech_Note_Hydrology:7.11}
\begin{split}W_{can,\,sno}^{max } =p_{ice}\left(L+S\right).\end{split}
\end{equation}
The maximum storage of liquid water is \(p_{liq}=0.1\) kg m$^{\text{-2}}$
({\hyperref[\detokenize{tech_note/References/CLM50_Tech_Note_References:dickinsonetal1993}]{\sphinxcrossref{\DUrole{std,std-ref}{Dickinson et al. 1993}}}}), and that of snow
is \(p_{sno}=6\) kg m$^{\text{-2}}$, consistent with reported
field measurements ({\hyperref[\detokenize{tech_note/References/CLM50_Tech_Note_References:pomeroyetal1998}]{\sphinxcrossref{\DUrole{std,std-ref}{Pomeroy et al. 1998}}}}).

Canopy snow unloading from wind speed \(u\) and above-freezing temperatures are modeled from linear
fluxes and e-folding times similar to {\hyperref[\detokenize{tech_note/References/CLM50_Tech_Note_References:roeschetal2001}]{\sphinxcrossref{\DUrole{std,std-ref}{Roesch et al. (2001)}}}}
\phantomsection\label{\detokenize{tech_note/Hydrology/CLM50_Tech_Note_Hydrology:equation-7.12}}\begin{equation}\label{equation:tech_note/Hydrology/CLM50_Tech_Note_Hydrology:7.12}
\begin{split}q_{unl,\, wind} =\frac{u W_{can,sno}}{1.56\times 10^5 \text{ m}}\end{split}
\end{equation}\phantomsection\label{\detokenize{tech_note/Hydrology/CLM50_Tech_Note_Hydrology:equation-7.13}}\begin{equation}\label{equation:tech_note/Hydrology/CLM50_Tech_Note_Hydrology:7.13}
\begin{split}q_{unl,\, temp} =\frac{W_{can,sno}(T-270 \textrm{ K})}{1.87\times 10^5 \text{ K s}} > 0\end{split}
\end{equation}\phantomsection\label{\detokenize{tech_note/Hydrology/CLM50_Tech_Note_Hydrology:equation-7.14}}\begin{equation}\label{equation:tech_note/Hydrology/CLM50_Tech_Note_Hydrology:7.14}
\begin{split}q_{unl,\, tot} =\min \left( q_{unl,\, wind} +q_{unl,\, temp} ,W_{can,\, sno} \right)\end{split}
\end{equation}
The canopy liquid water and snow water equivalent are updated as
\phantomsection\label{\detokenize{tech_note/Hydrology/CLM50_Tech_Note_Hydrology:equation-7.15}}\begin{equation}\label{equation:tech_note/Hydrology/CLM50_Tech_Note_Hydrology:7.15}
\begin{split} W_{can,\, liq}^{n+1} =W_{can,liq}^{n} + q_{intr,\, liq} - q_{drip,\, liq} \Delta t - E_{v}^{liq} \Delta t \ge 0\end{split}
\end{equation}
and
\phantomsection\label{\detokenize{tech_note/Hydrology/CLM50_Tech_Note_Hydrology:equation-7.16}}\begin{equation}\label{equation:tech_note/Hydrology/CLM50_Tech_Note_Hydrology:7.16}
\begin{split}W_{can,\, sno}^{n+1} =W_{can,sno}^{n} + q_{intr,\, ice} - \left(q_{drip,\, ice}+q_{unl,\, tot} \right)\Delta t
                      - E_{v}^{ice} \Delta t \ge 0\end{split}
\end{equation}
where \(E_{v}^{liq}\) and \(E_{v}^{ice}\) are partitioned from the stem and leaf
surface evaporation \(E_{v}^{w}\) (Chapter \hyperref[\detokenize{tech_note/Fluxes/CLM50_Tech_Note_Fluxes:rst-momentum-sensible-heat-and-latent-heat-fluxes}]{\ref{\detokenize{tech_note/Fluxes/CLM50_Tech_Note_Fluxes:rst-momentum-sensible-heat-and-latent-heat-fluxes}}}) based on the vegetation temperature \(T_{v}\) (K) (Chapter \hyperref[\detokenize{tech_note/Fluxes/CLM50_Tech_Note_Fluxes:rst-momentum-sensible-heat-and-latent-heat-fluxes}]{\ref{\detokenize{tech_note/Fluxes/CLM50_Tech_Note_Fluxes:rst-momentum-sensible-heat-and-latent-heat-fluxes}}}) and its relation to the freezing temperature of water \(T_{f}\) (K) (\hyperref[\detokenize{tech_note/Ecosystem/CLM50_Tech_Note_Ecosystem:table-physical-constants}]{Table \ref{\detokenize{tech_note/Ecosystem/CLM50_Tech_Note_Ecosystem:table-physical-constants}}})
\phantomsection\label{\detokenize{tech_note/Hydrology/CLM50_Tech_Note_Hydrology:equation-7.17}}\begin{equation}\label{equation:tech_note/Hydrology/CLM50_Tech_Note_Hydrology:7.17}
\begin{split}E_{v}^{liq} =
\left\{\begin{array}{lr}
E_{v}^{w} &  T_v > T_{f} \\
0         &  T_v \le T_f
\end{array}\right\}\end{split}
\end{equation}\phantomsection\label{\detokenize{tech_note/Hydrology/CLM50_Tech_Note_Hydrology:equation-7.18}}\begin{equation}\label{equation:tech_note/Hydrology/CLM50_Tech_Note_Hydrology:7.18}
\begin{split}E_{v}^{ice} =
\left\{\begin{array}{lr}
0         & T_v > T_f \\
E_{v}^{w} & T_v \le T_f
\end{array}\right\}.\end{split}
\end{equation}
The total rate of liquid and solid precipitation reaching the ground is then
\phantomsection\label{\detokenize{tech_note/Hydrology/CLM50_Tech_Note_Hydrology:equation-7.19}}\begin{equation}\label{equation:tech_note/Hydrology/CLM50_Tech_Note_Hydrology:7.19}
\begin{split}q_{grnd,liq} =q_{thru,\, liq} +q_{drip,\, liq}\end{split}
\end{equation}\phantomsection\label{\detokenize{tech_note/Hydrology/CLM50_Tech_Note_Hydrology:equation-7.20}}\begin{equation}\label{equation:tech_note/Hydrology/CLM50_Tech_Note_Hydrology:7.20}
\begin{split}q_{grnd,ice} =q_{thru,\, ice} +q_{drip,\, ice} +q_{unl,\, tot} .\end{split}
\end{equation}
Solid precipitation reaching the soil or snow surface,
\(q_{grnd,\, ice} \Delta t\), is added immediately to the snow pack
(Chapter \hyperref[\detokenize{tech_note/Snow_Hydrology/CLM50_Tech_Note_Snow_Hydrology:rst-snow-hydrology}]{\ref{\detokenize{tech_note/Snow_Hydrology/CLM50_Tech_Note_Snow_Hydrology:rst-snow-hydrology}}}). The liquid part,
\(q_{grnd,\, liq} \Delta t\) is added after surface fluxes
(Chapter \hyperref[\detokenize{tech_note/Fluxes/CLM50_Tech_Note_Fluxes:rst-momentum-sensible-heat-and-latent-heat-fluxes}]{\ref{\detokenize{tech_note/Fluxes/CLM50_Tech_Note_Fluxes:rst-momentum-sensible-heat-and-latent-heat-fluxes}}})
and snow/soil temperatures (Chapter \hyperref[\detokenize{tech_note/Soil_Snow_Temperatures/CLM50_Tech_Note_Soil_Snow_Temperatures:rst-soil-and-snow-temperatures}]{\ref{\detokenize{tech_note/Soil_Snow_Temperatures/CLM50_Tech_Note_Soil_Snow_Temperatures:rst-soil-and-snow-temperatures}}})
have been determined.

The wetted fraction of the canopy (stems plus leaves), which is required
for surface flux (Chapter \hyperref[\detokenize{tech_note/Fluxes/CLM50_Tech_Note_Fluxes:rst-momentum-sensible-heat-and-latent-heat-fluxes}]{\ref{\detokenize{tech_note/Fluxes/CLM50_Tech_Note_Fluxes:rst-momentum-sensible-heat-and-latent-heat-fluxes}}})
calculations, is ({\hyperref[\detokenize{tech_note/References/CLM50_Tech_Note_References:dickinsonetal1993}]{\sphinxcrossref{\DUrole{std,std-ref}{Dickinson et al.1993}}}})
\phantomsection\label{\detokenize{tech_note/Hydrology/CLM50_Tech_Note_Hydrology:equation-7.21}}\begin{equation}\label{equation:tech_note/Hydrology/CLM50_Tech_Note_Hydrology:7.21}
\begin{split}f_{wet} =
\left\{\begin{array}{lr}
\left[\frac{W_{can} }{p_{liq}\left(L+S\right)} \right]^{{2\mathord{\left/ {\vphantom {2 3}} \right. \kern-\nulldelimiterspace} 3} } \le 1 & \qquad L+S > 0 \\
0 &\qquad L+S = 0
\end{array}\right\}\end{split}
\end{equation}
while the fraction of the canopy that is dry and transpiring is
\phantomsection\label{\detokenize{tech_note/Hydrology/CLM50_Tech_Note_Hydrology:equation-7.22}}\begin{equation}\label{equation:tech_note/Hydrology/CLM50_Tech_Note_Hydrology:7.22}
\begin{split}f_{dry} =
\left\{\begin{array}{lr}
\frac{\left(1-f_{wet} \right)L}{L+S} & \qquad L+S > 0 \\
0 &\qquad L+S = 0
\end{array}\right\}.\end{split}
\end{equation}
Similarly, the snow-covered fraction of the canopy is used for surface alebdo when intercepted snow is present (Chapter \hyperref[\detokenize{tech_note/Surface_Albedos/CLM50_Tech_Note_Surface_Albedos:rst-surface-albedos}]{\ref{\detokenize{tech_note/Surface_Albedos/CLM50_Tech_Note_Surface_Albedos:rst-surface-albedos}}})
\phantomsection\label{\detokenize{tech_note/Hydrology/CLM50_Tech_Note_Hydrology:equation-7.23}}\begin{equation}\label{equation:tech_note/Hydrology/CLM50_Tech_Note_Hydrology:7.23}
\begin{split}f_{can,\, sno} =
\left\{\begin{array}{lr}
\left[\frac{W_{can,\, sno} }{p_{sno}\left(L+S\right)} \right]^{{3\mathord{\left/ {\vphantom {3 20}} \right. \kern-\nulldelimiterspace} 20} } \le 1 & \qquad L+S > 0 \\
0 &\qquad L+S = 0
\end{array}\right\}.\end{split}
\end{equation}

\subsection{Surface Runoff, Surface Water Storage, and Infiltration}
\label{\detokenize{tech_note/Hydrology/CLM50_Tech_Note_Hydrology:surface-runoff-surface-water-storage-and-infiltration}}\label{\detokenize{tech_note/Hydrology/CLM50_Tech_Note_Hydrology:id2}}
The moisture input at the grid cell surface ,\(q_{liq,\, 0}\) , is
the sum of liquid precipitation reaching the ground and melt water from
snow (kg m$^{\text{-2}}$ s$^{\text{-1}}$). The moisture flux is
then partitioned between surface runoff, surface water storage, and
infiltration into the soil.


\subsubsection{Surface Runoff}
\label{\detokenize{tech_note/Hydrology/CLM50_Tech_Note_Hydrology:id3}}\label{\detokenize{tech_note/Hydrology/CLM50_Tech_Note_Hydrology:surface-runoff}}
The simple TOPMODEL-based ({\hyperref[\detokenize{tech_note/References/CLM50_Tech_Note_References:bevenkirkby1979}]{\sphinxcrossref{\DUrole{std,std-ref}{Beven and Kirkby 1979}}}})
runoff model (SIMTOP) described by {\hyperref[\detokenize{tech_note/References/CLM50_Tech_Note_References:niuetal2005}]{\sphinxcrossref{\DUrole{std,std-ref}{Niu et al. (2005)}}}}
is implemented to parameterize runoff. A
key concept underlying this approach is that of fractional saturated
area \(f_{sat}\) , which is determined by the topographic
characteristics and soil moisture state of a grid cell. The saturated
portion of a grid cell contributes to surface runoff, \(q_{over}\) ,
by the saturation excess mechanism (Dunne runoff)
\phantomsection\label{\detokenize{tech_note/Hydrology/CLM50_Tech_Note_Hydrology:equation-7.64}}\begin{equation}\label{equation:tech_note/Hydrology/CLM50_Tech_Note_Hydrology:7.64}
\begin{split}q_{over} =f_{sat} \ q_{liq,\, 0}\end{split}
\end{equation}
The fractional saturated area is a function of soil moisture
\phantomsection\label{\detokenize{tech_note/Hydrology/CLM50_Tech_Note_Hydrology:equation-7.65}}\begin{equation}\label{equation:tech_note/Hydrology/CLM50_Tech_Note_Hydrology:7.65}
\begin{split}f_{sat} =f_{\max } \ \exp \left(-0.5f_{over} z_{\nabla } \right)\end{split}
\end{equation}
where \(f_{\max }\)  is the potential or maximum value of
\(f_{sat}\) , \(f_{over}\)  is a decay factor (m$^{\text{-1}}$), and
\(z_{\nabla}\) is the water table depth (m) (section
\hyperref[\detokenize{tech_note/Hydrology/CLM50_Tech_Note_Hydrology:lateral-sub-surface-runoff}]{\ref{\detokenize{tech_note/Hydrology/CLM50_Tech_Note_Hydrology:lateral-sub-surface-runoff}}}). The maximum saturated fraction,
\(f_{\max }\), is defined as the value of the discrete cumulative
distribution function (CDF) of the topographic index when the grid cell
mean water table depth is zero. Thus, \(f_{\max }\)  is the percent of
pixels in a grid cell whose topographic index is larger than or equal to
the grid cell mean topographic index. It should be calculated explicitly
from the CDF at each grid cell at the resolution that the model is run.
However, because this is a computationally intensive task for global
applications, \(f_{\max }\)  is calculated once at 0.125$^{\text{o}}$
resolution using the 1-km compound topographic indices (CTIs) based on
the HYDRO1K dataset ({\hyperref[\detokenize{tech_note/References/CLM50_Tech_Note_References:verdingreenlee1996}]{\sphinxcrossref{\DUrole{std,std-ref}{Verdin and Greenlee 1996}}}})
from USGS following the algorithm in {\hyperref[\detokenize{tech_note/References/CLM50_Tech_Note_References:niuetal2005}]{\sphinxcrossref{\DUrole{std,std-ref}{Niu et al. (2005)}}}}
and then area-averaged to the desired model resolution (section
\hyperref[\detokenize{tech_note/Ecosystem/CLM50_Tech_Note_Ecosystem:surface-data}]{\ref{\detokenize{tech_note/Ecosystem/CLM50_Tech_Note_Ecosystem:surface-data}}}). Pixels
with CTIs exceeding the 95 percentile threshold in each
0.125$^{\text{o}}$ grid cell are excluded from the calculation to
eliminate biased estimation of statistics due to large CTI values at
pixels on stream networks. For grid cells over regions without CTIs such
as Australia, the global mean \(f_{\max }\)  is used to fill the
gaps. See {\hyperref[\detokenize{tech_note/References/CLM50_Tech_Note_References:lietal2013b}]{\sphinxcrossref{\DUrole{std,std-ref}{Li et al. (2013b)}}}} for additional details. The decay factor
\(f_{over}\)  for global simulations was determined through
sensitivity analysis and comparison with observed runoff to be 0.5
m$^{\text{-1}}$.


\subsubsection{Surface Water Storage}
\label{\detokenize{tech_note/Hydrology/CLM50_Tech_Note_Hydrology:id4}}\label{\detokenize{tech_note/Hydrology/CLM50_Tech_Note_Hydrology:surface-water-storage}}
A surface water store has been added to the model to represent wetlands
and small, sub-grid scale water bodies. As a result, the wetland land
unit has been removed. The state variables for surface water are the
mass of water \(W_{sfc}\)  (kg m$^{\text{-2}}$) and temperature
\(T_{h2osfc}\)  (Chapter \hyperref[\detokenize{tech_note/Soil_Snow_Temperatures/CLM50_Tech_Note_Soil_Snow_Temperatures:rst-soil-and-snow-temperatures}]{\ref{\detokenize{tech_note/Soil_Snow_Temperatures/CLM50_Tech_Note_Soil_Snow_Temperatures:rst-soil-and-snow-temperatures}}}).
Surface water storage and outflow are
functions of fine spatial scale elevation variations called
microtopography. The microtopography is assumed to be distributed
normally around the grid cell mean elevation. Given the standard
deviation of the microtopographic distribution, \(\sigma _{micro}\)
(m), the fractional area of the grid cell that is inundated can be
calculated. Surface water storage, \(Wsfc\), is related to the
height (relative to the grid cell mean elevation) of the surface water,
\(d\), by
\phantomsection\label{\detokenize{tech_note/Hydrology/CLM50_Tech_Note_Hydrology:equation-7.66}}\begin{equation}\label{equation:tech_note/Hydrology/CLM50_Tech_Note_Hydrology:7.66}
\begin{split}W_{sfc} =\frac{d}{2} \left(1+erf\left(\frac{d}{\sigma _{micro} \sqrt{2} } \right)\right)+\frac{\sigma _{micro} }{\sqrt{2\pi } } e^{\frac{-d^{2} }{2\sigma _{micro} ^{2} } }\end{split}
\end{equation}
where \(erf\) is the error function. For a given value of
\(W_{sfc}\) , \eqref{equation:tech_note/Hydrology/CLM50_Tech_Note_Hydrology:7.66} can be solved for \(d\) using the
Newton-Raphson method. Once \(d\) is known, one can determine the
fraction of the area that is inundated as
\phantomsection\label{\detokenize{tech_note/Hydrology/CLM50_Tech_Note_Hydrology:equation-7.67}}\begin{equation}\label{equation:tech_note/Hydrology/CLM50_Tech_Note_Hydrology:7.67}
\begin{split}f_{h2osfc} =\frac{1}{2} \left(1+erf\left(\frac{d}{\sigma _{micro} \sqrt{2} } \right)\right)\end{split}
\end{equation}
No global datasets exist for microtopography, so the default
parameterization is a simple function of slope
\phantomsection\label{\detokenize{tech_note/Hydrology/CLM50_Tech_Note_Hydrology:equation-7.68}}\begin{equation}\label{equation:tech_note/Hydrology/CLM50_Tech_Note_Hydrology:7.68}
\begin{split}\sigma _{micro} =\left(\beta +\beta _{0} \right)^{\eta }\end{split}
\end{equation}
where \(\beta\)  is the topographic slope,
\(\beta_{0} =\left(\sigma_{\max } \right)^{\frac{1}{\eta } }\) determines
the maximum value of \(\sigma_{micro}\) , and \(\eta\)  is an
adjustable parameter. Default values in the model are
\(\sigma_{\max } =0.4\) and \(\eta =-3\).

If the spatial scale of the microtopography is small relative to that of
the grid cell, one can assume that the inundated areas are distributed
randomly within the grid cell. With this assumption, a result from
percolation theory can be used to quantify the fraction of the inundated
portion of the grid cell that is interconnected
\phantomsection\label{\detokenize{tech_note/Hydrology/CLM50_Tech_Note_Hydrology:equation-7.69}}\begin{equation}\label{equation:tech_note/Hydrology/CLM50_Tech_Note_Hydrology:7.69}
\begin{split}\begin{array}{lr} f_{connected} =\left(f_{h2osfc} -f_{c} \right)^{\mu } & \qquad f_{h2osfc} >f_{c}  \\ f_{connected} =0 &\qquad  f_{h2osfc} \le f_{c}  \end{array}\end{split}
\end{equation}
where \(f_{c}\)  is a threshold below which no single connected
inundated area spans the grid cell and \(\mu\)  is a scaling
exponent. Default values of \(f_{c}\)  and \(\mu\) are 0.4 and
0.14, respectively. When the inundated fraction of the grid cell
surpasses \(f_{c}\) , the surface water store acts as a linear
reservoir
\phantomsection\label{\detokenize{tech_note/Hydrology/CLM50_Tech_Note_Hydrology:equation-7.70}}\begin{equation}\label{equation:tech_note/Hydrology/CLM50_Tech_Note_Hydrology:7.70}
\begin{split}q_{out,h2osfc}=k_{h2osfc} \ f_{connected} \ (Wsfc-Wc)\frac{1}{\Delta t}\end{split}
\end{equation}
where \(q_{out,h2osfc}\) is the surface water runoff, \(k_{h2osfc}\)
is a constant, \(Wc\) is the amount of surface water present when
\(f_{h2osfc} =f_{c}\) , and \(\Delta t\) is the model time step.
The linear storage coefficent \(k_{h2osfc} = \sin \left(\beta \right)\)
is a function of grid cell mean topographic slope where \(\beta\)
is the slope in radians.


\subsubsection{Infiltration}
\label{\detokenize{tech_note/Hydrology/CLM50_Tech_Note_Hydrology:id5}}\label{\detokenize{tech_note/Hydrology/CLM50_Tech_Note_Hydrology:infiltration}}
The surface moisture flux remaining after surface runoff has been
removed,
\phantomsection\label{\detokenize{tech_note/Hydrology/CLM50_Tech_Note_Hydrology:equation-7.71}}\begin{equation}\label{equation:tech_note/Hydrology/CLM50_Tech_Note_Hydrology:7.71}
\begin{split}q_{in,surface} = (1-f_{sat}) \ q_{liq,\, 0}\end{split}
\end{equation}
is divided into inputs to surface water (\(q_{in,\, h2osfc}\) ) and
the soil \(q_{in,soil}\) . If \(q_{in,soil}\)  exceeds the
maximum soil infiltration capacity (kg m$^{\text{-2}}$
s$^{\text{-1}}$),
\phantomsection\label{\detokenize{tech_note/Hydrology/CLM50_Tech_Note_Hydrology:equation-7.72}}\begin{equation}\label{equation:tech_note/Hydrology/CLM50_Tech_Note_Hydrology:7.72}
\begin{split}q_{infl,\, \max } =(1-f_{sat}) \ \Theta_{ice} k_{sat}\end{split}
\end{equation}
where \(\Theta_{ice}\) is an ice impedance factor (section
\hyperref[\detokenize{tech_note/Hydrology/CLM50_Tech_Note_Hydrology:hydraulic-properties}]{\ref{\detokenize{tech_note/Hydrology/CLM50_Tech_Note_Hydrology:hydraulic-properties}}}), infiltration excess (Hortonian) runoff is generated
\phantomsection\label{\detokenize{tech_note/Hydrology/CLM50_Tech_Note_Hydrology:equation-7.73}}\begin{equation}\label{equation:tech_note/Hydrology/CLM50_Tech_Note_Hydrology:7.73}
\begin{split}q_{infl,\, excess} =\max \left(q_{in,soil} -\left(1-f_{h2osfc} \right)q_{\inf l,\max } ,0\right)\end{split}
\end{equation}
and transferred from \(q_{in,soil}\)  to \(q_{in,h2osfc}\) .
After evaporative losses have been removed, these moisture fluxes are
\phantomsection\label{\detokenize{tech_note/Hydrology/CLM50_Tech_Note_Hydrology:equation-7.74}}\begin{equation}\label{equation:tech_note/Hydrology/CLM50_Tech_Note_Hydrology:7.74}
\begin{split}q_{in,\, h2osfc} = f_{h2osfc} q_{in,surface} + q_{infl,excess} - q_{evap,h2osfc}\end{split}
\end{equation}
and
\phantomsection\label{\detokenize{tech_note/Hydrology/CLM50_Tech_Note_Hydrology:equation-7.75}}\begin{equation}\label{equation:tech_note/Hydrology/CLM50_Tech_Note_Hydrology:7.75}
\begin{split}q_{in,soil} = (1-f_{h2osfc} ) \ q_{in,surface} - q_{\inf l,excess} - (1 - f_{sno} - f_{h2osfc} ) \ q_{evap,soil}.\end{split}
\end{equation}
The balance of surface water is then calculated as
\phantomsection\label{\detokenize{tech_note/Hydrology/CLM50_Tech_Note_Hydrology:equation-7.76}}\begin{equation}\label{equation:tech_note/Hydrology/CLM50_Tech_Note_Hydrology:7.76}
\begin{split}\Delta W_{sfc} =\left(q_{in,h2osfc} - q_{out,h2osfc} - q_{drain,h2osfc} \right) \ \Delta t.\end{split}
\end{equation}
Bottom drainage from the surface water store
\phantomsection\label{\detokenize{tech_note/Hydrology/CLM50_Tech_Note_Hydrology:equation-7.77}}\begin{equation}\label{equation:tech_note/Hydrology/CLM50_Tech_Note_Hydrology:7.77}
\begin{split}q_{drain,h2osfc} = \min \left(f_{h2osfc} q_{\inf l,\max } ,\frac{W_{sfc} }{\Delta t} \right)\end{split}
\end{equation}
is then added to \(q_{in,soil}\)  giving the total infiltration
into the surface soil layer
\phantomsection\label{\detokenize{tech_note/Hydrology/CLM50_Tech_Note_Hydrology:equation-7.78}}\begin{equation}\label{equation:tech_note/Hydrology/CLM50_Tech_Note_Hydrology:7.78}
\begin{split}q_{infl} = q_{in,soil} + q_{drain,h2osfc}\end{split}
\end{equation}
Infiltration \(q_{infl}\)  and explicit surface runoff
\(q_{over}\)  are not allowed for glaciers.


\subsection{Soil Water}
\label{\detokenize{tech_note/Hydrology/CLM50_Tech_Note_Hydrology:id6}}\label{\detokenize{tech_note/Hydrology/CLM50_Tech_Note_Hydrology:soil-water}}
Soil water is predicted from a multi-layer model, in which the vertical
soil moisture transport is governed by infiltration, surface and
sub-surface runoff, gradient diffusion, gravity, and canopy transpiration
through root extraction (\hyperref[\detokenize{tech_note/Hydrology/CLM50_Tech_Note_Hydrology:figure-hydrologic-processes}]{Figure \ref{\detokenize{tech_note/Hydrology/CLM50_Tech_Note_Hydrology:figure-hydrologic-processes}}}).

For one-dimensional vertical water flow in soils, the conservation of
mass is stated as
\phantomsection\label{\detokenize{tech_note/Hydrology/CLM50_Tech_Note_Hydrology:equation-7.79}}\begin{equation}\label{equation:tech_note/Hydrology/CLM50_Tech_Note_Hydrology:7.79}
\begin{split}\frac{\partial \theta }{\partial t} =-\frac{\partial q}{\partial z} - e\end{split}
\end{equation}
where \(\theta\)  is the volumetric soil water content
(mm$^{\text{3}}$ of water / mm$^{\text{-3}}$ of soil), \(t\) is
time (s), \(z\) is height above some datum in the soil column (mm)
(positive upwards), \(q\) is soil water flux (kg m$^{\text{-2}}$
s$^{\text{-1}}$ or mm s$^{\text{-1}}$) (positive upwards), and
\(e\) is a soil moisture sink term (mm of water mm$^{\text{-1}}$
of soil s$^{\text{-1}}$) (ET loss). This equation is solved
numerically by dividing the soil column into multiple layers in the
vertical and integrating downward over each layer with an upper boundary
condition of the infiltration flux into the top soil layer
\(q_{infl}\)  and a zero-flux lower boundary condition at the
bottom of the soil column (sub-surface runoff is removed later in the
timestep, section \hyperref[\detokenize{tech_note/Hydrology/CLM50_Tech_Note_Hydrology:lateral-sub-surface-runoff}]{\ref{\detokenize{tech_note/Hydrology/CLM50_Tech_Note_Hydrology:lateral-sub-surface-runoff}}}).

The soil water flux \(q\) in equation can be described by Darcy’s
law {\hyperref[\detokenize{tech_note/References/CLM50_Tech_Note_References:dingman2002}]{\sphinxcrossref{\DUrole{std,std-ref}{(Dingman 2002)}}}}
\phantomsection\label{\detokenize{tech_note/Hydrology/CLM50_Tech_Note_Hydrology:equation-7.80}}\begin{equation}\label{equation:tech_note/Hydrology/CLM50_Tech_Note_Hydrology:7.80}
\begin{split}q = -k \frac{\partial \psi _{h} }{\partial z}\end{split}
\end{equation}
where \(k\) is the hydraulic conductivity (mm s$^{\text{-1}}$),
and \(\psi _{h}\)  is the hydraulic potential (mm). The hydraulic
potential is
\phantomsection\label{\detokenize{tech_note/Hydrology/CLM50_Tech_Note_Hydrology:equation-7.81}}\begin{equation}\label{equation:tech_note/Hydrology/CLM50_Tech_Note_Hydrology:7.81}
\begin{split}\psi _{h} =\psi _{m} +\psi _{z}\end{split}
\end{equation}
where \(\psi _{m}\)  is the soil matric potential (mm) (which is
related to the adsorptive and capillary forces within the soil matrix),
and \(\psi _{z}\)  is the gravitational potential (mm) (the vertical
distance from an arbitrary reference elevation to a point in the soil).
If the reference elevation is the soil surface, then
\(\psi _{z} =z\). Letting \(\psi =\psi _{m}\) , Darcy’s law
becomes
\phantomsection\label{\detokenize{tech_note/Hydrology/CLM50_Tech_Note_Hydrology:equation-7.82}}\begin{equation}\label{equation:tech_note/Hydrology/CLM50_Tech_Note_Hydrology:7.82}
\begin{split}q = -k \left[\frac{\partial \left(\psi +z\right)}{\partial z} \right].\end{split}
\end{equation}
Equation \eqref{equation:tech_note/Hydrology/CLM50_Tech_Note_Hydrology:7.82} can be further manipulated to yield
\phantomsection\label{\detokenize{tech_note/Hydrology/CLM50_Tech_Note_Hydrology:equation-7.83}}\begin{equation}\label{equation:tech_note/Hydrology/CLM50_Tech_Note_Hydrology:7.83}
\begin{split}q = -k \left[\frac{\partial \left(\psi +z\right)}{\partial z} \right]
= -k \left(\frac{\partial \psi }{\partial z} + 1 \right) \ .\end{split}
\end{equation}
Substitution of this equation into equation \eqref{equation:tech_note/Hydrology/CLM50_Tech_Note_Hydrology:7.79}, with \(e = 0\), yields
the Richards equation {\hyperref[\detokenize{tech_note/References/CLM50_Tech_Note_References:dingman2002}]{\sphinxcrossref{\DUrole{std,std-ref}{(Dingman 2002)}}}}
\phantomsection\label{\detokenize{tech_note/Hydrology/CLM50_Tech_Note_Hydrology:equation-7.84}}\begin{equation}\label{equation:tech_note/Hydrology/CLM50_Tech_Note_Hydrology:7.84}
\begin{split}\frac{\partial \theta }{\partial t} =
\frac{\partial }{\partial z} \left[k\left(\frac{\partial \psi }{\partial z} + 1
\right)\right].\end{split}
\end{equation}
In practice (Section \hyperref[\detokenize{tech_note/Hydrology/CLM50_Tech_Note_Hydrology:numerical-solution-hydrology}]{\ref{\detokenize{tech_note/Hydrology/CLM50_Tech_Note_Hydrology:numerical-solution-hydrology}}}), changes in soil
water content are predicted from \eqref{equation:tech_note/Hydrology/CLM50_Tech_Note_Hydrology:7.79} using finite-difference approximations
for \eqref{equation:tech_note/Hydrology/CLM50_Tech_Note_Hydrology:7.84}.


\subsubsection{Hydraulic Properties}
\label{\detokenize{tech_note/Hydrology/CLM50_Tech_Note_Hydrology:id7}}\label{\detokenize{tech_note/Hydrology/CLM50_Tech_Note_Hydrology:hydraulic-properties}}
The hydraulic conductivity \(k_{i}\)  (mm s$^{\text{-1}}$) and
the soil matric potential \(\psi _{i}\)  (mm) for layer \(i\)
vary with volumetric soil water \(\theta _{i}\)  and soil texture.
As with the soil thermal properties (section
\hyperref[\detokenize{tech_note/Soil_Snow_Temperatures/CLM50_Tech_Note_Soil_Snow_Temperatures:soil-and-snow-thermal-properties}]{\ref{\detokenize{tech_note/Soil_Snow_Temperatures/CLM50_Tech_Note_Soil_Snow_Temperatures:soil-and-snow-thermal-properties}}}) the hydraulic
properties of the soil are assumed to be a weighted combination of the
mineral properties, which are determined according to sand and clay
contents based on work by {\hyperref[\detokenize{tech_note/References/CLM50_Tech_Note_References:clapphornberger1978}]{\sphinxcrossref{\DUrole{std,std-ref}{Clapp and Hornberger (1978)}}}} and {\hyperref[\detokenize{tech_note/References/CLM50_Tech_Note_References:cosbyetal1984}]{\sphinxcrossref{\DUrole{std,std-ref}{Cosby et al. (1984)}}}},
and organic properties of the soil
({\hyperref[\detokenize{tech_note/References/CLM50_Tech_Note_References:lawrenceslater2008}]{\sphinxcrossref{\DUrole{std,std-ref}{Lawrence and Slater 2008}}}}).

The hydraulic conductivity is defined at the depth of the interface of
two adjacent layers \(z_{h,\, i}\)  (\hyperref[\detokenize{tech_note/Hydrology/CLM50_Tech_Note_Hydrology:figure-water-flux-schematic}]{Figure \ref{\detokenize{tech_note/Hydrology/CLM50_Tech_Note_Hydrology:figure-water-flux-schematic}}})
and is a function of the saturated hydraulic conductivity
\(k_{sat} \left[z_{h,\, i} \right]\), the liquid volumetric soil
moisture of the two layers \(\theta _{i}\)  and \(\theta_{i+1}\)
and an ice impedance factor \(\Theta_{ice}\)
\phantomsection\label{\detokenize{tech_note/Hydrology/CLM50_Tech_Note_Hydrology:equation-7.85}}\begin{equation}\label{equation:tech_note/Hydrology/CLM50_Tech_Note_Hydrology:7.85}
\begin{split}k\left[z_{h,\, i} \right] =
\left\{\begin{array}{lr}
\Theta_{ice} k_{sat} \left[z_{h,\, i} \right]\left[\frac{0.5\left(\theta_{\, i} +\theta_{\, i+1} \right)}{0.5\left(\theta_{sat,\, i} +\theta_{sat,\, i+1} \right)} \right]^{2B_{i} +3} & \qquad 1 \le i \le N_{levsoi} - 1 \\
\Theta_{ice} k_{sat} \left[z_{h,\, i} \right]\left(\frac{\theta_{\, i} }{\theta_{sat,\, i} } \right)^{2B_{i} +3} & \qquad i = N_{levsoi}
\end{array}\right\}.\end{split}
\end{equation}
The ice impedance factor is a function of ice content, and is meant to
quantify the increased tortuosity of the water flow when part of the
pore space is filled with ice. {\hyperref[\detokenize{tech_note/References/CLM50_Tech_Note_References:swensonetal2012}]{\sphinxcrossref{\DUrole{std,std-ref}{Swenson et al. (2012)}}}}
used a power law form
\phantomsection\label{\detokenize{tech_note/Hydrology/CLM50_Tech_Note_Hydrology:equation-7.86}}\begin{equation}\label{equation:tech_note/Hydrology/CLM50_Tech_Note_Hydrology:7.86}
\begin{split}\Theta_{ice} = 10^{-\Omega F_{ice} }\end{split}
\end{equation}
where \(\Omega = 6\)and \(F_{ice} = \frac{\theta_{ice} }{\theta_{sat} }\)
is the ice-filled fraction of the pore space.

Because the hydraulic properties of mineral and organic soil may differ
significantly, the bulk hydraulic properties of each soil layer are
computed as weighted averages of the properties of the mineral and
organic components. The water content at saturation (i.e. porosity) is
\phantomsection\label{\detokenize{tech_note/Hydrology/CLM50_Tech_Note_Hydrology:equation-7.90}}\begin{equation}\label{equation:tech_note/Hydrology/CLM50_Tech_Note_Hydrology:7.90}
\begin{split}\theta_{sat,i} =(1-f_{om,i} )\theta_{sat,\min ,i} +f_{om,i} \theta_{sat,om}\end{split}
\end{equation}
where \(f_{om,i}\)  is the soil organic matter fraction,
\(\theta_{sat,om} =0.9\) ({\hyperref[\detokenize{tech_note/References/CLM50_Tech_Note_References:farouki1981}]{\sphinxcrossref{\DUrole{std,std-ref}{Farouki 1981}}}};
{\hyperref[\detokenize{tech_note/References/CLM50_Tech_Note_References:lettsetal2000}]{\sphinxcrossref{\DUrole{std,std-ref}{Letts et al. 2000}}}}) is the
porosity of organic matter and the porosity of the mineral soil
\(\theta_{sat,\min ,i}\)  is
\phantomsection\label{\detokenize{tech_note/Hydrology/CLM50_Tech_Note_Hydrology:equation-7.91}}\begin{equation}\label{equation:tech_note/Hydrology/CLM50_Tech_Note_Hydrology:7.91}
\begin{split}\theta_{sat,\min ,i} = 0.489 - 0.00126(\% sand)_{i} .\end{split}
\end{equation}
The exponent \(B_{i}\) is
\phantomsection\label{\detokenize{tech_note/Hydrology/CLM50_Tech_Note_Hydrology:equation-7.92}}\begin{equation}\label{equation:tech_note/Hydrology/CLM50_Tech_Note_Hydrology:7.92}
\begin{split}B_{i} =(1-f_{om,i} )B_{\min ,i} +f_{om,i} B_{om}\end{split}
\end{equation}
where \(B_{om} = 2.7\) ({\hyperref[\detokenize{tech_note/References/CLM50_Tech_Note_References:lettsetal2000}]{\sphinxcrossref{\DUrole{std,std-ref}{Letts et al. 2000}}}}) and
\phantomsection\label{\detokenize{tech_note/Hydrology/CLM50_Tech_Note_Hydrology:equation-7.93}}\begin{equation}\label{equation:tech_note/Hydrology/CLM50_Tech_Note_Hydrology:7.93}
\begin{split}B_{\min ,i} =2.91+0.159(\% clay)_{i} .\end{split}
\end{equation}
The soil matric potential (mm) is defined at the node depth
\(z_{i}\)  of each layer \(i\) (\hyperref[\detokenize{tech_note/Hydrology/CLM50_Tech_Note_Hydrology:figure-water-flux-schematic}]{Figure \ref{\detokenize{tech_note/Hydrology/CLM50_Tech_Note_Hydrology:figure-water-flux-schematic}}})
\phantomsection\label{\detokenize{tech_note/Hydrology/CLM50_Tech_Note_Hydrology:equation-7.94}}\begin{equation}\label{equation:tech_note/Hydrology/CLM50_Tech_Note_Hydrology:7.94}
\begin{split}\psi _{i} =\psi _{sat,\, i} \left(\frac{\theta_{\, i} }{\theta_{sat,\, i} } \right)^{-B_{i} } \ge -1\times 10^{8} \qquad 0.01\le \frac{\theta_{i} }{\theta_{sat,\, i} } \le 1\end{split}
\end{equation}
where the saturated soil matric potential (mm) is
\phantomsection\label{\detokenize{tech_note/Hydrology/CLM50_Tech_Note_Hydrology:equation-7.95}}\begin{equation}\label{equation:tech_note/Hydrology/CLM50_Tech_Note_Hydrology:7.95}
\begin{split}\psi _{sat,i} =(1-f_{om,i} )\psi _{sat,\min ,i} +f_{om,i} \psi _{sat,om}\end{split}
\end{equation}
where \(\psi _{sat,om} = -10.3\) mm ({\hyperref[\detokenize{tech_note/References/CLM50_Tech_Note_References:lettsetal2000}]{\sphinxcrossref{\DUrole{std,std-ref}{Letts et al. 2000}}}}) is the
saturated organic matter matric potential and the saturated mineral soil
matric potential \(\psi _{sat,\min ,i}\) is
\phantomsection\label{\detokenize{tech_note/Hydrology/CLM50_Tech_Note_Hydrology:equation-7.96}}\begin{equation}\label{equation:tech_note/Hydrology/CLM50_Tech_Note_Hydrology:7.96}
\begin{split}\psi _{sat,\, \min ,\, i} =-10.0\times 10^{1.88-0.0131(\% sand)_{i} } .\end{split}
\end{equation}
The saturated hydraulic conductivity,
\(k_{sat} \left[z_{h,\, i} \right]\) (mm s$^{\text{-1}}$), for
organic soils (\(k_{sat,\, om}\) ) may be two to three orders of
magnitude larger than that of mineral soils (\(k_{sat,\, \min }\) ).
Bulk soil layer values of \(k_{sat}\) calculated as weighted
averages based on \(f_{om}\)  may therefore be determined primarily
by the organic soil properties even for values of \(f_{om}\)  as low
as 1 \%. To better represent the influence of organic soil material on
the grid cell average saturated hydraulic conductivity, the soil organic
matter fraction is further subdivided into “connected” and “unconnected”
fractions using a result from percolation theory ({\hyperref[\detokenize{tech_note/References/CLM50_Tech_Note_References:staufferaharony1994}]{\sphinxcrossref{\DUrole{std,std-ref}{Stauffer and Aharony
1994}}}}, {\hyperref[\detokenize{tech_note/References/CLM50_Tech_Note_References:berkowitzbalberg1992}]{\sphinxcrossref{\DUrole{std,std-ref}{Berkowitz and Balberg 1992}}}}).
Assuming that the organic and mineral fractions are randomly distributed throughout
a soil layer, percolation theory predicts that above a threshold value
\(f_{om} = f_{threshold}\), connected flow pathways consisting of
organic material only exist and span the soil space. Flow through these
pathways interacts only with organic material, and thus can be described
by \(k_{sat,\, om}\). This fraction of the grid cell is given by
\phantomsection\label{\detokenize{tech_note/Hydrology/CLM50_Tech_Note_Hydrology:equation-7.97}}\begin{equation}\label{equation:tech_note/Hydrology/CLM50_Tech_Note_Hydrology:7.97}
\begin{split}\begin{array}{lr}
f_{perc} =\; N_{perc} \left(f_{om} {\rm \; }-f_{threshold} \right)^{\beta_{perc} } f_{om} {\rm \; } & \qquad f_{om} \ge f_{threshold}  \\
f_{perc} = 0 & \qquad f_{om} <f_{threshold}
\end{array}\end{split}
\end{equation}
where \(\beta ^{perc} =0.139\), \(f_{threshold} =0.5\), and
\(N_{perc} =\left(1-f_{threshold} \right)^{-\beta_{perc} }\) . In
the unconnected portion of the grid cell,
\(f_{uncon} =\; \left(1-f_{perc} {\rm \; }\right)\), the saturated
hydraulic conductivity is assumed to correspond to flow pathways that
pass through the mineral and organic components in series
\phantomsection\label{\detokenize{tech_note/Hydrology/CLM50_Tech_Note_Hydrology:equation-7.98}}\begin{equation}\label{equation:tech_note/Hydrology/CLM50_Tech_Note_Hydrology:7.98}
\begin{split}k_{sat,\, uncon} =f_{uncon} \left(\frac{\left(1-f_{om} \right)}{k_{sat,\, \min } } +\frac{\left(f_{om} -f_{perc} \right)}{k_{sat,\, om} } \right)^{-1} .\end{split}
\end{equation}
where saturated hydraulic conductivity for mineral soil depends on soil
texture ({\hyperref[\detokenize{tech_note/References/CLM50_Tech_Note_References:cosbyetal1984}]{\sphinxcrossref{\DUrole{std,std-ref}{Cosby et al. 1984}}}}) as
\phantomsection\label{\detokenize{tech_note/Hydrology/CLM50_Tech_Note_Hydrology:equation-7.99}}\begin{equation}\label{equation:tech_note/Hydrology/CLM50_Tech_Note_Hydrology:7.99}
\begin{split}k_{sat,\, \min } \left[z_{h,\, i} \right]=0.0070556\times 10^{-0.884+0.0153\left(\% sand\right)_{i} } .\end{split}
\end{equation}
The bulk soil layer saturated hydraulic conductivity is then computed
as
\phantomsection\label{\detokenize{tech_note/Hydrology/CLM50_Tech_Note_Hydrology:equation-7.100}}\begin{equation}\label{equation:tech_note/Hydrology/CLM50_Tech_Note_Hydrology:7.100}
\begin{split}k_{sat} \left[z_{h,\, i} \right]=f_{uncon,\, i} k_{sat,\, uncon} \left[z_{h,\, i} \right]+(1-f_{uncon,\, i} )k_{sat,\, om} \left[z_{h,\, i} \right].\end{split}
\end{equation}

\subsubsection{Numerical Solution}
\label{\detokenize{tech_note/Hydrology/CLM50_Tech_Note_Hydrology:numerical-solution}}\label{\detokenize{tech_note/Hydrology/CLM50_Tech_Note_Hydrology:numerical-solution-hydrology}}
With reference to \hyperref[\detokenize{tech_note/Hydrology/CLM50_Tech_Note_Hydrology:figure-water-flux-schematic}]{Figure \ref{\detokenize{tech_note/Hydrology/CLM50_Tech_Note_Hydrology:figure-water-flux-schematic}}}, the equation for
conservation of mass (equation \eqref{equation:tech_note/Hydrology/CLM50_Tech_Note_Hydrology:7.79}) can be integrated over each layer as
\phantomsection\label{\detokenize{tech_note/Hydrology/CLM50_Tech_Note_Hydrology:equation-7.101}}\begin{equation}\label{equation:tech_note/Hydrology/CLM50_Tech_Note_Hydrology:7.101}
\begin{split}\int _{-z_{h,\, i} }^{-z_{h,\, i-1} }\frac{\partial \theta }{\partial t} \,  dz=-\int _{-z_{h,\, i} }^{-z_{h,\, i-1} }\frac{\partial q}{\partial z}  \, dz-\int _{-z_{h,\, i} }^{-z_{h,\, i-1} } e\, dz .\end{split}
\end{equation}
Note that the integration limits are negative since \(z\) is defined
as positive upward from the soil surface. This equation can be written
as
\phantomsection\label{\detokenize{tech_note/Hydrology/CLM50_Tech_Note_Hydrology:equation-7.102}}\begin{equation}\label{equation:tech_note/Hydrology/CLM50_Tech_Note_Hydrology:7.102}
\begin{split}\Delta z_{i} \frac{\partial \theta_{liq,\, i} }{\partial t} =-q_{i-1} +q_{i} -e_{i}\end{split}
\end{equation}
where \(q_{i}\)  is the flux of water across interface
\(z_{h,\, i}\) , \(q_{i-1}\)  is the flux of water across
interface \(z_{h,\, i-1}\) , and \(e_{i}\)  is a layer-averaged
soil moisture sink term (ET loss) defined as positive for flow out of
the layer (mm s$^{\text{-1}}$). Taking the finite difference with
time and evaluating the fluxes implicitly at time \(n+1\) yields
\phantomsection\label{\detokenize{tech_note/Hydrology/CLM50_Tech_Note_Hydrology:equation-7.103}}\begin{equation}\label{equation:tech_note/Hydrology/CLM50_Tech_Note_Hydrology:7.103}
\begin{split}\frac{\Delta z_{i} \Delta \theta_{liq,\, i} }{\Delta t} =-q_{i-1}^{n+1} +q_{i}^{n+1} -e_{i}\end{split}
\end{equation}
where
\(\Delta \theta_{liq,\, i} =\theta_{liq,\, i}^{n+1} -\theta_{liq,\, i}^{n}\)
is the change in volumetric soil liquid water of layer \(i\) in time
\(\Delta t\)and \(\Delta z_{i}\)  is the thickness of layer
\(i\) (mm).

The water removed by transpiration in each layer \(e_{i}\)  is a
function of the total transpiration \(E_{v}^{t}\)  (Chapter \hyperref[\detokenize{tech_note/Fluxes/CLM50_Tech_Note_Fluxes:rst-momentum-sensible-heat-and-latent-heat-fluxes}]{\ref{\detokenize{tech_note/Fluxes/CLM50_Tech_Note_Fluxes:rst-momentum-sensible-heat-and-latent-heat-fluxes}}}) and
the effective root fraction \(r_{e,\, i}\)
\phantomsection\label{\detokenize{tech_note/Hydrology/CLM50_Tech_Note_Hydrology:equation-7.104}}\begin{equation}\label{equation:tech_note/Hydrology/CLM50_Tech_Note_Hydrology:7.104}
\begin{split}e_{i} =r_{e,\, i} E_{v}^{t} .\end{split}
\end{equation}
\begin{figure}[htbp]
\centering
\capstart

\noindent\sphinxincludegraphics{{image21}.png}
\caption{Schematic diagram of numerical scheme used to solve for soil water fluxes.}\label{\detokenize{tech_note/Hydrology/CLM50_Tech_Note_Hydrology:figure-water-flux-schematic}}\label{\detokenize{tech_note/Hydrology/CLM50_Tech_Note_Hydrology:id13}}\end{figure}

Shown are three soil layers, \(i-1\), \(i\), and \(i+1\).
The soil matric potential \(\psi\)  and volumetric soil water
\(\theta_{liq}\)  are defined at the layer node depth \(z\).
The hydraulic conductivity \(k\left[z_{h} \right]\) is defined at
the interface of two layers \(z_{h}\) . The layer thickness is
\(\Delta z\). The soil water fluxes \(q_{i-1}\)  and
\(q_{i}\)  are defined as positive upwards. The soil moisture sink
term \(e\) (ET loss) is defined as positive for flow out of the
layer.

Note that because more than one plant functional type (PFT) may share a
soil column, the transpiration \(E_{v}^{t}\)  is a weighted sum of
transpiration from all PFTs whose weighting depends on PFT area as
\phantomsection\label{\detokenize{tech_note/Hydrology/CLM50_Tech_Note_Hydrology:equation-7.105}}\begin{equation}\label{equation:tech_note/Hydrology/CLM50_Tech_Note_Hydrology:7.105}
\begin{split}E_{v}^{t} =\sum _{j=1}^{npft}\left(E_{v}^{t} \right)_{j} \left(wt\right)_{j}\end{split}
\end{equation}
where \(npft\) is the number of PFTs sharing a soil column,
\(\left(E_{v}^{t} \right)_{j}\)  is the transpiration from the
\(j^{th}\)  PFT on the column, and \(\left(wt\right)_{j}\)  is
the relative area of the \(j^{th}\)  PFT with respect to the column.
The effective root fraction \(r_{e,\, i}\)  is also a column-level
quantity that is a weighted sum over all PFTs. The weighting depends on
the per unit area transpiration of each PFT and its relative area as
\phantomsection\label{\detokenize{tech_note/Hydrology/CLM50_Tech_Note_Hydrology:equation-7.106}}\begin{equation}\label{equation:tech_note/Hydrology/CLM50_Tech_Note_Hydrology:7.106}
\begin{split}r_{e,\, i} =\frac{\sum _{j=1}^{npft}\left(r_{e,\, i} \right)_{j} \left(E_{v}^{t} \right)_{j} \left(wt\right)_{j}  }{\sum _{j=1}^{npft}\left(E_{v}^{t} \right)_{j} \left(wt\right)_{j}  }\end{split}
\end{equation}
where \(\left(r_{e,\, i} \right)_{j}\)  is the effective root
fraction for the \(j^{th}\)  PFT
\phantomsection\label{\detokenize{tech_note/Hydrology/CLM50_Tech_Note_Hydrology:equation-7.107}}\begin{equation}\label{equation:tech_note/Hydrology/CLM50_Tech_Note_Hydrology:7.107}
\begin{split}\begin{array}{lr}
\left(r_{e,\, i} \right)_{j} =\frac{\left(r_{i} \right)_{j} \left(w_{i} \right)_{j} }{\left(\beta _{t} \right)_{j} } & \qquad \left(\beta _{t} \right)_{j} >0 \\
\left(r_{e,\, i} \right)_{j} =0 & \qquad \left(\beta _{t} \right)_{j} =0
\end{array}\end{split}
\end{equation}
and \(\left(r_{i} \right)_{j}\)  is the fraction of roots in layer
\(i\) (Chapter \hyperref[\detokenize{tech_note/Photosynthesis/CLM50_Tech_Note_Photosynthesis:rst-stomatal-resistance-and-photosynthesis}]{\ref{\detokenize{tech_note/Photosynthesis/CLM50_Tech_Note_Photosynthesis:rst-stomatal-resistance-and-photosynthesis}}}),
\(\left(w_{i} \right)_{j}\)  is a soil dryness or plant wilting factor
for layer \(i\) (Chapter \hyperref[\detokenize{tech_note/Photosynthesis/CLM50_Tech_Note_Photosynthesis:rst-stomatal-resistance-and-photosynthesis}]{\ref{\detokenize{tech_note/Photosynthesis/CLM50_Tech_Note_Photosynthesis:rst-stomatal-resistance-and-photosynthesis}}}), and \(\left(\beta_{t} \right)_{j}\)  is a wetness factor for the total
soil column for the \(j^{th}\)  PFT (Chapter \hyperref[\detokenize{tech_note/Photosynthesis/CLM50_Tech_Note_Photosynthesis:rst-stomatal-resistance-and-photosynthesis}]{\ref{\detokenize{tech_note/Photosynthesis/CLM50_Tech_Note_Photosynthesis:rst-stomatal-resistance-and-photosynthesis}}}).

The soil water fluxes in \eqref{equation:tech_note/Hydrology/CLM50_Tech_Note_Hydrology:7.103},, which are a function of
\(\theta_{liq,\, i}\)  and \(\theta_{liq,\, i+1}\)  because of
their dependence on hydraulic conductivity and soil matric potential,
can be linearized about \(\theta\)  using a Taylor series expansion
as
\phantomsection\label{\detokenize{tech_note/Hydrology/CLM50_Tech_Note_Hydrology:equation-7.108}}\begin{equation}\label{equation:tech_note/Hydrology/CLM50_Tech_Note_Hydrology:7.108}
\begin{split}q_{i}^{n+1} =q_{i}^{n} +\frac{\partial q_{i} }{\partial \theta_{liq,\, i} } \Delta \theta_{liq,\, i} +\frac{\partial q_{i} }{\partial \theta_{liq,\, i+1} } \Delta \theta_{liq,\, i+1}\end{split}
\end{equation}\phantomsection\label{\detokenize{tech_note/Hydrology/CLM50_Tech_Note_Hydrology:equation-7.109}}\begin{equation}\label{equation:tech_note/Hydrology/CLM50_Tech_Note_Hydrology:7.109}
\begin{split}q_{i-1}^{n+1} =q_{i-1}^{n} +\frac{\partial q_{i-1} }{\partial \theta_{liq,\, i-1} } \Delta \theta_{liq,\, i-1} +\frac{\partial q_{i-1} }{\partial \theta_{liq,\, i} } \Delta \theta_{liq,\, i} .\end{split}
\end{equation}
Substitution of these expressions for \(q_{i}^{n+1}\)  and
\(q_{i-1}^{n+1}\)  into \eqref{equation:tech_note/Hydrology/CLM50_Tech_Note_Hydrology:7.103} results in a general tridiagonal
equation set of the form
\phantomsection\label{\detokenize{tech_note/Hydrology/CLM50_Tech_Note_Hydrology:equation-7.110}}\begin{equation}\label{equation:tech_note/Hydrology/CLM50_Tech_Note_Hydrology:7.110}
\begin{split}r_{i} =a_{i} \Delta \theta_{liq,\, i-1} +b_{i} \Delta \theta_{liq,\, i} +c_{i} \Delta \theta_{liq,\, i+1}\end{split}
\end{equation}
where
\phantomsection\label{\detokenize{tech_note/Hydrology/CLM50_Tech_Note_Hydrology:equation-7.111}}\begin{equation}\label{equation:tech_note/Hydrology/CLM50_Tech_Note_Hydrology:7.111}
\begin{split}a_{i} =-\frac{\partial q_{i-1} }{\partial \theta_{liq,\, i-1} }\end{split}
\end{equation}\phantomsection\label{\detokenize{tech_note/Hydrology/CLM50_Tech_Note_Hydrology:equation-7.112}}\begin{equation}\label{equation:tech_note/Hydrology/CLM50_Tech_Note_Hydrology:7.112}
\begin{split}b_{i} =\frac{\partial q_{i} }{\partial \theta_{liq,\, i} } -\frac{\partial q_{i-1} }{\partial \theta_{liq,\, i} } -\frac{\Delta z_{i} }{\Delta t}\end{split}
\end{equation}\phantomsection\label{\detokenize{tech_note/Hydrology/CLM50_Tech_Note_Hydrology:equation-7.113}}\begin{equation}\label{equation:tech_note/Hydrology/CLM50_Tech_Note_Hydrology:7.113}
\begin{split}c_{i} =\frac{\partial q_{i} }{\partial \theta_{liq,\, i+1} }\end{split}
\end{equation}\phantomsection\label{\detokenize{tech_note/Hydrology/CLM50_Tech_Note_Hydrology:equation-7.114}}\begin{equation}\label{equation:tech_note/Hydrology/CLM50_Tech_Note_Hydrology:7.114}
\begin{split}r_{i} =q_{i-1}^{n} -q_{i}^{n} +e_{i} .\end{split}
\end{equation}
The tridiagonal equation set is solved over
\(i=1,\ldots ,N_{levsoi}\).

The finite-difference forms of the fluxes and partial derivatives in
equations \eqref{equation:tech_note/Hydrology/CLM50_Tech_Note_Hydrology:7.111} - \eqref{equation:tech_note/Hydrology/CLM50_Tech_Note_Hydrology:7.114} can be obtained from equation as
\phantomsection\label{\detokenize{tech_note/Hydrology/CLM50_Tech_Note_Hydrology:equation-7.115}}\begin{equation}\label{equation:tech_note/Hydrology/CLM50_Tech_Note_Hydrology:7.115}
\begin{split}q_{i-1}^{n} =-k\left[z_{h,\, i-1} \right]\left[\frac{\left(\psi _{i-1} -\psi _{i} \right)+\left(z_{i} - z_{i-1} \right)}{z_{i} -z_{i-1} } \right]\end{split}
\end{equation}\phantomsection\label{\detokenize{tech_note/Hydrology/CLM50_Tech_Note_Hydrology:equation-7.116}}\begin{equation}\label{equation:tech_note/Hydrology/CLM50_Tech_Note_Hydrology:7.116}
\begin{split}q_{i}^{n} =-k\left[z_{h,\, i} \right]\left[\frac{\left(\psi _{i} -\psi _{i+1} \right)+\left(z_{i+1} - z_{i} \right)}{z_{i+1} -z_{i} } \right]\end{split}
\end{equation}\phantomsection\label{\detokenize{tech_note/Hydrology/CLM50_Tech_Note_Hydrology:equation-7.117}}\begin{equation}\label{equation:tech_note/Hydrology/CLM50_Tech_Note_Hydrology:7.117}
\begin{split}\frac{\partial q_{i-1} }{\partial \theta _{liq,\, i-1} } =-\left[\frac{k\left[z_{h,\, i-1} \right]}{z_{i} -z_{i-1} } \frac{\partial \psi _{i-1} }{\partial \theta _{liq,\, i-1} } \right]-\frac{\partial k\left[z_{h,\, i-1} \right]}{\partial \theta _{liq,\, i-1} } \left[\frac{\left(\psi _{i-1} -\psi _{i} \right)+\left(z_{i} - z_{i-1} \right)}{z_{i} - z_{i-1} } \right]\end{split}
\end{equation}\phantomsection\label{\detokenize{tech_note/Hydrology/CLM50_Tech_Note_Hydrology:equation-7.118}}\begin{equation}\label{equation:tech_note/Hydrology/CLM50_Tech_Note_Hydrology:7.118}
\begin{split}\frac{\partial q_{i-1} }{\partial \theta _{liq,\, i} } =\left[\frac{k\left[z_{h,\, i-1} \right]}{z_{i} -z_{i-1} } \frac{\partial \psi _{i} }{\partial \theta _{liq,\, i} } \right]-\frac{\partial k\left[z_{h,\, i-1} \right]}{\partial \theta _{liq,\, i} } \left[\frac{\left(\psi _{i-1} -\psi _{i} \right)+\left(z_{i} - z_{i-1} \right)}{z_{i} - z_{i-1} } \right]\end{split}
\end{equation}\phantomsection\label{\detokenize{tech_note/Hydrology/CLM50_Tech_Note_Hydrology:equation-7.119}}\begin{equation}\label{equation:tech_note/Hydrology/CLM50_Tech_Note_Hydrology:7.119}
\begin{split}\frac{\partial q_{i} }{\partial \theta _{liq,\, i} } =-\left[\frac{k\left[z_{h,\, i} \right]}{z_{i+1} -z_{i} } \frac{\partial \psi _{i} }{\partial \theta _{liq,\, i} } \right]-\frac{\partial k\left[z_{h,\, i} \right]}{\partial \theta _{liq,\, i} } \left[\frac{\left(\psi _{i} -\psi _{i+1} \right)+\left(z_{i+1} - z_{i} \right)}{z_{i+1} - z_{i} } \right]\end{split}
\end{equation}\phantomsection\label{\detokenize{tech_note/Hydrology/CLM50_Tech_Note_Hydrology:equation-7.120}}\begin{equation}\label{equation:tech_note/Hydrology/CLM50_Tech_Note_Hydrology:7.120}
\begin{split}\frac{\partial q_{i} }{\partial \theta _{liq,\, i+1} } =\left[\frac{k\left[z_{h,\, i} \right]}{z_{i+1} -z_{i} } \frac{\partial \psi _{i+1} }{\partial \theta _{liq,\, i+1} } \right]-\frac{\partial k\left[z_{h,\, i} \right]}{\partial \theta _{liq,\, i+1} } \left[\frac{\left(\psi _{i} -\psi _{i+1} \right)+\left(z_{i+1} - z_{i} \right)}{z_{i+1} - z_{i} } \right].\end{split}
\end{equation}
The derivatives of the soil matric potential at the node depth are
derived from \eqref{equation:tech_note/Hydrology/CLM50_Tech_Note_Hydrology:7.94}
\phantomsection\label{\detokenize{tech_note/Hydrology/CLM50_Tech_Note_Hydrology:equation-7.121}}\begin{equation}\label{equation:tech_note/Hydrology/CLM50_Tech_Note_Hydrology:7.121}
\begin{split}\frac{\partial \psi _{i-1} }{\partial \theta_{liq,\, \, i-1} } =-B_{i-1} \frac{\psi _{i-1} }{\theta_{\, \, i-1} }\end{split}
\end{equation}\phantomsection\label{\detokenize{tech_note/Hydrology/CLM50_Tech_Note_Hydrology:equation-7.122}}\begin{equation}\label{equation:tech_note/Hydrology/CLM50_Tech_Note_Hydrology:7.122}
\begin{split}\frac{\partial \psi _{i} }{\partial \theta_{\, liq,\, i} } =-B_{i} \frac{\psi _{i} }{\theta_{i} }\end{split}
\end{equation}\phantomsection\label{\detokenize{tech_note/Hydrology/CLM50_Tech_Note_Hydrology:equation-7.123}}\begin{equation}\label{equation:tech_note/Hydrology/CLM50_Tech_Note_Hydrology:7.123}
\begin{split}\frac{\partial \psi _{i+1} }{\partial \theta_{liq,\, i+1} } =-B_{i+1} \frac{\psi _{i+1} }{\theta_{\, i+1} }\end{split}
\end{equation}
with the constraint
\(0.01\, \theta_{sat,\, i} \le \theta_{\, i} \le \theta_{sat,\, i}\) .

The derivatives of the hydraulic conductivity at the layer interface are
derived from \eqref{equation:tech_note/Hydrology/CLM50_Tech_Note_Hydrology:7.85}
\phantomsection\label{\detokenize{tech_note/Hydrology/CLM50_Tech_Note_Hydrology:equation-7.124}}\begin{equation}\label{equation:tech_note/Hydrology/CLM50_Tech_Note_Hydrology:7.124}
\begin{split}\begin{array}{l}
{\frac{\partial k\left[z_{h,\, i-1} \right]}{\partial \theta _{liq,\, i-1} }
= \frac{\partial k\left[z_{h,\, i-1} \right]}{\partial \theta _{liq,\, i} }
= \left(2B_{i-1} +3\right) \ \overline{\Theta}_{ice} \ k_{sat} \left[z_{h,\, i-1} \right] \ \left[\frac{\overline{\theta}_{liq}}{\overline{\theta}_{sat}} \right]^{2B_{i-1} +2} \left(\frac{0.5}{\overline{\theta}_{sat}} \right)} \end{array}\end{split}
\end{equation}
where \(\overline{\Theta}_{ice} = \Theta(\overline{\theta}_{ice})\) \eqref{equation:tech_note/Hydrology/CLM50_Tech_Note_Hydrology:7.86},
\(\overline{\theta}_{ice} = 0.5\left(\theta_{ice\, i-1} +\theta_{ice\, i} \right)\),
\(\overline{\theta}_{liq} = 0.5\left(\theta_{liq\, i-1} +\theta_{liq\, i} \right)\),
and
\(\overline{\theta}_{sat} = 0.5\left(\theta_{sat,\, i-1} +\theta_{sat,\, i} \right)\)

and
\phantomsection\label{\detokenize{tech_note/Hydrology/CLM50_Tech_Note_Hydrology:equation-7.125}}\begin{equation}\label{equation:tech_note/Hydrology/CLM50_Tech_Note_Hydrology:7.125}
\begin{split}\begin{array}{l}
{\frac{\partial k\left[z_{h,\, i} \right]}{\partial \theta _{liq,\, i} }
= \frac{\partial k\left[z_{h,\, i} \right]}{\partial \theta _{liq,\, i+1} }
= \left(2B_{i} +3\right) \ \overline{\Theta}_{ice} \ k_{sat} \left[z_{h,\, i} \right] \ \left[\frac{\overline{\theta}_{liq}}{\overline{\theta}_{sat}} \right]^{2B_{i} +2} \left(\frac{0.5}{\overline{\theta}_{sat}} \right)} \end{array}.\end{split}
\end{equation}
where \(\overline{\theta}_{liq} = 0.5\left(\theta_{\, i} +\theta_{\, i+1} \right)\),
\(\overline{\theta}_{sat} = 0.5\left(\theta_{sat,\, i} +\theta_{sat,\, i+1} \right)\).


\paragraph{Equation set for layer \protect\(i=1\protect\)}
\label{\detokenize{tech_note/Hydrology/CLM50_Tech_Note_Hydrology:equation-set-for-layer}}
For the top soil layer (\(i=1\)), the boundary condition is the
infiltration rate (section \hyperref[\detokenize{tech_note/Hydrology/CLM50_Tech_Note_Hydrology:surface-runoff}]{\ref{\detokenize{tech_note/Hydrology/CLM50_Tech_Note_Hydrology:surface-runoff}}}),
\(q_{i-1}^{n+1} =-q_{infl}^{n+1}\) , and the water balance equation
is
\phantomsection\label{\detokenize{tech_note/Hydrology/CLM50_Tech_Note_Hydrology:equation-7.135}}\begin{equation}\label{equation:tech_note/Hydrology/CLM50_Tech_Note_Hydrology:7.135}
\begin{split}\frac{\Delta z_{i} \Delta \theta_{liq,\, i} }{\Delta t} =q_{infl}^{n+1} +q_{i}^{n+1} -e_{i} .\end{split}
\end{equation}
After grouping like terms, the coefficients of the tridiagonal set of
equations for \(i=1\) are
\phantomsection\label{\detokenize{tech_note/Hydrology/CLM50_Tech_Note_Hydrology:equation-7.136}}\begin{equation}\label{equation:tech_note/Hydrology/CLM50_Tech_Note_Hydrology:7.136}
\begin{split}a_{i} =0\end{split}
\end{equation}\phantomsection\label{\detokenize{tech_note/Hydrology/CLM50_Tech_Note_Hydrology:equation-7.137}}\begin{equation}\label{equation:tech_note/Hydrology/CLM50_Tech_Note_Hydrology:7.137}
\begin{split}b_{i} =\frac{\partial q_{i} }{\partial \theta_{liq,\, i} } -\frac{\Delta z_{i} }{\Delta t}\end{split}
\end{equation}\phantomsection\label{\detokenize{tech_note/Hydrology/CLM50_Tech_Note_Hydrology:equation-7.138}}\begin{equation}\label{equation:tech_note/Hydrology/CLM50_Tech_Note_Hydrology:7.138}
\begin{split}c_{i} =\frac{\partial q_{i} }{\partial \theta_{liq,\, i+1} }\end{split}
\end{equation}\phantomsection\label{\detokenize{tech_note/Hydrology/CLM50_Tech_Note_Hydrology:equation-7.139}}\begin{equation}\label{equation:tech_note/Hydrology/CLM50_Tech_Note_Hydrology:7.139}
\begin{split}r_{i} =q_{infl}^{n+1} -q_{i}^{n} +e_{i} .\end{split}
\end{equation}

\paragraph{Equation set for layers \protect\(i=2,\ldots ,N_{levsoi} -1\protect\)}
\label{\detokenize{tech_note/Hydrology/CLM50_Tech_Note_Hydrology:equation-set-for-layers}}
The coefficients of the tridiagonal set of equations for
\(i=2,\ldots ,N_{levsoi} -1\) are
\phantomsection\label{\detokenize{tech_note/Hydrology/CLM50_Tech_Note_Hydrology:equation-7.140}}\begin{equation}\label{equation:tech_note/Hydrology/CLM50_Tech_Note_Hydrology:7.140}
\begin{split}a_{i} =-\frac{\partial q_{i-1} }{\partial \theta_{liq,\, i-1} }\end{split}
\end{equation}\phantomsection\label{\detokenize{tech_note/Hydrology/CLM50_Tech_Note_Hydrology:equation-7.141}}\begin{equation}\label{equation:tech_note/Hydrology/CLM50_Tech_Note_Hydrology:7.141}
\begin{split}b_{i} =\frac{\partial q_{i} }{\partial \theta_{liq,\, i} } -\frac{\partial q_{i-1} }{\partial \theta_{liq,\, i} } -\frac{\Delta z_{i} }{\Delta t}\end{split}
\end{equation}\phantomsection\label{\detokenize{tech_note/Hydrology/CLM50_Tech_Note_Hydrology:equation-7.142}}\begin{equation}\label{equation:tech_note/Hydrology/CLM50_Tech_Note_Hydrology:7.142}
\begin{split}c_{i} =\frac{\partial q_{i} }{\partial \theta_{liq,\, i+1} }\end{split}
\end{equation}\phantomsection\label{\detokenize{tech_note/Hydrology/CLM50_Tech_Note_Hydrology:equation-7.143}}\begin{equation}\label{equation:tech_note/Hydrology/CLM50_Tech_Note_Hydrology:7.143}
\begin{split}r_{i} =q_{i-1}^{n} -q_{i}^{n} +e_{i} .\end{split}
\end{equation}

\paragraph{Equation set for layer \protect\(i=N_{levsoi}\protect\)}
\label{\detokenize{tech_note/Hydrology/CLM50_Tech_Note_Hydrology:id8}}
For the lowest soil layer (\(i=N_{levsoi}\) ), a zero-flux bottom boundary
condition is applied (\(q_{i}^{n} =0\))
and the coefficients of the tridiagonal set of equations for
\(i=N_{levsoi}\)  are
\phantomsection\label{\detokenize{tech_note/Hydrology/CLM50_Tech_Note_Hydrology:equation-7.148}}\begin{equation}\label{equation:tech_note/Hydrology/CLM50_Tech_Note_Hydrology:7.148}
\begin{split}a_{i} =-\frac{\partial q_{i-1} }{\partial \theta_{liq,\, i-1} }\end{split}
\end{equation}\phantomsection\label{\detokenize{tech_note/Hydrology/CLM50_Tech_Note_Hydrology:equation-7.149}}\begin{equation}\label{equation:tech_note/Hydrology/CLM50_Tech_Note_Hydrology:7.149}
\begin{split}b_{i} =\frac{\partial q_{i} }{\partial \theta_{liq,\, i} } -\frac{\partial q_{i-1} }{\partial \theta_{liq,\, i} } -\frac{\Delta z_{i} }{\Delta t}\end{split}
\end{equation}\phantomsection\label{\detokenize{tech_note/Hydrology/CLM50_Tech_Note_Hydrology:equation-7.150}}\begin{equation}\label{equation:tech_note/Hydrology/CLM50_Tech_Note_Hydrology:7.150}
\begin{split}c_{i} =0\end{split}
\end{equation}\phantomsection\label{\detokenize{tech_note/Hydrology/CLM50_Tech_Note_Hydrology:equation-7.151}}\begin{equation}\label{equation:tech_note/Hydrology/CLM50_Tech_Note_Hydrology:7.151}
\begin{split}r_{i} =q_{i-1}^{n} +e_{i} .\end{split}
\end{equation}

\paragraph{Adaptive Time Stepping}
\label{\detokenize{tech_note/Hydrology/CLM50_Tech_Note_Hydrology:adaptive-time-stepping}}
The length of the time step is adjusted in order to improve the accuracy
and stability of the numerical solutions.  The difference between two numerical
approximations is used to  estimate the temporal truncation error, and then
the step size \(\Delta t_{sub}\) is adjusted to meet a user-prescribed error tolerance
{\hyperref[\detokenize{tech_note/References/CLM50_Tech_Note_References:kavetskietal2002}]{\sphinxcrossref{\DUrole{std,std-ref}{{[}Kavetski et al., 2002{]}}}}}.  The temporal truncation
error is estimated by comparing the flux obtained from the first-order
Taylor series expansion (\(q_{i-1}^{n+1}\) and \(q_{i}^{n+1}\),
equations \eqref{equation:tech_note/Hydrology/CLM50_Tech_Note_Hydrology:7.108} and \eqref{equation:tech_note/Hydrology/CLM50_Tech_Note_Hydrology:7.109}) against the flux at the start of the
time step (\(q_{i-1}^{n}\) and \(q_{i}^{n}\)). Since the tridiagonal
solution already provides an estimate of \(\Delta \theta_{liq,i}\), it is
convenient to compute the error for each of the \(i\) layers from equation
\eqref{equation:tech_note/Hydrology/CLM50_Tech_Note_Hydrology:7.103} as
\phantomsection\label{\detokenize{tech_note/Hydrology/CLM50_Tech_Note_Hydrology:equation-7.152}}\begin{equation}\label{equation:tech_note/Hydrology/CLM50_Tech_Note_Hydrology:7.152}
\begin{split}\epsilon_{i} = \left[ \frac{\Delta \theta_{liq,\, i} \Delta z_{i}}{\Delta t_{sub}} -
\left( q_{i-1}^{n} - q_{i}^{n} + e_{i}\right) \right] \ \frac{\Delta t_{sub}}{2}\end{split}
\end{equation}
and the maximum absolute error across all layers as
\phantomsection\label{\detokenize{tech_note/Hydrology/CLM50_Tech_Note_Hydrology:equation-7.153}}\begin{equation}\label{equation:tech_note/Hydrology/CLM50_Tech_Note_Hydrology:7.153}
\begin{split}\begin{array}{lr}
\epsilon_{crit} = {\rm max} \left( \left| \epsilon_{i} \right| \right) & \qquad 1 \le i \le nlevsoi
\end{array} \ .\end{split}
\end{equation}
The adaptive step size selection is based on specified upper and lower error
tolerances, \(\tau_{U}\) and \(\tau_{L}\). The solution is accepted if
\(\epsilon_{crit} \le \tau_{U}\) and the procedure repeats until the adaptive
sub-stepping  spans the full model time step (the sub-steps are doubled if
\(\epsilon_{crit} \le \tau_{L}\), i.e., if the solution is very  accurate).
Conversely, the solution is rejected if \(\epsilon_{crit} > \tau_{U}\).  In
this case the length of the sub-steps is halved and a new solution is obtained.
The halving of substeps continues until either \(\epsilon_{crit} \le \tau_{U}\)
or the specified minimum time step length is reached.

Upon solution of the tridiagonal equation set, the liquid water contents are updated
as follows
\phantomsection\label{\detokenize{tech_note/Hydrology/CLM50_Tech_Note_Hydrology:equation-7.164}}\begin{equation}\label{equation:tech_note/Hydrology/CLM50_Tech_Note_Hydrology:7.164}
\begin{split}w_{liq,\, i}^{n+1} =w_{liq,\, i}^{n} +\Delta \theta_{liq,\, i} \Delta z_{i} \qquad i=1,\ldots ,N_{levsoi} .\end{split}
\end{equation}
The volumetric water content is
\phantomsection\label{\detokenize{tech_note/Hydrology/CLM50_Tech_Note_Hydrology:equation-7.165}}\begin{equation}\label{equation:tech_note/Hydrology/CLM50_Tech_Note_Hydrology:7.165}
\begin{split}\theta_{i} =\frac{w_{liq,\, i} }{\Delta z_{i} \rho _{liq} } +\frac{w_{ice,\, i} }{\Delta z_{i} \rho _{ice} } .\end{split}
\end{equation}

\subsection{Frozen Soils and Perched Water Table}
\label{\detokenize{tech_note/Hydrology/CLM50_Tech_Note_Hydrology:frozen-soils-and-perched-water-table}}\label{\detokenize{tech_note/Hydrology/CLM50_Tech_Note_Hydrology:id9}}
When soils freeze, the power-law form of the ice impedance factor
(section \hyperref[\detokenize{tech_note/Hydrology/CLM50_Tech_Note_Hydrology:hydraulic-properties}]{\ref{\detokenize{tech_note/Hydrology/CLM50_Tech_Note_Hydrology:hydraulic-properties}}}) can greatly decrease the hydraulic
conductivity of the soil, leading to nearly impermeable soil layers. When unfrozen
soil layers are present above relatively ice-rich frozen layers, the
possibility exists for perched saturated zones. Lateral drainage from
perched saturated regions is parameterized as a function of the
thickness of the saturated zone
\phantomsection\label{\detokenize{tech_note/Hydrology/CLM50_Tech_Note_Hydrology:equation-7.166}}\begin{equation}\label{equation:tech_note/Hydrology/CLM50_Tech_Note_Hydrology:7.166}
\begin{split}q_{drai,perch} =k_{drai,\, perch} \left(z_{frost} -z_{\nabla ,perch} \right)\end{split}
\end{equation}
where \(k_{drai,\, perch}\)  depends on topographic slope and soil
hydraulic conductivity,
\phantomsection\label{\detokenize{tech_note/Hydrology/CLM50_Tech_Note_Hydrology:equation-7.167}}\begin{equation}\label{equation:tech_note/Hydrology/CLM50_Tech_Note_Hydrology:7.167}
\begin{split}k_{drai,\, perch} =10^{-5} \sin (\beta )\left(\frac{\sum _{i=N_{perch} }^{i=N_{frost} }\Theta_{ice,i} k_{sat} \left[z_{i} \right]\Delta z_{i}  }{\sum _{i=N_{perch} }^{i=N_{frost} }\Delta z_{i}  } \right)\end{split}
\end{equation}
where \(\Theta_{ice}\)  is an ice impedance factor, \(\beta\)
is the mean grid cell topographic slope in
radians, \(z_{frost}\) is the depth to the frost table, and
\(z_{\nabla ,perch}\)  is the depth to the perched saturated zone.
The frost table \(z_{frost}\)  is defined as the shallowest frozen
layer having an unfrozen layer above it, while the perched water table
\(z_{\nabla ,perch}\)  is defined as the depth at which the
volumetric water content drops below a specified threshold. The default
threshold is set to 0.9. Drainage from the perched saturated zone
\(q_{drai,perch}\)  is removed from layers \(N_{perch}\)
through \(N_{frost}\) , which are the layers containing
\(z_{\nabla ,perch}\)  and, \(z_{frost}\) respectively.


\subsection{Lateral Sub-surface Runoff}
\label{\detokenize{tech_note/Hydrology/CLM50_Tech_Note_Hydrology:id10}}\label{\detokenize{tech_note/Hydrology/CLM50_Tech_Note_Hydrology:lateral-sub-surface-runoff}}
Lateral sub-surface runoff occurs when saturated soil moisture conditions
exist within the soil column.  Sub-surface runoff is
\phantomsection\label{\detokenize{tech_note/Hydrology/CLM50_Tech_Note_Hydrology:equation-7.168}}\begin{equation}\label{equation:tech_note/Hydrology/CLM50_Tech_Note_Hydrology:7.168}
\begin{split}q_{drai} = \Theta_{ice} K_{baseflow} tan \left( \beta \right)
\Delta z_{sat}^{N_{baseflow}} \ ,\end{split}
\end{equation}
where \(K_{baseflow}\) is a calibration parameter, \(\beta\) is the
topographic slope, the exponent \(N_{baseflow}\) = 1, and \(\Delta z_{sat}\)
is the thickness of the saturated portion of the soil column.

The saturated thickness is
\phantomsection\label{\detokenize{tech_note/Hydrology/CLM50_Tech_Note_Hydrology:equation-7.1681}}\begin{equation}\label{equation:tech_note/Hydrology/CLM50_Tech_Note_Hydrology:7.1681}
\begin{split}\Delta z_{sat} = z_{bedrock} - z_{\nabla},\end{split}
\end{equation}
where the water table \(z_{\nabla}\) is determined by finding the
irst soil layer above the bedrock depth (section \hyperref[\detokenize{tech_note/Ecosystem/CLM50_Tech_Note_Ecosystem:depth-to-bedrock}]{\ref{\detokenize{tech_note/Ecosystem/CLM50_Tech_Note_Ecosystem:depth-to-bedrock}}})
in which the volumetric water content drops below a specified threshold.
The default threshold is set to 0.9.

The specific yield, \(S_{y}\) , which depends on the soil
properties and the water table location, is derived by taking the
difference between two equilibrium soil moisture profiles whose water
tables differ by an infinitesimal amount
\phantomsection\label{\detokenize{tech_note/Hydrology/CLM50_Tech_Note_Hydrology:equation-7.174}}\begin{equation}\label{equation:tech_note/Hydrology/CLM50_Tech_Note_Hydrology:7.174}
\begin{split}S_{y} =\theta_{sat} \left(1-\left(1+\frac{z_{\nabla } }{\Psi _{sat} } \right)^{\frac{-1}{B} } \right)\end{split}
\end{equation}
where B is the Clapp-Hornberger exponent. Because \(S_{y}\)  is a
function of the soil properties, it results in water table dynamics that
are consistent with the soil water fluxes described in section \hyperref[\detokenize{tech_note/Hydrology/CLM50_Tech_Note_Hydrology:soil-water}]{\ref{\detokenize{tech_note/Hydrology/CLM50_Tech_Note_Hydrology:soil-water}}}.

After the above calculations, two numerical adjustments are implemented
to keep the liquid water content of each soil layer
(\(w_{liq,\, i}\) ) within physical constraints of
\(w_{liq}^{\min } \le w_{liq,\, i} \le \left(\theta_{sat,\, i} -\theta_{ice,\, i} \right)\Delta z_{i}\)
where \(w_{liq}^{\min } =0.01\) (mm). First, beginning with the
bottom soil layer \(i=N_{levsoi}\) , any excess liquid water in each
soil layer
(\(w_{liq,\, i}^{excess} =w_{liq,\, i} -\left(\theta_{sat,\, i} -\theta_{ice,\, i} \right)\Delta z_{i} \ge 0\))
is successively added to the layer above. Any excess liquid water that
remains after saturating the entire soil column (plus a maximum surface
ponding depth \(w_{liq}^{pond} =10\) kg m$^{\text{-2}}$), is
added to drainage \(q_{drai}\) . Second, to prevent negative
\(w_{liq,\, i}\) , each layer is successively brought up to
\(w_{liq,\, i} =w_{liq}^{\min }\)  by taking the required amount of
water from the layer below. If this results in
\(w_{liq,\, N_{levsoi} } <w_{liq}^{\min }\) , then the layers above
are searched in succession for the required amount of water
(\(w_{liq}^{\min } -w_{liq,\, N_{levsoi} }\) ) and removed from
those layers subject to the constraint
\(w_{liq,\, i} \ge w_{liq}^{\min }\) . If sufficient water is not
found, then the water is removed from \(W_{t}\)  and
\(q_{drai}\) .

The soil surface layer liquid water and ice contents are then updated
for dew \(q_{sdew}\) , frost \(q_{frost}\) , or sublimation \(q_{subl}\)
(section \hyperref[\detokenize{tech_note/Fluxes/CLM50_Tech_Note_Fluxes:update-of-ground-sensible-and-latent-heat-fluxes}]{\ref{\detokenize{tech_note/Fluxes/CLM50_Tech_Note_Fluxes:update-of-ground-sensible-and-latent-heat-fluxes}}}) as
\phantomsection\label{\detokenize{tech_note/Hydrology/CLM50_Tech_Note_Hydrology:equation-7.175}}\begin{equation}\label{equation:tech_note/Hydrology/CLM50_Tech_Note_Hydrology:7.175}
\begin{split}w_{liq,\, 1}^{n+1} =w_{liq,\, 1}^{n} +q_{sdew} \Delta t\end{split}
\end{equation}\phantomsection\label{\detokenize{tech_note/Hydrology/CLM50_Tech_Note_Hydrology:equation-7.176}}\begin{equation}\label{equation:tech_note/Hydrology/CLM50_Tech_Note_Hydrology:7.176}
\begin{split}w_{ice,\, 1}^{n+1} =w_{ice,\, 1}^{n} +q_{frost} \Delta t\end{split}
\end{equation}\phantomsection\label{\detokenize{tech_note/Hydrology/CLM50_Tech_Note_Hydrology:equation-7.177}}\begin{equation}\label{equation:tech_note/Hydrology/CLM50_Tech_Note_Hydrology:7.177}
\begin{split}w_{ice,\, 1}^{n+1} =w_{ice,\, 1}^{n} -q_{subl} \Delta t.\end{split}
\end{equation}
Sublimation of ice is limited to the amount of ice available.


\subsection{Runoff from glaciers and snow-capped surfaces}
\label{\detokenize{tech_note/Hydrology/CLM50_Tech_Note_Hydrology:runoff-from-glaciers-and-snow-capped-surfaces}}\label{\detokenize{tech_note/Hydrology/CLM50_Tech_Note_Hydrology:id11}}
All surfaces are constrained to have a snow water equivalent
\(W_{sno} \le 1000\) kg m$^{\text{-2}}$. For snow-capped
surfaces, the solid and liquid precipitation reaching the snow surface
and dew in solid or liquid form, is separated into solid
\(q_{snwcp,ice}\) and liquid \(q_{snwcp,liq}\)  runoff terms
\phantomsection\label{\detokenize{tech_note/Hydrology/CLM50_Tech_Note_Hydrology:equation-7.178}}\begin{equation}\label{equation:tech_note/Hydrology/CLM50_Tech_Note_Hydrology:7.178}
\begin{split}q_{snwcp,ice} =q_{grnd,ice} +q_{frost}\end{split}
\end{equation}\phantomsection\label{\detokenize{tech_note/Hydrology/CLM50_Tech_Note_Hydrology:equation-7.179}}\begin{equation}\label{equation:tech_note/Hydrology/CLM50_Tech_Note_Hydrology:7.179}
\begin{split}q_{snwcp,liq} =q_{grnd,liq} +q_{dew}\end{split}
\end{equation}
and snow pack properties are unchanged. The \(q_{snwcp,ice}\)
runoff is sent to the River Transport Model (RTM) (Chapter 11) where it
is routed to the ocean as an ice stream and, if applicable, the ice is
melted there.

For snow-capped surfaces other than glaciers and lakes the
\(q_{snwcp,liq}\)  runoff is assigned to the glaciers and lakes
runoff term \(q_{rgwl}\)  (e.g. \(q_{rgwl} =q_{snwcp,liq}\) ).
For glacier surfaces the runoff term \(q_{rgwl}\)  is calculated
from the residual of the water balance
\phantomsection\label{\detokenize{tech_note/Hydrology/CLM50_Tech_Note_Hydrology:equation-7.180}}\begin{equation}\label{equation:tech_note/Hydrology/CLM50_Tech_Note_Hydrology:7.180}
\begin{split}q_{rgwl} =q_{grnd,ice} +q_{grnd,liq} -E_{g} -E_{v} -\frac{\left(W_{b}^{n+1} -W_{b}^{n} \right)}{\Delta t} -q_{snwcp,ice}\end{split}
\end{equation}
where \(W_{b}^{n}\)  and \(W_{b}^{n+1}\)  are the water balances
at the beginning and ending of the time step defined as
\phantomsection\label{\detokenize{tech_note/Hydrology/CLM50_Tech_Note_Hydrology:equation-7.181}}\begin{equation}\label{equation:tech_note/Hydrology/CLM50_Tech_Note_Hydrology:7.181}
\begin{split}W_{b} =W_{can} +W_{sno} +\sum _{i=1}^{N}\left(w_{ice,i} +w_{liq,i} \right) .\end{split}
\end{equation}
Currently, glaciers are non-vegetated and \(E_{v} =W_{can} =0\).
The contribution of lake runoff to \(q_{rgwl}\)  is described in
section \hyperref[\detokenize{tech_note/Lake/CLM50_Tech_Note_Lake:precipitation-evaporation-and-runoff-lake}]{\ref{\detokenize{tech_note/Lake/CLM50_Tech_Note_Lake:precipitation-evaporation-and-runoff-lake}}}. The runoff
term \(q_{rgwl}\)  may be negative for glaciers and lakes, which reduces
the total amount of runoff available to the river routing model (Chapter \hyperref[\detokenize{tech_note/MOSART/CLM50_Tech_Note_MOSART:rst-river-transport-model-rtm}]{\ref{\detokenize{tech_note/MOSART/CLM50_Tech_Note_MOSART:rst-river-transport-model-rtm}}}).

\textgreater{}


\section{Snow Hydrology}
\label{\detokenize{tech_note/Snow_Hydrology/CLM50_Tech_Note_Snow_Hydrology:rst-snow-hydrology}}\label{\detokenize{tech_note/Snow_Hydrology/CLM50_Tech_Note_Snow_Hydrology::doc}}\label{\detokenize{tech_note/Snow_Hydrology/CLM50_Tech_Note_Snow_Hydrology:snow-hydrology}}
The parameterizations for snow are based primarily on
{\hyperref[\detokenize{tech_note/References/CLM50_Tech_Note_References:anderson1976}]{\sphinxcrossref{\DUrole{std,std-ref}{Anderson (1976)}}}}, {\hyperref[\detokenize{tech_note/References/CLM50_Tech_Note_References:jordan1991}]{\sphinxcrossref{\DUrole{std,std-ref}{Jordan (1991)}}}},
and {\hyperref[\detokenize{tech_note/References/CLM50_Tech_Note_References:daizeng1997}]{\sphinxcrossref{\DUrole{std,std-ref}{Dai and Zeng (1997)}}}}. The snowpack
can have up to five layers. These layers are indexed in the Fortran code
as \(i=-4,-3,-2,-1,0\) where layer \(i=0\) is the snow layer
next to the top soil layer and layer \(i=-4\) is the top layer of a
five-layer snow pack. Since the number of snow layers varies according
to the snow depth, we use the notation \(snl+1\) to describe the top
layer of snow for the variable layer snow pack, where \(snl\) is the
negative of the number of snow layers. Refer to \hyperref[\detokenize{tech_note/Snow_Hydrology/CLM50_Tech_Note_Snow_Hydrology:figure-three-layer-snow-pack}]{Figure \ref{\detokenize{tech_note/Snow_Hydrology/CLM50_Tech_Note_Snow_Hydrology:figure-three-layer-snow-pack}}} for an example of the snow layer structure for a three layer
snow pack.

\begin{figure}[htbp]
\centering
\capstart

\noindent\sphinxincludegraphics{{image16}.png}
\caption{Example of three layer snow pack (\(snl=-3\)).}\label{\detokenize{tech_note/Snow_Hydrology/CLM50_Tech_Note_Snow_Hydrology:figure-three-layer-snow-pack}}\label{\detokenize{tech_note/Snow_Hydrology/CLM50_Tech_Note_Snow_Hydrology:id10}}\end{figure}

Shown are three snow layers, \(i=-2\), \(i=-1\), and
\(i=0\). The layer node depth is \(z\), the layer interface is
\(z_{h}\) , and the layer thickness is \(\Delta z\).

The state variables for snow are the mass of water \(w_{liq,i}\)
(kg m$^{\text{-2}}$), mass of ice \(w_{ice,i}\)  (kg m$^{\text{-2}}$), layer
thickness \(\Delta z_{i}\)  (m), and temperature \(T_{i}\)
(Chapter \hyperref[\detokenize{tech_note/Soil_Snow_Temperatures/CLM50_Tech_Note_Soil_Snow_Temperatures:rst-soil-and-snow-temperatures}]{\ref{\detokenize{tech_note/Soil_Snow_Temperatures/CLM50_Tech_Note_Soil_Snow_Temperatures:rst-soil-and-snow-temperatures}}}). The water vapor phase is
neglected. Snow can also exist in the model without being represented by
explicit snow layers. This occurs when the snowpack is less than a
specified minimum snow depth (\(z_{sno} < 0.01\) m). In this case,
the state variable is the mass of snow \(W_{sno}\)  (kg m$^{\text{-2}}$).

Section \hyperref[\detokenize{tech_note/Snow_Hydrology/CLM50_Tech_Note_Snow_Hydrology:snow-covered-area-fraction}]{\ref{\detokenize{tech_note/Snow_Hydrology/CLM50_Tech_Note_Snow_Hydrology:snow-covered-area-fraction}}} describes the calculation
of fractional snow covered area, which is used in the surface albedo
calculation (Chapter \hyperref[\detokenize{tech_note/Surface_Albedos/CLM50_Tech_Note_Surface_Albedos:rst-surface-albedos}]{\ref{\detokenize{tech_note/Surface_Albedos/CLM50_Tech_Note_Surface_Albedos:rst-surface-albedos}}}) and thesurface flux
calculations (Chapter \hyperref[\detokenize{tech_note/Fluxes/CLM50_Tech_Note_Fluxes:rst-momentum-sensible-heat-and-latent-heat-fluxes}]{\ref{\detokenize{tech_note/Fluxes/CLM50_Tech_Note_Fluxes:rst-momentum-sensible-heat-and-latent-heat-fluxes}}}). The following two sections (\hyperref[\detokenize{tech_note/Snow_Hydrology/CLM50_Tech_Note_Snow_Hydrology:ice-content}]{\ref{\detokenize{tech_note/Snow_Hydrology/CLM50_Tech_Note_Snow_Hydrology:ice-content}}} and
\hyperref[\detokenize{tech_note/Snow_Hydrology/CLM50_Tech_Note_Snow_Hydrology:water-content}]{\ref{\detokenize{tech_note/Snow_Hydrology/CLM50_Tech_Note_Snow_Hydrology:water-content}}}) describe the ice and water content of the snow
pack assuming that at least one snow layer exists. Section \hyperref[\detokenize{tech_note/Snow_Hydrology/CLM50_Tech_Note_Snow_Hydrology:black-and-organic-carbon-and-mineral-dust-within-snow}]{\ref{\detokenize{tech_note/Snow_Hydrology/CLM50_Tech_Note_Snow_Hydrology:black-and-organic-carbon-and-mineral-dust-within-snow}}} describes how black and
organic carbon and mineral dust particles are represented within snow,
including meltwater flushing. See Section \hyperref[\detokenize{tech_note/Snow_Hydrology/CLM50_Tech_Note_Snow_Hydrology:initialization-of-snow-layer}]{\ref{\detokenize{tech_note/Snow_Hydrology/CLM50_Tech_Note_Snow_Hydrology:initialization-of-snow-layer}}} for a description of how a snow layer is initialized.


\subsection{Snow Covered Area Fraction}
\label{\detokenize{tech_note/Snow_Hydrology/CLM50_Tech_Note_Snow_Hydrology:snow-covered-area-fraction}}\label{\detokenize{tech_note/Snow_Hydrology/CLM50_Tech_Note_Snow_Hydrology:id1}}
The fraction of the ground covered by snow, \(f_{sno}\) , is based
on the method of {\hyperref[\detokenize{tech_note/References/CLM50_Tech_Note_References:swensonlawrence2012}]{\sphinxcrossref{\DUrole{std,std-ref}{Swenson and Lawrence (2012)}}}}.
Because the processes
governing snowfall and snowmelt differ, changes in \(f_{sno}\)  are
calculated separately for accumulation and depletion. When snowfall
occurs, \(f_{sno}\)  is updated as
\phantomsection\label{\detokenize{tech_note/Snow_Hydrology/CLM50_Tech_Note_Snow_Hydrology:equation-8.14}}\begin{equation}\label{equation:tech_note/Snow_Hydrology/CLM50_Tech_Note_Snow_Hydrology:8.14}
\begin{split}f^{n+1} _{sno} =1-\left(\left(1-\tanh (k_{accum} q_{sno} \Delta t)\right)\left(1-f^{n} _{sno} \right)\right)\end{split}
\end{equation}
where \(k_{accum}\)  is a constant whose default value is 0.1,
\(q_{sno} \Delta t\) is the amount of new snow,
\(f^{n+1} _{sno}\)  is the updated snow covered fraction (SCF), and
\(f^{n} _{sno}\)  is the SCF from the previous time step.

When snow melt occurs, \(f_{sno}\)  is calculated from the depletion
curve
\phantomsection\label{\detokenize{tech_note/Snow_Hydrology/CLM50_Tech_Note_Snow_Hydrology:equation-8.15}}\begin{equation}\label{equation:tech_note/Snow_Hydrology/CLM50_Tech_Note_Snow_Hydrology:8.15}
\begin{split}f_{sno} =1-\left(\frac{\cos ^{-1} \left(2R_{sno} -1\right)}{\pi } \right)^{N_{melt} }\end{split}
\end{equation}
where \(R_{sno}\)  is the ratio of \(W_{sno}\)  to the maximum
accumulated snow \(W_{\max }\) , and \(N_{melt}\)  is a
parameter that depends on the topographic variability within the grid
cell. Whenever \(W_{sno}\)  reaches zero, \(W_{\max }\)  is
reset to zero. The depletion curve shape parameter is defined as
\phantomsection\label{\detokenize{tech_note/Snow_Hydrology/CLM50_Tech_Note_Snow_Hydrology:equation-8.16}}\begin{equation}\label{equation:tech_note/Snow_Hydrology/CLM50_Tech_Note_Snow_Hydrology:8.16}
\begin{split}N_{melt} =\frac{200}{\min \left(10,\sigma _{topo} \right)}\end{split}
\end{equation}
The standard deviation of the elevation within a grid cell,
\(\sigma _{topo}\)  , is calculated from a high resolution DEM (a
1km DEM is used for CLM).


\subsection{Ice Content}
\label{\detokenize{tech_note/Snow_Hydrology/CLM50_Tech_Note_Snow_Hydrology:id2}}\label{\detokenize{tech_note/Snow_Hydrology/CLM50_Tech_Note_Snow_Hydrology:ice-content}}
The conservation equation for mass of ice in snow layers is
\phantomsection\label{\detokenize{tech_note/Snow_Hydrology/CLM50_Tech_Note_Snow_Hydrology:equation-8.17}}\begin{equation}\label{equation:tech_note/Snow_Hydrology/CLM50_Tech_Note_Snow_Hydrology:8.17}
\begin{split}\frac{\partial w_{ice,\, i} }{\partial t} =
\left\{\begin{array}{lr}
f_{sno} \ q_{ice,\, i-1} -\frac{\left(\Delta w_{ice,\, i} \right)_{p} }{\Delta t} & \qquad i=snl+1 \\
-\frac{\left(\Delta w_{ice,\, i} \right)_{p} }{\Delta t} & \qquad i=snl+2,\ldots ,0
\end{array}\right\}\end{split}
\end{equation}
where \(q_{ice,\, i-1}\)  is the rate of ice accumulation from
precipitation or frost or the rate of ice loss from sublimation (kg
m$^{\text{-2}}$ s$^{\text{-1}}$) in the top layer and
\({\left(\Delta w_{ice,\, i} \right)_{p} \mathord{\left/ {\vphantom {\left(\Delta w_{ice,\, i} \right)_{p}  \Delta t}} \right. \kern-\nulldelimiterspace} \Delta t}\)
is the change in ice due to phase change (melting rate) (section \hyperref[\detokenize{tech_note/Soil_Snow_Temperatures/CLM50_Tech_Note_Soil_Snow_Temperatures:phase-change}]{\ref{\detokenize{tech_note/Soil_Snow_Temperatures/CLM50_Tech_Note_Soil_Snow_Temperatures:phase-change}}}).
The term \(q_{ice,\, i-1}\)  is computed in two steps as
\phantomsection\label{\detokenize{tech_note/Snow_Hydrology/CLM50_Tech_Note_Snow_Hydrology:equation-8.18}}\begin{equation}\label{equation:tech_note/Snow_Hydrology/CLM50_Tech_Note_Snow_Hydrology:8.18}
\begin{split}q_{ice,\, i-1} =q_{grnd,\, ice} +\left(q_{frost} -q_{subl} \right)\end{split}
\end{equation}
where \(q_{grnd,\, ice}\)  is the rate of solid precipitation
reaching the ground (section \hyperref[\detokenize{tech_note/Hydrology/CLM50_Tech_Note_Hydrology:canopy-water}]{\ref{\detokenize{tech_note/Hydrology/CLM50_Tech_Note_Hydrology:canopy-water}}}) and \(q_{frost}\)  and
\(q_{subl}\)  are gains due to frost and losses due to sublimation,
respectively (sectio \hyperref[\detokenize{tech_note/Fluxes/CLM50_Tech_Note_Fluxes:update-of-ground-sensible-and-latent-heat-fluxes}]{\ref{\detokenize{tech_note/Fluxes/CLM50_Tech_Note_Fluxes:update-of-ground-sensible-and-latent-heat-fluxes}}}). In the first step, immediately after
\(q_{grnd,\, ice}\)  has been determined after accounting for
interception (section \hyperref[\detokenize{tech_note/Hydrology/CLM50_Tech_Note_Hydrology:canopy-water}]{\ref{\detokenize{tech_note/Hydrology/CLM50_Tech_Note_Hydrology:canopy-water}}}), a new snow depth \(z_{sno}\)  (m) is
calculated from
\phantomsection\label{\detokenize{tech_note/Snow_Hydrology/CLM50_Tech_Note_Snow_Hydrology:equation-8.19}}\begin{equation}\label{equation:tech_note/Snow_Hydrology/CLM50_Tech_Note_Snow_Hydrology:8.19}
\begin{split}z_{sno}^{n+1} =z_{sno}^{n} +\Delta z_{sno}\end{split}
\end{equation}
where
\phantomsection\label{\detokenize{tech_note/Snow_Hydrology/CLM50_Tech_Note_Snow_Hydrology:equation-8.20}}\begin{equation}\label{equation:tech_note/Snow_Hydrology/CLM50_Tech_Note_Snow_Hydrology:8.20}
\begin{split}\Delta z_{sno} =\frac{q_{grnd,\, ice} \Delta t}{f_{sno} \rho _{sno} }\end{split}
\end{equation}
and \(\rho _{sno}\)  is the bulk density of newly fallen snow (kg
m$^{\text{-3}}$) ({\hyperref[\detokenize{tech_note/References/CLM50_Tech_Note_References:vankampenhoutetal2017}]{\sphinxcrossref{\DUrole{std,std-ref}{van Kampenhout et al. (2017)}}}},
{\hyperref[\detokenize{tech_note/References/CLM50_Tech_Note_References:anderson1976}]{\sphinxcrossref{\DUrole{std,std-ref}{Anderson (1976)}}}})
\phantomsection\label{\detokenize{tech_note/Snow_Hydrology/CLM50_Tech_Note_Snow_Hydrology:equation-8.21}}\begin{equation}\label{equation:tech_note/Snow_Hydrology/CLM50_Tech_Note_Snow_Hydrology:8.21}
\begin{split}\rho_{sno} =
\left\{\begin{array}{lr}
50 + 1.7 \left(17\right)^{1.5} & \qquad T_{atm} >T_{f} +2 \ \\
50+1.7 \left(T_{atm} -T_{f} + 15\right)^{1.5} & \qquad T_{f} - 15 < T_{atm} \le T_{f} + 2 \ \\
-3.833 \ \left( T_{atm} -T_{f} \right) - 0.0333 \ \left( T_{atm} -T_{f} \right)^{2}
&\qquad T_{atm} \le T_{f} - 15
\end{array}\right\}\end{split}
\end{equation}
where \(T_{atm}\)  is the atmospheric temperature (K), and \(T_{f}\) is
the freezing temperature of water (K) (\hyperref[\detokenize{tech_note/Ecosystem/CLM50_Tech_Note_Ecosystem:table-physical-constants}]{Table \ref{\detokenize{tech_note/Ecosystem/CLM50_Tech_Note_Ecosystem:table-physical-constants}}}). When
wind speed \(W_{atm}\) is greater than 0.1 m $_{\text{-1}}$, snow density
increases due to wind-driven compaction according to
{\hyperref[\detokenize{tech_note/References/CLM50_Tech_Note_References:vankampenhoutetal2017}]{\sphinxcrossref{\DUrole{std,std-ref}{(van Kampenhout et al. 2017)}}}}
\phantomsection\label{\detokenize{tech_note/Snow_Hydrology/CLM50_Tech_Note_Snow_Hydrology:equation-8.21b}}\begin{equation}\label{equation:tech_note/Snow_Hydrology/CLM50_Tech_Note_Snow_Hydrology:8.21b}
\begin{split}\Delta \rho_{sno} = 266.861 \left(\frac{1 + tanh(\frac{W_{atm}}{5})}{2}\right)^{8.8}\end{split}
\end{equation}
where \(\Delta \rho_{sno}\) (kg m$^{\text{-3}}$) is the increase in snow
density relative to \eqref{equation:tech_note/Snow_Hydrology/CLM50_Tech_Note_Snow_Hydrology:8.21}.

The mass of snow \(W_{sno}\)  is
\phantomsection\label{\detokenize{tech_note/Snow_Hydrology/CLM50_Tech_Note_Snow_Hydrology:equation-8.22}}\begin{equation}\label{equation:tech_note/Snow_Hydrology/CLM50_Tech_Note_Snow_Hydrology:8.22}
\begin{split}W_{sno}^{n+1} =W_{sno}^{n} +q_{grnd,\, ice} \Delta t.\end{split}
\end{equation}
The ice content of the top layer and the layer thickness are updated as
\phantomsection\label{\detokenize{tech_note/Snow_Hydrology/CLM50_Tech_Note_Snow_Hydrology:equation-8.23}}\begin{equation}\label{equation:tech_note/Snow_Hydrology/CLM50_Tech_Note_Snow_Hydrology:8.23}
\begin{split}w_{ice,\, snl+1}^{n+1} =w_{ice,\, snl+1}^{n} +q_{grnd,\, ice} \Delta t\end{split}
\end{equation}\phantomsection\label{\detokenize{tech_note/Snow_Hydrology/CLM50_Tech_Note_Snow_Hydrology:equation-8.24}}\begin{equation}\label{equation:tech_note/Snow_Hydrology/CLM50_Tech_Note_Snow_Hydrology:8.24}
\begin{split}\Delta z_{snl+1}^{n+1} =\Delta z_{snl+1}^{n} +\Delta z_{sno} .\end{split}
\end{equation}
Since wetlands are modeled as columns of water (no soil), snow is not
allowed to accumulate if the surface temperature is above freezing
(\(T_{g} >T_{f}\) ). In this case, the incoming solid precipitation
is assigned to the runoff term \(q_{rgwl}\)  (section
\hyperref[\detokenize{tech_note/Hydrology/CLM50_Tech_Note_Hydrology:runoff-from-glaciers-and-snow-capped-surfaces}]{\ref{\detokenize{tech_note/Hydrology/CLM50_Tech_Note_Hydrology:runoff-from-glaciers-and-snow-capped-surfaces}}}).

In the second step, after surface fluxes and snow/soil temperatures have
been determined (Chapters \hyperref[\detokenize{tech_note/Fluxes/CLM50_Tech_Note_Fluxes:rst-momentum-sensible-heat-and-latent-heat-fluxes}]{\ref{\detokenize{tech_note/Fluxes/CLM50_Tech_Note_Fluxes:rst-momentum-sensible-heat-and-latent-heat-fluxes}}} and \hyperref[\detokenize{tech_note/Soil_Snow_Temperatures/CLM50_Tech_Note_Soil_Snow_Temperatures:rst-soil-and-snow-temperatures}]{\ref{\detokenize{tech_note/Soil_Snow_Temperatures/CLM50_Tech_Note_Soil_Snow_Temperatures:rst-soil-and-snow-temperatures}}}),
\(w_{ice,\, snl+1}\)  is updated for frost or sublimation as
\phantomsection\label{\detokenize{tech_note/Snow_Hydrology/CLM50_Tech_Note_Snow_Hydrology:equation-8.25}}\begin{equation}\label{equation:tech_note/Snow_Hydrology/CLM50_Tech_Note_Snow_Hydrology:8.25}
\begin{split}w_{ice,\, snl+1}^{n+1} =w_{ice,\, snl+1}^{n} +f_{sno} \left(q_{frost} -q_{subl} \right)\Delta t.\end{split}
\end{equation}
If \(w_{ice,\, snl+1}^{n+1} <0\) upon solution of equation , the ice
content is reset to zero and the liquid water content
\(w_{liq,\, snl+1}\)  is reduced by the amount required to bring
\(w_{ice,\, snl+1}^{n+1}\)  up to zero.

The snow water equivalent \(W_{sno}\)  is capped to not exceed 1000
kg m$^{\text{-2}}$. If the addition of \(q_{frost}\)  were to
result in \(W_{sno} >1000\) kg m$^{\text{-2}}$, the frost term
\(q_{frost}\)  is instead added to the ice runoff term
\(q_{snwcp,\, ice}\)  (section \hyperref[\detokenize{tech_note/Hydrology/CLM50_Tech_Note_Hydrology:runoff-from-glaciers-and-snow-capped-surfaces}]{\ref{\detokenize{tech_note/Hydrology/CLM50_Tech_Note_Hydrology:runoff-from-glaciers-and-snow-capped-surfaces}}}).


\subsection{Water Content}
\label{\detokenize{tech_note/Snow_Hydrology/CLM50_Tech_Note_Snow_Hydrology:water-content}}\label{\detokenize{tech_note/Snow_Hydrology/CLM50_Tech_Note_Snow_Hydrology:id3}}
The conservation equation for mass of water in snow layers is
\phantomsection\label{\detokenize{tech_note/Snow_Hydrology/CLM50_Tech_Note_Snow_Hydrology:equation-8.26}}\begin{equation}\label{equation:tech_note/Snow_Hydrology/CLM50_Tech_Note_Snow_Hydrology:8.26}
\begin{split}\frac{\partial w_{liq,\, i} }{\partial t} =\left(q_{liq,\, i-1} -q_{liq,\, i} \right)+\frac{\left(\Delta w_{liq,\, i} \right)_{p} }{\Delta t}\end{split}
\end{equation}
where \(q_{liq,\, i-1}\)  is the flow of liquid water into layer
\(i\) from the layer above, \(q_{liq,\, i}\)  is the flow of
water out of layer \(i\) to the layer below,
\({\left(\Delta w_{liq,\, i} \right)_{p} \mathord{\left/ {\vphantom {\left(\Delta w_{liq,\, i} \right)_{p}  \Delta t}} \right. \kern-\nulldelimiterspace} \Delta t}\)
is the change in liquid water due to phase change (melting rate)
(section \hyperref[\detokenize{tech_note/Soil_Snow_Temperatures/CLM50_Tech_Note_Soil_Snow_Temperatures:phase-change}]{\ref{\detokenize{tech_note/Soil_Snow_Temperatures/CLM50_Tech_Note_Soil_Snow_Temperatures:phase-change}}}). For the top snow layer only,
\phantomsection\label{\detokenize{tech_note/Snow_Hydrology/CLM50_Tech_Note_Snow_Hydrology:equation-8.27}}\begin{equation}\label{equation:tech_note/Snow_Hydrology/CLM50_Tech_Note_Snow_Hydrology:8.27}
\begin{split}q_{liq,\, i-1} =f_{sno} \left(q_{grnd,\, liq} +\left(q_{sdew} -q_{seva} \right)\right)\end{split}
\end{equation}
where \(q_{grnd,\, liq}\)  is the rate of liquid precipitation
reaching the snow (section \hyperref[\detokenize{tech_note/Hydrology/CLM50_Tech_Note_Hydrology:canopy-water}]{\ref{\detokenize{tech_note/Hydrology/CLM50_Tech_Note_Hydrology:canopy-water}}}), \(q_{seva}\) is the
evaporation of liquid water and \(q_{sdew}\)  is the liquid dew (section
\hyperref[\detokenize{tech_note/Fluxes/CLM50_Tech_Note_Fluxes:update-of-ground-sensible-and-latent-heat-fluxes}]{\ref{\detokenize{tech_note/Fluxes/CLM50_Tech_Note_Fluxes:update-of-ground-sensible-and-latent-heat-fluxes}}}).  After surface
fluxes and snow/soil temperatures have been determined
(Chapters \hyperref[\detokenize{tech_note/Fluxes/CLM50_Tech_Note_Fluxes:rst-momentum-sensible-heat-and-latent-heat-fluxes}]{\ref{\detokenize{tech_note/Fluxes/CLM50_Tech_Note_Fluxes:rst-momentum-sensible-heat-and-latent-heat-fluxes}}} and
\hyperref[\detokenize{tech_note/Soil_Snow_Temperatures/CLM50_Tech_Note_Soil_Snow_Temperatures:rst-soil-and-snow-temperatures}]{\ref{\detokenize{tech_note/Soil_Snow_Temperatures/CLM50_Tech_Note_Soil_Snow_Temperatures:rst-soil-and-snow-temperatures}}}), \(w_{liq,\, snl+1}\)  is
updated for the liquid precipitation reaching the ground and dew or
evaporation as
\phantomsection\label{\detokenize{tech_note/Snow_Hydrology/CLM50_Tech_Note_Snow_Hydrology:equation-8.28}}\begin{equation}\label{equation:tech_note/Snow_Hydrology/CLM50_Tech_Note_Snow_Hydrology:8.28}
\begin{split}w_{liq,\, snl+1}^{n+1} =w_{liq,\, snl+1}^{n} +f_{sno} \left(q_{grnd,\, liq} +q_{sdew} -q_{seva} \right)\Delta t.\end{split}
\end{equation}
When the liquid water within a snow layer exceeds the layer’s holding
capacity, the excess water is added to the underlying layer, limited by
the effective porosity (\(1-\theta _{ice}\) ) of the layer. The flow
of water is assumed to be zero (\(q_{liq,\, i} =0\)) if the
effective porosity of either of the two layers
(\(1-\theta _{ice,\, i} {\rm \; and\; }1-\theta _{ice,\, i+1}\) ) is
less than \(\theta _{imp} =0.05\), the water impermeable volumetric
water content. Thus, water flow between layers, \(q_{liq,\, i}\) ,
for layers \(i=snl+1,\ldots ,0\), is initially calculated as
\phantomsection\label{\detokenize{tech_note/Snow_Hydrology/CLM50_Tech_Note_Snow_Hydrology:equation-8.29}}\begin{equation}\label{equation:tech_note/Snow_Hydrology/CLM50_Tech_Note_Snow_Hydrology:8.29}
\begin{split}q_{liq,\, i} =\frac{\rho _{liq} \left[\theta _{liq,\, i} -S_{r} \left(1-\theta _{ice,\, i} \right)\right]f_{sno} \Delta z_{i} }{\Delta t} \ge 0\end{split}
\end{equation}
where the volumetric liquid water \(\theta _{liq,\, i}\)  and ice
\(\theta _{ice,\, i}\)  contents are
\phantomsection\label{\detokenize{tech_note/Snow_Hydrology/CLM50_Tech_Note_Snow_Hydrology:equation-8.30}}\begin{equation}\label{equation:tech_note/Snow_Hydrology/CLM50_Tech_Note_Snow_Hydrology:8.30}
\begin{split}\theta _{ice,\, i} =\frac{w_{ice,\, i} }{f_{sno} \Delta z_{i} \rho _{ice} } \le 1\end{split}
\end{equation}\phantomsection\label{\detokenize{tech_note/Snow_Hydrology/CLM50_Tech_Note_Snow_Hydrology:equation-8.31}}\begin{equation}\label{equation:tech_note/Snow_Hydrology/CLM50_Tech_Note_Snow_Hydrology:8.31}
\begin{split}\theta _{liq,\, i} =\frac{w_{liq,\, i} }{f_{sno} \Delta z_{i} \rho _{liq} } \le 1-\theta _{ice,\, i} ,\end{split}
\end{equation}
and \(S_{r} =0.033\) is the irreducible water saturation (snow
holds a certain amount of liquid water due to capillary retention after
drainage has ceased ({\hyperref[\detokenize{tech_note/References/CLM50_Tech_Note_References:anderson1976}]{\sphinxcrossref{\DUrole{std,std-ref}{Anderson (1976)}}}})). The water holding capacity of the
underlying layer limits the flow of water \(q_{liq,\, i}\)
calculated in equation , unless the underlying layer is the surface soil
layer, as
\phantomsection\label{\detokenize{tech_note/Snow_Hydrology/CLM50_Tech_Note_Snow_Hydrology:equation-8.32}}\begin{equation}\label{equation:tech_note/Snow_Hydrology/CLM50_Tech_Note_Snow_Hydrology:8.32}
\begin{split}q_{liq,\, i} \le \frac{\rho _{liq} \left[1-\theta _{ice,\, i+1} -\theta _{liq,\, i+1} \right]\Delta z_{i+1} }{\Delta t} \qquad i=snl+1,\ldots ,-1.\end{split}
\end{equation}
The liquid water content \(w_{liq,\, i}\)  is updated as
\phantomsection\label{\detokenize{tech_note/Snow_Hydrology/CLM50_Tech_Note_Snow_Hydrology:equation-8.33}}\begin{equation}\label{equation:tech_note/Snow_Hydrology/CLM50_Tech_Note_Snow_Hydrology:8.33}
\begin{split}w_{liq,\, i}^{n+1} =w_{liq,\, i}^{n} +\left(q_{i-1} -q_{i} \right)\Delta t.\end{split}
\end{equation}
Equations - are solved sequentially from top (\(i=snl+1\)) to
bottom (\(i=0\)) snow layer in each time step. The total flow of
liquid water reaching the soil surface is then \(q_{liq,\, 0}\)
which is used in the calculation of surface runoff and infiltration
(sections \hyperref[\detokenize{tech_note/Hydrology/CLM50_Tech_Note_Hydrology:surface-runoff}]{\ref{\detokenize{tech_note/Hydrology/CLM50_Tech_Note_Hydrology:surface-runoff}}} and \hyperref[\detokenize{tech_note/Hydrology/CLM50_Tech_Note_Hydrology:infiltration}]{\ref{\detokenize{tech_note/Hydrology/CLM50_Tech_Note_Hydrology:infiltration}}}).


\subsection{Black and organic carbon and mineral dust within snow}
\label{\detokenize{tech_note/Snow_Hydrology/CLM50_Tech_Note_Snow_Hydrology:id4}}\label{\detokenize{tech_note/Snow_Hydrology/CLM50_Tech_Note_Snow_Hydrology:black-and-organic-carbon-and-mineral-dust-within-snow}}
Particles within snow originate from atmospheric aerosol deposition
(\(D_{sp}\)  in Table 2.3 (kg m$^{\text{-2}}$ s$^{\text{-1}}$)
and influence snow radiative transfer (sections \hyperref[\detokenize{tech_note/Surface_Albedos/CLM50_Tech_Note_Surface_Albedos:snow-albedo}]{\ref{\detokenize{tech_note/Surface_Albedos/CLM50_Tech_Note_Surface_Albedos:snow-albedo}}},
\hyperref[\detokenize{tech_note/Surface_Albedos/CLM50_Tech_Note_Surface_Albedos:snowpack-optical-properties}]{\ref{\detokenize{tech_note/Surface_Albedos/CLM50_Tech_Note_Surface_Albedos:snowpack-optical-properties}}}, and \hyperref[\detokenize{tech_note/Surface_Albedos/CLM50_Tech_Note_Surface_Albedos:snow-aging}]{\ref{\detokenize{tech_note/Surface_Albedos/CLM50_Tech_Note_Surface_Albedos:snow-aging}}}).
Particle masses and mixing ratios are represented with a simple
mass-conserving scheme. The model maintains masses of the following
eight particle species within each snow layer: hydrophilic black carbon,
hydrophobic black carbon, hydrophilic organic carbon, hydrophobic
organic carbon, and four species of mineral dust with the following
particle sizes: 0.1-1.0, 1.0-2.5, 2.5-5.0, and 5.0-10.0 \(\mu m\).
Each of these species has unique optical properties
(\hyperref[\detokenize{tech_note/Surface_Albedos/CLM50_Tech_Note_Surface_Albedos:table-single-scatter-albedo-values-used-for-snowpack-impurities-and-ice}]{Table \ref{\detokenize{tech_note/Surface_Albedos/CLM50_Tech_Note_Surface_Albedos:table-single-scatter-albedo-values-used-for-snowpack-impurities-and-ice}}})
and meltwater removal efficiencies (\hyperref[\detokenize{tech_note/Snow_Hydrology/CLM50_Tech_Note_Snow_Hydrology:table-meltwater-scavenging}]{Table \ref{\detokenize{tech_note/Snow_Hydrology/CLM50_Tech_Note_Snow_Hydrology:table-meltwater-scavenging}}}).

The black carbon and organic carbon deposition rates described in Table
2.3 are combined into four categories as follows
\phantomsection\label{\detokenize{tech_note/Snow_Hydrology/CLM50_Tech_Note_Snow_Hydrology:equation-8.34}}\begin{equation}\label{equation:tech_note/Snow_Hydrology/CLM50_Tech_Note_Snow_Hydrology:8.34}
\begin{split}D_{bc,\, hphil} =D_{bc,\, dryhphil} +D_{bc,\, wethphil}\end{split}
\end{equation}\phantomsection\label{\detokenize{tech_note/Snow_Hydrology/CLM50_Tech_Note_Snow_Hydrology:equation-8.35}}\begin{equation}\label{equation:tech_note/Snow_Hydrology/CLM50_Tech_Note_Snow_Hydrology:8.35}
\begin{split}D_{bc,\, hphob} =D_{bc,\, dryhphob}\end{split}
\end{equation}\phantomsection\label{\detokenize{tech_note/Snow_Hydrology/CLM50_Tech_Note_Snow_Hydrology:equation-8.36}}\begin{equation}\label{equation:tech_note/Snow_Hydrology/CLM50_Tech_Note_Snow_Hydrology:8.36}
\begin{split}D_{oc,\, hphil} =D_{oc,\, dryhphil} +D_{oc,\, wethphil}\end{split}
\end{equation}\phantomsection\label{\detokenize{tech_note/Snow_Hydrology/CLM50_Tech_Note_Snow_Hydrology:equation-8.37}}\begin{equation}\label{equation:tech_note/Snow_Hydrology/CLM50_Tech_Note_Snow_Hydrology:8.37}
\begin{split}D_{oc,\, hphob} =D_{oc,\, dryhphob}\end{split}
\end{equation}
Deposited particles are assumed to be instantly mixed (homogeneously)
within the surface snow layer and are added after the inter-layer water
fluxes are computed (section \hyperref[\detokenize{tech_note/Snow_Hydrology/CLM50_Tech_Note_Snow_Hydrology:water-content}]{\ref{\detokenize{tech_note/Snow_Hydrology/CLM50_Tech_Note_Snow_Hydrology:water-content}}}) so that some aerosol is in the top
layer after deposition and is not immediately washed out before radiative
calculations are done. Particle masses are then redistributed each time
step based on meltwater drainage through the snow column (section
\hyperref[\detokenize{tech_note/Snow_Hydrology/CLM50_Tech_Note_Snow_Hydrology:water-content}]{\ref{\detokenize{tech_note/Snow_Hydrology/CLM50_Tech_Note_Snow_Hydrology:water-content}}}) and snow layer combination and subdivision
(section \hyperref[\detokenize{tech_note/Snow_Hydrology/CLM50_Tech_Note_Snow_Hydrology:snow-layer-combination-and-subdivision}]{\ref{\detokenize{tech_note/Snow_Hydrology/CLM50_Tech_Note_Snow_Hydrology:snow-layer-combination-and-subdivision}}}). The change in
mass of each of the particle species \(\Delta m_{sp,\, i}\)
(kg m$^{\text{-2}}$) is
\phantomsection\label{\detokenize{tech_note/Snow_Hydrology/CLM50_Tech_Note_Snow_Hydrology:equation-8.38}}\begin{equation}\label{equation:tech_note/Snow_Hydrology/CLM50_Tech_Note_Snow_Hydrology:8.38}
\begin{split}\Delta m_{sp,\, i} =\left[k_{sp} \left(q_{liq,\, i-1} c_{sp,\, i-1} -q_{liq,\, i} c_{i} \right)+D_{sp} \right]\Delta t\end{split}
\end{equation}
where \(k_{sp}\)  is the meltwater scavenging efficiency that is
unique for each species (\hyperref[\detokenize{tech_note/Snow_Hydrology/CLM50_Tech_Note_Snow_Hydrology:table-meltwater-scavenging}]{Table \ref{\detokenize{tech_note/Snow_Hydrology/CLM50_Tech_Note_Snow_Hydrology:table-meltwater-scavenging}}}), \(q_{liq,\, i-1}\)  is the flow
of liquid water into layer \(i\) from the layer above,
\(q_{liq,\, i}\)  is the flow of water out of layer \(i\) into
the layer below (kg m$^{\text{-2}}$ s$^{\text{-1}}$) (section
\hyperref[\detokenize{tech_note/Snow_Hydrology/CLM50_Tech_Note_Snow_Hydrology:water-content}]{\ref{\detokenize{tech_note/Snow_Hydrology/CLM50_Tech_Note_Snow_Hydrology:water-content}}}), \(c_{sp,\, i-1}\)  and \(c_{sp,\, i}\)  are the particle
mass mixing ratios in layers \(i-1\) and \(i\) (kg
kg$^{\text{-1}}$), \(D_{sp}\)  is the atmospheric deposition rate
(zero for all layers except layer \(snl+1\)), and \(\Delta t\)
is the model time step (s). The particle mass mixing ratio is
\phantomsection\label{\detokenize{tech_note/Snow_Hydrology/CLM50_Tech_Note_Snow_Hydrology:equation-8.39}}\begin{equation}\label{equation:tech_note/Snow_Hydrology/CLM50_Tech_Note_Snow_Hydrology:8.39}
\begin{split}c_{i} =\frac{m_{sp,\, i} }{w_{liq,\, i} +w_{ice,\, i} } .\end{split}
\end{equation}
Values of \(k_{sp}\)  are partially derived from experiments
published by {\hyperref[\detokenize{tech_note/References/CLM50_Tech_Note_References:conwayetal1996}]{\sphinxcrossref{\DUrole{std,std-ref}{Conway et al. (1996)}}}}. Particles masses are re-distributed
proportionately with snow mass when layers are combined or divided, thus
conserving particle mass within the snow column. The mass of particles
carried out with meltwater through the bottom snow layer is assumed to
be permanently lost from the snowpack, and is not maintained within the
model.


\begin{savenotes}\sphinxattablestart
\centering
\sphinxcapstartof{table}
\sphinxcaption{Meltwater scavenging efficiency for particles within snow}\label{\detokenize{tech_note/Snow_Hydrology/CLM50_Tech_Note_Snow_Hydrology:table-meltwater-scavenging}}\label{\detokenize{tech_note/Snow_Hydrology/CLM50_Tech_Note_Snow_Hydrology:id11}}
\sphinxaftercaption
\begin{tabulary}{\linewidth}[t]{|T|T|}
\hline
\sphinxstylethead{\sphinxstyletheadfamily 
Species
\unskip}\relax &
\(k_{sp}\)
\\
\hline
Hydrophilic black carbon
&
0.20
\\
\hline
Hydrophobic black carbon
&
0.03
\\
\hline
Hydrophilic organic carbon
&
0.20
\\
\hline
Hydrophobic organic carbon
&
0.03
\\
\hline
Dust species 1 (0.1-1.0 \(\mu m\))
&
0.02
\\
\hline
Dust species 2 (1.0-2.5 \(\mu m\))
&
0.02
\\
\hline
Dust species 3 (2.5-5.0 \(\mu m\))
&
0.01
\\
\hline
Dust species 4 (5.0-10.0 \(\mu m\))
&
0.01
\\
\hline
\end{tabulary}
\par
\sphinxattableend\end{savenotes}


\subsection{Initialization of snow layer}
\label{\detokenize{tech_note/Snow_Hydrology/CLM50_Tech_Note_Snow_Hydrology:initialization-of-snow-layer}}\label{\detokenize{tech_note/Snow_Hydrology/CLM50_Tech_Note_Snow_Hydrology:id5}}
If there are no existing snow layers (\(snl+1=1\)) but
\(z_{sno} \ge 0.01\) m after accounting for solid precipitation
\(q_{sno}\) , then a snow layer is initialized (\(snl=-1\)) as
follows
\phantomsection\label{\detokenize{tech_note/Snow_Hydrology/CLM50_Tech_Note_Snow_Hydrology:equation-8.40}}\begin{equation}\label{equation:tech_note/Snow_Hydrology/CLM50_Tech_Note_Snow_Hydrology:8.40}
\begin{split}\begin{array}{lcr}
\Delta z_{0} & = & z_{sno}  \\
z_{o} & = & -0.5\Delta z_{0}  \\
z_{h,\, -1} & = & -\Delta z_{0}  \\
T_{0} & = & \min \left(T_{f} ,T_{atm} \right) \\
w_{ice,\, 0} & = & W_{sno}  \\
w_{liq,\, 0} & = & 0
\end{array}.\end{split}
\end{equation}

\subsection{Snow Compaction}
\label{\detokenize{tech_note/Snow_Hydrology/CLM50_Tech_Note_Snow_Hydrology:id6}}\label{\detokenize{tech_note/Snow_Hydrology/CLM50_Tech_Note_Snow_Hydrology:snow-compaction}}
Snow compaction is initiated after the soil hydrology calculations
{[}surface runoff (section \hyperref[\detokenize{tech_note/Hydrology/CLM50_Tech_Note_Hydrology:surface-runoff}]{\ref{\detokenize{tech_note/Hydrology/CLM50_Tech_Note_Hydrology:surface-runoff}}}), infiltration (section
\hyperref[\detokenize{tech_note/Hydrology/CLM50_Tech_Note_Hydrology:infiltration}]{\ref{\detokenize{tech_note/Hydrology/CLM50_Tech_Note_Hydrology:infiltration}}}), soil water (section \hyperref[\detokenize{tech_note/Hydrology/CLM50_Tech_Note_Hydrology:soil-water}]{\ref{\detokenize{tech_note/Hydrology/CLM50_Tech_Note_Hydrology:soil-water}}}){]} are
complete. Compaction of snow includes three types of processes:
destructive metamorphism of new snow (crystal breakdown due to wind or
thermodynamic stress); snow load or overburden (pressure); and melting
(changes in snow structure due to melt-freeze cycles plus changes in
crystals due to liquid water). The total fractional compaction rate for
each snow layer \(C_{R,\, i}\)  (s$^{\text{-1}}$) is the sum of the
three compaction processes
\phantomsection\label{\detokenize{tech_note/Snow_Hydrology/CLM50_Tech_Note_Snow_Hydrology:equation-8.41}}\begin{equation}\label{equation:tech_note/Snow_Hydrology/CLM50_Tech_Note_Snow_Hydrology:8.41}
\begin{split}C_{R,\, i} =\frac{1}{\Delta z_{i} } \frac{\partial \Delta z_{i} }{\partial t} =C_{R1,\, i} +C_{R2,\, i} +C_{R3,\, i} .\end{split}
\end{equation}
Compaction is not allowed if the layer is saturated
\phantomsection\label{\detokenize{tech_note/Snow_Hydrology/CLM50_Tech_Note_Snow_Hydrology:equation-8.42}}\begin{equation}\label{equation:tech_note/Snow_Hydrology/CLM50_Tech_Note_Snow_Hydrology:8.42}
\begin{split}1-\left(\frac{w_{ice,\, i} }{f_{sno} \Delta z_{i} \rho _{ice} } +\frac{w_{liq,\, i} }{f_{sno} \Delta z_{i} \rho _{liq} } \right)\le 0.001\end{split}
\end{equation}
or if the ice content is below a minimum value
(\(w_{ice,\, i} \le 0.1\)).

Compaction as a result of destructive metamorphism \(C_{R1,\; i}\) (s$_{\text{-1}}$) is temperature dependent ({\hyperref[\detokenize{tech_note/References/CLM50_Tech_Note_References:anderson1976}]{\sphinxcrossref{\DUrole{std,std-ref}{Anderson (1976)}}}})
\phantomsection\label{\detokenize{tech_note/Snow_Hydrology/CLM50_Tech_Note_Snow_Hydrology:equation-8.43}}\begin{equation}\label{equation:tech_note/Snow_Hydrology/CLM50_Tech_Note_Snow_Hydrology:8.43}
\begin{split}C_{R1,\, i} =\left[\frac{1}{\Delta z_{i} } \frac{\partial \Delta z_{i} }{\partial t} \right]_{metamorphism} =-c_{3} c_{1} c_{2} \exp \left[-c_{4} \left(T_{f} -T_{i} \right)\right]\end{split}
\end{equation}
where \(c_{3} =2.777\times 10^{-6}\)  (s$^{\text{-1}}$) is the fractional compaction rate for \(T_{i} =T_{f}\), \(c_{4} =0.04\) K$^{\text{-1}}$, and
\phantomsection\label{\detokenize{tech_note/Snow_Hydrology/CLM50_Tech_Note_Snow_Hydrology:equation-8.44}}\begin{equation}\label{equation:tech_note/Snow_Hydrology/CLM50_Tech_Note_Snow_Hydrology:8.44}
\begin{split}\begin{array}{lr}
c_{1}  = 1 & \qquad \frac{w_{ice,\, i} }{f_{sno} \Delta z_{i} } \le 100{\rm \; kg\; m}^{{\rm -3}}  \\
c_{1}  = \exp \left[-0.046\left(\frac{w_{ice,\, i} }{f_{sno} \Delta z_{i} } -100\right)\right] & \qquad \frac{w_{ice,\, i} }{f_{sno} \Delta z_{i} } >100{\rm \; kg\; m}^{{\rm -3}}  \\
c_{2}  = 2 & \qquad \frac{w_{liq,\, i} }{f_{sno} \Delta z_{i} } >0.01 \\
c_{2}  = 1 & \qquad \frac{w_{liq,\, i} }{f_{sno} \Delta z_{i} } \le 0.01
\end{array}\end{split}
\end{equation}
where
\({w_{ice,\, i} \mathord{\left/ {\vphantom {w_{ice,\, i}  \left(f_{sno} \Delta z_{i} \right)}} \right. \kern-\nulldelimiterspace} \left(f_{sno} \Delta z_{i} \right)}\)
and
\({w_{liq,\, i} \mathord{\left/ {\vphantom {w_{liq,\, i}  \left(f_{sno} \Delta z_{i} \right)}} \right. \kern-\nulldelimiterspace} \left(f_{sno} \Delta z_{i} \right)}\)
are the bulk densities of liquid water and ice (kg m$^{\text{-3}}$).

The compaction rate as a result of overburden \(C_{R2,\; i}\) (s$^{\text{-1}}$) is a linear function of the snow load pressure \(P_{s,\, i}\) (kg m$^{\text{-2}}$) ({\hyperref[\detokenize{tech_note/References/CLM50_Tech_Note_References:anderson1976}]{\sphinxcrossref{\DUrole{std,std-ref}{Anderson (1976)}}}})
\phantomsection\label{\detokenize{tech_note/Snow_Hydrology/CLM50_Tech_Note_Snow_Hydrology:equation-8.45}}\begin{equation}\label{equation:tech_note/Snow_Hydrology/CLM50_Tech_Note_Snow_Hydrology:8.45}
\begin{split}C_{R2,\, i} =\left[\frac{1}{\Delta z_{i} } \frac{\partial \Delta z_{i} }{\partial t} \right]_{overburden} =-\frac{P_{s,\, i} }{\eta }\end{split}
\end{equation}
where \(\eta\)  is a viscosity coefficient (kg s m$^{\text{-2}}$) that varies with density and temperature as
\phantomsection\label{\detokenize{tech_note/Snow_Hydrology/CLM50_Tech_Note_Snow_Hydrology:equation-8.46}}\begin{equation}\label{equation:tech_note/Snow_Hydrology/CLM50_Tech_Note_Snow_Hydrology:8.46}
\begin{split}\eta =\eta _{0} \exp \left[c_{5} \left(T_{f} -T_{i} \right)+c_{6} \frac{w_{ice,\, i} }{f_{sno} \Delta z_{i} } \right]\end{split}
\end{equation}
where \(\eta _{0} =9\times 10^{5}\)  kg s m$^{\text{-2}}$, and
\(c_{5} =0.08\) K$^{\text{-1}}$, \(c_{6} =0.023\)
m$^{\text{3}}$ kg$^{\text{-1}}$ are constants. The snow load
pressure \(P_{s,\, i}\)  is calculated for each layer as the sum of
the ice \(w_{ice,\, i}\)  and liquid water contents
\(w_{liq,\, i}\)  of the layers above plus half the ice and liquid
water contents of the layer being compacted
\phantomsection\label{\detokenize{tech_note/Snow_Hydrology/CLM50_Tech_Note_Snow_Hydrology:equation-8.47}}\begin{equation}\label{equation:tech_note/Snow_Hydrology/CLM50_Tech_Note_Snow_Hydrology:8.47}
\begin{split}P_{s,\, i} =\frac{w_{ice,\, i} +w_{liq,\, i} }{2} +\sum _{j=snl+1}^{j=i-1}\left(w_{ice,\, j} +w_{liq,\, j} \right) .\end{split}
\end{equation}
The compaction rate due to melting \(C_{R3,\; i}\) (s$^{\text{-1}}$) is taken to be the ratio of the change in snow ice
mass after the melting to the mass before melting
\phantomsection\label{\detokenize{tech_note/Snow_Hydrology/CLM50_Tech_Note_Snow_Hydrology:equation-8.48}}\begin{equation}\label{equation:tech_note/Snow_Hydrology/CLM50_Tech_Note_Snow_Hydrology:8.48}
\begin{split}C_{R3,\, i} =\left[\frac{1}{\Delta z_{i} } \frac{\partial \Delta z_{i} }{\partial t} \right]_{melt} =-\frac{1}{\Delta t} \max \left(0,\frac{W_{sno,\, i}^{n} -W_{sno,\, i}^{n+1} }{W_{sno,\, i}^{n} } \right)\end{split}
\end{equation}
and melting is identified during the phase change calculations (section
\hyperref[\detokenize{tech_note/Soil_Snow_Temperatures/CLM50_Tech_Note_Soil_Snow_Temperatures:phase-change}]{\ref{\detokenize{tech_note/Soil_Snow_Temperatures/CLM50_Tech_Note_Soil_Snow_Temperatures:phase-change}}}). Because snow depth is defined as the average depth of the snow
covered area, the snow depth must also be updated for changes in
\(f_{sno}\) .
\phantomsection\label{\detokenize{tech_note/Snow_Hydrology/CLM50_Tech_Note_Snow_Hydrology:equation-8.49}}\begin{equation}\label{equation:tech_note/Snow_Hydrology/CLM50_Tech_Note_Snow_Hydrology:8.49}
\begin{split}C_{R4,\, i} =\left[\frac{1}{\Delta z_{i} } \frac{\partial \Delta z_{i} }{\partial t} \right]_{fsno} =-\frac{1}{\Delta t} \max \left(0,\frac{f_{sno,\, i}^{n} -f_{sno,\, i}^{n+1} }{f_{sno,\, i}^{n} } \right)\end{split}
\end{equation}
The snow layer thickness after compaction is then
\phantomsection\label{\detokenize{tech_note/Snow_Hydrology/CLM50_Tech_Note_Snow_Hydrology:equation-8.50}}\begin{equation}\label{equation:tech_note/Snow_Hydrology/CLM50_Tech_Note_Snow_Hydrology:8.50}
\begin{split}\Delta z_{i}^{n+1} =\Delta z_{i}^{n} \left(1+C_{R,\, i} \Delta t\right).\end{split}
\end{equation}

\subsection{Snow Layer Combination and Subdivision}
\label{\detokenize{tech_note/Snow_Hydrology/CLM50_Tech_Note_Snow_Hydrology:snow-layer-combination-and-subdivision}}\label{\detokenize{tech_note/Snow_Hydrology/CLM50_Tech_Note_Snow_Hydrology:id7}}
After the determination of snow temperature including phase change(Chapter
\hyperref[\detokenize{tech_note/Soil_Snow_Temperatures/CLM50_Tech_Note_Soil_Snow_Temperatures:rst-soil-and-snow-temperatures}]{\ref{\detokenize{tech_note/Soil_Snow_Temperatures/CLM50_Tech_Note_Soil_Snow_Temperatures:rst-soil-and-snow-temperatures}}}), snow hydrology (Chapter
\hyperref[\detokenize{tech_note/Snow_Hydrology/CLM50_Tech_Note_Snow_Hydrology:rst-snow-hydrology}]{\ref{\detokenize{tech_note/Snow_Hydrology/CLM50_Tech_Note_Snow_Hydrology:rst-snow-hydrology}}}), and the compaction calculations (section
\hyperref[\detokenize{tech_note/Snow_Hydrology/CLM50_Tech_Note_Snow_Hydrology:snow-compaction}]{\ref{\detokenize{tech_note/Snow_Hydrology/CLM50_Tech_Note_Snow_Hydrology:snow-compaction}}}) , the number of snow layers is adjusted by
either combining or subdividing layers. The combination and subdivision
of snow layers is based on {\hyperref[\detokenize{tech_note/References/CLM50_Tech_Note_References:jordan1991}]{\sphinxcrossref{\DUrole{std,std-ref}{Jordan (1991)}}}}.


\subsubsection{Combination}
\label{\detokenize{tech_note/Snow_Hydrology/CLM50_Tech_Note_Snow_Hydrology:id8}}\label{\detokenize{tech_note/Snow_Hydrology/CLM50_Tech_Note_Snow_Hydrology:combination}}
If a snow layer has nearly melted or if its thickness
\(\Delta z_{i}\)  is less than the prescribed minimum thickness
\(\Delta z_{\min }\)  (\hyperref[\detokenize{tech_note/Snow_Hydrology/CLM50_Tech_Note_Snow_Hydrology:table-snow-layer-thickness}]{Table \ref{\detokenize{tech_note/Snow_Hydrology/CLM50_Tech_Note_Snow_Hydrology:table-snow-layer-thickness}}}), the layer is combined with a
neighboring layer. The overlying or underlying layer is selected as the
neighboring layer according to the following rules
\begin{enumerate}
\item {} 
If the top layer is being removed, it is combined with the underlying
layer

\item {} 
If the underlying layer is not snow (i.e., it is the top soil layer),
the layer is combined with the overlying layer

\item {} 
If the layer is nearly completely melted, the layer is combined with
the underlying layer

\item {} 
If none of the above rules apply, the layer is combined with the
thinnest neighboring layer.

\end{enumerate}

A first pass is made through all snow layers to determine if any layer
is nearly melted (\(w_{ice,\, i} \le 0.1\)). If so, the remaining
liquid water and ice content of layer \(i\) is combined with the
underlying neighbor \(i+1\) as
\phantomsection\label{\detokenize{tech_note/Snow_Hydrology/CLM50_Tech_Note_Snow_Hydrology:equation-8.51}}\begin{equation}\label{equation:tech_note/Snow_Hydrology/CLM50_Tech_Note_Snow_Hydrology:8.51}
\begin{split}w_{liq,\, i+1} =w_{liq,\, i+1} +w_{liq,\, i}\end{split}
\end{equation}\phantomsection\label{\detokenize{tech_note/Snow_Hydrology/CLM50_Tech_Note_Snow_Hydrology:equation-8.52}}\begin{equation}\label{equation:tech_note/Snow_Hydrology/CLM50_Tech_Note_Snow_Hydrology:8.52}
\begin{split}w_{ice,\, i+1} =w_{ice,\, i+1} +w_{ice,\, i} .\end{split}
\end{equation}
This includes the snow layer directly above the top soil layer. In this
case, the liquid water and ice content of the melted snow layer is added
to the contents of the top soil layer. The layer properties,
\(T_{i}\) , \(w_{ice,\, i}\) , \(w_{liq,\, i}\) ,
\(\Delta z_{i}\) , are then re-indexed so that the layers above the
eliminated layer are shifted down by one and the number of snow layers
is decremented accordingly.

At this point, if there are no explicit snow layers remaining
(\(snl=0\)), the snow water equivalent \(W_{sno}\)  and snow
depth \(z_{sno}\)  are set to zero, otherwise, \(W_{sno}\)  and
\(z_{sno}\)  are re-calculated as
\phantomsection\label{\detokenize{tech_note/Snow_Hydrology/CLM50_Tech_Note_Snow_Hydrology:equation-8.53}}\begin{equation}\label{equation:tech_note/Snow_Hydrology/CLM50_Tech_Note_Snow_Hydrology:8.53}
\begin{split}W_{sno} =\sum _{i=snl+1}^{i=0}\left(w_{ice,\, i} +w_{liq,\, i} \right)\end{split}
\end{equation}\phantomsection\label{\detokenize{tech_note/Snow_Hydrology/CLM50_Tech_Note_Snow_Hydrology:equation-8.54}}\begin{equation}\label{equation:tech_note/Snow_Hydrology/CLM50_Tech_Note_Snow_Hydrology:8.54}
\begin{split}z_{sno} =\sum _{i=snl+1}^{i=0}\Delta z_{i}  .\end{split}
\end{equation}
If the snow depth \(0<z_{sno} <0.01\) m or the snow density
\(\frac{W_{sno} }{f_{sno} z_{sno} } <50\) kg/m3, the number of snow
layers is set to zero, the total ice content of the snowpack
\(\sum _{i=snl+1}^{i=0}w_{ice,\; i}\)  is assigned to
\(W_{sno}\) , and the total liquid water
\(\sum _{i=snl+1}^{i=0}w_{liq,\; i}\)  is assigned to the top soil
layer. Otherwise, the layers are combined according to the rules above.

When two snow layers are combined (denoted here as 1 and 2), their
thickness combination (\(c\)) is
\phantomsection\label{\detokenize{tech_note/Snow_Hydrology/CLM50_Tech_Note_Snow_Hydrology:equation-8.55}}\begin{equation}\label{equation:tech_note/Snow_Hydrology/CLM50_Tech_Note_Snow_Hydrology:8.55}
\begin{split}\Delta z_{c} =\Delta z_{1} +\Delta z_{2} ,\end{split}
\end{equation}
their mass combination is
\phantomsection\label{\detokenize{tech_note/Snow_Hydrology/CLM50_Tech_Note_Snow_Hydrology:equation-8.56}}\begin{equation}\label{equation:tech_note/Snow_Hydrology/CLM50_Tech_Note_Snow_Hydrology:8.56}
\begin{split}w_{liq,\, c} =w_{liq,\, 1} +w_{liq,\, 2}\end{split}
\end{equation}\phantomsection\label{\detokenize{tech_note/Snow_Hydrology/CLM50_Tech_Note_Snow_Hydrology:equation-8.57}}\begin{equation}\label{equation:tech_note/Snow_Hydrology/CLM50_Tech_Note_Snow_Hydrology:8.57}
\begin{split}w_{ice,\, c} =w_{ice,\, 1} +w_{ice,\, 2} ,\end{split}
\end{equation}
and their temperatures are combined as
\phantomsection\label{\detokenize{tech_note/Snow_Hydrology/CLM50_Tech_Note_Snow_Hydrology:equation-8.58}}\begin{equation}\label{equation:tech_note/Snow_Hydrology/CLM50_Tech_Note_Snow_Hydrology:8.58}
\begin{split}T_{c} =T_{f} +\frac{h_{c} -L_{f} w_{liq,\, c} }{C_{ice} w_{ice,\, c} +C_{liq} w_{liq,\, c} }\end{split}
\end{equation}
where \(h_{c} =h_{1} +h_{2}\)  is the combined enthalpy
\(h_{i}\)  of the two layers where
\phantomsection\label{\detokenize{tech_note/Snow_Hydrology/CLM50_Tech_Note_Snow_Hydrology:equation-8.59}}\begin{equation}\label{equation:tech_note/Snow_Hydrology/CLM50_Tech_Note_Snow_Hydrology:8.59}
\begin{split}h_{i} =\left(C_{ice} w_{ice,\, i} +C_{liq} w_{liq,\, i} \right)\left(T_{i} -T_{f} \right)+L_{f} w_{liq,\, i} .\end{split}
\end{equation}
In these equations, \(L_{f}\)  is the latent heat of fusion (J kg$^{\text{-1}}$) and \(C_{liq}\)  and \(C_{ice}\)  are the specific
heat capacities (J kg$^{\text{-1}}$ K$^{\text{-1}}$) of liquid water and ice,
respectively (\hyperref[\detokenize{tech_note/Ecosystem/CLM50_Tech_Note_Ecosystem:table-physical-constants}]{Table \ref{\detokenize{tech_note/Ecosystem/CLM50_Tech_Note_Ecosystem:table-physical-constants}}}). After layer combination,
the node depths and layer interfaces (\hyperref[\detokenize{tech_note/Snow_Hydrology/CLM50_Tech_Note_Snow_Hydrology:figure-three-layer-snow-pack}]{Figure \ref{\detokenize{tech_note/Snow_Hydrology/CLM50_Tech_Note_Snow_Hydrology:figure-three-layer-snow-pack}}})
are recalculated from
\phantomsection\label{\detokenize{tech_note/Snow_Hydrology/CLM50_Tech_Note_Snow_Hydrology:equation-8.60}}\begin{equation}\label{equation:tech_note/Snow_Hydrology/CLM50_Tech_Note_Snow_Hydrology:8.60}
\begin{split}z_{i} =z_{h,\, i} -0.5\Delta z_{i} \qquad i=0,\ldots ,snl+1\end{split}
\end{equation}\phantomsection\label{\detokenize{tech_note/Snow_Hydrology/CLM50_Tech_Note_Snow_Hydrology:equation-8.61}}\begin{equation}\label{equation:tech_note/Snow_Hydrology/CLM50_Tech_Note_Snow_Hydrology:8.61}
\begin{split}z_{h,\, i-1} =z_{h,\, i} -\Delta z_{i} \qquad i=0,\ldots ,snl+1\end{split}
\end{equation}
where \(\Delta z_{i}\)  is the layer thickness.


\begin{savenotes}\sphinxattablestart
\centering
\sphinxcapstartof{table}
\sphinxcaption{Minimum and maximum thickness of snow layers (m)}\label{\detokenize{tech_note/Snow_Hydrology/CLM50_Tech_Note_Snow_Hydrology:table-snow-layer-thickness}}\label{\detokenize{tech_note/Snow_Hydrology/CLM50_Tech_Note_Snow_Hydrology:id12}}
\sphinxaftercaption
\begin{tabular}[t]{|*{6}{\X{1}{6}|}}
\hline
\sphinxstylethead{\sphinxstyletheadfamily 
Layer
\unskip}\relax &
\(\Delta z_{\min }\)
&
\(N_{l}\)
&
\(N_{u}\)
&
\(\left(\Delta z_{\max } \right)_{l}\)
&
\(\left(\Delta z_{\max } \right)_{u}\)
\\
\hline
1 (top)
&
0.010
&
1
&
\(>\)1
&
0.03
&
0.02
\\
\hline
2
&
0.015
&
2
&
\(>\)2
&
0.07
&
0.05
\\
\hline
3
&
0.025
&
3
&
\(>\)3
&
0.18
&
0.11
\\
\hline
4
&
0.055
&
4
&
\(>\)4
&
0.41
&
0.23
\\
\hline
5 (bottom)
&
0.115
&
5
&\begin{itemize}
\item {} 
\end{itemize}
&\begin{itemize}
\item {} 
\end{itemize}
&\begin{itemize}
\item {} 
\end{itemize}
\\
\hline
\end{tabular}
\par
\sphinxattableend\end{savenotes}

The maximum snow layer thickness, \(\Delta z_{\max }\) , depends on
the number of layers, \(N_{l}\)  and \(N_{u}\)  (section
\hyperref[\detokenize{tech_note/Snow_Hydrology/CLM50_Tech_Note_Snow_Hydrology:subdivision}]{\ref{\detokenize{tech_note/Snow_Hydrology/CLM50_Tech_Note_Snow_Hydrology:subdivision}}}).


\subsubsection{Subdivision}
\label{\detokenize{tech_note/Snow_Hydrology/CLM50_Tech_Note_Snow_Hydrology:subdivision}}\label{\detokenize{tech_note/Snow_Hydrology/CLM50_Tech_Note_Snow_Hydrology:id9}}
The snow layers are subdivided when the layer thickness exceeds the
prescribed maximum thickness \(\Delta z_{\max }\)  with lower and
upper bounds that depend on the number of snow layers (numref:\sphinxtitleref{Table snow layer thickness}). For
example, if there is only one layer, then the maximum thickness of that
layer is 0.03 m, however, if there is more than one layer, then the
maximum thickness of the top layer is 0.02 m. Layers are checked
sequentially from top to bottom for this limit. If there is only one
snow layer and its thickness is greater than 0.03 m (\hyperref[\detokenize{tech_note/Snow_Hydrology/CLM50_Tech_Note_Snow_Hydrology:table-snow-layer-thickness}]{Table \ref{\detokenize{tech_note/Snow_Hydrology/CLM50_Tech_Note_Snow_Hydrology:table-snow-layer-thickness}}}), the
layer is subdivided into two layers of equal thickness, liquid water and
ice contents, and temperature. If there is an existing layer below the
layer to be subdivided, the thickness \(\Delta z_{i}\) , liquid
water and ice contents, \(w_{liq,\; i}\)  and \(w_{ice,\; i}\) ,
and temperature \(T_{i}\)  of the excess snow are combined with the
underlying layer according to equations -. If there is no underlying
layer after adjusting the layer for the excess snow, the layer is
subdivided into two layers of equal thickness, liquid water and ice
contents. The vertical snow temperature profile is maintained by
calculating the slope between the layer above the splitting layer
(\(T_{1}\) ) and the splitting layer (\(T_{2}\) ) and
constraining the new temperatures (\(T_{2}^{n+1}\) ,
\(T_{3}^{n+1}\) ) to lie along this slope. The temperature of the
lower layer is first evaluated from
\phantomsection\label{\detokenize{tech_note/Snow_Hydrology/CLM50_Tech_Note_Snow_Hydrology:equation-8.62}}\begin{equation}\label{equation:tech_note/Snow_Hydrology/CLM50_Tech_Note_Snow_Hydrology:8.62}
\begin{split}T'_{3} =T_{2}^{n} -\left(\frac{T_{1}^{n} -T_{2}^{n} }{{\left(\Delta z_{1}^{n} +\Delta z_{2}^{n} \right)\mathord{\left/ {\vphantom {\left(\Delta z_{1}^{n} +\Delta z_{2}^{n} \right) 2}} \right. \kern-\nulldelimiterspace} 2} } \right)\left(\frac{\Delta z_{2}^{n+1} }{2} \right),\end{split}
\end{equation}
then adjusted as,
\phantomsection\label{\detokenize{tech_note/Snow_Hydrology/CLM50_Tech_Note_Snow_Hydrology:equation-8.63}}\begin{equation}\label{equation:tech_note/Snow_Hydrology/CLM50_Tech_Note_Snow_Hydrology:8.63}
\begin{split}\begin{array}{lr}
T_{3}^{n+1} = T_{2}^{n} & \qquad T'_{3} \ge T_{f}  \\
T_{2}^{n+1} = T_{2}^{n} +\left(\frac{T_{1}^{n} -T_{2}^{n} }{{\left(\Delta z_{1} +\Delta z_{2}^{n} \right)\mathord{\left/ {\vphantom {\left(\Delta z_{1} +\Delta z_{2}^{n} \right) 2}} \right. \kern-\nulldelimiterspace} 2} } \right)\left(\frac{\Delta z_{2}^{n+1} }{2} \right) & \qquad T'_{3} <T_{f}
\end{array}\end{split}
\end{equation}
where here the subscripts 1, 2, and 3 denote three layers numbered from
top to bottom. After layer subdivision, the node depths and layer
interfaces are recalculated from equations and .


\section{Stomatal Resistance and Photosynthesis}
\label{\detokenize{tech_note/Photosynthesis/CLM50_Tech_Note_Photosynthesis:rst-stomatal-resistance-and-photosynthesis}}\label{\detokenize{tech_note/Photosynthesis/CLM50_Tech_Note_Photosynthesis::doc}}\label{\detokenize{tech_note/Photosynthesis/CLM50_Tech_Note_Photosynthesis:stomatal-resistance-and-photosynthesis}}

\subsection{Summary of CLM5.0 updates relative to the CLM4.5}
\label{\detokenize{tech_note/Photosynthesis/CLM50_Tech_Note_Photosynthesis:summary-of-clm5-0-updates-relative-to-the-clm4-5}}
We describe here the complete photosynthesis and stomatal conductance parameterizations that
appear in CLM5.0. Corresponding information for CLM4.5 appeared in the
CLM4.5 Technical Note ({\hyperref[\detokenize{tech_note/References/CLM50_Tech_Note_References:olesonetal2013}]{\sphinxcrossref{\DUrole{std,std-ref}{Oleson et al. 2013}}}}).

CLM5 includes the following new changes to photosynthesis and stomatal conductance:
\begin{itemize}
\item {} 
Default stomatal conductance calculation uses the Medlyn conductance model

\item {} 
\(J_{max}\) is predicted by the LUNA model (Chapter \hyperref[\detokenize{tech_note/Photosynthetic_Capacity/CLM50_Tech_Note_Photosynthetic_Capacity:rst-photosynthetic-capacity}]{\ref{\detokenize{tech_note/Photosynthetic_Capacity/CLM50_Tech_Note_Photosynthetic_Capacity:rst-photosynthetic-capacity}}})

\item {} 
Leaf N concentration and the fraction of leaf N in Rubisco used to calculate \(V_{cmax25}\) are determined by the LUNA model (Chapter \hyperref[\detokenize{tech_note/Photosynthetic_Capacity/CLM50_Tech_Note_Photosynthetic_Capacity:rst-photosynthetic-capacity}]{\ref{\detokenize{tech_note/Photosynthetic_Capacity/CLM50_Tech_Note_Photosynthetic_Capacity:rst-photosynthetic-capacity}}})

\item {} 
Water stress is applied by the hydraulic conductance model (Chapter \hyperref[\detokenize{tech_note/Plant_Hydraulics/CLM50_Tech_Note_Plant_Hydraulics:rst-plant-hydraulics}]{\ref{\detokenize{tech_note/Plant_Hydraulics/CLM50_Tech_Note_Plant_Hydraulics:rst-plant-hydraulics}}})

\end{itemize}


\subsection{Introduction}
\label{\detokenize{tech_note/Photosynthesis/CLM50_Tech_Note_Photosynthesis:introduction}}
Leaf stomatal resistance, which is needed for the water vapor flux
(Chapter \hyperref[\detokenize{tech_note/Fluxes/CLM50_Tech_Note_Fluxes:rst-momentum-sensible-heat-and-latent-heat-fluxes}]{\ref{\detokenize{tech_note/Fluxes/CLM50_Tech_Note_Fluxes:rst-momentum-sensible-heat-and-latent-heat-fluxes}}}),
is coupled to leaf photosynthesis similar to Collatz et al.
({\hyperref[\detokenize{tech_note/References/CLM50_Tech_Note_References:collatzetal1991}]{\sphinxcrossref{\DUrole{std,std-ref}{1991}}}}, {\hyperref[\detokenize{tech_note/References/CLM50_Tech_Note_References:collatzetal1992}]{\sphinxcrossref{\DUrole{std,std-ref}{1992}}}}). These equations are solved separately for sunlit and
shaded leaves using average absorbed photosynthetically active radiation
for sunlit and shaded leaves
{[}\(\phi ^{sun}\) ,\(\phi ^{sha}\)  W m$^{\text{-2}}$
(section \hyperref[\detokenize{tech_note/Radiative_Fluxes/CLM50_Tech_Note_Radiative_Fluxes:solar-fluxes}]{\ref{\detokenize{tech_note/Radiative_Fluxes/CLM50_Tech_Note_Radiative_Fluxes:solar-fluxes}}}){]} to give sunlit and shaded stomatal resistance
(\(r_{s}^{sun}\) ,\(r_{s}^{sha}\) s m$^{\text{-1}}$) and
photosynthesis (\(A^{sun}\) ,\(A^{sha}\)  µmol CO$_{\text{2}}$ m$^{\text{-2}}$ s$^{\text{-1}}$). Canopy
photosynthesis is \(A^{sun} L^{sun} +A^{sha} L^{sha}\) , where
\(L^{sun}\)  and \(L^{sha}\)  are the sunlit and shaded leaf
area indices (section \hyperref[\detokenize{tech_note/Radiative_Fluxes/CLM50_Tech_Note_Radiative_Fluxes:solar-fluxes}]{\ref{\detokenize{tech_note/Radiative_Fluxes/CLM50_Tech_Note_Radiative_Fluxes:solar-fluxes}}}). Canopy conductance is
\(\frac{1}{r_{b} +r_{s}^{sun} } L^{sun} +\frac{1}{r_{b} +r_{s}^{sha} } L^{sha}\) ,
where \(r_{b}\)  is the leaf boundary layer resistance (section
\hyperref[\detokenize{tech_note/Fluxes/CLM50_Tech_Note_Fluxes:sensible-and-latent-heat-fluxes-and-temperature-for-vegetated-surfaces}]{\ref{\detokenize{tech_note/Fluxes/CLM50_Tech_Note_Fluxes:sensible-and-latent-heat-fluxes-and-temperature-for-vegetated-surfaces}}}).
The implementation is described by Bonan et al. ({\hyperref[\detokenize{tech_note/References/CLM50_Tech_Note_References:bonanetal2011}]{\sphinxcrossref{\DUrole{std,std-ref}{2011}}}}), though different
methods of calculating stomatal conductance, \(J_{max}\), and the nitrogen variables
used to calculate \(V_{cmax25}\) are used in CLM5.


\subsection{Stomatal resistance}
\label{\detokenize{tech_note/Photosynthesis/CLM50_Tech_Note_Photosynthesis:stomatal-resistance}}\label{\detokenize{tech_note/Photosynthesis/CLM50_Tech_Note_Photosynthesis:id1}}
CLM5 calculates stomatal conductance using the Medlyn stomatal conductance model ({\hyperref[\detokenize{tech_note/References/CLM50_Tech_Note_References:medlynetal2011}]{\sphinxcrossref{\DUrole{std,std-ref}{Medlyn et al. 2011}}}}).
Previous versions of CLM calculated leaf stomatal resistance is using the Ball-Berry conductance
model as described by {\hyperref[\detokenize{tech_note/References/CLM50_Tech_Note_References:collatzetal1991}]{\sphinxcrossref{\DUrole{std,std-ref}{Collatz et al. (1991)}}}} and implemented in global
climate models ({\hyperref[\detokenize{tech_note/References/CLM50_Tech_Note_References:sellersetal1996}]{\sphinxcrossref{\DUrole{std,std-ref}{Sellers et al. 1996}}}}). The Medlyn model
calculates stomatal conductance (i.e., the inverse of resistance) based on net leaf
photosynthesis, the vapor pressure deficit, and the CO$_{\text{2}}$ concentration at the leaf surface.
Leaf stomatal resistance is:
\phantomsection\label{\detokenize{tech_note/Photosynthesis/CLM50_Tech_Note_Photosynthesis:equation-9.1}}\begin{equation}\label{equation:tech_note/Photosynthesis/CLM50_Tech_Note_Photosynthesis:9.1}
\begin{split}\frac{1}{r_{s} } =g_{s} = g_{o} + (1 + \frac{g_{1} }{\sqrt{D}}) \frac{A_{n} }{{c_{s} \mathord{\left/ {\vphantom {c_{s}  P_{atm} }} \right. \kern-\nulldelimiterspace} P_{atm} } }\end{split}
\end{equation}
where \(r_{s}\) is leaf stomatal resistance (s m$^{\text{2}}$
\(\mu\)mol$^{\text{-1}}$), \(g_{o}\) is the minimum stomatal conductance
(\(\mu\) mol m $^{\text{-2}}$ s$^{\text{-1}}$), \(A_{n}\) is leaf net
photosynthesis (\(\mu\)mol CO$_{\text{2}}$ m$^{\text{-2}}$
s$^{\text{-1}}$), \(c_{s}\) is the CO$_{\text{2}}$ partial
pressure at the leaf surface (Pa), \(P_{atm}\) is the atmospheric
pressure (Pa), and \(D\) is the vapor pressure deficit at the leaf surface (Pa).
\(g_{1}\) is a plant functional type dependent parameter (\hyperref[\detokenize{tech_note/Photosynthesis/CLM50_Tech_Note_Photosynthesis:table-plant-functional-type-pft-stomatal-conductance-parameters}]{Table \ref{\detokenize{tech_note/Photosynthesis/CLM50_Tech_Note_Photosynthesis:table-plant-functional-type-pft-stomatal-conductance-parameters}}})
and can be calculated as
\phantomsection\label{\detokenize{tech_note/Photosynthesis/CLM50_Tech_Note_Photosynthesis:equation-9.2}}\begin{equation}\label{equation:tech_note/Photosynthesis/CLM50_Tech_Note_Photosynthesis:9.2}
\begin{split}g_{1} = {\sqrt{\frac{3\Gamma\lambda}{1.6}}}\end{split}
\end{equation}
where \(\Gamma\) (mol mol$^{\text{-1}}$) is the CO$_{\text{2}}$ compensation point
for photosynthesis without dark respiration,
1.6 is the ratio of diffusivity of CO$_{\text{2}}$ to H$_{\text{2}}$ O  and \(\lambda\)
(mol H$_{\text{2}}$ O mol$^{\text{-1}}$) is a parameter
describing the marginal water cost of carbon gain.
The value for \(g_{o}=100\) \(\mu\) mol m $^{\text{-2}}$ s$^{\text{-1}}$ for
C$_{\text{3}}$ and C$_{\text{4}}$ plants.
Photosynthesis is calculated for sunlit (\(A^{sun}\)) and shaded
(\(A^{sha}\)) leaves to give \(r_{s}^{sun}\) and
\(r_{s}^{sha}\). Additionally, soil water influences stomatal
resistance through plant hydraulic stress, detailed in
the {\hyperref[\detokenize{tech_note/Plant_Hydraulics/CLM50_Tech_Note_Plant_Hydraulics:rst-plant-hydraulics}]{\sphinxcrossref{\DUrole{std,std-ref}{Plant Hydraulics}}}} chapter.

Resistance is converted from units of
s m$^{\text{2}}$ \(\mu\) mol$^{\text{-1}}$ to  s m$^{\text{-1}}$ as:
1 s m$^{\text{-1}}$ = \(1\times 10^{-9} R_{gas} \frac{\theta _{atm} }{P_{atm} }\)
\(\mu\) mol$^{\text{-1}}$ m$^{\text{2}}$ s, {[}same as 4.5, but units seem off. check that units are correct{]}
where \(R_{gas}\) is the universal gas constant (J K$^{\text{-1}}$
kmol$^{\text{-1}}$) (\hyperref[\detokenize{tech_note/Ecosystem/CLM50_Tech_Note_Ecosystem:table-physical-constants}]{Table \ref{\detokenize{tech_note/Ecosystem/CLM50_Tech_Note_Ecosystem:table-physical-constants}}}) and \(\theta _{atm}\) is the
atmospheric potential temperature (K).


\begin{savenotes}\sphinxattablestart
\centering
\sphinxcapstartof{table}
\sphinxcaption{Plant functional type (PFT) stomatal conductance parameters.}\label{\detokenize{tech_note/Photosynthesis/CLM50_Tech_Note_Photosynthesis:table-plant-functional-type-pft-stomatal-conductance-parameters}}\label{\detokenize{tech_note/Photosynthesis/CLM50_Tech_Note_Photosynthesis:id4}}
\sphinxaftercaption
\begin{tabulary}{\linewidth}[t]{|T|T|}
\hline
\sphinxstylethead{\sphinxstyletheadfamily 
PFT
\unskip}\relax &\sphinxstylethead{\sphinxstyletheadfamily 
g$_{\text{1}}$ (Pa)
\unskip}\relax \\
\hline
NET Temperate
&
2.35
\\
\hline
NET Boreal
&
2.35
\\
\hline
NDT Boreal
&
2.35
\\
\hline
BET Tropical
&
4.12
\\
\hline
BET temperate
&
4.12
\\
\hline
BDT tropical
&
4.45
\\
\hline
BDT temperate
&
4.45
\\
\hline
BDT boreal
&
4.45
\\
\hline
BES temperate
&
4.70
\\
\hline
BDS temperate
&
4.70
\\
\hline
BDS boreal
&
4.70
\\
\hline
C$_{\text{3}}$ arctic grass
&
2.22
\\
\hline
C$_{\text{3}}$ grass
&
5.25
\\
\hline
C$_{\text{4}}$ grass
&
1.62
\\
\hline
Temperate Corn
&
1.79
\\
\hline
Spring Wheat
&
5.79
\\
\hline
Temperate Soybean
&
5.79
\\
\hline
Cotton
&
5.79
\\
\hline
Rice
&
5.79
\\
\hline
Sugarcane
&
1.79
\\
\hline
Tropical Corn
&
1.79
\\
\hline
Tropical Soybean
&
5.79
\\
\hline
\end{tabulary}
\par
\sphinxattableend\end{savenotes}


\subsection{Photosynthesis}
\label{\detokenize{tech_note/Photosynthesis/CLM50_Tech_Note_Photosynthesis:id2}}\label{\detokenize{tech_note/Photosynthesis/CLM50_Tech_Note_Photosynthesis:photosynthesis}}
Photosynthesis in C$_{\text{3}}$ plants is based on the model of
{\hyperref[\detokenize{tech_note/References/CLM50_Tech_Note_References:farquharetal1980}]{\sphinxcrossref{\DUrole{std,std-ref}{Farquhar et al. (1980)}}}}. Photosynthesis in C$_{\text{4}}$ plants is
based on the model of {\hyperref[\detokenize{tech_note/References/CLM50_Tech_Note_References:collatzetal1992}]{\sphinxcrossref{\DUrole{std,std-ref}{Collatz et al. (1992)}}}}. {\hyperref[\detokenize{tech_note/References/CLM50_Tech_Note_References:bonanetal2011}]{\sphinxcrossref{\DUrole{std,std-ref}{Bonan et al. (2011)}}}}
describe the implementation, modified here. In its simplest form, leaf
net photosynthesis after accounting for respiration (\(R_{d}\) ) is
\phantomsection\label{\detokenize{tech_note/Photosynthesis/CLM50_Tech_Note_Photosynthesis:equation-9.3}}\begin{equation}\label{equation:tech_note/Photosynthesis/CLM50_Tech_Note_Photosynthesis:9.3}
\begin{split}A_{n} =\min \left(A_{c} ,A_{j} ,A_{p} \right)-R_{d} .\end{split}
\end{equation}
The RuBP carboxylase (Rubisco) limited rate of carboxylation
\(A_{c}\)  (\(\mu\) mol CO$_{\text{2}}$ m$^{\text{-2}}$
s$^{\text{-1}}$) is
\phantomsection\label{\detokenize{tech_note/Photosynthesis/CLM50_Tech_Note_Photosynthesis:equation-9.4}}\begin{equation}\label{equation:tech_note/Photosynthesis/CLM50_Tech_Note_Photosynthesis:9.4}
\begin{split}A_{c} =\left\{\begin{array}{l} {\frac{V_{c\max } \left(c_{i} -\Gamma _{\*} \right)}{c_{i} +K_{c} \left(1+{o_{i} \mathord{\left/ {\vphantom {o_{i}  K_{o} }} \right. \kern-\nulldelimiterspace} K_{o} } \right)} \qquad {\rm for\; C}_{{\rm 3}} {\rm \; plants}} \\ {V_{c\max } \qquad \qquad \qquad {\rm for\; C}_{{\rm 4}} {\rm \; plants}} \end{array}\right\}\qquad \qquad c_{i} -\Gamma _{\*} \ge 0.\end{split}
\end{equation}
The maximum rate of carboxylation allowed by the capacity to regenerate
RuBP (i.e., the light-limited rate) \(A_{j}\)  (\(\mu\) mol
CO$_{\text{2}}$ m$^{\text{-2}}$ s$^{\text{-1}}$) is
\phantomsection\label{\detokenize{tech_note/Photosynthesis/CLM50_Tech_Note_Photosynthesis:equation-9.5}}\begin{equation}\label{equation:tech_note/Photosynthesis/CLM50_Tech_Note_Photosynthesis:9.5}
\begin{split}A_{j} =\left\{\begin{array}{l} {\frac{J_{x}\left(c_{i} -\Gamma _{\*} \right)}{4c_{i} +8\Gamma _{\*} } \qquad \qquad {\rm for\; C}_{{\rm 3}} {\rm \; plants}} \\ {\alpha (4.6\phi )\qquad \qquad {\rm for\; C}_{{\rm 4}} {\rm \; plants}} \end{array}\right\}\qquad \qquad c_{i} -\Gamma _{\*} \ge 0.\end{split}
\end{equation}
The product-limited rate of carboxylation for C$_{\text{3}}$ plants
and the PEP carboxylase-limited rate of carboxylation for
C$_{\text{4}}$ plants \(A_{p}\)  (\(\mu\) mol
CO$_{\text{2}}$ m$^{\text{-2}}$ s$^{\text{-1}}$) is
\phantomsection\label{\detokenize{tech_note/Photosynthesis/CLM50_Tech_Note_Photosynthesis:equation-9.6}}\begin{equation}\label{equation:tech_note/Photosynthesis/CLM50_Tech_Note_Photosynthesis:9.6}
\begin{split}A_{p} =\left\{\begin{array}{l} {3T_{p\qquad } \qquad \qquad {\rm for\; C}_{{\rm 3}} {\rm \; plants}} \\ {k_{p} \frac{c_{i} }{P_{atm} } \qquad \qquad \qquad {\rm for\; C}_{{\rm 4}} {\rm \; plants}} \end{array}\right\}.\end{split}
\end{equation}
In these equations, \(c_{i}\)  is the internal leaf
CO$_{\text{2}}$ partial pressure (Pa) and \(o_{i} =0.20P_{atm}\)
is the O$_{\text{2}}$ partial pressure (Pa). \(K_{c}\)  and
\(K_{o}\)  are the Michaelis-Menten constants (Pa) for
CO$_{\text{2}}$ and O$_{\text{2}}$. \(\Gamma _{\*}\)  (Pa) is
the CO$_{\text{2}}$ compensation point. \(V_{c\max }\)  is the
maximum rate of carboxylation (µmol m$^{\text{-2}}$
s$^{\text{-1}}$, Chapter \hyperref[\detokenize{tech_note/Photosynthetic_Capacity/CLM50_Tech_Note_Photosynthetic_Capacity:rst-photosynthetic-capacity}]{\ref{\detokenize{tech_note/Photosynthetic_Capacity/CLM50_Tech_Note_Photosynthetic_Capacity:rst-photosynthetic-capacity}}})
and \(J_{x}\) is the electron transport rate (µmol
m$^{\text{-2}}$ s$^{\text{-1}}$). \(T_{p}\)  is the triose
phosphate utilization rate (µmol m$^{\text{-2}}$ s$^{\text{-1}}$),
taken as \(T_{p} =0.167V_{c\max }\)  so that
\(A_{p} =0.5V_{c\max }\)  for C$_{\text{3}}$ plants (as in
{\hyperref[\detokenize{tech_note/References/CLM50_Tech_Note_References:collatzetal1992}]{\sphinxcrossref{\DUrole{std,std-ref}{Collatz et al. 1992}}}}). For C$_{\text{4}}$ plants, the light-limited
rate \(A_{j}\)  varies with \(\phi\)  in relation to the quantum
efficiency (\(\alpha =0.05\) mol CO$_{\text{2}}$
mol$^{\text{-1}}$ photon). \(\phi\)  is the absorbed
photosynthetically active radiation (W m$^{\text{-2}}$) (section \hyperref[\detokenize{tech_note/Radiative_Fluxes/CLM50_Tech_Note_Radiative_Fluxes:solar-fluxes}]{\ref{\detokenize{tech_note/Radiative_Fluxes/CLM50_Tech_Note_Radiative_Fluxes:solar-fluxes}}})
, which is converted to photosynthetic photon flux assuming 4.6
\(\mu\) mol photons per joule. \(k_{p}\)  is the initial slope
of C$_{\text{4}}$ CO$_{\text{2}}$ response curve.

For C$_{\text{3}}$ plants, the electron transport rate depends on the
photosynthetically active radiation absorbed by the leaf. A common
expression is the smaller of the two roots of the equation
\phantomsection\label{\detokenize{tech_note/Photosynthesis/CLM50_Tech_Note_Photosynthesis:equation-9.7}}\begin{equation}\label{equation:tech_note/Photosynthesis/CLM50_Tech_Note_Photosynthesis:9.7}
\begin{split}\Theta _{PSII} J_{x}^{2} -\left(I_{PSII} +J_{\max } \right)J_{x}+I_{PSII} J_{\max } =0\end{split}
\end{equation}
where \(J_{\max }\)  is the maximum potential rate of electron
transport (\(\mu\)mol m$^{\text{-2}}$ s$^{\text{-1}}$, Chapter \hyperref[\detokenize{tech_note/Photosynthetic_Capacity/CLM50_Tech_Note_Photosynthetic_Capacity:rst-photosynthetic-capacity}]{\ref{\detokenize{tech_note/Photosynthetic_Capacity/CLM50_Tech_Note_Photosynthetic_Capacity:rst-photosynthetic-capacity}}}),
\(I_{PSII}\)  is the light utilized in electron transport by
photosystem II (µmol m$^{\text{-2}}$ s$^{\text{-1}}$), and
\(\Theta _{PSII}\)  is a curvature parameter. For a given amount of
photosynthetically active radiation absorbed by a leaf (\(\phi\),  W
m$^{\text{-2}}$), converted to photosynthetic photon flux density
with 4.6 \(\mu\)mol J$^{\text{-1}}$, the light utilized in
electron transport is
\phantomsection\label{\detokenize{tech_note/Photosynthesis/CLM50_Tech_Note_Photosynthesis:equation-9.8}}\begin{equation}\label{equation:tech_note/Photosynthesis/CLM50_Tech_Note_Photosynthesis:9.8}
\begin{split}I_{PSII} =0.5\Phi _{PSII} (4.6\phi )\end{split}
\end{equation}
where \(\Phi _{PSII}\)  is the quantum yield of photosystem II, and
the term 0.5 arises because one photon is absorbed by each of the two
photosystems to move one electron. Parameter values are
\(\Theta _{PSII}\) = 0.7 and \(\Phi _{PSII}\) = 0.85. In
calculating \(A_{j}\)  (for both C$_{\text{3}}$ and
C$_{\text{4}}$ plants), \(\phi =\phi ^{sun}\)  for sunlit leaves
and \(\phi =\phi ^{sha}\)  for shaded leaves.

The model uses co-limitation as described by {\hyperref[\detokenize{tech_note/References/CLM50_Tech_Note_References:collatzetal1991}]{\sphinxcrossref{\DUrole{std,std-ref}{Collatz et al. (1991, 1992)}}}}. The actual gross photosynthesis rate, \(A\), is given by the
smaller root of the equations
\phantomsection\label{\detokenize{tech_note/Photosynthesis/CLM50_Tech_Note_Photosynthesis:equation-9.9}}\begin{equation}\label{equation:tech_note/Photosynthesis/CLM50_Tech_Note_Photosynthesis:9.9}
\begin{split}\begin{array}{rcl} {\Theta _{cj} A_{i}^{2} -\left(A_{c} +A_{j} \right)A_{i} +A_{c} A_{j} } & {=} & {0} \\ {\Theta _{ip} A^{2} -\left(A_{i} +A_{p} \right)A+A_{i} A_{p} } & {=} & {0} \end{array} .\end{split}
\end{equation}
Values are \(\Theta _{cj} =0.98\) and \(\Theta _{ip} =0.95\) for
C$_{\text{3}}$ plants; and \(\Theta _{cj} =0.80\)and
\(\Theta _{ip} =0.95\) for C$_{\text{4}}$ plants.
\(A_{i}\) is the intermediate co-limited photosynthesis.
\(A_{n} =A-R_{d}\) .

The parameters \(K_{c}\), \(K_{o}\), and \(\Gamma\)
depend on temperature. Values at 25 $^{\text{o}}$ C are
\(K_{c25} ={\rm 4}0{\rm 4}.{\rm 9}\times 10^{-6} P_{atm}\),
\(K_{o25} =278.4\times 10^{-3} P_{atm}\), and
\(\Gamma _{25} {\rm =42}.75\times 10^{-6} P_{atm}\).
\(V_{c\max }\), \(J_{\max }\), \(T_{p}\), \(k_{p}\),
and \(R_{d}\) also vary with temperature. Parameter values at 25
$^{\text{o}}$C are calculated from \(V_{c\max }\) at 25
$^{\text{o}}$C: \(J_{\max 25} =1.97V_{c\max 25}\),
\(T_{p25} =0.167V_{c\max 25}\), and
\(R_{d25} =0.015V_{c\max 25}\) (C$_{\text{3}}$) and
\(R_{d25} =0.025V_{c\max 25}\) (C$_{\text{4}}$). For
C$_{\text{4}}$ plants, \(k_{p25} =20000\; V_{c\max 25}\).
However, when the biogeochemistry is active, \(R_{d25}\)  is
calculated from leaf nitrogen as \(R_{d25} = 0.2577LNC_{a}\),
where \(LNC_{a}\) is the area-based leaf nitrogen concentration
(g N m$^{\text{-2}}$ leaf area, Chapter \hyperref[\detokenize{tech_note/Photosynthetic_Capacity/CLM50_Tech_Note_Photosynthetic_Capacity:rst-photosynthetic-capacity}]{\ref{\detokenize{tech_note/Photosynthetic_Capacity/CLM50_Tech_Note_Photosynthetic_Capacity:rst-photosynthetic-capacity}}}),
and 0.2577 \(\mu\)mol CO$_{\text{2}}$ g$^{\text{-1}}$ N s$^{\text{-1}}$ is the base respiration rate.
{[}this doesn’t look correct based on the code, which lists two options; verify with Rosie{]}
The parameters \(V_{c\max 25}\),
\(J_{\max 25}\), \(T_{p25}\), \(k_{p25}\), and
\(R_{d25}\) are scaled over the canopy for sunlit and shaded leaves
(section \hyperref[\detokenize{tech_note/Photosynthesis/CLM50_Tech_Note_Photosynthesis:canopy-scaling}]{\ref{\detokenize{tech_note/Photosynthesis/CLM50_Tech_Note_Photosynthesis:canopy-scaling}}}). In C$_{\text{3}}$ plants, these are adjusted for leaf temperature,
\(T_{v}\) (K), as:
\phantomsection\label{\detokenize{tech_note/Photosynthesis/CLM50_Tech_Note_Photosynthesis:equation-9.10}}\begin{equation}\label{equation:tech_note/Photosynthesis/CLM50_Tech_Note_Photosynthesis:9.10}
\begin{split}\begin{array}{rcl} {V_{c\max } } & {=} & {V_{c\max 25} \; f\left(T_{v} \right)f_{H} \left(T_{v} \right)} \\ {J_{\max } } & {=} & {J_{\max 25} \; f\left(T_{v} \right)f_{H} \left(T_{v} \right)} \\ {T_{p} } & {=} & {T_{p25} \; f\left(T_{v} \right)f_{H} \left(T_{v} \right)} \\ {R_{d} } & {=} & {R_{d25} \; f\left(T_{v} \right)f_{H} \left(T_{v} \right)} \\ {K_{c} } & {=} & {K_{c25} \; f\left(T_{v} \right)} \\ {K_{o} } & {=} & {K_{o25} \; f\left(T_{v} \right)} \\ {\Gamma } & {=} & {\Gamma _{25} \; f\left(T_{v} \right)} \end{array}\end{split}
\end{equation}
with
\phantomsection\label{\detokenize{tech_note/Photosynthesis/CLM50_Tech_Note_Photosynthesis:equation-9.11}}\begin{equation}\label{equation:tech_note/Photosynthesis/CLM50_Tech_Note_Photosynthesis:9.11}
\begin{split}f\left(T_{v} \right)=\; \exp \left[\frac{\Delta H_{a} }{298.15\times 0.001R_{gas} } \left(1-\frac{298.15}{T_{v} } \right)\right]\end{split}
\end{equation}
and
\phantomsection\label{\detokenize{tech_note/Photosynthesis/CLM50_Tech_Note_Photosynthesis:equation-9.12}}\begin{equation}\label{equation:tech_note/Photosynthesis/CLM50_Tech_Note_Photosynthesis:9.12}
\begin{split}f_{H} \left(T_{v} \right)=\frac{1+\exp \left(\frac{298.15\Delta S-\Delta H_{d} }{298.15\times 0.001R_{gas} } \right)}{1+\exp \left(\frac{\Delta ST_{v} -\Delta H_{d} }{0.001R_{gas} T_{v} } \right)}  .\end{split}
\end{equation}
\hyperref[\detokenize{tech_note/Photosynthesis/CLM50_Tech_Note_Photosynthesis:table-temperature-dependence-parameters-for-c3-photosynthesis}]{Table \ref{\detokenize{tech_note/Photosynthesis/CLM50_Tech_Note_Photosynthesis:table-temperature-dependence-parameters-for-c3-photosynthesis}}}
lists parameter values for \(\Delta H_{a}\)  and
\(\Delta H_{d}\) . \(\Delta S\) is calculated
separately for \(V_{c\max }\) and \(J_{max }\)
to allow for temperature acclimation of photosynthesis (see equation \eqref{equation:tech_note/Photosynthesis/CLM50_Tech_Note_Photosynthesis:9.16}),
and \(\Delta S\) is 490 J mol $^{\text{-1}}$ K $^{\text{-1}}$ for \(R_d\)
({\hyperref[\detokenize{tech_note/References/CLM50_Tech_Note_References:bonanetal2011}]{\sphinxcrossref{\DUrole{std,std-ref}{Bonan et al. 2011}}}}, {\hyperref[\detokenize{tech_note/References/CLM50_Tech_Note_References:lombardozzietal2015}]{\sphinxcrossref{\DUrole{std,std-ref}{Lombardozzi et al. 2015}}}}).
Because \(T_{p}\)  as implemented here varies with
\(V_{c\max }\) , \(T_{p}\) uses the same temperature parameters as
\(V_{c\max}\) . For C$_{\text{4}}$ plants,
\phantomsection\label{\detokenize{tech_note/Photosynthesis/CLM50_Tech_Note_Photosynthesis:equation-9.13}}\begin{equation}\label{equation:tech_note/Photosynthesis/CLM50_Tech_Note_Photosynthesis:9.13}
\begin{split}\begin{array}{l} {V_{c\max } =V_{c\max 25} \left[\frac{Q_{10} ^{(T_{v} -298.15)/10} }{f_{H} \left(T_{v} \right)f_{L} \left(T_{v} \right)} \right]} \\ {f_{H} \left(T_{v} \right)=1+\exp \left[s_{1} \left(T_{v} -s_{2} \right)\right]} \\ {f_{L} \left(T_{v} \right)=1+\exp \left[s_{3} \left(s_{4} -T_{v} \right)\right]} \end{array}\end{split}
\end{equation}
with \(Q_{10} =2\),
\(s_{1} =0.3\)K$^{\text{-1}}$
\(s_{2} =313.15\) K,
\(s_{3} =0.2\)K$^{\text{-1}}$, and \(s_{4} =288.15\) K.
Additionally,
\phantomsection\label{\detokenize{tech_note/Photosynthesis/CLM50_Tech_Note_Photosynthesis:equation-9.14}}\begin{equation}\label{equation:tech_note/Photosynthesis/CLM50_Tech_Note_Photosynthesis:9.14}
\begin{split}R_{d} =R_{d25} \left\{\frac{Q_{10} ^{(T_{v} -298.15)/10} }{1+\exp \left[s_{5} \left(T_{v} -s_{6} \right)\right]} \right\}\end{split}
\end{equation}
with \(Q_{10} =2\), \(s_{5} =1.3\)
K$^{\text{-1}}$ and \(s_{6} =328.15\)K, and
\phantomsection\label{\detokenize{tech_note/Photosynthesis/CLM50_Tech_Note_Photosynthesis:equation-9.15}}\begin{equation}\label{equation:tech_note/Photosynthesis/CLM50_Tech_Note_Photosynthesis:9.15}
\begin{split}k_{p} =k_{p25} \, Q_{10} ^{(T_{v} -298.15)/10}\end{split}
\end{equation}
with \(Q_{10} =2\).


\begin{savenotes}\sphinxattablestart
\centering
\sphinxcapstartof{table}
\sphinxcaption{Temperature dependence parameters for C3 photosynthesis.}\label{\detokenize{tech_note/Photosynthesis/CLM50_Tech_Note_Photosynthesis:table-temperature-dependence-parameters-for-c3-photosynthesis}}\label{\detokenize{tech_note/Photosynthesis/CLM50_Tech_Note_Photosynthesis:id5}}
\sphinxaftercaption
\begin{tabulary}{\linewidth}[t]{|T|T|T|}
\hline
\sphinxstylethead{\sphinxstyletheadfamily 
Parameter
\unskip}\relax &\sphinxstylethead{\sphinxstyletheadfamily 
\(\Delta H_{a}\)  (J mol$^{\text{-1}}$)
\unskip}\relax &\sphinxstylethead{\sphinxstyletheadfamily 
\(\Delta H_{d}\)  (J mol$^{\text{-1}}$)
\unskip}\relax \\
\hline
\(V_{c\max }\)
&
72000
&
200000
\\
\hline
\(J_{\max }\)
&
50000
&
200000
\\
\hline
\(T_{p}\)
&
72000
&
200000
\\
\hline
\(R_{d}\)
&
46390
&
150650
\\
\hline
\(K_{c}\)
&
79430
&
\textendash{}
\\
\hline
\(K_{o}\)
&
36380
&
\textendash{}
\\
\hline
\(\Gamma _{\*}\)
&
37830
&
\textendash{}
\\
\hline
\end{tabulary}
\par
\sphinxattableend\end{savenotes}

In the model, acclimation is
implemented as in {\hyperref[\detokenize{tech_note/References/CLM50_Tech_Note_References:kattgeknorr2007}]{\sphinxcrossref{\DUrole{std,std-ref}{Kattge and Knorr (2007)}}}}. In this parameterization,
\(V_{c\max }\) and \(J_{\max }\)  vary with the plant growth temperature. This is
achieved by allowing \(\Delta S\) to vary with growth temperature
according to
\phantomsection\label{\detokenize{tech_note/Photosynthesis/CLM50_Tech_Note_Photosynthesis:equation-9.16}}\begin{equation}\label{equation:tech_note/Photosynthesis/CLM50_Tech_Note_Photosynthesis:9.16}
\begin{split}\begin{array}{l} {\Delta S=668.39-1.07(T_{10} -T_{f} )\qquad \qquad {\rm for\; }V_{c\max } } \\ {\Delta S=659.70-0.75(T_{10} -T_{f} )\qquad \qquad {\rm for\; }J_{\max } } \end{array}\end{split}
\end{equation}
The effect is to cause the temperature optimum of \(V_{c\max }\)
and \(J_{\max }\)  to increase with warmer temperatures.
Additionally, the
ratio \(J_{\max 25} /V_{c\max 25}\)  at 25 $^{\text{o}}$C decreases with growth temperature as
\phantomsection\label{\detokenize{tech_note/Photosynthesis/CLM50_Tech_Note_Photosynthesis:equation-9.17}}\begin{equation}\label{equation:tech_note/Photosynthesis/CLM50_Tech_Note_Photosynthesis:9.17}
\begin{split}J_{\max 25} /V_{c\max 25} =2.59-0.035(T_{10} -T_{f} ).\end{split}
\end{equation}
In these acclimation functions, \(T_{10}\)  is the 10-day mean air
temperature (K) and \(T_{f}\)  is the freezing point of water (K).
For lack of data, \(T_{p}\)  acclimates similar to \(V_{c\max }\). Acclimation is restricted over the temperature
range \(T_{10} -T_{f} \ge 11\)$^{\text{o}}$C and \(T_{10} -T_{f} \le 35\)$^{\text{o}}$C.


\subsection{Canopy scaling}
\label{\detokenize{tech_note/Photosynthesis/CLM50_Tech_Note_Photosynthesis:canopy-scaling}}\label{\detokenize{tech_note/Photosynthesis/CLM50_Tech_Note_Photosynthesis:id3}}
\(V_{c\max 25}\)  is calculated separately for sunlit and shaded
leaves using an exponential profile to area-based leaf nitrogen
(\(LNC_{a}\), see Chapter \hyperref[\detokenize{tech_note/Photosynthetic_Capacity/CLM50_Tech_Note_Photosynthetic_Capacity:rst-photosynthetic-capacity}]{\ref{\detokenize{tech_note/Photosynthetic_Capacity/CLM50_Tech_Note_Photosynthetic_Capacity:rst-photosynthetic-capacity}}} ),
as in {\hyperref[\detokenize{tech_note/References/CLM50_Tech_Note_References:bonanetal2011}]{\sphinxcrossref{\DUrole{std,std-ref}{Bonan et al. (2011)}}}}. \(V_{c\max 25}\)  at
cumulative leaf area index \(x\) from the canopy top scales directly
with \(LNC_{a}\) , which decreases exponentially with greater
cumulative leaf area, so that {[}Verify with Rosie- different based on Vcmax option 0, 3, \& 4{]}
\phantomsection\label{\detokenize{tech_note/Photosynthesis/CLM50_Tech_Note_Photosynthesis:equation-9.18}}\begin{equation}\label{equation:tech_note/Photosynthesis/CLM50_Tech_Note_Photosynthesis:9.18}
\begin{split}V_{c\; \max 25}^{} \left(x\right)=V_{c\; \max 25}^{} \left(0\right)e^{-K_{n} x}\end{split}
\end{equation}
where \(V_{c\; \max 25}^{} \left(0\right)\) is defined at the top of
the canopy using \(SLA_{0}\), which is the specific leaf area at
the canopy top and \(K_{n}\)  is the decay
coefficient for nitrogen. The canopy integrated value for sunlit and
shaded leaves is
\phantomsection\label{\detokenize{tech_note/Photosynthesis/CLM50_Tech_Note_Photosynthesis:equation-9.20}}\begin{equation}\label{equation:tech_note/Photosynthesis/CLM50_Tech_Note_Photosynthesis:9.20}
\begin{split}\begin{array}{rcl} {V_{c\; \max 25}^{sun} } & {=} & {\int _{0}^{L}V_{c\; \max 25}^{} \left(x\right)f_{sun} \left(x\right)\,  dx} \\ {} & {=} & {V_{c\; \max 25}^{} \left(0\right)\left[1-e^{-\left(K_{n} +K\right)L} \right]\frac{1}{K_{n} +K} } \end{array}\end{split}
\end{equation}\phantomsection\label{\detokenize{tech_note/Photosynthesis/CLM50_Tech_Note_Photosynthesis:equation-9.21}}\begin{equation}\label{equation:tech_note/Photosynthesis/CLM50_Tech_Note_Photosynthesis:9.21}
\begin{split}\begin{array}{rcl} {V_{c\; \max 25}^{sha} } & {=} & {\int _{0}^{L}V_{c\; \max 25}^{} \left(x\right)\left[1-f_{sun} \left(x\right)\right] \, dx} \\ {} & {=} & {V_{c\; \max 25}^{} \left(0\right)\left\{\left[1-e^{-K_{n} L} \right]\frac{1}{K_{n} } -\left[1-e^{-\left(K_{n} +K\right)L} \right]\frac{1}{K_{n} +K} \right\}} \end{array}\end{split}
\end{equation}
and the average value for the sunlit and shaded leaves is
\phantomsection\label{\detokenize{tech_note/Photosynthesis/CLM50_Tech_Note_Photosynthesis:equation-9.22}}\begin{equation}\label{equation:tech_note/Photosynthesis/CLM50_Tech_Note_Photosynthesis:9.22}
\begin{split}\bar{V}_{c\; \max 25}^{sun} ={V_{c\; \max 25}^{sun} \mathord{\left/ {\vphantom {V_{c\; \max 25}^{sun}  L^{sun} }} \right. \kern-\nulldelimiterspace} L^{sun} }\end{split}
\end{equation}\phantomsection\label{\detokenize{tech_note/Photosynthesis/CLM50_Tech_Note_Photosynthesis:equation-9.23}}\begin{equation}\label{equation:tech_note/Photosynthesis/CLM50_Tech_Note_Photosynthesis:9.23}
\begin{split}\bar{V}_{c\; \max 25}^{sha} ={V_{c\; \max 25}^{sha} \mathord{\left/ {\vphantom {V_{c\; \max 25}^{sha}  L^{sha} }} \right. \kern-\nulldelimiterspace} L^{sha} } .\end{split}
\end{equation}
This integration is over all leaf area (\(L\)) with
\(f_{sun} (x)=\exp \left(-Kx\right)\) and \(K\) the direct beam
extinction coefficient (equation \eqref{equation:tech_note/Radiative_Fluxes/CLM50_Tech_Note_Radiative_Fluxes:4.8} in chapter \hyperref[\detokenize{tech_note/Radiative_Fluxes/CLM50_Tech_Note_Radiative_Fluxes:rst-radiative-fluxes}]{\ref{\detokenize{tech_note/Radiative_Fluxes/CLM50_Tech_Note_Radiative_Fluxes:rst-radiative-fluxes}}}). Photosynthetic parameters
\(J_{\max 25}\) , \(T_{p25}\) , \(k_{p25}\), and
\(R_{d25}\)  scale similarly.

The model uses \(K_{n} =0.30\) to match an explicit multi-layer canopy, as in
{\hyperref[\detokenize{tech_note/References/CLM50_Tech_Note_References:bonanetal2012}]{\sphinxcrossref{\DUrole{std,std-ref}{Bonan et al. (2012)}}}}.
The value \(K_{n} = 0.11\) chosen by {\hyperref[\detokenize{tech_note/References/CLM50_Tech_Note_References:bonanetal2011}]{\sphinxcrossref{\DUrole{std,std-ref}{Bonan et al. (2011)}}}} is
consistent with observationally-derived estimates for forests, mostly
tropical, and provides a gradient in V$_{\text{cmax}}$ similar to
the original CLM4 specific leaf area scaling. However,
{\hyperref[\detokenize{tech_note/References/CLM50_Tech_Note_References:bonanetal2012}]{\sphinxcrossref{\DUrole{std,std-ref}{Bonan et al. (2012)}}}} showed that the sunlit/shaded canopy parameterization does not
match an explicit multi-layer canopy parameterization. The discrepancy
arises from absorption of scattered radiation by shaded leaves and can
be tuned out with higher \(K_{n}\).


\subsection{Numerical implementation}
\label{\detokenize{tech_note/Photosynthesis/CLM50_Tech_Note_Photosynthesis:numerical-implementation}}\label{\detokenize{tech_note/Photosynthesis/CLM50_Tech_Note_Photosynthesis:numerical-implementation-photosynthesis}}
The CO$_{\text{2}}$ partial pressure at the leaf surface,
\(c_{s}\)  (Pa), and the vapor pressure at the leaf surface,
\(e_{s}\)  (Pa), needed for the stomatal resistance model in
equation \eqref{equation:tech_note/Photosynthesis/CLM50_Tech_Note_Photosynthesis:9.1}, and the internal leaf CO$_{\text{2}}$ partial pressure
\(c_{i}\)  (Pa), needed for the photosynthesis model in equations \eqref{equation:tech_note/Photosynthesis/CLM50_Tech_Note_Photosynthesis:9.3}-\eqref{equation:tech_note/Photosynthesis/CLM50_Tech_Note_Photosynthesis:9.5},
are calculated assuming there is negligible capacity to store
CO$_{\text{2}}$ and water vapor at the leaf surface so that
\phantomsection\label{\detokenize{tech_note/Photosynthesis/CLM50_Tech_Note_Photosynthesis:equation-9.31}}\begin{equation}\label{equation:tech_note/Photosynthesis/CLM50_Tech_Note_Photosynthesis:9.31}
\begin{split}A_{n} =\frac{c_{a} -c_{i} }{\left(1.4r_{b} +1.6r_{s} \right)P_{atm} } =\frac{c_{a} -c_{s} }{1.4r_{b} P_{atm} } =\frac{c_{s} -c_{i} }{1.6r_{s} P_{atm} }\end{split}
\end{equation}
and the transpiration fluxes are related as
\phantomsection\label{\detokenize{tech_note/Photosynthesis/CLM50_Tech_Note_Photosynthesis:equation-9.32}}\begin{equation}\label{equation:tech_note/Photosynthesis/CLM50_Tech_Note_Photosynthesis:9.32}
\begin{split}\frac{e_{a} -e_{i} }{r_{b} +r_{s} } =\frac{e_{a} -e_{s} }{r_{b} } =\frac{e_{s} -e_{i} }{r_{s} }\end{split}
\end{equation}
where \(r_{b}\)  is leaf boundary layer resistance (s
m$^{\text{2}}$ \(\mu\) mol$^{\text{-1}}$) (section \hyperref[\detokenize{tech_note/Fluxes/CLM50_Tech_Note_Fluxes:sensible-and-latent-heat-fluxes-and-temperature-for-vegetated-surfaces}]{\ref{\detokenize{tech_note/Fluxes/CLM50_Tech_Note_Fluxes:sensible-and-latent-heat-fluxes-and-temperature-for-vegetated-surfaces}}}), the
terms 1.4 and 1.6 are the ratios of diffusivity of CO$_{\text{2}}$ to
H$_{\text{2}}$O for the leaf boundary layer resistance and stomatal
resistance,
\(c_{a} ={\rm CO}_{{\rm 2}} \left({\rm mol\; mol}^{{\rm -1}} \right)\), \(P_{atm}\)
is the atmospheric CO$_{\text{2}}$ partial pressure (Pa) calculated
from CO$_{\text{2}}$ concentration (ppmv), \(e_{i}\)  is the
saturation vapor pressure (Pa) evaluated at the leaf temperature
\(T_{v}\) , and \(e_{a}\)  is the vapor pressure of air (Pa).
The vapor pressure of air in the plant canopy \(e_{a}\)  (Pa) is
determined from
\phantomsection\label{\detokenize{tech_note/Photosynthesis/CLM50_Tech_Note_Photosynthesis:equation-9.33}}\begin{equation}\label{equation:tech_note/Photosynthesis/CLM50_Tech_Note_Photosynthesis:9.33}
\begin{split}e_{a} =\frac{P_{atm} q_{s} }{0.622}\end{split}
\end{equation}
where \(q_{s}\)  is the specific humidity of canopy air (kg
kg$^{\text{-1}}$, section \hyperref[\detokenize{tech_note/Fluxes/CLM50_Tech_Note_Fluxes:sensible-and-latent-heat-fluxes-and-temperature-for-vegetated-surfaces}]{\ref{\detokenize{tech_note/Fluxes/CLM50_Tech_Note_Fluxes:sensible-and-latent-heat-fluxes-and-temperature-for-vegetated-surfaces}}}).
Equations and are solved for
\(c_{s}\)  and \(e_{s}\)
\phantomsection\label{\detokenize{tech_note/Photosynthesis/CLM50_Tech_Note_Photosynthesis:equation-9.34}}\begin{equation}\label{equation:tech_note/Photosynthesis/CLM50_Tech_Note_Photosynthesis:9.34}
\begin{split}c_{s} =c_{a} -1.4r_{b} P_{atm} A_{n}\end{split}
\end{equation}\phantomsection\label{\detokenize{tech_note/Photosynthesis/CLM50_Tech_Note_Photosynthesis:equation-9.35}}\begin{equation}\label{equation:tech_note/Photosynthesis/CLM50_Tech_Note_Photosynthesis:9.35}
\begin{split}e_{s} =\frac{e_{a} r_{s} +e_{i} r_{b} }{r_{b} +r_{s} }\end{split}
\end{equation}
Substitution of equation \eqref{equation:tech_note/Photosynthesis/CLM50_Tech_Note_Photosynthesis:9.35} into equation \eqref{equation:tech_note/Photosynthesis/CLM50_Tech_Note_Photosynthesis:9.1} gives an expression for stomatal
resistance (\(r_{s}\) ) as a function of photosynthesis
(\(A_{n}\) ), given here in terms of conductance with
\(g_{s} =1/r_{s}\)  and \(g_{b} =1/r_{b}\)
\phantomsection\label{\detokenize{tech_note/Photosynthesis/CLM50_Tech_Note_Photosynthesis:equation-9.36}}\begin{equation}\label{equation:tech_note/Photosynthesis/CLM50_Tech_Note_Photosynthesis:9.36}
\begin{split}c_{s} g_{s}^{2} +\left[c_{s} \left(g_{b} -b\right)-m{\it A}_{n} P_{atm} \right]g_{s} -g_{b} \left[c_{s} b+mA_{n} P_{atm} {e_{a} \mathord{\left/ {\vphantom {e_{a}  e_{\*} \left(T_{v} \right)}} \right. \kern-\nulldelimiterspace} e_{\*} \left(T_{v} \right)} \right]=0.\end{split}
\end{equation}
Stomatal conductance is the larger of the two roots that satisfy the
quadratic equation. Values for \(c_{i}\)  are given by
\phantomsection\label{\detokenize{tech_note/Photosynthesis/CLM50_Tech_Note_Photosynthesis:equation-9.37}}\begin{equation}\label{equation:tech_note/Photosynthesis/CLM50_Tech_Note_Photosynthesis:9.37}
\begin{split}c_{i} =c_{a} -\left(1.4r_{b} +1.6r_{s} \right)P_{atm} A{}_{n}\end{split}
\end{equation}
The equations for \(c_{i}\) , \(c_{s}\) , \(r_{s}\) , and
\(A_{n}\)  are solved iteratively until \(c_{i}\)  converges.
{\hyperref[\detokenize{tech_note/References/CLM50_Tech_Note_References:sunetal2012}]{\sphinxcrossref{\DUrole{std,std-ref}{Sun et al. (2012)}}}} pointed out that the CLM4 numerical approach does not
always converge. Therefore, the model uses a hybrid algorithm that
combines the secant method and Brent’s method to solve for
\(c_{i}\) . The equation set is solved separately for sunlit
(\(A_{n}^{sun}\) , \(r_{s}^{sun}\) ) and shaded
(\(A_{n}^{sha}\) , \(r_{s}^{sha}\) ) leaves.

The model has an optional (though not supported) multi-layer canopy, as
described by {\hyperref[\detokenize{tech_note/References/CLM50_Tech_Note_References:bonanetal2012}]{\sphinxcrossref{\DUrole{std,std-ref}{Bonan et al. (2012)}}}}. The multi-layer model is only intended
to address the non-linearity of light profiles, photosynthesis, and
stomatal conductance in the plant canopy. In the multi-layer canopy,
sunlit (\(A_{n}^{sun}\) , \(r_{s}^{sun}\) ) and shaded
(\(A_{n}^{sha}\) , \(r_{s}^{sha}\) ) leaves are explicitly
resolved at depths in the canopy using a light profile (Chapter \hyperref[\detokenize{tech_note/Radiative_Fluxes/CLM50_Tech_Note_Radiative_Fluxes:rst-radiative-fluxes}]{\ref{\detokenize{tech_note/Radiative_Fluxes/CLM50_Tech_Note_Radiative_Fluxes:rst-radiative-fluxes}}}). In
this case, \(V_{c\max 25}\)  is not integrated over the canopy, but
is instead given explicitly for each canopy layer using equation \eqref{equation:tech_note/Photosynthesis/CLM50_Tech_Note_Photosynthesis:9.18}. This
also uses the {\hyperref[\detokenize{tech_note/References/CLM50_Tech_Note_References:lloydetal2010}]{\sphinxcrossref{\DUrole{std,std-ref}{Lloyd et al. (2010)}}}} relationship whereby
K$_{\text{n}}$ scales with V$_{\text{cmax}}$ as
\phantomsection\label{\detokenize{tech_note/Photosynthesis/CLM50_Tech_Note_Photosynthesis:equation-9.38}}\begin{equation}\label{equation:tech_note/Photosynthesis/CLM50_Tech_Note_Photosynthesis:9.38}
\begin{split}K_{n} =\exp \left(0.00963V_{c\max } -2.43\right)\end{split}
\end{equation}
such that higher values of V$_{\text{cmax}}$ imply steeper declines
in photosynthetic capacity through the canopy with respect to cumulative
leaf area.


\section{Photosynthetic Capacity}
\label{\detokenize{tech_note/Photosynthetic_Capacity/CLM50_Tech_Note_Photosynthetic_Capacity:rst-photosynthetic-capacity}}\label{\detokenize{tech_note/Photosynthetic_Capacity/CLM50_Tech_Note_Photosynthetic_Capacity::doc}}\label{\detokenize{tech_note/Photosynthetic_Capacity/CLM50_Tech_Note_Photosynthetic_Capacity:photosynthetic-capacity}}
The photosynthetic capacity is represented by two key parameters: 1) the maximum rate of carboxylation at
25 $^{\text{o}}$C, \(V_{\text{c,max25}}\); and 2) the maximum rate of electron transport at
25 $^{\text{o}}$C, \(J_{\text{max25}}\) . They are predicted by a mechanistic model of leaf
utilization of nitrogen for assimilation (LUNA V1.0) (Ali et al. 2016) based on an optimality hypothesis to nitrogen allocation
among light capture, electron transport, carboxylation, respiration and storage.
Specifically, the model allocates the nitrogen by maximizing the daily
net photosynthetic carbon gain under following two key assumptions:
\begin{itemize}
\item {} 
nitrogen allocated for light capture, electron transport and carboxylation are co-limiting;

\item {} 
respiratory nitrogen is allocated to maintain dark respiration determined by \(V_{\text{c,max}}\).

\end{itemize}

Compared to traditional photosynthetic capacity models, a key advantage of LUNA is that the model is able to predict the potential
acclimation of photosynthetic capacities at different environmental conditions as determined by temperature, radiation,
CO $_{\text{2}}$ concentrations, day length, and humidity.


\subsection{Model inputs and parameter estimations}
\label{\detokenize{tech_note/Photosynthetic_Capacity/CLM50_Tech_Note_Photosynthetic_Capacity:model-inputs-and-parameter-estimations}}\label{\detokenize{tech_note/Photosynthetic_Capacity/CLM50_Tech_Note_Photosynthetic_Capacity:id1}}
The LUNA model includes the following four unitless parameters:
\begin{itemize}
\item {} 
\(J_{maxb0}\) , which specifies the baseline proportion of nitrogen allocated for electron transport;

\item {} 
\(J_{maxb1}\) , which determines response of electron transport rate to light availability;

\item {} 
\(t_{c,j0}\) , which defines the baseline ratio of Rubisco-limited rate to light-limited rate;

\item {} 
\(H\) , which determines the response of electron transport rate to relative humidity.

\end{itemize}

The above four parameters are estimated by fitting the LUNA model to a global compilation of \textgreater{}800 obervations
located at different biomes, canopy locations, and time of the year from 1993-2013 (Ali et al 2015). The model inputs
are area-based leaf nitrogen content, leaf mass per unit leaf area and the driving environmental conditions (average of past 10 days)
including temperature, CO $_{\text{2}}$ concentrations, daily mean and maximum radiation, relative humidity and day length.
The estimated values in CLM5 for the listed parameters are 0.0311, 0.1745, 0.8054, and 6.0999, repectively. In LUNA V1.0, the estimated
parameter values are for C3 natural vegetations. In view that potentially large differences in photosythetic capacity could exist
between crops and natural vegetations due to human selection and genetic modifications, in CLM5,
the LUNA model are used only for C3 natural vegetations. The photosynthetic capacity for crops and C4 plants are thus
still kept the same as CLM4.5. Namely, it is estimated based on the leaf nitrogen content, fixed RUBISCO allocations for
\(V_{c\max 25}\) and an adjusting factor to account for the impact of day length. In CLM5, the model simulates both sun-lit and shaded leaves;
however, because the sun-lit and shaded leaves can changes through the day based on the sun angles,
we do not differentiate the photosynthetic capacity difference for sun-lit or shaded leaves.


\subsection{Model structure}
\label{\detokenize{tech_note/Photosynthetic_Capacity/CLM50_Tech_Note_Photosynthetic_Capacity:model-structure}}\label{\detokenize{tech_note/Photosynthetic_Capacity/CLM50_Tech_Note_Photosynthetic_Capacity:id2}}

\subsubsection{Plant Nitrogen}
\label{\detokenize{tech_note/Photosynthetic_Capacity/CLM50_Tech_Note_Photosynthetic_Capacity:plant-nitrogen}}
The structure of the LUNA model is adapted from Xu et al.(2012), where the plant nitrogen at the leaf level ( \(\text{LNC}_{a}\);  gN/ m $^{\text{2}}$ leaf) is divided into
four pools: structural nitrogen( \(N_{\text{str}}\);  gN/m $^{\text{2}}$ leaf),
photosynthetic nitrogen ( \(N_{\text{psn}}\); gN/ m:sup:\sphinxtitleref{2} leaf),
storage nitrogen( \(N_{\text{store}}\);  gN/m $^{\text{2}}$ leaf),
and respiratory nitrogen ( \(N_{\text{resp}}\);  gN/m $^{\text{2}}$ leaf).
Namely,
\phantomsection\label{\detokenize{tech_note/Photosynthetic_Capacity/CLM50_Tech_Note_Photosynthetic_Capacity:equation-10.1)}}\begin{equation}\label{equation:tech_note/Photosynthetic_Capacity/CLM50_Tech_Note_Photosynthetic_Capacity:10.1)}
\begin{split} \text{LNC}_{a} = N_{\text{psn}} + N_{\text{str}}+ N_{\text{store}} + N_{\text{resp}}.\end{split}
\end{equation}
The photosynthetic nitrogen, \(N_{\text{psn}}\), is further divided into
nitrogen for light capture ( \(N_{\text{lc}}\);   gN/m $^{\text{2}}$ leaf),
nitrogen for electron transport ( \(N_{\text{et}}\);  gN/m $^{\text{2}}$ leaf),
and nitrogen for carboxylation ( \(N_{\text{cb}}\);  gN/m $^{\text{2}}$ leaf).
Namely,
\phantomsection\label{\detokenize{tech_note/Photosynthetic_Capacity/CLM50_Tech_Note_Photosynthetic_Capacity:equation-10.2)}}\begin{equation}\label{equation:tech_note/Photosynthetic_Capacity/CLM50_Tech_Note_Photosynthetic_Capacity:10.2)}
\begin{split} N_{\text{psn}} =N_{\text{et}} + N_{\text{cb}} + N_{\text{lc}}.\end{split}
\end{equation}
The structural nitrogen,  \(N_{\text{str}}\), is calculated as the
multiplication of leaf mass per unit area (LMA: g biomass/m $^{\text{2}}$ leaf), and the structural nitrogen content (SNC; gN/g biomass). Namely,
\phantomsection\label{\detokenize{tech_note/Photosynthetic_Capacity/CLM50_Tech_Note_Photosynthetic_Capacity:equation-10.3)}}\begin{equation}\label{equation:tech_note/Photosynthetic_Capacity/CLM50_Tech_Note_Photosynthetic_Capacity:10.3)}
\begin{split} N_{\text{str}} = \text{SNC} \cdot \text{LMA}\end{split}
\end{equation}
where SNC is set to be fixed at 0.002 (gN/g biomass), based on data on C:N ratio from dead wood (White etal.,2000).

We assume that plants optimize their nitrogen allocations (i.e., \(N_{\text{store}}\),  \(N_{\text{resp}}\),  \(N_{\text{lc}}\),  \(N_{\text{et}}\), \(N_{\text{cb}}\)) to maximize the photosynthetic carbon gain, defined as
the gross photosynthesis ( \(A\) ) minus the maintenance respiration for
photosynthetic enzymes ( \(R_{\text{psn}}\) ), under specific
environmental conditions and given plant’s strategy of leaf nitrogen
use. Namely, the solutions of nitrogen allocations \{ \(N_{\text{store}}\),  \(N_{\text{resp}}\),  \(N_{\text{lc}}\),  \(N_{\text{et}}\), \(N_{\text{cb}}\) \} can be estimated as follows,
\phantomsection\label{\detokenize{tech_note/Photosynthetic_Capacity/CLM50_Tech_Note_Photosynthetic_Capacity:equation-10.4)}}\begin{equation}\label{equation:tech_note/Photosynthetic_Capacity/CLM50_Tech_Note_Photosynthetic_Capacity:10.4)}
\begin{split}\left\{\hat{N}_{\text{{store}}}, \hat{N}_{\text{{resp}}},
  \hat{\mathrm{N}}_{\text{lc}}, \hat{N}_{\text{et}}, \hat{\mathrm{N}}_{\text{cb}}
\right\} = \underset{\mathrm{N}_{\text{store}}\,+\,\mathrm{N}_{\text{resp}}\,+\,\mathrm{N}_{\text{lc}}\,+\,\mathrm{N}_{\text{et}}\,+\,\mathrm{N}_{\text{cb}}\,<\text{FNC}_{\mathrm{a}}}{\text{argmax}} (A-R_{\text{psn}}),\end{split}
\end{equation}
where  \(\text{FNC}_{a}\) is the functional nitrogen content defined as the total leaf nitrogen content ( \(\text{LNC}_{a}\)) minus the structural nitrogen content ( \(N_{\text{str}}\) ).

The gross photosynthesis, \(A\), was calculated with a coupled leaf gas  exchange model based on the Farquhar et al. (1980) model of
photosynthesis and Ball\textendash{}Berry-type stomatal conductance model (Ball et al., 1987). The maintenance respiration for photosynthetic enzymes, \(R_{\text{psn}}\), is
calculated by the multiplication of total photosynthetic nitrogen ( \(N_{\text{psn}}\) ) and the maintenance respiration cost for photosynthetic enzymes.


\subsubsection{Maximum electron transport rate}
\label{\detokenize{tech_note/Photosynthetic_Capacity/CLM50_Tech_Note_Photosynthetic_Capacity:maximum-electron-transport-rate}}
In the LUNA model, the maximum electron transport rate
( \(J_{\text{max}}\); \({\mu} mol\)  electron / m $^{\text{-2}}$/s)
is simulated to have a baseline allocation of nitrogen and additional
nitrogen allocation to change depending on the average daytime
photosynthetic active radiation (PAR;  \({\mu} mol\)  electron / m $^{\text{-2}}$/s), day length (hours) and air humidity.
Specifically, the LUNA model has
\phantomsection\label{\detokenize{tech_note/Photosynthetic_Capacity/CLM50_Tech_Note_Photosynthetic_Capacity:equation-10.5)}}\begin{equation}\label{equation:tech_note/Photosynthetic_Capacity/CLM50_Tech_Note_Photosynthetic_Capacity:10.5)}
\begin{split}J_{\text{{max}}} = J_{\text{max}0} + J_{\text{max}b1}
f\left(\text{day length} \right)f\left(\text{humidity}
\right)\alpha \text{PAR}\end{split}
\end{equation}
The baseline electron transport rate, \(J_{\text{max}0}\), is calculated as follows,
\phantomsection\label{\detokenize{tech_note/Photosynthetic_Capacity/CLM50_Tech_Note_Photosynthetic_Capacity:equation-10.6)}}\begin{equation}\label{equation:tech_note/Photosynthetic_Capacity/CLM50_Tech_Note_Photosynthetic_Capacity:10.6)}
\begin{split}J_{\text{max}0} = J_{\text{max}b0}{\text{FNC}}_{\mathrm{a}}{\text{NUE}}_{J_{\text{{max}}}}\end{split}
\end{equation}
where \(J_{\text{max}b0}\) (unitless) is the baseline proportion of nitrogen
allocated for electron transport rate.  \({\text{NUE}}_{J_{\text{{max}}}}\) ( \({\mu} mol\)  electron /s/g N)
is the nitrogen use efficiency of \(J_{\text{{max}}}\). \(J_{\text{max}b1}\) (unitless) is a coefficient determining the response of the electron
transport rate to amount of absorbed light (i.e., \(\alpha \text{PAR}\)).
\(f\left(\text{day length} \right)\) is a function specifies the impact of day
length (hours) on \(J_{\text{max}}\) in view that longer day length has been demonstrated by previous studies to alter \(V_{\mathrm{c}\text{max}25}\) and
\(J_{\text{max}25}\) (Bauerle et al., 2012; Comstock and Ehleringer, 1986) through photoperiod sensing and regulation (e.g. Song et al., 2013).
Following Bauerle et al. (2012), \(f\left(\text{day length} \right)\) is simulated as follows,
\phantomsection\label{\detokenize{tech_note/Photosynthetic_Capacity/CLM50_Tech_Note_Photosynthetic_Capacity:equation-10.7)}}\begin{equation}\label{equation:tech_note/Photosynthetic_Capacity/CLM50_Tech_Note_Photosynthetic_Capacity:10.7)}
\begin{split}f\left(\text{day length} \right) = \left(\frac{\text{day length}}{12} \right)^{2}.\end{split}
\end{equation}
\(f\left(\text{humidity} \right)\) represents the impact of air humitidy on
\(J_{\text{{max}}}\). We assume that higher humidity leads to higher
\(J_{\text{{max}}}\) with less water limiation on stomta opening and that low
relative humidity has a stronger impact on nitrogen allocation due to greater
water limitation. When relative humidity (RH; unitless) is too low, we assume
that plants are physiologically unable to reallocate nitrogen. We therefore
assume that there exists a critical value of relative humidity ( \(RH_{0} =
0.25\); unitless), below which there is no optimal nitrogen allocation. Based
on the above assumptions, we have
\phantomsection\label{\detokenize{tech_note/Photosynthetic_Capacity/CLM50_Tech_Note_Photosynthetic_Capacity:equation-10.8)}}\begin{equation}\label{equation:tech_note/Photosynthetic_Capacity/CLM50_Tech_Note_Photosynthetic_Capacity:10.8)}
\begin{split}f\left(\text{humidity}
\right) = \left(1-\mathrm{e}^{\left(-H
      \frac{\text{max}\left(\text{RH}-{\text{RH}}_{0}, 0 \right)}{1-\text{RH}_{0}} \right)} \right),\end{split}
\end{equation}
where \(H\) (unitless) specifies the impact of relative humidity on electron transport rate.

The efficiency of light energy absorption (unitless),  \(\alpha\), is calculated
depending on the amount of nitrogen allocated for light capture,
\(\mathrm{N}_{\text{lc}}\). Following Niinemets and Tenhunen (1997), the LUNA model has,
\phantomsection\label{\detokenize{tech_note/Photosynthetic_Capacity/CLM50_Tech_Note_Photosynthetic_Capacity:equation-10.9)}}\begin{equation}\label{equation:tech_note/Photosynthetic_Capacity/CLM50_Tech_Note_Photosynthetic_Capacity:10.9)}
\begin{split}\alpha =\frac{0.292}{1+\frac{0.076}{\mathrm{N}_{\text{lc}}C_{b}}}\end{split}
\end{equation}
where 0.292 is the conversion factor from photon to electron. \(C_{b}\)
is the conversion factor (1.78) from nitrogen to chlorophyll. After we
estimate \(J_{\text{{max}}}\), the actual electron transport rate with
the daily maximum radiation ( \(J_{x}\)) can be calculated using the
empirical expression of  leaf (1937),
\phantomsection\label{\detokenize{tech_note/Photosynthetic_Capacity/CLM50_Tech_Note_Photosynthetic_Capacity:equation-10.10)}}\begin{equation}\label{equation:tech_note/Photosynthetic_Capacity/CLM50_Tech_Note_Photosynthetic_Capacity:10.10)}
\begin{split}J_{x} = \frac{\alpha \text{PAR}_{\text{max}}} {\left(1 + \frac{\alpha^{2}{\text{PAR}}_{\text{{max}}}^{2}}{J_{\text{{max}}}^{2}}
  \right)^{0.5}}\end{split}
\end{equation}
where \(\text{PAR}_{\text{{max}}}\) ( \(\mu mol\)/m $^{\text{2}}$/s) is the
maximum photosynthetically active radiation during the day.


\subsubsection{Maximum rate of carboxylation}
\label{\detokenize{tech_note/Photosynthetic_Capacity/CLM50_Tech_Note_Photosynthetic_Capacity:maximum-rate-of-carboxylation}}
The maximum rate of carboxylation at 25 $^{\text{o}}$C varies with
foliage nitrogen concentration and specific leaf area and is calculated
as in Thornton and Zimmermann (2007). At 25ºC,
\phantomsection\label{\detokenize{tech_note/Photosynthetic_Capacity/CLM50_Tech_Note_Photosynthetic_Capacity:equation-10.11)}}\begin{equation}\label{equation:tech_note/Photosynthetic_Capacity/CLM50_Tech_Note_Photosynthetic_Capacity:10.11)}
\begin{split} V_{c\max 25} =N_{str} N_{cb} F_{NR} a_{R25}\end{split}
\end{equation}
where \(N_{str}\) is the area-based leaf nitrogen concentration (g N
m$^{\text{-2}}$ leaf area), \(N_{cb}\)  is the fraction of leaf
nitrogen in Rubisco (g N in Rubisco g$^{\text{-1}}$ N),
\(F_{NR} =7.16\) is the mass ratio of total Rubisco molecular mass
to nitrogen in Rubisco (g Rubisco g$^{\text{-1}}$ N in Rubisco), and
\(a_{R25} =60\) is the specific activity of Rubisco (µmol
CO$_{\text{2}}$ g$^{\text{-1}}$ Rubisco s$^{\text{-1}}$).

\(V_{c\max 25}\)  additionally varies with daylength (\(DYL\))
using the function \(f(DYL)\), which introduces seasonal variation
to \(V_{c\max }\)
\phantomsection\label{\detokenize{tech_note/Photosynthetic_Capacity/CLM50_Tech_Note_Photosynthetic_Capacity:equation-10.12)}}\begin{equation}\label{equation:tech_note/Photosynthetic_Capacity/CLM50_Tech_Note_Photosynthetic_Capacity:10.12)}
\begin{split} f\left(DYL\right)=\frac{\left(DYL\right)^{2} }{\left(DYL_{\max } \right)^{2} }\end{split}
\end{equation}
with \(0.01\le f\left(DYL\right)\le 1\). Daylength (seconds) is
given by
\phantomsection\label{\detokenize{tech_note/Photosynthetic_Capacity/CLM50_Tech_Note_Photosynthetic_Capacity:equation-10.13)}}\begin{equation}\label{equation:tech_note/Photosynthetic_Capacity/CLM50_Tech_Note_Photosynthetic_Capacity:10.13)}
\begin{split} DYL=2\times 13750.9871\cos ^{-1} \left[\frac{-\sin \left(lat\right)\sin \left(decl\right)}{\cos \left(lat\right)\cos \left(decl\right)} \right]\end{split}
\end{equation}
where \(lat\) (latitude) and \(decl\) (declination angle) are
from section \hyperref[\detokenize{tech_note/Surface_Albedos/CLM50_Tech_Note_Surface_Albedos:solar-zenith-angle}]{\ref{\detokenize{tech_note/Surface_Albedos/CLM50_Tech_Note_Surface_Albedos:solar-zenith-angle}}}. Maximum daylength (\(DYL_{\max }\) ) is calculated
similarly but using the maximum declination angle for present-day
orbital geometry (\(\pm\)23.4667º {[}\(\pm\)0.409571 radians{]},
positive for Northern Hemisphere latitudes and negative for Southern
Hemisphere).


\subsubsection{Implementation of Photosynthetic Capacity}
\label{\detokenize{tech_note/Photosynthetic_Capacity/CLM50_Tech_Note_Photosynthetic_Capacity:implementation-of-photosynthetic-capacity}}
Based on Farquhar et al. (1980) and Wullschleger (1993), we can calculate the
electron-limited photosynthetic rate under daily maximum radiation ( \(W_{jx}\))
and the Rubisco-limited photosynthetic rate ( \(W_{\mathrm{c}}\)) as follows,
\phantomsection\label{\detokenize{tech_note/Photosynthetic_Capacity/CLM50_Tech_Note_Photosynthetic_Capacity:equation-10.14)}}\begin{equation}\label{equation:tech_note/Photosynthetic_Capacity/CLM50_Tech_Note_Photosynthetic_Capacity:10.14)}
\begin{split}W_{J_{x}} = K_{j}J_{x} ,\end{split}
\end{equation}\phantomsection\label{\detokenize{tech_note/Photosynthetic_Capacity/CLM50_Tech_Note_Photosynthetic_Capacity:equation-10.15)}}\begin{equation}\label{equation:tech_note/Photosynthetic_Capacity/CLM50_Tech_Note_Photosynthetic_Capacity:10.15)}
\begin{split}W_{\mathrm{c}} = K_{\mathrm{c}} V_{{\mathrm{c}, \text{max}}},\end{split}
\end{equation}
where \(K_{j}\) and \(K_{\mathrm{c}}\) as the conversion factors for
\(J_{x}\) and  \(V_{{\mathrm{c}, \text{max}}}\) ( \(V_{{\mathrm{c}, \text{max}}}\) to
\(W_{\mathrm{c}}\) and \(J_{x}\) to \(W_{J_{x}}\)), respectively. Based on
Xu et al. (2012), Maire et al. (2012) and Walker et al. (2014), we
assume that \(W_{\mathrm{c}}\) is proportional to
\(W_{J_{x}}\). Specifically, we have
\phantomsection\label{\detokenize{tech_note/Photosynthetic_Capacity/CLM50_Tech_Note_Photosynthetic_Capacity:equation-10.16)}}\begin{equation}\label{equation:tech_note/Photosynthetic_Capacity/CLM50_Tech_Note_Photosynthetic_Capacity:10.16)}
\begin{split}W_{\mathrm{c}}=t_{\alpha}t_{\mathrm{c}, j0}W_{J_{x}}\end{split}
\end{equation}
where \(t_{\mathrm{c}, j0}\) is the baseline ratio of \(W_{\mathrm{c}}\) to
\(W_{J_{x}}\). We recognize that this ratio may change depending on the
nitrogen use efficiency of carboxylation and electron transport (Ainsworth and Rogers, 2007),
therefore the LUNA model has the modification factor, \(t_{\alpha}\), to adjust baseline
the ratio depending on the nitrogen use efficiency for electron vs carboxylation (Ali et al 2016).


\subsubsection{Total Respiration}
\label{\detokenize{tech_note/Photosynthetic_Capacity/CLM50_Tech_Note_Photosynthetic_Capacity:total-respiration}}
Following Collatz et al.(1991a), the total respiration ( \(R_{\mathrm{t}}\)) is
calculated in proportion to \(V_{\text{c,max}}\),
\phantomsection\label{\detokenize{tech_note/Photosynthetic_Capacity/CLM50_Tech_Note_Photosynthetic_Capacity:equation-10.17)}}\begin{equation}\label{equation:tech_note/Photosynthetic_Capacity/CLM50_Tech_Note_Photosynthetic_Capacity:10.17)}
\begin{split}R_{\mathrm{t}} = 0.015 V_{\text{c,max}}.\end{split}
\end{equation}
Accounting for the daytime and nighttime temperature, the daily respirations is calculated as follows,
\phantomsection\label{\detokenize{tech_note/Photosynthetic_Capacity/CLM50_Tech_Note_Photosynthetic_Capacity:equation-10.18)}}\begin{equation}\label{equation:tech_note/Photosynthetic_Capacity/CLM50_Tech_Note_Photosynthetic_Capacity:10.18)}
\begin{split} R_{\text{td}}={R}_{\mathrm{t}} [D_{\text{day}} + D_{\text{night}} f_{\mathrm{r}}{(T_{\text{night}})/f_{\mathrm{r}}{(T_{\text{day}})}}],\end{split}
\end{equation}
where \(D_{\text{day}}\) and \(D_{\text{night}}\) are daytime and
nighttime durations in seconds. \(f_{\mathrm{r}}(T_{\text{night}})\) and
\(f_{\mathrm{r}}(T_{\text{day}})\) are the temperature response functions for
respiration {[}see Appendix B in Ali et al (2016) for details {]}.


\subsection{Numerical scheme}
\label{\detokenize{tech_note/Photosynthetic_Capacity/CLM50_Tech_Note_Photosynthetic_Capacity:numerical-scheme}}\label{\detokenize{tech_note/Photosynthetic_Capacity/CLM50_Tech_Note_Photosynthetic_Capacity:id3}}
The LUNA model searches for the “optimal” nitrogen allocations for maximum net photosynthetic carbon gain
by incrementally increase the nitrogen allocated for light capture (i.e.,  \(N_{\text{lc}}\)) (see Ali et al 2016 for details) .
We assume that  plants only optimize the nitrogen allocation when they can grow (i.e., GPP\textgreater{}0.0).
If GPP become zero under stress, then the LUNA model assume a certain amount of enzyme will decay at daily rates of 0.1,
in view that the half-life time for photosynthetic enzymes are short (\textasciitilde{}7 days) (Suzuki et al. 2001).
To avoid unrealistic low values of photosynthetic capacity, the decay is only limited to 50 percent of the original enzyme levels.


\section{Plant Hydraulics}
\label{\detokenize{tech_note/Plant_Hydraulics/CLM50_Tech_Note_Plant_Hydraulics::doc}}\label{\detokenize{tech_note/Plant_Hydraulics/CLM50_Tech_Note_Plant_Hydraulics:plant-hydraulics}}\label{\detokenize{tech_note/Plant_Hydraulics/CLM50_Tech_Note_Plant_Hydraulics:rst-plant-hydraulics}}

\subsection{Roots}
\label{\detokenize{tech_note/Plant_Hydraulics/CLM50_Tech_Note_Plant_Hydraulics:roots}}\label{\detokenize{tech_note/Plant_Hydraulics/CLM50_Tech_Note_Plant_Hydraulics:id1}}

\subsubsection{Vertical Root Distribution}
\label{\detokenize{tech_note/Plant_Hydraulics/CLM50_Tech_Note_Plant_Hydraulics:id2}}\label{\detokenize{tech_note/Plant_Hydraulics/CLM50_Tech_Note_Plant_Hydraulics:vertical-root-distribution}}
The root fraction \(r_{i}\)  in each soil layer depends on the plant
functional type
\phantomsection\label{\detokenize{tech_note/Plant_Hydraulics/CLM50_Tech_Note_Plant_Hydraulics:equation-11.1}}\begin{equation}\label{equation:tech_note/Plant_Hydraulics/CLM50_Tech_Note_Plant_Hydraulics:11.1}
\begin{split}r_{i} =
\begin{array}{lr}
\left(\beta^{z_{h,\, i-1} \cdot 100} - \beta^{z_{h,\, i} \cdot 100} \right) & \qquad {\rm for\; }1 \le i \le N_{levsoi}
\end{array}\end{split}
\end{equation}
where \(z_{h,\, i}\) (m) is the depth from the soil surface to the
interface between layers \(i\) and \(i+1\) (\(z_{h,\, 0}\) ,
the soil surface) (section \hyperref[\detokenize{tech_note/Ecosystem/CLM50_Tech_Note_Ecosystem:vertical-discretization}]{\ref{\detokenize{tech_note/Ecosystem/CLM50_Tech_Note_Ecosystem:vertical-discretization}}}), the factor of 100
converts from m to cm, and \(\beta\) is a plant-dependent root
distribution parameter adopted from {\hyperref[\detokenize{tech_note/References/CLM50_Tech_Note_References:jacksonetal1996}]{\sphinxcrossref{\DUrole{std,std-ref}{Jackson et al. (1996)}}}}
(\hyperref[\detokenize{tech_note/Plant_Hydraulics/CLM50_Tech_Note_Plant_Hydraulics:table-plant-functional-type-root-distribution-parameters}]{Table \ref{\detokenize{tech_note/Plant_Hydraulics/CLM50_Tech_Note_Plant_Hydraulics:table-plant-functional-type-root-distribution-parameters}}}).


\begin{savenotes}\sphinxattablestart
\centering
\sphinxcapstartof{table}
\sphinxcaption{Plant functional type root distribution parameters}\label{\detokenize{tech_note/Plant_Hydraulics/CLM50_Tech_Note_Plant_Hydraulics:table-plant-functional-type-root-distribution-parameters}}\label{\detokenize{tech_note/Plant_Hydraulics/CLM50_Tech_Note_Plant_Hydraulics:id12}}
\sphinxaftercaption
\begin{tabulary}{\linewidth}[t]{|T|T|}
\hline
\sphinxstylethead{\sphinxstyletheadfamily 
Plant Functional Type
\unskip}\relax &
\(\beta\)
\\
\hline
NET Temperate
&
0.976
\\
\hline
NET Boreal
&
0.943
\\
\hline
NDT Boreal
&
0.943
\\
\hline
BET Tropical
&
0.993
\\
\hline
BET temperate
&
0.966
\\
\hline
BDT tropical
&
0.993
\\
\hline
BDT temperate
&
0.966
\\
\hline
BDT boreal
&
0.943
\\
\hline
BES temperate
&
0.964
\\
\hline
BDS temperate
&
0.964
\\
\hline
BDS boreal
&
0.914
\\
\hline
C$_{\text{3}}$ grass arctic
&
0.914
\\
\hline
C$_{\text{3}}$ grass
&
0.943
\\
\hline
C$_{\text{4}}$ grass
&
0.943
\\
\hline
Crop R
&
0.943
\\
\hline
Crop I
&
0.943
\\
\hline
Corn R
&
0.943
\\
\hline
Corn I
&
0.943
\\
\hline
Temp Cereal R
&
0.943
\\
\hline
Temp Cereal I
&
0.943
\\
\hline
Winter Cereal R
&
0.943
\\
\hline
Winter Cereal I
&
0.943
\\
\hline
Soybean R
&
0.943
\\
\hline
Soybean I
&
0.943
\\
\hline
\end{tabulary}
\par
\sphinxattableend\end{savenotes}


\subsubsection{Root Spacing}
\label{\detokenize{tech_note/Plant_Hydraulics/CLM50_Tech_Note_Plant_Hydraulics:root-spacing}}\label{\detokenize{tech_note/Plant_Hydraulics/CLM50_Tech_Note_Plant_Hydraulics:id3}}
To determine the conductance along the soil to root pathway (section
\hyperref[\detokenize{tech_note/Plant_Hydraulics/CLM50_Tech_Note_Plant_Hydraulics:soil-to-root}]{\ref{\detokenize{tech_note/Plant_Hydraulics/CLM50_Tech_Note_Plant_Hydraulics:soil-to-root}}}) an estimate of the spacing between the roots within
a soil layer is required.  The distance between roots \(dx_{root,i}\) (m)
is calculated by assuming that roots are distributed uniformly throughout
the soil ({\hyperref[\detokenize{tech_note/References/CLM50_Tech_Note_References:gardner1960}]{\sphinxcrossref{\DUrole{std,std-ref}{Gardner 1960}}}})
\phantomsection\label{\detokenize{tech_note/Plant_Hydraulics/CLM50_Tech_Note_Plant_Hydraulics:equation-11.12}}\begin{equation}\label{equation:tech_note/Plant_Hydraulics/CLM50_Tech_Note_Plant_Hydraulics:11.12}
\begin{split}dx_{root,i} = \left(\pi \cdot L_i\right)^{- \frac{1}{2}}\end{split}
\end{equation}
where \(L_{i}\) is the root length density (m m $^{\text{-3}}$)
\phantomsection\label{\detokenize{tech_note/Plant_Hydraulics/CLM50_Tech_Note_Plant_Hydraulics:equation-11.13}}\begin{equation}\label{equation:tech_note/Plant_Hydraulics/CLM50_Tech_Note_Plant_Hydraulics:11.13}
\begin{split}L_{i} = \frac{B_{root,i}}{\rho_{root} {CA}_{root}} \ ,\end{split}
\end{equation}
\(B_{root,i}\) is the root biomass density (kg m $^{\text{-3}}$)
\phantomsection\label{\detokenize{tech_note/Plant_Hydraulics/CLM50_Tech_Note_Plant_Hydraulics:equation-11.14}}\begin{equation}\label{equation:tech_note/Plant_Hydraulics/CLM50_Tech_Note_Plant_Hydraulics:11.14}
\begin{split}B_{root,i} = \frac{c\_to\_b \cdot C_{fineroot} \cdot r_{i}}{dz_{i}}\end{split}
\end{equation}
where \(c\_to\_b = 2\) (kg biomass kg carbon $^{\text{-1}}$) and
\(C_{fineroot}\) is the amount of fine root carbon (kg m $^{\text{-2}}$).

\(\rho_{root}\) is the root density  (kg m $^{\text{-3}}$), and
\({CA}_{root}\) is the fine root cross sectional area (m $^{\text{2}}$)
\phantomsection\label{\detokenize{tech_note/Plant_Hydraulics/CLM50_Tech_Note_Plant_Hydraulics:equation-11.15}}\begin{equation}\label{equation:tech_note/Plant_Hydraulics/CLM50_Tech_Note_Plant_Hydraulics:11.15}
\begin{split}CA_{root} = \pi r_{root}^{2}\end{split}
\end{equation}
where \(r_{root}\) is the root radius (m).


\subsection{Plant Hydraulic Stress}
\label{\detokenize{tech_note/Plant_Hydraulics/CLM50_Tech_Note_Plant_Hydraulics:plant-hydraulic-stress}}\label{\detokenize{tech_note/Plant_Hydraulics/CLM50_Tech_Note_Plant_Hydraulics:id4}}
The Plant Hydraulic Stress (PHS) routine explicitly models water transport
through the vegetation according to a simple hydraulic framework following
Darcy’s Law for porous media flow equations influenced by
{\hyperref[\detokenize{tech_note/References/CLM50_Tech_Note_References:bonanetal2014}]{\sphinxcrossref{\DUrole{std,std-ref}{Bonan et al. (2014)}}}},
{\hyperref[\detokenize{tech_note/References/CLM50_Tech_Note_References:chuangetal2006}]{\sphinxcrossref{\DUrole{std,std-ref}{Chuang et al. (2006)}}}},
{\hyperref[\detokenize{tech_note/References/CLM50_Tech_Note_References:sperryetal1998}]{\sphinxcrossref{\DUrole{std,std-ref}{Sperry et al. (1998)}}}},
{\hyperref[\detokenize{tech_note/References/CLM50_Tech_Note_References:sperryandlove2015}]{\sphinxcrossref{\DUrole{std,std-ref}{Sperry and Love (2015)}}}},
{\hyperref[\detokenize{tech_note/References/CLM50_Tech_Note_References:williamsetal1996}]{\sphinxcrossref{\DUrole{std,std-ref}{Williams et al (1996)}}}}.

PHS solves for the vegetation water potential that matches water supply with
transpiration demand. Water supply is modeled according to the circuit analog
in \hyperref[\detokenize{tech_note/Plant_Hydraulics/CLM50_Tech_Note_Plant_Hydraulics:figure-plant-hydraulic-circuit}]{Figure \ref{\detokenize{tech_note/Plant_Hydraulics/CLM50_Tech_Note_Plant_Hydraulics:figure-plant-hydraulic-circuit}}}. Transpiration demand is modeled
relative to maximum transpiration by a transpiration loss function dependent
on leaf water potential.

\begin{figure}[htbp]
\centering
\capstart

\noindent\sphinxincludegraphics{{circuit}.jpg}
\caption{Circuit diagram of plant hydraulics scheme}\label{\detokenize{tech_note/Plant_Hydraulics/CLM50_Tech_Note_Plant_Hydraulics:figure-plant-hydraulic-circuit}}\label{\detokenize{tech_note/Plant_Hydraulics/CLM50_Tech_Note_Plant_Hydraulics:id13}}\end{figure}


\subsubsection{Plant Water Supply}
\label{\detokenize{tech_note/Plant_Hydraulics/CLM50_Tech_Note_Plant_Hydraulics:id5}}\label{\detokenize{tech_note/Plant_Hydraulics/CLM50_Tech_Note_Plant_Hydraulics:plant-water-supply}}
The supply equations are used to solve for vegetation water potential forced
by transpiration demand and the set of layer-by-layer soil water potentials.
The water supply is discretized into segments: soil-to-root, root-to-stem, and
stem-to-leaf. There are typically several (1-49) soil-to-root flows operating
in parallel, one per soil layer. There are two stem-to-leaf flows operating in
parallel, corresponding to the sunlit and shaded “leaves”.

In general the water fluxes (e.g. soil-to-root, root-to-stem, etc.) are
modeled according to Darcy’s Law for porous media flow as:
\phantomsection\label{\detokenize{tech_note/Plant_Hydraulics/CLM50_Tech_Note_Plant_Hydraulics:equation-11.101}}\begin{equation}\label{equation:tech_note/Plant_Hydraulics/CLM50_Tech_Note_Plant_Hydraulics:11.101}
\begin{split}q = kA\left( \psi_1 - \psi_2 \right)\end{split}
\end{equation}
\(q\) is the flux of water (mmH$_{\text{2}}$O) spanning the segment
between \(\psi_1\) and \(\psi_2\)

\(k\) is the hydraulic conductance (s$^{\text{-1}}$)

\(A\) is the area basis (m$^{\text{2}}$/m$^{\text{2}}$) relating the
conducting area basis to ground area

\(\psi_1 - \psi_2\) is the gradient in water potential (mmH$_{\text{2}}$O)
across the segment

The segments in \hyperref[\detokenize{tech_note/Plant_Hydraulics/CLM50_Tech_Note_Plant_Hydraulics:figure-plant-hydraulic-circuit}]{Figure \ref{\detokenize{tech_note/Plant_Hydraulics/CLM50_Tech_Note_Plant_Hydraulics:figure-plant-hydraulic-circuit}}} have variable resistance,
as the water potentials become lower, hydraulic conductance decreases.  This is
captured by multiplying the maximum segment conductance by a sigmoidal function
capturing the percent loss of conductivity. The function uses two parameters to
fit experimental vulnerability curves: the water potential at 50\% loss of
conductivity (\(p50\)) and a shape fitting parameter (\(c_k\)).
\phantomsection\label{\detokenize{tech_note/Plant_Hydraulics/CLM50_Tech_Note_Plant_Hydraulics:equation-11.102}}\begin{equation}\label{equation:tech_note/Plant_Hydraulics/CLM50_Tech_Note_Plant_Hydraulics:11.102}
\begin{split}k=k_{max}\cdot 2^{-\left(\dfrac{\psi_1}{p50}\right)^{c_k}}\end{split}
\end{equation}
\(k_{max}\) is the maximum segment conductance (s-1)

\(p50\) is the water potential at 50\% loss of conductivity (mmH2O)

\(\psi_1\) is the water potential of the lower segment terminus (mmH2O)


\paragraph{Stem-to-leaf}
\label{\detokenize{tech_note/Plant_Hydraulics/CLM50_Tech_Note_Plant_Hydraulics:id6}}\label{\detokenize{tech_note/Plant_Hydraulics/CLM50_Tech_Note_Plant_Hydraulics:stem-to-leaf}}
The area basis and conductance parameterization varies by segment. There
are two stem-to-leaf fluxes in parallel, from stem to sunlit leaf and from
stem to shaded leaf (\(q_{1a}\) and \(q_{1a}\)).  The water flux from
stem-to-leaf is the product of the segment conductance, the conducting area
basis, and the water potential gradient from stem to leaf. Stem-to-leaf
conductance is defined as the maximum conductance multiplied by the percent
of maximum conductance, as calculated by the sigmoidal vulnerability curve.
The maximum conductance is a PFT parameter representing the maximum
conductance of water from stem to leaf per unit leaf area.  This parameter
can be defined separately for sunlit and shaded segments and should already
include the appropriate length scaling (in other words this is a conductance,
not conductivity). The water potential gradient is the difference between
leaf water potential and stem water potential. There is no gravity term,
assuming a negligible difference in height across the segment. The area
basis is the leaf area index (either sunlit or shaded).
\phantomsection\label{\detokenize{tech_note/Plant_Hydraulics/CLM50_Tech_Note_Plant_Hydraulics:equation-11.103}}\begin{equation}\label{equation:tech_note/Plant_Hydraulics/CLM50_Tech_Note_Plant_Hydraulics:11.103}
\begin{split}q_{1a}=k_{1a}\cdot\mbox{LAI}_{sun}\cdot\left(\psi_{stem}-\psi_{sunleaf} \right)\end{split}
\end{equation}\phantomsection\label{\detokenize{tech_note/Plant_Hydraulics/CLM50_Tech_Note_Plant_Hydraulics:equation-11.104}}\begin{equation}\label{equation:tech_note/Plant_Hydraulics/CLM50_Tech_Note_Plant_Hydraulics:11.104}
\begin{split}q_{1b}=k_{1b}\cdot\mbox{LAI}_{shade}\cdot\left(\psi_{stem}-\psi_{shadeleaf} \right)\end{split}
\end{equation}\phantomsection\label{\detokenize{tech_note/Plant_Hydraulics/CLM50_Tech_Note_Plant_Hydraulics:equation-11.105}}\begin{equation}\label{equation:tech_note/Plant_Hydraulics/CLM50_Tech_Note_Plant_Hydraulics:11.105}
\begin{split}k_{1a}=k_{1a,max}\cdot 2^{-\left(\dfrac{\psi_{stem}}{p50_1}\right)^{c_k}}\end{split}
\end{equation}\phantomsection\label{\detokenize{tech_note/Plant_Hydraulics/CLM50_Tech_Note_Plant_Hydraulics:equation-11.106}}\begin{equation}\label{equation:tech_note/Plant_Hydraulics/CLM50_Tech_Note_Plant_Hydraulics:11.106}
\begin{split}k_{1b}=k_{1b,max}\cdot 2^{-\left(\dfrac{\psi_{stem}}{p50_1}\right)^{c_k}}\end{split}
\end{equation}
Variables:

\(q_{1a}\) = flux of water (mmH2O/s) from stem to sunlit leaf

\(q_{1b}\) = flux of water (mmH2O/s) from stem to shaded leaf

\(LAI_{sun}\) = sunlit leaf area index (m2/m2)

\(LAI_{shade}\) = shaded leaf area index (m2/m2)

\(\psi_{stem}\) = stem water potential (mmH2O)

\(\psi_{sunleaf}\) = sunlit leaf water potential (mmH2O)

\(\psi_{shadeleaf}\) = shaded leaf water potential (mmH2O)

Parameters:

\(k_{1a,max}\) = maximum leaf conductance (s-1)

\(k_{1b,max}\) = maximum leaf conductance (s-1)

\(p50_{1}\) = water potential at 50\% loss of conductance (mmH2O)

\(c_{k}\) = vulnerability curve shape-fitting parameter (-)


\paragraph{Root-to-stem}
\label{\detokenize{tech_note/Plant_Hydraulics/CLM50_Tech_Note_Plant_Hydraulics:id7}}\label{\detokenize{tech_note/Plant_Hydraulics/CLM50_Tech_Note_Plant_Hydraulics:root-to-stem}}
There is one root-to-stem flux. This represents a flux from the root collar
to the upper branch reaches. The water flux from root-to-stem is the product
of the segment conductance, the conducting area basis, and the water
potential gradient from root to stem. Root-to-stem conductance is defined
as the maximum conductance multiplied by the percent of maximum conductance,
as calculated by the sigmoidal vulnerability curve (two parameters). The
maximum conductance is defined as the maximum root-to-stem conductivity per
unit stem area (PFT parameter) divided by the length of the conducting path,
which is taken to be the vegetation height. The area basis is the stem area
index. The gradient in water potential is the difference between the root
water potential and the stem water potential less the difference in
gravitational potential.
\phantomsection\label{\detokenize{tech_note/Plant_Hydraulics/CLM50_Tech_Note_Plant_Hydraulics:equation-11.107}}\begin{equation}\label{equation:tech_note/Plant_Hydraulics/CLM50_Tech_Note_Plant_Hydraulics:11.107}
\begin{split}q_2=k_2 \cdot SAI \cdot \left( \psi_{root} - \psi_{stem} - \Delta \psi_z  \right)\end{split}
\end{equation}\phantomsection\label{\detokenize{tech_note/Plant_Hydraulics/CLM50_Tech_Note_Plant_Hydraulics:equation-11.108}}\begin{equation}\label{equation:tech_note/Plant_Hydraulics/CLM50_Tech_Note_Plant_Hydraulics:11.108}
\begin{split}k_2=\dfrac{k_{2,max}}{z_2} \cdot 2^{-\left(\dfrac{\psi_{root}}{p50_2}\right)^{c_k}}\end{split}
\end{equation}
Variables:

\(q_2\) = flux of water (mmH2O/s) from root to stem

\(SAI\) = stem area index (m2/m2)

\(\Delta\psi_z\) = gravitational potential (mmH2O)

\(\psi_{root}\) = root water potential (mmH2O)

\(\psi_{stem}\) = stem water potential (mmH2O)

Parameters:

\(k_{2,max}\) = maximum stem conductivity (m/s)

\(p50_2\) = water potential at 50\% loss of conductivity (mmH2O)

\(z_2\) = vegetation height (m)


\paragraph{Soil-to-root}
\label{\detokenize{tech_note/Plant_Hydraulics/CLM50_Tech_Note_Plant_Hydraulics:id8}}\label{\detokenize{tech_note/Plant_Hydraulics/CLM50_Tech_Note_Plant_Hydraulics:soil-to-root}}
There are several soil-to-root fluxes operating in parallel (one for each
root-containing soil layer). Each represents a flux from the given soil
layer to the root collar. The water flux from soil-to-root is the product
of the segment conductance, the conducting area basis, and the water
potential gradient from soil to root. The area basis is a proxy for root
area index, defined as the summed leaf and stem area index multiplied by
the root-to-shoot ratio (PFT parameter) multiplied by the layer root
fraction. The root fraction comes from an empirical root profile (section
\hyperref[\detokenize{tech_note/Plant_Hydraulics/CLM50_Tech_Note_Plant_Hydraulics:vertical-root-distribution}]{\ref{\detokenize{tech_note/Plant_Hydraulics/CLM50_Tech_Note_Plant_Hydraulics:vertical-root-distribution}}}).

The gradient in water potential is the difference between the soil water
potential and the root water potential less the difference in gravitational
potential. There is only one root water potential to which all soil layers
are connected in parallel. A soil-to-root flux can be either positive
(vegetation water uptake) or negative (water deposition), depending on the
relative values of the root and soil water potentials. This allows for the
occurrence of hydraulic redistribution where water moves through vegetation
tissue from one soil layer to another.

Soil-to-root conductance is the result of two resistances in series, first
across the soil-root interface and then through the root tissue. The root
tissue conductance is defined as the maximum conductance multiplied by the
percent of maximum conductance, as calculated by the sigmoidal vulnerability
curve. The maximum conductance is defined as the maximum root-tissue
conductivity (PFT parameter) divided by the length of the conducting path,
which is taken to be the soil layer depth plus lateral root length.

The soil-root interface conductance is defined as the soil conductivity
divided by the conducting length from soil to root. The soil conductivity
varies by soil layer and is calculated based on soil potential and soil
properties, via the Brooks-Corey theory. The conducting length is determined
from the characteristic root spacing (section \hyperref[\detokenize{tech_note/Plant_Hydraulics/CLM50_Tech_Note_Plant_Hydraulics:root-spacing}]{\ref{\detokenize{tech_note/Plant_Hydraulics/CLM50_Tech_Note_Plant_Hydraulics:root-spacing}}}).
\phantomsection\label{\detokenize{tech_note/Plant_Hydraulics/CLM50_Tech_Note_Plant_Hydraulics:equation-11.109}}\begin{equation}\label{equation:tech_note/Plant_Hydraulics/CLM50_Tech_Note_Plant_Hydraulics:11.109}
\begin{split}q_{3,i}=k_{3,i} \cdot RAI \cdot \left(\psi_{soil,i}-\psi_{root} + \Delta\psi_{z,i} \right)\end{split}
\end{equation}\phantomsection\label{\detokenize{tech_note/Plant_Hydraulics/CLM50_Tech_Note_Plant_Hydraulics:equation-11.110}}\begin{equation}\label{equation:tech_note/Plant_Hydraulics/CLM50_Tech_Note_Plant_Hydraulics:11.110}
\begin{split}RAI=\left(LAI+SAI \right) \cdot r_i \cdot f_{root-leaf}\end{split}
\end{equation}\phantomsection\label{\detokenize{tech_note/Plant_Hydraulics/CLM50_Tech_Note_Plant_Hydraulics:equation-11.111}}\begin{equation}\label{equation:tech_note/Plant_Hydraulics/CLM50_Tech_Note_Plant_Hydraulics:11.111}
\begin{split}k_{3,i}=\dfrac{k_{r,i} \cdot k_{s,i}}{k_{r,i}+k_{s,i}}\end{split}
\end{equation}\phantomsection\label{\detokenize{tech_note/Plant_Hydraulics/CLM50_Tech_Note_Plant_Hydraulics:equation-11.112}}\begin{equation}\label{equation:tech_note/Plant_Hydraulics/CLM50_Tech_Note_Plant_Hydraulics:11.112}
\begin{split}k_{r,i}=\dfrac{k_{3,max}}{z_{3,i}} \cdot 2^{-\left(\dfrac{\psi_{soil,i}}{p50_3}\right)^{c_k}}\end{split}
\end{equation}\phantomsection\label{\detokenize{tech_note/Plant_Hydraulics/CLM50_Tech_Note_Plant_Hydraulics:equation-11.113}}\begin{equation}\label{equation:tech_note/Plant_Hydraulics/CLM50_Tech_Note_Plant_Hydraulics:11.113}
\begin{split}k_{s,i} = \dfrac{k_{soil,i}}{dx_{root,i}}\end{split}
\end{equation}
Variables:

\(q_{3,i}\) = flux of water (mmH2O/s) from soil layer \(i\) to root

\(\Delta\psi_{z,i}\) = change in gravitational potential from soil layer \(i\) to surface (mmH2O)

\(LAI\) = total leaf area index (m2/m2)

\(SAI\) = stem area index (m2/m2)

\(\psi_{soil,i}\) = water potential in soil layer \(i\) (mmH2O)

\(\psi_{root}\) = root water potential (mmH2O)

\(z_{3,i}\) = length of root tissue conducting path = soil layer depth + root lateral length (m)

\(r_i\) = root fraction in soil layer \(i\) (-)

\(k_{soil,i}\) = Brooks-Corey soil conductivity in soil layer \(i\) (m/s)

Parameters:

\(f_{root-leaf}\) = root-to-shoot ratio (-)

\(p50_3\) = water potential at 50\% loss of root tissue conductance (mmH2O)

\(ck\) = shape-fitting parameter for vulnerability curve (-)


\subsubsection{Plant Water Demand}
\label{\detokenize{tech_note/Plant_Hydraulics/CLM50_Tech_Note_Plant_Hydraulics:plant-water-demand}}\label{\detokenize{tech_note/Plant_Hydraulics/CLM50_Tech_Note_Plant_Hydraulics:id9}}
Plant water demand depends on stomatal conductance, which is described in section \hyperref[\detokenize{tech_note/Photosynthesis/CLM50_Tech_Note_Photosynthesis:stomatal-resistance}]{\ref{\detokenize{tech_note/Photosynthesis/CLM50_Tech_Note_Photosynthesis:stomatal-resistance}}}.
Here we describe the influence of PHS and the coupling of vegetation water demand and supply.
PHS models vegetation water demand as transpiration attenuated by a transpiration loss function based on leaf water potential.
Sunlit leaf transpiration is modeled as the maximum sunlit leaf transpiration multiplied by the percent of maximum transpiration as modeled by the sigmoidal loss function.
The same follows for shaded leaf transpiration.
Maximum stomatal conductance is calculated from the Medlyn model {\hyperref[\detokenize{tech_note/References/CLM50_Tech_Note_References:medlynetal2011}]{\sphinxcrossref{\DUrole{std,std-ref}{(Medlyn et al. 2011)}}}} absent water stress and used to calculate the maximum transpiration (see section \hyperref[\detokenize{tech_note/Fluxes/CLM50_Tech_Note_Fluxes:sensible-and-latent-heat-fluxes-and-temperature-for-vegetated-surfaces}]{\ref{\detokenize{tech_note/Fluxes/CLM50_Tech_Note_Fluxes:sensible-and-latent-heat-fluxes-and-temperature-for-vegetated-surfaces}}}).
Water stress is calculated as the ratio of attenuated stomatal conductance to maximum stomatal conductance.
Water stress is calculated with distinct values for sunlit and shaded leaves.
Vegetation water stress is calculated based on leaf water potential and is used to attenuate photosynthesis (see section \hyperref[\detokenize{tech_note/Photosynthesis/CLM50_Tech_Note_Photosynthesis:photosynthesis}]{\ref{\detokenize{tech_note/Photosynthesis/CLM50_Tech_Note_Photosynthesis:photosynthesis}}})
\phantomsection\label{\detokenize{tech_note/Plant_Hydraulics/CLM50_Tech_Note_Plant_Hydraulics:equation-11.201}}\begin{equation}\label{equation:tech_note/Plant_Hydraulics/CLM50_Tech_Note_Plant_Hydraulics:11.201}
\begin{split}E_{sun} = E_{sun,max} \cdot 2^{-\left(\dfrac{\psi_{sunleaf}}{p50_e}\right)^{c_k}}\end{split}
\end{equation}\phantomsection\label{\detokenize{tech_note/Plant_Hydraulics/CLM50_Tech_Note_Plant_Hydraulics:equation-11.202}}\begin{equation}\label{equation:tech_note/Plant_Hydraulics/CLM50_Tech_Note_Plant_Hydraulics:11.202}
\begin{split}E_{shade} = E_{shade,max} \cdot 2^{-\left(\dfrac{\psi_{shadeleaf}}{p50_e}\right)^{c_k}}\end{split}
\end{equation}\phantomsection\label{\detokenize{tech_note/Plant_Hydraulics/CLM50_Tech_Note_Plant_Hydraulics:equation-11.203}}\begin{equation}\label{equation:tech_note/Plant_Hydraulics/CLM50_Tech_Note_Plant_Hydraulics:11.203}
\begin{split}\beta_{t,sun} = \dfrac{g_{s,sun}}{g_{s,sun,\beta_t=1}}\end{split}
\end{equation}\phantomsection\label{\detokenize{tech_note/Plant_Hydraulics/CLM50_Tech_Note_Plant_Hydraulics:equation-11.204}}\begin{equation}\label{equation:tech_note/Plant_Hydraulics/CLM50_Tech_Note_Plant_Hydraulics:11.204}
\begin{split}\beta_{t,shade} = \dfrac{g_{s,shade}}{g_{s,shade,\beta_t=1}}\end{split}
\end{equation}
\(E_{sun}\) = sunlit leaf transpiration (mm/s)

\(E_{shade}\) = shaded leaf transpiration (mm/s)

\(E_{sun,max}\) = sunlit leaf transpiration absent water stress (mm/s)

\(E_{shade,max}\) = shaded leaf transpiration absent water stress (mm/s)

\(\psi_{sunleaf}\) = sunlit leaf water potential (mmH2O)

\(\psi_{shadeleaf}\) = shaded leaf water potential (mmH2O)

\(\beta_{t,sun}\) = sunlit transpiration water stress (-)

\(\beta_{t,shade}\) = shaded transpiration water stress (-)

\(g_{s,sun}\) = stomatal conductance of water corresponding to \(E_{sun}\)

\(g_{s,shade}\) = stomatal conductance of water corresponding to \(E_{shade}\)

\(g_{s,sun,max}\) = stomatal conductance of water corresponding to \(E_{sun,max}\)

\(g_{s,shade,max}\) = stomatal conductance of water corresponding to \(E_{shade,max}\)


\subsubsection{Vegetation Water Potential}
\label{\detokenize{tech_note/Plant_Hydraulics/CLM50_Tech_Note_Plant_Hydraulics:id10}}\label{\detokenize{tech_note/Plant_Hydraulics/CLM50_Tech_Note_Plant_Hydraulics:vegetation-water-potential}}
Both plant water supply and demand are functions of vegetation water potential. PHS explicitly models root, stem, shaded leaf, and sunlit leaf water potential at each timestep. PHS iterates to find the vegetation water potential \(\psi\) (vector) that satisfies continuity between the non-linear vegetation water supply and demand (equations \eqref{equation:tech_note/Plant_Hydraulics/CLM50_Tech_Note_Plant_Hydraulics:11.103}, \eqref{equation:tech_note/Plant_Hydraulics/CLM50_Tech_Note_Plant_Hydraulics:11.104}, \eqref{equation:tech_note/Plant_Hydraulics/CLM50_Tech_Note_Plant_Hydraulics:11.107}, \eqref{equation:tech_note/Plant_Hydraulics/CLM50_Tech_Note_Plant_Hydraulics:11.109}, \eqref{equation:tech_note/Plant_Hydraulics/CLM50_Tech_Note_Plant_Hydraulics:11.201}, \eqref{equation:tech_note/Plant_Hydraulics/CLM50_Tech_Note_Plant_Hydraulics:11.202}).
\phantomsection\label{\detokenize{tech_note/Plant_Hydraulics/CLM50_Tech_Note_Plant_Hydraulics:equation-11.301}}\begin{equation}\label{equation:tech_note/Plant_Hydraulics/CLM50_Tech_Note_Plant_Hydraulics:11.301}
\begin{split}\psi=\left[\psi_{sunleaf},\psi_{shadeleaf},\psi_{stem},\psi_{root}\right]\end{split}
\end{equation}\phantomsection\label{\detokenize{tech_note/Plant_Hydraulics/CLM50_Tech_Note_Plant_Hydraulics:equation-11.302}}\begin{equation}\label{equation:tech_note/Plant_Hydraulics/CLM50_Tech_Note_Plant_Hydraulics:11.302}
\begin{split}\begin{aligned}
E_{sun}&=q_{1a}\\
E_{shade}&=q_{1b}\\
E_{sun}+E_{shade}&=q_{1a}+q_{1b}\\
&=q_2\\
&=\sum_{i=1}^{nlevsoi}{q_{3,i}}
\end{aligned}\end{split}
\end{equation}
PHS finds the water potentials that match supply and demand. In the plant water transport equations \eqref{equation:tech_note/Plant_Hydraulics/CLM50_Tech_Note_Plant_Hydraulics:11.302}, the demand terms (left-hand side) are decreasing functions of absolute leaf water potential. As absolute leaf water potential becomes larger, water stress increases, causing a decrease in transpiration demand. The supply terms (right-hand side) are increasing functions of absolute leaf water potential. As absolute leaf water potential becomes larger, the gradients in water potential increase, causing an increase in vegetation water supply. PHS takes a Newton’s method approach to iteratively solve for the vegetation water potentials that satisfy continuity \eqref{equation:tech_note/Plant_Hydraulics/CLM50_Tech_Note_Plant_Hydraulics:11.302}.


\subsubsection{Numerical Implementation}
\label{\detokenize{tech_note/Plant_Hydraulics/CLM50_Tech_Note_Plant_Hydraulics:phs-numerical-implementation}}\label{\detokenize{tech_note/Plant_Hydraulics/CLM50_Tech_Note_Plant_Hydraulics:numerical-implementation}}
The four plant water potential nodes are ( \(\psi_{root}\), \(\psi_{xylem}\), \(\psi_{shadeleaf}\), \(\psi_{sunleaf}\)).
The fluxes between each pair of nodes are labeled in Figure 1.
\(E_{sun}\) and \(E_{sha}\) are the transpiration from sunlit and shaded leaves, respectively.
We use the circuit-analog model to calculate the vegetation water potential ( \(\psi\)) for the four plant nodes, forced by soil matric potential and unstressed transpiration.
The unstressed transpiration is acquired by running the photosynthesis model with \(\beta_t=1\).
The unstressed transpiration flux is attenuated based on the leaf-level vegetation water potential.
Using the attenuated transpiration, we solve for \(g_{s,stressed}\) and output \(\beta_t=\dfrac{g_{s,stressed}}{g_{s,unstressed}}\).

The continuity of water flow through the system yields four equations
\phantomsection\label{\detokenize{tech_note/Plant_Hydraulics/CLM50_Tech_Note_Plant_Hydraulics:equation-11.401}}\begin{equation}\label{equation:tech_note/Plant_Hydraulics/CLM50_Tech_Note_Plant_Hydraulics:11.401}
\begin{split}\begin{aligned}
E_{sun}&=q_{1a}\\
E_{shade}&=q_{1b}\\
q_{1a}+q_{1b}&=q_2\\
q_2&=\sum_{i=1}^{nlevsoi}{q_{3,i}}
\end{aligned}\end{split}
\end{equation}
We seek the set of vegetation water potential values,
\phantomsection\label{\detokenize{tech_note/Plant_Hydraulics/CLM50_Tech_Note_Plant_Hydraulics:equation-11.402}}\begin{equation}\label{equation:tech_note/Plant_Hydraulics/CLM50_Tech_Note_Plant_Hydraulics:11.402}
\begin{split}\psi=\left[ \begin {array}{c}
\psi_{sunleaf}\cr\psi_{shadeleaf}\cr\psi_{stem}\cr\psi_{root}
\end {array} \right]\end{split}
\end{equation}
that satisfies these equations, as forced by the soil moisture and atmospheric state.
Each flux on the schematic can be represented in terms of the relevant water potentials. Defining the transpiration fluxes:
\phantomsection\label{\detokenize{tech_note/Plant_Hydraulics/CLM50_Tech_Note_Plant_Hydraulics:equation-11.403}}\begin{equation}\label{equation:tech_note/Plant_Hydraulics/CLM50_Tech_Note_Plant_Hydraulics:11.403}
\begin{split}\begin{aligned}
E_{sun} &= E_{sun,max} \cdot 2^{-\left(\dfrac{\psi_{sunleaf}}{p50_e}\right)^{c_k}} \\
E_{shade} &= E_{shade,max} \cdot 2^{-\left(\dfrac{\psi_{shadeleaf}}{p50_e}\right)^{c_k}}
\end{aligned}\end{split}
\end{equation}
Defining the water supply fluxes:
\phantomsection\label{\detokenize{tech_note/Plant_Hydraulics/CLM50_Tech_Note_Plant_Hydraulics:equation-11.404}}\begin{equation}\label{equation:tech_note/Plant_Hydraulics/CLM50_Tech_Note_Plant_Hydraulics:11.404}
\begin{split}\begin{aligned}
q_{1a}&=k_{1a,max}\cdot 2^{-\left(\dfrac{\psi_{stem}}{p50_1}\right)^{c_k}} \cdot\mbox{LAI}_{sun}\cdot\left(\psi_{stem}-\psi_{sunleaf} \right) \\
q_{1b}&=k_{1b,max}\cdot 2^{-\left(\dfrac{\psi_{stem}}{p50_1}\right)^{c_k}}\cdot\mbox{LAI}_{shade}\cdot\left(\psi_{stem}-\psi_{shadeleaf} \right) \\
q_2&=\dfrac{k_{2,max}}{z_2} \cdot 2^{-\left(\dfrac{\psi_{root}}{p50_2}\right)^{c_k}} \cdot SAI \cdot \left( \psi_{root} - \psi_{stem} - \Delta \psi_z  \right) \\
q_{soil}&=\sum_{i=1}^{nlevsoi}{q_{3,i}}=\sum_{i=1}^{nlevsoi}{k_{3,i}\cdot RAI\cdot\left(\psi_{soil,i}-\psi_{root} + \Delta\psi_{z,i} \right)}
\end{aligned}\end{split}
\end{equation}
We’re looking to find the vector \(\psi\)
that fits with soil and atmospheric forcings while satisfying water flow continuity.
Due to the model non-linearity, we use a linearized explicit approach, iterating with Newton’s method.
The initial guess is the solution for \(\psi\) (vector) from the previous time step.
The general framework, from iteration \sphinxtitleref{m} to \sphinxtitleref{m+1} is:
\phantomsection\label{\detokenize{tech_note/Plant_Hydraulics/CLM50_Tech_Note_Plant_Hydraulics:equation-11.405}}\begin{equation}\label{equation:tech_note/Plant_Hydraulics/CLM50_Tech_Note_Plant_Hydraulics:11.405}
\begin{split}q^{m+1}=q^m+\dfrac{\delta q}{\delta\psi}\Delta\psi \\
\psi^{m+1}=\psi^{m}+\Delta\psi\end{split}
\end{equation}
So for our first flux balance equation, at iteration \sphinxtitleref{m+1}, we have:
\phantomsection\label{\detokenize{tech_note/Plant_Hydraulics/CLM50_Tech_Note_Plant_Hydraulics:equation-11.406}}\begin{equation}\label{equation:tech_note/Plant_Hydraulics/CLM50_Tech_Note_Plant_Hydraulics:11.406}
\begin{split}E_{sun}^{m+1}=q_{1a}^{m+1}\end{split}
\end{equation}
Which can be linearized to:
\phantomsection\label{\detokenize{tech_note/Plant_Hydraulics/CLM50_Tech_Note_Plant_Hydraulics:equation-11.407}}\begin{equation}\label{equation:tech_note/Plant_Hydraulics/CLM50_Tech_Note_Plant_Hydraulics:11.407}
\begin{split}E_{sun}^{m}+\dfrac{\delta E_{sun}}{\delta\psi}\Delta\psi=q_{1a}^{m}+\dfrac{\delta q_{1a}}{\delta\psi}\Delta\psi\end{split}
\end{equation}
And rearranged to be:
\phantomsection\label{\detokenize{tech_note/Plant_Hydraulics/CLM50_Tech_Note_Plant_Hydraulics:equation-11.408}}\begin{equation}\label{equation:tech_note/Plant_Hydraulics/CLM50_Tech_Note_Plant_Hydraulics:11.408}
\begin{split}\dfrac{\delta q_{1a}}{\delta\psi}\Delta\psi-\dfrac{\delta E_{sun}}{\delta\psi}\Delta\psi=E_{sun}^{m}-q_{1a}^{m}\end{split}
\end{equation}
And for the other 3 flux balance equations:
\phantomsection\label{\detokenize{tech_note/Plant_Hydraulics/CLM50_Tech_Note_Plant_Hydraulics:equation-11.409}}\begin{equation}\label{equation:tech_note/Plant_Hydraulics/CLM50_Tech_Note_Plant_Hydraulics:11.409}
\begin{split}\begin{aligned}
\dfrac{\delta q_{1b}}{\delta\psi}\Delta\psi-\dfrac{\delta E_{sha}}{\delta\psi}\Delta\psi&=E_{sha}^{m}-q_{1b}^{m} \\
\dfrac{\delta q_2}{\delta\psi}\Delta\psi-\dfrac{\delta q_{1a}}{\delta\psi}\Delta\psi-\dfrac{\delta q_{1b}}{\delta\psi}\Delta\psi&=q_{1a}^{m}+q_{1b}^{m}-q_2^{m} \\
\dfrac{\delta q_{soil}}{\delta\psi}\Delta\psi-\dfrac{\delta q_2}{\delta\psi}\Delta\psi&=q_2^{m}-q_{soil}^{m}
\end{aligned}\end{split}
\end{equation}
Putting all four together in matrix form:
\phantomsection\label{\detokenize{tech_note/Plant_Hydraulics/CLM50_Tech_Note_Plant_Hydraulics:equation-11.410}}\begin{equation}\label{equation:tech_note/Plant_Hydraulics/CLM50_Tech_Note_Plant_Hydraulics:11.410}
\begin{split}\left[ \begin {array}{c}
\dfrac{\delta q_{1a}}{\delta\psi}-\dfrac{\delta E_{sun}}{\delta\psi} \cr
\dfrac{\delta q_{1b}}{\delta\psi}-\dfrac{\delta E_{sha}}{\delta\psi} \cr
\dfrac{\delta q_2}{\delta\psi}-\dfrac{\delta q_{1a}}{\delta\psi}-\dfrac{\delta q_{1b}}{\delta\psi} \cr
\dfrac{\delta q_{soil}}{\delta\psi}-\dfrac{\delta q_2}{\delta\psi}
\end {array} \right]
\Delta\psi=
\left[ \begin {array}{c}
E_{sun}^{m}-q_{1a}^{m} \cr
E_{sha}^{m}-q_{1b}^{m} \cr
q_{1a}^{m}+q_{1b}^{m}-q_2^{m} \cr
q_2^{m}-q_{soil}^{m}
\end {array} \right]\end{split}
\end{equation}
Now to expand the left-hand side, from generic \(\psi\) to all four plant water potential nodes, noting that many derivatives are zero (e.g. \(\dfrac{\delta E_{sun}}{\delta\psi_{sha}}=0\))

Introducing the notation:
\(A\Delta\psi=b\)
\phantomsection\label{\detokenize{tech_note/Plant_Hydraulics/CLM50_Tech_Note_Plant_Hydraulics:equation-11.411}}\begin{equation}\label{equation:tech_note/Plant_Hydraulics/CLM50_Tech_Note_Plant_Hydraulics:11.411}
\begin{split}\Delta\psi=\left[ \begin {array}{c}
\Delta\psi_{sunleaf} \cr
\Delta\psi_{shadeleaf} \cr
\Delta\psi_{stem} \cr
\Delta\psi_{root}
\end {array} \right]\end{split}
\end{equation}\phantomsection\label{\detokenize{tech_note/Plant_Hydraulics/CLM50_Tech_Note_Plant_Hydraulics:equation-11.412}}\begin{equation}\label{equation:tech_note/Plant_Hydraulics/CLM50_Tech_Note_Plant_Hydraulics:11.412}
\begin{split}A=
\left[ \begin {array}{cccc}
\dfrac{\delta q_{1a}}{\delta \psi_{sun}}-\dfrac{\delta E_{sun}}{\delta \psi_{sun}}&0&\dfrac{\delta q_{1a}}{\delta \psi_{stem}}&0\cr
0&\dfrac{\delta q_{1b}}{\delta \psi_{sha}}-\dfrac{\delta E_{sha}}{\delta \psi_{sha}}&\dfrac{\delta q_{1b}}{\delta \psi_{stem}}&0\cr
-\dfrac{\delta q_{1a}}{\delta \psi_{sun}}&
-\dfrac{\delta q_{1b}}{\delta \psi_{sha}}&
\dfrac{\delta q_2}{\delta \psi_{stem}}-\dfrac{\delta q_{1a}}{\delta \psi_{stem}}-\dfrac{\delta q_{1b}}{\delta \psi_{stem}}&
\dfrac{\delta q_2}{\delta \psi_{root}}\cr
0&0&-\dfrac{\delta q_2}{\delta \psi_{stem}}&\dfrac{\delta q_{soil}}{\delta \psi_{root}}-\dfrac{\delta q_2}{\delta \psi_{root}}
\end {array} \right]\end{split}
\end{equation}\phantomsection\label{\detokenize{tech_note/Plant_Hydraulics/CLM50_Tech_Note_Plant_Hydraulics:equation-11.413}}\begin{equation}\label{equation:tech_note/Plant_Hydraulics/CLM50_Tech_Note_Plant_Hydraulics:11.413}
\begin{split}b=
\left[ \begin {array}{c}
E_{sun}^{m}-q_{b1}^{m} \cr
E_{sha}^{m}-q_{b2}^{m} \cr
q_{b1}^{m}+q_{b2}^{m}-q_{stem}^{m} \cr
q_{stem}^{m}-q_{soil}^{m}
\end {array} \right]\end{split}
\end{equation}
Now we compute all the entries for \(A\) and \(b\) based on the soil moisture and maximum transpiration forcings and can solve to find:
\phantomsection\label{\detokenize{tech_note/Plant_Hydraulics/CLM50_Tech_Note_Plant_Hydraulics:equation-11.414}}\begin{equation}\label{equation:tech_note/Plant_Hydraulics/CLM50_Tech_Note_Plant_Hydraulics:11.414}
\begin{split}\Delta\psi=A^{-1}b\end{split}
\end{equation}\phantomsection\label{\detokenize{tech_note/Plant_Hydraulics/CLM50_Tech_Note_Plant_Hydraulics:equation-11.415}}\begin{equation}\label{equation:tech_note/Plant_Hydraulics/CLM50_Tech_Note_Plant_Hydraulics:11.415}
\begin{split}\psi_{m+1}=\psi_m+\Delta\psi\end{split}
\end{equation}
We iterate until \(b\to 0\), signifying water flux balance through the system. The result is a final set of water potentials ( \(\psi_{root}\), \(\psi_{xylem}\), \(\psi_{shadeleaf}\), \(\psi_{sunleaf}\)) satisfying non-divergent water flux through the system.
The magnitude of the water flux is driven by soil matric potential and unstressed ( \(\beta_t=1\)) transpiration.

We use the transpiration solution (corresponding to the final solution for \(\psi\)) to compute stomatal conductance. The stomatal conductance is then used to compute \(\beta_t\).
\phantomsection\label{\detokenize{tech_note/Plant_Hydraulics/CLM50_Tech_Note_Plant_Hydraulics:equation-11.416}}\begin{equation}\label{equation:tech_note/Plant_Hydraulics/CLM50_Tech_Note_Plant_Hydraulics:11.416}
\begin{split}\beta_{t,sun} = \dfrac{g_{s,sun}}{g_{s,sun,\beta_t=1}}\end{split}
\end{equation}\phantomsection\label{\detokenize{tech_note/Plant_Hydraulics/CLM50_Tech_Note_Plant_Hydraulics:equation-11.417}}\begin{equation}\label{equation:tech_note/Plant_Hydraulics/CLM50_Tech_Note_Plant_Hydraulics:11.417}
\begin{split}\beta_{t,shade} = \dfrac{g_{s,shade}}{g_{s,shade,\beta_t=1}}\end{split}
\end{equation}
The \(\beta_t\) values are used in the Photosynthesis module (see section \hyperref[\detokenize{tech_note/Photosynthesis/CLM50_Tech_Note_Photosynthesis:photosynthesis}]{\ref{\detokenize{tech_note/Photosynthesis/CLM50_Tech_Note_Photosynthesis:photosynthesis}}}) to apply water stress.
The solution for \(\psi\) is saved as a new variable (vegetation water potential) and is indicative of plant water status.
The soil-to-root fluxes \(\left( q_{3,1},q_{3,2},\mbox{...},q_{3,n}\right)\) are used as the soil transpiration sink in the Richards’ equation subsurface flow equations (see section \hyperref[\detokenize{tech_note/Hydrology/CLM50_Tech_Note_Hydrology:soil-water}]{\ref{\detokenize{tech_note/Hydrology/CLM50_Tech_Note_Hydrology:soil-water}}}).


\subsubsection{Flow Diagram of Leaf Flux Calculations:}
\label{\detokenize{tech_note/Plant_Hydraulics/CLM50_Tech_Note_Plant_Hydraulics:id11}}\label{\detokenize{tech_note/Plant_Hydraulics/CLM50_Tech_Note_Plant_Hydraulics:flow-diagram-of-leaf-flux-calculations}}
PHS runs nested in the loop that solves for sensible and latent heat fluxes and temperature for vegetated surfaces (see section \hyperref[\detokenize{tech_note/Fluxes/CLM50_Tech_Note_Fluxes:sensible-and-latent-heat-fluxes-and-temperature-for-vegetated-surfaces}]{\ref{\detokenize{tech_note/Fluxes/CLM50_Tech_Note_Fluxes:sensible-and-latent-heat-fluxes-and-temperature-for-vegetated-surfaces}}}).
The scheme iterates for convergence of leaf temperature (\(T_l\)), transpiration water stress (\(\beta_t\)), and intercellular CO2 concentration (\(c_i\)).
PHS is forced by maximum transpiration (absent water stress, \(\beta_t=1\)), whereby we first solve for assimilation, stomatal conductance, and intercellular CO2 with \(\beta_{t,sun}\) and \(\beta_{t,shade}\) both set to 1.
This involves iterating to convergence of \(c_i\) (see section \hyperref[\detokenize{tech_note/Photosynthesis/CLM50_Tech_Note_Photosynthesis:photosynthesis}]{\ref{\detokenize{tech_note/Photosynthesis/CLM50_Tech_Note_Photosynthesis:photosynthesis}}}).

Next, using the solutions for \(E_{sun,max}\) and \(E_{shade,max}\), PHS solves for \(\psi\), \(\beta_{t,sun}\), and \(\beta_{t,shade}\).
The values for \(\beta_{t,sun}\), and \(\beta_{t,shade}\) are inputs to the photosynthesis routine, which now solves for attenuated photosynthesis and stomatal conductance (reflecting water stress).
Again this involves iterating to convergence of \(c_i\).
Non-linearities between \(\beta_t\) and transpiration require also iterating to convergence of \(\beta_t\).
The outermost level of iteration works towards convergence of leaf temperature, reflecting leaf surface energy balance.

\begin{figure}[htbp]
\centering
\capstart

\noindent\sphinxincludegraphics{{flow}.png}
\caption{Flow diagram of leaf flux calculations}\label{\detokenize{tech_note/Plant_Hydraulics/CLM50_Tech_Note_Plant_Hydraulics:figure-phs-flow-diagram}}\label{\detokenize{tech_note/Plant_Hydraulics/CLM50_Tech_Note_Plant_Hydraulics:id14}}\end{figure}


\section{Lake Model}
\label{\detokenize{tech_note/Lake/CLM50_Tech_Note_Lake:lake-model}}\label{\detokenize{tech_note/Lake/CLM50_Tech_Note_Lake::doc}}\label{\detokenize{tech_note/Lake/CLM50_Tech_Note_Lake:rst-lake-model}}
The lake model, denoted the \sphinxstyleemphasis{Lake, Ice, Snow, and Sediment Simulator}
(LISSS), is from {\hyperref[\detokenize{tech_note/References/CLM50_Tech_Note_References:subinetal2012a}]{\sphinxcrossref{\DUrole{std,std-ref}{Subin et al. (2012a)}}}}.
It includes extensive modifications to the lake code of
{\hyperref[\detokenize{tech_note/References/CLM50_Tech_Note_References:zengetal2002}]{\sphinxcrossref{\DUrole{std,std-ref}{Zeng et al. (2002)}}}} used in CLM
versions 2 through 4, which utilized concepts from the lake models of
{\hyperref[\detokenize{tech_note/References/CLM50_Tech_Note_References:bonan1996}]{\sphinxcrossref{\DUrole{std,std-ref}{Bonan (1996)}}}},
{\hyperref[\detokenize{tech_note/References/CLM50_Tech_Note_References:henderson-sellers1985}]{\sphinxcrossref{\DUrole{std,std-ref}{Henderson-Sellers  (1985)}}}},
{\hyperref[\detokenize{tech_note/References/CLM50_Tech_Note_References:henderson-sellers1986}]{\sphinxcrossref{\DUrole{std,std-ref}{Henderson-Sellers  (1986)}}}},
{\hyperref[\detokenize{tech_note/References/CLM50_Tech_Note_References:hostetlerbartlein1990}]{\sphinxcrossref{\DUrole{std,std-ref}{Hostetler and Bartlein (1990)}}}},
and the coupled lake-atmosphere model of {\hyperref[\detokenize{tech_note/References/CLM50_Tech_Note_References:hostetleretal1993}]{\sphinxcrossref{\DUrole{std,std-ref}{Hostetler et al. (1993)}}}}, {\hyperref[\detokenize{tech_note/References/CLM50_Tech_Note_References:hostetleretal1993}]{\sphinxcrossref{\DUrole{std,std-ref}{Hostetler et al. (1993)}}}}.
Lakes have spatially variable depth prescribed in the surface data (section
{\hyperref[\detokenize{tech_note/Lake/CLM50_Tech_Note_Lake:external-data-lake}]{\sphinxcrossref{\DUrole{std,std-ref}{External Data}}}}); the surface data optionally includes lake optical
extinction coeffient and horizontal fetch, currently only used for site
simulations. Lake physics includes freezing and thawing in the lake
body, resolved snow layers, and “soil” and bedrock layers below the lake
body. Temperatures and ice fractions are simulated for
\(N_{levlak} =10\) layers (for global simulations) or
\(N_{levlak} =25\) (for site simulations) with discretization
described in section \hyperref[\detokenize{tech_note/Lake/CLM50_Tech_Note_Lake:vertical-discretization-lake}]{\ref{\detokenize{tech_note/Lake/CLM50_Tech_Note_Lake:vertical-discretization-lake}}}. Lake albedo is
described in section \hyperref[\detokenize{tech_note/Lake/CLM50_Tech_Note_Lake:surface-albedo-lake}]{\ref{\detokenize{tech_note/Lake/CLM50_Tech_Note_Lake:surface-albedo-lake}}}. Lake surface fluxes
(section \hyperref[\detokenize{tech_note/Lake/CLM50_Tech_Note_Lake:surface-fluxes-and-surface-temperature-lake}]{\ref{\detokenize{tech_note/Lake/CLM50_Tech_Note_Lake:surface-fluxes-and-surface-temperature-lake}}}) generally
follow the formulations for non-vegetated surfaces, including the calculations
of aerodynamic resistances (section
\hyperref[\detokenize{tech_note/Fluxes/CLM50_Tech_Note_Fluxes:sensible-and-latent-heat-fluxes-for-non-vegetated-surfaces}]{\ref{\detokenize{tech_note/Fluxes/CLM50_Tech_Note_Fluxes:sensible-and-latent-heat-fluxes-for-non-vegetated-surfaces}}});
however, the lake surface temperature
\(T_{g}\)  (representing an infinitesimal interface layer between
the top resolved lake layer and the atmosphere) is solved for
simultaneously with the surface fluxes. After surface fluxes are
evaluated, temperatures are solved simultaneously in the resolved snow
layers (if present), the lake body, and the soil and bedrock, using the
ground heat flux \sphinxstyleemphasis{G} as a top boundary condition. Snow, soil, and
bedrock models generally follow the formulations for non-vegetated
surfaces (Chapter \hyperref[\detokenize{tech_note/Soil_Snow_Temperatures/CLM50_Tech_Note_Soil_Snow_Temperatures:rst-soil-and-snow-temperatures}]{\ref{\detokenize{tech_note/Soil_Snow_Temperatures/CLM50_Tech_Note_Soil_Snow_Temperatures:rst-soil-and-snow-temperatures}}}), with
modifications described below.


\subsection{Vertical Discretization}
\label{\detokenize{tech_note/Lake/CLM50_Tech_Note_Lake:vertical-discretization-lake}}\label{\detokenize{tech_note/Lake/CLM50_Tech_Note_Lake:vertical-discretization}}
Currently, there is one lake modeled in each grid cell (with prescribed
or assumed depth \sphinxstyleemphasis{d}, extinction coefficient \(\eta\), and fetch
\sphinxstyleemphasis{f}), although this could be modified with changes to the CLM subgrid
decomposition algorithm in future model versions. As currently
implemented, the lake consists of 0-5 snow layers; water and ice layers
(10 for global simulations and 25 for site simulations) comprising the
“lake body;” 10 “soil” layers; and 5 bedrock layers. Each lake body
layer has a fixed water mass (set by the nominal layer thickness and the
liquid density), with frozen mass-fraction \sphinxstyleemphasis{I} a state variable.
Resolved snow layers are present if the snow thickness
\(z_{sno} \ge s_{\min }\)  , where \sphinxstyleemphasis{s}$_{\text{min}}$ = 4 cm by
default, and is adjusted for model timesteps other than 1800 s in order
to maintain numerical stability (section \hyperref[\detokenize{tech_note/Lake/CLM50_Tech_Note_Lake:modifications-to-snow-layer-logic-lake}]{\ref{\detokenize{tech_note/Lake/CLM50_Tech_Note_Lake:modifications-to-snow-layer-logic-lake}}}). For global simulations
with 10 body layers, the default (50 m lake) body layer thicknesses are
given by: \(\Delta z_{i}\)  of 0.1, 1, 2, 3, 4, 5, 7, 7, 10.45, and
10.45 m, with node depths \(z_{i}\)  located at the center of each
layer (i.e., 0.05, 0.6, 2.1, 4.6, 8.1, 12.6, 18.6, 25.6, 34.325, 44.775
m). For site simulations with 25 layers, the default thicknesses are
(m): 0.1 for layer 1; 0.25 for layers 2-5; 0.5 for layers 6-9; 0.75 for
layers 10-13; 2 for layers 14-15; 2.5 for layers 16-17; 3.5 for layers
18-21; and 5.225 for layers 22-25. For lakes with depth \sphinxstyleemphasis{d}
\(\neq\) 50 m and \sphinxstyleemphasis{d} \(\ge\) 1 m, the top
layer is kept at 10 cm and the other 9 layer thicknesses are adjusted to
maintain fixed proportions. For lakes with \sphinxstyleemphasis{d} \(<\) 1 m, all layers
have equal thickness. Thicknesses of snow, soil, and bedrock layers
follow the scheme used over non-vegetated surfaces (Chapter \hyperref[\detokenize{tech_note/Soil_Snow_Temperatures/CLM50_Tech_Note_Soil_Snow_Temperatures:rst-soil-and-snow-temperatures}]{\ref{\detokenize{tech_note/Soil_Snow_Temperatures/CLM50_Tech_Note_Soil_Snow_Temperatures:rst-soil-and-snow-temperatures}}}), with
modifications to the snow layer thickness rules to keep snow layers at
least as thick as \sphinxstyleemphasis{s}$_{\text{min}}$ (section \hyperref[\detokenize{tech_note/Lake/CLM50_Tech_Note_Lake:modifications-to-snow-layer-logic-lake}]{\ref{\detokenize{tech_note/Lake/CLM50_Tech_Note_Lake:modifications-to-snow-layer-logic-lake}}}).


\subsection{External Data}
\label{\detokenize{tech_note/Lake/CLM50_Tech_Note_Lake:external-data-lake}}\label{\detokenize{tech_note/Lake/CLM50_Tech_Note_Lake:external-data}}
As discussed in {\hyperref[\detokenize{tech_note/References/CLM50_Tech_Note_References:subinetal2012a}]{\sphinxcrossref{\DUrole{std,std-ref}{Subin et al. (2012a, b)}}}}, the
Global Lake and Wetland Database ({\hyperref[\detokenize{tech_note/References/CLM50_Tech_Note_References:lehnerdoll2004}]{\sphinxcrossref{\DUrole{std,std-ref}{Lehner and Doll 2004}}}})
is currently used to prescribe lake fraction in each land model grid cell,
for a total of 2.3 million km$^{\text{-2}}$. As in
{\hyperref[\detokenize{tech_note/References/CLM50_Tech_Note_References:subinetal2012a}]{\sphinxcrossref{\DUrole{std,std-ref}{Subin et al. (2012a, b)}}}}, the
{\hyperref[\detokenize{tech_note/References/CLM50_Tech_Note_References:kourzenevaetal2012}]{\sphinxcrossref{\DUrole{std,std-ref}{Kourzeneva et al. (2012)}}}} global gridded dataset is currently
used to estimate a mean lake depth in each grid cell, based on interpolated
compilations of geographic information.


\subsection{Surface Albedo}
\label{\detokenize{tech_note/Lake/CLM50_Tech_Note_Lake:surface-albedo-lake}}\label{\detokenize{tech_note/Lake/CLM50_Tech_Note_Lake:surface-albedo}}
For direct radiation, the albedo \sphinxstyleemphasis{a} for lakes with ground temperature
\({T}_{g}\) (K) above freezing is given by ({\hyperref[\detokenize{tech_note/References/CLM50_Tech_Note_References:pivovarov1972}]{\sphinxcrossref{\DUrole{std,std-ref}{Pivovarov, 1972}}}})
\phantomsection\label{\detokenize{tech_note/Lake/CLM50_Tech_Note_Lake:equation-12.1}}\begin{equation}\label{equation:tech_note/Lake/CLM50_Tech_Note_Lake:12.1}
\begin{split}a=\frac{0.5}{\cos z+0.15}\end{split}
\end{equation}
where \sphinxstyleemphasis{z} is the zenith angle. For diffuse radiation, the expression in
eq. is integrated over the full sky to yield \sphinxstyleemphasis{a} = 0.10.

For frozen lakes without resolved snow layers, the albedo at cold
temperatures \sphinxstyleemphasis{a}$_{\text{0}}$ is 0.60 for visible and 0.40 for near
infrared radiation. As the temperature at the ice surface,
\({T}_{g}\), approaches freezing {[} \({T}_{f}\) (K) (\hyperref[\detokenize{tech_note/Ecosystem/CLM50_Tech_Note_Ecosystem:table-physical-constants}]{Table \ref{\detokenize{tech_note/Ecosystem/CLM50_Tech_Note_Ecosystem:table-physical-constants}}}){]}, the albedo is relaxed towards 0.10 based on
{\hyperref[\detokenize{tech_note/References/CLM50_Tech_Note_References:mironovetal2010}]{\sphinxcrossref{\DUrole{std,std-ref}{Mironov et al. (2010)}}}}:
\phantomsection\label{\detokenize{tech_note/Lake/CLM50_Tech_Note_Lake:equation-12.2}}\begin{equation}\label{equation:tech_note/Lake/CLM50_Tech_Note_Lake:12.2}
\begin{split}a=a_{0} \left(1-x\right)+0.10x,x=\exp \left(-95\frac{T_{f} -T_{g} }{T_{f} } \right)\end{split}
\end{equation}
where \sphinxstyleemphasis{a} is restricted to be no less than that given in \eqref{equation:tech_note/Lake/CLM50_Tech_Note_Lake:12.1}.

For frozen lakes with resolved snow layers, the reflectance of the ice
surface is fixed at \sphinxstyleemphasis{a}$_{\text{0}}$, and the snow reflectance is
calculated as over non-vegetated surfaces (Chapter \hyperref[\detokenize{tech_note/Surface_Albedos/CLM50_Tech_Note_Surface_Albedos:rst-surface-albedos}]{\ref{\detokenize{tech_note/Surface_Albedos/CLM50_Tech_Note_Surface_Albedos:rst-surface-albedos}}}).
These two reflectances are combined to obtain the snow-fraction-weighted albedo as
in over non-vegetated surfaces (Chapter \hyperref[\detokenize{tech_note/Surface_Albedos/CLM50_Tech_Note_Surface_Albedos:rst-surface-albedos}]{\ref{\detokenize{tech_note/Surface_Albedos/CLM50_Tech_Note_Surface_Albedos:rst-surface-albedos}}}).


\subsection{Surface Fluxes and Surface Temperature}
\label{\detokenize{tech_note/Lake/CLM50_Tech_Note_Lake:surface-fluxes-and-surface-temperature}}\label{\detokenize{tech_note/Lake/CLM50_Tech_Note_Lake:surface-fluxes-and-surface-temperature-lake}}

\subsubsection{Surface Properties}
\label{\detokenize{tech_note/Lake/CLM50_Tech_Note_Lake:surface-properties-lake}}\label{\detokenize{tech_note/Lake/CLM50_Tech_Note_Lake:surface-properties}}
The fraction of shortwave radiation absorbed at the surface,
\(\beta\), depends on the lake state. If resolved snow layers are
present, then \(\beta\) is set equal to the absorption fraction
predicted by the snow-optics submodel (Chapter \hyperref[\detokenize{tech_note/Surface_Albedos/CLM50_Tech_Note_Surface_Albedos:rst-surface-albedos}]{\ref{\detokenize{tech_note/Surface_Albedos/CLM50_Tech_Note_Surface_Albedos:rst-surface-albedos}}})
for the top snow
layer. Otherwise, \(\beta\) is set equal to the near infrared
fraction of the shortwave radiation reaching the surface simulated by
the atmospheric model or atmospheric data model used for offline
simulations (Chapter \hyperref[\detokenize{tech_note/Land-Only_Mode/CLM50_Tech_Note_Land-Only_Mode:rst-land-only-mode}]{\ref{\detokenize{tech_note/Land-Only_Mode/CLM50_Tech_Note_Land-Only_Mode:rst-land-only-mode}}}). The remainder of the shortwave radiation
fraction (1 \({-}\) \(\beta\)) is absorbed in the lake
body or soil as described in section \hyperref[\detokenize{tech_note/Lake/CLM50_Tech_Note_Lake:radiation-penetration}]{\ref{\detokenize{tech_note/Lake/CLM50_Tech_Note_Lake:radiation-penetration}}}.

The surface roughnesses are functions of the lake state and atmospheric
forcing. For frozen lakes ( \(T_{g} \le T_{f}\) ) with resolved
snow layers, the momentum roughness length
\(z_{0m} =2.4 \times 10^{-3} {\rm m}\) (as over non-vegetated
surfaces; Chapter \hyperref[\detokenize{tech_note/Fluxes/CLM50_Tech_Note_Fluxes:rst-momentum-sensible-heat-and-latent-heat-fluxes}]{\ref{\detokenize{tech_note/Fluxes/CLM50_Tech_Note_Fluxes:rst-momentum-sensible-heat-and-latent-heat-fluxes}}}), and the scalar roughness lengths
(\sphinxstyleemphasis{z}$_{\text{0q}}$ for latent heat; and \sphinxstyleemphasis{z}$_{\text{0h}}$, for sensible heat) are given by
({\hyperref[\detokenize{tech_note/References/CLM50_Tech_Note_References:zilitinkevich1970}]{\sphinxcrossref{\DUrole{std,std-ref}{Zilitinkevich 1970}}}})
\phantomsection\label{\detokenize{tech_note/Lake/CLM50_Tech_Note_Lake:equation-12.3}}\begin{equation}\label{equation:tech_note/Lake/CLM50_Tech_Note_Lake:12.3}
\begin{split}\begin{array}{l} {R_{0} =\frac{z_{0m} u_{*} }{\nu } ,} \\ {z_{0h} =z_{0q} =z_{0m} \exp \left\{-0.13R_{0} ^{0.45} \right\}} \end{array}\end{split}
\end{equation}
where \(R_{0}\) is the near-surface atmospheric roughness
Reynolds number, \(z_{0h}\) is the roughness
length for sensible heat,  \(z_{0q}\) is the
roughness length for latent heat, \(\nu\) (m$^{\text{2}}$ s$^{\text{-1}}$) is the kinematic viscosity of air, and
\(u_{\*}\)  (m s$^{\text{-1}}$) is the friction velocity in the
atmospheric surface layer. For frozen lakes without resolved snow
layers, \(z_{0m} =1\times 10^{-3} {\rm m}\) ({\hyperref[\detokenize{tech_note/References/CLM50_Tech_Note_References:subinetal2012a}]{\sphinxcrossref{\DUrole{std,std-ref}{Subin et al. (2012a)}}}}),
and the scalar roughness lengths are given by .

For unfrozen lakes, \sphinxstyleemphasis{z}$_{\text{0m}}$ is given by ({\hyperref[\detokenize{tech_note/References/CLM50_Tech_Note_References:subinetal2012a}]{\sphinxcrossref{\DUrole{std,std-ref}{Subin et al. (2012a)}}}})
\phantomsection\label{\detokenize{tech_note/Lake/CLM50_Tech_Note_Lake:equation-12.4}}\begin{equation}\label{equation:tech_note/Lake/CLM50_Tech_Note_Lake:12.4}
\begin{split}z_{0m} =\max \left(\frac{\alpha \nu }{u_{*} } ,C\frac{u_{*} ^{2} }{g} \right)\end{split}
\end{equation}
where \(\alpha\) = 0.1, \(\nu\) is the kinematic viscosity
of air given below, \sphinxstyleemphasis{C} is the effective Charnock coefficient given
below, and \sphinxstyleemphasis{g} is the acceleration of gravity (\hyperref[\detokenize{tech_note/Ecosystem/CLM50_Tech_Note_Ecosystem:table-physical-constants}]{Table \ref{\detokenize{tech_note/Ecosystem/CLM50_Tech_Note_Ecosystem:table-physical-constants}}}). The kinematic
viscosity is given by
\phantomsection\label{\detokenize{tech_note/Lake/CLM50_Tech_Note_Lake:equation-12.5}}\begin{equation}\label{equation:tech_note/Lake/CLM50_Tech_Note_Lake:12.5}
\begin{split}\nu =\nu _{0} \left(\frac{T_{g} }{T_{0} } \right)^{1.5} \frac{P_{0} }{P_{ref} }\end{split}
\end{equation}
where
\(\nu _{0} =1.51\times 10^{-5} {\textstyle\frac{{\rm m}^{{\rm 2}} }{{\rm s}}}\)
, \(T_{0} ={\rm 293.15\; K}\),
\(P_{0} =1.013\times 10^{5} {\rm \; Pa}\) , and
\(P_{ref}\) is the pressure at the atmospheric reference
height. The Charnock coefficient \sphinxstyleemphasis{C} is a function of the lake fetch \sphinxstyleemphasis{F}
(m), given in the surface data or set to 25 times the lake depth \sphinxstyleemphasis{d} by
default:
\phantomsection\label{\detokenize{tech_note/Lake/CLM50_Tech_Note_Lake:equation-12.6}}\begin{equation}\label{equation:tech_note/Lake/CLM50_Tech_Note_Lake:12.6}
\begin{split}\begin{array}{l} {C=C_{\min } +(C_{\max } -C_{\min } )\exp \left\{-\min \left(A,B\right)\right\}} \\ {A={\left(\frac{Fg}{u_{\*} ^{2} } \right)^{{1\mathord{\left/ {\vphantom {1 3}} \right. \kern-\nulldelimiterspace} 3} } \mathord{\left/ {\vphantom {\left(\frac{Fg}{u_{\*} ^{2} } \right)^{{1\mathord{\left/ {\vphantom {1 3}} \right. \kern-\nulldelimiterspace} 3} }  f_{c} }} \right. \kern-\nulldelimiterspace} f_{c} } } \\ {B=\varepsilon \frac{\sqrt{dg} }{u} } \end{array}\end{split}
\end{equation}
where \sphinxstyleemphasis{A} and \sphinxstyleemphasis{B} define the fetch- and depth-limitation, respectively;
\(C_{\min } =0.01\) , \(C_{\max } =0.01\),
\(\varepsilon =1\) , \(f_{c} =100\) , and \sphinxstyleemphasis{u} (m
s$^{\text{-1}}$) is the atmospheric forcing wind.


\subsubsection{Surface Flux Solution}
\label{\detokenize{tech_note/Lake/CLM50_Tech_Note_Lake:surface-flux-solution}}\label{\detokenize{tech_note/Lake/CLM50_Tech_Note_Lake:surface-flux-solution-lake}}
Conservation of energy at the lake surface requires
\phantomsection\label{\detokenize{tech_note/Lake/CLM50_Tech_Note_Lake:equation-12.7}}\begin{equation}\label{equation:tech_note/Lake/CLM50_Tech_Note_Lake:12.7}
\begin{split}\beta \vec{S}_{g} -\vec{L}_{g} -H_{g} -\lambda E_{g} -G=0\end{split}
\end{equation}
where \(\vec{S}_{g}\) is the absorbed solar radiation in the lake,
\(\beta\) is the fraction absorbed at the surface,
\(\vec{L}_{g}\) is the net emitted longwave radiation (+ upwards),
\(H_{g}\) is the sensible heat flux (+ upwards),
\(E_{g}\) is the water vapor flux (+ upwards), and \sphinxstyleemphasis{G} is the
ground heat flux (+ downwards). All of these fluxes depend implicitly on
the temperature at the lake surface \({T}_{g}\).
\(\lambda\)  converts \(E_{g}\)  to an energy flux based on
\phantomsection\label{\detokenize{tech_note/Lake/CLM50_Tech_Note_Lake:equation-12.8}}\begin{equation}\label{equation:tech_note/Lake/CLM50_Tech_Note_Lake:12.8}
\begin{split}\lambda =\left\{\begin{array}{l} {\lambda _{sub} \qquad T_{g} \le T_{f} } \\ {\lambda _{vap} \qquad T_{g} >T_{f} } \end{array}\right\}.\end{split}
\end{equation}
The sensible heat flux (W m$^{\text{-2}}$) is
\phantomsection\label{\detokenize{tech_note/Lake/CLM50_Tech_Note_Lake:equation-12.9}}\begin{equation}\label{equation:tech_note/Lake/CLM50_Tech_Note_Lake:12.9}
\begin{split}H_{g} =-\rho _{atm} C_{p} \frac{\left(\theta _{atm} -T_{g} \right)}{r_{ah} }\end{split}
\end{equation}
where \(\rho _{atm}\)  is the density of moist air (kg
m$^{\text{-3}}$) (Chapter \hyperref[\detokenize{tech_note/Fluxes/CLM50_Tech_Note_Fluxes:rst-momentum-sensible-heat-and-latent-heat-fluxes}]{\ref{\detokenize{tech_note/Fluxes/CLM50_Tech_Note_Fluxes:rst-momentum-sensible-heat-and-latent-heat-fluxes}}}), \(C_{p}\)  is the specific heat
capacity of air (J kg$^{\text{-1}}$ K$^{\text{-1}}$) (\hyperref[\detokenize{tech_note/Ecosystem/CLM50_Tech_Note_Ecosystem:table-physical-constants}]{Table \ref{\detokenize{tech_note/Ecosystem/CLM50_Tech_Note_Ecosystem:table-physical-constants}}}),
\(\theta _{atm}\)  is the atmospheric potential temperature (K)
(Chapter \hyperref[\detokenize{tech_note/Fluxes/CLM50_Tech_Note_Fluxes:rst-momentum-sensible-heat-and-latent-heat-fluxes}]{\ref{\detokenize{tech_note/Fluxes/CLM50_Tech_Note_Fluxes:rst-momentum-sensible-heat-and-latent-heat-fluxes}}}), \(T_{g}\)  is the lake surface temperature (K) (at an
infinitesimal interface just above the top resolved model layer: snow,
ice, or water), and \(r_{ah}\)  is the aerodynamic resistance to
sensible heat transfer (s m$^{\text{-1}}$) (section \hyperref[\detokenize{tech_note/Fluxes/CLM50_Tech_Note_Fluxes:monin-obukhov-similarity-theory}]{\ref{\detokenize{tech_note/Fluxes/CLM50_Tech_Note_Fluxes:monin-obukhov-similarity-theory}}}).

The water vapor flux (kg m$^{\text{-2}}$ s$^{\text{-1}}$) is
\phantomsection\label{\detokenize{tech_note/Lake/CLM50_Tech_Note_Lake:equation-12.10}}\begin{equation}\label{equation:tech_note/Lake/CLM50_Tech_Note_Lake:12.10}
\begin{split}E_{g} =-\frac{\rho _{atm} \left(q_{atm} -q_{sat}^{T_{g} } \right)}{r_{aw} }\end{split}
\end{equation}
where \(q_{atm}\)  is the atmospheric specific humidity (kg
kg$^{\text{-1}}$) (section \hyperref[\detokenize{tech_note/Ecosystem/CLM50_Tech_Note_Ecosystem:atmospheric-coupling}]{\ref{\detokenize{tech_note/Ecosystem/CLM50_Tech_Note_Ecosystem:atmospheric-coupling}}}),
\(q_{sat}^{T_{g} }\) is the saturated specific humidity
(kg kg$^{\text{-1}}$) (section \hyperref[\detokenize{tech_note/Fluxes/CLM50_Tech_Note_Fluxes:saturation-vapor-pressure}]{\ref{\detokenize{tech_note/Fluxes/CLM50_Tech_Note_Fluxes:saturation-vapor-pressure}}}) at
the lake surface temperature \(T_{g}\) , and \(r_{aw}\)  is the
aerodynamic resistance to water vapor transfer (s m$^{\text{-1}}$)
(section \hyperref[\detokenize{tech_note/Fluxes/CLM50_Tech_Note_Fluxes:monin-obukhov-similarity-theory}]{\ref{\detokenize{tech_note/Fluxes/CLM50_Tech_Note_Fluxes:monin-obukhov-similarity-theory}}}).

The zonal and meridional momentum fluxes are
\phantomsection\label{\detokenize{tech_note/Lake/CLM50_Tech_Note_Lake:equation-12.11}}\begin{equation}\label{equation:tech_note/Lake/CLM50_Tech_Note_Lake:12.11}
\begin{split}\tau _{x} =-\rho _{atm} \frac{u_{atm} }{r_{atm} }\end{split}
\end{equation}\phantomsection\label{\detokenize{tech_note/Lake/CLM50_Tech_Note_Lake:equation-12.12}}\begin{equation}\label{equation:tech_note/Lake/CLM50_Tech_Note_Lake:12.12}
\begin{split}\tau _{y} =-\rho _{atm} \frac{v_{atm} }{r_{atm} }\end{split}
\end{equation}
where \(u_{atm}\)  and \(v_{atm}\)  are the zonal and
meridional atmospheric winds (m s$^{\text{-1}}$) (section
\hyperref[\detokenize{tech_note/Ecosystem/CLM50_Tech_Note_Ecosystem:atmospheric-coupling}]{\ref{\detokenize{tech_note/Ecosystem/CLM50_Tech_Note_Ecosystem:atmospheric-coupling}}}), and
\(r_{am}\)  is the aerodynamic resistance for momentum (s
m$^{\text{-1}}$) (section \hyperref[\detokenize{tech_note/Fluxes/CLM50_Tech_Note_Fluxes:monin-obukhov-similarity-theory}]{\ref{\detokenize{tech_note/Fluxes/CLM50_Tech_Note_Fluxes:monin-obukhov-similarity-theory}}}).

The heat flux into the lake surface \(G\) (W m$^{\text{-2}}$) is
\phantomsection\label{\detokenize{tech_note/Lake/CLM50_Tech_Note_Lake:equation-12.13}}\begin{equation}\label{equation:tech_note/Lake/CLM50_Tech_Note_Lake:12.13}
\begin{split}G=\frac{2\lambda _{T} }{\Delta z_{T} } \left(T_{g} -T_{T} \right)\end{split}
\end{equation}
where \(\lambda _{T}\)  is the thermal conductivity (W
m$^{\text{-1}}$ K$^{\text{-1}}$), \(\Delta z_{T}\)  is the
thickness (m), and \(T_{T}\)  is the temperature (K) of the top
resolved lake layer (snow, ice, or water). The top thermal conductivity
\(\lambda _{T}\)  of unfrozen lakes ( \(T_{g} >T_{f}\) )
includes conductivities due to molecular ( \(\lambda _{liq}\) ) and
eddy (\(\lambda _{K}\) ) diffusivities (section \hyperref[\detokenize{tech_note/Lake/CLM50_Tech_Note_Lake:eddy-diffusivity-and-thermal-conductivities}]{\ref{\detokenize{tech_note/Lake/CLM50_Tech_Note_Lake:eddy-diffusivity-and-thermal-conductivities}}}), as evaluated
in the top lake layer at the previous timestep, where
\(\lambda _{liq}\)  is the thermal conductivity of water (\hyperref[\detokenize{tech_note/Ecosystem/CLM50_Tech_Note_Ecosystem:table-physical-constants}]{Table \ref{\detokenize{tech_note/Ecosystem/CLM50_Tech_Note_Ecosystem:table-physical-constants}}}). For frozen lakes without resolved snow layers,
\(\lambda _{T} =\lambda _{ice}\)  . When resolved snow layers are
present, \(\lambda _{T}\) is calculated based on the water
content, ice content, and thickness of the top snow layer, as for
non-vegetated surfaces.

The absorbed solar radiation \(\vec{S}_{g}\)  is
\phantomsection\label{\detokenize{tech_note/Lake/CLM50_Tech_Note_Lake:equation-12.14}}\begin{equation}\label{equation:tech_note/Lake/CLM50_Tech_Note_Lake:12.14}
\begin{split}\vec{S}_{g} =\sum _{\Lambda }S_{atm} \, \downarrow _{\Lambda }^{\mu } \left(1-\alpha _{g,\, \Lambda }^{\mu } \right) +S_{atm} \, \downarrow _{\Lambda } \left(1-\alpha _{g,\, \Lambda } \right)\end{split}
\end{equation}
where \(S_{atm} \, \downarrow _{\Lambda }^{\mu }\)  and
\(S_{atm} \, \downarrow _{\Lambda }\)  are the incident direct beam
and diffuse solar fluxes (W m$^{\text{-2}}$) and \(\Lambda\)
denotes the visible (\(<\) 0.7\(\mu {\rm m}\)) and
near-infrared (\(\ge\)  0.7\(\mu {\rm m}\)) wavebands (section
\hyperref[\detokenize{tech_note/Ecosystem/CLM50_Tech_Note_Ecosystem:atmospheric-coupling}]{\ref{\detokenize{tech_note/Ecosystem/CLM50_Tech_Note_Ecosystem:atmospheric-coupling}}}), and \(\alpha _{g,\, \Lambda }^{\mu }\)  and
\(\alpha _{g,\, \mu }\)  are the direct beam and diffuse lake
albedos (section \hyperref[\detokenize{tech_note/Lake/CLM50_Tech_Note_Lake:surface-albedo-lake}]{\ref{\detokenize{tech_note/Lake/CLM50_Tech_Note_Lake:surface-albedo-lake}}}).

The net emitted longwave radiation is
\phantomsection\label{\detokenize{tech_note/Lake/CLM50_Tech_Note_Lake:equation-12.15}}\begin{equation}\label{equation:tech_note/Lake/CLM50_Tech_Note_Lake:12.15}
\begin{split}\vec{L}_{g} =L_{g} \, \uparrow -L_{atm} \, \downarrow\end{split}
\end{equation}
where \(L_{g} \, \uparrow\)  is the upward longwave radiation from
the surface, \(L_{atm} \, \downarrow\)  is the downward atmospheric
longwave radiation (section \hyperref[\detokenize{tech_note/Ecosystem/CLM50_Tech_Note_Ecosystem:atmospheric-coupling}]{\ref{\detokenize{tech_note/Ecosystem/CLM50_Tech_Note_Ecosystem:atmospheric-coupling}}}). The upward
longwave radiation from the surface is
\phantomsection\label{\detokenize{tech_note/Lake/CLM50_Tech_Note_Lake:equation-12.16}}\begin{equation}\label{equation:tech_note/Lake/CLM50_Tech_Note_Lake:12.16}
\begin{split}L\, \uparrow =\left(1-\varepsilon _{g} \right)L_{atm} \, \downarrow +\varepsilon _{g} \sigma \left(T_{g}^{n} \right)^{4} +4\varepsilon _{g} \sigma \left(T_{g}^{n} \right)^{3} \left(T_{g}^{n+1} -T_{g}^{n} \right)\end{split}
\end{equation}
where \(\varepsilon _{g} =0.97\) is the lake surface emissivity,
\(\sigma\) is the Stefan-Boltzmann constant (W m$^{\text{-2}}$ K$^{\text{-4}}$) (\hyperref[\detokenize{tech_note/Ecosystem/CLM50_Tech_Note_Ecosystem:table-physical-constants}]{Table \ref{\detokenize{tech_note/Ecosystem/CLM50_Tech_Note_Ecosystem:table-physical-constants}}}), and
\(T_{g}^{n+1} -T_{g}^{n}\) is the difference in lake surface
temperature between Newton-Raphson iterations (see below).

The sensible heat \(H_{g}\) , the water vapor flux \(E_{g}\)
through its dependence on the saturated specific humidity, the net
longwave radiation \(\vec{L}_{g}\) , and the ground heat flux
\(G\), all depend on the lake surface temperature \(T_{g}\) .
Newton-Raphson iteration is applied to solve for \(T_{g}\)  and the
surface fluxes as
\phantomsection\label{\detokenize{tech_note/Lake/CLM50_Tech_Note_Lake:equation-12.17}}\begin{equation}\label{equation:tech_note/Lake/CLM50_Tech_Note_Lake:12.17}
\begin{split}\Delta T_{g} =\frac{\beta \overrightarrow{S}_{g} -\overrightarrow{L}_{g} -H_{g} -\lambda E_{g} -G}{\frac{\partial \overrightarrow{L}_{g} }{\partial T_{g} } +\frac{\partial H_{g} }{\partial T_{g} } +\frac{\partial \lambda E_{g} }{\partial T_{g} } +\frac{\partial G}{\partial T_{g} } }\end{split}
\end{equation}
where \(\Delta T_{g} =T_{g}^{n+1} -T_{g}^{n}\)  and the subscript
“n” indicates the iteration. Therefore, the surface temperature
\(T_{g}^{n+1}\)  can be written as
\phantomsection\label{\detokenize{tech_note/Lake/CLM50_Tech_Note_Lake:equation-12.18}}\begin{equation}\label{equation:tech_note/Lake/CLM50_Tech_Note_Lake:12.18}
\begin{split}T_{g}^{n+1} =\frac{\beta \overrightarrow{S}_{g} -\overrightarrow{L}_{g} -H_{g} -\lambda E_{g} -G+T_{g}^{n} \left(\frac{\partial \overrightarrow{L}_{g} }{\partial T_{g} } +\frac{\partial H_{g} }{\partial T_{g} } +\frac{\partial \lambda E_{g} }{\partial T_{g} } +\frac{\partial G}{\partial T_{g} } \right)}{\frac{\partial \overrightarrow{L}_{g} }{\partial T_{g} } +\frac{\partial H_{g} }{\partial T_{g} } +\frac{\partial \lambda E_{g} }{\partial T_{g} } +\frac{\partial G}{\partial T_{g} } }\end{split}
\end{equation}
where the partial derivatives are
\phantomsection\label{\detokenize{tech_note/Lake/CLM50_Tech_Note_Lake:equation-12.19}}\begin{equation}\label{equation:tech_note/Lake/CLM50_Tech_Note_Lake:12.19}
\begin{split}\frac{\partial \overrightarrow{L}_{g} }{\partial T_{g} } =4\varepsilon _{g} \sigma \left(T_{g}^{n} \right)^{3} ,\end{split}
\end{equation}\phantomsection\label{\detokenize{tech_note/Lake/CLM50_Tech_Note_Lake:equation-12.20}}\begin{equation}\label{equation:tech_note/Lake/CLM50_Tech_Note_Lake:12.20}
\begin{split}\frac{\partial H_{g} }{\partial T_{g} } =\frac{\rho _{atm} C_{p} }{r_{ah} } ,\end{split}
\end{equation}\phantomsection\label{\detokenize{tech_note/Lake/CLM50_Tech_Note_Lake:equation-12.21}}\begin{equation}\label{equation:tech_note/Lake/CLM50_Tech_Note_Lake:12.21}
\begin{split}\frac{\partial \lambda E_{g} }{\partial T_{g} } =\frac{\lambda \rho _{atm} }{r_{aw} } \frac{dq_{sat}^{T_{g} } }{dT_{g} } ,\end{split}
\end{equation}\phantomsection\label{\detokenize{tech_note/Lake/CLM50_Tech_Note_Lake:equation-12.22}}\begin{equation}\label{equation:tech_note/Lake/CLM50_Tech_Note_Lake:12.22}
\begin{split}\frac{\partial G}{\partial T_{g} } =\frac{2\lambda _{T} }{\Delta z_{T} } .\end{split}
\end{equation}
The fluxes of momentum, sensible heat, and water vapor are solved for
simultaneously with lake surface temperature as follows. The
stability-related equations are the same as for non-vegetated surfaces
(section \hyperref[\detokenize{tech_note/Fluxes/CLM50_Tech_Note_Fluxes:sensible-and-latent-heat-fluxes-for-non-vegetated-surfaces}]{\ref{\detokenize{tech_note/Fluxes/CLM50_Tech_Note_Fluxes:sensible-and-latent-heat-fluxes-for-non-vegetated-surfaces}}}),
except that the surface roughnesses are here (weakly varying) functions
of the friction velocity \(u_{\*}\) . To begin, \sphinxstyleemphasis{z}$_{\text{0m}}$ is set
based on the value calculated for the last timestep (for
\(T_{g} >T_{f}\) ) or based on the values in section
\hyperref[\detokenize{tech_note/Lake/CLM50_Tech_Note_Lake:surface-properties-lake}]{\ref{\detokenize{tech_note/Lake/CLM50_Tech_Note_Lake:surface-properties-lake}}} (otherwise), and the scalar roughness
lengths are set based on the relationships in section \hyperref[\detokenize{tech_note/Lake/CLM50_Tech_Note_Lake:surface-properties-lake}]{\ref{\detokenize{tech_note/Lake/CLM50_Tech_Note_Lake:surface-properties-lake}}}.
\begin{enumerate}
\item {} 
An initial guess for the wind speed \(V_{a}\)  including the
convective velocity \(U_{c}\)  is obtained from \eqref{equation:tech_note/Fluxes/CLM50_Tech_Note_Fluxes:5.24} assuming an
initial convective velocity \(U_{c} =0\) m s$^{\text{-1}}$ for
stable conditions (\(\theta _{v,\, atm} -\theta _{v,\, s} \ge 0\)
as evaluated from \eqref{equation:tech_note/Fluxes/CLM50_Tech_Note_Fluxes:5.50}) and \(U_{c} =0.5\) for unstable
conditions (\(\theta _{v,\, atm} -\theta _{v,\, s} <0\)).

\item {} 
An initial guess for the Monin-Obukhov length \(L\) is obtained
from the bulk Richardson number using \eqref{equation:tech_note/Fluxes/CLM50_Tech_Note_Fluxes:5.46} and \eqref{equation:tech_note/Fluxes/CLM50_Tech_Note_Fluxes:5.48}.

\item {} 
The following system of equations is iterated four times:

\item {} 
Heat of vaporization / sublimation \(\lambda\) (\eqref{equation:tech_note/Lake/CLM50_Tech_Note_Lake:12.8})

\item {} 
Thermal conductivity \(\lambda _{T}\) (above)

\item {} 
Friction velocity \(u_{\*}\)  (\eqref{equation:tech_note/Fluxes/CLM50_Tech_Note_Fluxes:5.32}, \eqref{equation:tech_note/Fluxes/CLM50_Tech_Note_Fluxes:5.33}, \eqref{equation:tech_note/Fluxes/CLM50_Tech_Note_Fluxes:5.34}, \eqref{equation:tech_note/Fluxes/CLM50_Tech_Note_Fluxes:5.35})

\item {} 
Potential temperature scale \(\theta _{\*}\)  (\eqref{equation:tech_note/Fluxes/CLM50_Tech_Note_Fluxes:5.37} , \eqref{equation:tech_note/Fluxes/CLM50_Tech_Note_Fluxes:5.38}, \eqref{equation:tech_note/Fluxes/CLM50_Tech_Note_Fluxes:5.39}, \eqref{equation:tech_note/Fluxes/CLM50_Tech_Note_Fluxes:5.40})

\item {} 
Humidity scale \(q_{\*}\)  (\eqref{equation:tech_note/Fluxes/CLM50_Tech_Note_Fluxes:5.41}, \eqref{equation:tech_note/Fluxes/CLM50_Tech_Note_Fluxes:5.42}, \eqref{equation:tech_note/Fluxes/CLM50_Tech_Note_Fluxes:5.43}, \eqref{equation:tech_note/Fluxes/CLM50_Tech_Note_Fluxes:5.44})

\item {} 
Aerodynamic resistances \(r_{am}\) , \(r_{ah}\) , and
\(r_{aw}\)  (\eqref{equation:tech_note/Fluxes/CLM50_Tech_Note_Fluxes:5.55}, \eqref{equation:tech_note/Fluxes/CLM50_Tech_Note_Fluxes:5.56}, \eqref{equation:tech_note/Fluxes/CLM50_Tech_Note_Fluxes:5.57})

\item {} 
Lake surface temperature \(T_{g}^{n+1}\)  (\eqref{equation:tech_note/Lake/CLM50_Tech_Note_Lake:12.18})

\item {} 
Heat of vaporization / sublimation \(\lambda\)  (\eqref{equation:tech_note/Lake/CLM50_Tech_Note_Lake:12.8})

\item {} 
Sensible heat flux \(H_{g}\)  is updated for \(T_{g}^{n+1}\)
(\eqref{equation:tech_note/Lake/CLM50_Tech_Note_Lake:12.9})

\item {} 
Water vapor flux \(E_{g}\)  is updated for \(T_{g}^{n+1}\)
as
\phantomsection\label{\detokenize{tech_note/Lake/CLM50_Tech_Note_Lake:equation-12.23}}\begin{equation}\label{equation:tech_note/Lake/CLM50_Tech_Note_Lake:12.23}
\begin{split}E_{g} =-\frac{\rho _{atm} }{r_{aw} } \left[q_{atm} -q_{sat}^{T_{g} } -\frac{\partial q_{sat}^{T_{g} } }{\partial T_{g} } \left(T_{g}^{n+1} -T_{g}^{n} \right)\right]\end{split}
\end{equation}
\end{enumerate}

where the last term on the right side of equation is the change in
saturated specific humidity due to the change in \(T_{g}\)  between
iterations.
\begin{enumerate}
\item {} 
Saturated specific humidity \(q_{sat}^{T_{g} }\)  and its
derivative \(\frac{dq_{sat}^{T_{g} } }{dT_{g} }\)  are updated
for \(T_{g}^{n+1}\)  (section \hyperref[\detokenize{tech_note/Fluxes/CLM50_Tech_Note_Fluxes:monin-obukhov-similarity-theory}]{\ref{\detokenize{tech_note/Fluxes/CLM50_Tech_Note_Fluxes:monin-obukhov-similarity-theory}}}).

\item {} 
Virtual potential temperature scale \(\theta _{v\*}\)  (\eqref{equation:tech_note/Fluxes/CLM50_Tech_Note_Fluxes:5.17})

\item {} 
Wind speed including the convective velocity, \(V_{a}\)  (\eqref{equation:tech_note/Fluxes/CLM50_Tech_Note_Fluxes:5.24})

\item {} 
Monin-Obukhov length \(L\) (\eqref{equation:tech_note/Fluxes/CLM50_Tech_Note_Fluxes:5.49})

\item {} 
Roughness lengths (\eqref{equation:tech_note/Lake/CLM50_Tech_Note_Lake:12.3}, \eqref{equation:tech_note/Lake/CLM50_Tech_Note_Lake:12.4}).

\end{enumerate}

Once the four iterations for lake surface temperature have been yielded
a tentative solution \(T_{g} ^{{'} }\) , several restrictions
are imposed in order to maintain consistency with the top lake model
layer temperature \(T_{T}\) ({\hyperref[\detokenize{tech_note/References/CLM50_Tech_Note_References:subinetal2012a}]{\sphinxcrossref{\DUrole{std,std-ref}{Subin et al. (2012a)}}}}).
\phantomsection\label{\detokenize{tech_note/Lake/CLM50_Tech_Note_Lake:equation-12.24}}\begin{equation}\label{equation:tech_note/Lake/CLM50_Tech_Note_Lake:12.24}
\begin{split}\begin{array}{l} {{\rm 1)\; }T_{T} \le T_{f} <T_{g} ^{{'} } \Rightarrow T_{g} =T_{f} ,} \\ {{\rm 2)\; }T_{T} >T_{g} ^{{'} } >T_{m} \Rightarrow T_{g} =T_{T} ,} \\ {{\rm 3)\; }T_{m} >T_{g} ^{{'} } >T_{T} >T_{f} \Rightarrow T_{g} =T_{T} } \end{array}\end{split}
\end{equation}
where \(T_{m}\) is the temperature of maximum liquid water
density, 3.85$^{\text{o}}$ C ({\hyperref[\detokenize{tech_note/References/CLM50_Tech_Note_References:hostetlerbartlein1990}]{\sphinxcrossref{\DUrole{std,std-ref}{Hostetler and Bartlein (1990)}}}}). The
first condition requires that, if there is any snow or ice present, the
surface temperature is restricted to be less than or equal to freezing.
The second and third conditions maintain convective stability in the top
lake layer.

If eq. XXX is applied, the turbulent fluxes \(H_{g}\) and
\(E_{g}\) are re-evaluated. The emitted longwave radiation and
the momentum fluxes are re-evaluated in any case. The final ground heat
flux \(G\) is calculated from the residual of the energy balance eq.
XXX in order to precisely conserve energy. XXX This ground heat flux
is taken as a prescribed flux boundary condition for the lake
temperature solution (section \hyperref[\detokenize{tech_note/Lake/CLM50_Tech_Note_Lake:boundary-conditions-lake}]{\ref{\detokenize{tech_note/Lake/CLM50_Tech_Note_Lake:boundary-conditions-lake}}}).
An energy balance check is
included at each timestep to insure that eq. XXX is obeyed to within
0.1 W m$^{\text{-2}}$.


\subsection{Lake Temperature}
\label{\detokenize{tech_note/Lake/CLM50_Tech_Note_Lake:lake-temperature}}\label{\detokenize{tech_note/Lake/CLM50_Tech_Note_Lake:id1}}

\subsubsection{Introduction}
\label{\detokenize{tech_note/Lake/CLM50_Tech_Note_Lake:introduction}}\label{\detokenize{tech_note/Lake/CLM50_Tech_Note_Lake:introduction-lake}}
The (optional-) snow, lake body (water and/or ice), soil, and bedrock
system is unified for the lake temperature solution. The governing
equation, similar to that for the snow-soil-bedrock system for vegetated
land units (Chapter \hyperref[\detokenize{tech_note/Soil_Snow_Temperatures/CLM50_Tech_Note_Soil_Snow_Temperatures:rst-soil-and-snow-temperatures}]{\ref{\detokenize{tech_note/Soil_Snow_Temperatures/CLM50_Tech_Note_Soil_Snow_Temperatures:rst-soil-and-snow-temperatures}}}), is
\phantomsection\label{\detokenize{tech_note/Lake/CLM50_Tech_Note_Lake:equation-12.25}}\begin{equation}\label{equation:tech_note/Lake/CLM50_Tech_Note_Lake:12.25}
\begin{split}\tilde{c}_{v} \frac{\partial T}{\partial t} =\frac{\partial }{\partial z} \left(\tau \frac{\partial T}{\partial z} \right)-\frac{d\phi }{dz}\end{split}
\end{equation}
where \(\tilde{c}_{v}\)  is the volumetric heat capacity (J
m$^{\text{-3}}$ K$^{\text{-1}}$), \(t\) is time (s), \sphinxstyleemphasis{T} is
the temperature (K), \(\tau\) is the thermal conductivity (W
m$^{\text{-1}}$ K$^{\text{-1}}$), and \(\phi\) is the solar
radiation (W m$^{\text{-2}}$) penetrating to depth \sphinxstyleemphasis{z} (m). The
system is discretized into \sphinxstyleemphasis{N} layers, where
\phantomsection\label{\detokenize{tech_note/Lake/CLM50_Tech_Note_Lake:equation-12.26}}\begin{equation}\label{equation:tech_note/Lake/CLM50_Tech_Note_Lake:12.26}
\begin{split}N=n_{sno} +N_{levlak} +N_{levgrnd}  ,\end{split}
\end{equation}
\(n_{sno}\)  is the number of actively modeled snow layers at the
current timestep (Chapter \hyperref[\detokenize{tech_note/Snow_Hydrology/CLM50_Tech_Note_Snow_Hydrology:rst-snow-hydrology}]{\ref{\detokenize{tech_note/Snow_Hydrology/CLM50_Tech_Note_Snow_Hydrology:rst-snow-hydrology}}}), and
\(N_{levgrnd}\) is as for vegetated land units (Chapter
\hyperref[\detokenize{tech_note/Soil_Snow_Temperatures/CLM50_Tech_Note_Soil_Snow_Temperatures:rst-soil-and-snow-temperatures}]{\ref{\detokenize{tech_note/Soil_Snow_Temperatures/CLM50_Tech_Note_Soil_Snow_Temperatures:rst-soil-and-snow-temperatures}}}). Energy is conserved as
\phantomsection\label{\detokenize{tech_note/Lake/CLM50_Tech_Note_Lake:equation-12.27}}\begin{equation}\label{equation:tech_note/Lake/CLM50_Tech_Note_Lake:12.27}
\begin{split}\frac{d}{dt} \sum _{j=1}^{N}\left[\tilde{c}_{v,j} (t)\left(T_{j} -T_{f} \right)+L_{j} (t)\right] \Delta z_{j} =G+\left(1-\beta \right)\vec{S}_{g}\end{split}
\end{equation}
where \(\tilde{c}_{v,j} (t)\)is the volumetric heat capacity of
the \sphinxstyleemphasis{j}th layer (section \hyperref[\detokenize{tech_note/Lake/CLM50_Tech_Note_Lake:radiation-penetration}]{\ref{\detokenize{tech_note/Lake/CLM50_Tech_Note_Lake:radiation-penetration}}}),
\(L_{j} (t)\)is the latent heat
of fusion per unit volume of the \sphinxstyleemphasis{j}th layer (proportional to the mass
of liquid water present), and the right-hand side represents the net
influx of energy to the lake system. Note that
\(\tilde{c}_{v,j} (t)\) can only change due to phase change (except
for changing snow layer mass, which, apart from energy required to melt
snow, represents an untracked energy flux in the land model, along with
advected energy associated with water flows in general), and this is
restricted to occur at \(T_{j} =T_{f}\) in the snow-lake-soil
system, allowing eq. to be precisely enforced and justifying the
exclusion of \(c_{v,j}\)  from the time derivative in eq. .


\subsubsection{Overview of Changes from CLM4}
\label{\detokenize{tech_note/Lake/CLM50_Tech_Note_Lake:overview-of-changes-from-clm4}}\label{\detokenize{tech_note/Lake/CLM50_Tech_Note_Lake:overview-of-changes-from-clm4-2}}
Thermal conductivities include additional eddy diffusivity, beyond the
{\hyperref[\detokenize{tech_note/References/CLM50_Tech_Note_References:hostetlerbartlein1990}]{\sphinxcrossref{\DUrole{std,std-ref}{Hostetler and Bartlein (1990)}}}} formulation,
due to unresolved processes ({\hyperref[\detokenize{tech_note/References/CLM50_Tech_Note_References:fangstefan1996}]{\sphinxcrossref{\DUrole{std,std-ref}{Fang and Stefan 1996}}}};
{\hyperref[\detokenize{tech_note/References/CLM50_Tech_Note_References:subinetal2012a}]{\sphinxcrossref{\DUrole{std,std-ref}{Subin et al. (2012a)}}}}). Lake water is now allowed to
freeze by an arbitrary fraction for each layer, which releases latent
heat and changes thermal properties. Convective mixing occurs for all
lakes, even if frozen. Soil and bedrock are included beneath the lake.
The full snow model is used if the snow thickness exceeds a threshold;
if there are resolved snow layers, radiation transfer is predicted by
the snow-optics submodel (Chapter \hyperref[\detokenize{tech_note/Surface_Albedos/CLM50_Tech_Note_Surface_Albedos:rst-surface-albedos}]{\ref{\detokenize{tech_note/Surface_Albedos/CLM50_Tech_Note_Surface_Albedos:rst-surface-albedos}}}), and the remaining radiation
penetrating the bottom snow layer is absorbed in the top layer of lake
ice; conversely, if there are no snow layers, the solar radiation
penetrating the bottom lake layer is absorbed in the top soil layer. The
lakes have variable depth, and all physics is assumed valid for
arbitrary depth, except for a depth-dependent enhanced mixing (section
\hyperref[\detokenize{tech_note/Lake/CLM50_Tech_Note_Lake:eddy-diffusivity-and-thermal-conductivities}]{\ref{\detokenize{tech_note/Lake/CLM50_Tech_Note_Lake:eddy-diffusivity-and-thermal-conductivities}}}). Finally, a previous sign error in the calculation of eddy
diffusivity (specifically, the Brunt-Väisälä frequency term; eq. ) was
corrected.


\subsubsection{Boundary Conditions}
\label{\detokenize{tech_note/Lake/CLM50_Tech_Note_Lake:boundary-conditions}}\label{\detokenize{tech_note/Lake/CLM50_Tech_Note_Lake:boundary-conditions-lake}}
The top boundary condition, imposed at the top modeled layer
\(i=j_{top}\)  , where \(j_{top} =-n_{sno} +1\), is the downwards
surface flux \sphinxstyleemphasis{G} defined by the energy flux residual during the surface
temperature solution (section \hyperref[\detokenize{tech_note/Lake/CLM50_Tech_Note_Lake:boundary-conditions-lake}]{\ref{\detokenize{tech_note/Lake/CLM50_Tech_Note_Lake:boundary-conditions-lake}}}). The bottom
boundary condition, imposed at \(i=N_{levlak} +N_{levgrnd}\)  , is zero flux.
The 2-m windspeed \(u_{2}\) (m s$^{\text{-1}}$) is used in the
calculation of eddy diffusivity:
\phantomsection\label{\detokenize{tech_note/Lake/CLM50_Tech_Note_Lake:equation-12.28}}\begin{equation}\label{equation:tech_note/Lake/CLM50_Tech_Note_Lake:12.28}
\begin{split}u_{2} =\frac{u_{*} }{k} \ln \left(\frac{2}{z_{0m} } \right)\ge 0.1.\end{split}
\end{equation}
where \(u_{*}\) is the friction velocity calculated in section
\hyperref[\detokenize{tech_note/Lake/CLM50_Tech_Note_Lake:boundary-conditions-lake}]{\ref{\detokenize{tech_note/Lake/CLM50_Tech_Note_Lake:boundary-conditions-lake}}} and \sphinxstyleemphasis{k} is the von Karman constant
(\hyperref[\detokenize{tech_note/Ecosystem/CLM50_Tech_Note_Ecosystem:table-physical-constants}]{Table \ref{\detokenize{tech_note/Ecosystem/CLM50_Tech_Note_Ecosystem:table-physical-constants}}}).


\subsubsection{Eddy Diffusivity and Thermal Conductivities}
\label{\detokenize{tech_note/Lake/CLM50_Tech_Note_Lake:eddy-diffusivity-and-thermal-conductivities}}\label{\detokenize{tech_note/Lake/CLM50_Tech_Note_Lake:id2}}
The total eddy diffusivity \(K_{W}\)  (m$^{\text{2}}$ s$^{\text{-1}}$) for liquid water in the lake body is given by ({\hyperref[\detokenize{tech_note/References/CLM50_Tech_Note_References:subinetal2012a}]{\sphinxcrossref{\DUrole{std,std-ref}{Subin et al. (2012a)}}}})
\phantomsection\label{\detokenize{tech_note/Lake/CLM50_Tech_Note_Lake:equation-12.29}}\begin{equation}\label{equation:tech_note/Lake/CLM50_Tech_Note_Lake:12.29}
\begin{split}K_{W} = m_{d} \left(\kappa _{e} +K_{ed} +\kappa _{m} \right)\end{split}
\end{equation}
where \(\kappa _{e}\)  is due to wind-driven eddies
({\hyperref[\detokenize{tech_note/References/CLM50_Tech_Note_References:hostetlerbartlein1990}]{\sphinxcrossref{\DUrole{std,std-ref}{Hostetler and Bartlein (1990)}}}}),
\(K_{ed}\)  is a modest enhanced diffusivity
intended to represent unresolved mixing processes
({\hyperref[\detokenize{tech_note/References/CLM50_Tech_Note_References:fangstefan1996}]{\sphinxcrossref{\DUrole{std,std-ref}{Fang and Stefan 1996}}}}),
\(\kappa _{m} =\frac{\lambda _{liq} }{c_{liq} \rho _{liq} }\) is
the molecular diffusivity of water (given by the ratio of its thermal
conductivity (W m$^{\text{-1}}$ K$^{\text{-1}}$) to the product of
its heat capacity (J kg$^{\text{-1}}$ K$^{\text{-1}}$) and density
(kg m$^{\text{-3}}$), values given in \hyperref[\detokenize{tech_note/Ecosystem/CLM50_Tech_Note_Ecosystem:table-physical-constants}]{Table \ref{\detokenize{tech_note/Ecosystem/CLM50_Tech_Note_Ecosystem:table-physical-constants}}}), and \(m_{d}\)
(unitless) is a factor which increases the overall diffusivity for large
lakes, intended to represent 3-dimensional mixing processes such as
caused by horizontal temperature gradients. As currently implemented,
\phantomsection\label{\detokenize{tech_note/Lake/CLM50_Tech_Note_Lake:equation-12.30}}\begin{equation}\label{equation:tech_note/Lake/CLM50_Tech_Note_Lake:12.30}
\begin{split}m_{d} =\left\{\begin{array}{l} {1,\qquad d<25{\rm m}} \\ {10,\qquad d\ge 25{\rm m}} \end{array}\right\}\end{split}
\end{equation}
where \sphinxstyleemphasis{d} is the lake depth.

The wind-driven eddy diffusion coefficient \(\kappa _{e,\, i}\) (m$^{\text{2}}$ s$^{\text{-1}}$) for layers \(1\le i\le N_{levlak}\)  is
\phantomsection\label{\detokenize{tech_note/Lake/CLM50_Tech_Note_Lake:equation-12.31}}\begin{equation}\label{equation:tech_note/Lake/CLM50_Tech_Note_Lake:12.31}
\begin{split}\kappa _{e,\, i} =\left\{\begin{array}{l} {\frac{kw^{*} z_{i} }{P_{0} \left(1+37Ri^{2} \right)} \exp \left(-k^{*} z_{i} \right)\qquad T_{g} >T_{f} } \\ {0\qquad T_{g} \le T_{f} } \end{array}\right\}\end{split}
\end{equation}
where \(P_{0} =1\) is the neutral value of the turbulent Prandtl
number, \(z_{i}\)  is the node depth (m), the surface friction
velocity (m s$^{\text{-1}}$) is \(w^{*} =0.0012u_{2}\) , and
\(k^{*}\)  varies with latitude \(\phi\)  as
\(k^{*} =6.6u_{2}^{-1.84} \sqrt{\left|\sin \phi \right|}\) . For the
bottom layer,
\(\kappa _{e,\, N_{levlak} } =\kappa _{e,N_{levlak} -1\, }\) . As in
{\hyperref[\detokenize{tech_note/References/CLM50_Tech_Note_References:hostetlerbartlein1990}]{\sphinxcrossref{\DUrole{std,std-ref}{Hostetler and Bartlein (1990)}}}},
the 2-m wind speed \(u_{2}\)  (m s$^{\text{-1}}$) (eq. ) is used to evaluate
\(w^{*}\)  and \(k^{*}\)  rather than the 10-m wind used by
{\hyperref[\detokenize{tech_note/References/CLM50_Tech_Note_References:henderson-sellers1985}]{\sphinxcrossref{\DUrole{std,std-ref}{Henderson-Sellers  (1985)}}}}.

The Richardson number is
\phantomsection\label{\detokenize{tech_note/Lake/CLM50_Tech_Note_Lake:equation-12.32}}\begin{equation}\label{equation:tech_note/Lake/CLM50_Tech_Note_Lake:12.32}
\begin{split}R_{i} =\frac{-1+\sqrt{1+\frac{40N^{2} k^{2} z_{i}^{2} }{w^{*^{2} } \exp \left(-2k^{*} z_{i} \right)} } }{20}\end{split}
\end{equation}
where
\phantomsection\label{\detokenize{tech_note/Lake/CLM50_Tech_Note_Lake:equation-12.33}}\begin{equation}\label{equation:tech_note/Lake/CLM50_Tech_Note_Lake:12.33}
\begin{split}N^{2} =\frac{g}{\rho _{i} } \frac{\partial \rho }{\partial z}\end{split}
\end{equation}
and \(g\) is the acceleration due to gravity (m s$^{\text{-2}}$)
(\hyperref[\detokenize{tech_note/Ecosystem/CLM50_Tech_Note_Ecosystem:table-physical-constants}]{Table \ref{\detokenize{tech_note/Ecosystem/CLM50_Tech_Note_Ecosystem:table-physical-constants}}}), \(\rho _{i}\)  is the density of water (kg
m$^{\text{-3}}$), and \(\frac{\partial \rho }{\partial z}\)  is
approximated as
\(\frac{\rho _{i+1} -\rho _{i} }{z_{i+1} -z_{i} }\) . Note that
because here, \sphinxstyleemphasis{z} is increasing downwards (unlike in {\hyperref[\detokenize{tech_note/References/CLM50_Tech_Note_References:hostetlerbartlein1990}]{\sphinxcrossref{\DUrole{std,std-ref}{Hostetler and Bartlein (1990)}}}}), eq. contains no negative sign; this is a correction
from CLM4. The density of water is
({\hyperref[\detokenize{tech_note/References/CLM50_Tech_Note_References:hostetlerbartlein1990}]{\sphinxcrossref{\DUrole{std,std-ref}{Hostetler and Bartlein (1990)}}}})
\phantomsection\label{\detokenize{tech_note/Lake/CLM50_Tech_Note_Lake:equation-12.34}}\begin{equation}\label{equation:tech_note/Lake/CLM50_Tech_Note_Lake:12.34}
\begin{split}\rho _{i} =1000\left(1-1.9549\times 10^{-5} \left|T_{i} -277\right|^{1.68} \right).\end{split}
\end{equation}
The enhanced diffusivity \(K_{ed}\)  is given by
({\hyperref[\detokenize{tech_note/References/CLM50_Tech_Note_References:fangstefan1996}]{\sphinxcrossref{\DUrole{std,std-ref}{Fang and Stefan 1996}}}})
\phantomsection\label{\detokenize{tech_note/Lake/CLM50_Tech_Note_Lake:equation-12.35}}\begin{equation}\label{equation:tech_note/Lake/CLM50_Tech_Note_Lake:12.35}
\begin{split}K_{ed} =1.04\times 10^{-8} \left(N^{2} \right)^{-0.43} ,N^{2} \ge 7.5\times 10^{-5} {\rm s}^{2}\end{split}
\end{equation}
where \(N^{2}\) is calculated as in eq. except for the minimum
value imposed in .

The thermal conductivity for the liquid water portion of lake body layer
\sphinxstyleemphasis{i}, \(\tau _{liq,i}\)  (W m$^{\text{-1}}$ K$^{\text{-1}}$) is
given by
\phantomsection\label{\detokenize{tech_note/Lake/CLM50_Tech_Note_Lake:equation-12.36}}\begin{equation}\label{equation:tech_note/Lake/CLM50_Tech_Note_Lake:12.36}
\begin{split}\tau _{liq,i} =K_{W} c_{liq} \rho _{liq}  .\end{split}
\end{equation}
The thermal conductivity of the ice portion of lake body layer \sphinxstyleemphasis{i},
\(\tau _{ice,eff}\) (W m$^{\text{-1}}$ K$^{\text{-1}}$), is
constant among layers, and is given by
\phantomsection\label{\detokenize{tech_note/Lake/CLM50_Tech_Note_Lake:equation-12.37}}\begin{equation}\label{equation:tech_note/Lake/CLM50_Tech_Note_Lake:12.37}
\begin{split}\tau _{ice,eff} =\tau _{ice} \frac{\rho _{ice} }{\rho _{liq} }\end{split}
\end{equation}
where \(\tau _{ice}\) (\hyperref[\detokenize{tech_note/Ecosystem/CLM50_Tech_Note_Ecosystem:table-physical-constants}]{Table \ref{\detokenize{tech_note/Ecosystem/CLM50_Tech_Note_Ecosystem:table-physical-constants}}}) is the nominal thermal
conductivity of ice: \(\tau _{ice,eff}\) is adjusted for the fact
that the nominal model layer thicknesses remain constant even while the
physical ice thickness exceeds the water thickness.

The overall thermal conductivity \(\tau _{i}\)  for layer \sphinxstyleemphasis{i} with
ice mass-fraction \(I_{i}\) is the harmonic mean of the liquid
and water fractions, assuming that they will be physically vertically
stacked, and is given by
\phantomsection\label{\detokenize{tech_note/Lake/CLM50_Tech_Note_Lake:equation-12.38}}\begin{equation}\label{equation:tech_note/Lake/CLM50_Tech_Note_Lake:12.38}
\begin{split}\tau _{i} =\frac{\tau _{ice,eff} \tau _{liq,i} }{\tau _{liq,i} I_{i} +\tau _{ice} \left(1-I_{i} \right)}  .\end{split}
\end{equation}
The thermal conductivity of snow, soil, and bedrock layers above and
below the lake, respectively, are computed identically to those for
vegetated land units (Chapter \hyperref[\detokenize{tech_note/Soil_Snow_Temperatures/CLM50_Tech_Note_Soil_Snow_Temperatures:rst-soil-and-snow-temperatures}]{\ref{\detokenize{tech_note/Soil_Snow_Temperatures/CLM50_Tech_Note_Soil_Snow_Temperatures:rst-soil-and-snow-temperatures}}}), except for the adjustment of thermal
conductivity for frost heave or excess ice ({\hyperref[\detokenize{tech_note/References/CLM50_Tech_Note_References:subinetal2012a}]{\sphinxcrossref{\DUrole{std,std-ref}{Subin et al., 2012a,
Supporting Information}}}}).


\subsubsection{Radiation Penetration}
\label{\detokenize{tech_note/Lake/CLM50_Tech_Note_Lake:radiation-penetration}}\label{\detokenize{tech_note/Lake/CLM50_Tech_Note_Lake:id3}}
If there are no resolved snow layers, the surface absorption fraction \(\beta\) is set according to the near-infrared fraction simulated
by the atmospheric model. This is apportioned to the surface energy
budget (section \hyperref[\detokenize{tech_note/Lake/CLM50_Tech_Note_Lake:surface-properties-lake}]{\ref{\detokenize{tech_note/Lake/CLM50_Tech_Note_Lake:surface-properties-lake}}}), and thus no additional radiation is absorbed in
the top \(z_{a}\)  (currently 0.6 m) of unfrozen lakes, for which
the light extinction coefficient \(\eta\) (m$^{\text{-1}}$)
varies between lake columns (eq. ). For frozen lakes
(\(T_{g} \le T_{f}\) ), the remaining \(\left(1-\beta \right)\vec{S}_{g}\)  fraction of surface absorbed
radiation that is not apportioned to the surface energy budget is
absorbed in the top lake body layer. This is a simplification, as lake
ice is partially transparent. If there are resolved snow layers, then
the snow optics submodel (Chapter \hyperref[\detokenize{tech_note/Surface_Albedos/CLM50_Tech_Note_Surface_Albedos:rst-surface-albedos}]{\ref{\detokenize{tech_note/Surface_Albedos/CLM50_Tech_Note_Surface_Albedos:rst-surface-albedos}}}) is used to calculate the snow layer
absorption (except for the absorption predicted for the top layer by the
snow optics submodel, which is assigned to the surface energy budget),
with the remainder penetrating snow layers absorbed in the top lake body
ice layer.

For unfrozen lakes, the solar radiation remaining at depth
\(z>z_{a}\)  in the lake body is given by
\phantomsection\label{\detokenize{tech_note/Lake/CLM50_Tech_Note_Lake:equation-12.39}}\begin{equation}\label{equation:tech_note/Lake/CLM50_Tech_Note_Lake:12.39}
\begin{split}\phi =\left(1-\beta \vec{S}_{g} \right)\exp \left\{-\eta \left(z-z_{a} \right)\right\} .\end{split}
\end{equation}
For all lake body layers, the flux absorbed by the layer \sphinxstyleemphasis{i},
\(\phi _{i}\)  , is
\phantomsection\label{\detokenize{tech_note/Lake/CLM50_Tech_Note_Lake:equation-12.40}}\begin{equation}\label{equation:tech_note/Lake/CLM50_Tech_Note_Lake:12.40}
\begin{split}\phi _{i} =\left(1-\beta \vec{S}_{g} \right)\left[\exp \left\{-\eta \left(z_{i} -\frac{\Delta z_{i} }{2} -z_{a} \right)\right\}-\exp \left\{-\eta \left(z_{i} +\frac{\Delta z_{i} }{2} -z_{a} \right)\right\}\right] .\end{split}
\end{equation}
The argument of each exponent is constrained to be non-negative (so
\(\phi _{i}\)  = 0 for layers contained within \({z}_{a}\)).
The remaining flux exiting the bottom of layer \(i=N_{levlak}\)  is
absorbed in the top soil layer.

The light extinction coefficient \(\eta\) (m$^{\text{-1}}$), if
not provided as external data, is a function of depth \sphinxstyleemphasis{d} (m)
({\hyperref[\detokenize{tech_note/References/CLM50_Tech_Note_References:subinetal2012a}]{\sphinxcrossref{\DUrole{std,std-ref}{Subin et al. (2012a)}}}}):
\phantomsection\label{\detokenize{tech_note/Lake/CLM50_Tech_Note_Lake:equation-12.41}}\begin{equation}\label{equation:tech_note/Lake/CLM50_Tech_Note_Lake:12.41}
\begin{split}\eta =1.1925d^{-0.424}  .\end{split}
\end{equation}

\subsubsection{Heat Capacities}
\label{\detokenize{tech_note/Lake/CLM50_Tech_Note_Lake:heat-capacities}}\label{\detokenize{tech_note/Lake/CLM50_Tech_Note_Lake:heat-capacities-lake}}
The vertically-integrated heat capacity for each lake layer,
\(\text{c}_{v,i}\) (J m$^{\text{-2}}$) is determined by the mass-weighted average over the heat capacities for the
water and ice fractions:
\phantomsection\label{\detokenize{tech_note/Lake/CLM50_Tech_Note_Lake:equation-12.42}}\begin{equation}\label{equation:tech_note/Lake/CLM50_Tech_Note_Lake:12.42}
\begin{split}c_{v,i} =\Delta z_{i} \rho _{liq} \left[c_{liq} \left(1-I_{i} \right)+c_{ice} I_{i} \right] .\end{split}
\end{equation}
Note that the density of water is used for both ice and water fractions,
as the thickness of the layer is fixed.

The total heat capacity \(c_{v,i}\)  for each soil, snow, and
bedrock layer (J m$^{\text{-2}}$) is determined as for vegetated land
units (Chapter \hyperref[\detokenize{tech_note/Soil_Snow_Temperatures/CLM50_Tech_Note_Soil_Snow_Temperatures:rst-soil-and-snow-temperatures}]{\ref{\detokenize{tech_note/Soil_Snow_Temperatures/CLM50_Tech_Note_Soil_Snow_Temperatures:rst-soil-and-snow-temperatures}}}), as the sum of the heat capacities for the water, ice,
and mineral constituents.


\subsubsection{Crank-Nicholson Solution}
\label{\detokenize{tech_note/Lake/CLM50_Tech_Note_Lake:crank-nicholson-solution-lake}}\label{\detokenize{tech_note/Lake/CLM50_Tech_Note_Lake:crank-nicholson-solution}}
The solution method for thermal diffusion is similar to that used for
soil (Chapter \hyperref[\detokenize{tech_note/Soil_Snow_Temperatures/CLM50_Tech_Note_Soil_Snow_Temperatures:rst-soil-and-snow-temperatures}]{\ref{\detokenize{tech_note/Soil_Snow_Temperatures/CLM50_Tech_Note_Soil_Snow_Temperatures:rst-soil-and-snow-temperatures}}}), except that the
lake body layers are sandwiched between the snow and soil layers
(section \hyperref[\detokenize{tech_note/Lake/CLM50_Tech_Note_Lake:introduction-lake}]{\ref{\detokenize{tech_note/Lake/CLM50_Tech_Note_Lake:introduction-lake}}}), and radiation flux is
absorbed throughout the lake layers. Before solution, layer temperatures
\(T_{i}\)  (K), thermal conductivities \(\tau _{i}\)  (W
m$^{\text{-1}}$ K$^{\text{-1}}$), heat capacities \(c_{v,i}\)
(J m$^{\text{-2}}$), and layer and interface depths from all
components are transformed into a uniform set of vectors with length
\(N=n_{sno} +N_{levlak} +N_{levgrnd}\)  and consistent units to
simplify the solution. Thermal conductivities at layer interfaces are
calculated as the harmonic mean of the conductivities of the neighboring
layers:
\phantomsection\label{\detokenize{tech_note/Lake/CLM50_Tech_Note_Lake:equation-12.43}}\begin{equation}\label{equation:tech_note/Lake/CLM50_Tech_Note_Lake:12.43}
\begin{split}\lambda _{i} =\frac{\tau _{i} \tau _{i+1} \left(z_{i+1} -z_{i} \right)}{\tau _{i} \left(z_{i+1} -\hat{z}_{i} \right)+\tau _{i+1} \left(\hat{z}_{i} -z_{i} \right)}  ,\end{split}
\end{equation}
where \(\lambda _{i}\)  is the conductivity at the interface between
layer \sphinxstyleemphasis{i} and layer \sphinxstyleemphasis{i +} 1, \(z_{i}\)  is the depth of the node of
layer \sphinxstyleemphasis{i}, and \(\hat{z}_{i}\)  is the depth of the interface below
layer \sphinxstyleemphasis{i}. Care is taken at the boundaries between snow and lake and
between lake and soil. The governing equation is discretized for each
layer as
\phantomsection\label{\detokenize{tech_note/Lake/CLM50_Tech_Note_Lake:equation-12.44}}\begin{equation}\label{equation:tech_note/Lake/CLM50_Tech_Note_Lake:12.44}
\begin{split}\frac{c_{v,i} }{\Delta t} \left(T_{i}^{n+1} -T_{i}^{n} \right)=F_{i-1} -F_{i} +\phi _{i}\end{split}
\end{equation}
where superscripts \sphinxstyleemphasis{n} + 1 and \sphinxstyleemphasis{n} denote values at the end and
beginning of the timestep \(\Delta t\), respectively, \(F_{i}\)
(W m$^{\text{-2}}$) is the downward heat flux at the bottom of layer
\sphinxstyleemphasis{i}, and \(\phi _{i}\)  is the solar radiation absorbed in layer
\sphinxstyleemphasis{i}.

Eq. is solved using the semi-implicit Crank-Nicholson Method, resulting
in a tridiagonal system of equations:
\phantomsection\label{\detokenize{tech_note/Lake/CLM50_Tech_Note_Lake:equation-12.45}}\begin{equation}\label{equation:tech_note/Lake/CLM50_Tech_Note_Lake:12.45}
\begin{split}\begin{array}{l} {r_{i} =a_{i} T_{i-1}^{n+1} +b_{i} T_{i}^{n+1} +cT_{i+1}^{n+1} ,} \\ {a_{i} =-0.5\frac{\Delta t}{c_{v,i} } \frac{\partial F_{i-1} }{\partial T_{i-1}^{n} } ,} \\ {b_{i} =1+0.5\frac{\Delta t}{c_{v,i} } \left(\frac{\partial F_{i-1} }{\partial T_{i-1}^{n} } +\frac{\partial F_{i} }{\partial T_{i}^{n} } \right),} \\ {c_{i} =-0.5\frac{\Delta t}{c_{v,i} } \frac{\partial F_{i} }{\partial T_{i}^{n} } ,} \\ {r_{i} =T_{i}^{n} +0.5\frac{\Delta t}{c_{v,i} } \left(F_{i-1} -F_{i} \right)+\frac{\Delta t}{c_{v,i} } \phi _{i} .} \end{array}\end{split}
\end{equation}
The fluxes \(F_{i}\)  are defined as follows: for the top layer,
\(F_{j_{top} -1} =2G;a_{j_{top} } =0\), where \sphinxstyleemphasis{G} is defined as in
section \hyperref[\detokenize{tech_note/Lake/CLM50_Tech_Note_Lake:boundary-conditions-lake}]{\ref{\detokenize{tech_note/Lake/CLM50_Tech_Note_Lake:boundary-conditions-lake}}} (the factor of 2 merely cancels
out the Crank-Nicholson 0.5 in the equation for \(r_{j_{top} }\) ).
For the bottom layer, \(F_{N_{levlak} +N_{levgrnd} } =0\).
For all other layers:
\phantomsection\label{\detokenize{tech_note/Lake/CLM50_Tech_Note_Lake:equation-12.46}}\begin{equation}\label{equation:tech_note/Lake/CLM50_Tech_Note_Lake:12.46}
\begin{split}F_{i} =\lambda _{i} \frac{T_{i} ^{n} -T_{i+1}^{n} }{z_{n+1} -z_{n} }  .\end{split}
\end{equation}

\subsubsection{Phase Change}
\label{\detokenize{tech_note/Lake/CLM50_Tech_Note_Lake:phase-change}}\label{\detokenize{tech_note/Lake/CLM50_Tech_Note_Lake:phase-change-lake}}
Phase change in the lake, snow, and soil is done similarly to that done
for the soil and snow for vegetated land units (Chapter \hyperref[\detokenize{tech_note/Soil_Snow_Temperatures/CLM50_Tech_Note_Soil_Snow_Temperatures:rst-soil-and-snow-temperatures}]{\ref{\detokenize{tech_note/Soil_Snow_Temperatures/CLM50_Tech_Note_Soil_Snow_Temperatures:rst-soil-and-snow-temperatures}}}), except
without the allowance for freezing point depression in soil underlying
lakes. After the heat diffusion is calculated, phase change occurs in a
given layer if the temperature is below freezing and liquid water
remains, or if the temperature is above freezing and ice remains.

If melting occurs, the available energy for melting, \(Q_{avail}\)
(J m$^{\text{-2}}$), is computed as
\phantomsection\label{\detokenize{tech_note/Lake/CLM50_Tech_Note_Lake:equation-12.47}}\begin{equation}\label{equation:tech_note/Lake/CLM50_Tech_Note_Lake:12.47}
\begin{split}Q_{avail} =\left(T_{i} -T_{f} \right)c_{v,i}\end{split}
\end{equation}
where \(T_{i}\)  is the temperature of the layer after thermal
diffusion (section \hyperref[\detokenize{tech_note/Lake/CLM50_Tech_Note_Lake:crank-nicholson-solution-lake}]{\ref{\detokenize{tech_note/Lake/CLM50_Tech_Note_Lake:crank-nicholson-solution-lake}}}), and
\(c_{v,i}\) is as calculated in section
\hyperref[\detokenize{tech_note/Lake/CLM50_Tech_Note_Lake:heat-capacities-lake}]{\ref{\detokenize{tech_note/Lake/CLM50_Tech_Note_Lake:heat-capacities-lake}}}. The mass of melt in the layer \sphinxstyleemphasis{M}
(kg m$^{\text{-2}}$) is given by
\phantomsection\label{\detokenize{tech_note/Lake/CLM50_Tech_Note_Lake:equation-12.48}}\begin{equation}\label{equation:tech_note/Lake/CLM50_Tech_Note_Lake:12.48}
\begin{split}M=\min \left\{M_{ice} ,\frac{Q_{avail} }{H_{fus} } \right\}\end{split}
\end{equation}
where \(H_{fus}\)  (J kg$^{\text{-1}}$) is the latent heat of
fusion of water (\hyperref[\detokenize{tech_note/Ecosystem/CLM50_Tech_Note_Ecosystem:table-physical-constants}]{Table \ref{\detokenize{tech_note/Ecosystem/CLM50_Tech_Note_Ecosystem:table-physical-constants}}}), and \(M_{ice}\)  is the mass of ice in
the layer: \(I_{i} \rho _{liq} \Delta z_{i}\)  for a lake body
layer, or simply the soil / snow ice content state variable
(\(w_{ice}\) ) for a soil / snow layer. The heat remainder,
\(Q_{rem}\) is given by
\phantomsection\label{\detokenize{tech_note/Lake/CLM50_Tech_Note_Lake:equation-12.49}}\begin{equation}\label{equation:tech_note/Lake/CLM50_Tech_Note_Lake:12.49}
\begin{split}Q_{rem} =Q_{avail} -MH_{fus}  .\end{split}
\end{equation}
Finally, the mass of ice in the layer \(M_{ice}\)  is adjusted
downwards by \(M\), and the temperature \(T_{i}\)  of the
layer is adjusted to
\phantomsection\label{\detokenize{tech_note/Lake/CLM50_Tech_Note_Lake:equation-12.50}}\begin{equation}\label{equation:tech_note/Lake/CLM50_Tech_Note_Lake:12.50}
\begin{split}T_{i} =T_{f} +\frac{Q_{rem} }{c'_{v,i} }\end{split}
\end{equation}
where \(c'_{v,i} =c_{v,i} +M\left(c_{liq} -c_{ice} \right)\).

If freezing occurs, \(Q_{avail}\)  is again given by but will be
negative. The melt \(M\), also negative, is given by
\phantomsection\label{\detokenize{tech_note/Lake/CLM50_Tech_Note_Lake:equation-12.51}}\begin{equation}\label{equation:tech_note/Lake/CLM50_Tech_Note_Lake:12.51}
\begin{split}M=\max \left\{-M_{liq} ,\frac{Q_{avail} }{H_{fus} } \right\}\end{split}
\end{equation}
where \(M_{liq}\)  is the mass of water in the layer:
\(\left(1-I_{i} \right)\rho _{liq} \Delta z_{i}\)  for a lake body
layer, or the soil / snow water content state variable
(\(w_{liq}\) ). The heat remainder \(Q_{rem}\)  is given by eq.
and will be negative or zero. Finally, \(M_{liq}\)  is adjusted
downwards by \(-M\) and the temperature is reset according to eq. .

In the presence of nonzero snow water \(W_{sno}\)  without resolved
snow layers over

an unfrozen top lake layer, the available energy in the top lake layer
\(\left(T_{1} -T_{f} \right)c_{v,1}\)  is used to melt the snow.
Similar to above, \(W_{sno}\)  is either completely melted and the
remainder of heat returned to the top lake layer, or the available heat
is exhausted and the top lake layer is set to freezing. The snow
thickness is adjusted downwards in proportion to the amount of melt,
maintaining constant density.


\subsubsection{Convection}
\label{\detokenize{tech_note/Lake/CLM50_Tech_Note_Lake:convection}}\label{\detokenize{tech_note/Lake/CLM50_Tech_Note_Lake:convection-lake}}
Convective mixing is based on
{\hyperref[\detokenize{tech_note/References/CLM50_Tech_Note_References:hostetleretal1993}]{\sphinxcrossref{\DUrole{std,std-ref}{Hostetler et al.’s (1993, 1994)}}}} coupled
lake-atmosphere model, adjusting the lake temperature after diffusion
and phase change to maintain a stable density profile. Unfrozen lakes
overturn when \(\rho _{i} >\rho _{i+1}\) , in which case the layer
thickness weighted average temperature for layers 1 to \(i+1\) is
applied to layers 1 to \(i+1\) and the densities are updated. This
scheme is applied iteratively to layers \(1\le i<N_{levlak} -1\).
Unstable profiles occurring at the bottom of the lake (i.e., between
layers \(i=N_{levlak} -1\) and \(i=N_{levlak}\) ) are treated
separately ({\hyperref[\detokenize{tech_note/References/CLM50_Tech_Note_References:subinetal2012a}]{\sphinxcrossref{\DUrole{std,std-ref}{Subin et al. (2012a)}}}}), as occasionally
these can be induced by
heat expelled from the sediments (not present in the original
{\hyperref[\detokenize{tech_note/References/CLM50_Tech_Note_References:hostetleretal1994}]{\sphinxcrossref{\DUrole{std,std-ref}{Hostetler et al. (1994)}}}} model). Mixing proceeds
from the bottom upward in this
case (i.e., first mixing layers \(i=N_{levlak} -1\) and
\(i=N_{levlak}\) , then checking \(i=N_{levlak} -2\) and
\(i=N_{levlak} -1\) and mixing down to \(i=N_{levlak}\)  if
needed, and on to the top), so as not to mix in with warmer over-lying
layers.

For frozen lakes, this algorithm is generalized to conserve total
enthalpy and ice content, and to maintain ice contiguous at the top of
the lake. Thus, an additional mixing criterion is added: the presence of
ice in a layer that is below a layer which is not completely frozen.
When this occurs, these two lake layers and all those above mix. Total
enthalpy \sphinxstyleemphasis{Q} is conserved as
\phantomsection\label{\detokenize{tech_note/Lake/CLM50_Tech_Note_Lake:equation-12.52}}\begin{equation}\label{equation:tech_note/Lake/CLM50_Tech_Note_Lake:12.52}
\begin{split}Q=\sum _{j=1}^{i+1}\Delta z_{j} \rho _{liq} \left(T_{j} -T_{f} \right)\left[\left(1-I_{j} \right)c_{liq} +I_{j} c_{ice} \right]  .\end{split}
\end{equation}
Once the average ice fraction \(I_{av}\)  is calculated from
\phantomsection\label{\detokenize{tech_note/Lake/CLM50_Tech_Note_Lake:equation-12.53}}\begin{equation}\label{equation:tech_note/Lake/CLM50_Tech_Note_Lake:12.53}
\begin{split}\begin{array}{l} {I_{av} =\frac{\sum _{j=1}^{i+1}I_{j} \Delta z_{j}  }{Z_{i+1} } ,} \\ {Z_{i+1} =\sum _{j=1}^{i+1}\Delta z_{j}  ,} \end{array}\end{split}
\end{equation}
the temperatures are calculated. A separate temperature is calculated
for the frozen (\(T_{froz}\) ) and unfrozen (\(T_{unfr}\) )
fractions of the mixed layers. If the total heat content \sphinxstyleemphasis{Q} is positive
(e.g. some layers will be above freezing), then the extra heat is all
assigned to the unfrozen layers, while the fully frozen layers are kept
at freezing. Conversely, if \(Q < 0\), the heat deficit will all
be assigned to the ice, and the liquid layers will be kept at freezing.
For the layer that contains both ice and liquid (if present), a weighted
average temperature will have to be calculated.

If \(Q > 0\), then \(T_{froz} =T_{f}\) , and \(T_{unfr}\)
is given by
\phantomsection\label{\detokenize{tech_note/Lake/CLM50_Tech_Note_Lake:equation-12.54}}\begin{equation}\label{equation:tech_note/Lake/CLM50_Tech_Note_Lake:12.54}
\begin{split}T_{unfr} =\frac{Q}{\rho _{liq} Z_{i+1} \left[\left(1-I_{av} \right)c_{liq} \right]} +T_{f}  .\end{split}
\end{equation}
If \(Q < 0\), then \(T_{unfr} =T_{f}\) , and \(T_{froz}\)
is given by
\phantomsection\label{\detokenize{tech_note/Lake/CLM50_Tech_Note_Lake:equation-12.55}}\begin{equation}\label{equation:tech_note/Lake/CLM50_Tech_Note_Lake:12.55}
\begin{split}T_{froz} =\frac{Q}{\rho _{liq} Z_{i+1} \left[I_{av} c_{ice} \right]} +T_{f}  .\end{split}
\end{equation}
The ice is lumped together at the top. For each lake layer \sphinxstyleemphasis{j} from 1
to \sphinxstyleemphasis{i} + 1, the ice fraction and temperature are set as follows, where
\(Z_{j} =\sum _{m=1}^{j}\Delta z_{m}\)  :
\begin{enumerate}
\item {} 
If \(Z_{j} \le Z_{i+1} I_{av}\)  , then \(I_{j} =1\) and
\(T_{j} =T_{froz}\)  .

\item {} 
Otherwise, if \(Z_{j-1} <Z_{i+1} I_{av}\)  , then the layer will
contain both ice and water. The ice fraction is given by
\(I_{j} =\frac{Z_{i+1} I_{av} -Z_{j-1} }{\Delta z_{j} }\)  . The
temperature is set to conserve the desired heat content that would be
present if the layer could have two temperatures, and then dividing
by the heat capacity of the layer to yield
\phantomsection\label{\detokenize{tech_note/Lake/CLM50_Tech_Note_Lake:equation-12.56}}\begin{equation}\label{equation:tech_note/Lake/CLM50_Tech_Note_Lake:12.56}
\begin{split}T_{j} =\frac{T_{froz} I_{j} c_{ice} +T_{unfr} \left(1-I_{j} \right)c_{liq} }{I_{j} c_{ice} +\left(1-I_{j} \right)c_{liq} }  .\end{split}
\end{equation}
\item {} 
Otherwise, \(I_{j} =0\) and \(T_{j} =T_{unfr}\) .

\end{enumerate}


\subsubsection{Energy Conservation}
\label{\detokenize{tech_note/Lake/CLM50_Tech_Note_Lake:energy-conservation}}\label{\detokenize{tech_note/Lake/CLM50_Tech_Note_Lake:energy-conservation-lake}}
To check energy conservation, the left-hand side of eq. XXX is
re-written to yield the total enthalpy of the lake system (J
m$^{\text{-2}}$) \(H_{tot}\) :
\phantomsection\label{\detokenize{tech_note/Lake/CLM50_Tech_Note_Lake:equation-12.57}}\begin{equation}\label{equation:tech_note/Lake/CLM50_Tech_Note_Lake:12.57}
\begin{split}H_{tot} =\sum _{i=j_{top} }^{N_{levlak} +N_{levgrnd} }\left[c_{v,i} \left(T_{i} -T_{f} \right)+M_{liq,i} H_{fus} \right] -W_{sno,bulk} H_{fus}\end{split}
\end{equation}
where \(M_{liq,i}\)  is the water mass of the \sphinxstyleemphasis{i}th layer (similar
to section \hyperref[\detokenize{tech_note/Lake/CLM50_Tech_Note_Lake:phase-change-lake}]{\ref{\detokenize{tech_note/Lake/CLM50_Tech_Note_Lake:phase-change-lake}}}), and \(W_{sno,bulk}\)  is the mass of snow-ice not
present in resolved snow layers. This expression is evaluated once at
the beginning and once at the end of the timestep (re-evaluating each
\(c_{v,i}\) ), and the change is compared with the net surface
energy flux to yield the error flux \(E_{soi}\)  (W
m$^{\text{-2}}$):
\phantomsection\label{\detokenize{tech_note/Lake/CLM50_Tech_Note_Lake:equation-12.58}}\begin{equation}\label{equation:tech_note/Lake/CLM50_Tech_Note_Lake:12.58}
\begin{split}E_{soi} =\frac{\Delta H_{tot} }{\Delta t} -G-\sum _{i=j_{top} }^{N_{levlak} +N_{levgrnd} }\phi _{i}\end{split}
\end{equation}
If \(\left|E_{soi} \right|<0.1\)W m$^{\text{-2}}$, it is
subtracted from the sensible heat flux and added to \sphinxstyleemphasis{G}. Otherwise, the
model is aborted.


\subsection{Lake Hydrology}
\label{\detokenize{tech_note/Lake/CLM50_Tech_Note_Lake:lake-hydrology}}\label{\detokenize{tech_note/Lake/CLM50_Tech_Note_Lake:id4}}

\subsubsection{Overview}
\label{\detokenize{tech_note/Lake/CLM50_Tech_Note_Lake:overview}}\label{\detokenize{tech_note/Lake/CLM50_Tech_Note_Lake:overview-lake-hydrology}}
Hydrology is done similarly to other impervious non-vegetated columns
(e.g., glaciers) where snow layers may be resolved but infiltration into
the permanent ground is not allowed. The water mass of lake columns is
currently maintained constant, aside from overlying snow. The water
budget is balanced with \(q_{rgwl}\)  (eq. ; kg m$^{\text{-2}}$
s$^{\text{-1}}$), a generalized runoff term for impervious land units
that may be negative.

There are some modifications to the soil and snow parameterizations as
compared with the soil in vegetated land units, or the snow overlying
other impervious columns. The soil can freeze or thaw, with the
allowance for frost heave (or the initialization of excess ice)
(sections \hyperref[\detokenize{tech_note/Lake/CLM50_Tech_Note_Lake:eddy-diffusivity-and-thermal-conductivities}]{\ref{\detokenize{tech_note/Lake/CLM50_Tech_Note_Lake:eddy-diffusivity-and-thermal-conductivities}}} and
\hyperref[\detokenize{tech_note/Lake/CLM50_Tech_Note_Lake:phase-change-lake}]{\ref{\detokenize{tech_note/Lake/CLM50_Tech_Note_Lake:phase-change-lake}}}), but no air-filled pore space is allowed in
the soil. To preserve numerical stability in the lake model (which uses
a slightly different surface flux algorithm than over other
non-vegetated land units), two changes are made to the snow model.
First, dew or frost is not allowed to be absorbed by a top snow layer
which has become completely melted during the timestep. Second, because
occasional instabilities occurred during model testing when the
Courant\textendash{}Friedrichs\textendash{}Lewy (CFL) condition was violated, due to the
explicit time-stepping integration of the surface flux solution,
resolved snow layers must be a minimum of \(s_{\min }\)  = 4 cm
thick rather than 1 cm when the default timestep of 1800 s is used.


\subsubsection{Water Balance}
\label{\detokenize{tech_note/Lake/CLM50_Tech_Note_Lake:water-balance-lake}}\label{\detokenize{tech_note/Lake/CLM50_Tech_Note_Lake:water-balance}}
The total water balance of the system is given by
\phantomsection\label{\detokenize{tech_note/Lake/CLM50_Tech_Note_Lake:equation-12.59}}\begin{equation}\label{equation:tech_note/Lake/CLM50_Tech_Note_Lake:12.59}
\begin{split}\Delta W_{sno} +\sum _{i=1}^{n_{levsoi} }\left(\Delta w_{liq,i} +\Delta w_{ice,i} \right) =\left(q_{rain} +q_{sno} -E_{g} -q_{rgwl} -q_{snwcp,\, ice} \right)\Delta t\end{split}
\end{equation}
where \(W_{sno}\)  (kg m$^{\text{-2}}$) is the total mass of snow
(both liquid and ice, in resolved snow layers or bulk snow),
\(w_{liq,i}\)  and \(w_{ice,i}\)  are the masses of water phases
(kg m$^{\text{-2}}$) in soil layer \sphinxstyleemphasis{i}, \(q_{rain}\)  and
\(q_{sno}\)  are the precipitation forcing from the atmosphere (kg
m$^{\text{-2}}$ s$^{\text{-1}}$), \(q_{snwcp,\, ice}\)  is the ice runoff
associated with snow-capping (below), \(E_{g}\)  is the ground
evaporation (section \hyperref[\detokenize{tech_note/Lake/CLM50_Tech_Note_Lake:surface-flux-solution-lake}]{\ref{\detokenize{tech_note/Lake/CLM50_Tech_Note_Lake:surface-flux-solution-lake}}}), and \(n_{levsoi}\)
is the number of hydrologically active soil layers (as opposed to dry bedrock
layers).


\subsubsection{Precipitation, Evaporation, and Runoff}
\label{\detokenize{tech_note/Lake/CLM50_Tech_Note_Lake:precipitation-evaporation-and-runoff}}\label{\detokenize{tech_note/Lake/CLM50_Tech_Note_Lake:precipitation-evaporation-and-runoff-lake}}
All precipitation reaches the ground, as there is no vegetated fraction.
As for other land types, incident snowfall accumulates (with ice mass
\(W_{sno}\)  and thickness \(z_{sno}\) ) until its thickness
exceeds a minimum thickness \(s_{\min }\) , at which point a
resolved snow layer is initiated, with water, ice, dissolved aerosol,
snow-grain radius, etc., state variables tracked by the Snow Hydrology
submodel (Chapter \hyperref[\detokenize{tech_note/Snow_Hydrology/CLM50_Tech_Note_Snow_Hydrology:rst-snow-hydrology}]{\ref{\detokenize{tech_note/Snow_Hydrology/CLM50_Tech_Note_Snow_Hydrology:rst-snow-hydrology}}}). The density of fresh snow is
assigned as for other land types (Chapter \hyperref[\detokenize{tech_note/Snow_Hydrology/CLM50_Tech_Note_Snow_Hydrology:rst-snow-hydrology}]{\ref{\detokenize{tech_note/Snow_Hydrology/CLM50_Tech_Note_Snow_Hydrology:rst-snow-hydrology}}}).
Solid precipitation is added immediately
to the snow, while liquid precipitation is added to snow layers, if they
exist, after accounting for dew, frost, and sublimation (below). If
\(z_{sno}\)  exceeds \(s_{\min }\)  after solid precipitation is
added but no snow layers are present, a new snow layer is initiated
immediately, and then dew, frost, and sublimation are accounted for.
Snow-capping is invoked if the snow depth \(z_{sno} >1000{\rm m}\),
in which case additional precipitation and frost deposition is added to
\(q_{snwcp,\, ice}\) .

If there are resolved snow layers, the generalized “evaporation”
\(E_{g}\)  (i.e., evaporation, dew, frost, and sublimation) is
treated as over other land units, except that the allowed evaporation
from the ground is unlimited (though the top snow layer cannot lose more
water mass than it contains). If there are no resolved snow layers but
\(W_{sno} >0\) and \(E_{g} >0\), sublimation
\(q_{sub,sno}\) (kg m$^{\text{-2}}$ s$^{\text{-1}}$) will be
given by
\phantomsection\label{\detokenize{tech_note/Lake/CLM50_Tech_Note_Lake:equation-12.60}}\begin{equation}\label{equation:tech_note/Lake/CLM50_Tech_Note_Lake:12.60}
\begin{split}q_{sub,sno} =\min \left\{E_{g} ,\frac{W_{sno} }{\Delta t} \right\} .\end{split}
\end{equation}
If \(E_{g} <0,T_{g} \le T_{f}\)  , and there are no resolved snow
layers or the top snow layer is not unfrozen, then the rate of frost
production \(q_{frost} =\left|E_{g} \right|\). If \(E_{g} <0\)
but the top snow layer has completely thawed during the Phase Change step
of the Lake Temperature solution (section \hyperref[\detokenize{tech_note/Lake/CLM50_Tech_Note_Lake:phase-change-lake}]{\ref{\detokenize{tech_note/Lake/CLM50_Tech_Note_Lake:phase-change-lake}}}), then
frost (or dew) is not allowed to accumulate (\(q_{frost} =0\)), to
insure that the layer is eliminated by the Snow Hydrology
(Chapter \hyperref[\detokenize{tech_note/Snow_Hydrology/CLM50_Tech_Note_Snow_Hydrology:rst-snow-hydrology}]{\ref{\detokenize{tech_note/Snow_Hydrology/CLM50_Tech_Note_Snow_Hydrology:rst-snow-hydrology}}}) code. (If \(T_{g} >T_{f}\),
then no snow is present (section \hyperref[\detokenize{tech_note/Lake/CLM50_Tech_Note_Lake:surface-flux-solution-lake}]{\ref{\detokenize{tech_note/Lake/CLM50_Tech_Note_Lake:surface-flux-solution-lake}}}), and
evaporation or dew deposition is balanced by \(q_{rgwl}\) .) The
snowpack is updated for frost and sublimation:
\phantomsection\label{\detokenize{tech_note/Lake/CLM50_Tech_Note_Lake:equation-12.61}}\begin{equation}\label{equation:tech_note/Lake/CLM50_Tech_Note_Lake:12.61}
\begin{split}W_{sno} =W_{sno} +\Delta t\left(q_{frost} -q_{sub,sno} \right) .\end{split}
\end{equation}
If there are resolved snow layers, then this update occurs using the Snow
Hydrology submodel (Chapter \hyperref[\detokenize{tech_note/Snow_Hydrology/CLM50_Tech_Note_Snow_Hydrology:rst-snow-hydrology}]{\ref{\detokenize{tech_note/Snow_Hydrology/CLM50_Tech_Note_Snow_Hydrology:rst-snow-hydrology}}}). Otherwise, the
snow ice mass is updated directly, and \(z_{sno}\) is adjusted by the same
proportion as the snow ice (i.e., maintaining the same density), unless
there was no snow before adding the frost, in which case the density is
assumed to be 250 kg m$^{\text{-3}}$.


\subsubsection{Soil Hydrology}
\label{\detokenize{tech_note/Lake/CLM50_Tech_Note_Lake:soil-hydrology}}\label{\detokenize{tech_note/Lake/CLM50_Tech_Note_Lake:soil-hydrology-lake}}
The combined water and ice soil volume fraction in a soil layer
\(\theta _{i}\)  is given by
\phantomsection\label{\detokenize{tech_note/Lake/CLM50_Tech_Note_Lake:equation-12.62}}\begin{equation}\label{equation:tech_note/Lake/CLM50_Tech_Note_Lake:12.62}
\begin{split}\theta _{i} =\frac{1}{\Delta z_{i} } \left(\frac{w_{ice,i} }{\rho _{ice} } +\frac{w_{liq,i} }{\rho _{liq} } \right) .\end{split}
\end{equation}
If \(\theta _{i} <\theta _{sat,i}\)  , the pore volume fraction at
saturation (as may occur when ice melts), then the liquid water mass is
adjusted to
\phantomsection\label{\detokenize{tech_note/Lake/CLM50_Tech_Note_Lake:equation-12.63}}\begin{equation}\label{equation:tech_note/Lake/CLM50_Tech_Note_Lake:12.63}
\begin{split}w_{liq,i} =\left(\theta _{sat,i} \Delta z_{i} -\frac{w_{ice,i} }{\rho _{ice} } \right)\rho _{liq}  .\end{split}
\end{equation}
Otherwise, if excess ice is melting and
\(w_{liq,i} >\theta _{sat,i} \rho _{liq} \Delta z_{i}\)  , then the
water in the layer is reset to
\phantomsection\label{\detokenize{tech_note/Lake/CLM50_Tech_Note_Lake:equation-12.64}}\begin{equation}\label{equation:tech_note/Lake/CLM50_Tech_Note_Lake:12.64}
\begin{split}w_{liq,i} = \theta _{sat,i} \rho _{liq} \Delta z_{i}\end{split}
\end{equation}
This allows excess ice to be initialized (and begin to be lost only
after the pore ice is melted, which is realistic if the excess ice is
found in heterogeneous chunks) but irreversibly lost when melt occurs.


\subsubsection{Modifications to Snow Layer Logic}
\label{\detokenize{tech_note/Lake/CLM50_Tech_Note_Lake:modifications-to-snow-layer-logic}}\label{\detokenize{tech_note/Lake/CLM50_Tech_Note_Lake:modifications-to-snow-layer-logic-lake}}
A thickness difference \(z_{lsa} =s_{\min } -\tilde{s}_{\min }\)
adjusts the minimum resolved snow layer thickness for lake columns as
compared to non-lake columns. The value of \(z_{lsa}\)  is chosen to
satisfy the CFL condition for the model timestep. By default,
\(\tilde{s}_{\min }\) = 1 cm and \(s_{\min }\) = 4 cm. See
{\hyperref[\detokenize{tech_note/References/CLM50_Tech_Note_References:subinetal2012a}]{\sphinxcrossref{\DUrole{std,std-ref}{Subin et al. (2012a; including Supporting Information)}}}}
for further discussion.

The rules for combining and sub-dividing snow layers (section
\hyperref[\detokenize{tech_note/Snow_Hydrology/CLM50_Tech_Note_Snow_Hydrology:snow-layer-combination-and-subdivision}]{\ref{\detokenize{tech_note/Snow_Hydrology/CLM50_Tech_Note_Snow_Hydrology:snow-layer-combination-and-subdivision}}}) are
adjusted for lakes to maintain minimum thicknesses of \(s_{\min }\)
and to increase all target layer thicknesses by \(z_{lsa}\) . The
rules for combining layers are modified by simply increasing layer
thickness thresholds by \(z_{lsa}\) . The rules for dividing snow
layers are contained in a separate subroutine that is modified for
lakes, and is a function of the number of layers and the layer
thicknesses. There are two types of operations: (a) subdividing layers
in half, and (b) shifting some volume from higher layers to lower layers
(without increasing the layer number). For subdivisions of type (a), the
thickness thresholds triggering subdivision are increased by
\(2z_{lsa}\)  for lakes. For shifts of type (b), the thickness
thresholds triggering the shifts are increased by \(z_{lsa}\) . At
the end of the modified subroutine, a snow ice and liquid balance check
are performed.

In rare instances, resolved snow layers may be present over an unfrozen
top lake body layer. In this case, the snow layers may be eliminated if
enough heat is present in the top layer to melt the snow: see
{\hyperref[\detokenize{tech_note/References/CLM50_Tech_Note_References:subinetal2012a}]{\sphinxcrossref{\DUrole{std,std-ref}{Subin et al. (2012a, Supporting Information)}}}}.


\section{Glaciers}
\label{\detokenize{tech_note/Glacier/CLM50_Tech_Note_Glacier::doc}}\label{\detokenize{tech_note/Glacier/CLM50_Tech_Note_Glacier:glaciers}}\label{\detokenize{tech_note/Glacier/CLM50_Tech_Note_Glacier:rst-glaciers}}
This chapter describes features of CLM that are specific to coupling to
an ice sheet model (in the CESM context, this is the Glimmer-CISM model;
{\hyperref[\detokenize{tech_note/References/CLM50_Tech_Note_References:lipscombsacks2012}]{\sphinxcrossref{\DUrole{std,std-ref}{Lipscomb and Sacks (2012)}}}} provide documentation
and user’s guide for Glimmer-CISM). General information about glacier
land units can be found elsewhere in this document (see Chapter
\hyperref[\detokenize{tech_note/Ecosystem/CLM50_Tech_Note_Ecosystem:rst-surface-characterization-vertical-discretization-and-model-input-requirements}]{\ref{\detokenize{tech_note/Ecosystem/CLM50_Tech_Note_Ecosystem:rst-surface-characterization-vertical-discretization-and-model-input-requirements}}} for an overview).


\subsection{Overview}
\label{\detokenize{tech_note/Glacier/CLM50_Tech_Note_Glacier:overview}}\label{\detokenize{tech_note/Glacier/CLM50_Tech_Note_Glacier:overview-glaciers}}
CLM is responsible for computing three quantities that are passed to the
ice sheet model:
\begin{enumerate}
\item {} 
Surface mass balance (SMB) \textendash{} the net annual accumulation/ablation of
mass at the upper surface (section
\hyperref[\detokenize{tech_note/Glacier/CLM50_Tech_Note_Glacier:computation-of-the-surface-mass-balance}]{\ref{\detokenize{tech_note/Glacier/CLM50_Tech_Note_Glacier:computation-of-the-surface-mass-balance}}})

\item {} 
Ground surface temperature, which serves as an upper boundary
condition for Glimmer-CISM’s temperature calculation

\item {} 
Surface topography, which currently is fixed in time, and is provided
on CLM’s surface dataset

\end{enumerate}

The ice sheet model is typically run at much higher resolution than CLM
(e.g., \(\sim\)5 km rather than \(\sim\)100 km). To improve
the downscaling from CLM’s grid to the ice sheet grid, the glaciated
portion of each grid cell is divided into multiple elevation classes
(section 10.2). The above quantities are computed separately in each
elevation class. Glimmer-CISM then computes high-resolution quantities
via horizontal and vertical interpolation.

There are several reasons for computing the SMB in CLM rather than in
Glimmer-CISM:
\begin{enumerate}
\item {} 
It is much cheaper to compute the SMB in CLM for \(\sim\)10
elevation classes than in Glimmer-CISM. For example, suppose we are
running CLM at a resolution of \(\sim\)50 km and Glimmer at
\(\sim\)5 km. Greenland has dimensions of about 1000 x 2000 km.
For CLM we would have 20 x 40 x 10 = 8,000 columns, whereas for
Glimmer we would have 200 x 400 = 80,000 columns.

\item {} 
We can use the sophisticated snow physics parameterization already in
CLM instead of implementing a separate scheme for Glimmer-CISM. Any
improvements to the CLM are applied to ice sheets automatically.

\item {} 
The atmosphere model can respond during runtime to ice-sheet surface
changes. As shown by {\hyperref[\detokenize{tech_note/References/CLM50_Tech_Note_References:pritchardetal2008}]{\sphinxcrossref{\DUrole{std,std-ref}{Pritchard et al. (2008)}}}},
runtime albedo feedback
from the ice sheet is critical for simulating ice-sheet retreat on
paleoclimate time scales. Without this feedback the atmosphere warms
much less, and the retreat is delayed.

\item {} 
Mass is more nearly conserved, given that the rate of surface ice
growth or melting computed in CLM is equal to the rate seen by the
dynamic ice sheet model. (Mass conservation is not exact, however,
because of approximations made in interpolating from the CLM grid to
the ice-sheet grid.)

\item {} 
The improved SMB is available in CLM for all glaciated grid cells
(e.g., in the Alps, Rockies, Andes, and Himalayas), not just those
which are part of ice sheets.

\end{enumerate}

The current coupling between CLM and Glimmer-CISM is one-way only. That
is, CLM sends the SMB and surface temperature to Glimmer-CISM but does
not do anything with the fields that are returned. The CLM glacier
fraction and surface topography are therefore fixed in time. One-way
coupling is reasonable for runs of \(\sim\)100 years or less, in
which ice-sheet elevation changes are modest. For longer runs with
larger elevation changes, two-way coupling is highly desirable. A
two-way coupling scheme is under development.


\subsection{Multiple elevation class scheme}
\label{\detokenize{tech_note/Glacier/CLM50_Tech_Note_Glacier:multiple-elevation-class-scheme}}\label{\detokenize{tech_note/Glacier/CLM50_Tech_Note_Glacier:id1}}
In the typical operation of CLM, the glacier land unit contains a single
column (section \hyperref[\detokenize{tech_note/Ecosystem/CLM50_Tech_Note_Ecosystem:surface-heterogeneity-and-data-structure}]{\ref{\detokenize{tech_note/Ecosystem/CLM50_Tech_Note_Ecosystem:surface-heterogeneity-and-data-structure}}}).
However, when running CESM with an active ice
sheet model, the glacier land unit is replaced by a glacier\_mec land
unit, where “mec” denotes “multiple elevation classes”. In most ways,
glacier\_mec land units behave the same as standard glacier land units.
However, each glacier\_mec land unit is divided into a user-defined set
of columns based on surface elevation. The default is 10 elevation
classes whose lower limits are 0, 200, 400, 700, 1000, 1300, 1600, 2000,
2500, and 3000 m. Each column is characterized by a fractional area and
surface elevation that are read in during model initialization. Each
glacier\_mec column within a grid cell has distinct ice and snow
temperatures, snow water content, surface fluxes, and SMB.

Glacier\_mec columns, like glacier columns, are initialized with a
temperature of 250 K. While glacier columns are initialized with a snow
liquid water equivalent (LWE) equal to the maximum allowed value of 1 m,
glacier\_mec columns begin with a snow LWE of 0.5 m so that they will
reach their equilibrium mean snow depth sooner. Glacier\_mec columns
typically require several decades of spin-up to equilibrate with a given
climate.

The atmospheric surface temperature, potential temperature, specific
humidity, density, and pressure are downscaled from the mean grid cell
elevation to the glacier\_mec column elevation using a specified lapse
rate (typically 6.0 deg/km) and an assumption of uniform relative
humidity. At a given time, lower-elevation columns can undergo surface
melting while columns at higher elevations remain frozen. This gives a
more accurate simulation of summer melting, which is a highly nonlinear
function of air temperature. The precipitation rate and radiative fluxes
are not currently downscaled, but could be in the future if care were
taken to preserve the cell-integrated values.

In contrast to most CLM subgrid units, glacier\_mec columns can be
active (i.e., have model calculations run there) even if their area is
zero. This is done because the ice sheet model may require a SMB even
for some grid cells where CLM does not have glacier land units. To allow
for this, grid overlap files have been pre-computed. For given
resolutions of CLM and Glimmer-CISM, these files identify all
land-covered grid cells that overlap any part of the ice sheet grid. In
these overlapping cells, glacier\_mec columns are defined in all
elevation classes. Some columns may have zero area and are called
“virtual” columns. These columns do not affect energy exchange between
the land and the atmosphere, but are included for potential forcing of
Glimmer-CISM.


\subsection{Computation of the surface mass balance}
\label{\detokenize{tech_note/Glacier/CLM50_Tech_Note_Glacier:id2}}\label{\detokenize{tech_note/Glacier/CLM50_Tech_Note_Glacier:computation-of-the-surface-mass-balance}}
The SMB of a glacier or ice sheet is the net annual
accumulation/ablation of mass at the upper surface. Ablation is defined
as the mass of water that runs off to the ocean. Not all the surface
meltwater runs off; some of the melt percolates into the snow and
refreezes. Accumulation is primarily by snowfall and deposition, and
ablation is primarily by melting and evaporation/sublimation. CLM uses a
surface-energy-balance (SEB) scheme to compute the SMB. In this scheme,
the melting depends on the sum of the radiative, turbulent, and
conductive fluxes reaching the surface, as described elsewhere in this
document.

CLM has a somewhat unrealistic treatment of accumulation and melting for
standard glacier land units. The snow depth is limited to a prescribed
depth of 1 m liquid water equivalent, with any additional snow assumed
to run off to the ocean. (This amounts to a crude parameterization of
iceberg calving.) Snow melting is treated in a realistic fashion, with
meltwater percolating downward through snow layers as long as the snow
is unsaturated. Once the underlying snow is saturated, any additional
meltwater runs off. When glacier ice melts, however, the meltwater is
assumed to remain in place until it refreezes. In warm parts of the ice
sheet, the meltwater does not refreeze, but stays in place indefinitely.

In the modified glacier\_mec columns, the treatment of melting and
freezing depends on the logical variable \sphinxstyleemphasis{glc\_dyntopo}. This variable
controls whether CLM surface topography changes dynamically as the ice
sheet evolves (i.e., whether the coupling is one-way or two-way). If
\sphinxstyleemphasis{glc\_dyntopo} is true, then CLM receives updated topographic
information from the ice sheet model. In this case, snow in excess of
the prescribed maximum depth is assumed to turn into ice, contributing a
positive SMB to the ice sheet model. Melting ice is assumed to run off
to the ocean, giving a negative SMB. The net SMB associated with ice
formation (by conversion from snow) and melting/runoff is computed for
each column, averaged over the coupling interval, and sent to the
coupler (\sphinxstyleemphasis{qice, mm/s}). If \sphinxstyleemphasis{glc\_dyntopo} is false, then surface runoff
for glacier\_mec land units is computed as for glacier land units: Any
snow in excess of 1 m LWE runs off to the ocean, and Melted ice remains
in place until it refreezes. Excess snow and melted ice still contribute
to positive and negative values, respectively, of \sphinxstyleemphasis{qice}, but only for
the purpose of forcing Glimmer-CISM. Currently, \sphinxstyleemphasis{glc\_dyntopo} = false
is the default, and the only supported option.

Note that the SMB typically is defined as the total accumulation of ice
and snow, minus the total ablation. The \sphinxstyleemphasis{qice} flux passed to
Glimmer-CISM is the mass balance for ice alone, not snow. We can think
of CLM as owning the snow, whereas Glimmer-CISM owns the underlying ice.
Fluctuations in snow depth between 0 and 1 m LWE are not reflected in
the SMB passed to Glimmer-CISM.


\section{Model for Scale Adaptive River Transport (MOSART)}
\label{\detokenize{tech_note/MOSART/CLM50_Tech_Note_MOSART:model-for-scale-adaptive-river-transport-mosart}}\label{\detokenize{tech_note/MOSART/CLM50_Tech_Note_MOSART::doc}}\label{\detokenize{tech_note/MOSART/CLM50_Tech_Note_MOSART:rst-river-transport-model-rtm}}

\subsection{Overview}
\label{\detokenize{tech_note/MOSART/CLM50_Tech_Note_MOSART:overview}}\label{\detokenize{tech_note/MOSART/CLM50_Tech_Note_MOSART:overview-mosart}}
MOSART is a river transport model designed for applications across local,
regional and global scales {\hyperref[\detokenize{tech_note/References/CLM50_Tech_Note_References:lietal2013b}]{\sphinxcrossref{\DUrole{std,std-ref}{(Li et al., 2013b)}}}}. A
major purpose of MOSART is to provide freshwater input for the ocean
model in coupled Earth system model. MOSART also provides an effective
way of evaluating and diagnosing the soil hydrology simulated by land
surface models through direction comparison of the simulated river flow
with observations of natural streamflow at gauging stations
{\hyperref[\detokenize{tech_note/References/CLM50_Tech_Note_References:lietal2015a}]{\sphinxcrossref{\DUrole{std,std-ref}{(Li et al., 2015a)}}}}.  Moreover, MOSART provides a
modeling framework for representing riverine transport and transformation
of energy and biogeochemical fluxes under both natural and human-influenced
conditions ( {\hyperref[\detokenize{tech_note/References/CLM50_Tech_Note_References:lietal2015b}]{\sphinxcrossref{\DUrole{std,std-ref}{(Li et al., 2015b)}}}}.


\subsection{Routing Processes}
\label{\detokenize{tech_note/MOSART/CLM50_Tech_Note_MOSART:routing-processes}}\label{\detokenize{tech_note/MOSART/CLM50_Tech_Note_MOSART:id1}}
MOSART divides each spatial unit such as a lat/lon grid or watershed into
three categories of hydrologic units (as shown in
\hyperref[\detokenize{tech_note/MOSART/CLM50_Tech_Note_MOSART:figure-mosart-conceptual-diagram}]{Figure \ref{\detokenize{tech_note/MOSART/CLM50_Tech_Note_MOSART:figure-mosart-conceptual-diagram}}}): hillslopes
that contribute both surface and subsurface runoff into tributaries,
tributaries that discharge into a single main channel, and the main channel
connects the local spatial unit with upstream/downstream units through the
river network. MOSART assumes that all the tributaries within a spatial unit
can be treated as a single hypothetical sub-network channel with a transport
capacity equivalent to all the tributaries combined. Correspondingly, three
routing processes are represented in MOSART: 1) hillslope routing: in each
spatial unit, surface runoff is routed as overland flow into the sub-network
channel, while subsurface runoff generated in the spatial unit directly enters
the sub-network channel; 2) sub-network channel routing: the sub-network channel
receives water from the hillslopes, routes water through the channel and discharges
it into the main channel; 3) main channel routing: the main channel receives water
from the sub-network channel and/or inflow, if any, from the upstream spatial units,
and discharges the water to its downstream spatial unit or the ocean.
\phantomsection\label{\detokenize{tech_note/MOSART/CLM50_Tech_Note_MOSART:figure-mosart-conceptual-diagram}}
\begin{figure}[htbp]
\centering

\noindent\sphinxincludegraphics[width=800\sphinxpxdimen,height=400\sphinxpxdimen]{{mosart_diagram}.png}
\label{\detokenize{tech_note/MOSART/CLM50_Tech_Note_MOSART:figure-mosart-conceptual-diagram}}\end{figure}

MOSART only route positive runoff, although negative runoff could be generated
occasionally by the land model (e.g., \(q_{gwl}\)). Negative runoff in any
runoff component including  \(q_{sur}\),  \(q_{sub}\),  \(q_{gwl}\)
is not routed through MOSART, but instead is mapped directly from the spatial unit
where it is generated at any time step to the coupler.

In MOSART, the travel velocities of water across hillslopes, sub-network and main
channel are all estimated using the Manning’s equation with different levels of
simplifications. Generally the Manning’s equation is in the form of
\phantomsection\label{\detokenize{tech_note/MOSART/CLM50_Tech_Note_MOSART:equation-14.1}}\begin{equation}\label{equation:tech_note/MOSART/CLM50_Tech_Note_MOSART:14.1}
\begin{split}V = \frac{R^{\frac{2}{3}} S_{f}}{n}\end{split}
\end{equation}
where  :math: \sphinxtitleref{V} is the travel velocity (m s $^{\text{-1}}$ ),  \(R\) is the hydraulic
radius (m). \(S_{f}\)  is the friction slope, and  accounting for the effects
of gravity, friction, inertia and other forces onthe water. If the channel slope
is steep enough, the gravity force dominates over the others so one can approximate
\(S_{f}\) by the channel bed slope \(S\) , which is the key assumption
underpinning the kinematic wave method. \(n\)  is the Manning’s roughness
coefficient, which is mainly controlled by surface roughness and sinuosity of the
flow path.

If the water surface is sufficiently large or the water depth \(h\) is
sufficiently shallow, the hydraulic radius can be approximated by the water depth.
This is the case for both hillslope and sub-network channel routing.
\phantomsection\label{\detokenize{tech_note/MOSART/CLM50_Tech_Note_MOSART:equation-14.2}}\begin{equation}\label{equation:tech_note/MOSART/CLM50_Tech_Note_MOSART:14.2}
\begin{split}R_{h} = h_{h}
R_{t} = h_{t}\end{split}
\end{equation}
Here \(R_{h}\) (m) and \(R_{t}\) (m) are hydraulic radius for hillslope and
sub-network channel routing respectively, and \(h_{h}\) (m) and \(h_{t}\)
(m) are water depth during hillslope and sub-network channel routing respectively.

For the main channel, the hydraulic radius is given by
\phantomsection\label{\detokenize{tech_note/MOSART/CLM50_Tech_Note_MOSART:equation-14.3}}\begin{equation}\label{equation:tech_note/MOSART/CLM50_Tech_Note_MOSART:14.3}
\begin{split}R_{r} = \frac{A_{r}}{P_{r}}\end{split}
\end{equation}
where \(A_{r}\) (m $^{\text{2}}$ ) is the wetted area defined as the part of the
channel cross-section area below the water surface,  \(P_{r}\) (m) is the
wetted perimeter (m), the perimeter confines in the wetted area.

For hillslopes, sub-network and main channels, a common continuity equation can
be written as
\phantomsection\label{\detokenize{tech_note/MOSART/CLM50_Tech_Note_MOSART:equation-14.4}}\begin{equation}\label{equation:tech_note/MOSART/CLM50_Tech_Note_MOSART:14.4}
\begin{split}\frac{dS}{dt} = Q_{in} - Q_{out} + R\end{split}
\end{equation}
where \(Q_{in}\) (m $^{\text{3}}$ s $^{\text{-1}}$ ) is the main channel flow from
the upstream grid(s) into the main channel of the current grid, which is zero for
hillslope and sub-network routing. \(Q_{out}\) (m $^{\text{3}}$ s $^{\text{-1}}$ ) is
the outflow rate from hillslope into the sub-network, from the sub-network into
the main channel, or from the current main channel to the main channel of its
downstream grid (if not the outlet grid) or ocean (if the current grid is the
basin outlet).  \(R\) (m $^{\text{3}}$ s $^{\text{-1}}$ ) is a source term, which
could be the surface
runoff generation rate for hillslopes, or lateral inflow (from hillslopes) into
sub-network channel or water-atmosphere exchange fluxes such as precipitation
and evaporation. It is assumed that surface runoff is generated uniformly
across all the hillslopes. Currently, MOSART does not exchange water with
the atmosphere or return water to the land model so its function is strictly
to transport water from runoff generation through the hillslope, tributaries,
and main channels to the basin outlets.


\subsection{Numerical Solution}
\label{\detokenize{tech_note/MOSART/CLM50_Tech_Note_MOSART:numerical-solution}}\label{\detokenize{tech_note/MOSART/CLM50_Tech_Note_MOSART:numerical-solution-mosart}}
The numerical implementation of MOSART is mainly based on a subcycling
scheme and a local time-stepping algorithm. There are two levels of
subcycling. For convenience, we denote \(T_{inputs}\) (s),
\(T_{mosart}\) (s), \(T_{hillslope}\) (s) and
\(T_{channel}\) (s) as the time steps of runoff inputs (from CLM
to MOSART via the flux coupler), MOSART routing, hillslope routing and
channel routing respectively. The first level of subcycling is between
the runoff inputs and MOSART routing. If \(T_{inputs}\) is 10800s
and \(T_{mosart}\) is 3600s, three MOSART time steps will be
invoked each time the runoff inputs are updated. The second level of
subcycling is between the hillslope routing and channel routing. This
is to account for the fact that the travel velocity of water across
hillslope is usually much slower than that in the channels.
\(T_{hillslope}\) is usually set as the same as \(T_{mosart}\),
but within each time step of hillslope routing there are a few time
steps for channel routing, i.e.,
\(T_{hillslope} = D_{levelH2R} \cdot T_{channel}\). The local
time-stepping algorithm is to account for the fact that the travel
velocity of water is much faster in some river channels (e.g., with
steeper bed slope, narrower channel width) than others. That is, for
each channel (either a sub-network or main channel), the final time
step of local channel routing is given as
\(T_{local}=T_{channel}/D_{local}\).  \(D_{local}\) is
currently estimated empirically as a function of local channel slope,
width, length and upstream drainage area. If MOSART crashes due to a
numerical issue, we recommend to increase \(D_{levelH2R}\) and, if
the issue remains, reducing \(T_{mosart}\).


\subsection{Parameters and Input Data}
\label{\detokenize{tech_note/MOSART/CLM50_Tech_Note_MOSART:id2}}\label{\detokenize{tech_note/MOSART/CLM50_Tech_Note_MOSART:parameters-and-input-data}}
MOSART is supported by a comprehensive, global hydrography dataset at 0.5
$^{\text{o}}$ resolution. As such, the fundamental spatial unit of MOSART is a 0.5
$^{\text{o}}$ lat/lon grid.  The topographic parameters (such as flow direction,
channel length, topographic and channel slopes etc.) were derived using the
Dominant River Tracing (DRT) algorithm ({\hyperref[\detokenize{tech_note/References/CLM50_Tech_Note_References:wuetal2011}]{\sphinxcrossref{\DUrole{std,std-ref}{Wu et al., 2011}}}} ;
{\hyperref[\detokenize{tech_note/References/CLM50_Tech_Note_References:wuetal2012}]{\sphinxcrossref{\DUrole{std,std-ref}{Wu et al. 2012}}}}). The DRT algorithm produces the topographic
parameters in a scale-consistent way to preserve/upscale the key features of
a baseline high-resolution hydrography dataset at multiple coarser spatial
resolutions. Here the baseline high-resolution hydrography dataset is the
1km resolution Hydrological data and maps based on SHuttle Elevation
Derivatives at multiple Scales (HydroSHEDS)
({\hyperref[\detokenize{tech_note/References/CLM50_Tech_Note_References:lehnerdoll2004}]{\sphinxcrossref{\DUrole{std,std-ref}{Lehner and Döll, 2004}}}} ;
{\hyperref[\detokenize{tech_note/References/CLM50_Tech_Note_References:lehneretal2008}]{\sphinxcrossref{\DUrole{std,std-ref}{Lehner et al., 2008}}}}).  The channel geometry
parameters, e.g., bankfull width and depth, were estimated from empirical
hydraulic geometry relationships as functions of the mean annual discharge.
The Manning roughness coefficients for overland and channel flow were
calculated as functions of landcover and water depth. For more details
on the methodology to derive channel geometry and the Manning’s roughness
coefficients, please refer to
{\hyperref[\detokenize{tech_note/References/CLM50_Tech_Note_References:getiranaetal2012}]{\sphinxcrossref{\DUrole{std,std-ref}{Getirana et al. (2012)}}}} . The full list of
parameters included in this global hydrography dataset is provided in
the \hyperref[\detokenize{tech_note/MOSART/CLM50_Tech_Note_MOSART:table-mosart-parameters}]{Table \ref{\detokenize{tech_note/MOSART/CLM50_Tech_Note_MOSART:table-mosart-parameters}}}.  Evaluation of global simulations
by MOSART using the aforementioned parameters is described in
{\hyperref[\detokenize{tech_note/References/CLM50_Tech_Note_References:lietal2015b}]{\sphinxcrossref{\DUrole{std,std-ref}{Li et al. (2015b)}}}} .


\begin{savenotes}\sphinxattablestart
\centering
\sphinxcapstartof{table}
\sphinxcaption{List of parameters in the global hydrography dataset}\label{\detokenize{tech_note/MOSART/CLM50_Tech_Note_MOSART:table-mosart-parameters}}\label{\detokenize{tech_note/MOSART/CLM50_Tech_Note_MOSART:id3}}
\sphinxaftercaption
\begin{tabulary}{\linewidth}[t]{|T|T|T|}
\hline
\sphinxstylethead{\sphinxstyletheadfamily 
Name
\unskip}\relax &\sphinxstylethead{\sphinxstyletheadfamily 
Unit
\unskip}\relax &\sphinxstylethead{\sphinxstyletheadfamily 
Description
\unskip}\relax \\
\hline
\(F_{dir}\)
&
-
&
The D8 single flow direction for each coarse grid cell coded using 1 (E), 2 (SE), 4 (S), 8 (SW), 16 (W), 32 (NW), 64 (N), 128 (NE)
\\
\hline
\(A_{total}\)
&
km $^{\text{2}}$
&
The upstream drainage area of each coarse grid cell
\\
\hline
\(F_{dis}\)
&
m
&
The dominant river length for each coarse grid cell
\\
\hline
\(S_{channel}\)
&
-
&
The average channel slope for each coarse grid cell
\\
\hline
\(S_{topographic}\)
&
-
&
The average topographic slope (for overland flow routing) for each coarse grid cell
\\
\hline
\(A_{local}\)
&
km $^{\text{2}}$
&
The surface area for each coarse grid cell
\\
\hline
\(D_{p}\)
&
m $^{\text{-1}}$
&
Drainage density, calculated  as the total channel length within each coarse grid cell divided by the local cell area
\\
\hline
\(D_{r}\)
&
m
&
The bankfull depth of main channel
\\
\hline
\(W_{r}\)
&
m
&
The bankfull width of main channel
\\
\hline
\(D_{t}\)
&
m
&
The average bankfull depth of tributary channels
\\
\hline
\(W_{t}\)
&
m
&
The average bankfull width of tributary channels
\\
\hline
\(n_{r}\)
&
-
&
Manning’s roughness coefficient for channel flow routing
\\
\hline
\(n_{h}\)
&
-
&
Manning’s roughness coefficient for overland flow routing
\\
\hline
\end{tabulary}
\par
\sphinxattableend\end{savenotes}


\subsection{Difference between CLM5.0 and CLM4.5}
\label{\detokenize{tech_note/MOSART/CLM50_Tech_Note_MOSART:difference-between-clm5-0-and-clm4-5}}
1. Routing methods: RTM, a linear reservoir method, is used in CLM4.5 for
river routing, whilst in CLM5.0, MOSART is an added option for river routing
based on the more physically-based kinematic wave method.
2. Runoff treatment: In RTM runoff is routed regardless of its sign so
negative streamflow can be simulated at times. MOSART routes only nonnegative
runoff and always produces positive streamflow, which is important for
future extension for modeling riverine heat and biogeochemical fluxes.
3. Input parameters: RTM in CLM4.5 only requires one layer of spatial variable
of channel velocity, whilst MOSART in CLM5.0 requires 13 parameters that
are all available globally at 0.5 $^{\text{o}}$ resolution.
4. Outputs: RTM only produces streamflow simulation, whilst MOSART
additionally simulates the time-varying channel velocities and channel
water depth and channel surface water variation.


\section{Urban Model (CLMU)}
\label{\detokenize{tech_note/Urban/CLM50_Tech_Note_Urban:rst-urban-model-clmu}}\label{\detokenize{tech_note/Urban/CLM50_Tech_Note_Urban::doc}}\label{\detokenize{tech_note/Urban/CLM50_Tech_Note_Urban:urban-model-clmu}}
At the global scale, and at the coarse spatial resolution of current
climate models, urbanization has negligible impact on climate. However,
the urban parameterization (CLMU; {\hyperref[\detokenize{tech_note/References/CLM50_Tech_Note_References:olesonetal2008b}]{\sphinxcrossref{\DUrole{std,std-ref}{Oleson et al. (2008b)}}}};
{\hyperref[\detokenize{tech_note/References/CLM50_Tech_Note_References:olesonetal2008c}]{\sphinxcrossref{\DUrole{std,std-ref}{Oleson et al. (2008c)}}}}) allows
simulation of the urban environment within a climate model, and
particularly the temperature where people live. As such, the urban model
allows scientific study of how climate change affects the urban heat
island and possible urban planning and design strategies to mitigate
warming (e.g., white roofs).

Urban areas in CLM are represented by up to three urban landunits per
gridcell according to density class. The urban landunit is based on the
“urban canyon” concept of {\hyperref[\detokenize{tech_note/References/CLM50_Tech_Note_References:oke1987}]{\sphinxcrossref{\DUrole{std,std-ref}{Oke (1987)}}}} in which
the canyon geometry is
described by building height (\(H\)) and street width (\(W\))
(\hyperref[\detokenize{tech_note/Urban/CLM50_Tech_Note_Urban:figure-schematic-representation-of-the-urban-landunit}]{Figure \ref{\detokenize{tech_note/Urban/CLM50_Tech_Note_Urban:figure-schematic-representation-of-the-urban-landunit}}}). The canyon system
consists of roofs, walls, and canyon
floor. Walls are further divided into shaded and sunlit components. The
canyon floor is divided into pervious (e.g., to represent residential
lawns, parks) and impervious (e.g., to represent roads, parking lots,
sidewalks) fractions. Vegetation is not explicitly modeled for the
pervious fraction; instead evaporation is parameterized by a simplified
bulk scheme.

Each of the five urban surfaces is treated as a column within the
landunit (\hyperref[\detokenize{tech_note/Urban/CLM50_Tech_Note_Urban:figure-schematic-representation-of-the-urban-landunit}]{Figure \ref{\detokenize{tech_note/Urban/CLM50_Tech_Note_Urban:figure-schematic-representation-of-the-urban-landunit}}}).
Radiation parameterizations account for trapping
of solar and longwave radiation inside the canyon. Momentum fluxes are
determined for the urban landunit using a roughness length and
displacement height appropriate for the urban canyon and stability
formulations from CLM. A one-dimensional heat conduction equation is
solved numerically for a multiple-layer (\(N_{levurb} =10\)) column
to determine conduction fluxes into and out of canyon surfaces.

A new building energy model has been developed for CLM5.0.  It accounts
for the conduction of heat through interior surfaces (roof, sunlit and
shaded walls, and floors), convection (sensible heat exchange) between
interior surfaces and building air, longwave radiation exchange between
interior surfaces, and ventilation (natural infiltration and exfiltration).
Idealized HAC systems are assumed where the system capacity is infinite and
the system supplies the amount of energy needed to keep the indoor air
temperature (\(T_{iB}\)) within maximum and minimum emperatures
(\(T_{iB,\, \max } ,\, T_{iB,\, \min }\) ), thus explicitly
resolving space heating and air conditioning fluxes. Anthropogenic sources
of waste heat (\(Q_{H,\, waste}\) ) from HAC that account for inefficiencies
in the heating and air conditioning equipment and from energy lost in the
conversion of primary energy sources to end use energy are derived from
{\hyperref[\detokenize{tech_note/References/CLM50_Tech_Note_References:sivak2013}]{\sphinxcrossref{\DUrole{std,std-ref}{Sivak (2013)}}}}.  These sources of waste heat are incorporated
as modifications to the canyon energy budget.

Turbulent {[}sensible heat (\(Q_{H,\, u}\) ) and
latent heat (\(Q_{E,\, u}\) ){]} and storage (\(Q_{S,\, u}\) )
heat fluxes and surface (\(T_{u,\, s}\) ) and internal
(\(T_{u,\, i=1,\, N_{levgrnd} }\) ) temperatures are determined for
each urban surface \(u\). Hydrology on the roof and canyon floor is
simulated and walls are hydrologically inactive. A snowpack can form on
the active surfaces. A certain amount of liquid water is allowed to pond
on these surfaces which supports evaporation. Water in excess of the
maximum ponding depth runs off
(\(R_{roof} ,\, R_{imprvrd} ,\, R_{prvrd}\) ).

The heat and moisture fluxes from each surface interact with each other
through a bulk air mass that represents air in the urban canopy layer
for which specific humidity (\(q_{ac}\) ) and temperature
(\(T_{ac}\) ) are prognosed (\hyperref[\detokenize{tech_note/Urban/CLM50_Tech_Note_Urban:figure-schematic-of-urban-and-atmospheric-model-coupling}]{Figure \ref{\detokenize{tech_note/Urban/CLM50_Tech_Note_Urban:figure-schematic-of-urban-and-atmospheric-model-coupling}}}).
The air temperature can
be compared with that from surrounding vegetated/soil (rural) surfaces
in the model to ascertain heat island characteristics. As with other
landunits, the CLMU is forced either with output from a host atmospheric
model (e.g., the Community Atmosphere Model (CAM)) or
observed forcing (e.g., reanalysis or field observations). The urban
model produces sensible, latent heat, and momentum fluxes, emitted
longwave, and reflected solar radiation, which are area-averaged with
fluxes from non-urban “landunits” (e.g., vegetation, lakes) to supply
grid cell averaged fluxes to the atmospheric model.

Present day global urban extent and urban properties were developed by
{\hyperref[\detokenize{tech_note/References/CLM50_Tech_Note_References:jacksonetal2010}]{\sphinxcrossref{\DUrole{std,std-ref}{Jackson et al. (2010)}}}}. Urban extent, defined for four classes {[}tall
building district (TBD), and high, medium, and low density (HD, MD,
LD){]}, was derived from LandScan 2004, a population density dataset
derived from census data, nighttime lights satellite observations, road
proximity, and slope ({\hyperref[\detokenize{tech_note/References/CLM50_Tech_Note_References:dobsonetal2000}]{\sphinxcrossref{\DUrole{std,std-ref}{Dobson et al. 2000}}}}). The urban extent data for
TBD, HD, and MD classes are aggregated from the original 1 km resolution
to both a 0.05$^{\text{o}}$ by 0.05$^{\text{o}}$ global grid
for high-resolution studies or a 0.5$^{\text{o}}$ by
0.5$^{\text{o}}$ grid. For the current implementation, the LD class
is not used because it is highly rural and better modeled as a
vegetated/soil surface. Although the TBD, HD, and MD classes are
represented as individual urban landunits, urban model history output is
currently a weighted average of the output for individual classes.

For each of 33 distinct regions across the globe, thermal (e.g., heat
capacity and thermal conductivity), radiative (e.g., albedo and
emissivity) and morphological (e.g., height to width ratio, roof
fraction, average building height, and pervious fraction of the canyon
floor) properties are provided for each of the density classes. Building
interior minimum and maximum temperatures are prescribed based on
climate and socioeconomic considerations. The surface dataset creation
routines (see CLM5.0 User’s Guide) aggregate the data to the desired
resolution.

\begin{figure}[htbp]
\centering
\capstart

\noindent\sphinxincludegraphics{{image19}.png}
\caption{Schematic representation of the urban land unit. See the text for description of notation. Incident, reflected, and net solar and longwave radiation are calculated for each individual surface but are not shown for clarity.}\label{\detokenize{tech_note/Urban/CLM50_Tech_Note_Urban:figure-schematic-representation-of-the-urban-landunit}}\label{\detokenize{tech_note/Urban/CLM50_Tech_Note_Urban:id1}}\end{figure}

\begin{figure}[htbp]
\centering
\capstart

\noindent\sphinxincludegraphics{{image23}.png}
\caption{Schematic of urban and atmospheric model coupling.  The urban model is forced by the atmospheric model wind (\(u_{atm}\) ), temperature (\(T_{atm}\) ), specific humidity (\(q_{atm}\) ), precipitation (\(P_{atm}\) ), solar (\(S_{atm} \, \downarrow\) ) and longwave (\(L_{atm} \, \downarrow\) ) radiation at reference height \(z'_{atm}\)  (section \hyperref[\detokenize{tech_note/Ecosystem/CLM50_Tech_Note_Ecosystem:atmospheric-coupling}]{\ref{\detokenize{tech_note/Ecosystem/CLM50_Tech_Note_Ecosystem:atmospheric-coupling}}}). Fluxes from the urban landunit to the atmosphere are turbulent sensible (\(H\)) and latent heat (\(\lambda E\)), momentum (\(\tau\) ), albedo (\(I\uparrow\) ), emitted longwave (\(L\uparrow\) ), and absorbed shortwave (\(\vec{S}\)) radiation. Air temperature (\(T_{ac}\) ), specific humidity (\(q_{ac}\) ), and wind speed (\(u_{c}\) ) within the urban canopy layer are diagnosed by the urban model. \(H\) is the average building height.}\label{\detokenize{tech_note/Urban/CLM50_Tech_Note_Urban:figure-schematic-of-urban-and-atmospheric-model-coupling}}\label{\detokenize{tech_note/Urban/CLM50_Tech_Note_Urban:id2}}\end{figure}

The urban model that was first released as a component of CLM4.0 is separately
described in the urban technical note ({\hyperref[\detokenize{tech_note/References/CLM50_Tech_Note_References:olesonetal2010b}]{\sphinxcrossref{\DUrole{std,std-ref}{Oleson et al. (2010b)}}}}).
The main changes in the urban model from CLM4.0 to CLM4.5 were 1)
an expansion of the single urban landunit to up to three landunits per
grid cell stratified by urban density types, 2) the number of urban
layers for roofs and walls was no longer constrained to be equal to the
number of ground layers, 3) space heating and air conditioning wasteheat
factors were set to zero by default so that the user could customize
these factors for their own application, 4) the elevation threshold used
to eliminate urban areas in the surface dataset creation routines was
increased from 2200 meters to 2600 meters, 5) hydrologic and thermal
calculations for the pervious road followed CLM4.5 parameterizations.

The main changes in the urban model from CLM4.5 to CLM5.0 are 1) a more
sophisticated and realistic building space heating and air conditioning
submodel that prognoses interior building air temperature and includes more
realistic space heating and air conditioning wasteheat factors (see above), 2) the maximum
building temperature (which determines air conditioning demand) is now read in
from a namelist-defined file which allows for dynamic control of this input
variable.  The maximum building temperatures that are defined in
{\hyperref[\detokenize{tech_note/References/CLM50_Tech_Note_References:jacksonetal2010}]{\sphinxcrossref{\DUrole{std,std-ref}{Jackson et al. (2010)}}}} are implemented in year 1950 (thus
air conditioning is off in prior years) and air conditioning is turned off in year
2100 (because the buildings are not suitable for air conditioning in some extreme
global warming scenarios).  These feature will be described in more detail in
a forthcoming paper. In addition, a module of heat stress indices calculated online
in the model that can be used to assess human thermal comfort for rural and urban
areas has been added.  This last development is described and evaluated by
{\hyperref[\detokenize{tech_note/References/CLM50_Tech_Note_References:buzanetal2015}]{\sphinxcrossref{\DUrole{std,std-ref}{Buzan et al. (2015)}}}}.


\section{CN Pools}
\label{\detokenize{tech_note/CN_Pools/CLM50_Tech_Note_CN_Pools::doc}}\label{\detokenize{tech_note/CN_Pools/CLM50_Tech_Note_CN_Pools:rst-cn-pools}}\label{\detokenize{tech_note/CN_Pools/CLM50_Tech_Note_CN_Pools:cn-pools}}

\subsection{Introduction}
\label{\detokenize{tech_note/CN_Pools/CLM50_Tech_Note_CN_Pools:introduction}}
CLM includes a prognostic treatment of the terrestrial carbon and
nitrogen cycles including natural vegetation, crops, and soil biogeochemistry. The model is
fully prognostic with respect to all carbon and nitrogen state variables
in the vegetation, litter, and soil organic matter. The seasonal timing
of new vegetation growth and litterfall is also prognostic, responding
to soil and air temperature, soil water availability, daylength, and
crop management practices in
varying degrees depending on a specified phenology type or management for each PFT
(Chapter
\hyperref[\detokenize{tech_note/Vegetation_Phenology_Turnover/CLM50_Tech_Note_Vegetation_Phenology_Turnover:rst-vegetation-phenology-and-turnover}]{\ref{\detokenize{tech_note/Vegetation_Phenology_Turnover/CLM50_Tech_Note_Vegetation_Phenology_Turnover:rst-vegetation-phenology-and-turnover}}}). The
prognostic LAI, SAI,
tissue stoichiometry, and vegetation heights are
utilized by the biophysical model that couples carbon, water, and
energy cycles.

Separate state variables for C and N are tracked for leaf, live stem,
dead stem, live coarse root, dead coarse root, fine root, and grain pools
(\hyperref[\detokenize{tech_note/CN_Pools/CLM50_Tech_Note_CN_Pools:figure-vegetation-fluxes-and-pools}]{Figure \ref{\detokenize{tech_note/CN_Pools/CLM50_Tech_Note_CN_Pools:figure-vegetation-fluxes-and-pools}}}). Each of these pools has two corresponding
storage pools representing, respectively, short-term and long-term
storage of non-structural carbohydrates and labile nitrogen. There are
two additional carbon pools, one for the storage of growth respiration
reserves, and another used to meet excess demand for maintenance
respiration during periods with low photosynthesis. One additional
nitrogen pool tracks retranslocated nitrogen, mobilized from leaf tissue
prior to abscission and litterfall. Altogether there are 23 state
variables for vegetation carbon, and 22 for vegetation nitrogen.

\begin{figure}[htbp]
\centering
\capstart

\noindent\sphinxincludegraphics[width=753\sphinxpxdimen,height=513\sphinxpxdimen]{{CLMCN_pool_structure_v2_lores}.png}
\caption{Vegetation fluxes and pools for carbon cycle in CLM5.}\label{\detokenize{tech_note/CN_Pools/CLM50_Tech_Note_CN_Pools:figure-vegetation-fluxes-and-pools}}\label{\detokenize{tech_note/CN_Pools/CLM50_Tech_Note_CN_Pools:id1}}\end{figure}

In addition to the vegetation pools, CLM includes a series of
decomposing carbon and nitrogen pools as vegetation successively
breaks down to CWD, and/or litter, and subsequently to soil organic
matter. Discussion of the decomposition model, alternate
specifications of decomposition rates, and methods to rapidly
equilibrate the decomposition model, is in Chapter
\hyperref[\detokenize{tech_note/Decomposition/CLM50_Tech_Note_Decomposition:rst-decomposition}]{\ref{\detokenize{tech_note/Decomposition/CLM50_Tech_Note_Decomposition:rst-decomposition}}}.


\subsection{Tissue Stoichiometry}
\label{\detokenize{tech_note/CN_Pools/CLM50_Tech_Note_CN_Pools:tissue-stoichiometry}}
As of CLM5, vegetation tissues have a flexible stoichiometry, as
described in {\hyperref[\detokenize{tech_note/References/CLM50_Tech_Note_References:ghimireetal2016}]{\sphinxcrossref{\DUrole{std,std-ref}{Ghimire et al. (2016)}}}}. Each
tissue has a target C:N ratio, and nitrogen is allocated at each
timestep in order to allow the plant to best match the target
stoichiometry.  Nitrogen downregulation of productivity acts by
increasing the C:N ratio of leaves when insufficient nitrogen is
available to meet stoichiometric demands of leaf growth, thereby
reducing the N available for photosynthesis and reducing the \(V_{\text{c,max25}}\) and
\(J_{\text{max25}}\) terms, as described in Chapter
\hyperref[\detokenize{tech_note/Photosynthetic_Capacity/CLM50_Tech_Note_Photosynthetic_Capacity:rst-photosynthetic-capacity}]{\ref{\detokenize{tech_note/Photosynthetic_Capacity/CLM50_Tech_Note_Photosynthetic_Capacity:rst-photosynthetic-capacity}}}.  Details of the flexible tissue
stoichiometry are described in Chapter \hyperref[\detokenize{tech_note/CN_Allocation/CLM50_Tech_Note_CN_Allocation:rst-cn-allocation}]{\ref{\detokenize{tech_note/CN_Allocation/CLM50_Tech_Note_CN_Allocation:rst-cn-allocation}}}.


\section{Plant Respiration}
\label{\detokenize{tech_note/Plant_Respiration/CLM50_Tech_Note_Plant_Respiration:plant-respiration}}\label{\detokenize{tech_note/Plant_Respiration/CLM50_Tech_Note_Plant_Respiration::doc}}\label{\detokenize{tech_note/Plant_Respiration/CLM50_Tech_Note_Plant_Respiration:rst-plant-respiration}}

\subsection{Autotrophic Respiration}
\label{\detokenize{tech_note/Plant_Respiration/CLM50_Tech_Note_Plant_Respiration:autotrophic-respiration}}
The model treats maintenance and growth respiration fluxes separately,
even though it is difficult to measure them as separate fluxes (Lavigne
and Ryan, 1997; Sprugel et al., 1995). Maintenance respiration is
defined as the carbon cost to support the metabolic activity of existing
live tissue, while growth respiration is defined as the additional
carbon cost for the synthesis of new growth.


\subsubsection{Maintenance Respiration}
\label{\detokenize{tech_note/Plant_Respiration/CLM50_Tech_Note_Plant_Respiration:maintenance-respiration}}
Atkin et al. (2016) propose a model for respiration that is based on the leaf nitrogen content per unit area (\(NS_{narea}\) (gN m $^{\text{2}}$ leaf), with an intercept parameter that is PFT dependant, and an acclimation term that depends upon the average temperature of the previous 10 day period \(t_{2m,10days}\), in Celsius.
\phantomsection\label{\detokenize{tech_note/Plant_Respiration/CLM50_Tech_Note_Plant_Respiration:equation-17.46)}}\begin{equation}\label{equation:tech_note/Plant_Respiration/CLM50_Tech_Note_Plant_Respiration:17.46)}
\begin{split}CF_{mr\_ leaf} =  i_{atkin,pft} + (NS_{narea}  0.2061) - (0.0402 (t_{2m,10days}))\end{split}
\end{equation}
The temperature dependance of leaf maintenance (dark) respiration is described in Chapter \hyperref[\detokenize{tech_note/Photosynthesis/CLM50_Tech_Note_Photosynthesis:rst-stomatal-resistance-and-photosynthesis}]{\ref{\detokenize{tech_note/Photosynthesis/CLM50_Tech_Note_Photosynthesis:rst-stomatal-resistance-and-photosynthesis}}}.
\phantomsection\label{\detokenize{tech_note/Plant_Respiration/CLM50_Tech_Note_Plant_Respiration:equation-17.47)}}\begin{equation}\label{equation:tech_note/Plant_Respiration/CLM50_Tech_Note_Plant_Respiration:17.47)}
\begin{split}CF_{mr\_ livestem} \_ =NS_{livestem} MR_{base} MR_{Q10} ^{(T_{2m} -20)/10}\end{split}
\end{equation}\phantomsection\label{\detokenize{tech_note/Plant_Respiration/CLM50_Tech_Note_Plant_Respiration:equation-17.48)}}\begin{equation}\label{equation:tech_note/Plant_Respiration/CLM50_Tech_Note_Plant_Respiration:17.48)}
\begin{split}CF_{mr\_ livecroot} \_ =NS_{livecroot} MR_{base} MR_{Q10} ^{(T_{2m} -20)/10}\end{split}
\end{equation}\phantomsection\label{\detokenize{tech_note/Plant_Respiration/CLM50_Tech_Note_Plant_Respiration:equation-17.49)}}\begin{equation}\label{equation:tech_note/Plant_Respiration/CLM50_Tech_Note_Plant_Respiration:17.49)}
\begin{split}CF_{mr\_ froot} \_ =\sum _{j=1}^{nlevsoi}NS_{froot} rootfr_{j} MR_{base} MR_{Q10} ^{(Ts_{j} -20)/10}\end{split}
\end{equation}
where \(MR_{q10}\) (= 2.0) is
the temperature sensitivity for maintenance respiration,
\(T_{2m}\) ($^{\text{o}}$C) is the air temperature at 2m
height, \(Ts_{j}`* (:sup:`o\)C) is the soil
temperature at level \sphinxstyleemphasis{j}, and \(rootfr_{j}\) is the fraction
of fine roots distributed in soil level \sphinxstyleemphasis{j}.


\begin{savenotes}\sphinxattablestart
\centering
\sphinxcapstartof{table}
\sphinxcaption{Atkin leaf respiration model intercept values.}\label{\detokenize{tech_note/Plant_Respiration/CLM50_Tech_Note_Plant_Respiration:id3}}
\sphinxaftercaption
\begin{tabulary}{\linewidth}[t]{|T|T|}
\hline
\sphinxstylethead{\sphinxstyletheadfamily 
Plant functional type
\unskip}\relax &
\(i_{atkin}\)
\\
\hline
NET Temperate
&
1.499
\\
\hline
NET Boreal
&
1.499
\\
\hline
NDT Boreal
&
1.499
\\
\hline
BET Tropical
&
1.756
\\
\hline
BET temperate
&
1.756
\\
\hline
BDT tropical
&
1.756
\\
\hline
BDT temperate
&
1.756
\\
\hline
BDT boreal
&
1.756
\\
\hline
BES temperate
&
2.075
\\
\hline
BDS temperate
&
2.075
\\
\hline
BDS boreal
&
2.075
\\
\hline
C$_{\text{3}}$ arctic grass
&
2.196
\\
\hline
C$_{\text{3}}$ grass
&
2.196
\\
\hline
C$_{\text{4}}$ grass
&
2.196
\\
\hline
\end{tabulary}
\par
\sphinxattableend\end{savenotes}

Note that, for woody vegetation, maintenance respiration costs are not
calculated for the dead stem and dead coarse root components. These
components are assumed to consist of dead xylem cells, with no metabolic
function. By separating the small live component of the woody tissue
(ray parenchyma, phloem, and sheathing lateral meristem cells) from the
larger fraction of dead woody tissue, it is reasonable to assume a
common base maintenance respiration rate for all live tissue types.

The total maintenance respiration cost is then given as:
\phantomsection\label{\detokenize{tech_note/Plant_Respiration/CLM50_Tech_Note_Plant_Respiration:equation-17.50)}}\begin{equation}\label{equation:tech_note/Plant_Respiration/CLM50_Tech_Note_Plant_Respiration:17.50)}
\begin{split}CF_{mr} =CF_{mr\_ leaf} +CF_{mr\_ froot} +CF_{mr\_ livestem} +CF_{mr\_ livecroot} .\end{split}
\end{equation}

\subsubsection{Growth Respiration}
\label{\detokenize{tech_note/Plant_Respiration/CLM50_Tech_Note_Plant_Respiration:growth-respiration}}
Growth respiration is calculated as a factor of 0.11 times the total
carbon allocation to new growth (\(CF_{growth}\), after allocating carbon for N acquisition,
Chapter \hyperref[\detokenize{tech_note/FUN/CLM50_Tech_Note_FUN:rst-fun}]{\ref{\detokenize{tech_note/FUN/CLM50_Tech_Note_FUN:rst-fun}}}.)on a given timestep, based on construction costs
for a range of woody and non-woody tissues (Atkin et al. in prep). For new
carbon and nitrogen allocation that enters storage pools for subsequent
display, it is not clear what fraction of the associated growth
respiration should occur at the time of initial allocation, and what
fraction should occur later, at the time of display of new growth from
storage. Eddy covariance estimates of carbon fluxes in forest ecosystems
suggest that the growth respiration associated with transfer of
allocated carbon and nitrogen from storage into displayed tissue is not
significant (Churkina et al., 2003), and so it is assumed in CLM that
all of the growth respiration cost is incurred at the time of initial
allocation, regardless of the fraction of allocation that is displayed
immediately (i.e. regardless of the value of \(f_{cur}\),
section 13.5). This behavior is parameterized in such a way that if
future research suggests that some fraction of the growth respiration
cost should be incurred at the time of display from storage, a simple
parameter modification will effect the change. %
\begin{footnote}[1]\sphinxAtStartFootnote
Parameter \(\text{grpnow}\)  in routines CNGResp and  CNAllocation, currently set to 1.0, could be changed to a smaller
value to transfer some portion (1 - \(\text{grpnow}\) ) of the growth respiration forward in time to occur at the time of growth
display from storage.
%
\end{footnote}


\section{Fixation and Uptake of Nitrogen (FUN)}
\label{\detokenize{tech_note/FUN/CLM50_Tech_Note_FUN:rst-fun}}\label{\detokenize{tech_note/FUN/CLM50_Tech_Note_FUN:fixation-and-uptake-of-nitrogen-fun}}\label{\detokenize{tech_note/FUN/CLM50_Tech_Note_FUN::doc}}

\subsection{Introduction}
\label{\detokenize{tech_note/FUN/CLM50_Tech_Note_FUN:introduction}}
The Fixation and Uptake of Nitrogen model is based on work by {\hyperref[\detokenize{tech_note/References/CLM50_Tech_Note_References:fisheretal2010}]{\sphinxcrossref{\DUrole{std,std-ref}{Fisher et al. (2010)}}}}, {\hyperref[\detokenize{tech_note/References/CLM50_Tech_Note_References:brzosteketal2014}]{\sphinxcrossref{\DUrole{std,std-ref}{Brzostek et al. (2014)}}}}, and {\hyperref[\detokenize{tech_note/References/CLM50_Tech_Note_References:shietal2016}]{\sphinxcrossref{\DUrole{std,std-ref}{Shi et al. (2016)}}}}.  The concept of FUN is that in most cases, Nitrogen uptake requires the expenditure of energy in the form of carbon, and further, that there are numerous potential sources of Nitrogen in the environment which a plant may exchange for carbon. The ratio of carbon expended to Nitrogen acquired is referred to here as the cost, or exchange rate,  of N acquisition (\(E_{nacq}\), gC/gN)). There are eight pathways for N uptake:
\begin{enumerate}
\item {} 
Fixation by symbiotic bacteria in root nodules (for N fixing plants) (\(_{fix}\))

\item {} 
Retranslocation of N from senescing tissues (\(_{ret}\))

\item {} 
Active uptake of NH4 by arbuscular mycorrhizal plants (\(_{active,nh4}\))

\item {} 
Active uptake of NH4 by ectomycorrhizal plants (\(_{active,nh4}\))

\item {} 
Active uptake of NO3 by arbuscular mycorrhizal plants (\(_{active,no3}\))

\item {} 
Active uptake of NO3 by ectomycorrhizal plants (\(_{active,no3}\))

\item {} 
Nonmycorrhizal uptake of NH4 (\(_{nonmyc,no3}\))

\item {} 
Nonmycorrhizal uptake of NO3 (\(_{nonmyc,nh4}\))

\end{enumerate}

The notation suffix for each pathway is given in parentheses here. At each timestep, each of these pathways is associated with a cost term (\(N_{cost,x}\)), a payment in carbon (\(C_{nuptake,x}\)), and an influx of Nitrogen (\(N_{uptake,x}\)) where \(x\) is one of the eight uptake streams listed above.

For each PFT, we define a fraction of the total C acquisition that can be used for N fixation (\(f_{fixers}\)), which is broadly equivalent to the fraction of a given PFT that is capable of fixing Nitrogen, and thus represents an upper limit on the amount to which fixation can be increased in low n conditions.  For each PFT, the cost calculation is conducted twice. Once where fixation is possible and once where it is not. (\(f_{fixers}\))

For all of the active uptake pathways, whose cost depends on varying concentrations of N through the soil profile, the costs and fluxes are also determined by soil layer \(j\).


\subsection{Boundary conditions of FUN}
\label{\detokenize{tech_note/FUN/CLM50_Tech_Note_FUN:boundary-conditions-of-fun}}

\subsubsection{Available Carbon}
\label{\detokenize{tech_note/FUN/CLM50_Tech_Note_FUN:available-carbon}}
The carbon available for FUN, \(C_{avail}\) (gC m$^{\text{-2}}$) is the total canopy  photosynthetic uptake (GPP), minus the maintenance respiration fluxes (\(m_r\)) and multiplied by the time step in seconds (\(\delta t\)). Thus, the remainder of this chapter considers fluxes per timestep, and integrates these fluxes as they are calculated.
\begin{quote}
\begin{equation*}
\begin{split}C_{avail} = (GPP - m_r) \delta t\end{split}
\end{equation*}\end{quote}

Growth respiration is thus only calculated on the part of the carbon uptake that remains after expenditure of C by the FUN module.


\subsubsection{Available Soil Nitrogen}
\label{\detokenize{tech_note/FUN/CLM50_Tech_Note_FUN:available-soil-nitrogen}}

\subsubsection{Cost of Nitrogen Fixation}
\label{\detokenize{tech_note/FUN/CLM50_Tech_Note_FUN:cost-of-nitrogen-fixation}}\begin{description}
\item[{The cost of fixation is derived from {\hyperref[\detokenize{tech_note/References/CLM50_Tech_Note_References:houltonetal2008}]{\sphinxcrossref{\DUrole{std,std-ref}{Houlton et al. (2008)}}}}.}] \leavevmode\begin{equation*}
\begin{split}N_{cost,fix} = -s_{fix}/(1.25 e^{a_{fix} + b_{fix} . t_{soil}  (1 - 0.5 t_{soil}/ c_{fix}) })\end{split}
\end{equation*}
\end{description}

Herein, \(a_{fix}\), \(b_{fix}\) and \(c_{fix}\) are all parameters of the temperature response function of fixation reported by Houlton et al. (2008) (\(exp[a+bT_s(1-0.5T_s/c)\)).   t\_\{soil\} is the soil temperature in C. The values of these parameters are fitted to empirical data as a=-3.62 \(\pm\) 0.52, b=0.27:math:\sphinxtitleref{pm} 0.04 and c=25.15 \(\pm\) 0.66. 1.25 converts from the temperature response function to a 0-1 limitation factor (as specifically employed by Houlton et al.).  This function is a ‘rate’ of uptake for a given temperature. Here we assimilated the rate of fixation into the cost term by assuming that the rate is analagous to a conductance for N, and inverting the term to produce a cost/resistance analagoue. We then multiply this temperature term by the minimum cost at optimal temperature (\(s_{fix}\)) to give a temperature limited cost in terms of C to N ratios.


\subsubsection{Cost of Active Uptake}
\label{\detokenize{tech_note/FUN/CLM50_Tech_Note_FUN:cost-of-active-uptake}}
The cost of N uptake from soil, for each layer \(j\), is controlled by two uptake parameters that pertain respectively to the relationship between soil N content and N uptake, and root C density and N uptake.

For non-mycorrhizal uptake:
\begin{quote}
\begin{equation*}
\begin{split}N_{cost,nonmyc,j} = \frac{k_{n,nonmyc}}{N_{smin,j}} + \frac{k_{c,nonmyc}}{c_{root,j}}\end{split}
\end{equation*}\end{quote}

and for active uptake:
\begin{quote}
\begin{equation*}
\begin{split}N_{cost,active,j} = \frac{k_{n,active}}{N_{smin,j}} + \frac{k_{c,active}}{c_{root,j}}\end{split}
\end{equation*}\end{quote}

where \(k_{n,active}\) varies according to whether we are considering ecto or arbuscular mycorrhizal uptake.
\begin{quote}
\phantomsection\label{\detokenize{tech_note/FUN/CLM50_Tech_Note_FUN:equation-18.2}}\begin{equation}\label{equation:tech_note/FUN/CLM50_Tech_Note_FUN:18.2}
\begin{split}k_{n,active}  =
\left\{\begin{array}{lr}
k_{n,Eactive}&  e = 1\\
k_{n,Aactive}&  e = 0
\end{array}\right\}\end{split}
\end{equation}\end{quote}

where m=1 pertains to the fraction of the PFT that is ecotmycorrhizal, as opposed to arbuscular mycorrhizal.


\subsection{Resolving N cost across simultaneous uptake streams}
\label{\detokenize{tech_note/FUN/CLM50_Tech_Note_FUN:resolving-n-cost-across-simultaneous-uptake-streams}}
The total cost of N uptake is calculated based on the assumption that carbon is partitioned to each stream in proportion to the inverse of the cost of uptake. So, more expensive pathways receive less carbon. Earlier versions of FUN \DUrole{xref,std,std-ref}{(Fisher et al., 2010)\textless{}Fisheretal2010\textgreater{})} utilized a scheme whereby plants only took up N from the cheapest pathway. {\hyperref[\detokenize{tech_note/References/CLM50_Tech_Note_References:brzosteketal2014}]{\sphinxcrossref{\DUrole{std,std-ref}{Brzostek et al. (2014)}}}} introduced a scheme for the simultaneous uptake from different pathways. Here we calcualate a ‘conductance’ to N uptake (analagous to the inverse of the cost function conceptualized as a resistance term) \(N_{conductance}\) ( gN/gC) as:
\begin{quote}
\begin{equation*}
\begin{split}N_{conductance,f}=  \sum{(1/N_{cost,x})}\end{split}
\end{equation*}\end{quote}

From this, we then calculate the fraction of the carbon allocated to each pathway as
\begin{quote}
\begin{equation*}
\begin{split}C_{frac,x} = \frac{1/N_{cost,x}}{N_{conductance}}\end{split}
\end{equation*}\end{quote}

These fractions are used later, to calculate the carbon expended on different uptake pathways.  Next, the N acquired from each uptake stream per unit C spent (\(N_{exch,x}\), gN/gC)  is determined as
\begin{quote}
\begin{equation*}
\begin{split}N_{exch,x} = \frac{C_{frac,x}}{N_{cost,x}}\end{split}
\end{equation*}\end{quote}

We then determine the total amount of N uptake per unit C spent (\(N_{exch,tot}\), gN/gC) as the sum of all the uptake streams.
\begin{quote}
\begin{equation*}
\begin{split}N_{exch,tot} = \sum{N_{exch,x}}\end{split}
\end{equation*}\end{quote}

and thus the subsequent overall N cost is
\begin{quote}
\begin{equation*}
\begin{split}N_{cost,tot} = 1/{N_{exch,tot}}\end{split}
\end{equation*}
Retranslocation is determined via a different set of mechanisms, once the \(N_{cost,tot}\) is known.
\end{quote}


\subsection{Nitrogen Retranslocation}
\label{\detokenize{tech_note/FUN/CLM50_Tech_Note_FUN:nitrogen-retranslocation}}
The retranslocation uses an iterative algorithm to remove Nitrogen from each piece of falling litter.  There are two pathways for this, ‘free’ uptake which removes the labile N pool, and ‘paid-for’ uptake which uses C to extract N from increasingly more recalcitrant pools.

At each timestep, the pool of carbon in falling leaves (\(C_{fallingleaf}\), g m$^{\text{-2}}$) is generated from the quantity of litterfall on that day (see Phenology chapter for details). The amount of N in the litter pool (\(N_{fallingleaf}\), g m$^{\text{-2}}$) is calculated as the total leaf N multiplied by the fraction of the leaf pool passed to litter that timestep.
\begin{quote}
\begin{equation*}
\begin{split}N_{fallingleaf} = N_{leaf}.C_{fallingleaf}/C_{leaf}\end{split}
\end{equation*}\end{quote}

The carbon available at the beginning of the iterative retranslocation calculation is equal to the \(C_{avail}\) input into FUN.
\begin{quote}
\begin{equation*}
\begin{split}C_{avail,retrans,0} = C_{avail}\end{split}
\end{equation*}\end{quote}


\subsubsection{Free Retranslocation}
\label{\detokenize{tech_note/FUN/CLM50_Tech_Note_FUN:free-retranslocation}}
Some part of the leaf Nitrogen pool is removed without the need for an C expenditure.  This ‘free’ N uptake amount, (\(N_{retrans,free}\), gN m$^{\text{-2}}$) is calculated as
\begin{quote}
\begin{equation*}
\begin{split}N_{retrans,free}  = max(N_{fallingleaf} -  (C_{fallingleaf}/CN_{litter,min} ),0.0)\end{split}
\end{equation*}\end{quote}

where \(CN_{litter,min}\) is the minimum C:N ratio of the falling litter (currently set to 1.5 x the target C:N ratio).

The new \(N_{fallingleaf}\) (gN m$^{\text{-2}}$) is then determined as
\begin{quote}
\begin{equation*}
\begin{split}N_{fallingleaf} = N_{fallingleaf} - N_{retrans,free}\end{split}
\end{equation*}\end{quote}

and the new litter C:N ratio as
\begin{quote}
\begin{equation*}
\begin{split}CN_{fallingleaf}=C_{fallingleaf}/N_{fallingleaf}\end{split}
\end{equation*}\end{quote}


\subsubsection{Paid-for Retranslocation}
\label{\detokenize{tech_note/FUN/CLM50_Tech_Note_FUN:paid-for-retranslocation}}
The remaining calculations conduct an iterative calculation to determine the degree to which N retranslocation from leaves is paid for as C:N ratios and thus cost increase as N is extracted.  The iteration continues until either
\begin{enumerate}
\item {} 
The cost of retranslocation (\(cost_{retrans}\) increases beyond the cost of acquiring N from alternative pathways (\(N_{cost,tot}\)).

\item {} 
\(CN_{fallingleaf}\) rises to a maximum level, after which no more extraction is possible (representing unavoidable N loss) or

\item {} 
There is no more carbon left to pay for extraction.

\end{enumerate}

First we calculate the cost of extraction (\(cost_{retrans}\), gC/gN) for the current leaf C:N ratio as
\begin{quote}
\begin{equation*}
\begin{split}cost_{retrans}= k_{retrans} / (1/CN_{fallingleaf})^{1.3}\end{split}
\end{equation*}\end{quote}

where \(k_{retrans}\)  is a parameter controlling the overall cost of resorption, which also increases exponentially as the C:N ratio increases \sphinxstylestrong{Say something about 1.3 exponent}).
\begin{description}
\item[{Next, we calculate the amount of C needed to be spent to increase the falling leaf C:N ratio by 1.0 in this iteration \(i\) (\(C_{retrans_spent,i}\),  gC m$^{\text{-2}}$) as:}] \leavevmode\begin{equation*}
\begin{split}C_{retrans,spent,i}   = cost_{retrans}.(N_{fallingleaf} - C_{fallingleaf}/
                        (CN_{fallingleaf} + 1.0))\end{split}
\end{equation*}
\end{description}

(wherein the retranslocation cost is assumed to not change over the increment of 1.0 in C:N ratio).   Next, we calculate whether this is larger than the remaining C available to spend.
\begin{quote}
\begin{equation*}
\begin{split}C_{retrans,spent,i} = min(C_{retrans,spent,i}, C_{avail,retrans,i})\end{split}
\end{equation*}\end{quote}

The amount of N retranslocated from the leaf in this iteration (\(N_{retrans_paid,i}\),  gN m$^{\text{-2}}$) is calculated, checking that it does not fall below zero:
\begin{quote}
\begin{equation*}
\begin{split}N_{retrans,paid,i} = min(N_{fallingleaf},C_{retrans,spent,i} / cost_{retrans})\end{split}
\end{equation*}\end{quote}

The next step calculates the growth C which is accounted for by this amount of N extraction in this iteration (\(C_{retrans,accounted,i}\)).  This is calculated using the current plant C:N ratio, and also for the additional C which will need to be spent on growth respiration to build this amount of new tissue.
\begin{quote}
\begin{equation*}
\begin{split}C_{retrans,accounted,i} = N_{retrans,paid,i} . CN_{plant} . (1.0 + gr_{frac})\end{split}
\end{equation*}\end{quote}

Then the falling leaf N is updated:
\begin{quote}
\begin{equation*}
\begin{split}N_{fallingleaf}    = N_{fallingleaf} - N_{ret,i}\end{split}
\end{equation*}\end{quote}

and the \(CN_{fallingleaf}\) and cost\_\{retrans\} are updated. The amount of available carbon that is either unspent on N acquisition nor accounted for by N uptake is updated:
\begin{quote}
\begin{equation*}
\begin{split}C_{avail,retrans,i+1}  = C_{avail,retrans,i} - C_{retrans,spent,i} - C_{retrans,accounted,i}\end{split}
\end{equation*}\end{quote}


\subsubsection{Outputs of Retranslocation algorithm.}
\label{\detokenize{tech_note/FUN/CLM50_Tech_Note_FUN:outputs-of-retranslocation-algorithm}}
The final output of the retranslocation calculation are the retranslocated N (\(N_{retrans}\),  gN m$^{\text{-2}}$), C spent on retranslocation (\(C_{retrans_paid}\),  gC m$^{\text{-2}}$), and C accounted for by retranslocation (\(C_{retrans_accounted}\),  gC m$^{\text{-2}}$).

For paid-for uptake, we accumulate the total carbon spent on retranslocation (\(C_{spent_retrans}\)),
\begin{quote}
\begin{equation*}
\begin{split}C_{retrans,spent} = \sum{C_{retrans,i}}\end{split}
\end{equation*}\end{quote}

The total N acquired from retranslocation is
\begin{quote}
\begin{equation*}
\begin{split}N_{retrans} = N_{retrans,paid}+N_{retrans,free}\end{split}
\end{equation*}\end{quote}

where N acquired by paid-for retranslocation is
\begin{quote}
\begin{equation*}
\begin{split}N_{retrans,paid} = \sum{N_{retrans,paid,i}}\end{split}
\end{equation*}\end{quote}

The total carbon accounted for by retranslocation is the sum of the C accounted for by paid-for N uptake (\(N_{retrans_paid}\)) and by free N uptake (\(N_{retrans_free}\)).
\begin{quote}
\begin{equation*}
\begin{split}C_{retrans,accounted} = \sum{C_{retrans,accounted,i}}+N_{retrans,free}.CN_{plant} . (1.0 + gr_{frac})\end{split}
\end{equation*}\end{quote}

The total available carbon in FUN to spend on fixation and active uptake (\(C_{tospend}\),  gC m$^{\text{-2}}$) is calculated as the carbon available minus that account for by retranslocation:
\begin{quote}
\begin{equation*}
\begin{split}C_{tospend} = C_{avail} - C_{retrans,accounted}\end{split}
\end{equation*}\end{quote}


\subsection{Carbon expenditure on fixation and active uptake.}
\label{\detokenize{tech_note/FUN/CLM50_Tech_Note_FUN:carbon-expenditure-on-fixation-and-active-uptake}}
At each model timestep, the overall cost of N uptake is calculated (see below) in terms of C:N ratios. The available carbon (\(C_{avail}\), g m$^{\text{-2}}$ s$^{\text{-1}}$) is then allocated to two alternative outcomes, payment for N uptake, or conservation for growth. For each carbon conserved for growth, a corresponding quantity of N must be made available.  In the case where the plant target C:N ratio is fixed, the partitioning between carbon for growth (\(C_{growth}\)) and carbon for N uptake  (\(C_{nuptake}\)) is calculated by solving a system of simultaneous equations. First, the carbon available must equal the carbon spent on N uptake plus that saved for growth.
\begin{quote}
\begin{equation*}
\begin{split}C_{growth}+C_{nuptake}=C_{avail}\end{split}
\end{equation*}\end{quote}

Second, the nitrogen acquired from expenditure of N (left hand side of term below) must equal the N that is required to match the growth carbon (right hand side of term below).
\begin{quote}
\begin{equation*}
\begin{split}C_{nuptake}/N_{cost} =C_{growth}/CN_{target}\end{split}
\end{equation*}\end{quote}

The solution to these two equated terms can be used to estimate the ideal \(C_{nuptake}\) as follows,
\begin{quote}
\begin{equation*}
\begin{split}C_{nuptake} =C_{tospend}/ ( (1.0+f_{gr}*(CN_{target} / N_{cost}) + 1) .\end{split}
\end{equation*}\end{quote}

and the other C and N fluxes can be determined following the logic above.


\subsection{Modifications to allow variation in C:N ratios}
\label{\detokenize{tech_note/FUN/CLM50_Tech_Note_FUN:modifications-to-allow-variation-in-c-n-ratios}}
The original FUN model as developed by {\hyperref[\detokenize{tech_note/References/CLM50_Tech_Note_References:fisheretal2010}]{\sphinxcrossref{\DUrole{std,std-ref}{Fisher et al. (2010)}}}} and {\hyperref[\detokenize{tech_note/References/CLM50_Tech_Note_References:brzosteketal2014}]{\sphinxcrossref{\DUrole{std,std-ref}{Brzostek et al. (2014)}}}} assumes a fixed plant tissue C:N ratio. This means that in the case where N is especially limiting, all excess carbon will be utilized in an attempt to take up more Nitrogen. It has been repeatedly observed, however, that in these circumstances in real life, plants have some flexibility in the C:N stoichiometry of their tissues, and therefore, this assumption may not be realistic. \sphinxstylestrong{lit review on CN ratios}

Thus, in CLM5, we introduce the capacity for tissue C:N ratios to be prognostic, rather than static. Overall N and C availability (\(N_{uptake}\) and \(C_{growth}\)) and hence tissue C:N ratios, are both determined by FUN.  Allocation to individual tissues is discussed in the allocation chapter

Here we introduce an algorithm which adjusts the C expenditure on uptake to allow varying tissue C:N ratios. Increasing C spent on uptake will directly reduce the C:N ratio, and reducing C spent on uptake (retaining more for tissue growth) will increase it. C spent on uptake is impacted by both the N cost in the environment, and the existing tissue C:N ratio of the plant.    The output of this algorithm is \(\gamma_{FUN}\), the fraction of the ideal \(C_{nuptake}\) calculated from
the FUN equation above (\sphinxstylestrong{link equation}).
\begin{quote}
\begin{equation*}
\begin{split}C_{nuptake} = C_{nuptake}.\gamma_{FUN}\end{split}
\end{equation*}\end{quote}


\subsubsection{Response of C expenditure to Nitrogen uptake cost}
\label{\detokenize{tech_note/FUN/CLM50_Tech_Note_FUN:response-of-c-expenditure-to-nitrogen-uptake-cost}}
The environmental cost of Nitrogen (\(N_{cost,tot}\)) is used to determine \(\gamma_{FUN}\).
\begin{quote}
\begin{equation*}
\begin{split}\gamma_{FUN} = max(0.0,1.0 - (N_{cost,tot}-a_{cnflex})/b_{cnflex})\end{split}
\end{equation*}\end{quote}

where \(a_{cnflex}\) and \(b_{cnflex}\) are parameters fitted to give flexible C:N ranges over the operating range of N costs of the model. Calibration of these parameters should be subject to future testing in idealized experimental settings; they are here intended as a placeholder to allow some flexible stoichiometry, in the absence of adequate understanding of this process.  Here \(a_{cnflex}\) operates as the \(N_{cost,tot}\) above which there is a modification in the C expenditure (to allow higher C:N ratios), and \(b_{cnflex}\) is the scalar which determines how much the C expenditure is modified for a given discrepancy between \(a_{cnflex}\) and the actual cost of uptake.


\subsubsection{Response of C expenditure to plant C:N ratios}
\label{\detokenize{tech_note/FUN/CLM50_Tech_Note_FUN:response-of-c-expenditure-to-plant-c-n-ratios}}
We first calculate a \(\delta_{CN}\), which is the difference between the target C:N (\(target_{CN}\)) a model parameter, and the existing C:N ratio (\(CN_{plant}\)) \sphinxstylestrong{This isn’t strictly how it is worked out. Need to remember why we use c\_allometry instead}.
\begin{quote}
\begin{equation*}
\begin{split}CN_{plant} = \frac{C_{leaf} + C_{leaf,storage}}{N_{leaf} + N_{leaf,storage})}\end{split}
\end{equation*}\end{quote}
\begin{description}
\item[{and}] \leavevmode\begin{equation*}
\begin{split}\delta_{CN} = CN_{plant} - target_{CN}\end{split}
\end{equation*}
\end{description}

We then increase \(\gamma_{FUN}\) to  account for situations where (even if N is expensive) plant C:N ratios have increased too far from the target.  Where  \(\delta_{CN}\) is negative, we reduce C spent on N uptake and retain more C for growth
\begin{quote}
\begin{equation*}
\begin{split}\gamma_{FUN}  =
\left\{\begin{array}{lr}
\gamma_{FUN}+ 0.5.(delta_{CN}/c_{flexcn})& delta_{CN} > 0\\
\gamma_{FUN}+(1-\gamma_{FUN}).min(1,\delta_{CN}/c_{flexcn}) &  delta_{CN} < 0
\end{array}\right\}\end{split}
\end{equation*}\end{quote}

We then restrict the degree to which C expenditure can be reduced (to prevent unrealistically high C:N ratios) as
\begin{quote}
\begin{equation*}
\begin{split}\gamma_{FUN} = max(min(1.0,\gamma_{FUN}),0.5)\end{split}
\end{equation*}\end{quote}


\subsection{Calculation of N uptake streams from active uptake and fixation}
\label{\detokenize{tech_note/FUN/CLM50_Tech_Note_FUN:calculation-of-n-uptake-streams-from-active-uptake-and-fixation}}
Once the final \(C_{nuptake}\) is known, the fluxes of C to the individual pools can be derived as
\begin{quote}
\begin{equation*}
\begin{split}C_{nuptake,x}  = C_{frac,x}.C_{nuptake}\end{split}
\end{equation*}\begin{equation*}
\begin{split}N_{uptake,x}  = \frac{C_{nuptake}}{N_{cost}}\end{split}
\end{equation*}\end{quote}

Following this, we determine whether the extraction estimates exceed the pool size for each source of N.  Where \(N_{active,no3} + N_{nonmyc,no3} > N_{avail,no3}\), we calculate the unmet uptake, \(N_{unmet,no3}\)
\begin{quote}
\begin{equation*}
\begin{split}N_{unmet,no3}  = N_{active,no3} + N_{nonmyc,no3} - N_{avail,no3}\end{split}
\end{equation*}\end{quote}

then modify both fluxes to account
\begin{quote}
\begin{equation*}
\begin{split}N_{active,no3} = N_{active,no3} +  N_{unmet,no3}.\frac{N_{active,no3}}{N_{active,no3}+N_{nonmyc,no3}}\end{split}
\end{equation*}\begin{equation*}
\begin{split}N_{nonmyc,no3} = N_{nonmyc,no3} +  N_{unmet,no3}.\frac{N_{nonmyc,no3}}{N_{active,no3}+N_{nonmyc,no3}}\end{split}
\end{equation*}\end{quote}

and similarly, for NH4, where \(N_{active,nh4} + N_{nonmyc,nh4} > N_{avail,nh4}\), we calculate the unmet uptake, \(N_{unmet,no3}\)
\begin{quote}
\begin{equation*}
\begin{split}N_{unmet,nh4}  = N_{active,nh4} + N_{nonmyc,nh4} - N_{avail,nh4}\end{split}
\end{equation*}\end{quote}

then modify both fluxes to account
\begin{quote}
\begin{equation*}
\begin{split}N_{active,nh4} = N_{active,nh4} +  N_{unmet,nh4}.\frac{N_{active,nh4}}{N_{active,nh4}+N_{nonmyc,nh4}}\end{split}
\end{equation*}\begin{equation*}
\begin{split}N_{nonmyc,nh4} = N_{nonmyc,nh4} +  N_{unmet,nh4}.\frac{N_{nonmyc,nh4}}{N_{active,nh4}+N_{nonmyc,nh4}}\end{split}
\end{equation*}\end{quote}

and then update the C spent to account for hte new lower N acquisition in that layer/pool.
\begin{quote}
\begin{equation*}
\begin{split}C_{active,nh4} = N_{active,nh4}.N_{cost,active,nh4}\\
C_{active,no3} = N_{active,no3}.N_{cost,active,no3}\\
C_{nonmyc,no3} = N_{nonmyc,no3}.N_{cost,nonmyc,no3}\\
C_{nonmyc,no3} = N_{nonmyc,no3}.N_{cost,nonmyc,no3}\\\end{split}
\end{equation*}\end{quote}

Following this, we determine how much carbon is accounted for for each soil layer.
\begin{quote}
\begin{equation*}
\begin{split}C_{accounted,x,j}  =  C_{spent,j,x} - (N_{acquired,j,x}.CN_{plant}.(1.0+ gr_{frac}))\end{split}
\end{equation*}\end{quote}


\subsection{Types of N uptake streams}
\label{\detokenize{tech_note/FUN/CLM50_Tech_Note_FUN:types-of-n-uptake-streams}}
Arbuscular mycorrhizal fungi:
Ectomycorrhizal fungi:
Nonmycorrhizal plants.

ECK\_active (step 1) sets active components for Ectomycorrhizal fungi
ACK\_active (step 2) sets active components for Arbuscular fungi

kc\_nonmyc (step 1) sets nonmyc components for Ectomycorrhizal fungi
kc\_nonmyc (step 2) sets active components for Arbuscular fungi

ACTIVE vs NONMYC
ECTO vs ARBU for ACTIVE.


\section{Carbon and Nitrogen Allocation}
\label{\detokenize{tech_note/CN_Allocation/CLM50_Tech_Note_CN_Allocation:carbon-and-nitrogen-allocation}}\label{\detokenize{tech_note/CN_Allocation/CLM50_Tech_Note_CN_Allocation::doc}}\label{\detokenize{tech_note/CN_Allocation/CLM50_Tech_Note_CN_Allocation:rst-cn-allocation}}

\subsection{Introduction}
\label{\detokenize{tech_note/CN_Allocation/CLM50_Tech_Note_CN_Allocation:introduction}}
The carbon and nitrogen allocation routines in CLM determine the fate of
newly assimilated carbon, coming from the calculation of photosynthesis,
and available mineral nitrogen, coming from plant uptake of mineral
nitrogen in the soil or being drawn out of plant reserves. A significant change to CLM5 relative to prior versions is that allocation of carbon and nitrogen proceed independently rather than in a sequential manner.


\subsection{Carbon Allocation for Maintenance Respiration Costs}
\label{\detokenize{tech_note/CN_Allocation/CLM50_Tech_Note_CN_Allocation:carbon-allocation-for-maintenance-respiration-costs}}
Allocation of available carbon on each time step is prioritized, with
first priority given to the demand for carbon to support maintenance
respiration of live tissues (section 13.7). Second priority is to
replenish the internal plant carbon pool that supports maintenance
respiration during times when maintenance respiration exceeds
photosynthesis (e.g. at night, during winter for perennial vegetation,
or during periods of drought stress) (Sprugel et al., 1995). Third
priority is to support growth of new tissues, including allocation to
storage pools from which new growth will be displayed in subsequent time
steps.

The total maintenance respiration demand (\(CF_{mr}\), gC
m$^{\text{-2}}$ s$^{\text{-1}}$) is calculated as a function of
tissue mass and nitrogen concentration, and temperature (section 13.7).
The carbon supply to support this demand is composed of fluxes allocated
from carbon assimilated in the current timestep
(\(CF_{GPP,mr}\), gC m$^{\text{-2}}$ s$^{\text{-1}}$)
and from a storage pool that is drawn down when total demand exceeds
photosynthesis ( \(CF_{xs,mr}\), gC m$^{\text{-2}}$
s$^{\text{-1}}$):
\phantomsection\label{\detokenize{tech_note/CN_Allocation/CLM50_Tech_Note_CN_Allocation:equation-19.1}}\begin{equation}\label{equation:tech_note/CN_Allocation/CLM50_Tech_Note_CN_Allocation:19.1}
\begin{split}CF_{mr} =CF_{GPP,mr} +CF_{xs,mr}\end{split}
\end{equation}\phantomsection\label{\detokenize{tech_note/CN_Allocation/CLM50_Tech_Note_CN_Allocation:equation-19.2}}\begin{equation}\label{equation:tech_note/CN_Allocation/CLM50_Tech_Note_CN_Allocation:19.2}
\begin{split}CF_{GPP,mr} =\_ \left\{\begin{array}{l} {CF_{mr} \qquad \qquad {\rm for\; }CF_{mr} \le CF_{GPP} } \\ {CF_{GPP} \qquad {\rm for\; }CF_{mr} >CF_{GPP} } \end{array}\right.\end{split}
\end{equation}\phantomsection\label{\detokenize{tech_note/CN_Allocation/CLM50_Tech_Note_CN_Allocation:equation-19.3}}\begin{equation}\label{equation:tech_note/CN_Allocation/CLM50_Tech_Note_CN_Allocation:19.3}
\begin{split}CF_{xs,mr} =\_ \left\{\begin{array}{l} {0\qquad \qquad \qquad {\rm for\; }CF_{mr} \le CF_{GPP} } \\ {CF_{mr} -CF_{GPP} \qquad {\rm for\; }CF_{mr} >CF_{GPP} } \end{array}\right.\end{split}
\end{equation}
The storage pool that supplies carbon for maintenance respiration in
excess of current  \(CF_{GPP}\) ( \(CS_{xs}\), gC
m$^{\text{-2}}$) is permitted to run a deficit (negative state), and
the magnitude of this deficit determines an allocation demand which
gradually replenishes  \(CS_{xs}\). The logic for allowing a
negative state for this pool is to eliminate the need to know in advance
what the total maintenance respiration demand will be for a particular
combination of climate and plant type. Using the deficit approach, the
allocation to alleviate the deficit increases as the deficit increases,
until the supply of carbon into the pool balances the demand for carbon
leaving the pool in a quasi-steady state, with variability driven by the
seasonal cycle, climate variation, disturbance, and internal dynamics of
the plant-litter-soil system. In cases where the combination of climate
and plant type are not suitable to sustained growth, the deficit in this
pool increases until the available carbon is being allocated mostly to
alleviate the deficit, and new growth approaches zero. The allocation
flux to  \(CS_{xs}\) (\(CF_{GPP,xs}\), gC
m$^{\text{-2}}$ s$^{\text{-1}}$) is given as
\phantomsection\label{\detokenize{tech_note/CN_Allocation/CLM50_Tech_Note_CN_Allocation:equation-19.4}}\begin{equation}\label{equation:tech_note/CN_Allocation/CLM50_Tech_Note_CN_Allocation:19.4}
\begin{split}CF_{GPP,xs,pot} =\left\{\begin{array}{l} {0\qquad \qquad \qquad {\rm for\; }CS_{xs} \ge 0} \\ {-CS_{xs} /(86400\tau _{xs} )\qquad {\rm for\; }CS_{xs} <0} \end{array}\right.\end{split}
\end{equation}\phantomsection\label{\detokenize{tech_note/CN_Allocation/CLM50_Tech_Note_CN_Allocation:equation-19.5}}\begin{equation}\label{equation:tech_note/CN_Allocation/CLM50_Tech_Note_CN_Allocation:19.5}
\begin{split}CF_{GPP,xs} =\left\{\begin{array}{l} {CF_{GPP,xs,pot} \qquad \qquad \qquad {\rm for\; }CF_{GPP,xs,pot} \le CF_{GPP} -CF_{GPP,mr} } \\ {\max (CF_{GPP} -CF_{GPP,mr} ,0)\qquad {\rm for\; }CF_{GPP,xs,pot} >CF_{GPP} -CF_{GPP,mr} } \end{array}\right.\end{split}
\end{equation}
where \(\tau_{xs}\) is the time constant (currently
set to 30 days) controlling the rate of replenishment of \(CS_{xs}\).

Note that these two top-priority carbon allocation fluxes
(\(CF_{GPP,mr}\) and \(CF_{GPP,xs}\)) are not
stoichiometrically associated with any nitrogen fluxes.


\subsection{Carbon and Nitrogen Stoichiometry of New Growth}
\label{\detokenize{tech_note/CN_Allocation/CLM50_Tech_Note_CN_Allocation:carbon-and-nitrogen-stoichiometry-of-new-growth}}
After accounting for the carbon cost of maintenance respiration, the
remaining carbon flux from photosynthesis which can be allocated to new
growth (\(CF_{avail}\), gC m$^{\text{-2}}$ s$^{\text{-1}}$) is
\phantomsection\label{\detokenize{tech_note/CN_Allocation/CLM50_Tech_Note_CN_Allocation:equation-19.6}}\begin{equation}\label{equation:tech_note/CN_Allocation/CLM50_Tech_Note_CN_Allocation:19.6}
\begin{split}CF_{avail\_ alloc} =CF_{GPP} -CF_{GPP,mr} -CF_{GPP,xs} .\end{split}
\end{equation}
Potential allocation to new growth is calculated for all of the plant
carbon and nitrogen state variables based on specified C:N ratios for
each tissue type and allometric parameters that relate allocation
between various tissue types. The allometric parameters are defined as
follows:
\phantomsection\label{\detokenize{tech_note/CN_Allocation/CLM50_Tech_Note_CN_Allocation:equation-19.7}}\begin{equation}\label{equation:tech_note/CN_Allocation/CLM50_Tech_Note_CN_Allocation:19.7}
\begin{split}\begin{array}{l} {a_{1} ={\rm \; ratio\; of\; new\; fine\; root\; :\; new\; leaf\; carbon\; allocation}} \\ {a_{2} ={\rm \; ratio\; of\; new\; coarse\; root\; :\; new\; stem\; carbon\; allocation}} \\ {a_{3} ={\rm \; ratio\; of\; new\; stem\; :\; new\; leaf\; carbon\; allocation}} \\ {a_{4} ={\rm \; ratio\; new\; live\; wood\; :\; new\; total\; wood\; allocation}} \\ {g_{1} ={\rm ratio\; of\; growth\; respiration\; carbon\; :\; new\; growth\; carbon.\; }} \end{array}\end{split}
\end{equation}
Parameters \(a_{1}\), \(a_{2}\), and \(a_{4}\) are defined as constants for a given PFT (Table
13.1), while  \(g_{l }\) = 0.3 (unitless) is prescribed as a
constant for all PFTs, based on construction costs for a range of woody
and non-woody tissues (Larcher, 1995).

The model includes a dynamic allocation scheme for woody vegetation
(parameter \(a_{3}\) = -1, \hyperref[\detokenize{tech_note/CN_Allocation/CLM50_Tech_Note_CN_Allocation:table-allocation-and-cn-ratio-parameters}]{Table \ref{\detokenize{tech_note/CN_Allocation/CLM50_Tech_Note_CN_Allocation:table-allocation-and-cn-ratio-parameters}}}), in which case the
ratio for carbon allocation between new stem and new leaf increases with
increasing net primary production (NPP), as
\phantomsection\label{\detokenize{tech_note/CN_Allocation/CLM50_Tech_Note_CN_Allocation:equation-19.8}}\begin{equation}\label{equation:tech_note/CN_Allocation/CLM50_Tech_Note_CN_Allocation:19.8}
\begin{split}a_{3} =\frac{2.7}{1+e^{-0.004NPP_{ann} -300} } -0.4\end{split}
\end{equation}
where \(NPP_{ann}\) is the annual sum of NPP from the previous
year. This mechanism has the effect of increasing woody allocation in
favorable growth environments (Allen et al., 2005; Vanninen and Makela,
2005) and during the phase of stand growth prior to canopy closure
(Axelsson and Axelsson, 1986).


\begin{savenotes}\sphinxattablestart
\centering
\sphinxcapstartof{table}
\sphinxcaption{Allocation and target carbon:nitrogen ratio parameters}\label{\detokenize{tech_note/CN_Allocation/CLM50_Tech_Note_CN_Allocation:table-allocation-and-cn-ratio-parameters}}\label{\detokenize{tech_note/CN_Allocation/CLM50_Tech_Note_CN_Allocation:id1}}
\sphinxaftercaption
\begin{tabulary}{\linewidth}[t]{|T|T|T|T|T|T|T|T|T|}
\hline
\sphinxstylethead{\sphinxstyletheadfamily 
Plant functional type
\unskip}\relax &
\(a_{1}\)
&
\(a_{2}\)
&
\(a_{3}\)
&
\(a_{4}\)
&
\(Target CN_{leaf}\)
&
\(Target CN_{fr}\)
&
\(Target CN_{lw}\)
&
\(Target CN_{dw}\)
\\
\hline
NET Temperate
&
1
&
0.3
&
-1
&
0.1
&
35
&
42
&
50
&
500
\\
\hline
NET Boreal
&
1
&
0.3
&
-1
&
0.1
&
40
&
42
&
50
&
500
\\
\hline
NDT Boreal
&
1
&
0.3
&
-1
&
0.1
&
25
&
42
&
50
&
500
\\
\hline
BET Tropical
&
1
&
0.3
&
-1
&
0.1
&
30
&
42
&
50
&
500
\\
\hline
BET temperate
&
1
&
0.3
&
-1
&
0.1
&
30
&
42
&
50
&
500
\\
\hline
BDT tropical
&
1
&
0.3
&
-1
&
0.1
&
25
&
42
&
50
&
500
\\
\hline
BDT temperate
&
1
&
0.3
&
-1
&
0.1
&
25
&
42
&
50
&
500
\\
\hline
BDT boreal
&
1
&
0.3
&
-1
&
0.1
&
25
&
42
&
50
&
500
\\
\hline
BES temperate
&
1
&
0.3
&
0.2
&
0.5
&
30
&
42
&
50
&
500
\\
\hline
BDS temperate
&
1
&
0.3
&
0.2
&
0.5
&
25
&
42
&
50
&
500
\\
\hline
BDS boreal
C$_{\text{3}}$ arctic grass
&
1
1
&
0.3
0
&
0.2
0
&
0.1
0
&
25
25
&
42
42
&
50
0
&
500
0
\\
\hline
C$_{\text{3}}$ grass
&
2
&
0
&
0
&
0
&
25
&
42
&
0
&
0
\\
\hline
C$_{\text{4}}$ grass
&
2
&
0
&
0
&
0
&
25
&
42
&
0
&
0
\\
\hline
Crop R
&
2
&
0
&
0
&
0
&
25
&
42
&
0
&
0
\\
\hline
Crop I
&
2
&
0
&
0
&
0
&
25
&
42
&
0
&
0
\\
\hline
Corn R
&
2
&
0
&
0
&
1
&
25
&
42
&
50
&
500
\\
\hline
Corn I
&
2
&
0
&
0
&
1
&
25
&
42
&
50
&
500
\\
\hline
Temp Cereal R
&
2
&
0
&
0
&
1
&
25
&
42
&
50
&
500
\\
\hline
Temp Cereal I
&
2
&
0
&
0
&
1
&
25
&
42
&
50
&
500
\\
\hline
Winter Cereal R
&
2
&
0
&
0
&
1
&
25
&
42
&
50
&
500
\\
\hline
Winter Cereal I
&
2
&
0
&
0
&
1
&
25
&
42
&
50
&
500
\\
\hline
Soybean R
&
2
&
0
&
0
&
1
&
25
&
42
&
50
&
500
\\
\hline
Soybean I
&
2
&
0
&
0
&
1
&
25
&
42
&
50
&
500
\\
\hline
\end{tabulary}
\par
\sphinxattableend\end{savenotes}

Carbon to nitrogen ratios are defined for different tissue types as
follows:
\phantomsection\label{\detokenize{tech_note/CN_Allocation/CLM50_Tech_Note_CN_Allocation:equation-19.9}}\begin{equation}\label{equation:tech_note/CN_Allocation/CLM50_Tech_Note_CN_Allocation:19.9}
\begin{split}\begin{array}{l} {CN_{leaf} =\_ {\rm \; C:N\; for\; leaf}} \\ {CN_{fr} =\_ {\rm \; C:N\; for\; fine\; root}} \\ {CN_{lw} =\_ {\rm \; C:N\; for\; live\; wood\; (in\; stem\; and\; coarse\; root)}} \\ {CN_{dw} =\_ {\rm \; C:N\; for\; dead\; wood\; (in\; stem\; and\; coarse\; root)}} \end{array}\end{split}
\end{equation}
where all C:N parameters are defined as constants for a given PFT
(\hyperref[\detokenize{tech_note/CN_Allocation/CLM50_Tech_Note_CN_Allocation:table-allocation-and-cn-ratio-parameters}]{Table \ref{\detokenize{tech_note/CN_Allocation/CLM50_Tech_Note_CN_Allocation:table-allocation-and-cn-ratio-parameters}}}).

Given values for the parameters in and , total carbon and nitrogen
allocation to new growth ( \(CF_{alloc}\), gC
m$^{\text{-2}}$ s$^{\text{-1}}$, and \(NF_{alloc}\), gN
m$^{\text{-2}}$ s$^{\text{-1}}$, respectively) can be expressed as
functions of new leaf carbon allocation (\(CF_{GPP,leaf}\), gC
m$^{\text{-2}}$ s$^{\text{-1}}$):
\phantomsection\label{\detokenize{tech_note/CN_Allocation/CLM50_Tech_Note_CN_Allocation:equation-19.10}}\begin{equation}\label{equation:tech_note/CN_Allocation/CLM50_Tech_Note_CN_Allocation:19.10}
\begin{split}\begin{array}{l} {CF_{alloc} =CF_{GPP,leaf} {\kern 1pt} C_{allom} } \\ {NF_{alloc} =CF_{GPP,leaf} {\kern 1pt} N_{allom} } \end{array}\end{split}
\end{equation}
where
\phantomsection\label{\detokenize{tech_note/CN_Allocation/CLM50_Tech_Note_CN_Allocation:equation-19.11}}\begin{equation}\label{equation:tech_note/CN_Allocation/CLM50_Tech_Note_CN_Allocation:19.11}
\begin{split}\begin{array}{l} {C_{allom} =\left\{\begin{array}{l} {\left(1+g_{1} \right)\left(1+a_{1} +a_{3} \left(1+a_{2} \right)\right)\qquad {\rm for\; woody\; PFT}} \\ {1+g_{1} +a_{1} \left(1+g_{1} \right)\qquad \qquad {\rm for\; non-woody\; PFT}} \end{array}\right. } \\ {} \end{array}\end{split}
\end{equation}\phantomsection\label{\detokenize{tech_note/CN_Allocation/CLM50_Tech_Note_CN_Allocation:equation-19.12}}\begin{equation}\label{equation:tech_note/CN_Allocation/CLM50_Tech_Note_CN_Allocation:19.12}
\begin{split}N_{allom} =\left\{\begin{array}{l} {\frac{1}{CN_{leaf} } +\frac{a_{1} }{CN_{fr} } +\frac{a_{3} a_{4} \left(1+a_{2} \right)}{CN_{lw} } +} \\ {\qquad \frac{a_{3} \left(1-a_{4} \right)\left(1+a_{2} \right)}{CN_{dw} } \qquad {\rm for\; woody\; PFT}} \\ {\frac{1}{CN_{leaf} } +\frac{a_{1} }{CN_{fr} } \qquad \qquad \qquad {\rm for\; non-woody\; PFT.}} \end{array}\right.\end{split}
\end{equation}
Since the C:N stoichiometry for new growth allocation is defined, from
Eq. , as \(C_{allom}\)/ \(N_{allom}\), the total carbon available for new growth allocation
(\(CF_{avail\_alloc}\)) can be used to calculate the total
plant nitrogen demand for new growth ( \(NF_{plant\_demand}\),
gN m$^{\text{-2}}$ s$^{\text{-1}}$) as:
\phantomsection\label{\detokenize{tech_note/CN_Allocation/CLM50_Tech_Note_CN_Allocation:equation-19.13}}\begin{equation}\label{equation:tech_note/CN_Allocation/CLM50_Tech_Note_CN_Allocation:19.13}
\begin{split}NF_{plant\_ demand} =CF_{avail\_ alloc} \frac{N_{allom} }{C_{allom} } .\end{split}
\end{equation}

\subsection{Carbon Allocation to New Growth}
\label{\detokenize{tech_note/CN_Allocation/CLM50_Tech_Note_CN_Allocation:carbon-allocation-to-new-growth}}
There are two carbon pools associated with each plant tissue \textendash{} one which
represents the currently displayed tissue, and another which represents
carbon stored for display in a subsequent growth period. The nitrogen
pools follow this same organization. The model keeps track of stored
carbon according to which tissue type it will eventually be displayed
as, and the separation between display in the current timestep and
storage for later display depends on the parameter \(f_{cur}\)
(values 0 to 1). Given \(CF_{alloc,leaf}\) and \(f_{cur}\), the allocation fluxes of carbon to display and
storage pools (where storage is indicated with \sphinxstyleemphasis{\_stor}) for the various
tissue types are given as:
\phantomsection\label{\detokenize{tech_note/CN_Allocation/CLM50_Tech_Note_CN_Allocation:equation-19.14}}\begin{equation}\label{equation:tech_note/CN_Allocation/CLM50_Tech_Note_CN_Allocation:19.14}
\begin{split}CF_{alloc,leaf} \_ =CF_{alloc,leaf\_ tot} f_{cur}\end{split}
\end{equation}\phantomsection\label{\detokenize{tech_note/CN_Allocation/CLM50_Tech_Note_CN_Allocation:equation-19.15}}\begin{equation}\label{equation:tech_note/CN_Allocation/CLM50_Tech_Note_CN_Allocation:19.15}
\begin{split}CF_{alloc,leaf\_ stor} \_ =CF_{alloc,leaf\_ tot} \left(1-f_{cur} \right)\end{split}
\end{equation}\phantomsection\label{\detokenize{tech_note/CN_Allocation/CLM50_Tech_Note_CN_Allocation:equation-19.16}}\begin{equation}\label{equation:tech_note/CN_Allocation/CLM50_Tech_Note_CN_Allocation:19.16}
\begin{split}CF_{alloc,froot} \_ =CF_{alloc,leaf\_ tot} a_{1} f_{cur}\end{split}
\end{equation}\phantomsection\label{\detokenize{tech_note/CN_Allocation/CLM50_Tech_Note_CN_Allocation:equation-19.17}}\begin{equation}\label{equation:tech_note/CN_Allocation/CLM50_Tech_Note_CN_Allocation:19.17}
\begin{split}CF_{alloc,froot\_ stor} \_ =CF_{alloc,leaf\_ tot} a_{1} \left(1-f_{cur} \right)\end{split}
\end{equation}\phantomsection\label{\detokenize{tech_note/CN_Allocation/CLM50_Tech_Note_CN_Allocation:equation-19.18}}\begin{equation}\label{equation:tech_note/CN_Allocation/CLM50_Tech_Note_CN_Allocation:19.18}
\begin{split}CF_{alloc,livestem} \_ =CF_{alloc,leaf\_ tot} a_{3} a_{4} f_{cur}\end{split}
\end{equation}\phantomsection\label{\detokenize{tech_note/CN_Allocation/CLM50_Tech_Note_CN_Allocation:equation-19.19}}\begin{equation}\label{equation:tech_note/CN_Allocation/CLM50_Tech_Note_CN_Allocation:19.19}
\begin{split}CF_{alloc,livestem\_ stor} \_ =CF_{alloc,leaf\_ tot} a_{3} a_{4} \left(1-f_{cur} \right)\end{split}
\end{equation}\phantomsection\label{\detokenize{tech_note/CN_Allocation/CLM50_Tech_Note_CN_Allocation:equation-19.20}}\begin{equation}\label{equation:tech_note/CN_Allocation/CLM50_Tech_Note_CN_Allocation:19.20}
\begin{split}CF_{alloc,deadstem} \_ =CF_{alloc,leaf\_ tot} a_{3} \left(1-a_{4} \right)f_{cur}\end{split}
\end{equation}\phantomsection\label{\detokenize{tech_note/CN_Allocation/CLM50_Tech_Note_CN_Allocation:equation-19.21}}\begin{equation}\label{equation:tech_note/CN_Allocation/CLM50_Tech_Note_CN_Allocation:19.21}
\begin{split}CF_{alloc,deadstem\_ stor} \_ =CF_{alloc,leaf\_ tot} a_{3} \left(1-a_{4} \right)\left(1-f_{cur} \right)\end{split}
\end{equation}\phantomsection\label{\detokenize{tech_note/CN_Allocation/CLM50_Tech_Note_CN_Allocation:equation-19.22}}\begin{equation}\label{equation:tech_note/CN_Allocation/CLM50_Tech_Note_CN_Allocation:19.22}
\begin{split}CF_{alloc,livecroot} \_ =CF_{alloc,leaf\_ tot} a_{2} a_{3} a_{4} f_{cur}\end{split}
\end{equation}\phantomsection\label{\detokenize{tech_note/CN_Allocation/CLM50_Tech_Note_CN_Allocation:equation-19.23}}\begin{equation}\label{equation:tech_note/CN_Allocation/CLM50_Tech_Note_CN_Allocation:19.23}
\begin{split}CF_{alloc,livecroot\_ stor} \_ =CF_{alloc,leaf\_ tot} a_{2} a_{3} a_{4} \left(1-f_{cur} \right)\end{split}
\end{equation}\phantomsection\label{\detokenize{tech_note/CN_Allocation/CLM50_Tech_Note_CN_Allocation:equation-19.24}}\begin{equation}\label{equation:tech_note/CN_Allocation/CLM50_Tech_Note_CN_Allocation:19.24}
\begin{split}CF_{alloc,deadcroot} \_ =CF_{alloc,leaf\_ tot} a_{2} a_{3} \left(1-a_{4} \right)f_{cur}\end{split}
\end{equation}\phantomsection\label{\detokenize{tech_note/CN_Allocation/CLM50_Tech_Note_CN_Allocation:equation-19.25}}\begin{equation}\label{equation:tech_note/CN_Allocation/CLM50_Tech_Note_CN_Allocation:19.25}
\begin{split}CF_{alloc,deadcroot\_ stor} \_ =CF_{alloc,leaf\_ tot} a_{2} a_{3} \left(1-a_{4} \right)\left(1-f_{cur} \right).\end{split}
\end{equation}

\subsection{Nitrogen allocation}
\label{\detokenize{tech_note/CN_Allocation/CLM50_Tech_Note_CN_Allocation:nitrogen-allocation}}
The total flux of nitrogen to be allocated is given by the FUN model (Chapter \hyperref[\detokenize{tech_note/FUN/CLM50_Tech_Note_FUN:rst-fun}]{\ref{\detokenize{tech_note/FUN/CLM50_Tech_Note_FUN:rst-fun}}}).  This gives a total N to be allocated within a given timestep, \(N_{supply}\).  The total N allocated for a given tissue \(i\) is the minimum between the supply and the demand:
\phantomsection\label{\detokenize{tech_note/CN_Allocation/CLM50_Tech_Note_CN_Allocation:equation-19.26}}\begin{equation}\label{equation:tech_note/CN_Allocation/CLM50_Tech_Note_CN_Allocation:19.26}
\begin{split}NF_{alloc,i} = min \left( NF_{demand, i}, NF_{supply, i} \right)\end{split}
\end{equation}
The demand for each tissue, calculated for the tissue to remain on stoichiometry during growth, is:
\phantomsection\label{\detokenize{tech_note/CN_Allocation/CLM50_Tech_Note_CN_Allocation:equation-19.27}}\begin{equation}\label{equation:tech_note/CN_Allocation/CLM50_Tech_Note_CN_Allocation:19.27}
\begin{split}NF_{demand,leaf} \_ =\frac{CF_{alloc,leaf\_ tot} }{CN_{leaf} } f_{cur}\end{split}
\end{equation}\phantomsection\label{\detokenize{tech_note/CN_Allocation/CLM50_Tech_Note_CN_Allocation:equation-19.28}}\begin{equation}\label{equation:tech_note/CN_Allocation/CLM50_Tech_Note_CN_Allocation:19.28}
\begin{split}NF_{demand,leaf\_ stor} \_ =\frac{CF_{alloc,leaf\_ tot} }{CN_{leaf} } \left(1-f_{cur} \right)\end{split}
\end{equation}\phantomsection\label{\detokenize{tech_note/CN_Allocation/CLM50_Tech_Note_CN_Allocation:equation-19.29}}\begin{equation}\label{equation:tech_note/CN_Allocation/CLM50_Tech_Note_CN_Allocation:19.29}
\begin{split}NF_{demand,froot} \_ =\frac{CF_{alloc,leaf\_ tot} a_{1} }{CN_{fr} } f_{cur}\end{split}
\end{equation}\phantomsection\label{\detokenize{tech_note/CN_Allocation/CLM50_Tech_Note_CN_Allocation:equation-19.30}}\begin{equation}\label{equation:tech_note/CN_Allocation/CLM50_Tech_Note_CN_Allocation:19.30}
\begin{split}NF_{demand,froot\_ stor} \_ =\frac{CF_{alloc,leaf\_ tot} a_{1} }{CN_{fr} } \left(1-f_{cur} \right)\end{split}
\end{equation}\phantomsection\label{\detokenize{tech_note/CN_Allocation/CLM50_Tech_Note_CN_Allocation:equation-19.31}}\begin{equation}\label{equation:tech_note/CN_Allocation/CLM50_Tech_Note_CN_Allocation:19.31}
\begin{split}NF_{demand,livestem} \_ =\frac{CF_{alloc,leaf\_ tot} a_{3} a_{4} }{CN_{lw} } f_{cur}\end{split}
\end{equation}\phantomsection\label{\detokenize{tech_note/CN_Allocation/CLM50_Tech_Note_CN_Allocation:equation-19.32}}\begin{equation}\label{equation:tech_note/CN_Allocation/CLM50_Tech_Note_CN_Allocation:19.32}
\begin{split}NF_{demand,livestem\_ stor} \_ =\frac{CF_{alloc,leaf\_ tot} a_{3} a_{4} }{CN_{lw} } \left(1-f_{cur} \right)\end{split}
\end{equation}\phantomsection\label{\detokenize{tech_note/CN_Allocation/CLM50_Tech_Note_CN_Allocation:equation-19.33}}\begin{equation}\label{equation:tech_note/CN_Allocation/CLM50_Tech_Note_CN_Allocation:19.33}
\begin{split}NF_{demand,deadstem} \_ =\frac{CF_{alloc,leaf\_ tot} a_{3} \left(1-a_{4} \right)}{CN_{dw} } f_{cur}\end{split}
\end{equation}\phantomsection\label{\detokenize{tech_note/CN_Allocation/CLM50_Tech_Note_CN_Allocation:equation-19.34}}\begin{equation}\label{equation:tech_note/CN_Allocation/CLM50_Tech_Note_CN_Allocation:19.34}
\begin{split}NF_{demand,deadstem\_ stor} \_ =\frac{CF_{alloc,leaf\_ tot} a_{3} \left(1-a_{4} \right)}{CN_{dw} } \left(1-f_{cur} \right)\end{split}
\end{equation}\phantomsection\label{\detokenize{tech_note/CN_Allocation/CLM50_Tech_Note_CN_Allocation:equation-19.35}}\begin{equation}\label{equation:tech_note/CN_Allocation/CLM50_Tech_Note_CN_Allocation:19.35}
\begin{split}NF_{demand,livecroot} \_ =\frac{CF_{alloc,leaf\_ tot} a_{2} a_{3} a_{4} }{CN_{lw} } f_{cur}\end{split}
\end{equation}\phantomsection\label{\detokenize{tech_note/CN_Allocation/CLM50_Tech_Note_CN_Allocation:equation-19.36}}\begin{equation}\label{equation:tech_note/CN_Allocation/CLM50_Tech_Note_CN_Allocation:19.36}
\begin{split}NF_{demand,livecroot\_ stor} \_ =\frac{CF_{alloc,leaf\_ tot} a_{2} a_{3} a_{4} }{CN_{lw} } \left(1-f_{cur} \right)\end{split}
\end{equation}\phantomsection\label{\detokenize{tech_note/CN_Allocation/CLM50_Tech_Note_CN_Allocation:equation-19.37}}\begin{equation}\label{equation:tech_note/CN_Allocation/CLM50_Tech_Note_CN_Allocation:19.37}
\begin{split}NF_{demand,deadcroot} \_ =\frac{CF_{alloc,leaf\_ tot} a_{2} a_{3} \left(1-a_{4} \right)}{CN_{dw} } f_{cur}\end{split}
\end{equation}\phantomsection\label{\detokenize{tech_note/CN_Allocation/CLM50_Tech_Note_CN_Allocation:equation-19.38}}\begin{equation}\label{equation:tech_note/CN_Allocation/CLM50_Tech_Note_CN_Allocation:19.38}
\begin{split}NF_{demand,deadcroot\_ stor} \_ =\frac{CF_{alloc,leaf} a_{2} a_{3} \left(1-a_{4} \right)}{CN_{dw} } \left(1-f_{cur} \right).\end{split}
\end{equation}
After each pool’s demand is calculated, the total plant N demand is then the sum of each individual pool :math: \sphinxtitleref{i} corresponding to each tissue:
\phantomsection\label{\detokenize{tech_note/CN_Allocation/CLM50_Tech_Note_CN_Allocation:equation-19.39}}\begin{equation}\label{equation:tech_note/CN_Allocation/CLM50_Tech_Note_CN_Allocation:19.39}
\begin{split}NF_{demand,tot} = \sum _{i=tissues} NF_{demand,i}\end{split}
\end{equation}
and the total supply for each tissue :math: \sphinxtitleref{i} is the product of the fractional demand and the total available N, calculated as the term :math: \sphinxtitleref{N\_\{uptake\}} equal to the sum of the eight N uptake streams described in the FUN model (Chapter \hyperref[\detokenize{tech_note/FUN/CLM50_Tech_Note_FUN:rst-fun}]{\ref{\detokenize{tech_note/FUN/CLM50_Tech_Note_FUN:rst-fun}}}).
\phantomsection\label{\detokenize{tech_note/CN_Allocation/CLM50_Tech_Note_CN_Allocation:equation-19.40}}\begin{equation}\label{equation:tech_note/CN_Allocation/CLM50_Tech_Note_CN_Allocation:19.40}
\begin{split}NF_{alloc,i} = N_{uptake} NF_{demand,i} / NF_{demand,tot}\end{split}
\end{equation}

\section{Vegetation Phenology and Turnover}
\label{\detokenize{tech_note/Vegetation_Phenology_Turnover/CLM50_Tech_Note_Vegetation_Phenology_Turnover:vegetation-phenology-and-turnover}}\label{\detokenize{tech_note/Vegetation_Phenology_Turnover/CLM50_Tech_Note_Vegetation_Phenology_Turnover::doc}}\label{\detokenize{tech_note/Vegetation_Phenology_Turnover/CLM50_Tech_Note_Vegetation_Phenology_Turnover:rst-vegetation-phenology-and-turnover}}
The CLM phenology model consists of several algorithms controlling the
transfer of stored carbon and nitrogen out of storage pools for the
display of new growth and into litter pools for losses of displayed
growth. PFTs are classified into three distinct phenological types that
are represented by separate algorithms: an evergreen type, for which
some fraction of annual leaf growth persists in the displayed pool for
longer than one year; a seasonal-deciduous type with a single growing
season per year, controlled mainly by temperature and daylength; and a
stress-deciduous type with the potential for multiple growing seasons
per year, controlled by temperature and soil moisture conditions.

The three phenology types share a common set of control variables. The
calculation of the phenology fluxes is generalized, operating
identically for all three phenology types, given a specification of the
common control variables. The following sections describe first the
general flux parameterization, followed by the algorithms for setting
the control parameters for the three phenology types.


\subsection{General Phenology Flux Parameterization}
\label{\detokenize{tech_note/Vegetation_Phenology_Turnover/CLM50_Tech_Note_Vegetation_Phenology_Turnover:general-phenology-flux-parameterization}}
Fluxes of carbon and nitrogen from storage pools and into displayed
tissue pools pass through a special transfer pool (denoted \sphinxstyleemphasis{\_xfer}),
maintained as a separate state variable for each tissue type. Storage
(\sphinxstyleemphasis{\_stor}) and transfer (\sphinxstyleemphasis{\_xfer}) pools are maintained separately to
reduce the complexity of accounting for transfers into and out of
storage over the course of a single growing season.

\begin{figure}[htbp]
\centering
\capstart

\noindent\sphinxincludegraphics{{image110}.png}
\caption{Example of annual phenology cycle for seasonal deciduous.}\label{\detokenize{tech_note/Vegetation_Phenology_Turnover/CLM50_Tech_Note_Vegetation_Phenology_Turnover:figure-annual-phenology-cycle}}\label{\detokenize{tech_note/Vegetation_Phenology_Turnover/CLM50_Tech_Note_Vegetation_Phenology_Turnover:id1}}\end{figure}


\subsubsection{14.1.1 Onset Periods}
\label{\detokenize{tech_note/Vegetation_Phenology_Turnover/CLM50_Tech_Note_Vegetation_Phenology_Turnover:onset-periods}}
The deciduous phenology algorithms specify the occurrence of onset
growth periods (Figure 14.1). Carbon fluxes from the transfer pools into
displayed growth are calculated during these periods as:
\phantomsection\label{\detokenize{tech_note/Vegetation_Phenology_Turnover/CLM50_Tech_Note_Vegetation_Phenology_Turnover:equation-20.1)}}\begin{equation}\label{equation:tech_note/Vegetation_Phenology_Turnover/CLM50_Tech_Note_Vegetation_Phenology_Turnover:20.1)}
\begin{split}CF_{leaf\_ xfer,leaf} =r_{xfer\_ on} CS_{leaf\_ xfer}\end{split}
\end{equation}\phantomsection\label{\detokenize{tech_note/Vegetation_Phenology_Turnover/CLM50_Tech_Note_Vegetation_Phenology_Turnover:equation-20.2)}}\begin{equation}\label{equation:tech_note/Vegetation_Phenology_Turnover/CLM50_Tech_Note_Vegetation_Phenology_Turnover:20.2)}
\begin{split}CF_{froot\_ xfer,froot} =r_{xfer\_ on} CS_{froot\_ xfer}\end{split}
\end{equation}\phantomsection\label{\detokenize{tech_note/Vegetation_Phenology_Turnover/CLM50_Tech_Note_Vegetation_Phenology_Turnover:equation-20.3)}}\begin{equation}\label{equation:tech_note/Vegetation_Phenology_Turnover/CLM50_Tech_Note_Vegetation_Phenology_Turnover:20.3)}
\begin{split}CF_{livestem\_ xfer,livestem} =r_{xfer\_ on} CS_{livestem\_ xfer}\end{split}
\end{equation}\phantomsection\label{\detokenize{tech_note/Vegetation_Phenology_Turnover/CLM50_Tech_Note_Vegetation_Phenology_Turnover:equation-20.4)}}\begin{equation}\label{equation:tech_note/Vegetation_Phenology_Turnover/CLM50_Tech_Note_Vegetation_Phenology_Turnover:20.4)}
\begin{split}CF_{deadstem\_ xfer,deadstem} =r_{xfer\_ on} CS_{deadstem\_ xfer}\end{split}
\end{equation}\phantomsection\label{\detokenize{tech_note/Vegetation_Phenology_Turnover/CLM50_Tech_Note_Vegetation_Phenology_Turnover:equation-20.5)}}\begin{equation}\label{equation:tech_note/Vegetation_Phenology_Turnover/CLM50_Tech_Note_Vegetation_Phenology_Turnover:20.5)}
\begin{split}CF_{livecroot\_ xfer,livecroot} =r_{xfer\_ on} CS_{livecroot\_ xfer}\end{split}
\end{equation}\phantomsection\label{\detokenize{tech_note/Vegetation_Phenology_Turnover/CLM50_Tech_Note_Vegetation_Phenology_Turnover:equation-20.6)}}\begin{equation}\label{equation:tech_note/Vegetation_Phenology_Turnover/CLM50_Tech_Note_Vegetation_Phenology_Turnover:20.6)}
\begin{split}CF_{deadcroot\_ xfer,deadcroot} =r_{xfer\_ on} CS_{deadcroot\_ xfer} ,\end{split}
\end{equation}
with corresponding nitrogen fluxes:
\phantomsection\label{\detokenize{tech_note/Vegetation_Phenology_Turnover/CLM50_Tech_Note_Vegetation_Phenology_Turnover:equation-20.7)}}\begin{equation}\label{equation:tech_note/Vegetation_Phenology_Turnover/CLM50_Tech_Note_Vegetation_Phenology_Turnover:20.7)}
\begin{split}NF_{leaf\_ xfer,leaf} =r_{xfer\_ on} NS_{leaf\_ xfer}\end{split}
\end{equation}\phantomsection\label{\detokenize{tech_note/Vegetation_Phenology_Turnover/CLM50_Tech_Note_Vegetation_Phenology_Turnover:equation-20.8)}}\begin{equation}\label{equation:tech_note/Vegetation_Phenology_Turnover/CLM50_Tech_Note_Vegetation_Phenology_Turnover:20.8)}
\begin{split}NF_{froot\_ xfer,froot} =r_{xfer\_ on} NS_{froot\_ xfer}\end{split}
\end{equation}\phantomsection\label{\detokenize{tech_note/Vegetation_Phenology_Turnover/CLM50_Tech_Note_Vegetation_Phenology_Turnover:equation-20.9)}}\begin{equation}\label{equation:tech_note/Vegetation_Phenology_Turnover/CLM50_Tech_Note_Vegetation_Phenology_Turnover:20.9)}
\begin{split}NF_{livestem\_ xfer,livestem} =r_{xfer\_ on} NS_{livestem\_ xfer}\end{split}
\end{equation}\phantomsection\label{\detokenize{tech_note/Vegetation_Phenology_Turnover/CLM50_Tech_Note_Vegetation_Phenology_Turnover:equation-20.10)}}\begin{equation}\label{equation:tech_note/Vegetation_Phenology_Turnover/CLM50_Tech_Note_Vegetation_Phenology_Turnover:20.10)}
\begin{split}NF_{deadstem\_ xfer,deadstem} =r_{xfer\_ on} NS_{deadstem\_ xfer}\end{split}
\end{equation}\phantomsection\label{\detokenize{tech_note/Vegetation_Phenology_Turnover/CLM50_Tech_Note_Vegetation_Phenology_Turnover:equation-20.11)}}\begin{equation}\label{equation:tech_note/Vegetation_Phenology_Turnover/CLM50_Tech_Note_Vegetation_Phenology_Turnover:20.11)}
\begin{split}NF_{livecroot\_ xfer,livecroot} =r_{xfer\_ on} NS_{livecroot\_ xfer}\end{split}
\end{equation}\phantomsection\label{\detokenize{tech_note/Vegetation_Phenology_Turnover/CLM50_Tech_Note_Vegetation_Phenology_Turnover:equation-20.12)}}\begin{equation}\label{equation:tech_note/Vegetation_Phenology_Turnover/CLM50_Tech_Note_Vegetation_Phenology_Turnover:20.12)}
\begin{split}NF_{deadcroot\_ xfer,deadcroot} =r_{xfer\_ on} NS_{deadcroot\_ xfer} ,\end{split}
\end{equation}
where CF is the carbon flux, CS is stored carbon, NF is the nitrogen
flux, NS is stored nitrogen, \({r}_{xfer\_on}\) (s$^{\text{-1}}$) is a time-varying rate coefficient controlling flux
out of the transfer pool:
\phantomsection\label{\detokenize{tech_note/Vegetation_Phenology_Turnover/CLM50_Tech_Note_Vegetation_Phenology_Turnover:equation-ZEqnNum852972}}\begin{equation}\label{equation:tech_note/Vegetation_Phenology_Turnover/CLM50_Tech_Note_Vegetation_Phenology_Turnover:ZEqnNum852972}
\begin{split}r_{xfer\_ on} =\left\{\begin{array}{l} {{2\mathord{\left/ {\vphantom {2 t_{onset} }} \right. \kern-\nulldelimiterspace} t_{onset} } \qquad {\rm for\; }t_{onset} \ne \Delta t} \\ {{1\mathord{\left/ {\vphantom {1 \Delta t}} \right. \kern-\nulldelimiterspace} \Delta t} \qquad {\rm for\; }t_{onset} =\Delta t} \end{array}\right.\end{split}
\end{equation}
and \sphinxstyleemphasis{t}$_{\text{onset}}$ (s) is the number of seconds remaining in
the current phenology onset growth period (Figure 14.1). The form of Eq. \eqref{equation:tech_note/Vegetation_Phenology_Turnover/CLM50_Tech_Note_Vegetation_Phenology_Turnover:ZEqnNum852972}
produces a flux from the transfer pool which declines linearly over the
onset growth period, approaching zero flux in the final timestep.


\subsubsection{14.1.2 Offset Periods}
\label{\detokenize{tech_note/Vegetation_Phenology_Turnover/CLM50_Tech_Note_Vegetation_Phenology_Turnover:offset-periods}}
The deciduous phenology algorithms also specify the occurrence of
litterfall during offset periods. In contrast to the onset periods, only
leaf and fine root state variables are subject to litterfall fluxes.
Carbon fluxes from display pools into litter are calculated during these
periods as:
\phantomsection\label{\detokenize{tech_note/Vegetation_Phenology_Turnover/CLM50_Tech_Note_Vegetation_Phenology_Turnover:equation-20.14)}}\begin{equation}\label{equation:tech_note/Vegetation_Phenology_Turnover/CLM50_Tech_Note_Vegetation_Phenology_Turnover:20.14)}
\begin{split}CF_{leaf,litter}^{n} =\left\{\begin{array}{l} {CF_{leaf,litter}^{n-1} +r_{xfer\_ off} \left(CS_{leaf} -CF_{leaf,litter}^{n-1} {\kern 1pt} t_{offset} \right)\qquad {\rm for\; }t_{offset} \ne \Delta t} \\ {\left({CS_{leaf} \mathord{\left/ {\vphantom {CS_{leaf}  \Delta t}} \right. \kern-\nulldelimiterspace} \Delta t} \right)+CF_{alloc,leaf} \qquad \qquad \qquad \qquad {\rm for\; }t_{offset} =\Delta t} \end{array}\right.\end{split}
\end{equation}\phantomsection\label{\detokenize{tech_note/Vegetation_Phenology_Turnover/CLM50_Tech_Note_Vegetation_Phenology_Turnover:equation-20.15)}}\begin{equation}\label{equation:tech_note/Vegetation_Phenology_Turnover/CLM50_Tech_Note_Vegetation_Phenology_Turnover:20.15)}
\begin{split}CF_{froot,litter}^{n} =\left\{\begin{array}{l} {CF_{froot,litter}^{n-1} +r_{xfer\_ off} \left(CS_{froot} -CF_{froot,litter}^{n-1} {\kern 1pt} t_{offset} \right)\qquad {\rm for\; }t_{offset} \ne \Delta t} \\ {\left({CS_{froot} \mathord{\left/ {\vphantom {CS_{froot}  \Delta t}} \right. \kern-\nulldelimiterspace} \Delta t} \right)+CF_{alloc,\, froot} \qquad \qquad \qquad {\rm for\; }t_{offset} =\Delta t} \end{array}\right.\end{split}
\end{equation}\phantomsection\label{\detokenize{tech_note/Vegetation_Phenology_Turnover/CLM50_Tech_Note_Vegetation_Phenology_Turnover:equation-20.16)}}\begin{equation}\label{equation:tech_note/Vegetation_Phenology_Turnover/CLM50_Tech_Note_Vegetation_Phenology_Turnover:20.16)}
\begin{split}r_{xfer\_ off} =\frac{2\Delta t}{t_{offset} ^{2} }\end{split}
\end{equation}
where superscripts \sphinxstyleemphasis{n} and \sphinxstyleemphasis{n-1} refer to fluxes on the current and
previous timesteps, respectively. The rate coefficient \({r}_{xfer\_off}\) varies with time to produce a linearly
increasing litterfall rate throughout the offset period, and the special
case for fluxes in the final litterfall timestep
(\({t}_{offset}\) = \(\Delta t\)) ensures that all of the
displayed growth is sent to the litter pools for deciduous plant types.

Corresponding nitrogen fluxes during litterfall take into account retranslocation of nitrogen out of the displayed leaf pool prior to
litterfall (\({NF}_{leaf,retrans}\), gN m$^{\text{-2}}$ s$^{\text{-1}}$). Retranslocation of nitrogen out of fine roots is
assumed to be negligible. The fluxes are:
\phantomsection\label{\detokenize{tech_note/Vegetation_Phenology_Turnover/CLM50_Tech_Note_Vegetation_Phenology_Turnover:equation-20.17)}}\begin{equation}\label{equation:tech_note/Vegetation_Phenology_Turnover/CLM50_Tech_Note_Vegetation_Phenology_Turnover:20.17)}
\begin{split}NF_{leaf,litter} ={CF_{leaf,litter} \mathord{\left/ {\vphantom {CF_{leaf,litter}  CN_{leaf\_ litter} }} \right. \kern-\nulldelimiterspace} CN_{leaf\_ litter} }\end{split}
\end{equation}\phantomsection\label{\detokenize{tech_note/Vegetation_Phenology_Turnover/CLM50_Tech_Note_Vegetation_Phenology_Turnover:equation-20.18)}}\begin{equation}\label{equation:tech_note/Vegetation_Phenology_Turnover/CLM50_Tech_Note_Vegetation_Phenology_Turnover:20.18)}
\begin{split}NF_{froot,litter} ={CF_{leaf,litter} \mathord{\left/ {\vphantom {CF_{leaf,litter}  CN_{froot} }} \right. \kern-\nulldelimiterspace} CN_{froot} }\end{split}
\end{equation}\phantomsection\label{\detokenize{tech_note/Vegetation_Phenology_Turnover/CLM50_Tech_Note_Vegetation_Phenology_Turnover:equation-20.19)}}\begin{equation}\label{equation:tech_note/Vegetation_Phenology_Turnover/CLM50_Tech_Note_Vegetation_Phenology_Turnover:20.19)}
\begin{split}NF_{leaf,retrans} =\left({CF_{leaf,litter} \mathord{\left/ {\vphantom {CF_{leaf,litter}  CN_{leaf} }} \right. \kern-\nulldelimiterspace} CN_{leaf} } \right)-NF_{leaf,litter} .\end{split}
\end{equation}
where CN is C:N.


\subsubsection{14.1.3 Background Onset Growth}
\label{\detokenize{tech_note/Vegetation_Phenology_Turnover/CLM50_Tech_Note_Vegetation_Phenology_Turnover:background-onset-growth}}
The stress-deciduous phenology algorithm includes a provision for the
case when stress signals are absent, and the vegetation shifts from a
deciduous habit to an evergreen habit, until the next occurrence of an
offset stress trigger . In that case, the regular onset flux mechanism
is switched off and a background onset growth algorithm is invoked
(\({r}_{bgtr} >  0\)). During this period, small fluxes
of carbon and nitrogen from the storage pools into the associated
transfer pools are calculated on each time step, and the entire contents
of the transfer pool are added to the associated displayed growth pool
on each time step. The carbon fluxes from transfer to display pools
under these conditions are:
\phantomsection\label{\detokenize{tech_note/Vegetation_Phenology_Turnover/CLM50_Tech_Note_Vegetation_Phenology_Turnover:equation-20.20)}}\begin{equation}\label{equation:tech_note/Vegetation_Phenology_Turnover/CLM50_Tech_Note_Vegetation_Phenology_Turnover:20.20)}
\begin{split}CF_{leaf\_ xfer,leaf} ={CS_{leaf\_ xfer} \mathord{\left/ {\vphantom {CS_{leaf\_ xfer}  \Delta t}} \right. \kern-\nulldelimiterspace} \Delta t}\end{split}
\end{equation}\phantomsection\label{\detokenize{tech_note/Vegetation_Phenology_Turnover/CLM50_Tech_Note_Vegetation_Phenology_Turnover:equation-20.21)}}\begin{equation}\label{equation:tech_note/Vegetation_Phenology_Turnover/CLM50_Tech_Note_Vegetation_Phenology_Turnover:20.21)}
\begin{split}CF_{froot\_ xfer,froot} ={CS_{froot\_ xfer} \mathord{\left/ {\vphantom {CS_{froot\_ xfer}  \Delta t}} \right. \kern-\nulldelimiterspace} \Delta t}\end{split}
\end{equation}\phantomsection\label{\detokenize{tech_note/Vegetation_Phenology_Turnover/CLM50_Tech_Note_Vegetation_Phenology_Turnover:equation-20.22)}}\begin{equation}\label{equation:tech_note/Vegetation_Phenology_Turnover/CLM50_Tech_Note_Vegetation_Phenology_Turnover:20.22)}
\begin{split}CF_{livestem\_ xfer,livestem} ={CS_{livestem\_ xfer} \mathord{\left/ {\vphantom {CS_{livestem\_ xfer}  \Delta t}} \right. \kern-\nulldelimiterspace} \Delta t}\end{split}
\end{equation}\phantomsection\label{\detokenize{tech_note/Vegetation_Phenology_Turnover/CLM50_Tech_Note_Vegetation_Phenology_Turnover:equation-20.23)}}\begin{equation}\label{equation:tech_note/Vegetation_Phenology_Turnover/CLM50_Tech_Note_Vegetation_Phenology_Turnover:20.23)}
\begin{split}CF_{deadstem\_ xfer,deadstem} ={CS_{deadstem\_ xfer} \mathord{\left/ {\vphantom {CS_{deadstem\_ xfer}  \Delta t}} \right. \kern-\nulldelimiterspace} \Delta t}\end{split}
\end{equation}\phantomsection\label{\detokenize{tech_note/Vegetation_Phenology_Turnover/CLM50_Tech_Note_Vegetation_Phenology_Turnover:equation-20.24)}}\begin{equation}\label{equation:tech_note/Vegetation_Phenology_Turnover/CLM50_Tech_Note_Vegetation_Phenology_Turnover:20.24)}
\begin{split}CF_{livecroot\_ xfer,livecroot} ={CS_{livecroot\_ xfer} \mathord{\left/ {\vphantom {CS_{livecroot\_ xfer}  \Delta t}} \right. \kern-\nulldelimiterspace} \Delta t}\end{split}
\end{equation}\phantomsection\label{\detokenize{tech_note/Vegetation_Phenology_Turnover/CLM50_Tech_Note_Vegetation_Phenology_Turnover:equation-20.25)}}\begin{equation}\label{equation:tech_note/Vegetation_Phenology_Turnover/CLM50_Tech_Note_Vegetation_Phenology_Turnover:20.25)}
\begin{split}CF_{deadcroot\_ xfer,deadcroot} ={CS_{deadcroot\_ xfer} \mathord{\left/ {\vphantom {CS_{deadcroot\_ xfer}  \Delta t}} \right. \kern-\nulldelimiterspace} \Delta t} ,\end{split}
\end{equation}
and the corresponding nitrogen fluxes are:
\phantomsection\label{\detokenize{tech_note/Vegetation_Phenology_Turnover/CLM50_Tech_Note_Vegetation_Phenology_Turnover:equation-20.26)}}\begin{equation}\label{equation:tech_note/Vegetation_Phenology_Turnover/CLM50_Tech_Note_Vegetation_Phenology_Turnover:20.26)}
\begin{split}NF_{leaf\_ xfer,leaf} ={NS_{leaf\_ xfer} \mathord{\left/ {\vphantom {NS_{leaf\_ xfer}  \Delta t}} \right. \kern-\nulldelimiterspace} \Delta t}\end{split}
\end{equation}\phantomsection\label{\detokenize{tech_note/Vegetation_Phenology_Turnover/CLM50_Tech_Note_Vegetation_Phenology_Turnover:equation-20.27)}}\begin{equation}\label{equation:tech_note/Vegetation_Phenology_Turnover/CLM50_Tech_Note_Vegetation_Phenology_Turnover:20.27)}
\begin{split}NF_{froot\_ xfer,froot} ={NS_{froot\_ xfer} \mathord{\left/ {\vphantom {NS_{froot\_ xfer}  \Delta t}} \right. \kern-\nulldelimiterspace} \Delta t}\end{split}
\end{equation}\phantomsection\label{\detokenize{tech_note/Vegetation_Phenology_Turnover/CLM50_Tech_Note_Vegetation_Phenology_Turnover:equation-20.28)}}\begin{equation}\label{equation:tech_note/Vegetation_Phenology_Turnover/CLM50_Tech_Note_Vegetation_Phenology_Turnover:20.28)}
\begin{split}NF_{livestem\_ xfer,livestem} ={NS_{livestem\_ xfer} \mathord{\left/ {\vphantom {NS_{livestem\_ xfer}  \Delta t}} \right. \kern-\nulldelimiterspace} \Delta t}\end{split}
\end{equation}\phantomsection\label{\detokenize{tech_note/Vegetation_Phenology_Turnover/CLM50_Tech_Note_Vegetation_Phenology_Turnover:equation-20.29)}}\begin{equation}\label{equation:tech_note/Vegetation_Phenology_Turnover/CLM50_Tech_Note_Vegetation_Phenology_Turnover:20.29)}
\begin{split}NF_{deadstem\_ xfer,deadstem} ={NS_{deadstem\_ xfer} \mathord{\left/ {\vphantom {NS_{deadstem\_ xfer}  \Delta t}} \right. \kern-\nulldelimiterspace} \Delta t}\end{split}
\end{equation}\phantomsection\label{\detokenize{tech_note/Vegetation_Phenology_Turnover/CLM50_Tech_Note_Vegetation_Phenology_Turnover:equation-20.30)}}\begin{equation}\label{equation:tech_note/Vegetation_Phenology_Turnover/CLM50_Tech_Note_Vegetation_Phenology_Turnover:20.30)}
\begin{split}NF_{livecroot\_ xfer,livecroot} ={NS_{livecroot\_ xfer} \mathord{\left/ {\vphantom {NS_{livecroot\_ xfer}  \Delta t}} \right. \kern-\nulldelimiterspace} \Delta t}\end{split}
\end{equation}\phantomsection\label{\detokenize{tech_note/Vegetation_Phenology_Turnover/CLM50_Tech_Note_Vegetation_Phenology_Turnover:equation-20.31)}}\begin{equation}\label{equation:tech_note/Vegetation_Phenology_Turnover/CLM50_Tech_Note_Vegetation_Phenology_Turnover:20.31)}
\begin{split}NF_{deadcroot\_ xfer,deadcroot} ={NS_{deadcroot\_ xfer} \mathord{\left/ {\vphantom {NS_{deadcroot\_ xfer}  \Delta t}} \right. \kern-\nulldelimiterspace} \Delta t} .\end{split}
\end{equation}

\subsubsection{14.1.4 Background Litterfall}
\label{\detokenize{tech_note/Vegetation_Phenology_Turnover/CLM50_Tech_Note_Vegetation_Phenology_Turnover:background-litterfall}}
Both evergreen and stress-deciduous phenology algorithms can specify a
litterfall flux that is not associated with a specific offset period,
but which occurs instead at a slow rate over an extended period of time,
referred to as background litterfall. For evergreen types the background
litterfall is the only litterfall flux. For stress-deciduous types
either the offset period litterfall or the background litterfall
mechanism may be active, but not both at once. Given a specification of
the background litterfall rate (\({r}_{bglf}\), s$^{\text{-1}}$), litterfall carbon fluxes are calculated as
\phantomsection\label{\detokenize{tech_note/Vegetation_Phenology_Turnover/CLM50_Tech_Note_Vegetation_Phenology_Turnover:equation-20.32)}}\begin{equation}\label{equation:tech_note/Vegetation_Phenology_Turnover/CLM50_Tech_Note_Vegetation_Phenology_Turnover:20.32)}
\begin{split}CF_{leaf,litter} =r_{bglf} CS_{leaf}\end{split}
\end{equation}\phantomsection\label{\detokenize{tech_note/Vegetation_Phenology_Turnover/CLM50_Tech_Note_Vegetation_Phenology_Turnover:equation-20.33)}}\begin{equation}\label{equation:tech_note/Vegetation_Phenology_Turnover/CLM50_Tech_Note_Vegetation_Phenology_Turnover:20.33)}
\begin{split}CS_{froot,litter} =r_{bglf} CS_{froot} ,\end{split}
\end{equation}
with corresponding nitrogen litterfall and retranslocation fluxes:
\phantomsection\label{\detokenize{tech_note/Vegetation_Phenology_Turnover/CLM50_Tech_Note_Vegetation_Phenology_Turnover:equation-20.34)}}\begin{equation}\label{equation:tech_note/Vegetation_Phenology_Turnover/CLM50_Tech_Note_Vegetation_Phenology_Turnover:20.34)}
\begin{split}NF_{leaf,litter} ={CF_{leaf,litter} \mathord{\left/ {\vphantom {CF_{leaf,litter}  CN_{leaf\_ litter} }} \right. \kern-\nulldelimiterspace} CN_{leaf\_ litter} }\end{split}
\end{equation}\phantomsection\label{\detokenize{tech_note/Vegetation_Phenology_Turnover/CLM50_Tech_Note_Vegetation_Phenology_Turnover:equation-20.35)}}\begin{equation}\label{equation:tech_note/Vegetation_Phenology_Turnover/CLM50_Tech_Note_Vegetation_Phenology_Turnover:20.35)}
\begin{split}NF_{froot,litter} ={CF_{froot,litter} \mathord{\left/ {\vphantom {CF_{froot,litter}  CN_{froot} }} \right. \kern-\nulldelimiterspace} CN_{froot} }\end{split}
\end{equation}\phantomsection\label{\detokenize{tech_note/Vegetation_Phenology_Turnover/CLM50_Tech_Note_Vegetation_Phenology_Turnover:equation-20.36)}}\begin{equation}\label{equation:tech_note/Vegetation_Phenology_Turnover/CLM50_Tech_Note_Vegetation_Phenology_Turnover:20.36)}
\begin{split}NF_{leaf,retrans} =\left({CF_{leaf,litter} \mathord{\left/ {\vphantom {CF_{leaf,litter}  CN_{leaf} }} \right. \kern-\nulldelimiterspace} CN_{leaf} } \right)-NF_{leaf,litter} .\end{split}
\end{equation}

\subsubsection{14.1.5 Livewood Turnover}
\label{\detokenize{tech_note/Vegetation_Phenology_Turnover/CLM50_Tech_Note_Vegetation_Phenology_Turnover:livewood-turnover}}
The conceptualization of live wood vs. dead wood fractions for stem and
coarse root pools is intended to capture the difference in maintenance
respiration rates between these two physiologically distinct tissue
types. Unlike displayed pools for leaf and fine root, which are lost to
litterfall, live wood cells reaching the end of their lifespan are
retained as a part of the dead woody structure of stems and coarse
roots. A mechanism is therefore included in the phenology routine to
effect the transfer of live wood to dead wood pools, which also takes
into account the different nitrogen concentrations typical of these
tissue types.

A live wood turnover rate (\({r}_{lwt}\), s$^{\text{-1}}$) is
defined as
\phantomsection\label{\detokenize{tech_note/Vegetation_Phenology_Turnover/CLM50_Tech_Note_Vegetation_Phenology_Turnover:equation-20.37)}}\begin{equation}\label{equation:tech_note/Vegetation_Phenology_Turnover/CLM50_Tech_Note_Vegetation_Phenology_Turnover:20.37)}
\begin{split}r_{lwt} ={p_{lwt} \mathord{\left/ {\vphantom {p_{lwt}  \left(365\cdot 86400\right)}} \right. \kern-\nulldelimiterspace} \left(365\cdot 86400\right)}\end{split}
\end{equation}
where \({p}_{lwt} = 0.7\) is the assumed annual live wood
turnover fraction. Carbon fluxes from live to dead wood pools are:
\phantomsection\label{\detokenize{tech_note/Vegetation_Phenology_Turnover/CLM50_Tech_Note_Vegetation_Phenology_Turnover:equation-20.38)}}\begin{equation}\label{equation:tech_note/Vegetation_Phenology_Turnover/CLM50_Tech_Note_Vegetation_Phenology_Turnover:20.38)}
\begin{split}CF_{livestem,deadstem} =CS_{livestem} r_{lwt}\end{split}
\end{equation}\phantomsection\label{\detokenize{tech_note/Vegetation_Phenology_Turnover/CLM50_Tech_Note_Vegetation_Phenology_Turnover:equation-20.39)}}\begin{equation}\label{equation:tech_note/Vegetation_Phenology_Turnover/CLM50_Tech_Note_Vegetation_Phenology_Turnover:20.39)}
\begin{split}CF_{livecroot,deadcroot} =CS_{livecroot} r_{lwt} ,\end{split}
\end{equation}
and the associated nitrogen fluxes, including retranslocation of
nitrogen out of live wood during turnover, are:
\phantomsection\label{\detokenize{tech_note/Vegetation_Phenology_Turnover/CLM50_Tech_Note_Vegetation_Phenology_Turnover:equation-20.40)}}\begin{equation}\label{equation:tech_note/Vegetation_Phenology_Turnover/CLM50_Tech_Note_Vegetation_Phenology_Turnover:20.40)}
\begin{split}NF_{livestem,deadstem} ={CF_{livestem,deadstem} \mathord{\left/ {\vphantom {CF_{livestem,deadstem}  CN_{dw} }} \right. \kern-\nulldelimiterspace} CN_{dw} }\end{split}
\end{equation}\phantomsection\label{\detokenize{tech_note/Vegetation_Phenology_Turnover/CLM50_Tech_Note_Vegetation_Phenology_Turnover:equation-20.41)}}\begin{equation}\label{equation:tech_note/Vegetation_Phenology_Turnover/CLM50_Tech_Note_Vegetation_Phenology_Turnover:20.41)}
\begin{split}NF_{livestem,retrans} =\left({CF_{livestem,deadstem} \mathord{\left/ {\vphantom {CF_{livestem,deadstem}  CN_{lw} }} \right. \kern-\nulldelimiterspace} CN_{lw} } \right)-NF_{livestem,deadstem}\end{split}
\end{equation}\phantomsection\label{\detokenize{tech_note/Vegetation_Phenology_Turnover/CLM50_Tech_Note_Vegetation_Phenology_Turnover:equation-20.42)}}\begin{equation}\label{equation:tech_note/Vegetation_Phenology_Turnover/CLM50_Tech_Note_Vegetation_Phenology_Turnover:20.42)}
\begin{split}NF_{livecroot,deadcroot} ={CF_{livecroot,deadcroot} \mathord{\left/ {\vphantom {CF_{livecroot,deadcroot}  CN_{dw} }} \right. \kern-\nulldelimiterspace} CN_{dw} }\end{split}
\end{equation}\phantomsection\label{\detokenize{tech_note/Vegetation_Phenology_Turnover/CLM50_Tech_Note_Vegetation_Phenology_Turnover:equation-20.43)}}\begin{equation}\label{equation:tech_note/Vegetation_Phenology_Turnover/CLM50_Tech_Note_Vegetation_Phenology_Turnover:20.43)}
\begin{split}NF_{livecroot,retrans} =\left({CF_{livecroot,deadcroot} \mathord{\left/ {\vphantom {CF_{livecroot,deadcroot}  CN_{lw} }} \right. \kern-\nulldelimiterspace} CN_{lw} } \right)-NF_{livecroot,deadcroot} .\end{split}
\end{equation}

\subsection{Evergreen Phenology}
\label{\detokenize{tech_note/Vegetation_Phenology_Turnover/CLM50_Tech_Note_Vegetation_Phenology_Turnover:evergreen-phenology}}
The evergreen phenology algorithm is by far the simplest of the three
possible types. It is assumed for all evergreen types that all carbon
and nitrogen allocated for new growth in the current timestep goes
immediately to the displayed growth pools (i.e. f\({f}_{cur} = 1.0\)
(Chapter 13)). As such, there is never an accumulation of carbon or
nitrogen in the storage or transfer pools, and so the onset growth and
background onset growth mechanisms are never invoked for this type.
Litterfall is specified to occur only through the background litterfall
mechanism \textendash{} there are no distinct periods of litterfall for evergreen
types, but rather a continuous (slow) shedding of foliage and fine
roots. This is an obvious area for potential improvements in the model,
since it is known, at least for evergreen needleleaf trees in the
temperate and boreal zones, that there are distinct periods of higher
and lower leaf litterfall (Ferrari, 1999; Gholz et al., 1985). The rate
of background litterfall (\({r}_{bglf}\), section 14.1.4)
depends on the specified leaf longevity (\(\tau_{leaf}\), y), as
\phantomsection\label{\detokenize{tech_note/Vegetation_Phenology_Turnover/CLM50_Tech_Note_Vegetation_Phenology_Turnover:equation-20.44)}}\begin{equation}\label{equation:tech_note/Vegetation_Phenology_Turnover/CLM50_Tech_Note_Vegetation_Phenology_Turnover:20.44)}
\begin{split}r_{bglf} =\frac{1}{\tau _{leaf} \cdot 365\cdot 86400} .\end{split}
\end{equation}

\subsection{Seasonal-Deciduous Phenology}
\label{\detokenize{tech_note/Vegetation_Phenology_Turnover/CLM50_Tech_Note_Vegetation_Phenology_Turnover:seasonal-deciduous-phenology}}
The seasonal-deciduous phenology algorithm derives directly from the
treatment used in the offline model Biome-BGC v. 4.1.2, (Thornton et
al., 2002), which in turn is based on the parameterizations for leaf
onset and offset for temperate deciduous broadleaf forest from White et
al. (1997). Initiation of leaf onset is triggered when a common
degree-day summation exceeds a critical value, and leaf litterfall is
initiated when daylength is shorter than a critical value. Because of
the dependence on daylength, the seasonal deciduous phenology algorithm
is only valid for latitudes outside of the tropical zone, defined here
as \(\left|{\rm latitude}\right|>19.5{\rm {}^\circ }\). Neither the
background onset nor background litterfall mechanism is invoked for the
seasonal-deciduous phenology algorithm. The algorithm allows a maximum
of one onset period and one offset period each year.

The algorithms for initiation of onset and offset periods use the winter
and summer solstices as coordination signals. The period between winter
and summer solstice is identified as \({dayl}_{n} > {dayl}_{n-1}\),
and the period between summer and winter
solstice is identified as \({dayl}_{n} < {dayl}_{n-1}\),
where  \({dayl}_{n}\) and  \({dayl}_{n-1}\) are the day length(s) calculated for the
current and previous timesteps, respectively, using
\phantomsection\label{\detokenize{tech_note/Vegetation_Phenology_Turnover/CLM50_Tech_Note_Vegetation_Phenology_Turnover:equation-20.45)}}\begin{equation}\label{equation:tech_note/Vegetation_Phenology_Turnover/CLM50_Tech_Note_Vegetation_Phenology_Turnover:20.45)}
\begin{split}dayl=2\cdot 13750.9871\cdot acos\left(\frac{-\sin (lat)\sin (decl)}{\cos (lat)\cos (decl)} \right),\end{split}
\end{equation}
where \sphinxstyleemphasis{lat} and \sphinxstyleemphasis{decl} are the latitude and solar declination (radians),
respectively, and the factor 13750.9871 is the number of seconds per
radian of hour-angle.


\subsubsection{14.3.1 Seasonal-Deciduous Onset Trigger}
\label{\detokenize{tech_note/Vegetation_Phenology_Turnover/CLM50_Tech_Note_Vegetation_Phenology_Turnover:seasonal-deciduous-onset-trigger}}
The onset trigger for the seasonal-deciduous phenology algorithm is
based on an accumulated growing-degree-day approach (White et al.,
1997). The growing-degree-day summation (\({GDD}_{sum}\)) is
initiated ( \({GDD}_{sum} = 0\)) when the phenological state is
dormant and the model timestep crosses the winter solstice. Once these
conditions are met, \({GDD}_{sum}\) is updated on each timestep as
\phantomsection\label{\detokenize{tech_note/Vegetation_Phenology_Turnover/CLM50_Tech_Note_Vegetation_Phenology_Turnover:equation-ZEqnNum510730}}\begin{equation}\label{equation:tech_note/Vegetation_Phenology_Turnover/CLM50_Tech_Note_Vegetation_Phenology_Turnover:ZEqnNum510730}
\begin{split}GDD_{sum}^{n} =\left\{\begin{array}{l} {GDD_{sum}^{n-1} +\left(T_{s,3} -TKFRZ\right)f_{day} \qquad {\rm for\; }T_{s,3} >TKFRZ} \\ {GDD_{sum}^{n-1} \qquad \qquad \qquad {\rm for\; }T_{s,3} \le TKFRZ} \end{array}\right.\end{split}
\end{equation}
where \({T}_{s,3}\) (K) is the temperature of the third soil layer, and
\(f_{day} ={\Delta t\mathord{\left/ {\vphantom {\Delta t 86400}} \right. \kern-\nulldelimiterspace} 86400}\) .
The onset period is initiated if \(GDD_{sum} >GDD_{sum\_ crit}\) ,
where
\phantomsection\label{\detokenize{tech_note/Vegetation_Phenology_Turnover/CLM50_Tech_Note_Vegetation_Phenology_Turnover:equation-ZEqnNum598907}}\begin{equation}\label{equation:tech_note/Vegetation_Phenology_Turnover/CLM50_Tech_Note_Vegetation_Phenology_Turnover:ZEqnNum598907}
\begin{split}GDD_{sum\_ crit} =\exp \left(4.8+0.13{\kern 1pt} \left(T_{2m,ann\_ avg} -TKFRZ\right)\right)\end{split}
\end{equation}
and where \({T}_{2m,ann\_avg}\) (K) is the annual average of
the 2m air temperature, and TKFRZ is the freezing point of water (273.15 K). The following control variables are set when a new onset growth
period is initiated:
\phantomsection\label{\detokenize{tech_note/Vegetation_Phenology_Turnover/CLM50_Tech_Note_Vegetation_Phenology_Turnover:equation-20.48)}}\begin{equation}\label{equation:tech_note/Vegetation_Phenology_Turnover/CLM50_Tech_Note_Vegetation_Phenology_Turnover:20.48)}
\begin{split}GDD_{sum} =0\end{split}
\end{equation}\phantomsection\label{\detokenize{tech_note/Vegetation_Phenology_Turnover/CLM50_Tech_Note_Vegetation_Phenology_Turnover:equation-20.49)}}\begin{equation}\label{equation:tech_note/Vegetation_Phenology_Turnover/CLM50_Tech_Note_Vegetation_Phenology_Turnover:20.49)}
\begin{split}t_{onset} =86400\cdot n_{days\_ on} ,\end{split}
\end{equation}
where \({n}_{days\_on}\) is set to a constant value of 30 days.
Fluxes from storage into transfer pools occur in the timestep when a new
onset growth period is initiated. Carbon fluxes are:
\phantomsection\label{\detokenize{tech_note/Vegetation_Phenology_Turnover/CLM50_Tech_Note_Vegetation_Phenology_Turnover:equation-ZEqnNum904388}}\begin{equation}\label{equation:tech_note/Vegetation_Phenology_Turnover/CLM50_Tech_Note_Vegetation_Phenology_Turnover:ZEqnNum904388}
\begin{split}CF_{leaf\_ stor,leaf\_ xfer} ={f_{stor,xfer} CS_{leaf\_ stor} \mathord{\left/ {\vphantom {f_{stor,xfer} CS_{leaf\_ stor}  \Delta t}} \right. \kern-\nulldelimiterspace} \Delta t}\end{split}
\end{equation}\phantomsection\label{\detokenize{tech_note/Vegetation_Phenology_Turnover/CLM50_Tech_Note_Vegetation_Phenology_Turnover:equation-20.51)}}\begin{equation}\label{equation:tech_note/Vegetation_Phenology_Turnover/CLM50_Tech_Note_Vegetation_Phenology_Turnover:20.51)}
\begin{split}CF_{froot\_ stor,froot\_ xfer} ={f_{stor,xfer} CS_{froot\_ stor} \mathord{\left/ {\vphantom {f_{stor,xfer} CS_{froot\_ stor}  \Delta t}} \right. \kern-\nulldelimiterspace} \Delta t}\end{split}
\end{equation}\phantomsection\label{\detokenize{tech_note/Vegetation_Phenology_Turnover/CLM50_Tech_Note_Vegetation_Phenology_Turnover:equation-20.52)}}\begin{equation}\label{equation:tech_note/Vegetation_Phenology_Turnover/CLM50_Tech_Note_Vegetation_Phenology_Turnover:20.52)}
\begin{split}CF_{livestem\_ stor,livestem\_ xfer} ={f_{stor,xfer} CS_{livestem\_ stor} \mathord{\left/ {\vphantom {f_{stor,xfer} CS_{livestem\_ stor}  \Delta t}} \right. \kern-\nulldelimiterspace} \Delta t}\end{split}
\end{equation}\phantomsection\label{\detokenize{tech_note/Vegetation_Phenology_Turnover/CLM50_Tech_Note_Vegetation_Phenology_Turnover:equation-20.53)}}\begin{equation}\label{equation:tech_note/Vegetation_Phenology_Turnover/CLM50_Tech_Note_Vegetation_Phenology_Turnover:20.53)}
\begin{split}CF_{deadstem\_ stor,deadstem\_ xfer} ={f_{stor,xfer} CS_{deadstem\_ stor} \mathord{\left/ {\vphantom {f_{stor,xfer} CS_{deadstem\_ stor}  \Delta t}} \right. \kern-\nulldelimiterspace} \Delta t}\end{split}
\end{equation}\phantomsection\label{\detokenize{tech_note/Vegetation_Phenology_Turnover/CLM50_Tech_Note_Vegetation_Phenology_Turnover:equation-20.54)}}\begin{equation}\label{equation:tech_note/Vegetation_Phenology_Turnover/CLM50_Tech_Note_Vegetation_Phenology_Turnover:20.54)}
\begin{split}CF_{livecroot\_ stor,livecroot\_ xfer} ={f_{stor,xfer} CS_{livecroot\_ stor} \mathord{\left/ {\vphantom {f_{stor,xfer} CS_{livecroot\_ stor}  \Delta t}} \right. \kern-\nulldelimiterspace} \Delta t}\end{split}
\end{equation}\phantomsection\label{\detokenize{tech_note/Vegetation_Phenology_Turnover/CLM50_Tech_Note_Vegetation_Phenology_Turnover:equation-20.55)}}\begin{equation}\label{equation:tech_note/Vegetation_Phenology_Turnover/CLM50_Tech_Note_Vegetation_Phenology_Turnover:20.55)}
\begin{split}CF_{deadcroot\_ stor,deadcroot\_ xfer} ={f_{stor,xfer} CS_{deadcroot\_ stor} \mathord{\left/ {\vphantom {f_{stor,xfer} CS_{deadcroot\_ stor}  \Delta t}} \right. \kern-\nulldelimiterspace} \Delta t}\end{split}
\end{equation}\phantomsection\label{\detokenize{tech_note/Vegetation_Phenology_Turnover/CLM50_Tech_Note_Vegetation_Phenology_Turnover:equation-ZEqnNum195642}}\begin{equation}\label{equation:tech_note/Vegetation_Phenology_Turnover/CLM50_Tech_Note_Vegetation_Phenology_Turnover:ZEqnNum195642}
\begin{split}CF_{gresp\_ stor,gresp\_ xfer} ={f_{stor,xfer} CS_{gresp\_ stor} \mathord{\left/ {\vphantom {f_{stor,xfer} CS_{gresp\_ stor}  \Delta t}} \right. \kern-\nulldelimiterspace} \Delta t}\end{split}
\end{equation}
and the associated nitrogen fluxes are:
\phantomsection\label{\detokenize{tech_note/Vegetation_Phenology_Turnover/CLM50_Tech_Note_Vegetation_Phenology_Turnover:equation-ZEqnNum812152}}\begin{equation}\label{equation:tech_note/Vegetation_Phenology_Turnover/CLM50_Tech_Note_Vegetation_Phenology_Turnover:ZEqnNum812152}
\begin{split}NF_{leaf\_ stor,leaf\_ xfer} ={f_{stor,xfer} NS_{leaf\_ stor} \mathord{\left/ {\vphantom {f_{stor,xfer} NS_{leaf\_ stor}  \Delta t}} \right. \kern-\nulldelimiterspace} \Delta t}\end{split}
\end{equation}\phantomsection\label{\detokenize{tech_note/Vegetation_Phenology_Turnover/CLM50_Tech_Note_Vegetation_Phenology_Turnover:equation-20.58)}}\begin{equation}\label{equation:tech_note/Vegetation_Phenology_Turnover/CLM50_Tech_Note_Vegetation_Phenology_Turnover:20.58)}
\begin{split}NF_{froot\_ stor,froot\_ xfer} ={f_{stor,xfer} NS_{froot\_ stor} \mathord{\left/ {\vphantom {f_{stor,xfer} NS_{froot\_ stor}  \Delta t}} \right. \kern-\nulldelimiterspace} \Delta t}\end{split}
\end{equation}\phantomsection\label{\detokenize{tech_note/Vegetation_Phenology_Turnover/CLM50_Tech_Note_Vegetation_Phenology_Turnover:equation-20.59)}}\begin{equation}\label{equation:tech_note/Vegetation_Phenology_Turnover/CLM50_Tech_Note_Vegetation_Phenology_Turnover:20.59)}
\begin{split}NF_{livestem\_ stor,livestem\_ xfer} ={f_{stor,xfer} NS_{livestem\_ stor} \mathord{\left/ {\vphantom {f_{stor,xfer} NS_{livestem\_ stor}  \Delta t}} \right. \kern-\nulldelimiterspace} \Delta t}\end{split}
\end{equation}\phantomsection\label{\detokenize{tech_note/Vegetation_Phenology_Turnover/CLM50_Tech_Note_Vegetation_Phenology_Turnover:equation-20.60)}}\begin{equation}\label{equation:tech_note/Vegetation_Phenology_Turnover/CLM50_Tech_Note_Vegetation_Phenology_Turnover:20.60)}
\begin{split}NF_{deadstem\_ stor,deadstem\_ xfer} ={f_{stor,xfer} NS_{deadstem\_ stor} \mathord{\left/ {\vphantom {f_{stor,xfer} NS_{deadstem\_ stor}  \Delta t}} \right. \kern-\nulldelimiterspace} \Delta t}\end{split}
\end{equation}\phantomsection\label{\detokenize{tech_note/Vegetation_Phenology_Turnover/CLM50_Tech_Note_Vegetation_Phenology_Turnover:equation-20.61)}}\begin{equation}\label{equation:tech_note/Vegetation_Phenology_Turnover/CLM50_Tech_Note_Vegetation_Phenology_Turnover:20.61)}
\begin{split}NF_{livecroot\_ stor,livecroot\_ xfer} ={f_{stor,xfer} NS_{livecroot\_ stor} \mathord{\left/ {\vphantom {f_{stor,xfer} NS_{livecroot\_ stor}  \Delta t}} \right. \kern-\nulldelimiterspace} \Delta t}\end{split}
\end{equation}\phantomsection\label{\detokenize{tech_note/Vegetation_Phenology_Turnover/CLM50_Tech_Note_Vegetation_Phenology_Turnover:equation-ZEqnNum605338}}\begin{equation}\label{equation:tech_note/Vegetation_Phenology_Turnover/CLM50_Tech_Note_Vegetation_Phenology_Turnover:ZEqnNum605338}
\begin{split}NF_{deadcroot\_ stor,deadcroot\_ xfer} ={f_{stor,xfer} NS_{deadcroot\_ stor} \mathord{\left/ {\vphantom {f_{stor,xfer} NS_{deadcroot\_ stor}  \Delta t}} \right. \kern-\nulldelimiterspace} \Delta t}\end{split}
\end{equation}
where \({f}_{stor,xfer}\) is the fraction of current storage
pool moved into the transfer pool for display over the incipient onset
period. This fraction is set to 0.5, based on the observation that
seasonal deciduous trees are capable of replacing their canopies from
storage reserves in the event of a severe early-season disturbance such
as frost damage or defoliation due to insect herbivory.

If the onset criterion (\({GDD}_{sum} > {GDD}_{sum\_crit}\)) is not met before the summer solstice,
then \({GDD}_{sum}\) is set to 0.0 and the growing-degree-day
accumulation will not start again until the following winter solstice.
This mechanism prevents the initiation of very short growing seasons
late in the summer in cold climates. The onset counter is decremented on
each time step after initiation of the onset period, until it reaches
zero, signaling the end of the onset period:
\phantomsection\label{\detokenize{tech_note/Vegetation_Phenology_Turnover/CLM50_Tech_Note_Vegetation_Phenology_Turnover:equation-20.63)}}\begin{equation}\label{equation:tech_note/Vegetation_Phenology_Turnover/CLM50_Tech_Note_Vegetation_Phenology_Turnover:20.63)}
\begin{split}t_{onfset}^{n} =t_{onfset}^{n-1} -\Delta t\end{split}
\end{equation}

\subsubsection{14.3.2 Seasonal-Deciduous Offset Trigger}
\label{\detokenize{tech_note/Vegetation_Phenology_Turnover/CLM50_Tech_Note_Vegetation_Phenology_Turnover:seasonal-deciduous-offset-trigger}}
After the completion of an onset period, and once past the summer
solstice, the offset (litterfall) period is triggered when daylength is
shorter than 39300 s. The offset counter is set at the initiation of the
offset period: \(t_{offset} =86400\cdot n_{days\_ off}\) , where
\({n}_{days\_off}\) is set to a constant value of 15 days. The
offset counter is decremented on each time step after initiation of the
offset period, until it reaches zero, signaling the end of the offset
period:
\phantomsection\label{\detokenize{tech_note/Vegetation_Phenology_Turnover/CLM50_Tech_Note_Vegetation_Phenology_Turnover:equation-20.64)}}\begin{equation}\label{equation:tech_note/Vegetation_Phenology_Turnover/CLM50_Tech_Note_Vegetation_Phenology_Turnover:20.64)}
\begin{split}t_{offset}^{n} =t_{offset}^{n-1} -\Delta t\end{split}
\end{equation}

\subsection{Stress-Deciduous Phenology}
\label{\detokenize{tech_note/Vegetation_Phenology_Turnover/CLM50_Tech_Note_Vegetation_Phenology_Turnover:stress-deciduous-phenology}}
The stress-deciduous phenology algorithm was developed specifically for
the CLM based in part on the grass phenology model proposed by White et
al. (1997). The algorithm handles phenology for vegetation types such as
grasses and tropical drought-deciduous trees that respond to both cold
and drought-stress signals, and that can have multiple growing seasons
per year. The algorithm also allows for the possibility that leaves
might persist year-round in the absence of a suitable stress trigger. In
that case the phenology switches to an evergreen habit, maintaining a
marginally-deciduous leaf longevity (one year) until the occurrence of
the next stress trigger.


\subsubsection{14.4.1 Stress-Deciduous Onset Triggers}
\label{\detokenize{tech_note/Vegetation_Phenology_Turnover/CLM50_Tech_Note_Vegetation_Phenology_Turnover:stress-deciduous-onset-triggers}}
In climates that are warm year-round, onset triggering depends on soil
water availability. At the beginning of a dormant period (end of
previous offset period), an accumulated soil water index
(\({SWI}_{sum}\), d) is initialized (\({SWI}_{sum} = 0\)), with subsequent accumulation calculated as:
\phantomsection\label{\detokenize{tech_note/Vegetation_Phenology_Turnover/CLM50_Tech_Note_Vegetation_Phenology_Turnover:equation-ZEqnNum503826}}\begin{equation}\label{equation:tech_note/Vegetation_Phenology_Turnover/CLM50_Tech_Note_Vegetation_Phenology_Turnover:ZEqnNum503826}
\begin{split}SWI_{sum}^{n} =\left\{\begin{array}{l} {SWI_{sum}^{n-1} +f_{day} \qquad {\rm for\; }\Psi _{s,3} \ge \Psi _{onset} } \\ {SWI_{sum}^{n-1} \qquad \qquad {\rm for\; }\Psi _{s,3} <\Psi _{onset} } \end{array}\right.\end{split}
\end{equation}
where \(\Psi\)$_{\text{s,3}}$ is the soil water potential (MPa)
in the third soil layer and \({\Psi}_{onset} = -0.6 MPa\)
is the onset soil water potential threshold. Onset triggering is
possible once \({SWI}_{sum} > 15\). To avoid spurious onset triggering due to
soil moisture in the third soil layer exceeding the threshold due only to
soil water suction of water from deeper in the soil column, an additional precipitation trigger is included which requires
at least 20 mm of rain over the previous 10 days {\hyperref[\detokenize{tech_note/References/CLM50_Tech_Note_References:dahlinetal2015}]{\sphinxcrossref{\DUrole{std,std-ref}{(Dahlin et al., 2015)}}}}.  If the cold climate
growing degree-day accumulator is not active at the time when the soil moisture and precipitation
thresholds are reached (see below), and if the daylength is greater than 6
hours, then onset is triggered. Except as noted below,
\({SWI}_{sum}\) continues to accumulate according to Eq. \eqref{equation:tech_note/Vegetation_Phenology_Turnover/CLM50_Tech_Note_Vegetation_Phenology_Turnover:ZEqnNum503826} during
the dormant period if the daylength criterion prevents onset triggering,
and onset is then triggered at the timestep when daylength exceeds 6
hours.

In climates with a cold season, onset triggering depends on both
accumulated soil temperature summation and adequate soil moisture. At
the beginning of a dormant period a freezing day accumulator
(\({FD}_{sum}\), d) is initialized (\({FD}_{sum} = 0\)),
with subsequent accumulation calculated as:
\phantomsection\label{\detokenize{tech_note/Vegetation_Phenology_Turnover/CLM50_Tech_Note_Vegetation_Phenology_Turnover:equation-20.66)}}\begin{equation}\label{equation:tech_note/Vegetation_Phenology_Turnover/CLM50_Tech_Note_Vegetation_Phenology_Turnover:20.66)}
\begin{split}FD_{sum}^{n} =\left\{\begin{array}{l} {FD_{sum}^{n-1} +f_{day} \qquad {\rm for\; }T_{s,3} >TKFRZ} \\ {FD_{sum}^{n-1} \qquad \qquad {\rm for\; }T_{s,3} \le TKFRZ} \end{array}\right. .\end{split}
\end{equation}
If \({FD}_{sum} > 15\) during the dormant period, then a
cold-climate onset triggering criterion is introduced, following exactly
the growing degree-day summation (\({GDD}_{sum}\)) logic of Eqs. \eqref{equation:tech_note/Vegetation_Phenology_Turnover/CLM50_Tech_Note_Vegetation_Phenology_Turnover:ZEqnNum510730}
and \eqref{equation:tech_note/Vegetation_Phenology_Turnover/CLM50_Tech_Note_Vegetation_Phenology_Turnover:ZEqnNum598907}. At that time \({SWI}_{sum}\) is reset
(\({SWI}_{sum} = 0\)). Onset triggering under these conditions
depends on meeting all three of the following criteria:
\({SWI}_{sum} > 15\), \({GDD}_{sum} > {GDD}_{sum\_crit}\), and daylength greater than 6 hrs.

The following control variables are set when a new onset growth period
is initiated: \({SWI}_{sum} = 0\), \({FD}_{sum} = 0\), \({GDD}_{sum} = 0\), \({n}_{days\_active} = 0\), and
\(t_{onset} = 86400\cdot n_{days\_ on}\) , where \({n}_{days\_on}\) is set to a constant value of 30 days. Fluxes
from storage into transfer pools occur in the timestep when a new onset growth period is initiated, and are handled identically to Eqs. \eqref{equation:tech_note/Vegetation_Phenology_Turnover/CLM50_Tech_Note_Vegetation_Phenology_Turnover:ZEqnNum904388} -\eqref{equation:tech_note/Vegetation_Phenology_Turnover/CLM50_Tech_Note_Vegetation_Phenology_Turnover:ZEqnNum195642} for
carbon fluxes, and to Eqs. \eqref{equation:tech_note/Vegetation_Phenology_Turnover/CLM50_Tech_Note_Vegetation_Phenology_Turnover:ZEqnNum812152} - \eqref{equation:tech_note/Vegetation_Phenology_Turnover/CLM50_Tech_Note_Vegetation_Phenology_Turnover:ZEqnNum605338} for nitrogen fluxes. The onset counter is decremented on each time step after initiation of the onset period,
until it reaches zero, signaling the end of the onset period:
\phantomsection\label{\detokenize{tech_note/Vegetation_Phenology_Turnover/CLM50_Tech_Note_Vegetation_Phenology_Turnover:equation-20.67)}}\begin{equation}\label{equation:tech_note/Vegetation_Phenology_Turnover/CLM50_Tech_Note_Vegetation_Phenology_Turnover:20.67)}
\begin{split}t_{onfset}^{n} =t_{onfset}^{n-1} -\Delta t\end{split}
\end{equation}

\subsubsection{14.4.2 Stress-Deciduous Offset Triggers}
\label{\detokenize{tech_note/Vegetation_Phenology_Turnover/CLM50_Tech_Note_Vegetation_Phenology_Turnover:stress-deciduous-offset-triggers}}
Any one of the following three conditions is sufficient to initiate an
offset period for the stress-deciduous phenology algorithm: sustained
period of dry soil, sustained period of cold temperature, or daylength
shorter than 6 hours. Offset triggering due to dry soil or cold
temperature conditions is only allowed once the most recent onset period
is complete. Dry soil condition is evaluated with an offset soil water
index accumulator (\({OSWI}_{sum}\), d). To test for a sustained
period of dry soils, this control variable can increase or decrease, as
follows:
\phantomsection\label{\detokenize{tech_note/Vegetation_Phenology_Turnover/CLM50_Tech_Note_Vegetation_Phenology_Turnover:equation-20.68)}}\begin{equation}\label{equation:tech_note/Vegetation_Phenology_Turnover/CLM50_Tech_Note_Vegetation_Phenology_Turnover:20.68)}
\begin{split}OSWI_{sum}^{n} =\left\{\begin{array}{l} {OSWI_{sum}^{n-1} +f_{day} \qquad \qquad \qquad {\rm for\; }\Psi _{s,3} \le \Psi _{offset} } \\ {{\rm max}\left(OSWI_{sum}^{n-1} -f_{day} ,0\right)\qquad {\rm for\; }\Psi _{s,3} >\Psi _{onset} } \end{array}\right.\end{split}
\end{equation}
where \({\Psi}_{offset} = -2 MPa\) is the offset soil
water potential threshold. An offset period is triggered if the previous
onset period is complete and \({OSWI}_{sum}\)
\(\mathrm{\ge}\) \({OSWI}_{sum\_crit}\), where \({OSWI}_{sum\_crit} = 15\).

The cold temperature trigger is calculated with an offset freezing day
accumulator (\({OFD}_{sum}\), d). To test for a sustained period
of cold temperature, this variable can increase or decrease, as follows:
\phantomsection\label{\detokenize{tech_note/Vegetation_Phenology_Turnover/CLM50_Tech_Note_Vegetation_Phenology_Turnover:equation-20.69)}}\begin{equation}\label{equation:tech_note/Vegetation_Phenology_Turnover/CLM50_Tech_Note_Vegetation_Phenology_Turnover:20.69)}
\begin{split}OFD_{sum}^{n} =\left\{\begin{array}{l} {OFD_{sum}^{n-1} +f_{day} \qquad \qquad \qquad {\rm for\; }T_{s,3} \le TKFRZ} \\ {{\rm max}\left(OFD_{sum}^{n-1} -f_{day} ,0\right)\qquad \qquad {\rm for\; }T_{s,3} >TKFRZ} \end{array}\right.\end{split}
\end{equation}
An offset period is triggered if the previous onset period is complete
and  \({OFD}_{sum} > {OFD}_{sum\_crit}\),
where \({OFD}_{sum\_crit} = 15\).

The offset counter is set at the initiation of the offset period:
\(t_{offset} =86400\cdot n_{days\_ off}\) , where
\({n}_{days\_off}\) is set to a constant value of 15 days. The
offset counter is decremented on each time step after initiation of the
offset period, until it reaches zero, signaling the end of the offset
period:
\phantomsection\label{\detokenize{tech_note/Vegetation_Phenology_Turnover/CLM50_Tech_Note_Vegetation_Phenology_Turnover:equation-20.70)}}\begin{equation}\label{equation:tech_note/Vegetation_Phenology_Turnover/CLM50_Tech_Note_Vegetation_Phenology_Turnover:20.70)}
\begin{split}t_{offset}^{n} =t_{offset}^{n-1} -\Delta t\end{split}
\end{equation}

\subsubsection{14.4.3 Stress-Deciduous: Long Growing Season}
\label{\detokenize{tech_note/Vegetation_Phenology_Turnover/CLM50_Tech_Note_Vegetation_Phenology_Turnover:stress-deciduous-long-growing-season}}
Under conditions when the stress-deciduous conditions triggering offset
are not met for one year or longer, the stress-deciduous algorithm
shifts toward the evergreen behavior. This can happen in cases where a
stress-deciduous vegetation type is assigned in a climate where suitably
strong stresses occur less frequently than once per year. This condition
is evaluated by tracking the number of days since the beginning of the
most recent onset period (\({n}_{days\_active}\), d). At the end
of an offset period \({n}_{days\_active}\) is reset to 0. A long
growing season control variable (\sphinxstyleemphasis{LGS}, range 0 to 1) is calculated as:
\phantomsection\label{\detokenize{tech_note/Vegetation_Phenology_Turnover/CLM50_Tech_Note_Vegetation_Phenology_Turnover:equation-20.71)}}\begin{equation}\label{equation:tech_note/Vegetation_Phenology_Turnover/CLM50_Tech_Note_Vegetation_Phenology_Turnover:20.71)}
\begin{split}LGS=\left\{\begin{array}{l} {0\qquad \qquad \qquad {\rm for\; }n_{days\_ active} <365} \\ {\left({n_{days\_ active} \mathord{\left/ {\vphantom {n_{days\_ active}  365}} \right. \kern-\nulldelimiterspace} 365} \right)-1\qquad {\rm for\; }365\le n_{days\_ active} <730} \\ {1\qquad \qquad \qquad {\rm for\; }n_{days\_ active} \ge 730} \end{array}\right. .\end{split}
\end{equation}
The rate coefficient for background litterfall (\({r}_{bglf}\), s$^{\text{-1}}$) is calculated as a function of \sphinxstyleemphasis{LGS}:
\phantomsection\label{\detokenize{tech_note/Vegetation_Phenology_Turnover/CLM50_Tech_Note_Vegetation_Phenology_Turnover:equation-20.72)}}\begin{equation}\label{equation:tech_note/Vegetation_Phenology_Turnover/CLM50_Tech_Note_Vegetation_Phenology_Turnover:20.72)}
\begin{split}r_{bglf} =\frac{LGS}{\tau _{leaf} \cdot 365\cdot 86400}\end{split}
\end{equation}
where \({\tau}_{leaf}\) is the leaf longevity. The result is a shift to continuous litterfall as
\({n}_{days\_active}\) increases from 365 to 730. When a new offset period is triggered \({r}_{bglf}\) is set to 0.

The rate coefficient for background onset growth from the transfer pools ( \({r}_{bgtr}\), s$^{\text{-1}}$) also depends on \sphinxstyleemphasis{LGS}, as:
\phantomsection\label{\detokenize{tech_note/Vegetation_Phenology_Turnover/CLM50_Tech_Note_Vegetation_Phenology_Turnover:equation-20.73)}}\begin{equation}\label{equation:tech_note/Vegetation_Phenology_Turnover/CLM50_Tech_Note_Vegetation_Phenology_Turnover:20.73)}
\begin{split}r_{bgtr} =\frac{LGS}{365\cdot 86400} .\end{split}
\end{equation}
On each timestep with \({r}_{bgtr}\) \(\neq\) 0, carbon fluxes from storage to transfer pools are calculated as:
\phantomsection\label{\detokenize{tech_note/Vegetation_Phenology_Turnover/CLM50_Tech_Note_Vegetation_Phenology_Turnover:equation-20.74)}}\begin{equation}\label{equation:tech_note/Vegetation_Phenology_Turnover/CLM50_Tech_Note_Vegetation_Phenology_Turnover:20.74)}
\begin{split}CF_{leaf\_ stor,leaf\_ xfer} =CS_{leaf\_ stor} r_{bgtr}\end{split}
\end{equation}\phantomsection\label{\detokenize{tech_note/Vegetation_Phenology_Turnover/CLM50_Tech_Note_Vegetation_Phenology_Turnover:equation-20.75)}}\begin{equation}\label{equation:tech_note/Vegetation_Phenology_Turnover/CLM50_Tech_Note_Vegetation_Phenology_Turnover:20.75)}
\begin{split}CF_{froot\_ stor,froot\_ xfer} =CS_{froot\_ stor} r_{bgtr}\end{split}
\end{equation}\phantomsection\label{\detokenize{tech_note/Vegetation_Phenology_Turnover/CLM50_Tech_Note_Vegetation_Phenology_Turnover:equation-20.76)}}\begin{equation}\label{equation:tech_note/Vegetation_Phenology_Turnover/CLM50_Tech_Note_Vegetation_Phenology_Turnover:20.76)}
\begin{split}CF_{livestem\_ stor,livestem\_ xfer} =CS_{livestem\_ stor} r_{bgtr}\end{split}
\end{equation}\phantomsection\label{\detokenize{tech_note/Vegetation_Phenology_Turnover/CLM50_Tech_Note_Vegetation_Phenology_Turnover:equation-20.77)}}\begin{equation}\label{equation:tech_note/Vegetation_Phenology_Turnover/CLM50_Tech_Note_Vegetation_Phenology_Turnover:20.77)}
\begin{split}CF_{deadstem\_ stor,deadstem\_ xfer} =CS_{deadstem\_ stor} r_{bgtr}\end{split}
\end{equation}\phantomsection\label{\detokenize{tech_note/Vegetation_Phenology_Turnover/CLM50_Tech_Note_Vegetation_Phenology_Turnover:equation-20.78)}}\begin{equation}\label{equation:tech_note/Vegetation_Phenology_Turnover/CLM50_Tech_Note_Vegetation_Phenology_Turnover:20.78)}
\begin{split}CF_{livecroot\_ stor,livecroot\_ xfer} =CS_{livecroot\_ stor} r_{bgtr}\end{split}
\end{equation}\phantomsection\label{\detokenize{tech_note/Vegetation_Phenology_Turnover/CLM50_Tech_Note_Vegetation_Phenology_Turnover:equation-20.79)}}\begin{equation}\label{equation:tech_note/Vegetation_Phenology_Turnover/CLM50_Tech_Note_Vegetation_Phenology_Turnover:20.79)}
\begin{split}CF_{deadcroot\_ stor,deadcroot\_ xfer} =CS_{deadcroot\_ stor} r_{bgtr} ,\end{split}
\end{equation}
with corresponding nitrogen fluxes:
\phantomsection\label{\detokenize{tech_note/Vegetation_Phenology_Turnover/CLM50_Tech_Note_Vegetation_Phenology_Turnover:equation-20.80)}}\begin{equation}\label{equation:tech_note/Vegetation_Phenology_Turnover/CLM50_Tech_Note_Vegetation_Phenology_Turnover:20.80)}
\begin{split}NF_{leaf\_ stor,leaf\_ xfer} =NS_{leaf\_ stor} r_{bgtr}\end{split}
\end{equation}\phantomsection\label{\detokenize{tech_note/Vegetation_Phenology_Turnover/CLM50_Tech_Note_Vegetation_Phenology_Turnover:equation-20.81)}}\begin{equation}\label{equation:tech_note/Vegetation_Phenology_Turnover/CLM50_Tech_Note_Vegetation_Phenology_Turnover:20.81)}
\begin{split}NF_{froot\_ stor,froot\_ xfer} =NS_{froot\_ stor} r_{bgtr}\end{split}
\end{equation}\phantomsection\label{\detokenize{tech_note/Vegetation_Phenology_Turnover/CLM50_Tech_Note_Vegetation_Phenology_Turnover:equation-20.82)}}\begin{equation}\label{equation:tech_note/Vegetation_Phenology_Turnover/CLM50_Tech_Note_Vegetation_Phenology_Turnover:20.82)}
\begin{split}NF_{livestem\_ stor,livestem\_ xfer} =NS_{livestem\_ stor} r_{bgtr}\end{split}
\end{equation}\phantomsection\label{\detokenize{tech_note/Vegetation_Phenology_Turnover/CLM50_Tech_Note_Vegetation_Phenology_Turnover:equation-20.83)}}\begin{equation}\label{equation:tech_note/Vegetation_Phenology_Turnover/CLM50_Tech_Note_Vegetation_Phenology_Turnover:20.83)}
\begin{split}NF_{deadstem\_ stor,deadstem\_ xfer} =NS_{deadstem\_ stor} r_{bgtr}\end{split}
\end{equation}\phantomsection\label{\detokenize{tech_note/Vegetation_Phenology_Turnover/CLM50_Tech_Note_Vegetation_Phenology_Turnover:equation-20.84)}}\begin{equation}\label{equation:tech_note/Vegetation_Phenology_Turnover/CLM50_Tech_Note_Vegetation_Phenology_Turnover:20.84)}
\begin{split}NF_{livecroot\_ stor,livecroot\_ xfer} =NS_{livecroot\_ stor} r_{bgtr}\end{split}
\end{equation}\phantomsection\label{\detokenize{tech_note/Vegetation_Phenology_Turnover/CLM50_Tech_Note_Vegetation_Phenology_Turnover:equation-20.85)}}\begin{equation}\label{equation:tech_note/Vegetation_Phenology_Turnover/CLM50_Tech_Note_Vegetation_Phenology_Turnover:20.85)}
\begin{split}NF_{deadcroot\_ stor,deadcroot\_ xfer} =NS_{deadcroot\_ stor} r_{bgtr} .\end{split}
\end{equation}
The result, in conjunction with the treatment of background onset
growth, is a shift to continuous transfer from storage to display pools
at a rate that would result in complete turnover of the storage pools in
one year at steady state, once \sphinxstyleemphasis{LGS} reaches 1 (i.e. after two years
without stress-deciduous offset conditions). If and when conditions
cause stress-deciduous triggering again, \({r}_{bgtr}\) is rest
to 0.


\subsection{Litterfall Fluxes Merged to the Column Level}
\label{\detokenize{tech_note/Vegetation_Phenology_Turnover/CLM50_Tech_Note_Vegetation_Phenology_Turnover:litterfall-fluxes-merged-to-the-column-level}}
CLM uses three litter pools, defined on the basis of commonly measured
chemical fractionation of fresh litter into labile (LIT1 = hot water and
alcohol soluble fraction), cellulose/hemicellulose (LIT2 = acid soluble
fraction) and remaining material, referred to here for convenience as
lignin (LIT3 = acid insoluble fraction) (Aber et al., 1990; Taylor et
al., 1989). While multiple plant functional types can coexist on a
single CLM soil column, each soil column includes a single instance of
the litter pools. Fluxes entering the litter pools due to litterfall are
calculated using a weighted average of the fluxes originating at the PFT
level. Carbon fluxes are calculated as:
\phantomsection\label{\detokenize{tech_note/Vegetation_Phenology_Turnover/CLM50_Tech_Note_Vegetation_Phenology_Turnover:equation-20.86)}}\begin{equation}\label{equation:tech_note/Vegetation_Phenology_Turnover/CLM50_Tech_Note_Vegetation_Phenology_Turnover:20.86)}
\begin{split}CF_{leaf,lit1} =\sum _{p=0}^{npfts}CF_{leaf,litter} f_{lab\_ leaf,p} wcol_{p}\end{split}
\end{equation}\phantomsection\label{\detokenize{tech_note/Vegetation_Phenology_Turnover/CLM50_Tech_Note_Vegetation_Phenology_Turnover:equation-20.87)}}\begin{equation}\label{equation:tech_note/Vegetation_Phenology_Turnover/CLM50_Tech_Note_Vegetation_Phenology_Turnover:20.87)}
\begin{split}CF_{leaf,lit2} =\sum _{p=0}^{npfts}CF_{leaf,litter} f_{cel\_ leaf,p} wcol_{p}\end{split}
\end{equation}\phantomsection\label{\detokenize{tech_note/Vegetation_Phenology_Turnover/CLM50_Tech_Note_Vegetation_Phenology_Turnover:equation-20.88)}}\begin{equation}\label{equation:tech_note/Vegetation_Phenology_Turnover/CLM50_Tech_Note_Vegetation_Phenology_Turnover:20.88)}
\begin{split}CF_{leaf,lit3} =\sum _{p=0}^{npfts}CF_{leaf,litter} f_{lig\_ leaf,p} wcol_{p}\end{split}
\end{equation}\phantomsection\label{\detokenize{tech_note/Vegetation_Phenology_Turnover/CLM50_Tech_Note_Vegetation_Phenology_Turnover:equation-20.89)}}\begin{equation}\label{equation:tech_note/Vegetation_Phenology_Turnover/CLM50_Tech_Note_Vegetation_Phenology_Turnover:20.89)}
\begin{split}CF_{froot,lit1} =\sum _{p=0}^{npfts}CF_{froot,litter} f_{lab\_ froot,p} wcol_{p}\end{split}
\end{equation}\phantomsection\label{\detokenize{tech_note/Vegetation_Phenology_Turnover/CLM50_Tech_Note_Vegetation_Phenology_Turnover:equation-20.90)}}\begin{equation}\label{equation:tech_note/Vegetation_Phenology_Turnover/CLM50_Tech_Note_Vegetation_Phenology_Turnover:20.90)}
\begin{split}CF_{froot,lit2} =\sum _{p=0}^{npfts}CF_{froot,litter} f_{cel\_ froot,p} wcol_{p}\end{split}
\end{equation}\phantomsection\label{\detokenize{tech_note/Vegetation_Phenology_Turnover/CLM50_Tech_Note_Vegetation_Phenology_Turnover:equation-20.91)}}\begin{equation}\label{equation:tech_note/Vegetation_Phenology_Turnover/CLM50_Tech_Note_Vegetation_Phenology_Turnover:20.91)}
\begin{split}CF_{froot,lit3} =\sum _{p=0}^{npfts}CF_{froot,litter} f_{lig\_ froot,p} wcol_{p}  ,\end{split}
\end{equation}
where \({f}_{lab\_leaf,p}\), \({f}_{cel\_leaf,p}\), and
\({f}_{lig\_leaf,p}\) are the labile, cellulose/hemicellulose,
and lignin fractions of leaf litter for PFT \sphinxstyleemphasis{p},
\({f}_{lab\_froot,p}\), \({f}_{cel\_froot,p}\), and
\({f}_{lig\_froot,p}\) are the labile, cellulose/hemicellulose,
and lignin fractions of fine root litter for PFT \sphinxstyleemphasis{p},
\({wtcol}_{p}\) is the weight relative to the column for PFT
\sphinxstyleemphasis{p}, and \sphinxstyleemphasis{p} is an index through the plant functional types occurring on
a column. Nitrogen fluxes to the litter pools are assumed to follow the
C:N of the senescent tissue, and so are distributed using the same
fractions used for carbon fluxes:
\phantomsection\label{\detokenize{tech_note/Vegetation_Phenology_Turnover/CLM50_Tech_Note_Vegetation_Phenology_Turnover:equation-20.92)}}\begin{equation}\label{equation:tech_note/Vegetation_Phenology_Turnover/CLM50_Tech_Note_Vegetation_Phenology_Turnover:20.92)}
\begin{split}NF_{leaf,lit1} =\sum _{p=0}^{npfts}NF_{leaf,litter} f_{lab\_ leaf,p} wcol_{p}\end{split}
\end{equation}\phantomsection\label{\detokenize{tech_note/Vegetation_Phenology_Turnover/CLM50_Tech_Note_Vegetation_Phenology_Turnover:equation-20.93)}}\begin{equation}\label{equation:tech_note/Vegetation_Phenology_Turnover/CLM50_Tech_Note_Vegetation_Phenology_Turnover:20.93)}
\begin{split}NF_{leaf,lit2} =\sum _{p=0}^{npfts}NF_{leaf,litter} f_{cel\_ leaf,p} wcol_{p}\end{split}
\end{equation}\phantomsection\label{\detokenize{tech_note/Vegetation_Phenology_Turnover/CLM50_Tech_Note_Vegetation_Phenology_Turnover:equation-20.94)}}\begin{equation}\label{equation:tech_note/Vegetation_Phenology_Turnover/CLM50_Tech_Note_Vegetation_Phenology_Turnover:20.94)}
\begin{split}NF_{leaf,lit3} =\sum _{p=0}^{npfts}NF_{leaf,litter} f_{lig\_ leaf,p} wcol_{p}\end{split}
\end{equation}\phantomsection\label{\detokenize{tech_note/Vegetation_Phenology_Turnover/CLM50_Tech_Note_Vegetation_Phenology_Turnover:equation-20.95)}}\begin{equation}\label{equation:tech_note/Vegetation_Phenology_Turnover/CLM50_Tech_Note_Vegetation_Phenology_Turnover:20.95)}
\begin{split}NF_{froot,lit1} =\sum _{p=0}^{npfts}NF_{froot,litter} f_{lab\_ froot,p} wcol_{p}\end{split}
\end{equation}\phantomsection\label{\detokenize{tech_note/Vegetation_Phenology_Turnover/CLM50_Tech_Note_Vegetation_Phenology_Turnover:equation-20.96)}}\begin{equation}\label{equation:tech_note/Vegetation_Phenology_Turnover/CLM50_Tech_Note_Vegetation_Phenology_Turnover:20.96)}
\begin{split}NF_{froot,lit2} =\sum _{p=0}^{npfts}NF_{froot,litter} f_{cel\_ froot,p} wcol_{p}\end{split}
\end{equation}\phantomsection\label{\detokenize{tech_note/Vegetation_Phenology_Turnover/CLM50_Tech_Note_Vegetation_Phenology_Turnover:equation-20.97)}}\begin{equation}\label{equation:tech_note/Vegetation_Phenology_Turnover/CLM50_Tech_Note_Vegetation_Phenology_Turnover:20.97)}
\begin{split}NF_{froot,lit3} =\sum _{p=0}^{npfts}NF_{froot,litter} f_{lig\_ froot,p} wcol_{p}  .\end{split}
\end{equation}

\section{Decomposition}
\label{\detokenize{tech_note/Decomposition/CLM50_Tech_Note_Decomposition:rst-decomposition}}\label{\detokenize{tech_note/Decomposition/CLM50_Tech_Note_Decomposition:decomposition}}\label{\detokenize{tech_note/Decomposition/CLM50_Tech_Note_Decomposition::doc}}
Decomposition of fresh litter material into progressively more
recalcitrant forms of soil organic matter is represented in CLM is
defined as a cascade of \({k}_{tras}\) transformations between
\({m}_{pool}\) decomposing coarse woody debris (CWD), litter,
and soil organic matter (SOM) pools, each defined at
\({n}_{lev}\) vertical levels. CLM allows the user to define, at
compile time, between 2 contrasting hypotheses of decomposition as
embodied by two separate decomposition submodels: the CLM-CN pool
structure used in CLM4.0, or a second pool structure, characterized by
slower decomposition rates, based on the fCentury model (Parton et al.
1988). In addition, the user can choose, at compile time, whether to
allow \({n}_{lev}\) to equal 1, as in CLM4.0, or to equal the
number of soil levels used for the soil hydrological and thermal
calculations (see Section  \hyperref[\detokenize{tech_note/Ecosystem/CLM50_Tech_Note_Ecosystem:soil-layers}]{\ref{\detokenize{tech_note/Ecosystem/CLM50_Tech_Note_Ecosystem:soil-layers}}} for soil layering).

\begin{figure}[htbp]
\centering
\capstart

\noindent\sphinxincludegraphics{{CLM4_vertsoil_soilstruct_drawing}.png}
\caption{Schematic of decomposition model in CLM.}\label{\detokenize{tech_note/Decomposition/CLM50_Tech_Note_Decomposition:figure-schematic-of-decomposition-model-in-clm}}\label{\detokenize{tech_note/Decomposition/CLM50_Tech_Note_Decomposition:id1}}\end{figure}

Model is structured to allow different representations of the soil C and
N decomposition cascade, as well as a vertically-explicit treatment of
soil biogeochemistry.

For the single-level model structure, the fundamental equation for
carbon balance of the decomposing pools is:
\phantomsection\label{\detokenize{tech_note/Decomposition/CLM50_Tech_Note_Decomposition:equation-21.1)}}\begin{equation}\label{equation:tech_note/Decomposition/CLM50_Tech_Note_Decomposition:21.1)}
\begin{split}\frac{\partial C_{i} }{\partial t} =R_{i} +\sum _{j\ne i}\left(i-r_{j} \right)T_{ji} k_{j} C_{j} -k_{i} C_{i}\end{split}
\end{equation}
where \({C}_{i}\) is the carbon content of pool \sphinxstyleemphasis{i},
\({R}_{i}\) are the carbon inputs from plant tissues directly to
pool \sphinxstyleemphasis{i} (only non-zero for CWD and litter pools), \({k}_{i}\)
is the decay constant of pool \sphinxstyleemphasis{i}; \({T}_{ji}\) is the fraction
of carbon directed from pool \sphinxstyleemphasis{j} to pool \sphinxstyleemphasis{i} with fraction
\({r}_{j}\) lost as a respiration flux along the way.

Adding the vertical dimension to the decomposing pools changes the
balance equation to the following:
\phantomsection\label{\detokenize{tech_note/Decomposition/CLM50_Tech_Note_Decomposition:equation-21.2)}}\begin{equation}\label{equation:tech_note/Decomposition/CLM50_Tech_Note_Decomposition:21.2)}
\begin{split}\begin{array}{l} {\frac{\partial C_{i} (z)}{\partial t} =R_{i} (z)+\sum _{i\ne j}\left(1-r_{j} \right)T_{ji} k_{j} (z)C_{j} (z) -k_{i} (z)C_{i} (z)} \\ {+\frac{\partial }{\partial z} \left(D(z)\frac{\partial C_{i} }{\partial z} \right)+\frac{\partial }{\partial z} \left(A(z)C_{i} \right)} \end{array}\end{split}
\end{equation}
where \({C}_{i}\)(z) is now defined at each model level, and
in volumetric (gC m$^{\text{-3}}$) rather than areal (gC m$^{\text{-2}}$) units, along with \({R}_{i}\)(z) and
\({k}_{j}\)(z). In addition, vertical transport is handled by
the last two terms, for diffusive and advective transport. In the base
model, advective transport is set to zero, leaving only a diffusive flux
with diffusivity \sphinxstyleemphasis{D(z)} defined for all decomposing carbon and nitrogen
pools. Further discussion of the vertical distribution of carbon inputs
\({R}_{i}\)(z), vertical turnover times
\({k}_{j}\)(z), and vertical transport \sphinxstyleemphasis{D(z)} is below.
Discussion of the vertical model and analysis of both decomposition
structures is in {\hyperref[\detokenize{tech_note/References/CLM50_Tech_Note_References:kovenetal2013}]{\sphinxcrossref{\DUrole{std,std-ref}{Koven et al. (2013)}}}}.

\begin{figure}[htbp]
\centering
\capstart

\noindent\sphinxincludegraphics{{soil_C_pools_CN_century}.png}
\caption{Pool structure, transitions, respired fractions (numbers at
end of arrows), and turnover times (numbers in boxes) for the 2
alternate soil decomposition models included in CLM.}\label{\detokenize{tech_note/Decomposition/CLM50_Tech_Note_Decomposition:figure-pool-structure}}\label{\detokenize{tech_note/Decomposition/CLM50_Tech_Note_Decomposition:id2}}\end{figure}


\subsection{CLM-CN Pool Structure, Rate Constants and Parameters}
\label{\detokenize{tech_note/Decomposition/CLM50_Tech_Note_Decomposition:clm-cn-pool-structure-rate-constants-and-parameters}}
The CLM-CN structure in CLM45 uses three state variables for fresh
litter and four state variables for soil organic matter (SOM). The
masses of carbon and nitrogen in the live microbial community are not
modeled explicitly, but the activity of these organisms is represented
by decomposition fluxes transferring mass between the litter and SOM
pools, and heterotrophic respiration losses associated with these
transformations. The litter and SOM pools in CLM-CN are arranged as a
converging cascade (Figure 15.2), derived directly from the
implementation in Biome-BGC v4.1.2 (Thornton et al. 2002; Thornton and
Rosenbloom, 2005).

Model parameters are estimated based on a synthesis of microcosm
decomposition studies using radio-labeled substrates (Degens and
Sparling, 1996; Ladd et al. 1992; Martin et al. 1980; Mary et al. 1993;
Saggar et al. 1994; Sørensen, 1981; van Veen et al. 1984). Multiple
exponential models are fitted to data from the microcosm studies to
estimate exponential decay rates and respiration fractions (Thornton,
1998). The microcosm experiments used for parameterization were all
conducted at constant temperature and under moist conditions with
relatively high mineral nitrogen concentrations, and so the resulting
rate constants are assumed not limited by the availability of water or
mineral nitrogen. \hyperref[\detokenize{tech_note/Decomposition/CLM50_Tech_Note_Decomposition:table-decomposition-rate-constants}]{Table \ref{\detokenize{tech_note/Decomposition/CLM50_Tech_Note_Decomposition:table-decomposition-rate-constants}}} lists the base decomposition rates for each
litter and SOM pool, as well as a base rate for physical fragmentation
for the coarse woody debris pool (CWD).


\begin{savenotes}\sphinxattablestart
\centering
\sphinxcapstartof{table}
\sphinxcaption{Decomposition rate constants for litter and SOM pools, C:N ratios, and acceleration parameters (see section 15.8 for explanation) for the CLM-CN decomposition pool structure.}\label{\detokenize{tech_note/Decomposition/CLM50_Tech_Note_Decomposition:table-decomposition-rate-constants}}\label{\detokenize{tech_note/Decomposition/CLM50_Tech_Note_Decomposition:id3}}
\sphinxaftercaption
\begin{tabular}[t]{|*{5}{\X{1}{5}|}}
\hline
\sphinxstylethead{\sphinxstyletheadfamily \unskip}\relax &\sphinxstylethead{\sphinxstyletheadfamily 
Biome-BGC
\unskip}\relax &\sphinxstylethead{\sphinxstyletheadfamily 
CLM-CN
\unskip}\relax &\sphinxstylethead{\sphinxstyletheadfamily \unskip}\relax &\sphinxstylethead{\sphinxstyletheadfamily \unskip}\relax \\
\hline&
\({k}_{disc1}\)(d$^{\text{-1}}$)
&
\({k}_{disc2}\) (hr$^{\text{-1}}$)
&
\sphinxstyleemphasis{C:N ratio}
&
Acceleration term (\({a}_{i}\))
\\
\hline
\({k}_{Lit1}\)
&
0.7
&
0.04892
&\begin{itemize}
\item {} 
\end{itemize}
&
1
\\
\hline
\({k}_{Lit2}\)
&
0.07
&
0.00302
&\begin{itemize}
\item {} 
\end{itemize}
&
1
\\
\hline
\({k}_{Lit3}\)
&
0.014
&
0.00059
&\begin{itemize}
\item {} 
\end{itemize}
&
1
\\
\hline
\({k}_{SOM1}\)
&
0.07
&
0.00302
&
12
&
1
\\
\hline
\({k}_{SOM2}\)
&
0.014
&
0.00059
&
12
&
1
\\
\hline
\({k}_{SOM3}\)
&
0.0014
&
0.00006
&
10
&
5
\\
\hline
\({k}_{SOM4}\)
&
0.0001
&
0.000004
&
10
&
70
\\
\hline
\({k}_{CWD}\)
&
0.001
&
0.00004
&\begin{itemize}
\item {} 
\end{itemize}
&
1
\\
\hline
\end{tabular}
\par
\sphinxattableend\end{savenotes}

The first column of \hyperref[\detokenize{tech_note/Decomposition/CLM50_Tech_Note_Decomposition:table-decomposition-rate-constants}]{Table \ref{\detokenize{tech_note/Decomposition/CLM50_Tech_Note_Decomposition:table-decomposition-rate-constants}}} gives the rates as used for the Biome-BGC
model, which uses a discrete-time model with a daily timestep. The
second column of \hyperref[\detokenize{tech_note/Decomposition/CLM50_Tech_Note_Decomposition:table-decomposition-rate-constants}]{Table \ref{\detokenize{tech_note/Decomposition/CLM50_Tech_Note_Decomposition:table-decomposition-rate-constants}}} shows the rates transformed for a one-hour
discrete timestep typical of CLM-CN. The transformation is based on the
conversion of the initial discrete-time value (\({k}_{disc1}\))
first to a continuous time value (\({k}_{cont}\)), then to the
new discrete-time value with a different timestep
(\({k}_{disc2}\)) , following Olson (1963):
\phantomsection\label{\detokenize{tech_note/Decomposition/CLM50_Tech_Note_Decomposition:equation-ZEqnNum608251}}\begin{equation}\label{equation:tech_note/Decomposition/CLM50_Tech_Note_Decomposition:ZEqnNum608251}
\begin{split}k_{cont} =-\log \left(1-k_{disc1} \right)\end{split}
\end{equation}\phantomsection\label{\detokenize{tech_note/Decomposition/CLM50_Tech_Note_Decomposition:equation-ZEqnNum772630}}\begin{equation}\label{equation:tech_note/Decomposition/CLM50_Tech_Note_Decomposition:ZEqnNum772630}
\begin{split}k_{disc2} =1-\exp \left(-k_{cont} \frac{\Delta t_{2} }{\Delta t_{1} } \right)\end{split}
\end{equation}
where \(\Delta\)\({t}_{1}\) (s) and
\(\Delta\)t$_{\text{2}}$ (s) are the time steps of the
initial and new discrete-time models, respectively.

Respiration fractions are parameterized for decomposition fluxes out of
each litter and SOM pool. The respiration fraction (\sphinxstyleemphasis{rf}, unitless) is
the fraction of the decomposition carbon flux leaving one of the litter
or SOM pools that is released as CO$_{\text{2}}$ due to heterotrophic
respiration. Respiration fractions and exponential decay rates are
estimated simultaneously from the results of microcosm decomposition
experiments (Thornton, 1998). The same values are used in CLM-CN and
Biome-BGC (\hyperref[\detokenize{tech_note/Decomposition/CLM50_Tech_Note_Decomposition:table-respiration-fractions-for-litter-and-som-pools}]{Table \ref{\detokenize{tech_note/Decomposition/CLM50_Tech_Note_Decomposition:table-respiration-fractions-for-litter-and-som-pools}}}).


\begin{savenotes}\sphinxattablestart
\centering
\sphinxcapstartof{table}
\sphinxcaption{Respiration fractions for litter and SOM pools}\label{\detokenize{tech_note/Decomposition/CLM50_Tech_Note_Decomposition:table-respiration-fractions-for-litter-and-som-pools}}\label{\detokenize{tech_note/Decomposition/CLM50_Tech_Note_Decomposition:id4}}
\sphinxaftercaption
\begin{tabulary}{\linewidth}[t]{|T|T|}
\hline
\sphinxstylethead{\sphinxstyletheadfamily 
Pool
\unskip}\relax &\sphinxstylethead{\sphinxstyletheadfamily 
\sphinxstyleemphasis{rf}
\unskip}\relax \\
\hline
\({rf}_{Lit1}\)
&
0.39
\\
\hline
\({rf}_{Lit2}\)
&
0.55
\\
\hline
\({rf}_{Lit3}\)
&
0.29
\\
\hline
\({rf}_{SOM1}\)
&
0.28
\\
\hline
\({rf}_{SOM2}\)
&
0.46
\\
\hline
\({rf}_{SOM3}\)
&
0.55
\\
\hline
\({rf}_{SOM4}\)
&
\({1.0}^{a}\)
\\
\hline
\end{tabulary}
\par
\sphinxattableend\end{savenotes}

$^{\text{a}}$\({}^{a}\) The respiration fraction for pool SOM4 is 1.0 by
definition: since there is no pool downstream of SOM4, the entire carbon
flux leaving this pool is assumed to be respired as CO$_{\text{2}}$.


\subsection{Century-based Pool Structure, Rate Constants and Parameters}
\label{\detokenize{tech_note/Decomposition/CLM50_Tech_Note_Decomposition:century-based-pool-structure-rate-constants-and-parameters}}
The Century-based decomposition cascade is, like CLM-CN, a first-order
decay model; the two structures differ in the number of pools, the
connections between those pools, the turnover times of the pools, and
the respired fraction during each transition (Figure 15.2). The turnover
times are different for the Century-based pool structure, following
those described in Parton et al. (1988) (\hyperref[\detokenize{tech_note/Decomposition/CLM50_Tech_Note_Decomposition:table-turnover-times}]{Table \ref{\detokenize{tech_note/Decomposition/CLM50_Tech_Note_Decomposition:table-turnover-times}}}).


\begin{savenotes}\sphinxattablestart
\centering
\sphinxcapstartof{table}
\sphinxcaption{Turnover times, C:N ratios, and acceleration parameters (see section 15.8 for explanation) for the Century-based decomposition cascade.}\label{\detokenize{tech_note/Decomposition/CLM50_Tech_Note_Decomposition:table-turnover-times}}\label{\detokenize{tech_note/Decomposition/CLM50_Tech_Note_Decomposition:id5}}
\sphinxaftercaption
\begin{tabular}[t]{|*{4}{\X{1}{4}|}}
\hline
\sphinxstylethead{\sphinxstyletheadfamily \unskip}\relax &\sphinxstylethead{\sphinxstyletheadfamily 
Turnover time (year)
\unskip}\relax &\sphinxstylethead{\sphinxstyletheadfamily 
C:N ratio
\unskip}\relax &\sphinxstylethead{\sphinxstyletheadfamily 
Acceleration term (\({a}_{i}\))
\unskip}\relax \\
\hline
CWD
&
4.1
&\begin{itemize}
\item {} 
\end{itemize}
&
1
\\
\hline
Litter 1
&
0.066
&\begin{itemize}
\item {} 
\end{itemize}
&
1
\\
\hline
Litter 2
&
0.25
&\begin{itemize}
\item {} 
\end{itemize}
&
1
\\
\hline
Litter 3
&
0.25
&\begin{itemize}
\item {} 
\end{itemize}
&
1
\\
\hline
SOM 1
&
0.17
&
8
&
1
\\
\hline
SOM 2
&
6.1
&
11
&
15
\\
\hline
SOM 3
&
270
&
11
&
675
\\
\hline
\end{tabular}
\par
\sphinxattableend\end{savenotes}

Likewise, values for the respiration fraction of Century-based structure are in \hyperref[\detokenize{tech_note/Decomposition/CLM50_Tech_Note_Decomposition:table-respiration-fractions-for-century-based-structure}]{Table \ref{\detokenize{tech_note/Decomposition/CLM50_Tech_Note_Decomposition:table-respiration-fractions-for-century-based-structure}}}.


\begin{savenotes}\sphinxattablestart
\centering
\sphinxcapstartof{table}
\sphinxcaption{Respiration fractions for litter and SOM pools for Century-based structure}\label{\detokenize{tech_note/Decomposition/CLM50_Tech_Note_Decomposition:table-respiration-fractions-for-century-based-structure}}\label{\detokenize{tech_note/Decomposition/CLM50_Tech_Note_Decomposition:id6}}
\sphinxaftercaption
\begin{tabulary}{\linewidth}[t]{|T|T|}
\hline
\sphinxstylethead{\sphinxstyletheadfamily 
Pool
\unskip}\relax &\sphinxstylethead{\sphinxstyletheadfamily 
\sphinxstyleemphasis{rf}
\unskip}\relax \\
\hline
\({rf}_{Lit1}\)
&
0.55
\\
\hline
\({rf}_{Lit2}\)
&
0.5
\\
\hline
\({rf}_{Lit3}\)
&
0.5
\\
\hline
\({rf}_{SOM1}\)
&
f(txt)
\\
\hline
\({rf}_{SOM2}\)
&
0.55
\\
\hline
\({rf}_{SOM3}\)
&
0.55
\\
\hline
\end{tabulary}
\par
\sphinxattableend\end{savenotes}


\subsection{Environmental modifiers on decomposition rate}
\label{\detokenize{tech_note/Decomposition/CLM50_Tech_Note_Decomposition:environmental-modifiers-on-decomposition-rate}}
These base rates are modified on each timestep by functions of the
current soil environment. For the single-level model, there are two rate
modifiers, temperature (\({r}_{tsoil}\), unitless) and moisture
(\({r}_{water}\), unitless), both of which are calculated using
the average environmental conditions of the top five model levels (top
29 cm of soil column). For the vertically-resolved model, two additional
environmental modifiers are calculated beyond the temperature and
moisture limitations: an oxygen scalar (\({r}_{oxygen}\),
unitless), and a depth scalar (\({r}_{depth}\), unitless).

The Temperature scalar \({r}_{tsoil}\) is calculated in CLM
using a \({Q}_{10}\) approach, with \({Q}_{10} = 1.5\).
\phantomsection\label{\detokenize{tech_note/Decomposition/CLM50_Tech_Note_Decomposition:equation-21.5)}}\begin{equation}\label{equation:tech_note/Decomposition/CLM50_Tech_Note_Decomposition:21.5)}
\begin{split}r_{tsoil} =Q_{10} ^{\left(\frac{T_{soil,\, j} -T_{ref} }{10} \right)}\end{split}
\end{equation}
where \sphinxstyleemphasis{j} is the soil layer index, \({T}_{soil,j}\) (K) is the
temperature of soil level \sphinxstyleemphasis{j}. The reference temperature \({T}_{ref}\) = 25C.

The rate scalar for soil water potential (\({r}_{water}\),
unitless) is calculated using a relationship from Andrén and Paustian
(1987) and supported by additional data in Orchard and Cook (1983):
\phantomsection\label{\detokenize{tech_note/Decomposition/CLM50_Tech_Note_Decomposition:equation-21.6)}}\begin{equation}\label{equation:tech_note/Decomposition/CLM50_Tech_Note_Decomposition:21.6)}
\begin{split}r_{water} =\sum _{j=1}^{5}\left\{\begin{array}{l} {0\qquad {\rm for\; }\Psi _{j} <\Psi _{\min } } \\ {\frac{\log \left({\Psi _{\min } \mathord{\left/ {\vphantom {\Psi _{\min }  \Psi _{j} }} \right. \kern-\nulldelimiterspace} \Psi _{j} } \right)}{\log \left({\Psi _{\min } \mathord{\left/ {\vphantom {\Psi _{\min }  \Psi _{\max } }} \right. \kern-\nulldelimiterspace} \Psi _{\max } } \right)} w_{soil,\, j} \qquad {\rm for\; }\Psi _{\min } \le \Psi _{j} \le \Psi _{\max } } \\ {1\qquad {\rm for\; }\Psi _{j} >\Psi _{\max } \qquad \qquad } \end{array}\right\}\end{split}
\end{equation}
where \({\Psi}_{j}\) is the soil water potential in
layer \sphinxstyleemphasis{j}, \({\Psi}_{min}\) is a lower limit for soil
water potential control on decomposition rate (in CLM5, this was
changed from a default value of -10 MPa used in CLM4.5 and earlier to a
default value of -2.5 MPa). \({\Psi}_{max,j}\) (MPa) is the soil
moisture at which decomposition proceeds at a moisture-unlimited
rate. The default value of \({\Psi}_{max,j}\) for CLM5 is updated
from a saturated value used in CLM4.5 and earlier, to a value
nominally at field capacity, with a value of -0.002 MPa

For frozen soils, the bulk of the rapid dropoff in decomposition with
decreasing temperature is due to the moisture limitation, since matric
potential is limited by temperature in the supercooled water formulation
of Niu and Yang (2006),
\phantomsection\label{\detokenize{tech_note/Decomposition/CLM50_Tech_Note_Decomposition:equation-21.8)}}\begin{equation}\label{equation:tech_note/Decomposition/CLM50_Tech_Note_Decomposition:21.8)}
\begin{split}\psi \left(T\right)=-\frac{L_{f} \left(T-T_{f} \right)}{10^{3} T}\end{split}
\end{equation}
An additional frozen decomposition limitation can be specified using a
‘frozen Q$_{\text{10}}$’ following {\hyperref[\detokenize{tech_note/References/CLM50_Tech_Note_References:kovenetal2011}]{\sphinxcrossref{\DUrole{std,std-ref}{Koven et al. (2011)}}}}, however the
default value of this is the same as the unfrozen Q$_{\text{10}}$
value, and therefore the basic hypothesis is that frozen respiration is
limited by liquid water availability, and can be modeled following the
same approach as thawed but dry soils.

An additional rate scalar, \({r}_{oxygen}\) is enabled when the
CH$_{\text{4}}$ submodel is used (set equal to 1 for the single layer
model or when the CH$_{\text{4}}$ submodel is disabled). This limits
decomposition when there is insufficient molecular oxygen to satisfy
stoichiometric demand (1 mol O$_{\text{2}}$ consumed per mol
CO$_{\text{2}}$ produced) from heterotrophic decomposers, and supply
from diffusion through soil layers (unsaturated and saturated) or
aerenchyma (Chapter 19). A minimum value of  \({r}_{oxygen}\) is
set at 0.2, with the assumption that oxygen within organic tissues can
supply the necessary stoichiometric demand at this rate. This value lies
between estimates of 0.025\textendash{}0.1 (Frolking et al. 2001), and 0.35 (Wania
et al. 2009); the large range of these estimates poses a large
unresolved uncertainty.

Lastly, a possible explicit depth dependence, \({r}_{depth}\),
(set equal to 1 for the single layer model) can be applied to soil C
decomposition rates to account for processes other than temperature,
moisture, and anoxia that can limit decomposition. This depth dependence
of decomposition was shown by Jenkinson and Coleman (2008) to be an
important term in fitting total C and 14C profiles, and implies that
unresolved processes, such as priming effects, microscale anoxia, soil
mineral surface and/or aggregate stabilization may be important in
controlling the fate of carbon at depth {\hyperref[\detokenize{tech_note/References/CLM50_Tech_Note_References:kovenetal2013}]{\sphinxcrossref{\DUrole{std,std-ref}{Koven et al. (2013)}}}}. CLM
includes these unresolved depth controls via an exponential decrease in
the soil turnover time with depth:
\phantomsection\label{\detokenize{tech_note/Decomposition/CLM50_Tech_Note_Decomposition:equation-21.9)}}\begin{equation}\label{equation:tech_note/Decomposition/CLM50_Tech_Note_Decomposition:21.9)}
\begin{split}r_{depth} =\exp \left(-\frac{z}{z_{\tau } } \right)\end{split}
\end{equation}
where \({z}_{\tau}\) is the e-folding depth for decomposition. For
CLM4.5, the default value of this was 0.5m. For CLM5, this has been
changed to a default value of 10m, which effectively means that
intrinsic decomposition rates may proceed as quickly at depth as at the surface.

The combined decomposition rate scalar (\({r}_{total}\),unitless) is:
\phantomsection\label{\detokenize{tech_note/Decomposition/CLM50_Tech_Note_Decomposition:equation-21.10)}}\begin{equation}\label{equation:tech_note/Decomposition/CLM50_Tech_Note_Decomposition:21.10)}
\begin{split}r_{total} =r_{tsoil} r_{water} r_{oxygen} r_{depth} .\end{split}
\end{equation}

\subsection{N-limitation of Decomposition Fluxes}
\label{\detokenize{tech_note/Decomposition/CLM50_Tech_Note_Decomposition:n-limitation-of-decomposition-fluxes}}
Decomposition rates can also be limited by the availability of mineral
nitrogen, but calculation of this limitation depends on first estimating
the potential rates of decomposition, assuming an unlimited mineral
nitrogen supply. The general case is described here first, referring to
a generic decomposition flux from an “upstream” pool (\sphinxstyleemphasis{u}) to a
“downstream” pool (\sphinxstyleemphasis{d}), with an intervening loss due to respiration.
The potential carbon flux out of the upstream pool
(\({CF}_{pot,u}\), gC m$^{\text{-2}}$ s$^{\text{-1}}$) is:
\phantomsection\label{\detokenize{tech_note/Decomposition/CLM50_Tech_Note_Decomposition:equation-21.11)}}\begin{equation}\label{equation:tech_note/Decomposition/CLM50_Tech_Note_Decomposition:21.11)}
\begin{split}CF_{pot,\, u} =CS_{u} k_{u}\end{split}
\end{equation}
where \({CS}_{u}\) (gC m$^{\text{-2}}$) is the initial mass
in the upstream pool and \({k}_{u}\) is the decay rate constant
(s:sup:\sphinxtitleref{-1}) for the upstream pool, adjusted for temperature and
moisture conditions. Depending on the C:N ratios of the upstream and
downstream pools and the amount of carbon lost in the transformation due
to respiration (the respiration fraction), the execution of this
potential carbon flux can generate either a source or a sink of new
mineral nitrogen
(\({NF}_{pot\_min,u}\)\({}_{\rightarrow}\)\({}_{d}\), gN m$^{\text{-2}}$ s$^{\text{-1}}$). The governing equation
(Thornton and Rosenbloom, 2005) is:
\phantomsection\label{\detokenize{tech_note/Decomposition/CLM50_Tech_Note_Decomposition:equation-21.12)}}\begin{equation}\label{equation:tech_note/Decomposition/CLM50_Tech_Note_Decomposition:21.12)}
\begin{split}NF_{pot\_ min,\, u\to d} =\frac{CF_{pot,\, u} \left(1-rf_{u} -\frac{CN_{d} }{CN_{u} } \right)}{CN_{d} }\end{split}
\end{equation}
where \({rf}_{u}\) is the respiration fraction for fluxes
leaving the upstream pool, \({CN}_{u}\) and  \({CN}_{d}\)
are the C:N ratios for upstream and downstream pools, respectively.
Negative values of
\({NF}_{pot\_min,u}\)\({}_{\rightarrow}\)\({}_{d}\)
indicate that the decomposition flux results in a source of new mineral
nitrogen, while positive values indicate that the potential
decomposition flux results in a sink (demand) for mineral nitrogen.

Following from the general case, potential carbon fluxes leaving
individual pools in the decomposition cascade, for the example of the
CLM-CN pool structure, are given as:
\phantomsection\label{\detokenize{tech_note/Decomposition/CLM50_Tech_Note_Decomposition:equation-21.13)}}\begin{equation}\label{equation:tech_note/Decomposition/CLM50_Tech_Note_Decomposition:21.13)}
\begin{split}CF_{pot,\, Lit1} ={CS_{Lit1} k_{Lit1} r_{total} \mathord{\left/ {\vphantom {CS_{Lit1} k_{Lit1} r_{total}  \Delta t}} \right. \kern-\nulldelimiterspace} \Delta t}\end{split}
\end{equation}\phantomsection\label{\detokenize{tech_note/Decomposition/CLM50_Tech_Note_Decomposition:equation-21.14)}}\begin{equation}\label{equation:tech_note/Decomposition/CLM50_Tech_Note_Decomposition:21.14)}
\begin{split}CF_{pot,\, Lit2} ={CS_{Lit2} k_{Lit2} r_{total} \mathord{\left/ {\vphantom {CS_{Lit2} k_{Lit2} r_{total}  \Delta t}} \right. \kern-\nulldelimiterspace} \Delta t}\end{split}
\end{equation}\phantomsection\label{\detokenize{tech_note/Decomposition/CLM50_Tech_Note_Decomposition:equation-21.15)}}\begin{equation}\label{equation:tech_note/Decomposition/CLM50_Tech_Note_Decomposition:21.15)}
\begin{split}CF_{pot,\, Lit3} ={CS_{Lit3} k_{Lit3} r_{total} \mathord{\left/ {\vphantom {CS_{Lit3} k_{Lit3} r_{total}  \Delta t}} \right. \kern-\nulldelimiterspace} \Delta t}\end{split}
\end{equation}\phantomsection\label{\detokenize{tech_note/Decomposition/CLM50_Tech_Note_Decomposition:equation-21.16)}}\begin{equation}\label{equation:tech_note/Decomposition/CLM50_Tech_Note_Decomposition:21.16)}
\begin{split}CF_{pot,\, SOM1} ={CS_{SOM1} k_{SOM1} r_{total} \mathord{\left/ {\vphantom {CS_{SOM1} k_{SOM1} r_{total}  \Delta t}} \right. \kern-\nulldelimiterspace} \Delta t}\end{split}
\end{equation}\phantomsection\label{\detokenize{tech_note/Decomposition/CLM50_Tech_Note_Decomposition:equation-21.17)}}\begin{equation}\label{equation:tech_note/Decomposition/CLM50_Tech_Note_Decomposition:21.17)}
\begin{split}CF_{pot,\, SOM2} ={CS_{SOM2} k_{SOM2} r_{total} \mathord{\left/ {\vphantom {CS_{SOM2} k_{SOM2} r_{total}  \Delta t}} \right. \kern-\nulldelimiterspace} \Delta t}\end{split}
\end{equation}\phantomsection\label{\detokenize{tech_note/Decomposition/CLM50_Tech_Note_Decomposition:equation-21.18)}}\begin{equation}\label{equation:tech_note/Decomposition/CLM50_Tech_Note_Decomposition:21.18)}
\begin{split}CF_{pot,\, SOM3} ={CS_{SOM3} k_{SOM3} r_{total} \mathord{\left/ {\vphantom {CS_{SOM3} k_{SOM3} r_{total}  \Delta t}} \right. \kern-\nulldelimiterspace} \Delta t}\end{split}
\end{equation}\phantomsection\label{\detokenize{tech_note/Decomposition/CLM50_Tech_Note_Decomposition:equation-21.19)}}\begin{equation}\label{equation:tech_note/Decomposition/CLM50_Tech_Note_Decomposition:21.19)}
\begin{split}CF_{pot,\, SOM4} ={CS_{SOM4} k_{SOM4} r_{total} \mathord{\left/ {\vphantom {CS_{SOM4} k_{SOM4} r_{total}  \Delta t}} \right. \kern-\nulldelimiterspace} \Delta t}\end{split}
\end{equation}
where the factor (1/\(\Delta\)\sphinxstyleemphasis{t}) is included because the rate
constant is calculated for the entire timestep (Eqs. and ), but the
convention is to express all fluxes on a per-second basis. Potential
mineral nitrogen fluxes associated with these decomposition steps are,
again for the example of the CLM-CN pool structure (the CENTURY
structure will be similar but without the different terminal step):
\phantomsection\label{\detokenize{tech_note/Decomposition/CLM50_Tech_Note_Decomposition:equation-ZEqnNum934998}}\begin{equation}\label{equation:tech_note/Decomposition/CLM50_Tech_Note_Decomposition:ZEqnNum934998}
\begin{split}NF_{pot\_ min,\, Lit1\to SOM1} ={CF_{pot,\, Lit1} \left(1-rf_{Lit1} -\frac{CN_{SOM1} }{CN_{Lit1} } \right)\mathord{\left/ {\vphantom {CF_{pot,\, Lit1} \left(1-rf_{Lit1} -\frac{CN_{SOM1} }{CN_{Lit1} } \right) CN_{SOM1} }} \right. \kern-\nulldelimiterspace} CN_{SOM1} }\end{split}
\end{equation}\phantomsection\label{\detokenize{tech_note/Decomposition/CLM50_Tech_Note_Decomposition:equation-21.21)}}\begin{equation}\label{equation:tech_note/Decomposition/CLM50_Tech_Note_Decomposition:21.21)}
\begin{split}NF_{pot\_ min,\, Lit2\to SOM2} ={CF_{pot,\, Lit2} \left(1-rf_{Lit2} -\frac{CN_{SOM2} }{CN_{Lit2} } \right)\mathord{\left/ {\vphantom {CF_{pot,\, Lit2} \left(1-rf_{Lit2} -\frac{CN_{SOM2} }{CN_{Lit2} } \right) CN_{SOM2} }} \right. \kern-\nulldelimiterspace} CN_{SOM2} }\end{split}
\end{equation}\phantomsection\label{\detokenize{tech_note/Decomposition/CLM50_Tech_Note_Decomposition:equation-21.22)}}\begin{equation}\label{equation:tech_note/Decomposition/CLM50_Tech_Note_Decomposition:21.22)}
\begin{split}NF_{pot\_ min,\, Lit3\to SOM3} ={CF_{pot,\, Lit3} \left(1-rf_{Lit3} -\frac{CN_{SOM3} }{CN_{Lit3} } \right)\mathord{\left/ {\vphantom {CF_{pot,\, Lit3} \left(1-rf_{Lit3} -\frac{CN_{SOM3} }{CN_{Lit3} } \right) CN_{SOM3} }} \right. \kern-\nulldelimiterspace} CN_{SOM3} }\end{split}
\end{equation}\phantomsection\label{\detokenize{tech_note/Decomposition/CLM50_Tech_Note_Decomposition:equation-21.23)}}\begin{equation}\label{equation:tech_note/Decomposition/CLM50_Tech_Note_Decomposition:21.23)}
\begin{split}NF_{pot\_ min,\, SOM1\to SOM2} ={CF_{pot,\, SOM1} \left(1-rf_{SOM1} -\frac{CN_{SOM2} }{CN_{SOM1} } \right)\mathord{\left/ {\vphantom {CF_{pot,\, SOM1} \left(1-rf_{SOM1} -\frac{CN_{SOM2} }{CN_{SOM1} } \right) CN_{SOM2} }} \right. \kern-\nulldelimiterspace} CN_{SOM2} }\end{split}
\end{equation}\phantomsection\label{\detokenize{tech_note/Decomposition/CLM50_Tech_Note_Decomposition:equation-21.24)}}\begin{equation}\label{equation:tech_note/Decomposition/CLM50_Tech_Note_Decomposition:21.24)}
\begin{split}NF_{pot\_ min,\, SOM2\to SOM3} ={CF_{pot,\, SOM2} \left(1-rf_{SOM2} -\frac{CN_{SOM3} }{CN_{SOM2} } \right)\mathord{\left/ {\vphantom {CF_{pot,\, SOM2} \left(1-rf_{SOM2} -\frac{CN_{SOM3} }{CN_{SOM2} } \right) CN_{SOM3} }} \right. \kern-\nulldelimiterspace} CN_{SOM3} }\end{split}
\end{equation}\phantomsection\label{\detokenize{tech_note/Decomposition/CLM50_Tech_Note_Decomposition:equation-21.25)}}\begin{equation}\label{equation:tech_note/Decomposition/CLM50_Tech_Note_Decomposition:21.25)}
\begin{split}NF_{pot\_ min,\, SOM3\to SOM4} ={CF_{pot,\, SOM3} \left(1-rf_{SOM3} -\frac{CN_{SOM4} }{CN_{SOM3} } \right)\mathord{\left/ {\vphantom {CF_{pot,\, SOM3} \left(1-rf_{SOM3} -\frac{CN_{SOM4} }{CN_{SOM3} } \right) CN_{SOM4} }} \right. \kern-\nulldelimiterspace} CN_{SOM4} }\end{split}
\end{equation}\phantomsection\label{\detokenize{tech_note/Decomposition/CLM50_Tech_Note_Decomposition:equation-ZEqnNum473594}}\begin{equation}\label{equation:tech_note/Decomposition/CLM50_Tech_Note_Decomposition:ZEqnNum473594}
\begin{split}NF_{pot\_ min,\, SOM4} =-{CF_{pot,\, SOM4} \mathord{\left/ {\vphantom {CF_{pot,\, SOM4}  CN_{SOM4} }} \right. \kern-\nulldelimiterspace} CN_{SOM4} }\end{split}
\end{equation}
where the special form of Eq. arises because there is no SOM pool
downstream of SOM4 in the converging cascade: all carbon fluxes leaving
that pool are assumed to be in the form of respired CO$_{\text{2}}$,
and all nitrogen fluxes leaving that pool are assumed to be sources of
new mineral nitrogen.

Steps in the decomposition cascade that result in release of new mineral
nitrogen (mineralization fluxes) are allowed to proceed at their
potential rates, without modification for nitrogen availability. Steps
that result in an uptake of mineral nitrogen (immobilization fluxes) are
subject to rate limitation, depending on the availability of mineral
nitrogen, the total immobilization demand, and the total demand for soil
mineral nitrogen to support new plant growth. The potential mineral
nitrogen fluxes from Eqs. - are evaluated, summing all the positive
fluxes to generate the total potential nitrogen immobilization flux
(\({NF}_{immob\_demand}\), gN m$^{\text{-2}}$ s$^{\text{-1}}$), and summing absolute values of all the negative
fluxes to generate the total nitrogen mineralization flux
(\({NF}_{gross\_nmin}\), gN m$^{\text{-2}}$ s$^{\text{-1}}$). Since \({NF}_{griss\_nmin}\) is a source of
new mineral nitrogen to the soil mineral nitrogen pool it is not limited
by the availability of soil mineral nitrogen, and is therefore an actual
as opposed to a potential flux.


\subsection{N Competition between plant uptake and soil immobilization fluxes}
\label{\detokenize{tech_note/Decomposition/CLM50_Tech_Note_Decomposition:n-competition-between-plant-uptake-and-soil-immobilization-fluxes}}
Once \({NF}_{immob\_demand }\) and \({NF}_{nit\_demand }\)  for each layer \sphinxstyleemphasis{j} are known, the competition between plant and microbial nitrogen demand can be resolved. Mineral nitrogen in
the soil pool (\({NS}_{sminn}\), gN m$^{\text{-2}}$) at the
beginning of the timestep is considered the available supply.

Here, the \({NF}_{plant\_demand}\) is the theoretical maximum demand for nitrogen by plants to meet the entire carbon uptake given an N cost of zero (and therefore represents the upper bound on N requirements). N uptake costs that are
\(>\) 0 imply that the plant will take up less N that it demands, ultimately. However, given the heuristic nature of the N competition algorithm, this discrepancy is not explicitly resolved here.

The hypothetical plant nitrogen demand from the soil mineral pool is distributed between layers in proportion to the profile of available mineral N:
\phantomsection\label{\detokenize{tech_note/Decomposition/CLM50_Tech_Note_Decomposition:equation-21.291}}\begin{equation}\label{equation:tech_note/Decomposition/CLM50_Tech_Note_Decomposition:21.291}
\begin{split}NF_{plant\_ demand,j} =  NF_{plant\_ demand} NS_{sminn\_ j}  / \sum _{j=1}^{nj}NS_{sminn,j}\end{split}
\end{equation}
Plants first compete for ammonia (NH4).   For each soil layer (\sphinxstyleemphasis{j}), we calculate the total NH4 demand as:
\phantomsection\label{\detokenize{tech_note/Decomposition/CLM50_Tech_Note_Decomposition:equation-21.292}}\begin{equation}\label{equation:tech_note/Decomposition/CLM50_Tech_Note_Decomposition:21.292}
\begin{split}NF_{total\_ demand_nh4,j}  = NF_{immob\_ demand,j}  + NF_{immob\_ demand,j} + NF_{nit\_ demand,j}\end{split}
\end{equation}
where
If \({NF}_{total\_demand,j}\)\(\Delta\)\sphinxstyleemphasis{t} \(<\)
\({NS}_{sminn,j}\), then the available pool is large enough to
meet both the maximum plant and microbial demand, then immobilization proceeds at the maximum rate.
\phantomsection\label{\detokenize{tech_note/Decomposition/CLM50_Tech_Note_Decomposition:equation-21.29)}}\begin{equation}\label{equation:tech_note/Decomposition/CLM50_Tech_Note_Decomposition:21.29)}
\begin{split}f_{immob\_demand,j} = 1.0\end{split}
\end{equation}
where \({f}_{immob\_demand,j}\) is the fraction of potential immobilization demand that can be met given current supply of mineral nitrogen in this layer. We also set the actual nitrification flux to be the same as the potential flux (\(NF_{nit}\) = \(NF_{nit\_ demand}\)).

If \({NF}_{total\_demand,j}\)\(\Delta\)\sphinxstyleemphasis{t}
\(\mathrm{\ge}\) \({NS}_{sminn,j}\), then there is not enough
mineral nitrogen to meet the combined demands for plant growth and
heterotrophic immobilization, immobilization is reduced proportional to the discrepancy, by \(f_{immob\_ demand,j}\), where
\phantomsection\label{\detokenize{tech_note/Decomposition/CLM50_Tech_Note_Decomposition:equation-21.30)}}\begin{equation}\label{equation:tech_note/Decomposition/CLM50_Tech_Note_Decomposition:21.30)}
\begin{split}f_{immob\_ demand,j} = \frac{NS_{sminn,j} }{\Delta t\, NF_{total\_ demand,j} }\end{split}
\end{equation}
The N available to the FUN model for plant uptake (\({NF}_ {plant\_ avail\_ sminn}\) (gN m$^{\text{-2}}$), which determines both the cost of N uptake, and the absolute limit on the N which is available for acquisition, is calculated as the total mineralized pool minus the actual immobilized flux:
\phantomsection\label{\detokenize{tech_note/Decomposition/CLM50_Tech_Note_Decomposition:equation-21.311)}}\begin{equation}\label{equation:tech_note/Decomposition/CLM50_Tech_Note_Decomposition:21.311)}
\begin{split}NF_{plant\_ avail\_ sminn,j} = NS_{sminn,j} - f_{immob\_demand} NF_{immob\_ demand,j}\end{split}
\end{equation}
This treatment of competition for nitrogen as a limiting resource is
referred to a demand-based competition, where the fraction of the
available resource that eventually flows to a particular process depends
on the demand from that process in comparison to the total demand from
all processes. Processes expressing a greater demand acquire a larger
vfraction of the available resource.


\subsection{Final Decomposition Fluxes}
\label{\detokenize{tech_note/Decomposition/CLM50_Tech_Note_Decomposition:final-decomposition-fluxes}}
With \({f}_{immob\_demand}\) known, final decomposition fluxes
can be calculated. Actual carbon fluxes leaving the individual litter
and SOM pools, again for the example of the CLM-CN pool structure (the
CENTURY structure will be similar but, again without the different
terminal step), are calculated as:
\phantomsection\label{\detokenize{tech_note/Decomposition/CLM50_Tech_Note_Decomposition:equation-21.32)}}\begin{equation}\label{equation:tech_note/Decomposition/CLM50_Tech_Note_Decomposition:21.32)}
\begin{split}CF_{Lit1} =\left\{\begin{array}{l} {CF_{pot,\, Lit1} f_{immob\_ demand} \qquad {\rm for\; }NF_{pot\_ min,\, Lit1\to SOM1} >0} \\ {CF_{pot,\, Lit1} \qquad {\rm for\; }NF_{pot\_ min,\, Lit1\to SOM1} \le 0} \end{array}\right\}\end{split}
\end{equation}\phantomsection\label{\detokenize{tech_note/Decomposition/CLM50_Tech_Note_Decomposition:equation-21.33)}}\begin{equation}\label{equation:tech_note/Decomposition/CLM50_Tech_Note_Decomposition:21.33)}
\begin{split}CF_{Lit2} =\left\{\begin{array}{l} {CF_{pot,\, Lit2} f_{immob\_ demand} \qquad {\rm for\; }NF_{pot\_ min,\, Lit2\to SOM2} >0} \\ {CF_{pot,\, Lit2} \qquad {\rm for\; }NF_{pot\_ min,\, Lit2\to SOM2} \le 0} \end{array}\right\}\end{split}
\end{equation}\phantomsection\label{\detokenize{tech_note/Decomposition/CLM50_Tech_Note_Decomposition:equation-21.34)}}\begin{equation}\label{equation:tech_note/Decomposition/CLM50_Tech_Note_Decomposition:21.34)}
\begin{split}CF_{Lit3} =\left\{\begin{array}{l} {CF_{pot,\, Lit3} f_{immob\_ demand} \qquad {\rm for\; }NF_{pot\_ min,\, Lit3\to SOM3} >0} \\ {CF_{pot,\, Lit3} \qquad {\rm for\; }NF_{pot\_ min,\, Lit3\to SOM3} \le 0} \end{array}\right\}\end{split}
\end{equation}\phantomsection\label{\detokenize{tech_note/Decomposition/CLM50_Tech_Note_Decomposition:equation-21.35)}}\begin{equation}\label{equation:tech_note/Decomposition/CLM50_Tech_Note_Decomposition:21.35)}
\begin{split}CF_{SOM1} =\left\{\begin{array}{l} {CF_{pot,\, SOM1} f_{immob\_ demand} \qquad {\rm for\; }NF_{pot\_ min,\, SOM1\to SOM2} >0} \\ {CF_{pot,\, SOM1} \qquad {\rm for\; }NF_{pot\_ min,\, SOM1\to SOM2} \le 0} \end{array}\right\}\end{split}
\end{equation}\phantomsection\label{\detokenize{tech_note/Decomposition/CLM50_Tech_Note_Decomposition:equation-21.36)}}\begin{equation}\label{equation:tech_note/Decomposition/CLM50_Tech_Note_Decomposition:21.36)}
\begin{split}CF_{SOM2} =\left\{\begin{array}{l} {CF_{pot,\, SOM2} f_{immob\_ demand} \qquad {\rm for\; }NF_{pot\_ min,\, SOM2\to SOM3} >0} \\ {CF_{pot,\, SOM2} \qquad {\rm for\; }NF_{pot\_ min,\, SOM2\to SOM3} \le 0} \end{array}\right\}\end{split}
\end{equation}\phantomsection\label{\detokenize{tech_note/Decomposition/CLM50_Tech_Note_Decomposition:equation-21.37)}}\begin{equation}\label{equation:tech_note/Decomposition/CLM50_Tech_Note_Decomposition:21.37)}
\begin{split}CF_{SOM3} =\left\{\begin{array}{l} {CF_{pot,\, SOM3} f_{immob\_ demand} \qquad {\rm for\; }NF_{pot\_ min,\, SOM3\to SOM4} >0} \\ {CF_{pot,\, SOM3} \qquad {\rm for\; }NF_{pot\_ min,\, SOM3\to SOM4} \le 0} \end{array}\right\}\end{split}
\end{equation}\phantomsection\label{\detokenize{tech_note/Decomposition/CLM50_Tech_Note_Decomposition:equation-21.38)}}\begin{equation}\label{equation:tech_note/Decomposition/CLM50_Tech_Note_Decomposition:21.38)}
\begin{split}CF_{SOM4} =CF_{pot,\, SOM4}\end{split}
\end{equation}
Heterotrophic respiration fluxes (losses of carbon as
CO$_{\text{2}}$ to the atmosphere) are:
\phantomsection\label{\detokenize{tech_note/Decomposition/CLM50_Tech_Note_Decomposition:equation-21.39)}}\begin{equation}\label{equation:tech_note/Decomposition/CLM50_Tech_Note_Decomposition:21.39)}
\begin{split}CF_{Lit1,\, HR} =CF_{Lit1} rf_{Lit1}\end{split}
\end{equation}\phantomsection\label{\detokenize{tech_note/Decomposition/CLM50_Tech_Note_Decomposition:equation-21.40)}}\begin{equation}\label{equation:tech_note/Decomposition/CLM50_Tech_Note_Decomposition:21.40)}
\begin{split}CF_{Lit2,\, HR} =CF_{Lit2} rf_{Lit2}\end{split}
\end{equation}\phantomsection\label{\detokenize{tech_note/Decomposition/CLM50_Tech_Note_Decomposition:equation-21.41)}}\begin{equation}\label{equation:tech_note/Decomposition/CLM50_Tech_Note_Decomposition:21.41)}
\begin{split}CF_{Lit3,\, HR} =CF_{Lit3} rf_{Lit3}\end{split}
\end{equation}\phantomsection\label{\detokenize{tech_note/Decomposition/CLM50_Tech_Note_Decomposition:equation-21.42)}}\begin{equation}\label{equation:tech_note/Decomposition/CLM50_Tech_Note_Decomposition:21.42)}
\begin{split}CF_{SOM1,\, HR} =CF_{SOM1} rf_{SOM1}\end{split}
\end{equation}\phantomsection\label{\detokenize{tech_note/Decomposition/CLM50_Tech_Note_Decomposition:equation-21.43)}}\begin{equation}\label{equation:tech_note/Decomposition/CLM50_Tech_Note_Decomposition:21.43)}
\begin{split}CF_{SOM2,\, HR} =CF_{SOM2} rf_{SOM2}\end{split}
\end{equation}\phantomsection\label{\detokenize{tech_note/Decomposition/CLM50_Tech_Note_Decomposition:equation-21.44)}}\begin{equation}\label{equation:tech_note/Decomposition/CLM50_Tech_Note_Decomposition:21.44)}
\begin{split}CF_{SOM3,\, HR} =CF_{SOM3} rf_{SOM3}\end{split}
\end{equation}\phantomsection\label{\detokenize{tech_note/Decomposition/CLM50_Tech_Note_Decomposition:equation-21.45)}}\begin{equation}\label{equation:tech_note/Decomposition/CLM50_Tech_Note_Decomposition:21.45)}
\begin{split}CF_{SOM4,\, HR} =CF_{SOM4} rf_{SOM4}\end{split}
\end{equation}
Transfers of carbon from upstream to downstream pools in the
decomposition cascade are given as:
\phantomsection\label{\detokenize{tech_note/Decomposition/CLM50_Tech_Note_Decomposition:equation-21.46)}}\begin{equation}\label{equation:tech_note/Decomposition/CLM50_Tech_Note_Decomposition:21.46)}
\begin{split}CF_{Lit1,\, SOM1} =CF_{Lit1} \left(1-rf_{Lit1} \right)\end{split}
\end{equation}\phantomsection\label{\detokenize{tech_note/Decomposition/CLM50_Tech_Note_Decomposition:equation-21.47)}}\begin{equation}\label{equation:tech_note/Decomposition/CLM50_Tech_Note_Decomposition:21.47)}
\begin{split}CF_{Lit2,\, SOM2} =CF_{Lit2} \left(1-rf_{Lit2} \right)\end{split}
\end{equation}\phantomsection\label{\detokenize{tech_note/Decomposition/CLM50_Tech_Note_Decomposition:equation-21.48)}}\begin{equation}\label{equation:tech_note/Decomposition/CLM50_Tech_Note_Decomposition:21.48)}
\begin{split}CF_{Lit3,\, SOM3} =CF_{Lit3} \left(1-rf_{Lit3} \right)\end{split}
\end{equation}\phantomsection\label{\detokenize{tech_note/Decomposition/CLM50_Tech_Note_Decomposition:equation-21.49)}}\begin{equation}\label{equation:tech_note/Decomposition/CLM50_Tech_Note_Decomposition:21.49)}
\begin{split}CF_{SOM1,\, SOM2} =CF_{SOM1} \left(1-rf_{SOM1} \right)\end{split}
\end{equation}\phantomsection\label{\detokenize{tech_note/Decomposition/CLM50_Tech_Note_Decomposition:equation-21.50)}}\begin{equation}\label{equation:tech_note/Decomposition/CLM50_Tech_Note_Decomposition:21.50)}
\begin{split}CF_{SOM2,\, SOM3} =CF_{SOM2} \left(1-rf_{SOM2} \right)\end{split}
\end{equation}\phantomsection\label{\detokenize{tech_note/Decomposition/CLM50_Tech_Note_Decomposition:equation-21.51)}}\begin{equation}\label{equation:tech_note/Decomposition/CLM50_Tech_Note_Decomposition:21.51)}
\begin{split}CF_{SOM3,\, SOM4} =CF_{SOM3} \left(1-rf_{SOM3} \right)\end{split}
\end{equation}
In accounting for the fluxes of nitrogen between pools in the
decomposition cascade and associated fluxes to or from the soil mineral
nitrogen pool, the model first calculates a flux of nitrogen from an
upstream pool to a downstream pool, then calculates a flux either from
the soil mineral nitrogen pool to the downstream pool (immobilization)
or from the downstream pool to the soil mineral nitrogen pool
(mineralization). Transfers of nitrogen from upstream to downstream
pools in the decomposition cascade are given as:
\phantomsection\label{\detokenize{tech_note/Decomposition/CLM50_Tech_Note_Decomposition:equation-21.52)}}\begin{equation}\label{equation:tech_note/Decomposition/CLM50_Tech_Note_Decomposition:21.52)}
\begin{split}NF_{Lit1,\, SOM1} ={CF_{Lit1} \mathord{\left/ {\vphantom {CF_{Lit1}  CN_{Lit1} }} \right. \kern-\nulldelimiterspace} CN_{Lit1} }\end{split}
\end{equation}\phantomsection\label{\detokenize{tech_note/Decomposition/CLM50_Tech_Note_Decomposition:equation-21.53)}}\begin{equation}\label{equation:tech_note/Decomposition/CLM50_Tech_Note_Decomposition:21.53)}
\begin{split}NF_{Lit2,\, SOM2} ={CF_{Lit2} \mathord{\left/ {\vphantom {CF_{Lit2}  CN_{Lit2} }} \right. \kern-\nulldelimiterspace} CN_{Lit2} }\end{split}
\end{equation}\phantomsection\label{\detokenize{tech_note/Decomposition/CLM50_Tech_Note_Decomposition:equation-21.54)}}\begin{equation}\label{equation:tech_note/Decomposition/CLM50_Tech_Note_Decomposition:21.54)}
\begin{split}NF_{Lit3,\, SOM3} ={CF_{Lit3} \mathord{\left/ {\vphantom {CF_{Lit3}  CN_{Lit3} }} \right. \kern-\nulldelimiterspace} CN_{Lit3} }\end{split}
\end{equation}\phantomsection\label{\detokenize{tech_note/Decomposition/CLM50_Tech_Note_Decomposition:equation-21.55)}}\begin{equation}\label{equation:tech_note/Decomposition/CLM50_Tech_Note_Decomposition:21.55)}
\begin{split}NF_{SOM1,\, SOM2} ={CF_{SOM1} \mathord{\left/ {\vphantom {CF_{SOM1}  CN_{SOM1} }} \right. \kern-\nulldelimiterspace} CN_{SOM1} }\end{split}
\end{equation}\phantomsection\label{\detokenize{tech_note/Decomposition/CLM50_Tech_Note_Decomposition:equation-21.56)}}\begin{equation}\label{equation:tech_note/Decomposition/CLM50_Tech_Note_Decomposition:21.56)}
\begin{split}NF_{SOM2,\, SOM3} ={CF_{SOM2} \mathord{\left/ {\vphantom {CF_{SOM2}  CN_{SOM2} }} \right. \kern-\nulldelimiterspace} CN_{SOM2} }\end{split}
\end{equation}\phantomsection\label{\detokenize{tech_note/Decomposition/CLM50_Tech_Note_Decomposition:equation-21.57)}}\begin{equation}\label{equation:tech_note/Decomposition/CLM50_Tech_Note_Decomposition:21.57)}
\begin{split}NF_{SOM3,\, SOM4} ={CF_{SOM3} \mathord{\left/ {\vphantom {CF_{SOM3}  CN_{SOM3} }} \right. \kern-\nulldelimiterspace} CN_{SOM3} }\end{split}
\end{equation}
Corresponding fluxes to or from the soil mineral nitrogen pool depend on
whether the decomposition step is an immobilization flux or a
mineralization flux:
\phantomsection\label{\detokenize{tech_note/Decomposition/CLM50_Tech_Note_Decomposition:equation-21.58)}}\begin{equation}\label{equation:tech_note/Decomposition/CLM50_Tech_Note_Decomposition:21.58)}
\begin{split}NF_{sminn,\, Lit1\to SOM1} =\left\{\begin{array}{l} {NF_{pot\_ min,\, Lit1\to SOM1} f_{immob\_ demand} \qquad {\rm for\; }NF_{pot\_ min,\, Lit1\to SOM1} >0} \\ {NF_{pot\_ min,\, Lit1\to SOM1} \qquad {\rm for\; }NF_{pot\_ min,\, Lit1\to SOM1} \le 0} \end{array}\right\}\end{split}
\end{equation}\phantomsection\label{\detokenize{tech_note/Decomposition/CLM50_Tech_Note_Decomposition:equation-21.59)}}\begin{equation}\label{equation:tech_note/Decomposition/CLM50_Tech_Note_Decomposition:21.59)}
\begin{split}NF_{sminn,\, Lit2\to SOM2} =\left\{\begin{array}{l} {NF_{pot\_ min,\, Lit2\to SOM2} f_{immob\_ demand} \qquad {\rm for\; }NF_{pot\_ min,\, Lit2\to SOM2} >0} \\ {NF_{pot\_ min,\, Lit2\to SOM2} \qquad {\rm for\; }NF_{pot\_ min,\, Lit2\to SOM2} \le 0} \end{array}\right\}\end{split}
\end{equation}\phantomsection\label{\detokenize{tech_note/Decomposition/CLM50_Tech_Note_Decomposition:equation-21.60)}}\begin{equation}\label{equation:tech_note/Decomposition/CLM50_Tech_Note_Decomposition:21.60)}
\begin{split}NF_{sminn,\, Lit3\to SOM3} =\left\{\begin{array}{l} {NF_{pot\_ min,\, Lit3\to SOM3} f_{immob\_ demand} \qquad {\rm for\; }NF_{pot\_ min,\, Lit3\to SOM3} >0} \\ {NF_{pot\_ min,\, Lit3\to SOM3} \qquad {\rm for\; }NF_{pot\_ min,\, Lit3\to SOM3} \le 0} \end{array}\right\}\end{split}
\end{equation}\phantomsection\label{\detokenize{tech_note/Decomposition/CLM50_Tech_Note_Decomposition:equation-21.61)}}\begin{equation}\label{equation:tech_note/Decomposition/CLM50_Tech_Note_Decomposition:21.61)}
\begin{split}NF_{sminn,SOM1\to SOM2} =\left\{\begin{array}{l} {NF_{pot\_ min,\, SOM1\to SOM2} f_{immob\_ demand} \qquad {\rm for\; }NF_{pot\_ min,\, SOM1\to SOM2} >0} \\ {NF_{pot\_ min,\, SOM1\to SOM2} \qquad {\rm for\; }NF_{pot\_ min,\, SOM1\to SOM2} \le 0} \end{array}\right\}\end{split}
\end{equation}\phantomsection\label{\detokenize{tech_note/Decomposition/CLM50_Tech_Note_Decomposition:equation-21.62)}}\begin{equation}\label{equation:tech_note/Decomposition/CLM50_Tech_Note_Decomposition:21.62)}
\begin{split}NF_{sminn,SOM2\to SOM3} =\left\{\begin{array}{l} {NF_{pot\_ min,\, SOM2\to SOM3} f_{immob\_ demand} \qquad {\rm for\; }NF_{pot\_ min,\, SOM2\to SOM3} >0} \\ {NF_{pot\_ min,\, SOM2\to SOM3} \qquad {\rm for\; }NF_{pot\_ min,\, SOM2\to SOM3} \le 0} \end{array}\right\}\end{split}
\end{equation}\phantomsection\label{\detokenize{tech_note/Decomposition/CLM50_Tech_Note_Decomposition:equation-21.63)}}\begin{equation}\label{equation:tech_note/Decomposition/CLM50_Tech_Note_Decomposition:21.63)}
\begin{split}NF_{sminn,SOM3\to SOM4} =\left\{\begin{array}{l} {NF_{pot\_ min,\, SOM3\to SOM4} f_{immob\_ demand} \qquad {\rm for\; }NF_{pot\_ min,\, SOM3\to SOM4} >0} \\ {NF_{pot\_ min,\, SOM3\to SOM4} \qquad {\rm for\; }NF_{pot\_ min,\, SOM3\to SOM4} \le 0} \end{array}\right\}\end{split}
\end{equation}\phantomsection\label{\detokenize{tech_note/Decomposition/CLM50_Tech_Note_Decomposition:equation-21.64)}}\begin{equation}\label{equation:tech_note/Decomposition/CLM50_Tech_Note_Decomposition:21.64)}
\begin{split}NF_{sminn,\, SOM4} =NF_{pot\_ min,\, SOM4}\end{split}
\end{equation}

\subsection{Vertical Distribution and Transport of Decomposing C and N pools}
\label{\detokenize{tech_note/Decomposition/CLM50_Tech_Note_Decomposition:vertical-distribution-and-transport-of-decomposing-c-and-n-pools}}
Additional terms are needed to calculate the vertically-resolved soil C
and N budget: the initial vertical distribution of C and N from PFTs
delivered to the litter and CWD pools, and the vertical transport of C
and N pools.

For initial vertical inputs, CLM uses separate profiles for aboveground
(leaf, stem) and belowground (root) inputs. Aboveground inputs are given
a single exponential with default e-folding depth = 0.1m. Belowground
inputs are distributed according to rooting profiles with default values
based on the Jackson et al. (1996) exponential parameterization.

Vertical mixing is accomplished by an advection-diffusion equation. The
goal of this is to consider slow, soild- and adsorbed-phase transport
due to bioturbation, cryoturbation, and erosion. Faster aqueous-phase
transport is not included in CLM, but has been developed as part of the
CLM-BeTR suite of parameterizations (Tang and Riley 2013). The default
value of the advection term is 0 cm/yr, such that transport is purely
diffusive. Diffusive transport differs in rate between permafrost soils
(where cryoturbation is the dominant transport term) and non-permafrost
soils (where bioturbation dominates). For permafrost soils, a
parameterization based on that of {\hyperref[\detokenize{tech_note/References/CLM50_Tech_Note_References:kovenetal2009}]{\sphinxcrossref{\DUrole{std,std-ref}{Koven et al. (2009)}}}} is used: the
diffusivity parameter is constant through the active layer, and
decreases linearly from the base of the active layer to zero at a set
depth (default 3m); the default permafrost diffusivity is 5
cm$^{\text{2}}$/yr. For non-permafrost soils, the default diffusivity
is 1 cm$^{\text{2}}$/yr.


\subsection{Model Equilibration and its Acceleration}
\label{\detokenize{tech_note/Decomposition/CLM50_Tech_Note_Decomposition:model-equilibration-and-its-acceleration}}
For transient experiments, it is usually assumed that the carbon cycle
is starting from a point of relatively close equilibrium, i.e. that
productivity is balanced by ecosystem carbon losses through
respiratory and disturbance pathways.  In order to satisfy this
assumption, the model is generally run until the productivity and loss
terms find a stable long-term equilibrium; at this point the model is
considered ‘spun up’.

Because of the coupling between the slowest SOM pools and productivity
through N downregulation of photosynthesis, equilibration of the model
for initialization purposes will take an extremely long time in the
standard mode. This is particularly true for the CENTURY-based
decomposition cascade, which includes a passive pool. In order to
rapidly equilibrate the model, a modified version of the “accelerated
decomposition” {\hyperref[\detokenize{tech_note/References/CLM50_Tech_Note_References:thorntonrosenbloom2005}]{\sphinxcrossref{\DUrole{std,std-ref}{(Thornton and Rosenbloon, 2005)}}}} is used. The fundamental
idea of this approach is to allow fluxes between the various pools (both
turnover-defined and vertically-defined fluxes) adjust rapidly, while
keeping the pool sizes themselves small so that they can fill quickly.
To do this, the base decomposition rate  \({k}_{i}\) for each
pool \sphinxstyleemphasis{i} is accelerated by a term \({a}_{i}\) such that the slow
pools are collapsed onto an approximately annual timescale {\hyperref[\detokenize{tech_note/References/CLM50_Tech_Note_References:kovenetal2013}]{\sphinxcrossref{\DUrole{std,std-ref}{Koven et al. (2013)}}}}. Accelerating the pools beyond this timescale distorts the
seasonal and/or diurnal cycles of decomposition and N mineralization,
thus leading to a substantially different ecosystem productivity than
the full model. For the vertical model, the vertical transport terms are
also accelerated by the same term \({a}_{i}\), as is the
radioactive decay when \({}^{14}\)C is enabled, following the same
principle of keeping fluxes between pools (or fluxes lost to decay)
close to the full model while keeping the pools sizes small. When
leaving the accelerated decomposition mode, the concentration of C and N
in pools that had been accelerated are multiplied by the same term
\({a}_{i}\), to bring the model into approximate equilibrium.
Note that in CLM, the model can also transition into accelerated
decomposition mode from the standard mode (by dividing the pools by
\({a}_{i}\)), and that the transitions into and out of
accelerated decomposition mode are handled automatically by CLM upon
loading from restart files (which preserve information about the mode of
the model when restart files were written).

The base acceleration terms for the two decomposition cascades are shown in
Tables 15.1 and 15.3.  In addition to the base terms, CLM5 also
includes a geographic term to the acceleration in order to apply
larger values to high-latitude systems, where decomposition rates are
particularly slow and thus equilibration can take significantly longer
than in temperate or tropical climates.  This geographic term takes
the form of a logistic equation, where \({a}_{i}\) is equal to the
product of the base acceleration term and \({a}_{l}\) below:
\phantomsection\label{\detokenize{tech_note/Decomposition/CLM50_Tech_Note_Decomposition:equation-21.65)}}\begin{equation}\label{equation:tech_note/Decomposition/CLM50_Tech_Note_Decomposition:21.65)}
\begin{split} a_l = 1 + 50 / \left ( 1 + exp \left (-0.1 * (abs(latitude) -
 60 ) \right ) \right )\end{split}
\end{equation}

\section{External Nitrogen Cycle}
\label{\detokenize{tech_note/External_Nitrogen_Cycle/CLM50_Tech_Note_External_Nitrogen_Cycle::doc}}\label{\detokenize{tech_note/External_Nitrogen_Cycle/CLM50_Tech_Note_External_Nitrogen_Cycle:external-nitrogen-cycle}}\label{\detokenize{tech_note/External_Nitrogen_Cycle/CLM50_Tech_Note_External_Nitrogen_Cycle:rst-external-nitrogen-cycle}}

\subsection{Summary of CLM5.0 updates relative to CLM4.5}
\label{\detokenize{tech_note/External_Nitrogen_Cycle/CLM50_Tech_Note_External_Nitrogen_Cycle:summary-of-clm5-0-updates-relative-to-clm4-5}}\label{\detokenize{tech_note/External_Nitrogen_Cycle/CLM50_Tech_Note_External_Nitrogen_Cycle:id1}}
We describe external inputs to the nitrogen cycle in CLM5.0.  Much of the following information appeared in the CLM4.5 Technical Note ({\hyperref[\detokenize{tech_note/References/CLM50_Tech_Note_References:olesonetal2013}]{\sphinxcrossref{\DUrole{std,std-ref}{Oleson et al. 2013}}}}) as well as {\hyperref[\detokenize{tech_note/References/CLM50_Tech_Note_References:kovenetal2013}]{\sphinxcrossref{\DUrole{std,std-ref}{Koven et al. (2013)}}}}.

CLM5.0 includes the following changes to terrestrial nitrogen inputs:
\begin{itemize}
\item {} 
Time varrying deposition of reactive nitrogen. In off-line runs this changes monthly. In coupled simulations N deposition is passed at the coupling timestep (e.g., half-hourly).

\item {} 
Asymbiotic (or free living) N fixation is a function of evapotranspiration and is added to the inorganic nitrogen (NH$_{\text{4}}$$^{\text{+}}$) pool (described below).

\item {} 
Symbiotic N fixation is handled by the FUN model (chapter \hyperref[\detokenize{tech_note/FUN/CLM50_Tech_Note_FUN:rst-fun}]{\ref{\detokenize{tech_note/FUN/CLM50_Tech_Note_FUN:rst-fun}}}) and is passed straight to the plant, not the mineral nitrogen pool.

\end{itemize}


\subsection{Overview}
\label{\detokenize{tech_note/External_Nitrogen_Cycle/CLM50_Tech_Note_External_Nitrogen_Cycle:overview}}
In addition to the relatively rapid cycling of nitrogen within the plant
\textendash{} litter \textendash{} soil organic matter system, CLM also represents several
processes which couple the internal nitrogen cycle to external sources
and sinks. Inputs of new mineral nitrogen are from atmospheric
deposition and biological nitrogen fixation. Losses of mineral nitrogen
are due to nitrification, denitrification, leaching, and losses in fire.
While the short-term dynamics of nitrogen limitation depend on the
behavior of the internal nitrogen cycle, establishment of total
ecosystem nitrogen stocks depends on the balance between sources and
sinks in the external nitrogen cycle ({\hyperref[\detokenize{tech_note/References/CLM50_Tech_Note_References:thomasetal2015}]{\sphinxcrossref{\DUrole{std,std-ref}{Thomas et al. 2015}}}}).

As with CLM4.5, CLM5.0 represents inorganic N transformations based on the Century N-gas model; this
includes separate NH$_{\text{4}}$$^{\text{+}}$ and
NO$_{\text{3}}$$^{\text{-}}$ pools, as well as
environmentally controlled nitrification and denitrification rates that is described below.


\subsection{Atmospheric Nitrogen Deposition}
\label{\detokenize{tech_note/External_Nitrogen_Cycle/CLM50_Tech_Note_External_Nitrogen_Cycle:atmospheric-nitrogen-deposition}}
CLM uses a single variable to represent the total deposition of mineral
nitrogen onto the land surface, combining wet and dry deposition of
NO$_{\text{y}}$ and NH$_{\text{x}}$ as a single flux
(\({NF}_{ndep\_sminn}\), gN m$^{\text{-2}}$ s$^{\text{-1}}$). This flux
is intended to represent total reactive
nitrogen deposited to the land surface which originates from the
following natural and anthropogenic sources (Galloway et al. 2004):
formation of NO$_{\text{x}}$ during lightning,
NO\({}_{x }\)and NH$_{\text{3}}$ emission from wildfire,
NO$_{\text{x}}$ emission from natural soils, NH$_{\text{3}}$
emission from natural soils, vegetation, and wild animals,
NO$_{\text{x}}$ and NH$_{\text{3}}$ emission during fossil fuel
combustion (both thermal and fuel NO$_{\text{x}}$ production),
NO$_{\text{x}}$ and NH$_{\text{3}}$ emission from other industrial
processes, NO$_{\text{x}}$ and NH$_{\text{3}}$ emission from fire
associated with deforestation, NO$_{\text{x}}$ and NH$_{\text{3}}$
emission from agricultural burning, NO$_{\text{x}}$ emission from
agricultural soils, NH$_{\text{3}}$ emission from agricultural crops,
NH$_{\text{3}}$ emission from agricultural animal waste, and
NH$_{\text{3}}$ emission from human waste and waste water. The
deposition flux is provided as a spatially and (potentially) temporally
varying dataset (see section \hyperref[\detokenize{tech_note/Ecosystem/CLM50_Tech_Note_Ecosystem:atmospheric-coupling}]{\ref{\detokenize{tech_note/Ecosystem/CLM50_Tech_Note_Ecosystem:atmospheric-coupling}}} for a description of the default
input dataset).

The nitrogen deposition flux is assumed to enter the NH$_{\text{4}}$$^{\text{+}}$ pool,
and is vertically distributed throughout the soil profile. Although N deposition
inputs include both oxidized and reduced forms, CLM5 only reads in total
N deposition. This approach is held over from CLM4.0, which only represented a
single mineral nitrogen pool, however, real pathways for wet and dry
nitrogen deposition can be more complex than currently represented in
the CLM5.0, including release from melting snowpack and direct foliar
uptake of deposited NO$_{\text{y}}$ ({\hyperref[\detokenize{tech_note/References/CLM50_Tech_Note_References:tyeetal2005}]{\sphinxcrossref{\DUrole{std,std-ref}{Tye et al. 2005}}}};
{\hyperref[\detokenize{tech_note/References/CLM50_Tech_Note_References:vallanosparks2007}]{\sphinxcrossref{\DUrole{std,std-ref}{Vallano and Sparks, 2007}}}}).

In offline (uncoupled) CLM5.0 simulations monthly
estimates of N deposition are provided, as opposed to decadal files
supplied with previous versions of the model. In coupled simulations,
N depositions fluxes are passed to the land model at the frequency of
the time step (every half hour) through the coupler.


\subsection{Biological Nitrogen Fixation}
\label{\detokenize{tech_note/External_Nitrogen_Cycle/CLM50_Tech_Note_External_Nitrogen_Cycle:biological-nitrogen-fixation}}
The fixation of new reactive nitrogen from atmospheric N$_{\text{2}}$
by soil microorganisms is an important component of both preindustrial
and modern-day nitrogen budgets, but a mechanistic understanding of
global-scale controls on biological nitrogen fixation (BNF) is still
only poorly developed ({\hyperref[\detokenize{tech_note/References/CLM50_Tech_Note_References:clevelandetal1999}]{\sphinxcrossref{\DUrole{std,std-ref}{Cleveland et al. 1999}}}};
{\hyperref[\detokenize{tech_note/References/CLM50_Tech_Note_References:gallowayetal2004}]{\sphinxcrossref{\DUrole{std,std-ref}{Galloway et al. 2004}}}}). CLM5.0 uses the FUN
model (chapter \hyperref[\detokenize{tech_note/FUN/CLM50_Tech_Note_FUN:rst-fun}]{\ref{\detokenize{tech_note/FUN/CLM50_Tech_Note_FUN:rst-fun}}}) to
calculate the carbon cost and nitrogen acquired through symbotic
nitrogen fixation. This nitrogen is immediately available to plants.

{\hyperref[\detokenize{tech_note/References/CLM50_Tech_Note_References:clevelandetal1999}]{\sphinxcrossref{\DUrole{std,std-ref}{Cleveland et al. (1999)}}}} suggested
an empirical relationships that predicts BNF as a function of
either evapotranspiration rate or net primary productivity for
natural vegetation. CLM5.0 adopts the evapotranspiration approach
to calculate asymbiotic, or free-living, N fixation. This function
has been modified from the {\hyperref[\detokenize{tech_note/References/CLM50_Tech_Note_References:clevelandetal1999}]{\sphinxcrossref{\DUrole{std,std-ref}{Cleveland et al. (1999)}}}} estimates to provide lower estimate of
free-living nitrogen fixation in CLM5.0
(\({CF}_{ann\_ET}\), mm yr$^{\text{-1}}$).
This moves away from the NPP approach used in CLM4.0 and 4.5 and
avoids unrealistically increasing freeliving rates of N fixation
under global change scenarios ({\hyperref[\detokenize{tech_note/References/CLM50_Tech_Note_References:wiederetal2015}]{\sphinxcrossref{\DUrole{std,std-ref}{Wieder et al. 2015}}}} The expression used is:
\phantomsection\label{\detokenize{tech_note/External_Nitrogen_Cycle/CLM50_Tech_Note_External_Nitrogen_Cycle:equation-22.1)}}\begin{equation}\label{equation:tech_note/External_Nitrogen_Cycle/CLM50_Tech_Note_External_Nitrogen_Cycle:22.1)}
\begin{split}NF_{nfix,sminn} ={0.0006\left(0.0117+CF_{ann\_ ET}\right)\mathord{\left/ {\vphantom {0.0006\left(0.0117+ CF_{ann\_ ET}\right) \left(86400\cdot 365\right)}} \right. \kern-\nulldelimiterspace} \left(86400\cdot 365\right)}\end{split}
\end{equation}
Where \({NF}_{nfix,sminn}\) (gN m$^{\text{-2}}$ s$^{\text{-1}}$) is the rate of free-living nitrogen fixation in \hyperref[\detokenize{tech_note/External_Nitrogen_Cycle/CLM50_Tech_Note_External_Nitrogen_Cycle:figure-biological-nitrogen-fixation}]{Figure \ref{\detokenize{tech_note/External_Nitrogen_Cycle/CLM50_Tech_Note_External_Nitrogen_Cycle:figure-biological-nitrogen-fixation}}}.

\begin{figure}[htbp]
\centering
\capstart

\noindent\sphinxincludegraphics{{image11}.png}
\caption{Free-living nitrogen fixation as a function of annual evapotranspiration. Results here show annual N inputs from free-living N fixations, but the model actually calculates inputs on a per second basis.}\label{\detokenize{tech_note/External_Nitrogen_Cycle/CLM50_Tech_Note_External_Nitrogen_Cycle:figure-biological-nitrogen-fixation}}\label{\detokenize{tech_note/External_Nitrogen_Cycle/CLM50_Tech_Note_External_Nitrogen_Cycle:id2}}\end{figure}

As with Atmospheric N deposition, free-living N inputs are added directly to the
NH$_{\text{4}}$$^{\text{+}}$ pool.


\subsection{Nitrification and Denitrification Losses of Nitrogen}
\label{\detokenize{tech_note/External_Nitrogen_Cycle/CLM50_Tech_Note_External_Nitrogen_Cycle:nitrification-and-denitrification-losses-of-nitrogen}}
Nitrification is an autotrophic process that converts less mobile ammonium
ions into nitrate, that can more easily be lost from soil systems by leaching
or denitrification.  The process catalyzed by ammonia oxidizing archaea and
bacteria that convert ammonium (NH$_{\text{4}}$$^{\text{+}}$) into nitrite, which
is subsequently oxidized into nitrate (NO$_{\text{3}}$$^{\text{-}}$). Conditions
favoring nitrification include high NH$_{\text{4}}$$^{\text{+}}$ concentrations,
well aerated soils, a neutral pH and warmer temperatures.

Under aerobic conditions in the soil oxygen is the preferred electron
acceptor supporting the metabolism of heterotrophs, but anaerobic
conditions favor the activity of soil heterotrophs which use nitrate as
an electron acceptor (e.g. \sphinxstyleemphasis{Pseudomonas} and \sphinxstyleemphasis{Clostridium}) supporting
respiration. This process, known as denitrification, results in the
transformation of nitrate to gaseous N$_{\text{2}}$, with smaller
associated production of NO$_{\text{x}}$ and N$_{\text{2}}$O. It
is typically assumed that nitrogen fixation and denitrification
were approximately balanced in the preindustrial biosphere (
{\hyperref[\detokenize{tech_note/References/CLM50_Tech_Note_References:gallowayetal2004}]{\sphinxcrossref{\DUrole{std,std-ref}{Galloway et al. 2004}}}}). It is likely
that denitrification can occur within anaerobic
microsites within an otherwise aerobic soil environment, leading to
large global denitrification fluxes even when fluxes per unit area are
rather low ({\hyperref[\detokenize{tech_note/References/CLM50_Tech_Note_References:gallowayetal2004}]{\sphinxcrossref{\DUrole{std,std-ref}{Galloway et al. 2004}}}}).

CLM includes a detailed representation of nitrification and
denitrification based on the Century N model ({\hyperref[\detokenize{tech_note/References/CLM50_Tech_Note_References:partonetal1996}]{\sphinxcrossref{\DUrole{std,std-ref}{Parton
et al. 1996}}}}, {\hyperref[\detokenize{tech_note/References/CLM50_Tech_Note_References:partonetal2001}]{\sphinxcrossref{\DUrole{std,std-ref}{2001}}}};
{\hyperref[\detokenize{tech_note/References/CLM50_Tech_Note_References:delgrossoetal2000}]{\sphinxcrossref{\DUrole{std,std-ref}{del Grosso et al. 2000}}}}). In this
approach, nitrification of NH$_{\text{4}}$$^{\text{+}}$ to NO$_{\text{3}}$$^{\text{-}}$
is a function of temperature, moisture, and pH:
\phantomsection\label{\detokenize{tech_note/External_Nitrogen_Cycle/CLM50_Tech_Note_External_Nitrogen_Cycle:equation-22.2)}}\begin{equation}\label{equation:tech_note/External_Nitrogen_Cycle/CLM50_Tech_Note_External_Nitrogen_Cycle:22.2)}
\begin{split}f_{nitr,p} =\left[NH_{4} \right]k_{nitr} f\left(T\right)f\left(H_{2} O\right)f\left(pH\right)\end{split}
\end{equation}
where \({f}_{nitr,p}\) is the potential nitrification rate
(prior to competition for NH$_{\text{4}}$$^{\text{+}}$ by plant
uptake and N immobilization), \({k}_{nitr}\) is the maximum
nitrification rate (10 \% day\(\mathrm{-}\)1,
({\hyperref[\detokenize{tech_note/References/CLM50_Tech_Note_References:partonetal2001}]{\sphinxcrossref{\DUrole{std,std-ref}{Parton et al. 2001}}}}), and \sphinxstyleemphasis{f(T)} and
\sphinxstyleemphasis{f(H)}$_{\text{2}}$O) are rate modifiers for temperature and
moisture content. CLM uses the same rate modifiers as
are used in the decomposition routine. \sphinxstyleemphasis{f(pH)} is a rate
modifier for pH; however, because CLM does not calculate pH,
instead a fixed pH value of 6.5 is used in the pH function of
{\hyperref[\detokenize{tech_note/References/CLM50_Tech_Note_References:partonetal1996}]{\sphinxcrossref{\DUrole{std,std-ref}{Parton et al. (1996)}}}}.

The potential denitrification rate is co-limited by
NO$^{\text{-3}}$ concentration and C consumption rates, and occurs only in the anoxic fraction of soils:
\phantomsection\label{\detokenize{tech_note/External_Nitrogen_Cycle/CLM50_Tech_Note_External_Nitrogen_Cycle:equation-22.3)}}\begin{equation}\label{equation:tech_note/External_Nitrogen_Cycle/CLM50_Tech_Note_External_Nitrogen_Cycle:22.3)}
\begin{split}f_{denitr,p} =\min \left(f(decomp),f\left(\left[NO_{3} ^{-} \right]\right)\right)frac_{anox}\end{split}
\end{equation}
where \({f}_{denitr,p}\) is the potential denitrification rate
and \sphinxstyleemphasis{f(decomp)} and \sphinxstyleemphasis{f({[}NO}$_{\text{3}}$$^{\text{-}}$ \sphinxstyleemphasis{{]})}
are the carbon- and nitrate- limited denitrification rate functions,
respectively, ({\hyperref[\detokenize{tech_note/References/CLM50_Tech_Note_References:delgrossoetal2000}]{\sphinxcrossref{\DUrole{std,std-ref}{del Grosso et al. 2000}}}}).
Because the modified CLM includes explicit treatment of soil
biogeochemical vertical profiles, including diffusion of the trace
gases O$_{\text{2}}$ and CH$_{\text{4}}$ ({\hyperref[\detokenize{tech_note/References/CLM50_Tech_Note_References:rileyetal2011a}]{\sphinxcrossref{\DUrole{std,std-ref}{Riley et al. 2011a}}}}), the calculation of anoxic fraction  \({frac}_{anox}\)
uses this information following the anoxic microsite formulation
of {\hyperref[\detokenize{tech_note/References/CLM50_Tech_Note_References:arahvinten1995}]{\sphinxcrossref{\DUrole{std,std-ref}{Arah and Vinten (1995)}}}}.
\phantomsection\label{\detokenize{tech_note/External_Nitrogen_Cycle/CLM50_Tech_Note_External_Nitrogen_Cycle:equation-22.4)}}\begin{equation}\label{equation:tech_note/External_Nitrogen_Cycle/CLM50_Tech_Note_External_Nitrogen_Cycle:22.4)}
\begin{split}frac_{anox} =\exp \left(-aR_{\psi }^{-\alpha } V^{-\beta } C^{\gamma } \left[\theta +\chi \varepsilon \right]^{\delta } \right)\end{split}
\end{equation}
where \sphinxstyleemphasis{a}, \(\alpha\), \(\beta\), \(\gamma\), and \(\delta\) are constants (equal to
1.5x10$^{\text{-10}}$, 1.26, 0.6, 0.6, and 0.85, respectively), \({R}_{\psi}\) is the
radius of a typical pore space at moisture content \(\psi\), \sphinxstyleemphasis{V}
is the O$_{\text{2}}$ consumption rate, \sphinxstyleemphasis{C} is the O$_{\text{2}}$
concentration, \(\theta\) is the water-filled pore space,
\(\chi\) is the ratio of diffusivity of oxygen in water to that in
air, and \(\epsilon\) is the air-filled pore space ({\hyperref[\detokenize{tech_note/References/CLM50_Tech_Note_References:arahvinten1995}]{\sphinxcrossref{\DUrole{std,std-ref}{Arah and
Vinten (1995)}}}}). These parameters are all calculated
separately at each
layer to define a profile of anoxic porespace fraction in the soil.

The nitrification/denitrification models used here also predict fluxes
of N$_{\text{2}}$O via a “hole-in-the-pipe” approach ({\hyperref[\detokenize{tech_note/References/CLM50_Tech_Note_References:firestonedavidson1989}]{\sphinxcrossref{\DUrole{std,std-ref}{Firestone and
Davidson, 1989}}}}). A constant fraction
(6 * 10\({}^{-4}\), {\hyperref[\detokenize{tech_note/References/CLM50_Tech_Note_References:lietal2000}]{\sphinxcrossref{\DUrole{std,std-ref}{Li et al. 2000}}}}) of the
nitrification flux is assumed to be N$_{\text{2}}$O, while the fraction
of denitrification going to N$_{\text{2}}$O, \({P}_{N2:N2O}\), is variable, following
the Century ({\hyperref[\detokenize{tech_note/References/CLM50_Tech_Note_References:delgrossoetal2000}]{\sphinxcrossref{\DUrole{std,std-ref}{del Grosso et al. 2000}}}}) approach:
\phantomsection\label{\detokenize{tech_note/External_Nitrogen_Cycle/CLM50_Tech_Note_External_Nitrogen_Cycle:equation-22.5)}}\begin{equation}\label{equation:tech_note/External_Nitrogen_Cycle/CLM50_Tech_Note_External_Nitrogen_Cycle:22.5)}
\begin{split}P_{N_{2} :N_{2} O} =\max \left(0.16k_{1} ,k_{1} \exp \left(-0.8P_{NO_{3} :CO_{2} } \right)\right)f_{WFPS}\end{split}
\end{equation}
where \({P}_{NO3:CO2}\) is the ratio of CO$_{\text{2}}$
production in a given soil layer to the
NO$_{\text{3}}$$^{\text{-{}`}}$ concentration, \({k}_{1}\) is
a function of \({d}_{g}\), the gas diffusivity through the soil
matrix:
\phantomsection\label{\detokenize{tech_note/External_Nitrogen_Cycle/CLM50_Tech_Note_External_Nitrogen_Cycle:equation-22.6)}}\begin{equation}\label{equation:tech_note/External_Nitrogen_Cycle/CLM50_Tech_Note_External_Nitrogen_Cycle:22.6)}
\begin{split}k_{1} =\max \left(1.7,38.4-350*d_{g} \right)\end{split}
\end{equation}
and \({f}_{WFPS}\) is a function of the water filled pore space \sphinxstyleemphasis{WFPS:}
\phantomsection\label{\detokenize{tech_note/External_Nitrogen_Cycle/CLM50_Tech_Note_External_Nitrogen_Cycle:equation-22.16)}}\begin{equation}\label{equation:tech_note/External_Nitrogen_Cycle/CLM50_Tech_Note_External_Nitrogen_Cycle:22.16)}
\begin{split}f_{WFPS} =\max \left(0.1,0.015\times WFPS-0.32\right)\end{split}
\end{equation}

\subsection{Leaching Losses of Nitrogen}
\label{\detokenize{tech_note/External_Nitrogen_Cycle/CLM50_Tech_Note_External_Nitrogen_Cycle:leaching-losses-of-nitrogen}}
Soil mineral nitrogen remaining after plant uptake, immobilization, and
denitrification is subject to loss as a dissolved component of
hydrologic outflow from the soil column (leaching). This leaching loss
(\({NF}_{leached}\), gN m$^{\text{-2}}$ s$^{\text{-1}}$)
depends on the concentration of dissolved mineral (inorganic) nitrogen
in soil water solution (\sphinxstyleemphasis{DIN}, gN kgH$_{\text{2}}$O), and the rate
of hydrologic discharge from the soil column to streamflow
(\({Q}_{dis}\), kgH$_{\text{2}}$O m$^{\text{-2}}$
s$^{\text{-1}}$, section \hyperref[\detokenize{tech_note/Hydrology/CLM50_Tech_Note_Hydrology:lateral-sub-surface-runoff}]{\ref{\detokenize{tech_note/Hydrology/CLM50_Tech_Note_Hydrology:lateral-sub-surface-runoff}}}), as
\phantomsection\label{\detokenize{tech_note/External_Nitrogen_Cycle/CLM50_Tech_Note_External_Nitrogen_Cycle:equation-22.17)}}\begin{equation}\label{equation:tech_note/External_Nitrogen_Cycle/CLM50_Tech_Note_External_Nitrogen_Cycle:22.17)}
\begin{split}NF_{leached} =DIN\cdot Q_{dis} .\end{split}
\end{equation}
\sphinxstyleemphasis{DIN} is calculated assuming that a constant fraction (\sphinxstyleemphasis{sf}, proportion)
of the remaining soil mineral N pool is in soluble form, and that this
entire fraction is dissolved in the total soil water.  For the Century-
based formulation in CLM5.0, the leaching acts only on the
NO$_{\text{3}}$$^{\text{-{}`}}$ pool (which is assumed to be 100\%
soluble), while the NH$_{\text{4}}$$^{\text{+}}$ pool is assumed
to be 100\% adsorbed onto mineral surfaces and unaffected by leaching.
\sphinxstyleemphasis{DIN} is then given as
\phantomsection\label{\detokenize{tech_note/External_Nitrogen_Cycle/CLM50_Tech_Note_External_Nitrogen_Cycle:equation-22.18)}}\begin{equation}\label{equation:tech_note/External_Nitrogen_Cycle/CLM50_Tech_Note_External_Nitrogen_Cycle:22.18)}
\begin{split}DIN=\frac{NS_{sminn} sf}{WS_{tot\_ soil} }\end{split}
\end{equation}
where \({WS}_{tot\_soil}\) (kgH:sub:\sphinxtitleref{2}O m$^{\text{-2}}$) is the total mass of soil water content integrated
over the column. The total mineral nitrogen leaching flux is limited on
each time step to not exceed the soluble fraction of \({NS}_{sminn}\)
\phantomsection\label{\detokenize{tech_note/External_Nitrogen_Cycle/CLM50_Tech_Note_External_Nitrogen_Cycle:equation-22.19)}}\begin{equation}\label{equation:tech_note/External_Nitrogen_Cycle/CLM50_Tech_Note_External_Nitrogen_Cycle:22.19)}
\begin{split}NF_{leached} =\min \left(NF_{leached} ,\frac{NS_{sminn} sf}{\Delta t} \right).\end{split}
\end{equation}

\subsection{Losses of Nitrogen Due to Fire}
\label{\detokenize{tech_note/External_Nitrogen_Cycle/CLM50_Tech_Note_External_Nitrogen_Cycle:losses-of-nitrogen-due-to-fire}}
The final pathway for nitrogen loss is through combustion, also known as
pyrodenitrification. Detailed equations are provided, together with the
effects of fire on the carbon budget, in Chapter \hyperref[\detokenize{tech_note/Fire/CLM50_Tech_Note_Fire:rst-fire}]{\ref{\detokenize{tech_note/Fire/CLM50_Tech_Note_Fire:rst-fire}}}). It is assumed in
CLM-CN that losses of N due to fire are restricted to vegetation and
litter pools (including coarse woody debris). Loss rates of N are
determined by the fraction of biomass lost to combustion, assuming that
most of the nitrogen in the burned biomass is lost to the atmosphere
({\hyperref[\detokenize{tech_note/References/CLM50_Tech_Note_References:schlesinger1997}]{\sphinxcrossref{\DUrole{std,std-ref}{Schlesinger, 1997}}}}; {\hyperref[\detokenize{tech_note/References/CLM50_Tech_Note_References:smithetal2005}]{\sphinxcrossref{\DUrole{std,std-ref}{Smith et al. 2005}}}}). It is assumed that soil organic
matter pools of carbon and nitrogen are not directly affected by fire
({\hyperref[\detokenize{tech_note/References/CLM50_Tech_Note_References:neffetal2005}]{\sphinxcrossref{\DUrole{std,std-ref}{Neff et al. 2005}}}}).


\section{Fire}
\label{\detokenize{tech_note/Fire/CLM50_Tech_Note_Fire:rst-fire}}\label{\detokenize{tech_note/Fire/CLM50_Tech_Note_Fire:fire}}\label{\detokenize{tech_note/Fire/CLM50_Tech_Note_Fire::doc}}
The fire parameterization in CLM contains four components: non-peat
fires outside cropland and tropical closed forests, agricultural fires
in cropland, deforestation fires in the tropical closed forests, and
peat fires (see {\hyperref[\detokenize{tech_note/References/CLM50_Tech_Note_References:lietal2012a}]{\sphinxcrossref{\DUrole{std,std-ref}{Li et al. 2012a}}}},
{\hyperref[\detokenize{tech_note/References/CLM50_Tech_Note_References:lietal2012b}]{\sphinxcrossref{\DUrole{std,std-ref}{Li et al. 2012b}}}}, {\hyperref[\detokenize{tech_note/References/CLM50_Tech_Note_References:lietal2013a}]{\sphinxcrossref{\DUrole{std,std-ref}{Li et al. 2013}}}},
{\hyperref[\detokenize{tech_note/References/CLM50_Tech_Note_References:lilawrence2017}]{\sphinxcrossref{\DUrole{std,std-ref}{Li and Lawrence 2017}}}} for details).
In this fire parameterization, burned area is affected by climate and
weather conditions, vegetation composition and structure, and human
activities. After burned area is calculated, we estimate the fire impact,
including biomass and peat burning, fire-induced vegetation mortality,
adjustment of the carbon and nitrogen (C/N) pools, and fire emissions.


\subsection{Non-peat fires outside cropland and tropical closed forest}
\label{\detokenize{tech_note/Fire/CLM50_Tech_Note_Fire:non-peat-fires-outside-cropland-and-tropical-closed-forest}}\label{\detokenize{tech_note/Fire/CLM50_Tech_Note_Fire:id1}}
Burned area in a grid cell, \(A_{b}\) (km$^{\text{2}}$ s $^{\text{-1}}$),
is determined by
\phantomsection\label{\detokenize{tech_note/Fire/CLM50_Tech_Note_Fire:equation-23.1}}\begin{equation}\label{equation:tech_note/Fire/CLM50_Tech_Note_Fire:23.1}
\begin{split}A_{b} =N_{f} a\end{split}
\end{equation}
where \(N_{f}\)  (count s$^{\text{-1}}$) is fire
counts in the grid cell; \(a\) (km$^{\text{2}}$) is average fire
spread area of a fire.


\subsubsection{Fire counts}
\label{\detokenize{tech_note/Fire/CLM50_Tech_Note_Fire:fire-counts}}\label{\detokenize{tech_note/Fire/CLM50_Tech_Note_Fire:id2}}
Fire counts \(N_{f}\)  is taken as
\phantomsection\label{\detokenize{tech_note/Fire/CLM50_Tech_Note_Fire:equation-23.2}}\begin{equation}\label{equation:tech_note/Fire/CLM50_Tech_Note_Fire:23.2}
\begin{split}N_{f} = N_{i} f_{b} f_{m} f_{se,o}\end{split}
\end{equation}
where \(N_{i}\)  ( count s$^{\text{-1}}$) is the
number of ignition sources due to natural causes and human activities;
\(f_{b}\)  and \(f_{m}\)  (fractions) represent the availability
and combustibility of fuel, respectively; \(f_{se,o}\)  is the
fraction of anthropogenic and natural fires unsuppressed by humans and
related to the socioeconomic conditions.

\(N_{i}\)  (count s$^{\text{-1}}$) is given as
\phantomsection\label{\detokenize{tech_note/Fire/CLM50_Tech_Note_Fire:equation-23.3}}\begin{equation}\label{equation:tech_note/Fire/CLM50_Tech_Note_Fire:23.3}
\begin{split}N_{i} = \left(I_{n} +I_{a} \right) A_{g}\end{split}
\end{equation}
where \(I_{n}\) (count km$^{\text{-2}}$ s$^{\text{-1}}$) and \(I_{a}\)
(count km$^{\text{-2}}$ s$^{\text{-1}}$) are the number of natural and anthropogenic
ignitions per km$^{\text{2}}$, respectively; \(A_{g}\) is the area of the
grid cell (km$^{\text{2}}$). \(I_{n}\)  is estimated by
\phantomsection\label{\detokenize{tech_note/Fire/CLM50_Tech_Note_Fire:equation-23.4}}\begin{equation}\label{equation:tech_note/Fire/CLM50_Tech_Note_Fire:23.4}
\begin{split}I_{n} = \gamma \psi I_{l}\end{split}
\end{equation}
where \(\gamma\) =0.22 is ignition efficiency of cloud-to-ground
lightning; \(\psi =\frac{1}{5.16+2.16\cos [3min(60,\lambda )]}\)  is the
cloud-to-ground lightning fraction and depends on the latitude
\(\lambda\) (degrees) ; \(I_{l}\)  (flash km$^{\text{-2}}$ s$^{\text{-1}}$) is
the total lightning flashes. \(I_{a}\) is modeled as a monotonic
increasing function of population density:
\phantomsection\label{\detokenize{tech_note/Fire/CLM50_Tech_Note_Fire:equation-23.5}}\begin{equation}\label{equation:tech_note/Fire/CLM50_Tech_Note_Fire:23.5}
\begin{split}I_{a} =\frac{\alpha D_{P} k(D_{P} )}{n}\end{split}
\end{equation}
where \(\alpha =0.01\) (count person$^{\text{-1}}$ mon$^{\text{-1}}$) is the number of potential ignition sources by a
person per month; \(D_{P}\)  (person km$^{\text{-2}}$) is the population density; \(k(D_{P} )=6.8D_{P} ^{-0.6}\)  represents anthropogenic ignition
potential as a function of human population density \(D_{P}\) ; \sphinxstyleemphasis{n}
is the seconds in a month.

Fuel availability \(f_{b}\) is given as
\phantomsection\label{\detokenize{tech_note/Fire/CLM50_Tech_Note_Fire:equation-23.6}}\begin{equation}\label{equation:tech_note/Fire/CLM50_Tech_Note_Fire:23.6}
\begin{split}f_{b} =\left\{\begin{array}{c}
{0} \\ {\frac{B_{ag} -B_{low} }{B_{up} -B_{low} } } \\ {1} \end{array}
\begin{array}{cc} {} & {} \end{array}\begin{array}{c} {B_{ag} <B_{low} } \\ {\begin{array}{cc} {} & {} \end{array}B_{low} \le B_{ag} \le B_{up} } \\ {B_{ag} >B_{up} }
\end{array}\right\} \ ,\end{split}
\end{equation}
where \(B_{ag}\)  (g C m$^{\text{-2}}$) is the biomass of combined leaf,
stem, litter, and woody debris pools; \(B_{low}\)  = 105 g C m $^{\text{-2}}$
is the lower fuel threshold below which fire does not occur; \(B_{up}\)
= 1050 g C m$^{\text{-2}}$ is the upper fuel threshold above which fire
occurrence is not limited by fuel availability.

Fuel combustibility \(f_{m}\) is estimated by
\phantomsection\label{\detokenize{tech_note/Fire/CLM50_Tech_Note_Fire:equation-23.7}}\begin{equation}\label{equation:tech_note/Fire/CLM50_Tech_Note_Fire:23.7}
\begin{split}f_{m} = {f_{RH} f_{\beta}}, &\qquad T_{17cm} > T_{f}\end{split}
\end{equation}
where \(f_{RH}\) and \(f_{\beta }\) represent the dependence of
fuel combustibility on relative humidity \(RH\) (\%) and root-zone
soil moisture limitation \(\beta\) (fraction); \(T_{17cm}\) is
the temperature of the top 17 cm of soil (K) and \(T_{f}\) is the
freezing temperature. \(f_{RH}\) is a weighted average of real time
\(RH\) (\(RH_{0}\)) and 30-day running mean \(RH\)
(\(RH_{30d}\)):
\phantomsection\label{\detokenize{tech_note/Fire/CLM50_Tech_Note_Fire:equation-23.8}}\begin{equation}\label{equation:tech_note/Fire/CLM50_Tech_Note_Fire:23.8}
\begin{split}f_{RH} = (1-w) l_{RH_{0}} + wl_{RH_{30d}}\end{split}
\end{equation}
where weight \(w=\max [0,\min (1,\frac{B_{ag}-2500}{2500})]\),
\(l_{{RH}_{0}}=1-\max [0,\min (1,\frac{RH_{0}-30}{80-30})]\), and
\(l_{{RH}_{30d}}=1-\max [0.75,\min (1,\frac{RH_{30d}}{90})]\).
\(f_{\beta}\) is given by
\phantomsection\label{\detokenize{tech_note/Fire/CLM50_Tech_Note_Fire:equation-23.9}}\begin{equation}\label{equation:tech_note/Fire/CLM50_Tech_Note_Fire:23.9}
\begin{split}f_{\beta } =\left\{\begin{array}{cccc}
{1} & {} & {} & {\beta\le \beta_{low} } \\ {\frac{\beta_{up} -\beta}{\beta_{up} -\beta_{low} } } & {} & {} & {\beta_{low} <\beta<\beta_{up} } \\
{0} & {} & {} & {\beta\ge \beta_{up} }
\end{array}\right\} \ ,\end{split}
\end{equation}
where \(\beta _{low}\) =0.85 and \(\beta _{up}\) =0.98 are the
lower and upper thresholds, respectively.

For scarcely populated regions (\(D_{p} \le 0.1\) person
km $^{\text{-2}}$), we assume that anthropogenic suppression on fire
occurrence is negligible, i.e., \(f_{se,o} =1.0\). In regions of
\(D_{p} >0.1\) person km$^{\text{-2}}$, we parameterize the
fraction of anthropogenic and natural fires unsuppressed by human
activities as
\phantomsection\label{\detokenize{tech_note/Fire/CLM50_Tech_Note_Fire:equation-23.10}}\begin{equation}\label{equation:tech_note/Fire/CLM50_Tech_Note_Fire:23.10}
\begin{split}f_{se,o} =f_{d} f_{e}\end{split}
\end{equation}
where \({f}_{d}\) and \({f}_{e}\) are the effects of the
demographic and economic conditions on fire occurrence. The demographic
influence on fire occurrence is
\phantomsection\label{\detokenize{tech_note/Fire/CLM50_Tech_Note_Fire:equation-23.11}}\begin{equation}\label{equation:tech_note/Fire/CLM50_Tech_Note_Fire:23.11}
\begin{split}f_{d} =0.01 + 0.98 \exp (-0.025D_{P} ).\end{split}
\end{equation}
For shrub and grass PFTs, the economic influence on fire occurrence is
parameterized as a function of Gross Domestic Product GDP (k 1995US\$
capita$^{\text{-1}}$):
\phantomsection\label{\detokenize{tech_note/Fire/CLM50_Tech_Note_Fire:equation-23.12}}\begin{equation}\label{equation:tech_note/Fire/CLM50_Tech_Note_Fire:23.12}
\begin{split}f_{e} =0.1+0.9\times \exp [-\pi (\frac{GDP}{8} )^{0.5} ]\end{split}
\end{equation}
which captures 73\% of the observed MODIS fire counts with variable GDP
in regions where shrub and grass PFTs are dominant (fractional coverage
of shrub and grass PFTs \(>\) 50\%). In regions outside tropical
closed forests and dominated by trees (fractional coverage of tree PFTs
\(>\) 50\%), we use
\phantomsection\label{\detokenize{tech_note/Fire/CLM50_Tech_Note_Fire:equation-23.13}}\begin{equation}\label{equation:tech_note/Fire/CLM50_Tech_Note_Fire:23.13}
\begin{split}f_{e} =\left\{\begin{array}{c}
{0.39} \\ {0.79} \\ {1} \end{array}
\begin{array}{cc} {} & {} \end{array}\begin{array}{c} {GDP > 20 } \\
{ 8 < GDP \le 20 } \\  { GDP \le 8 }
\end{array}\right\} \ ,\end{split}
\end{equation}
to reproduce the relationship between MODIS fire counts and GDP.


\subsubsection{Average spread area of a fire}
\label{\detokenize{tech_note/Fire/CLM50_Tech_Note_Fire:id3}}\label{\detokenize{tech_note/Fire/CLM50_Tech_Note_Fire:average-spread-area-of-a-fire}}
Fire fighting capacity depends on socioeconomic conditions and affects
fire spread area. Due to a lack of observations, we consider the
socioeconomic impact on the average burned area rather than separately
on fire spread rate and fire duration:
\phantomsection\label{\detokenize{tech_note/Fire/CLM50_Tech_Note_Fire:equation-23.14}}\begin{equation}\label{equation:tech_note/Fire/CLM50_Tech_Note_Fire:23.14}
\begin{split}a=a^{*} F_{se}\end{split}
\end{equation}
where \(a^{*}\)  is the average burned area of a fire without
anthropogenic suppression and \(F_{se}\)  is the socioeconomic
effect on fire spread area.

Average burned area of a fire without anthropogenic suppression is
assumed elliptical in shape with the wind direction along the major axis
and the point of ignition at one of the foci. According to the area
formula for an ellipse, average burned area of a fire can be represented
as:
\phantomsection\label{\detokenize{tech_note/Fire/CLM50_Tech_Note_Fire:equation-23.15}}\begin{equation}\label{equation:tech_note/Fire/CLM50_Tech_Note_Fire:23.15}
\begin{split}a^{*} =\pi \frac{l}{2} \frac{w}{2} \times 10^{-6} =\frac{\pi u_{p}^{2} \tau ^{2} }{4L_{B} } (1+\frac{1}{H_{B} } )^{2} \times 10^{-6}\end{split}
\end{equation}
where \(u_{p}\) (m s$^{\text{-1}}$) is the fire spread rate in the
downwind direction; \(\tau\) (s) is average fire duration; \(L_{B}\)
and \(H_{B}\)  are length-to-breadth ratio and head-to-back ratio of
the ellipse; 10 $^{\text{-6}}$ converts m $^{\text{2}}$ to km $^{\text{2}}$.

According to {\hyperref[\detokenize{tech_note/References/CLM50_Tech_Note_References:aroraboer2005}]{\sphinxcrossref{\DUrole{std,std-ref}{Arora and Boer (2005)}}}},
\phantomsection\label{\detokenize{tech_note/Fire/CLM50_Tech_Note_Fire:equation-23.16}}\begin{equation}\label{equation:tech_note/Fire/CLM50_Tech_Note_Fire:23.16}
\begin{split}L_{B} =1.0+10.0[1-\exp (-0.06W)]\end{split}
\end{equation}
where \(W\)(m s$^{\text{-1}}$) is the wind speed. According to
the mathematical properties of the ellipse, the head-to-back ratio
\(H_{B}\)  is
\phantomsection\label{\detokenize{tech_note/Fire/CLM50_Tech_Note_Fire:equation-23.17}}\begin{equation}\label{equation:tech_note/Fire/CLM50_Tech_Note_Fire:23.17}
\begin{split}H_{B} =\frac{u_{p} }{u_{b} } =\frac{L_{B} +(L_{B} ^{2} -1)^{0.5} }{L_{B} -(L_{B} ^{2} -1)^{0.5} } .\end{split}
\end{equation}
The fire spread rate in the downwind direction is represented as
\phantomsection\label{\detokenize{tech_note/Fire/CLM50_Tech_Note_Fire:equation-23.18}}\begin{equation}\label{equation:tech_note/Fire/CLM50_Tech_Note_Fire:23.18}
\begin{split}u_{p} =u_{\max } C_{m} g(W)\end{split}
\end{equation}
({\hyperref[\detokenize{tech_note/References/CLM50_Tech_Note_References:aroraboer2005}]{\sphinxcrossref{\DUrole{std,std-ref}{Arora and Boer, 2005}}}}), where \(u_{\max }\)
(m s$^{\text{-1}}$) is the PFT-dependent average maximum fire spread
rate in natural vegetation regions; \(C_{m} =\sqrt{f_{m}}\)  and \(g(W)\)
represent the dependence of \(u_{p}\)  on fuel wetness and wind
speed \(W\), respectively. \(u_{\max }\)  is set to 0.33
m s $^{\text{-1}}$for grass PFTs, 0.28 m s $^{\text{-1}}$ for shrub PFTs, 0.26
m s$^{\text{-1}}$ for needleleaf tree PFTs, and 0.25 m s$^{\text{-1}}$ for
other tree PFTs. \(g(W)\) is derived from the mathematical properties
of the ellipse and equation \eqref{equation:tech_note/Fire/CLM50_Tech_Note_Fire:23.16} and \eqref{equation:tech_note/Fire/CLM50_Tech_Note_Fire:23.17}.
\phantomsection\label{\detokenize{tech_note/Fire/CLM50_Tech_Note_Fire:equation-23.19}}\begin{equation}\label{equation:tech_note/Fire/CLM50_Tech_Note_Fire:23.19}
\begin{split}g(W)=\frac{2L_{B} }{1+\frac{1}{H_{B} } } g(0).\end{split}
\end{equation}
Since g(\sphinxstyleemphasis{W})=1.0, and \(L_{B}\)  and \(H_{B}\)  are at their
maxima \(L_{B} ^{\max } =11.0\) and \(H_{B} ^{\max } =482.0\)
when \(W\to \infty\) , g(0) can be derived as
\phantomsection\label{\detokenize{tech_note/Fire/CLM50_Tech_Note_Fire:equation-23.20}}\begin{equation}\label{equation:tech_note/Fire/CLM50_Tech_Note_Fire:23.20}
\begin{split}g(0)=\frac{1+\frac{1}{H_{B} ^{\max } } }{2L_{B} ^{\max } } =0.05.\end{split}
\end{equation}
In the absence of globally gridded data on barriers to fire (e.g.
rivers, lakes, roads, firebreaks) and human fire-fighting efforts,
average fire duration is simply assumed equal to 1 which is the observed
2001\textendash{}2004 mean persistence of most fires in the world
({\hyperref[\detokenize{tech_note/References/CLM50_Tech_Note_References:giglioetal2006}]{\sphinxcrossref{\DUrole{std,std-ref}{Giglio et al. 2006}}}}).

As with the socioeconomic influence on fire occurrence, we assume that
the socioeconomic influence on fire spreading is negligible in regions
of \(D_{p} \le 0.1\) person km$^{\text{-2}}$, i.e.,
\(F_{se} = 1.0\). In regions of \(D_{p} >0.1\) person
km$^{\text{-2}}$, we parameterize such socioeconomic influence as:
\phantomsection\label{\detokenize{tech_note/Fire/CLM50_Tech_Note_Fire:equation-23.21}}\begin{equation}\label{equation:tech_note/Fire/CLM50_Tech_Note_Fire:23.21}
\begin{split}F_{se} =F_{d} F_{e}\end{split}
\end{equation}
where \({F}_{d}\) and \({F}_{e}\) are
effects of the demographic and economic conditions on the average spread
area of a fire, and are identified by maximizing the explained
variability of the GFED3 burned area fraction with both socioeconomic
indices in grid cells with various dominant vegetation types. For shrub
and grass PFTs, the demographic impact factor is
\phantomsection\label{\detokenize{tech_note/Fire/CLM50_Tech_Note_Fire:equation-23.22}}\begin{equation}\label{equation:tech_note/Fire/CLM50_Tech_Note_Fire:23.22}
\begin{split}F_{d} =0.2+0.8\times \exp [-\pi (\frac{D_{p} }{450} )^{0.5} ]\end{split}
\end{equation}
and the economic impact factor is
\phantomsection\label{\detokenize{tech_note/Fire/CLM50_Tech_Note_Fire:equation-23.23}}\begin{equation}\label{equation:tech_note/Fire/CLM50_Tech_Note_Fire:23.23}
\begin{split}F_{e} =0.2+0.8\times \exp (-\pi \frac{GDP}{7} ).\end{split}
\end{equation}
For tree PFTs outside tropical closed forests, the demographic and
economic impact factors are given as
\phantomsection\label{\detokenize{tech_note/Fire/CLM50_Tech_Note_Fire:equation-23.24}}\begin{equation}\label{equation:tech_note/Fire/CLM50_Tech_Note_Fire:23.24}
\begin{split}F_{d} =0.4+0.6\times \exp (-\pi \frac{D_{p} }{125} )\end{split}
\end{equation}
and
\phantomsection\label{\detokenize{tech_note/Fire/CLM50_Tech_Note_Fire:equation-23.25}}\begin{equation}\label{equation:tech_note/Fire/CLM50_Tech_Note_Fire:23.25}
\begin{split}F_{e} =\left\{\begin{array}{cc}
{0.62,} & {GDP>20} \\ {0.83,} & {8<GDP\le 20} \\
{1,} & {GDP\le 8}
\end{array}\right. .\end{split}
\end{equation}
Equations \eqref{equation:tech_note/Fire/CLM50_Tech_Note_Fire:23.22} - \eqref{equation:tech_note/Fire/CLM50_Tech_Note_Fire:23.25} reflect that more developed and more
densely populated regions have a higher fire fighting capability.


\subsubsection{Fire impact}
\label{\detokenize{tech_note/Fire/CLM50_Tech_Note_Fire:fire-impact}}\label{\detokenize{tech_note/Fire/CLM50_Tech_Note_Fire:id4}}
In post-fire regions, we calculate PFT-level fire carbon emissions from
biomass burning of the \(j\)th PFT, \({\phi}_{j}\)  (g C s$^{\text{-1}}$), as
\phantomsection\label{\detokenize{tech_note/Fire/CLM50_Tech_Note_Fire:equation-23.26}}\begin{equation}\label{equation:tech_note/Fire/CLM50_Tech_Note_Fire:23.26}
\begin{split}\phi _{j} =A_{b,j} \mathbf{C}_{j} \bullet \mathbf{CC}_{j}\end{split}
\end{equation}
where \(A_{b,j}\) (km$^{\text{2}}$ s$^{\text{-1}}$) is burned area for
the \(j\)th PFT; \sphinxstylestrong{C}$_{\text{j}}$ =(\(C_{leaf}\), \(C_{stem}\),
\(C_{root}\), \(C_{ts}\)) is a vector with carbon density (g C km
$^{\text{-2}}$) for leaf, stem (live and dead stem), root (fine, live coarse
and dead coarse root), and transfer and storage carbon pools as elements;
\(\mathbf{CC}_{j}\) = (\(\mathbf{CC}_{leaf}\),
\(\mathbf{CC}_{stem}\), \(\mathbf{CC}_{root}\), \(\mathbf{CC}_{ts}\))
is the corresponding combustion completeness factor vector
(\hyperref[\detokenize{tech_note/Fire/CLM50_Tech_Note_Fire:table-pft-specific-combustion-completeness-and-fire-mortality-factors}]{Table \ref{\detokenize{tech_note/Fire/CLM50_Tech_Note_Fire:table-pft-specific-combustion-completeness-and-fire-mortality-factors}}}).
Moreover, we assume that 50\% and 28\% of column-level litter and coarse woody
debris are burned and the corresponding carbon is transferred to atmosphere.

Tissue mortality due to fire leads to carbon transfers in two ways.
First, carbon from uncombusted leaf, live stem, dead stem, root, and
transfer and storage pools
\(\mathbf{C^{'} _{j1}} ={(C_{{leaf}} (1-CC_{{leaf}} ) ,C_{{livestem}} (1-CC_{{stem}} ) ,C_{{deadstem}} (1-CC_{{stem}} ),C_{{root}} (1-CC_{{root}} ),C_{{ts}} (1-CC_{{ts}} ))}_{j}\)
(g C km$^{\text{-2}}$) is transferred to litter as
\phantomsection\label{\detokenize{tech_note/Fire/CLM50_Tech_Note_Fire:equation-23.27}}\begin{equation}\label{equation:tech_note/Fire/CLM50_Tech_Note_Fire:23.27}
\begin{split}\Psi _{j1} =\frac{A_{b,j} }{f_{j} A_{g} } \mathbf{C^{'} _{j1}} \bullet M_{j1}\end{split}
\end{equation}
where
\(M_{j1} =(M_{{leaf}} ,M_{{livestem,1}} ,M_{{deadstem}} ,M_{{root}} ,M_{{ts}} )_{j}\)
is the corresponding mortality factor vector (\hyperref[\detokenize{tech_note/Fire/CLM50_Tech_Note_Fire:table-pft-specific-combustion-completeness-and-fire-mortality-factors}]{Table \ref{\detokenize{tech_note/Fire/CLM50_Tech_Note_Fire:table-pft-specific-combustion-completeness-and-fire-mortality-factors}}}). Second,
carbon from uncombusted live stems is transferred to dead stems as:
\phantomsection\label{\detokenize{tech_note/Fire/CLM50_Tech_Note_Fire:equation-23.28}}\begin{equation}\label{equation:tech_note/Fire/CLM50_Tech_Note_Fire:23.28}
\begin{split}\Psi _{j2} =\frac{A_{b,j} }{f_{j} A_{g} } C_{livestem} (1-CC_{stem} )M_{livestem,2}\end{split}
\end{equation}
where \(M_{livestem,2}\)  is the corresponding mortality factor
(\hyperref[\detokenize{tech_note/Fire/CLM50_Tech_Note_Fire:table-pft-specific-combustion-completeness-and-fire-mortality-factors}]{Table \ref{\detokenize{tech_note/Fire/CLM50_Tech_Note_Fire:table-pft-specific-combustion-completeness-and-fire-mortality-factors}}}).

Fire nitrogen emissions and nitrogen transfers due to fire-induced
mortality are calculated the same way as for carbon, using the same
values for combustion completeness and mortality factors. With CLM’s
dynamic vegetation option enabled, the number of tree PFT individuals
killed by fire per km$^{\text{2}}$ (individual km$^{\text{-2}}$
s$^{\text{-1}}$) is given by
\phantomsection\label{\detokenize{tech_note/Fire/CLM50_Tech_Note_Fire:equation-23.29}}\begin{equation}\label{equation:tech_note/Fire/CLM50_Tech_Note_Fire:23.29}
\begin{split}P_{disturb,j} =\frac{A_{b,j} }{f_{j} A_{g} } P_{j} \xi _{j}\end{split}
\end{equation}
where \(P_{j}\)  (individual km$^{\text{-2}}$) is the population
density for the \(j\) th tree PFT and \(\xi _{j}\)  is the
whole-plant mortality factor
(\hyperref[\detokenize{tech_note/Fire/CLM50_Tech_Note_Fire:table-pft-specific-combustion-completeness-and-fire-mortality-factors}]{Table \ref{\detokenize{tech_note/Fire/CLM50_Tech_Note_Fire:table-pft-specific-combustion-completeness-and-fire-mortality-factors}}}).


\subsection{Agricultural fires}
\label{\detokenize{tech_note/Fire/CLM50_Tech_Note_Fire:agricultural-fires}}\label{\detokenize{tech_note/Fire/CLM50_Tech_Note_Fire:id5}}
The burned area of cropland (km$^{\text{2}}$ s$^{\text{-1}}$) is taken as \({A}_{b}\):
\phantomsection\label{\detokenize{tech_note/Fire/CLM50_Tech_Note_Fire:equation-23.30}}\begin{equation}\label{equation:tech_note/Fire/CLM50_Tech_Note_Fire:23.30}
\begin{split}A_{b} =a_{1} f_{se} f_{t} f_{crop} A_{g}\end{split}
\end{equation}
where \(a_{1}\)  (s$^{\text{-1}}$) is a constant; \(f_{se}\) represents
the socioeconomic effect on fires; \(f_{t}\)  determines the seasonality
of agricultural fires; \(f_{crop}\)  is the fractional coverage of
cropland.  \(a_{1}\) = 1.6x10$^{\text{-4}}$ hr$^{\text{-1}}$ is estimated
using an inverse method, by matching 1997-2004 simulations to the analysis
of {\hyperref[\detokenize{tech_note/References/CLM50_Tech_Note_References:vanderwerfetal2010}]{\sphinxcrossref{\DUrole{std,std-ref}{van der Werf et al. (2010)}}}} that shows the
2001-2009 average contribution of cropland fires is 4.7\% of the total
global burned area.

The socioeconomic factor \(f_{se}\)  is given as follows:
\phantomsection\label{\detokenize{tech_note/Fire/CLM50_Tech_Note_Fire:equation-23.31}}\begin{equation}\label{equation:tech_note/Fire/CLM50_Tech_Note_Fire:23.31}
\begin{split}f_{se} =f_{d} f_{e} .\end{split}
\end{equation}
Here
\phantomsection\label{\detokenize{tech_note/Fire/CLM50_Tech_Note_Fire:equation-23.32}}\begin{equation}\label{equation:tech_note/Fire/CLM50_Tech_Note_Fire:23.32}
\begin{split}f_{d} =0.04+0.96\times \exp [-\pi (\frac{D_{p} }{350} )^{0.5} ]\end{split}
\end{equation}
and
\phantomsection\label{\detokenize{tech_note/Fire/CLM50_Tech_Note_Fire:equation-23.33}}\begin{equation}\label{equation:tech_note/Fire/CLM50_Tech_Note_Fire:23.33}
\begin{split}f_{e} =0.01+0.99\times \exp (-\pi \frac{GDP}{10} )\end{split}
\end{equation}
are the effects of population density and GDP on burned area, derived
in a similar way to equation \eqref{equation:tech_note/Fire/CLM50_Tech_Note_Fire:23.32} and \eqref{equation:tech_note/Fire/CLM50_Tech_Note_Fire:23.33}. \(f_{t}\)
is set to 1 at the first time step during the climatological peak month
for agricultural fires ({\hyperref[\detokenize{tech_note/References/CLM50_Tech_Note_References:vanderwerfetal2010}]{\sphinxcrossref{\DUrole{std,std-ref}{van der Werf et al. 2010}}}});
\({f}_{t}\) is set to 0 otherwise. Peak
month in this dataset correlates with the month after harvesting or the
month before planting. In CLM we use this dataset the same way whether
the CROP option is active or not, without regard to the CROP option’s
simulated planting and harvesting dates.

In the post-fire region, fire impact is parameterized similar to section
\hyperref[\detokenize{tech_note/Fire/CLM50_Tech_Note_Fire:fire-impact}]{\ref{\detokenize{tech_note/Fire/CLM50_Tech_Note_Fire:fire-impact}}} but with combustion completeness factors and tissue
mortality factors for crop PFTs
(\hyperref[\detokenize{tech_note/Fire/CLM50_Tech_Note_Fire:table-pft-specific-combustion-completeness-and-fire-mortality-factors}]{Table \ref{\detokenize{tech_note/Fire/CLM50_Tech_Note_Fire:table-pft-specific-combustion-completeness-and-fire-mortality-factors}}}).


\subsection{Deforestation fires}
\label{\detokenize{tech_note/Fire/CLM50_Tech_Note_Fire:id6}}\label{\detokenize{tech_note/Fire/CLM50_Tech_Note_Fire:deforestation-fires}}
CLM focuses on deforestation fires in tropical closed forests. Tropical
closed forests are defined as grid cells with tropical tree (BET and BDT tropical)
coverage \(>\) 60\% according to the FAO classification. Deforestation fires
are defined as fires caused by deforestation, including escaped
deforestation fires, termed degradation fires. Deforestation and
degradation fires are assumed to occur outside of cropland areas in
these grid cells. Burned area is controlled by the deforestation rate
and climate:
\phantomsection\label{\detokenize{tech_note/Fire/CLM50_Tech_Note_Fire:equation-23.34}}\begin{equation}\label{equation:tech_note/Fire/CLM50_Tech_Note_Fire:23.34}
\begin{split}A_{b} = b \ f_{lu} f_{cli,d} f_{b} A_{g}\end{split}
\end{equation}
where \(b\) (s$^{\text{-1}}$) is a global constant;
\(f_{lu}\)  (fraction) represents the effect of decreasing
fractional coverage of tree PFTs derived from land use data;
\(f_{cli,d}\)  (fraction) represents the effect of climate
conditions on the burned area.

Constants \(b\) and \({f}_{lu}\) are calibrated
based on observations and reanalysis datasets in the Amazon rainforest
(tropical closed forests within 15.5 $^{\text{o}}$ S \(\text{-}\) 10.5
$^{\text{o}}$ N, 30.5 $^{\text{o}}$ W \(\text{-}\) 91 $^{\text{o}}$ W).
\(b\) = 0.033 d$^{\text{-1}}$ and \(f_{lu}\)  is defined as
\phantomsection\label{\detokenize{tech_note/Fire/CLM50_Tech_Note_Fire:equation-23.35}}\begin{equation}\label{equation:tech_note/Fire/CLM50_Tech_Note_Fire:23.35}
\begin{split}f_{lu} = \max (0.0005,0.19D-0.001)\end{split}
\end{equation}
where \(D\) (yr$^{\text{-1}}$) is the annual loss of tree cover
based on CLM land use and land cover change data.

The effect of climate on deforestation fires is parameterized as:
\phantomsection\label{\detokenize{tech_note/Fire/CLM50_Tech_Note_Fire:equation-23.36}}\begin{equation}\label{equation:tech_note/Fire/CLM50_Tech_Note_Fire:23.36}
\begin{split}\begin{array}{ll}
f_{cli,d} \quad = & \quad \max \left[0,\min (1,\frac{b_{2} -P_{60d} }{b_{2} } )\right]^{0.5} \times \\
& \quad \max \left[0,\min (1,\frac{b_{3} -P_{10d} }{b_{3} } )\right]^{0.5} \times \\
& \quad \max \left[0,\min (1,\frac{0.25-P}{0.25} )\right]
\end{array}\end{split}
\end{equation}
where \(P\) (mm d $^{\text{-1}}$) is instantaneous precipitation, while
\(P_{60d}\)  (mm d$^{\text{-1}}$) and \(P_{10d}\)  (mm d $^{\text{-1}}$)
are 60-day and 10-day running means of precipitation, respectively;
\(b_{2}\) (mm d $^{\text{-1}}$) and \(b_{3}\)  (mm d $^{\text{-1}}$) are
the grid-cell dependent thresholds of \(P_{60d}\)  and \(P_{10d}\) ;
0.25 mm d $^{\text{-1}}$ is the maximum precipitation rate for drizzle.
{\hyperref[\detokenize{tech_note/References/CLM50_Tech_Note_References:lepageetal2010}]{\sphinxcrossref{\DUrole{std,std-ref}{Le Page et al. (2010)}}}} analyzed the relationship
between large-scale deforestation fire counts and precipitation during 2003
\(\text{-}\)2006 in southern Amazonia where tropical evergreen trees
(BET Tropical) are dominant. Figure 2 in
{\hyperref[\detokenize{tech_note/References/CLM50_Tech_Note_References:lepageetal2010}]{\sphinxcrossref{\DUrole{std,std-ref}{Le Page et al. (2010)}}}} showed that fires generally
occurred if both \(P_{60d}\)  and \(P_{10d}\)  were less than about
4.0 mm d $^{\text{-1}}$, and fires occurred more frequently in a drier environment.
Based on the 30-yr (1985 to 2004) precipitation data in
{\hyperref[\detokenize{tech_note/References/CLM50_Tech_Note_References:qianetal2006}]{\sphinxcrossref{\DUrole{std,std-ref}{Qian et al. (2006)}}}}. The climatological precipitation
of dry months (P \textless{} 4.0 mm d $^{\text{-1}}$) in a year over tropical deciduous
tree (BDT Tropical) dominated regions is 46\% of that over BET Tropical
dominated regions, so we set the PFT-dependent thresholds of \(P_{60d}\)
and \(P_{10d}\)  as 4.0 mm d $^{\text{-1}}$ for BET Tropical and 1.8 mm d
$^{\text{-1}}$ (= 4.0 mm d $^{\text{-1}}$ \(\times\) 46\%) for BDT Tropical, and
\(b\)$_{\text{2}}$ and \(b\)$_{\text{3}}$ are the average of thresholds
of BET Tropical and BDT Tropical weighted bytheir coverage.

The post-fire area due to deforestation is not limited to land-type
conversion regions. In the tree-reduced region, the maximum fire carbon
emissions are assumed to be 80\% of the total conversion flux. According
to the fraction of conversion flux for tropical trees in the
tree-reduced region (60\%) assigned by CLM4-CN, to reach the maximum fire
carbon emissions in a conversion region requires burning this region
about twice when we set PFT-dependent combustion completeness factors to
about 0.3 for stem {[}the mean of 0.2\({-}\)0.4 used in
{\hyperref[\detokenize{tech_note/References/CLM50_Tech_Note_References:vanderwerfetal2010}]{\sphinxcrossref{\DUrole{std,std-ref}{van der Werf et al. (2010)}}}}. Therefore, when
the burned area calculated from equation \eqref{equation:tech_note/Fire/CLM50_Tech_Note_Fire:23.36} is
no more than twice the tree-reduced area, we assume no escaped fires
outside the land-type conversion region, and the fire-related fraction
of the total conversion flux is estimated as
\(\frac{A_{b} /A_{g} }{2D}\) . Otherwise, 80\% of the total
conversion flux is assumed to be fire carbon emissions, and the biomass
combustion and vegetation mortality outside the tree-reduced regions
with an area fraction of \(\frac{A_{b} }{A_{g} } -2D\) are set as in
section \hyperref[\detokenize{tech_note/Fire/CLM50_Tech_Note_Fire:fire-impact}]{\ref{\detokenize{tech_note/Fire/CLM50_Tech_Note_Fire:fire-impact}}}.


\subsection{Peat fires}
\label{\detokenize{tech_note/Fire/CLM50_Tech_Note_Fire:peat-fires}}\label{\detokenize{tech_note/Fire/CLM50_Tech_Note_Fire:id7}}
The burned area due to peat fires is given as \({A}_{b}\):
\phantomsection\label{\detokenize{tech_note/Fire/CLM50_Tech_Note_Fire:equation-23.37}}\begin{equation}\label{equation:tech_note/Fire/CLM50_Tech_Note_Fire:23.37}
\begin{split}A_{b} = c \ f_{cli,p} f_{peat} (1 - f_{sat} ) A_{g}\end{split}
\end{equation}
where \(c\) (s$^{\text{-1}}$) is a constant; \(f_{cli,p}\) represents
the effect of climate on the burned area; \(f_{peat}\) is the fractional
coverage of peatland in the grid cell; and \(f_{sat}\)  is the fraction
of the grid cell with a water table at the surface or higher. \(c\) = 0.17
\(\times\) 10 $^{\text{-3}}$ hr$^{\text{-1}}$ for tropical peat fires and
\(c\) = 0.9 \(\times\) 10 $^{\text{-5}}$ hr $^{\text{-1}}$ for boreal peat fires
are derived using an inverse method, by matching simulations to earlier
studies: about 2.4 Mha peatland was burned over Indonesia in 1997
({\hyperref[\detokenize{tech_note/References/CLM50_Tech_Note_References:pageetal2002}]{\sphinxcrossref{\DUrole{std,std-ref}{Page et al. 2002}}}}) and the average burned area of peat
fires in Western Canada was 0.2 Mha yr $^{\text{-1}}$ for 1980-1999
({\hyperref[\detokenize{tech_note/References/CLM50_Tech_Note_References:turetskyetal2004}]{\sphinxcrossref{\DUrole{std,std-ref}{Turetsky et al. 2004}}}}).

For tropical peat fires, \(f_{cli,p}\)  is set as a function of
long-term precipitation \(P_{60d}\) :
\phantomsection\label{\detokenize{tech_note/Fire/CLM50_Tech_Note_Fire:equation-23.38}}\begin{equation}\label{equation:tech_note/Fire/CLM50_Tech_Note_Fire:23.38}
\begin{split}f_{cli,p} = \ max \left[0,\min \left(1,\frac{4-P_{60d} }{4} \right)\right]^{2} .\end{split}
\end{equation}
For boreal peat fires, \(f_{cli,p}\)  is set to
\phantomsection\label{\detokenize{tech_note/Fire/CLM50_Tech_Note_Fire:equation-23.39}}\begin{equation}\label{equation:tech_note/Fire/CLM50_Tech_Note_Fire:23.39}
\begin{split}f_{cli,p} = \exp (-\pi \frac{\theta _{17cm} }{0.3} )\cdot \max [0,\min (1,\frac{T_{17cm} -T_{f} }{10} )]\end{split}
\end{equation}
where \(\theta _{17cm}\) is the wetness of the top 17 cm of soil.

Peat fires lead to peat burning and the combustion and mortality of
vegetation over peatlands. For tropical peat fires, based on
{\hyperref[\detokenize{tech_note/References/CLM50_Tech_Note_References:pageetal2002}]{\sphinxcrossref{\DUrole{std,std-ref}{Page et al. (2002)}}}}, about 6\% of the peat carbon loss
from stored carbon is caused by 33.9\% of the peatland burned. Carbon emissions
due to peat burning (g C m$^{\text{-2}}$ s$^{\text{-1}}$) are therefore set as the
product of 6\%/33.9\%, burned area fraction of peat fire (s$^{\text{-1}}$), and
soil organic carbon (g C m$^{\text{-2}}$). For boreal peat fires, the carbon
emissions due to peat burning are set as 2.2 kg C m$^{\text{-2}}$ peat fire
area ({\hyperref[\detokenize{tech_note/References/CLM50_Tech_Note_References:turetskyetal2002}]{\sphinxcrossref{\DUrole{std,std-ref}{Turetsky et al. 2002}}}}). Biomass combustion
and vegetation mortality in post-fire peatlands are set the same as section
\hyperref[\detokenize{tech_note/Fire/CLM50_Tech_Note_Fire:fire-impact}]{\ref{\detokenize{tech_note/Fire/CLM50_Tech_Note_Fire:fire-impact}}} for non-crop PFTs and as section
\hyperref[\detokenize{tech_note/Fire/CLM50_Tech_Note_Fire:agricultural-fires}]{\ref{\detokenize{tech_note/Fire/CLM50_Tech_Note_Fire:agricultural-fires}}} for crops PFTs.


\subsection{Fire trace gas and aerosol emissions}
\label{\detokenize{tech_note/Fire/CLM50_Tech_Note_Fire:fire-trace-gas-and-aerosol-emissions}}\label{\detokenize{tech_note/Fire/CLM50_Tech_Note_Fire:id8}}
CESM2 is the first Earth system model that can model the full coupling
among fire, fire emissions, land, and atmosphere. CLM5, as the land
component of CESM2, calculates the surface trace gas and aerosol emissions
due to fire and fire emission heights, as the inputs of atmospheric
chemistry model and aerosol model.

Emissions for trace gas and aerosol species x and the j-th PFT, \(E_{x,j}\)
(g species s$^{\text{-1}}$), are given by
\phantomsection\label{\detokenize{tech_note/Fire/CLM50_Tech_Note_Fire:equation-23.40}}\begin{equation}\label{equation:tech_note/Fire/CLM50_Tech_Note_Fire:23.40}
\begin{split}E_{x,j} = EF_{x,j}\frac{\phi _{j} }{[C]}.\end{split}
\end{equation}
Here, \(EF_{x,j}\) (g species (g dm)$^{\text{-1}}$) is PFT-dependent emission
factor scaled from biome-level values (Li et al., in prep, also used for FireMIP
fire emissions data) by Dr. Val Martin and Dr. Li. \([C]\) = 0.5
(g C (g dm)$^{\text{-1}}$) is a conversion factor from dry matter to carbon.

Emission height is PFT-dependent: 4.3 km for needleleaf tree PFTs, 3 km for other
boreal and temperate tree PFTs, 2.5 km for tropical tree PFTs, 2 km for shrub
PFTs, and 1 km for grass and crop PFTs. These values are compiled from earlier
studies by Dr. Val Martin.


\begin{savenotes}\sphinxattablestart
\centering
\sphinxcapstartof{table}
\sphinxcaption{PFT-specific combustion completeness and fire mortality factors.}\label{\detokenize{tech_note/Fire/CLM50_Tech_Note_Fire:table-pft-specific-combustion-completeness-and-fire-mortality-factors}}\label{\detokenize{tech_note/Fire/CLM50_Tech_Note_Fire:id9}}
\sphinxaftercaption
\begin{tabulary}{\linewidth}[t]{|T|T|T|T|T|T|T|T|T|T|T|T|}
\hline
\sphinxstylethead{\sphinxstyletheadfamily 
PFT
\unskip}\relax &\sphinxstylethead{\sphinxstyletheadfamily 
\sphinxstyleemphasis{CC}$_{\text{leaf}}$
\unskip}\relax &\sphinxstylethead{\sphinxstyletheadfamily 
\sphinxstyleemphasis{CC}$_{\text{stem}}$
\unskip}\relax &\sphinxstylethead{\sphinxstyletheadfamily 
\sphinxstyleemphasis{CC}$_{\text{root}}$
\unskip}\relax &\sphinxstylethead{\sphinxstyletheadfamily 
\sphinxstyleemphasis{CC}$_{\text{ts}}$
\unskip}\relax &\sphinxstylethead{\sphinxstyletheadfamily 
\sphinxstyleemphasis{M}$_{\text{leaf}}$
\unskip}\relax &\sphinxstylethead{\sphinxstyletheadfamily 
\sphinxstyleemphasis{M}$_{\text{livestem,1}}$
\unskip}\relax &\sphinxstylethead{\sphinxstyletheadfamily 
\sphinxstyleemphasis{M}$_{\text{deadstem}}$
\unskip}\relax &\sphinxstylethead{\sphinxstyletheadfamily 
\sphinxstyleemphasis{M}$_{\text{root}}$
\unskip}\relax &\sphinxstylethead{\sphinxstyletheadfamily 
\sphinxstyleemphasis{M}$_{\text{ts}}$
\unskip}\relax &\sphinxstylethead{\sphinxstyletheadfamily 
\sphinxstyleemphasis{M}$_{\text{livestem,2}}$
\unskip}\relax &\sphinxstylethead{\sphinxstyletheadfamily 
\(\xi\)$_{\text{j}}$
\unskip}\relax \\
\hline
NET Temperate
&
0.80
&
0.30
&
0.00
&
0.50
&
0.80
&
0.15
&
0.15
&
0.15
&
0.50
&
0.35
&
0.15
\\
\hline
NET Boreal
&
0.80
&
0.30
&
0.00
&
0.50
&
0.80
&
0.15
&
0.15
&
0.15
&
0.50
&
0.35
&
0.15
\\
\hline
NDT Boreal
&
0.80
&
0.30
&
0.00
&
0.50
&
0.80
&
0.15
&
0.15
&
0.15
&
0.50
&
0.35
&
0.15
\\
\hline
BET Tropical
&
0.80
&
0.27
&
0.00
&
0.45
&
0.80
&
0.13
&
0.13
&
0.13
&
0.45
&
0.32
&
0.13
\\
\hline
BET Temperate
&
0.80
&
0.27
&
0.00
&
0.45
&
0.80
&
0.13
&
0.13
&
0.13
&
0.45
&
0.32
&
0.13
\\
\hline
BDT Tropical
&
0.80
&
0.27
&
0.00
&
0.45
&
0.80
&
0.10
&
0.10
&
0.10
&
0.35
&
0.25
&
0.10
\\
\hline
BDT Temperate
&
0.80
&
0.27
&
0.00
&
0.45
&
0.80
&
0.10
&
0.10
&
0.10
&
0.35
&
0.25
&
0.10
\\
\hline
BDT Boreal
&
0.80
&
0.27
&
0.00
&
0.45
&
0.80
&
0.13
&
0.13
&
0.13
&
0.45
&
0.32
&
0.13
\\
\hline
BES Temperate
&
0.80
&
0.35
&
0.00
&
0.55
&
0.80
&
0.17
&
0.17
&
0.17
&
0.55
&
0.38
&
0.17
\\
\hline
BDS Temperate
&
0.80
&
0.35
&
0.00
&
0.55
&
0.80
&
0.17
&
0.17
&
0.17
&
0.55
&
0.38
&
0.17
\\
\hline
BDS Boreal
&
0.80
&
0.35
&
0.00
&
0.55
&
0.80
&
0.17
&
0.17
&
0.17
&
0.55
&
0.38
&
0.17
\\
\hline
C$_{\text{3}}$ Grass Arctic
&
0.80
&
0.80
&
0.00
&
0.80
&
0.80
&
0.20
&
0.20
&
0.20
&
0.80
&
0.60
&
0.20
\\
\hline
C$_{\text{3}}$ Grass
&
0.80
&
0.80
&
0.00
&
0.80
&
0.80
&
0.20
&
0.20
&
0.20
&
0.80
&
0.60
&
0.20
\\
\hline
C$_{\text{4}}$ Grass
&
0.80
&
0.80
&
0.00
&
0.80
&
0.80
&
0.20
&
0.20
&
0.20
&
0.80
&
0.60
&
0.20
\\
\hline
Crop
&
0.80
&
0.80
&
0.00
&
0.80
&
0.80
&
0.20
&
0.20
&
0.20
&
0.80
&
0.60
&
0.20
\\
\hline
\end{tabulary}
\par
\sphinxattableend\end{savenotes}

Leaves (\(CC_{leaf}\) ), stems (\(CC_{stem}\) ),
roots (\(CC_{root}\) ) , and transfer and storage carbon
(\(CC_{ts}\) ); mortality factors for leaves
(\(M_{leaf}\) ), live stems (\(M_{livestem,1}\) ),
dead stems (\(M_{deadstem}\) ), roots
(\(M_{root}\) ), and transfer and storage carbon
(\(M_{ts}\) ) related to the carbon transfers from these pools
to litter pool; mortality factors for live stems
(\(M_{livestem,2}\) ) related to the carbon transfer from live
stems to dead stems; whole-plant mortality factor (\(\xi _{j}\) ).


\section{Methane Model}
\label{\detokenize{tech_note/Methane/CLM50_Tech_Note_Methane:rst-methane-model}}\label{\detokenize{tech_note/Methane/CLM50_Tech_Note_Methane::doc}}\label{\detokenize{tech_note/Methane/CLM50_Tech_Note_Methane:methane-model}}
The representation of processes in the methane biogeochemical model
integrated in CLM {[}CLM4Me; ({\hyperref[\detokenize{tech_note/References/CLM50_Tech_Note_References:rileyetal2011a}]{\sphinxcrossref{\DUrole{std,std-ref}{Riley et al. 2011a}}}}){]}
is based on several previously published models
({\hyperref[\detokenize{tech_note/References/CLM50_Tech_Note_References:caoetal1996}]{\sphinxcrossref{\DUrole{std,std-ref}{Cao et al. 1996}}}}; {\hyperref[\detokenize{tech_note/References/CLM50_Tech_Note_References:petrescuetal2010}]{\sphinxcrossref{\DUrole{std,std-ref}{Petrescu et al. 2010}}}};
{\hyperref[\detokenize{tech_note/References/CLM50_Tech_Note_References:tianetal2010}]{\sphinxcrossref{\DUrole{std,std-ref}{Tianet al. 2010}}}}; {\hyperref[\detokenize{tech_note/References/CLM50_Tech_Note_References:walteretal2001}]{\sphinxcrossref{\DUrole{std,std-ref}{Walter et al. 2001}}}};
{\hyperref[\detokenize{tech_note/References/CLM50_Tech_Note_References:waniaetal2010}]{\sphinxcrossref{\DUrole{std,std-ref}{Wania et al. 2010}}}}; {\hyperref[\detokenize{tech_note/References/CLM50_Tech_Note_References:zhangetal2002}]{\sphinxcrossref{\DUrole{std,std-ref}{Zhang et al. 2002}}}};
{\hyperref[\detokenize{tech_note/References/CLM50_Tech_Note_References:zhuangetal2004}]{\sphinxcrossref{\DUrole{std,std-ref}{Zhuang et al. 2004}}}}). Although the model has similarities
with these precursor models, a number of new process representations and
parameterization have been integrated into CLM.

Mechanistically modeling net surface CH$_{\text{4}}$ emissions requires
representing a complex and interacting series of processes. We first
(section \hyperref[\detokenize{tech_note/Methane/CLM50_Tech_Note_Methane:methane-model-structure-and-flow}]{\ref{\detokenize{tech_note/Methane/CLM50_Tech_Note_Methane:methane-model-structure-and-flow}}}) describe the overall
model structure and flow of
information in the CH$_{\text{4}}$ model, then describe the methods
used to represent: CH$_{\text{4}}$ mass balance; CH$_{\text{4}}$
production; ebullition; aerenchyma transport; CH$_{\text{4}}$
oxidation; reactive transport solution, including boundary conditions,
numerical solution, water table interface, etc.; seasonal inundation
effects; and impact of seasonal inundation on CH$_{\text{4}}$
production.


\subsection{Methane Model Structure and Flow}
\label{\detokenize{tech_note/Methane/CLM50_Tech_Note_Methane:methane-model-structure-and-flow}}\label{\detokenize{tech_note/Methane/CLM50_Tech_Note_Methane:id1}}
The driver routine for the methane biogeochemistry calculations (ch4, in
ch4Mod.F) controls the initialization of boundary conditions,
inundation, and impact of redox conditions; calls to routines to
calculate CH$_{\text{4}}$ production, oxidation, transport through
aerenchyma, ebullition, and the overall mass balance (for unsaturated
and saturated soils and, if desired, lakes); resolves changes to
CH$_{\text{4}}$ calculations associated with a changing inundated
fraction; performs a mass balance check; and calculates the average
gridcell CH$_{\text{4}}$ production, oxidation, and exchanges with
the atmosphere.


\subsection{Governing Mass-Balance Relationship}
\label{\detokenize{tech_note/Methane/CLM50_Tech_Note_Methane:governing-mass-balance-relationship}}\label{\detokenize{tech_note/Methane/CLM50_Tech_Note_Methane:id2}}
The model (\hyperref[\detokenize{tech_note/Methane/CLM50_Tech_Note_Methane:figure-methane-schematic}]{Figure \ref{\detokenize{tech_note/Methane/CLM50_Tech_Note_Methane:figure-methane-schematic}}}) accounts for CH$_{\text{4}}$
production in the anaerobic fraction of soil (\sphinxstyleemphasis{P}, mol m$^{\text{-3}}$
s$^{\text{-1}}$), ebullition (\sphinxstyleemphasis{E}, mol m$^{\text{-3}}$ s$^{\text{-1}}$),
aerenchyma transport (\sphinxstyleemphasis{A}, mol m$^{\text{-3}}$ s$^{\text{-1}}$), aqueous and
gaseous diffusion (\({F}_{D}\), mol m$^{\text{-2}}$ s$^{\text{-1}}$), and
oxidation (\sphinxstyleemphasis{O}, mol m$^{\text{-3}}$ s$^{\text{-1}}$) via a transient reaction
diffusion equation:
\phantomsection\label{\detokenize{tech_note/Methane/CLM50_Tech_Note_Methane:equation-24.1}}\begin{equation}\label{equation:tech_note/Methane/CLM50_Tech_Note_Methane:24.1}
\begin{split}\frac{\partial \left(RC\right)}{\partial t} =\frac{\partial F_{D} }{\partial z} +P\left(z,t\right)-E\left(z,t\right)-A\left(z,t\right)-O\left(z,t\right)\end{split}
\end{equation}
Here \sphinxstyleemphasis{z} (m) represents the vertical dimension, \sphinxstyleemphasis{t} (s) is time, and \sphinxstyleemphasis{R}
accounts for gas in both the aqueous and gaseous
phases:\(R = \epsilon _{a} +K_{H} \epsilon _{w}\), with
\(\epsilon _{a}\), \(\epsilon _{w}\), and \(K_{H}\) (-) the air-filled porosity, water-filled
porosity, and partitioning coefficient for the species of interest,
respectively, and \(C\) represents CH$_{\text{4}}$ or O$_{\text{2}}$ concentration with respect to water volume (mol m$^{\text{-3}}$).

An analogous version of equation is concurrently solved for
O$_{\text{2}}$, but with the following differences relative to
CH$_{\text{4}}$: \sphinxstyleemphasis{P} = \sphinxstyleemphasis{E} = 0 (i.e., no production or ebullition),
and the oxidation sink includes the O$_{\text{2}}$ demanded by
methanotrophs, heterotroph decomposers, nitrifiers, and autotrophic root
respiration.

As currently implemented, each gridcell contains an inundated and a
non-inundated fraction. Therefore, equation is solved four times for
each gridcell and time step: in the inundated and non-inundated
fractions, and for CH$_{\text{4}}$ and O$_{\text{2}}$. If desired,
the CH$_{\text{4}}$ and O$_{\text{2}}$ mass balance equation is
solved again for lakes (Chapter 9). For non-inundated areas, the water
table interface is defined at the deepest transition from greater than
95\% saturated to less than 95\% saturated that occurs above frozen soil
layers. The inundated fraction is allowed to change at each time step,
and the total soil CH$_{\text{4}}$ quantity is conserved by evolving
CH$_{\text{4}}$ to the atmosphere when the inundated fraction
decreases, and averaging a portion of the non-inundated concentration
into the inundated concentration when the inundated fraction increases.

\begin{figure}[htbp]
\centering
\capstart

\noindent\sphinxincludegraphics{{image14}.png}
\caption{Schematic representation of biological and physical
processes integrated in CLM that affect the net CH$_{\text{4}}$
surface flux ({\hyperref[\detokenize{tech_note/References/CLM50_Tech_Note_References:rileyetal2011a}]{\sphinxcrossref{\DUrole{std,std-ref}{Riley et al. 2011a}}}}). (left)
Fully inundated portion of a
CLM gridcell and (right) variably saturated portion of a gridcell.}\label{\detokenize{tech_note/Methane/CLM50_Tech_Note_Methane:figure-methane-schematic}}\label{\detokenize{tech_note/Methane/CLM50_Tech_Note_Methane:id13}}\end{figure}


\subsection{CH$_{\text{4}}$ Production}
\label{\detokenize{tech_note/Methane/CLM50_Tech_Note_Methane:id3}}\label{\detokenize{tech_note/Methane/CLM50_Tech_Note_Methane:ch4-production}}
Because CLM does not currently specifically represent wetland plant
functional types or soil biogeochemical processes, we used
gridcell-averaged decomposition rates as proxies. Thus, the upland
(default) heterotrophic respiration is used to estimate the wetland
decomposition rate after first dividing off the O$_{\text{2}}$
limitation. The O$_{\text{2}}$ consumption associated with anaerobic
decomposition is then set to the unlimited version so that it will be
reduced appropriately during O$_{\text{2}}$ competition.
CH$_{\text{4}}$ production at each soil level in the anaerobic
portion (i.e., below the water table) of the column is related to the
gridcell estimate of heterotrophic respiration from soil and litter
(R$_{\text{H}}$; mol C m$^{\text{-2}}$ s$_{\text{-1}}$) corrected for its soil temperature
(\({T}_{s}\)) dependence, soil temperature through a
\({A}_{10}\) factor (\(f_{T}\)), pH (\(f_{pH}\)),
redox potential (\(f_{pE}\)), and a factor accounting for the
seasonal inundation fraction (\sphinxstyleemphasis{S}, described below):
\phantomsection\label{\detokenize{tech_note/Methane/CLM50_Tech_Note_Methane:equation-24.2}}\begin{equation}\label{equation:tech_note/Methane/CLM50_Tech_Note_Methane:24.2}
\begin{split}P=R_{H} f_{CH_{4} } f_{T} f_{pH} f_{pE} S.\end{split}
\end{equation}
Here, \(f_{CH_{4} }\)  is the baseline ratio between CO$_{\text{2}}$
and CH$_{\text{4}}$ production (all parameters values are given in
\hyperref[\detokenize{tech_note/Methane/CLM50_Tech_Note_Methane:table-methane-parameter-descriptions}]{Table \ref{\detokenize{tech_note/Methane/CLM50_Tech_Note_Methane:table-methane-parameter-descriptions}}}). Currently, \(f_{CH_{4} }\)
is modified to account for our assumptions that methanogens may have a
higher Q\({}_{10}\) than aerobic decomposers; are not N limited;
and do not have a low-moisture limitation.

When the single BGC soil level is used in CLM (Chapter \hyperref[\detokenize{tech_note/Decomposition/CLM50_Tech_Note_Decomposition:rst-decomposition}]{\ref{\detokenize{tech_note/Decomposition/CLM50_Tech_Note_Decomposition:rst-decomposition}}}), the
temperature factor, \(f_{T}\) , is set to 0 for temperatures equal
to or below freezing, even though CLM allows heterotrophic respiration
below freezing. However, if the vertically resolved BGC soil column is
used, CH$_{\text{4}}$ production continues below freezing because
liquid water stress limits decomposition. The base temperature for the
\({Q}_{10}\) factor, \({T}_{B}\), is 22$^{\text{o}}$ C and effectively
modified the base \(f_{CH_{4}}\)  value.

For the single-layer BGC version,  \({R}_{H}\) is distributed
among soil levels by assuming that 50\% is associated with the roots
(using the CLM PFT-specific rooting distribution) and the rest is evenly
divided among the top 0.28 m of soil (to be consistent with CLM’s soil
decomposition algorithm). For the vertically resolved BGC version, the
prognosed distribution of \({R}_{H}\) is used to estimate CH$_{\text{4}}$ production.

The factor \(f_{pH}\)  is nominally set to 1, although a static
spatial map of \sphinxstyleemphasis{pH} can be used to determine this factor
({\hyperref[\detokenize{tech_note/References/CLM50_Tech_Note_References:dunfieldetal1993}]{\sphinxcrossref{\DUrole{std,std-ref}{Dunfield et al. 1993}}}}) by applying:
\phantomsection\label{\detokenize{tech_note/Methane/CLM50_Tech_Note_Methane:equation-24.3}}\begin{equation}\label{equation:tech_note/Methane/CLM50_Tech_Note_Methane:24.3}
\begin{split}f_{pH} =10^{-0.2235pH^{2} +2.7727pH-8.6} .\end{split}
\end{equation}
The \(f_{pE}\)  factor assumes that alternative electron acceptors
are reduced with an e-folding time of 30 days after inundation. The
default version of the model applies this factor to horizontal changes
in inundated area but not to vertical changes in the water table depth
in the upland fraction of the gridcell. We consider both \(f_{pH}\)
and \(f_{pE}\)  to be poorly constrained in the model and identify
these controllers as important areas for model improvement.

As a non-default option to account for CH$_{\text{4}}$ production in
anoxic microsites above the water table, we apply the Arah and Stephen
(1998) estimate of anaerobic fraction:
\phantomsection\label{\detokenize{tech_note/Methane/CLM50_Tech_Note_Methane:equation-24.4}}\begin{equation}\label{equation:tech_note/Methane/CLM50_Tech_Note_Methane:24.4}
\begin{split}\varphi =\frac{1}{1+\eta C_{O_{2} } } .\end{split}
\end{equation}
Here, \(\phi\) is the factor by which production is inhibited
above the water table (compared to production as calculated in equation
, \(C_{O_{2}}\)  (mol m$^{\text{-3}}$) is the bulk soil oxygen
concentration, and \(\eta\) = 400 mol m$^{\text{-3}}$.

The O$_{\text{2}}$ required to facilitate the vertically resolved
heterotrophic decomposition and root respiration is estimated assuming 1
mol O$_{\text{2}}$ is required per mol CO$_{\text{2}}$ produced.
The model also calculates the O$_{\text{2}}$ required during
nitrification, and the total O$_{\text{2}}$ demand is used in the
O$_{\text{2}}$ mass balance solution.


\begin{savenotes}\sphinxattablestart
\centering
\sphinxcapstartof{table}
\sphinxcaption{Parameter descriptions and sensitivity analysis ranges applied in the methane model}\label{\detokenize{tech_note/Methane/CLM50_Tech_Note_Methane:table-methane-parameter-descriptions}}\label{\detokenize{tech_note/Methane/CLM50_Tech_Note_Methane:id14}}
\sphinxaftercaption
\begin{tabular}[t]{|*{6}{\X{1}{6}|}}
\hline
\sphinxstylethead{\sphinxstyletheadfamily 
Mechanism
\unskip}\relax &\sphinxstylethead{\sphinxstyletheadfamily 
Parameter
\unskip}\relax &\sphinxstylethead{\sphinxstyletheadfamily 
Baseline Value
\unskip}\relax &\sphinxstylethead{\sphinxstyletheadfamily 
Range for Sensitivity Analysis
\unskip}\relax &\sphinxstylethead{\sphinxstyletheadfamily 
Units
\unskip}\relax &\sphinxstylethead{\sphinxstyletheadfamily 
Description
\unskip}\relax \\
\hline
Production
&
\({Q}_{10}\)
&
2
&
1.5 \textendash{} 4
&\begin{itemize}
\item {} 
\end{itemize}
&
CH$_{\text{4}}$ production \({Q}_{10}\)
\\
\hline&
\(f_{pH}\)
&
1
&
On, off
&\begin{itemize}
\item {} 
\end{itemize}
&
Impact of pH on CH$_{\text{4}}$ production
\\
\hline&
\(f_{pE}\)
&
1
&
On, off
&\begin{itemize}
\item {} 
\end{itemize}
&
Impact of redox potential on CH$_{\text{4}}$ production
\\
\hline&
\sphinxstyleemphasis{S}
&
Varies
&
NA
&\begin{itemize}
\item {} 
\end{itemize}
&
Seasonal inundation factor
\\
\hline&
\(\beta\)
&
0.2
&
NA
&\begin{itemize}
\item {} 
\end{itemize}
&
Effect of anoxia on decomposition rate (used to calculate \sphinxstyleemphasis{S} only)
\\
\hline&
\(f_{CH_{4} }\)
&
0.2
&
NA
&\begin{itemize}
\item {} 
\end{itemize}
&
Ratio between CH$_{\text{4}}$ and CO$_{\text{2}}$ production below the water table
\\
\hline
Ebullition
&
\({C}_{e,max}\)
&
0.15
&
NA
&
mol m$^{\text{-3}}$
&
CH$_{\text{4}}$ concentration to start ebullition
\\
\hline&
\({C}_{e,min}\)
&
0.15
&
NA
&\begin{itemize}
\item {} 
\end{itemize}
&
CH$_{\text{4}}$ concentration to end ebullition
\\
\hline
Diffusion
&
\(f_{D_{0} }\)
&
1
&
1, 10
&
m$^{\text{2}}$ s$^{\text{-1}}$
&
Diffusion coefficient multiplier (Table 24.2)
\\
\hline
Aerenchyma
&
\sphinxstyleemphasis{p}
&
0.3
&
NA
&\begin{itemize}
\item {} 
\end{itemize}
&
Grass aerenchyma porosity
\\
\hline&
\sphinxstyleemphasis{R}
&
2.9\(\times\)10$^{\text{-3}}$ m
&
NA
&
m
&
Aerenchyma radius
\\
\hline&
\({r}_{L}\)
&
3
&
NA
&\begin{itemize}
\item {} 
\end{itemize}
&
Root length to depth ratio
\\
\hline&
\({F}_{a}\)
&
1
&
0.5 \textendash{} 1.5
&\begin{itemize}
\item {} 
\end{itemize}
&
Aerenchyma conductance multiplier
\\
\hline
Oxidation
&
\(K_{CH_{4} }\)
&
5 x 10$^{\text{-3}}$
&
5\(\times\)10\({}^{-4}\)\({}_{ }\)- 5\(\times\)10$^{\text{-2}}$
&
mol m$^{\text{-3}}$
&
CH$_{\text{4}}$ half-saturation oxidation coefficient (wetlands)
\\
\hline&
\(K_{O_{2} }\)
&
2 x 10$^{\text{-2}}$
&
2\(\times\)10$^{\text{-3}}$ - 2\(\times\)10$^{\text{-1}}$
&
mol m$^{\text{-3}}$
&
O$_{\text{2}}$ half-saturation oxidation coefficient
\\
\hline&
\(R_{o,\max }\)
&
1.25 x 10\({}^{-5}\)
&
1.25\(\times\)10\({}^{-6}\) - 1.25\(\times\)10\({}^{-4}\)
&
mol m$^{\text{-3}}$ s$^{\text{-1}}$
&
Maximum oxidation rate (wetlands)
\\
\hline
\end{tabular}
\par
\sphinxattableend\end{savenotes}


\subsection{Ebullition}
\label{\detokenize{tech_note/Methane/CLM50_Tech_Note_Methane:ebullition}}
Briefly, the simulated aqueous CH$_{\text{4}}$ concentration in each
soil level is used to estimate the expected equilibrium gaseous partial
pressure (\(C_{e}\) ), as a function of temperature and depth below
the water table, by first estimating the Henry’s law partitioning
coefficient (\(k_{h}^{C}\) ) by the method described in
{\hyperref[\detokenize{tech_note/References/CLM50_Tech_Note_References:waniaetal2010}]{\sphinxcrossref{\DUrole{std,std-ref}{Wania et al. (2010)}}}}:
\phantomsection\label{\detokenize{tech_note/Methane/CLM50_Tech_Note_Methane:equation-24.5}}\begin{equation}\label{equation:tech_note/Methane/CLM50_Tech_Note_Methane:24.5}
\begin{split}\log \left(\frac{1}{k_{H} } \right)=\log k_{H}^{s} -\frac{1}{C_{H} } \left(\frac{1}{T} -\frac{1}{T^{s} } \right)\end{split}
\end{equation}\phantomsection\label{\detokenize{tech_note/Methane/CLM50_Tech_Note_Methane:equation-24.6}}\begin{equation}\label{equation:tech_note/Methane/CLM50_Tech_Note_Methane:24.6}
\begin{split}k_{h}^{C} =Tk_{H} R_{g}\end{split}
\end{equation}\phantomsection\label{\detokenize{tech_note/Methane/CLM50_Tech_Note_Methane:equation-24.7}}\begin{equation}\label{equation:tech_note/Methane/CLM50_Tech_Note_Methane:24.7}
\begin{split}C_{e} =\frac{C_{w} R_{g} T}{\theta _{s} k_{H}^{C} p}\end{split}
\end{equation}
where \(C_{H}\) is a constant, \(R_{g}\)  is the universal
gas constant, \(k_{H}^{s}\)  is Henry’s law partitioning coefficient
at standard temperature (\(T^{s}\) ),\(C_{w}\) is local
aqueous CH$_{\text{4}}$ concentration, and \sphinxstyleemphasis{p} is pressure.

The local pressure is calculated as the sum of the ambient pressure,
water pressure down to the local depth, and pressure from surface
ponding (if applicable). When the CH$_{\text{4}}$ partial pressure
exceeds 15\% of the local pressure (Baird et al. 2004; Strack et al.
2006; Wania et al. 2010), bubbling occurs to remove CH$_{\text{4}}$
to below this value, modified by the fraction of CH$_{\text{4}}$ in
the bubbles {[}taken as 57\%; ({\hyperref[\detokenize{tech_note/References/CLM50_Tech_Note_References:kellneretal2006}]{\sphinxcrossref{\DUrole{std,std-ref}{Kellner et al. 2006}}}};
{\hyperref[\detokenize{tech_note/References/CLM50_Tech_Note_References:waniaetal2010}]{\sphinxcrossref{\DUrole{std,std-ref}{Wania et al. 2010}}}}){]}.
Bubbles are immediately added to the surface flux for saturated columns
and are placed immediately above the water table interface in
unsaturated columns.


\subsection{Aerenchyma Transport}
\label{\detokenize{tech_note/Methane/CLM50_Tech_Note_Methane:id4}}\label{\detokenize{tech_note/Methane/CLM50_Tech_Note_Methane:aerenchyma-transport}}
Aerenchyma transport is modeled in CLM as gaseous diffusion driven by a
concentration gradient between the specific soil layer and the
atmosphere and, if specified, by vertical advection with the
transpiration stream. There is evidence that pressure driven flow can
also occur, but we did not include that mechanism in the current model.

The diffusive transport through aerenchyma (\sphinxstyleemphasis{A}, mol m$^{\text{-2}}$ s$^{\text{-1}}$) from each soil layer is represented in the model as:
\phantomsection\label{\detokenize{tech_note/Methane/CLM50_Tech_Note_Methane:equation-24.8}}\begin{equation}\label{equation:tech_note/Methane/CLM50_Tech_Note_Methane:24.8}
\begin{split}A=\frac{C\left(z\right)-C_{a} }{{\raise0.7ex\hbox{$ r_{L} z $}\!\mathord{\left/ {\vphantom {r_{L} z D}} \right. \kern-\nulldelimiterspace}\!\lower0.7ex\hbox{$ D $}} +r_{a} } pT\rho _{r} ,\end{split}
\end{equation}
where \sphinxstyleemphasis{D} is the free-air gas diffusion coefficient (m:sup:\sphinxtitleref{2} s$^{\text{-1}}$); \sphinxstyleemphasis{C(z)} (mol m$^{\text{-3}}$) is the gaseous
concentration at depth \sphinxstyleemphasis{z} (m); \(r_{L}\)  is the ratio of root
length to depth; \sphinxstyleemphasis{p} is the porosity (-); \sphinxstyleemphasis{T} is specific aerenchyma
area (m:sup:\sphinxtitleref{2} m$^{\text{-2}}$); \({r}_{a}\) is the
aerodynamic resistance between the surface and the atmospheric reference
height (s m:sup:\sphinxtitleref{-1}); and \(\rho _{r}\)  is the rooting
density as a function of depth (-). The gaseous concentration is
calculated with Henry’s law as described in equation .

Based on the ranges reported in {\hyperref[\detokenize{tech_note/References/CLM50_Tech_Note_References:colmer2003}]{\sphinxcrossref{\DUrole{std,std-ref}{Colmer (2003)}}}}, we have chosen
baseline aerenchyma porosity values of 0.3 for grass and crop PFTs and 0.1 for
tree and shrub PFTs:
\phantomsection\label{\detokenize{tech_note/Methane/CLM50_Tech_Note_Methane:equation-24.9}}\begin{equation}\label{equation:tech_note/Methane/CLM50_Tech_Note_Methane:24.9}
\begin{split}T=\frac{4 f_{N} N_{a}}{0.22} \pi R^{2} .\end{split}
\end{equation}
Here \(N_{a}\)  is annual net primary production (NPP, mol
m$^{\text{-2}}$ s$^{\text{-1}}$); \sphinxstyleemphasis{R} is the aerenchyma radius
(2.9 \(\times\)10$^{\text{-3}}$ m); \({f}_{N}\) is the
belowground fraction of annual NPP; and the 0.22 factor represents the
amount of C per tiller. O$_{\text{2}}$ can also diffuse in from the
atmosphere to the soil layer via the reverse of the same pathway, with
the same representation as Equation but with the gas diffusivity of
oxygen.

CLM also simulates the direct emission of CH$_{\text{4}}$ from leaves
to the atmosphere via transpiration of dissolved methane. We calculate
this flux (\(F_{CH_{4} -T}\) ; mol m\({}^{-}\)$^{\text{2}}$
s$^{\text{-1}}$) using the simulated soil water methane concentration
(\(C_{CH_{4} ,j}\)  (mol m$^{\text{-3}}$)) in each soil layer \sphinxstyleemphasis{j}
and the CLM predicted transpiration (\(F_{T}\) ) for each PFT,
assuming that no methane was oxidized inside the plant tissue:
\phantomsection\label{\detokenize{tech_note/Methane/CLM50_Tech_Note_Methane:equation-24.10}}\begin{equation}\label{equation:tech_note/Methane/CLM50_Tech_Note_Methane:24.10}
\begin{split}F_{CH_{4} -T} =\sum _{j}\rho _{r,j} F_{T} C_{CH_{4} ,j}  .\end{split}
\end{equation}

\subsection{CH$_{\text{4}}$ Oxidation}
\label{\detokenize{tech_note/Methane/CLM50_Tech_Note_Methane:id5}}\label{\detokenize{tech_note/Methane/CLM50_Tech_Note_Methane:ch4-oxidation}}
CLM represents CH$_{\text{4}}$ oxidation with double Michaelis-Menten
kinetics ({\hyperref[\detokenize{tech_note/References/CLM50_Tech_Note_References:arahstephen1998}]{\sphinxcrossref{\DUrole{std,std-ref}{Arah and Stephen 1998}}}}; {\hyperref[\detokenize{tech_note/References/CLM50_Tech_Note_References:segers1998}]{\sphinxcrossref{\DUrole{std,std-ref}{Segers 1998}}}}),
dependent on both the gaseous CH$_{\text{4}}$ and O$_{\text{2}}$ concentrations:
\phantomsection\label{\detokenize{tech_note/Methane/CLM50_Tech_Note_Methane:equation-24.11}}\begin{equation}\label{equation:tech_note/Methane/CLM50_Tech_Note_Methane:24.11}
\begin{split}R_{oxic} =R_{o,\max } \left[\frac{C_{CH_{4} } }{K_{CH_{4} } +C_{CH_{4} } } \right]\left[\frac{C_{O_{2} } }{K_{O_{2} } +C_{O_{2} } } \right]Q_{10} F_{\vartheta }\end{split}
\end{equation}
where \(K_{CH_{4} }\)  and \(K_{O_{2} }\) are the half
saturation coefficients (mol m$^{\text{-3}}$) with respect to
CH$_{\text{4}}$ and O$_{\text{2}}$ concentrations, respectively;
\(R_{o,\max }\)  is the maximum oxidation rate (mol
m$^{\text{-3}}$ s$^{\text{-1}}$); and \({Q}_{10}\)
specifies the temperature dependence of the reaction with a base
temperature set to 12 $^{\text{o}}$ C. The soil moisture limitation
factor \(F_{\theta }\) is applied above the water table to
represent water stress for methanotrophs. Based on the data in
{\hyperref[\detokenize{tech_note/References/CLM50_Tech_Note_References:schnellking1996}]{\sphinxcrossref{\DUrole{std,std-ref}{Schnell and King (1996)}}}}, we take
\(F_{\theta } = {e}^{-P/{P}_{c}}\), where \sphinxstyleemphasis{P} is the soil moisture
potential and \({P}_{c} = -2.4 \times {10}^{5}\) mm.


\subsection{Reactive Transport Solution}
\label{\detokenize{tech_note/Methane/CLM50_Tech_Note_Methane:id6}}\label{\detokenize{tech_note/Methane/CLM50_Tech_Note_Methane:reactive-transport-solution}}
The solution to equation is solved in several sequential steps: resolve
competition for CH$_{\text{4}}$ and O$_{\text{2}}$ (section
\hyperref[\detokenize{tech_note/Methane/CLM50_Tech_Note_Methane:competition-for-ch4and-o2}]{\ref{\detokenize{tech_note/Methane/CLM50_Tech_Note_Methane:competition-for-ch4and-o2}}}); add the ebullition flux into the
layer directly above the water
table or into the atmosphere; calculate the overall CH$_{\text{4}}$
or O$_{\text{2}}$ source term based on production, aerenchyma
transport, ebullition, and oxidation; establish boundary conditions,
including surface conductance to account for snow, ponding, and
turbulent conductances and bottom flux condition
(section \hyperref[\detokenize{tech_note/Methane/CLM50_Tech_Note_Methane:ch4-and-o2-source-terms}]{\ref{\detokenize{tech_note/Methane/CLM50_Tech_Note_Methane:ch4-and-o2-source-terms}}}); calculate diffusivity
(section \hyperref[\detokenize{tech_note/Methane/CLM50_Tech_Note_Methane:aqueous-and-gaseous-diffusion}]{\ref{\detokenize{tech_note/Methane/CLM50_Tech_Note_Methane:aqueous-and-gaseous-diffusion}}}); and solve the resulting
mass balance using a tridiagonal solver (section
\hyperref[\detokenize{tech_note/Methane/CLM50_Tech_Note_Methane:crank-nicholson-solution-methane}]{\ref{\detokenize{tech_note/Methane/CLM50_Tech_Note_Methane:crank-nicholson-solution-methane}}}).


\subsubsection{Competition for CH$_{\text{4}}$ and O$_{\text{2}}$}
\label{\detokenize{tech_note/Methane/CLM50_Tech_Note_Methane:competition-for-ch4and-o2}}\label{\detokenize{tech_note/Methane/CLM50_Tech_Note_Methane:competition-for-ch4-and-o2}}
For each time step, the unlimited CH$_{\text{4}}$ and
O$_{\text{2}}$ demands in each model depth interval are computed. If
the total demand over a time step for one of the species exceeds the
amount available in a particular control volume, the demand from each
process associated with the sink is scaled by the fraction required to
ensure non-negative concentrations. Since the methanotrophs are limited
by both CH$_{\text{4}}$ and O$_{\text{2}}$, the stricter
limitation is applied to methanotroph oxidation, and then the
limitations are scaled back for the other processes. The competition is
designed so that the sinks must not exceed the available concentration
over the time step, and if any limitation exists, the sinks must sum to
this value. Because the sinks are calculated explicitly while the
transport is semi-implicit, negative concentrations can occur after the
tridiagonal solution. When this condition occurs for O$_{\text{2}}$,
the concentrations are reset to zero; if it occurs for
CH$_{\text{4}}$, the surface flux is adjusted and the concentration
is set to zero if the adjustment is not too large.


\subsubsection{CH$_{\text{4}}$ and O$_{\text{2}}$ Source Terms}
\label{\detokenize{tech_note/Methane/CLM50_Tech_Note_Methane:id7}}\label{\detokenize{tech_note/Methane/CLM50_Tech_Note_Methane:ch4-and-o2-source-terms}}
The overall CH$_{\text{4}}$ net source term consists of production,
oxidation at the base of aerenchyma, transport through aerenchyma,
methanotrophic oxidation, and ebullition (either to the control volume
above the water table if unsaturated or directly to the atmosphere if
saturated). For O$_{\text{2}}$ below the top control volume, the net
source term consists of O$_{\text{2}}$ losses from methanotrophy, SOM
decomposition, and autotrophic respiration, and an O$_{\text{2}}$
source through aerenchyma.


\subsubsection{Aqueous and Gaseous Diffusion}
\label{\detokenize{tech_note/Methane/CLM50_Tech_Note_Methane:aqueous-and-gaseous-diffusion}}\label{\detokenize{tech_note/Methane/CLM50_Tech_Note_Methane:id8}}
For gaseous diffusion, we adopted the temperature dependence of
molecular free-air diffusion coefficients (\({D}_{0}\)
(m:sup:\sphinxtitleref{2} s$^{\text{-1}}$)) as described by
{\hyperref[\detokenize{tech_note/References/CLM50_Tech_Note_References:lerman1979}]{\sphinxcrossref{\DUrole{std,std-ref}{Lerman (1979)}}}} and applied by
{\hyperref[\detokenize{tech_note/References/CLM50_Tech_Note_References:waniaetal2010}]{\sphinxcrossref{\DUrole{std,std-ref}{Wania et al. (2010)}}}}
(\hyperref[\detokenize{tech_note/Methane/CLM50_Tech_Note_Methane:table-temperature-dependence-of-aqueous-and-gaseous-diffusion}]{Table \ref{\detokenize{tech_note/Methane/CLM50_Tech_Note_Methane:table-temperature-dependence-of-aqueous-and-gaseous-diffusion}}}).


\begin{savenotes}\sphinxattablestart
\centering
\sphinxcapstartof{table}
\sphinxcaption{Temperature dependence of aqueous and gaseous diffusion coefficients for CH$_{\text{4}}$ and O$_{\text{2}}$}\label{\detokenize{tech_note/Methane/CLM50_Tech_Note_Methane:table-temperature-dependence-of-aqueous-and-gaseous-diffusion}}\label{\detokenize{tech_note/Methane/CLM50_Tech_Note_Methane:id15}}
\sphinxaftercaption
\begin{tabulary}{\linewidth}[t]{|T|T|T|}
\hline
\sphinxstylethead{\sphinxstyletheadfamily 
\({D}_{0}\) (m$^{\text{2}}$ s$^{\text{-1}}$)
\unskip}\relax &\sphinxstylethead{\sphinxstyletheadfamily 
CH$_{\text{4}}$
\unskip}\relax &\sphinxstylethead{\sphinxstyletheadfamily 
O$_{\text{2}}$
\unskip}\relax \\
\hline
Aqueous
&
0.9798 + 0.02986\sphinxstyleemphasis{T} + 0.0004381\sphinxstyleemphasis{T}$^{\text{2}}$
&
1.172+ 0.03443\sphinxstyleemphasis{T} + 0.0005048\sphinxstyleemphasis{T}$^{\text{2}}$
\\
\hline
Gaseous
&
0.1875 + 0.0013\sphinxstyleemphasis{T}
&
0.1759 + 0.0011\sphinxstyleemphasis{T}
\\
\hline
\end{tabulary}
\par
\sphinxattableend\end{savenotes}

Gaseous diffusivity in soils also depends on the molecular diffusivity,
soil structure, porosity, and organic matter content.
{\hyperref[\detokenize{tech_note/References/CLM50_Tech_Note_References:moldrupetal2003}]{\sphinxcrossref{\DUrole{std,std-ref}{Moldrup et al. (2003)}}}}, using observations across a
range of unsaturated mineral soils, showed that the relationship between
effective diffusivity (\(D_{e}\)  (m:sup:\sphinxtitleref{2} s$^{\text{-1}}$)) and soil
properties can be represented as:
\phantomsection\label{\detokenize{tech_note/Methane/CLM50_Tech_Note_Methane:equation-24.12}}\begin{equation}\label{equation:tech_note/Methane/CLM50_Tech_Note_Methane:24.12}
\begin{split}D_{e} =D_{0} \theta _{a}^{2} \left(\frac{\theta _{a} }{\theta _{s} } \right)^{{\raise0.7ex\hbox{$ 3 $}\!\mathord{\left/ {\vphantom {3 b}} \right. \kern-\nulldelimiterspace}\!\lower0.7ex\hbox{$ b $}} } ,\end{split}
\end{equation}
where \(\theta _{a}\)  and \(\theta _{s}\)  are the air-filled
and total (saturated water-filled) porosities (-), respectively, and \sphinxstyleemphasis{b}
is the slope of the water retention curve (-). However, {\hyperref[\detokenize{tech_note/References/CLM50_Tech_Note_References:iiyamahasegawa2005}]{\sphinxcrossref{\DUrole{std,std-ref}{Iiyama and
Hasegawa (2005)}}}} have shown that the original Millington-Quirk
({\hyperref[\detokenize{tech_note/References/CLM50_Tech_Note_References:millingtonquirk1961}]{\sphinxcrossref{\DUrole{std,std-ref}{Millington and Quirk 1961}}}}) relationship matched
measurements more closely in unsaturated peat soils:
\phantomsection\label{\detokenize{tech_note/Methane/CLM50_Tech_Note_Methane:equation-24.13}}\begin{equation}\label{equation:tech_note/Methane/CLM50_Tech_Note_Methane:24.13}
\begin{split}D_{e} =D_{0} \frac{\theta _{a} ^{{\raise0.7ex\hbox{$ 10 $}\!\mathord{\left/ {\vphantom {10 3}} \right. \kern-\nulldelimiterspace}\!\lower0.7ex\hbox{$ 3 $}} } }{\theta _{s} ^{2} }\end{split}
\end{equation}
In CLM, we applied equation for soils with zero organic matter content
and equation for soils with more than 130 kg m$^{\text{-3}}$ organic
matter content. A linear interpolation between these two limits is
applied for soils with SOM content below 130 kg m$^{\text{-3}}$. For
aqueous diffusion in the saturated part of the soil column, we applied
({\hyperref[\detokenize{tech_note/References/CLM50_Tech_Note_References:moldrupetal2003}]{\sphinxcrossref{\DUrole{std,std-ref}{Moldrup et al. (2003)}}}}):
\phantomsection\label{\detokenize{tech_note/Methane/CLM50_Tech_Note_Methane:equation-24.14}}\begin{equation}\label{equation:tech_note/Methane/CLM50_Tech_Note_Methane:24.14}
\begin{split}D_{e} =D_{0} \theta _{s} ^{2} .\end{split}
\end{equation}
To simplify the solution, we assumed that gaseous diffusion dominates
above the water table interface and aqueous diffusion below the water
table interface. Descriptions, baseline values, and dimensions for
parameters specific to the CH$_{\text{4}}$ model are given in
\hyperref[\detokenize{tech_note/Methane/CLM50_Tech_Note_Methane:table-methane-parameter-descriptions}]{Table \ref{\detokenize{tech_note/Methane/CLM50_Tech_Note_Methane:table-methane-parameter-descriptions}}}. For freezing or frozen
soils below the water table, diffusion is limited to the remaining
liquid (CLM allows for some freezing point depression), and the diffusion
coefficients are scaled by the
volume-fraction of liquid. For unsaturated soils, Henry’s law
equilibrium is assumed at the interface with the water table.


\subsubsection{Boundary Conditions}
\label{\detokenize{tech_note/Methane/CLM50_Tech_Note_Methane:boundary-conditions}}\label{\detokenize{tech_note/Methane/CLM50_Tech_Note_Methane:id9}}
We assume the CH$_{\text{4}}$ and O$_{\text{2}}$ surface fluxes
can be calculated from an effective conductance and a gaseous
concentration gradient between the atmospheric concentration and either
the gaseous concentration in the first soil layer (unsaturated soils) or
in equilibrium with the water (saturated
soil\(w\left(C_{1}^{n} -C_{a} \right)\) and
\(w\left(C_{1}^{n+1} -C_{a} \right)\) for the fully explicit and
fully implicit cases, respectively (however, see
{\hyperref[\detokenize{tech_note/References/CLM50_Tech_Note_References:tangriley2013}]{\sphinxcrossref{\DUrole{std,std-ref}{Tang and Riley (2013)}}}}
for a more complete representation of this process). Here, \sphinxstyleemphasis{w} is the
surface boundary layer conductance as calculated in the existing CLM
surface latent heat calculations. If the top layer is not fully
saturated, the \(\frac{D_{m1} }{\Delta x_{m1} }\)  term is replaced
with a series combination:
\(\left[\frac{1}{w} +\frac{\Delta x_{1} }{D_{1} } \right]^{-1}\) ,
and if the top layer is saturated, this term is replaced with
\(\left[\frac{K_{H} }{w} +\frac{\frac{1}{2} \Delta x_{1} }{D_{1} } \right]^{-1}\) ,
where \({K}_{H}\) is the Henry’s law equilibrium constant.

When snow is present, a resistance is added to account for diffusion
through the snow based on the Millington-Quirk expression \eqref{equation:tech_note/Methane/CLM50_Tech_Note_Methane:24.13}
and CLM’s prediction of the liquid water, ice, and air fractions of each
snow layer. When the soil is ponded, the diffusivity is assumed to be
that of methane in pure water, and the resistance as the ratio of the
ponding depth to diffusivity. The overall conductance is taken as the
series combination of surface, snow, and ponding resistances. We assume
a zero flux gradient at the bottom of the soil column.


\subsubsection{Crank-Nicholson Solution}
\label{\detokenize{tech_note/Methane/CLM50_Tech_Note_Methane:crank-nicholson-solution-methane}}\label{\detokenize{tech_note/Methane/CLM50_Tech_Note_Methane:crank-nicholson-solution}}
Equation is solved using a Crank-Nicholson solution
({\hyperref[\detokenize{tech_note/References/CLM50_Tech_Note_References:pressetal1992}]{\sphinxcrossref{\DUrole{std,std-ref}{Press et al. 1992}}}}),
which combines fully explicit and implicit representations of the mass
balance. The fully explicit decomposition of equation can be written as
\phantomsection\label{\detokenize{tech_note/Methane/CLM50_Tech_Note_Methane:equation-24.15}}\begin{equation}\label{equation:tech_note/Methane/CLM50_Tech_Note_Methane:24.15}
\begin{split}\frac{R_{j}^{n+1} C_{j}^{n+1} -R_{j}^{n} C_{j}^{n} }{\Delta t} =\frac{1}{\Delta x_{j} } \left[\frac{D_{p1}^{n} }{\Delta x_{p1}^{} } \left(C_{j+1}^{n} -C_{j}^{n} \right)-\frac{D_{m1}^{n} }{\Delta x_{m1}^{} } \left(C_{j}^{n} -C_{j-1}^{n} \right)\right]+S_{j}^{n} ,\end{split}
\end{equation}
where \sphinxstyleemphasis{j} refers to the cell in the vertically discretized soil column
(increasing downward), \sphinxstyleemphasis{n} refers to the current time step,
\(\Delta\)\sphinxstyleemphasis{t} is the time step (s), \sphinxstyleemphasis{p1} is \sphinxstyleemphasis{j+½}, \sphinxstyleemphasis{m1} is \sphinxstyleemphasis{j-½},
and \(S_{j}^{n}\)  is the net source at time step \sphinxstyleemphasis{n} and position
\sphinxstyleemphasis{j}, i.e.,
\(S_{j}^{n} =P\left(j,n\right)-E\left(j,n\right)-A\left(j,n\right)-O\left(j,n\right)\).
The diffusivity coefficients are calculated as harmonic means of values
from the adjacent cells. Equation is solved for gaseous and aqueous
concentrations above and below the water table, respectively. The \sphinxstyleemphasis{R}
term ensure the total mass balance in both phases is properly accounted
for. An analogous relationship can be generated for the fully implicit
case by replacing \sphinxstyleemphasis{n} by \sphinxstyleemphasis{n+1} on the \sphinxstyleemphasis{C} and \sphinxstyleemphasis{S} terms of equation .
Using an average of the fully implicit and fully explicit relationships
gives:
\phantomsection\label{\detokenize{tech_note/Methane/CLM50_Tech_Note_Methane:equation-24.16}}\begin{equation}\label{equation:tech_note/Methane/CLM50_Tech_Note_Methane:24.16}
\begin{split}\begin{array}{l} {-\frac{1}{2\Delta x_{j} } \frac{D_{m1}^{} }{\Delta x_{m1}^{} } C_{j-1}^{n+1} +\left[\frac{R_{j}^{n+1} }{\Delta t} +\frac{1}{2\Delta x_{j} } \left(\frac{D_{p1}^{} }{\Delta x_{p1}^{} } +\frac{D_{m1}^{} }{\Delta x_{m1}^{} } \right)\right]C_{j}^{n+1} -\frac{1}{2\Delta x_{j} } \frac{D_{p1}^{} }{\Delta x_{p1}^{} } C_{j+1}^{n+1} =} \\ {\frac{R_{j}^{n} }{\Delta t} +\frac{1}{2\Delta x_{j} } \left[\frac{D_{p1}^{} }{\Delta x_{p1}^{} } \left(C_{j+1}^{n} -C_{j}^{n} \right)-\frac{D_{m1}^{} }{\Delta x_{m1}^{} } \left(C_{j}^{n} -C_{j-1}^{n} \right)\right]+\frac{1}{2} \left[S_{j}^{n} +S_{j}^{n+1} \right]} \end{array},\end{split}
\end{equation}
Equation is solved with a standard tridiagonal solver, i.e.:
\phantomsection\label{\detokenize{tech_note/Methane/CLM50_Tech_Note_Methane:equation-24.17}}\begin{equation}\label{equation:tech_note/Methane/CLM50_Tech_Note_Methane:24.17}
\begin{split}aC_{j-1}^{n+1} +bC_{j}^{n+1} +cC_{j+1}^{n+1} =r,\end{split}
\end{equation}
with coefficients specified in equation .

Two methane balance checks are performed at each timestep to insure that
the diffusion solution and the time-varying aggregation over inundated
and non-inundated areas strictly conserves methane molecules (except for
production minus consumption) and carbon atoms.


\subsubsection{Interface between water table and unsaturated zone}
\label{\detokenize{tech_note/Methane/CLM50_Tech_Note_Methane:id10}}\label{\detokenize{tech_note/Methane/CLM50_Tech_Note_Methane:interface-between-water-table-and-unsaturated-zone}}
We assume Henry’s Law equilibrium at the interface between the saturated
and unsaturated zone and constant flux from the soil element below the
interface to the center of the soil element above the interface. In this
case, the coefficients are the same as described above, except for the
soil element above the interface:
\begin{equation*}
\begin{split}\frac{D_{p1} }{\Delta x_{p1} } =\left[K_{H} \frac{\Delta x_{j} }{2D_{j} } +\frac{\Delta x_{j+1} }{2D_{j+1} } \right]^{-1}\end{split}
\end{equation*}\begin{equation*}
\begin{split}b=\left[\frac{R_{j}^{n+1} }{\Delta t} +\frac{1}{2\Delta x_{j} } \left(K_{H} \frac{D_{p1}^{} }{\Delta x_{p1} } +\frac{D_{m1}^{} }{\Delta x_{m1} } \right)\right]\end{split}
\end{equation*}\phantomsection\label{\detokenize{tech_note/Methane/CLM50_Tech_Note_Methane:equation-24.18}}\begin{equation}\label{equation:tech_note/Methane/CLM50_Tech_Note_Methane:24.18}
\begin{split}r=\frac{R_{j}^{n} }{\Delta t} C_{j}^{n} +\frac{1}{2\Delta x_{j} } \left[\frac{D_{p1}^{} }{\Delta x_{p1} } \left(C_{j+1}^{n} -K_{H} C_{j}^{n} \right)-\frac{D_{m1}^{} }{\Delta x_{m1} } \left(C_{j}^{n} -C_{j-1}^{n} \right)\right]+\frac{1}{2} \left[S_{j}^{n} +S_{j}^{n+1} \right]\end{split}
\end{equation}
and the soil element below the interface:
\begin{equation*}
\begin{split}\frac{D_{m1} }{\Delta x_{m1} } =\left[K_{H} \frac{\Delta x_{j-1} }{2D_{j-1} } +\frac{\Delta x_{j} }{2D_{j} } \right]^{-1}\end{split}
\end{equation*}\begin{equation*}
\begin{split}a=-K_{H} \frac{1}{2\Delta x_{j} } \frac{D_{m1}^{} }{\Delta x_{m1} }\end{split}
\end{equation*}\phantomsection\label{\detokenize{tech_note/Methane/CLM50_Tech_Note_Methane:equation-24.19}}\begin{equation}\label{equation:tech_note/Methane/CLM50_Tech_Note_Methane:24.19}
\begin{split}r=\frac{R_{j}^{n} }{\Delta t} +C_{j}^{n} +\frac{1}{2\Delta x_{j} } \left[\frac{D_{p1}^{} }{\Delta x_{p1} } \left(C_{j+1}^{n} -C_{j}^{n} \right)-\frac{D_{m1}^{} }{\Delta x_{m1} } \left(C_{j}^{n} -K_{H} C_{j-1}^{n} \right)\right]+\frac{1}{2} \left[S_{j}^{n} +S_{j}^{n+1} \right]\end{split}
\end{equation}

\subsection{Inundated Fraction Prediction}
\label{\detokenize{tech_note/Methane/CLM50_Tech_Note_Methane:id11}}\label{\detokenize{tech_note/Methane/CLM50_Tech_Note_Methane:inundated-fraction-prediction}}
A simplified dynamic representation of spatial inundation
based on recent work by {\hyperref[\detokenize{tech_note/References/CLM50_Tech_Note_References:prigentetal2007}]{\sphinxcrossref{\DUrole{std,std-ref}{Prigent et al. (2007)}}}} is used.  Prigent et al. (2007) described a
multi-satellite approach to estimate the global monthly inundated
fraction (\({F}_{i}\)) over an equal area grid
(0.25 \(\circ\)  \(\times\)0.25\(\circ\) at the equator)
from 1993 - 2000. They suggested that the IGBP estimate for inundation
could be used as a measure of sensitivity of their detection approach at
low inundation. We therefore used the sum of their satellite-derived
\({F}_{i}\) and the constant IGBP estimate when it was less than
10\% to perform a simple inversion for the inundated fraction for methane
production (\({f}_{s}\)). The method optimized two parameters
(\({fws}_{slope}\) and \({fws}_{intercept}\)) for each
grid cell in a simple model based on simulated total water storage
(\({TWS}\)):
\phantomsection\label{\detokenize{tech_note/Methane/CLM50_Tech_Note_Methane:equation-24.20}}\begin{equation}\label{equation:tech_note/Methane/CLM50_Tech_Note_Methane:24.20}
\begin{split}f_{s} =fws_{slope} TWS  + fws_{intercept} .\end{split}
\end{equation}
These parameters were evaluated at the
0.5$^{\text{o}}$ resolution, and aggregated for
coarser simulations. Ongoing work in the hydrology
submodel of CLM may alleviate the need for this crude simplification of
inundated fraction in future model versions.


\subsection{Seasonal Inundation}
\label{\detokenize{tech_note/Methane/CLM50_Tech_Note_Methane:id12}}\label{\detokenize{tech_note/Methane/CLM50_Tech_Note_Methane:seasonal-inundation}}
A simple scaling factor is used to mimic the impact of
seasonal inundation on CH$_{\text{4}}$ production (see appendix B in
{\hyperref[\detokenize{tech_note/References/CLM50_Tech_Note_References:rileyetal2011a}]{\sphinxcrossref{\DUrole{std,std-ref}{Riley et al. (2011a)}}}} for a discussion of this
simplified expression):
\phantomsection\label{\detokenize{tech_note/Methane/CLM50_Tech_Note_Methane:equation-24.21}}\begin{equation}\label{equation:tech_note/Methane/CLM50_Tech_Note_Methane:24.21}
\begin{split}S=\frac{\beta \left(f-\bar{f}\right)+\bar{f}}{f} ,S\le 1.\end{split}
\end{equation}
Here, \sphinxstyleemphasis{f} is the instantaneous inundated fraction, \(\bar{f}\) is
the annual average inundated fraction (evaluated for the previous
calendar year) weighted by heterotrophic respiration, and
\(\beta\) is the anoxia factor that relates the fully anoxic
decomposition rate to the fully oxygen-unlimited decomposition rate, all
other conditions being equal.


\section{Crops and Irrigation}
\label{\detokenize{tech_note/Crop_Irrigation/CLM50_Tech_Note_Crop_Irrigation:rst-crops-and-irrigation}}\label{\detokenize{tech_note/Crop_Irrigation/CLM50_Tech_Note_Crop_Irrigation::doc}}\label{\detokenize{tech_note/Crop_Irrigation/CLM50_Tech_Note_Crop_Irrigation:crops-and-irrigation}}

\subsection{Summary of CLM5.0 updates relative to the CLM4.5}
\label{\detokenize{tech_note/Crop_Irrigation/CLM50_Tech_Note_Crop_Irrigation:summary-of-clm5-0-updates-relative-to-the-clm4-5}}\label{\detokenize{tech_note/Crop_Irrigation/CLM50_Tech_Note_Crop_Irrigation:id1}}
We describe here the complete crop and irrigation parameterizations that
appear in CLM5.0. Corresponding information for CLM4.5 appeared in the
CLM4.5 Technical Note ({\hyperref[\detokenize{tech_note/References/CLM50_Tech_Note_References:olesonetal2013}]{\sphinxcrossref{\DUrole{std,std-ref}{Oleson et al. 2013}}}}).

CLM5.0 includes the following new updates to the CROP option, where CROP
refers to the interactive crop management model and is included as an option with the BGC configuration:
\begin{itemize}
\item {} 
New crop functional types

\item {} 
All crop areas are actively managed

\item {} 
Fertilization rates updated based on crop type and geographic region

\item {} 
New Irrigation triggers

\item {} 
Phenological triggers vary by latitude for some crop types

\item {} 
Ability to simulate transient crop management

\item {} 
Adjustments to allocation and phenological parameters

\item {} 
Crops reaching their maximum LAI triggers the grain fill phase

\item {} 
Grain C and N pools are included in a 1-year product pool

\item {} 
C for annual crop seeding comes from the grain C pool

\item {} 
Initial seed C for planting is increased from 1 to 3 g C/m\textasciicircum{}2

\end{itemize}

These updates appear in detail in the sections below. Many also appear in
{\hyperref[\detokenize{tech_note/References/CLM50_Tech_Note_References:levisetal2016}]{\sphinxcrossref{\DUrole{std,std-ref}{Levis et al. (2016)}}}}.


\subsection{The crop model}
\label{\detokenize{tech_note/Crop_Irrigation/CLM50_Tech_Note_Crop_Irrigation:id2}}\label{\detokenize{tech_note/Crop_Irrigation/CLM50_Tech_Note_Crop_Irrigation:the-crop-model}}

\subsubsection{Introduction}
\label{\detokenize{tech_note/Crop_Irrigation/CLM50_Tech_Note_Crop_Irrigation:introduction}}
Groups developing Earth System Models generally account for the human
footprint on the landscape in simulations of historical and future
climates. Traditionally we have represented this footprint with natural
vegetation types and particularly grasses because they resemble many
common crops. Most modeling efforts have not incorporated more explicit
representations of land management such as crop type, planting,
harvesting, tillage, fertilization, and irrigation, because global scale
datasets of these factors have lagged behind vegetation mapping. As this
begins to change, we increasingly find models that will simulate the
biogeophysical and biogeochemical effects not only of natural but also
human-managed land cover.

AgroIBIS is a state-of-the-art land surface model with options to
simulate dynamic vegetation ({\hyperref[\detokenize{tech_note/References/CLM50_Tech_Note_References:kuchariketal2000}]{\sphinxcrossref{\DUrole{std,std-ref}{Kucharik et al. 2000}}}}) and interactive
crop management ({\hyperref[\detokenize{tech_note/References/CLM50_Tech_Note_References:kucharikbrye2003}]{\sphinxcrossref{\DUrole{std,std-ref}{Kucharik and Brye 2003}}}}). The interactive crop
management parameterizations from AgroIBIS (March 2003 version) were
coupled as a proof-of-concept to the Community Land Model version 3
{[}CLM3.0, {\hyperref[\detokenize{tech_note/References/CLM50_Tech_Note_References:olesonetal2004}]{\sphinxcrossref{\DUrole{std,std-ref}{Oleson et al. (2004)}}}} {]} (not published), then coupled to the
CLM3.5 ({\hyperref[\detokenize{tech_note/References/CLM50_Tech_Note_References:levisetal2009}]{\sphinxcrossref{\DUrole{std,std-ref}{Levis et al. 2009}}}}) and later released to the community with
CLM4CN ({\hyperref[\detokenize{tech_note/References/CLM50_Tech_Note_References:levisetal2012}]{\sphinxcrossref{\DUrole{std,std-ref}{Levis et al. 2012}}}}), and CLM4.5BGC. Additional updates after the
release of CLM4.5 were available by request ({\hyperref[\detokenize{tech_note/References/CLM50_Tech_Note_References:levisetal2016}]{\sphinxcrossref{\DUrole{std,std-ref}{Levis et al. 2016}}}}),
and those are now incorporated into CLM5.

With interactive crop management and, therefore, a more accurate
representation of agricultural landscapes, we hope to improve the CLM’s
simulated biogeophysics and biogeochemistry. These advances may improve
fully coupled simulations with the Community Earth System Model (CESM),
while helping human societies answer questions about changing food,
energy, and water resources in response to climate, environmental, land
use, and land management change (e.g., {\hyperref[\detokenize{tech_note/References/CLM50_Tech_Note_References:kucharikbrye2003}]{\sphinxcrossref{\DUrole{std,std-ref}{Kucharik and Brye 2003}}}}; {\hyperref[\detokenize{tech_note/References/CLM50_Tech_Note_References:lobelletal2006}]{\sphinxcrossref{\DUrole{std,std-ref}{Lobell et al. 2006}}}}).
As implemented here, the crop model uses the same physiology as the
natural vegetation, though uses different crop-specific parameter values,
phenology, and allocation, as well as fertilizer and irrigation management.


\subsubsection{Crop plant functional types}
\label{\detokenize{tech_note/Crop_Irrigation/CLM50_Tech_Note_Crop_Irrigation:id3}}\label{\detokenize{tech_note/Crop_Irrigation/CLM50_Tech_Note_Crop_Irrigation:crop-plant-functional-types}}
To allow crops to coexist with natural vegetation in a grid cell, the
vegetated land unit is separated into a naturally vegetated land unit and
a managed crop land unit. Unlike the plant functional types (pfts) in the
naturally vegetated land unit, the managed crop pfts in the managed crop
land unit do not share soil columns and thus permit for differences in the
land management between crops. Each crop type has a rainfed and an irrigated
pft that are on independent soil columns. Crop grid cell coverage is assigned from
satellite data (similar to all natural pfts), and the managed crop type
proportions within the crop area is based on the dataset created by
{\hyperref[\detokenize{tech_note/References/CLM50_Tech_Note_References:portmannetal2010}]{\sphinxcrossref{\DUrole{std,std-ref}{Portmann et al. (2010)}}}} for present day. New in CLM5, crop area is
extrapolated through time using the dataset provided by Land Use Model
Intercomparison Project (LUMIP), which is part of CMIP6 Land use timeseries
({\hyperref[\detokenize{tech_note/References/CLM50_Tech_Note_References:lawrenceetal2016}]{\sphinxcrossref{\DUrole{std,std-ref}{Lawrence et al. 2016}}}}). For more details about how
crop distributions are determined, see Chapter \hyperref[\detokenize{tech_note/Transient_Landcover/CLM50_Tech_Note_Transient_Landcover:rst-transient-landcover-change}]{\ref{\detokenize{tech_note/Transient_Landcover/CLM50_Tech_Note_Transient_Landcover:rst-transient-landcover-change}}}.

CLM5 includes eight actively managed crop types
(temperate soybean, tropical soybean, temperate corn, tropical
corn, spring wheat, cotton, rice, and sugarcane) that are chosen
based on the availability of corresponding algorithms in AgroIBIS and as
developed by {\hyperref[\detokenize{tech_note/References/CLM50_Tech_Note_References:badgeranddirmeyer2015}]{\sphinxcrossref{\DUrole{std,std-ref}{Badger and Dirmeyer (2015)}}}} and
described by {\hyperref[\detokenize{tech_note/References/CLM50_Tech_Note_References:levisetal2016}]{\sphinxcrossref{\DUrole{std,std-ref}{Levis et al. (2016)}}}}. The representations of
sugarcane, rice, cotton, tropical corn, and tropical soy are new in CLM5.
Sugarcane and tropical corn are both C4 plants and are therefore represented
using the temperate corn functional form. Tropical soybean uses the temperate
soybean functional form, while rice and cotton use the wheat functional form.
In tropical regions, parameter values were developed for the Amazon Basin, and planting
date window is shifted by six months relative to the Northern Hemisphere.

In addition, CLM’s default list of plant functional types (pfts) includes an
irrigated and unirrigated unmanaged C3 crop (\hyperref[\detokenize{tech_note/Crop_Irrigation/CLM50_Tech_Note_Crop_Irrigation:table-crop-plant-functional-types}]{Table \ref{\detokenize{tech_note/Crop_Irrigation/CLM50_Tech_Note_Crop_Irrigation:table-crop-plant-functional-types}}}) treated as a second C3 grass.
The unmanaged C3 crop is only used when the crop model is not active and
has grid cell coverage assigned from satellite data, and
the unmanaged C3 irrigated crop type is currently not used
since irrigation requires the crop model to be active.
The default list of pfts also includes twenty-three inactive crop pfts
that do not yet have associated parameters required for active management.
Each of the inactive crop types is simulated using the parameters of the
spatially closest associated crop type that is most similar to the functional type (e.g., C3 or C4),
which is required to maintain similar phenological parameters based on temperature thresholds.
Information detailing which parameters are used for each crop type is
included in \hyperref[\detokenize{tech_note/Crop_Irrigation/CLM50_Tech_Note_Crop_Irrigation:table-crop-plant-functional-types}]{Table \ref{\detokenize{tech_note/Crop_Irrigation/CLM50_Tech_Note_Crop_Irrigation:table-crop-plant-functional-types}}}. It should be noted that pft-level history output merges
all crop types into the actively managed crop type, so analysis
of crop-specific output will require use of the land surface dataset to
remap the yields of each actively and inactively managed crop type. Otherwise, the
actively managed crop type will include yields for that crop type and all inactively
managed crop types that are using the same parameter set.


\begin{savenotes}\sphinxatlongtablestart\begin{longtable}{|l|l|l|l|}
\caption{Crop plant functional types (pfts) included in CLM5BGCCROP.\strut}\label{\detokenize{tech_note/Crop_Irrigation/CLM50_Tech_Note_Crop_Irrigation:table-crop-plant-functional-types}}\label{\detokenize{tech_note/Crop_Irrigation/CLM50_Tech_Note_Crop_Irrigation:id20}}\\*[\sphinxlongtablecapskipadjust]
\hline
\sphinxstylethead{\sphinxstyletheadfamily 
ITV
\unskip}\relax &\sphinxstylethead{\sphinxstyletheadfamily 
Plant function types (PFTs)
\unskip}\relax &\sphinxstylethead{\sphinxstyletheadfamily 
Management Class
\unskip}\relax &\sphinxstylethead{\sphinxstyletheadfamily 
Crop Parameters Used
\unskip}\relax \\
\hline
\endfirsthead

\multicolumn{4}{c}%
{\makebox[0pt]{\sphinxtablecontinued{\tablename\ \thetable{} -- continued from previous page}}}\\
\hline
\sphinxstylethead{\sphinxstyletheadfamily 
ITV
\unskip}\relax &\sphinxstylethead{\sphinxstyletheadfamily 
Plant function types (PFTs)
\unskip}\relax &\sphinxstylethead{\sphinxstyletheadfamily 
Management Class
\unskip}\relax &\sphinxstylethead{\sphinxstyletheadfamily 
Crop Parameters Used
\unskip}\relax \\
\hline
\endhead

\hline
\multicolumn{4}{r}{\makebox[0pt][r]{\sphinxtablecontinued{Continued on next page}}}\\
\endfoot

\endlastfoot

15
&
c3 unmanaged rainfed crop
&
none
&
not applicable
\\
\hline
16
&
c3 unmanaged irrigated crop
&
none
&
not applicable
\\
\hline
17
&
rainfed temperate corn
&
active
&
rainfed temperate corn
\\
\hline
18
&
irrigated temperate corn
&
active
&
irrigated temperate corn
\\
\hline
19
&
rainfed spring wheat
&
active
&
rainfed spring wheat
\\
\hline
20
&
irrigated spring wheat
&
active
&
irrigated spring wheat
\\
\hline
21
&
rainfed winter wheat
&
inactive
&
rainfed spring wheat
\\
\hline
22
&
irrigated winter wheat
&
inactive
&
irrigated spring wheat
\\
\hline
23
&
rainfed temperate soybean
&
active
&
rainfed temperate soybean
\\
\hline
24
&
irrigated temperate soybean
&
active
&
irrigated temperate soybean
\\
\hline
25
&
rainfed barley
&
inactive
&
rainfed spring wheat
\\
\hline
26
&
irrigated barley
&
inactive
&
irrigated spring wheat
\\
\hline
27
&
rainfed winter barley
&
inactive
&
rainfed spring wheat
\\
\hline
28
&
irrigated winter barley
&
inactive
&
irrigated spring wheat
\\
\hline
29
&
rainfed rye
&
inactive
&
rainfed spring wheat
\\
\hline
30
&
irrigated rye
&
inactive
&
irrigated spring wheat
\\
\hline
31
&
rainfed winter rye
&
inactive
&
rainfed spring wheat
\\
\hline
32
&
irrigated winter rye
&
inactive
&
irrigated spring wheat
\\
\hline
33
&
rainfed cassava
&
inactive
&
rainfed rice
\\
\hline
34
&
irrigated cassava
&
inactive
&
irrigated rice
\\
\hline
35
&
rainfed citrus
&
inactive
&
rainfed spring wheat
\\
\hline
36
&
irrigated citrus
&
inactive
&
irrigated spring wheat
\\
\hline
37
&
rainfed cocoa
&
inactive
&
rainfed rice
\\
\hline
38
&
irrigated cocoa
&
inactive
&
irrigated rice
\\
\hline
39
&
rainfed coffee
&
inactive
&
rainfed rice
\\
\hline
40
&
irrigated coffee
&
inactive
&
irrigated rice
\\
\hline
41
&
rainfed cotton
&
active
&
rainfed cotton
\\
\hline
42
&
irrigated cotton
&
active
&
irrigated cotton
\\
\hline
43
&
rainfed datepalm
&
inactive
&
rainfed cotton
\\
\hline
44
&
irrigated datepalm
&
inactive
&
irrigated cotton
\\
\hline
45
&
rainfed foddergrass
&
inactive
&
rainfed spring wheat
\\
\hline
46
&
irrigated foddergrass
&
inactive
&
irrigated spring wheat
\\
\hline
47
&
rainfed grapes
&
inactive
&
rainfed spring wheat
\\
\hline
48
&
irrigated grapes
&
inactive
&
irrigated spring wheat
\\
\hline
49
&
rainfed groundnuts
&
inactive
&
rainfed rice
\\
\hline
50
&
irrigated groundnuts
&
inactive
&
irrigated rice
\\
\hline
51
&
rainfed millet
&
inactive
&
rainfed tropical corn
\\
\hline
52
&
irrigated millet
&
inactive
&
irrigated tropical corn
\\
\hline
53
&
rainfed oilpalm
&
inactive
&
rainfed rice
\\
\hline
54
&
irrigated oilpalm
&
inactive
&
irrigated rice
\\
\hline
55
&
rainfed potatoes
&
inactive
&
rainfed spring wheat
\\
\hline
56
&
irrigated potatoes
&
inactive
&
irrigated spring wheat
\\
\hline
57
&
rainfed pulses
&
inactive
&
rainfed spring wheat
\\
\hline
58
&
irrigated pulses
&
inactive
&
irrigated spring wheat
\\
\hline
59
&
rainfed rapeseed
&
inactive
&
rainfed spring wheat
\\
\hline
60
&
irrigated rapeseed
&
inactive
&
irrigated spring wheat
\\
\hline
61
&
rainfed rice
&
active
&
rainfed rice
\\
\hline
62
&
irrigated rice
&
active
&
irrigated rice
\\
\hline
63
&
rainfed sorghum
&
inactive
&
rainfed tropical corn
\\
\hline
64
&
irrigated sorghum
&
inactive
&
irrigated tropical corn
\\
\hline
65
&
rainfed sugarbeet
&
inactive
&
rainfed spring wheat
\\
\hline
66
&
irrigated sugarbeet
&
inactive
&
irrigated spring wheat
\\
\hline
67
&
rainfed sugarcane
&
active
&
rainfed sugarcane
\\
\hline
68
&
irrigated sugarcane
&
active
&
irrigated sugarcane
\\
\hline
69
&
rainfed sunflower
&
inactive
&
rainfed spring wheat
\\
\hline
70
&
irrigated sunflower
&
inactive
&
irrigated spring wheat
\\
\hline
71
&
rainfed miscanthus
&
inactive
&
rainfed tropical corn
\\
\hline
72
&
irrigated miscanthus
&
inactive
&
irrigated tropical corn
\\
\hline
73
&
rainfed switchgrass
&
inactive
&
rainfed tropical corn
\\
\hline
74
&
irrigated switchgrass
&
inactive
&
irrigated tropical corn
\\
\hline
75
&
rainfed tropical corn
&
active
&
rainfed tropical corn
\\
\hline
76
&
irrigated tropical corn
&
active
&
irrigated tropical corn
\\
\hline
77
&
rainfed tropical soybean
&
active
&
rainfed tropical soybean
\\
\hline
78
&
irrigated tropical soybean
&
active
&
irrigated tropical soybean
\\
\hline
\end{longtable}\sphinxatlongtableend\end{savenotes}


\subsubsection{Phenology}
\label{\detokenize{tech_note/Crop_Irrigation/CLM50_Tech_Note_Crop_Irrigation:id4}}\label{\detokenize{tech_note/Crop_Irrigation/CLM50_Tech_Note_Crop_Irrigation:phenology}}
CLM5-BGC includes evergreen, seasonally deciduous (responding to changes
in day length), and stress deciduous (responding to changes in
temperature and/or soil moisture) phenology algorithms (Chapter \hyperref[\detokenize{tech_note/Vegetation_Phenology_Turnover/CLM50_Tech_Note_Vegetation_Phenology_Turnover:rst-vegetation-phenology-and-turnover}]{\ref{\detokenize{tech_note/Vegetation_Phenology_Turnover/CLM50_Tech_Note_Vegetation_Phenology_Turnover:rst-vegetation-phenology-and-turnover}}}).
CLM5-BGC-crop uses the AgroIBIS crop phenology algorithm,
consisting of three distinct phases.

Phase 1 starts at planting and ends with leaf emergence, phase 2
continues from leaf emergence to the beginning of grain fill, and phase
3 starts from the beginning of grain fill and ends with physiological
maturity and harvest.


\paragraph{Planting}
\label{\detokenize{tech_note/Crop_Irrigation/CLM50_Tech_Note_Crop_Irrigation:planting}}\label{\detokenize{tech_note/Crop_Irrigation/CLM50_Tech_Note_Crop_Irrigation:id5}}
All crops must meet the following requirements between the minimum planting date and the maximum
planting date (for the northern hemisphere) in \hyperref[\detokenize{tech_note/Crop_Irrigation/CLM50_Tech_Note_Crop_Irrigation:table-crop-phenology-parameters}]{Table \ref{\detokenize{tech_note/Crop_Irrigation/CLM50_Tech_Note_Crop_Irrigation:table-crop-phenology-parameters}}}:
\phantomsection\label{\detokenize{tech_note/Crop_Irrigation/CLM50_Tech_Note_Crop_Irrigation:equation-25.1}}\begin{equation}\label{equation:tech_note/Crop_Irrigation/CLM50_Tech_Note_Crop_Irrigation:25.1}
\begin{split}\begin{array}{c}
{T_{10d} >T_{p} } \\
{T_{10d}^{\min } >T_{p}^{\min } }  \\
{GDD_{8} \ge GDD_{\min } }
\end{array}\end{split}
\end{equation}
where \({T}_{10d}\) is the 10-day running mean of \({T}_{2m}\), (the simulated 2-m air
temperature during each model time step) and \(T_{10d}^{\min}\)  is
the 10-day running mean of \(T_{2m}^{\min }\)  (the daily minimum of
\({T}_{2m}\)). \({T}_{p}\) and \(T_{p}^{\min }\)  are crop-specific coldest planting temperatures
(\hyperref[\detokenize{tech_note/Crop_Irrigation/CLM50_Tech_Note_Crop_Irrigation:table-crop-phenology-parameters}]{Table \ref{\detokenize{tech_note/Crop_Irrigation/CLM50_Tech_Note_Crop_Irrigation:table-crop-phenology-parameters}}}), \({GDD}_{8}\) is the 20-year running mean growing
degree-days (units are degree-days or $^{\text{o}}$ days) tracked
from April through September (NH) above 8$^{\text{o}}$ C with
maximum daily increments of 30$^{\text{o}}$ days (see equation \eqref{equation:tech_note/Crop_Irrigation/CLM50_Tech_Note_Crop_Irrigation:25.3}), and
\({GDD}_{min }\)is the minimum growing degree day requirement
(\hyperref[\detokenize{tech_note/Crop_Irrigation/CLM50_Tech_Note_Crop_Irrigation:table-crop-phenology-parameters}]{Table \ref{\detokenize{tech_note/Crop_Irrigation/CLM50_Tech_Note_Crop_Irrigation:table-crop-phenology-parameters}}}). \({GDD}_{8}\) does not change as quickly as \({T}_{10d}\) and \(T_{10d}^{\min }\), so
it determines whether it is warm enough for the crop to be planted in a grid cell, while the
2-m air temperature variables determine the day when the crop may be planted if the \({GDD}_{8}\) threshold is met.
If the requirements in equation \eqref{equation:tech_note/Crop_Irrigation/CLM50_Tech_Note_Crop_Irrigation:25.1} are not met by the maximum planting date,
crops are still planted on the maximum planting date as long as  \({GDD}_{8} > 0\). In
the southern hemisphere (SH) the NH requirements apply 6 months later.

At planting, each crop seed pool is assigned 3 gC m$^{\text{-2}}$ from its
grain product pool. The seed carbon is transferred to the leaves upon leaf emergence. An
equivalent amount of seed leaf N is assigned given the pft’s C to N
ratio for leaves (\({CN}_{leaf}\) in \hyperref[\detokenize{tech_note/Crop_Irrigation/CLM50_Tech_Note_Crop_Irrigation:table-crop-allocation-parameters}]{Table \ref{\detokenize{tech_note/Crop_Irrigation/CLM50_Tech_Note_Crop_Irrigation:table-crop-allocation-parameters}}}; this differs from AgroIBIS,
which uses a seed leaf area index instead of seed C). The model updates the average growing degree-days necessary
for the crop to reach vegetative and physiological maturity,
\({GDD}_{mat}\), according to the following AgroIBIS rules:
\phantomsection\label{\detokenize{tech_note/Crop_Irrigation/CLM50_Tech_Note_Crop_Irrigation:equation-25.2}}\begin{equation}\label{equation:tech_note/Crop_Irrigation/CLM50_Tech_Note_Crop_Irrigation:25.2}
\begin{split}\begin{array}{lll}
GDD_{{\rm mat}}^{{\rm corn,sugarcane}} =0.85 GDD_{{\rm 8}} & {\rm \; \; \; and\; \; \; }& 950 <GDD_{{\rm mat}}^{{\rm corn,sugarcane}} <1850{}^\circ {\rm days} \\
GDD_{{\rm mat}}^{{\rm spring\ wheat,cotton}} =GDD_{{\rm 0}} & {\rm \; \; \; and\; \; \; } & GDD_{{\rm mat}}^{{\rm spring\ wheat,cotton}} <1700{}^\circ {\rm days} \\
GDD_{{\rm mat}}^{{\rm temp.soy}} =GDD_{{\rm 10}} & {\rm \; \; \; and\; \; \; } & GDD_{{\rm mat}}^{{\rm temp.soy}} <1900{}^\circ {\rm days} \\
GDD_{{\rm mat}}^{{\rm rice}} =GDD_{{\rm 0}} & {\rm \; \; \; and\; \; \; } & GDD_{{\rm mat}}^{{\rm rice}} <2100{}^\circ {\rm days} \\
GDD_{{\rm mat}}^{{\rm trop.soy}} =GDD_{{\rm 10}} & {\rm \; \; \; and\; \; \; } & GDD_{{\rm mat}}^{{\rm trop.soy}} <2100{}^\circ {\rm days}
\end{array}\end{split}
\end{equation}
where \({GDD}_{0}\), \({GDD}_{8}\), and \({GDD}_{10}\) are the 20-year running mean growing
degree-days tracked from April through September (NH) over 0$^{\text{o}}$C, 8$^{\text{o}}$C, and
10$^{\text{o}}$C, respectively, with maximum daily increments of
26$^{\text{o}}$days (for \({GDD}_{0}\)) or 30$^{\text{o}}$days (for \({GDD}_{8}\) and \({GDD}_{10}\)). Equation \eqref{equation:tech_note/Crop_Irrigation/CLM50_Tech_Note_Crop_Irrigation:25.3} shows how we calculate
\({GDD}_{0}\), \({GDD}_{8}\), and \({GDD}_{10}\) for each model timestep:
\phantomsection\label{\detokenize{tech_note/Crop_Irrigation/CLM50_Tech_Note_Crop_Irrigation:equation-25.3}}\begin{equation}\label{equation:tech_note/Crop_Irrigation/CLM50_Tech_Note_Crop_Irrigation:25.3}
\begin{split}\begin{array}{lll}
GDD_{{\rm 0}} =GDD_{0} +T_{2{\rm m}} -T_{f} & \quad {\rm \; \; \; where\; \; \; } & 0 \le T_{2{\rm m}} -T_{f} \le 26{}^\circ {\rm days} \\
GDD_{{\rm 8}} =GDD_{8} +T_{2{\rm m}} -T_{f} -8 & \quad {\rm \; \; \; where\; \; \; } & 0 \le T_{2{\rm m}} -T_{f} -8\le 30{}^\circ {\rm days} \\
GDD_{{\rm 10}} =GDD_{10} +T_{2{\rm m}} -T_{f} -10 & \quad {\rm \; \; \; where\; \; \; } & 0 \le T_{2{\rm m}} -T_{f} -10\le 30{}^\circ {\rm days}
\end{array}\end{split}
\end{equation}
where, if \({T}_{2m}\) -  \({T}_{f}\) takes on values
outside the above ranges within a day, then it equals the minimum or maximum value in
the range for that day. \({T}_{f}\) is the freezing temperature of water and equals 273.15 K,
\({T}_{2m}\) is the 2-m air temperature in units of K, and \sphinxstyleemphasis{GDD} is in units of ºdays.


\paragraph{Leaf emergence}
\label{\detokenize{tech_note/Crop_Irrigation/CLM50_Tech_Note_Crop_Irrigation:id6}}\label{\detokenize{tech_note/Crop_Irrigation/CLM50_Tech_Note_Crop_Irrigation:leaf-emergence}}
According to AgroIBIS, leaves may emerge when the growing degree-days of
soil temperature to 0.05 m depth (\(GDD_{T_{soi} }\) ), which is tracked since planting,
reaches 1 to 5\% of \({GDD}_{mat}\)
(see Phase 2 \% \({GDD}_{mat}\) in \hyperref[\detokenize{tech_note/Crop_Irrigation/CLM50_Tech_Note_Crop_Irrigation:table-crop-phenology-parameters}]{Table \ref{\detokenize{tech_note/Crop_Irrigation/CLM50_Tech_Note_Crop_Irrigation:table-crop-phenology-parameters}}}). The base temperature threshold values for \(GDD_{T_{soi} }\)
are listed in \hyperref[\detokenize{tech_note/Crop_Irrigation/CLM50_Tech_Note_Crop_Irrigation:table-crop-phenology-parameters}]{Table \ref{\detokenize{tech_note/Crop_Irrigation/CLM50_Tech_Note_Crop_Irrigation:table-crop-phenology-parameters}}} (the same base temperature threshold values are also used for
\(GDD_{T_{{\rm 2m}} }\) in section \hyperref[\detokenize{tech_note/Crop_Irrigation/CLM50_Tech_Note_Crop_Irrigation:grain-fill}]{\ref{\detokenize{tech_note/Crop_Irrigation/CLM50_Tech_Note_Crop_Irrigation:grain-fill}}}), and leaf emergence (crop phenology phase 2)
starts when this threshold is met. Leaf onset occurs in the first
time step of phase 2, at which moment all seed C is transferred to leaf
C. Subsequently, the leaf area index generally increases throughout phase 2 until it reaches
a predetermined maximum value. Stem and root C also increase throughout phase 2 based on
the carbon allocation algorithm in section \hyperref[\detokenize{tech_note/Crop_Irrigation/CLM50_Tech_Note_Crop_Irrigation:leaf-emergence-to-grain-fill}]{\ref{\detokenize{tech_note/Crop_Irrigation/CLM50_Tech_Note_Crop_Irrigation:leaf-emergence-to-grain-fill}}}.


\paragraph{Grain fill}
\label{\detokenize{tech_note/Crop_Irrigation/CLM50_Tech_Note_Crop_Irrigation:id7}}\label{\detokenize{tech_note/Crop_Irrigation/CLM50_Tech_Note_Crop_Irrigation:grain-fill}}
The grain fill phase (phase 3) begins in one of two ways. The first potential trigger is based on temperature, similar to phase 2. A variable tracked since
planting, similar to \(GDD_{T_{soi} }\)  but for 2-m air temperature,
\(GDD_{T_{{\rm 2m}} }\), must reach a heat unit threshold, \sphinxstyleemphasis{h}, of
of 40 to 65\% of  \({GDD}_{mat}\) (see Phase 3 \% \({GDD}_{mat}\) in \hyperref[\detokenize{tech_note/Crop_Irrigation/CLM50_Tech_Note_Crop_Irrigation:table-crop-phenology-parameters}]{Table \ref{\detokenize{tech_note/Crop_Irrigation/CLM50_Tech_Note_Crop_Irrigation:table-crop-phenology-parameters}}}).
For crops with the C4 photosynthetic pathway (temperate and tropical corn, sugarcane),
the \({GDD}_{mat}\) is based on an empirical function and ranges between 950 and 1850.
The second potential trigger for phase 3 is based on leaf area index.
When the maximum value of leaf area index is reached in phase 2 (\hyperref[\detokenize{tech_note/Crop_Irrigation/CLM50_Tech_Note_Crop_Irrigation:table-crop-allocation-parameters}]{Table \ref{\detokenize{tech_note/Crop_Irrigation/CLM50_Tech_Note_Crop_Irrigation:table-crop-allocation-parameters}}}), phase 3 begins.
In phase 3, the leaf area index begins to decline in
response to a background litterfall rate calculated as the inverse of
leaf longevity for the pft as done in the BGC part of the model.


\paragraph{Harvest}
\label{\detokenize{tech_note/Crop_Irrigation/CLM50_Tech_Note_Crop_Irrigation:harvest}}\label{\detokenize{tech_note/Crop_Irrigation/CLM50_Tech_Note_Crop_Irrigation:id8}}
Harvest is assumed to occur as soon as the crop reaches maturity. When
\(GDD_{T_{{\rm 2m}} }\) reaches 100\% of \({GDD}_{mat}\) or
the number of days past planting reaches a crop-specific maximum
(\hyperref[\detokenize{tech_note/Crop_Irrigation/CLM50_Tech_Note_Crop_Irrigation:table-crop-phenology-parameters}]{Table \ref{\detokenize{tech_note/Crop_Irrigation/CLM50_Tech_Note_Crop_Irrigation:table-crop-phenology-parameters}}}), then the crop is harvested.
Harvest occurs in one time step using the BGC leaf offset algorithm.


\begin{savenotes}\sphinxattablestart
\centering
\sphinxcapstartof{table}
\sphinxcaption{Crop phenology and morphology parameters for the active crop plant functional types (pfts) in CLM5BGCCROP. Numbers in the first row correspond to the list of pfts in \hyperref[\detokenize{tech_note/Crop_Irrigation/CLM50_Tech_Note_Crop_Irrigation:table-crop-plant-functional-types}]{Table \ref{\detokenize{tech_note/Crop_Irrigation/CLM50_Tech_Note_Crop_Irrigation:table-crop-plant-functional-types}}}.}\label{\detokenize{tech_note/Crop_Irrigation/CLM50_Tech_Note_Crop_Irrigation:table-crop-phenology-parameters}}\label{\detokenize{tech_note/Crop_Irrigation/CLM50_Tech_Note_Crop_Irrigation:id21}}
\sphinxaftercaption
\begin{tabulary}{\linewidth}[t]{|T|T|T|T|T|T|T|T|T|}
\hline


&\sphinxstylethead{\sphinxstyletheadfamily 
temperate corn
\unskip}\relax &\sphinxstylethead{\sphinxstyletheadfamily 
spring wheat
\unskip}\relax &\sphinxstylethead{\sphinxstyletheadfamily 
temperatue soybean
\unskip}\relax &\sphinxstylethead{\sphinxstyletheadfamily 
cotton
\unskip}\relax &\sphinxstylethead{\sphinxstyletheadfamily 
rice
\unskip}\relax &\sphinxstylethead{\sphinxstyletheadfamily 
sugarcane
\unskip}\relax &\sphinxstylethead{\sphinxstyletheadfamily 
tropical corn
\unskip}\relax &\sphinxstylethead{\sphinxstyletheadfamily 
tropical soybean
\unskip}\relax \\
\hline
IVT
&
17, 18
&
19, 20
&
23, 24
&
41, 42
&
61, 62
&
67, 68
&
75, 76
&
77, 78
\\
\hline
\(Date_{planting}^{min}\)
&
April 1
&
April 1
&
May 1
&
April 1
&
Janurary 1
&
Janurary 1
&
March 20
&
April 15
\\
\hline
\(Date_{planting}^{max}\)
&
June 15
&
June  15
&
June 15
&
May 31
&
Feburary 28
&
March 31
&
April 15
&
June 31
\\
\hline
\(T_{p}\)(K)
&
283.15
&
280.15
&
286.15
&
294.15
&
294.15
&
294.15
&
294.15
&
294.15
\\
\hline
\(T_{p}^{ min }\)(K)
&
279.15
&
272.15
&
279.15
&
283.15
&
283.15
&
283.15
&
283.15
&
283.15
\\
\hline
\({GDD}_{min}\)(ºdays)
&
50
&
50
&
50
&
50
&
50
&
50
&
50
&
50
\\
\hline
base temperature for GDD (ºC)
&
8
&
0
&
10
&
10
&
10
&
10
&
10
&
10
\\
\hline
\({GDD}_{mat}\)(ºdays)
&
950-1850
&
\(\mathrm{\le}\)1700
&
\(\mathrm{\le}\)1900
&
\(\mathrm{\le}\)1700
&
\(\mathrm{\le}\)2100
&
950-1850
&
950-1850
&
\(\mathrm{\le}\)2100
\\
\hline
Phase 2 \% \({GDD}_{mat}\)
&
0.03
&
0.05
&
0.03
&
0.03
&
0.01
&
0.03
&
0.03
&
0.03
\\
\hline
Phase 3 \% \({GDD}_{mat}\)
&
0.65
&
0.6
&
0.5
&
0.5
&
0.4
&
0.65
&
0.5
&
0.5
\\
\hline
Harvest: days past planting
&
\(\mathrm{\le}\)165
&
\(\mathrm{\le}\)150
&
\(\mathrm{\le}\)150
&
\(\mathrm{\le}\)160
&
\(\mathrm{\le}\)150
&
\(\mathrm{\le}\)300
&
\(\mathrm{\le}\)160
&
\(\mathrm{\le}\)150
\\
\hline
\(z_{top}^{\max }\) (m)
&
2.5
&
1.2
&
0.75
&
1.5
&
1.8
&
4
&
2.5
&
1
\\
\hline
SLA (m $^{\text{2}}$ leaf g $^{\text{-1}}$ C)
&
0.05
&
0.035
&
0.035
&
0.035
&
0.035
&
0.05
&
0.05
&
0.035
\\
\hline
\(\chi _{L}\) index
&
-0.5
&
-0.5
&
-0.5
&
-0.5
&
-0.5
&
-0.5
&
-0.5
&
-0.5
\\
\hline
grperc
&
0.11
&
0.11
&
0.11
&
0.11
&
0.11
&
0.11
&
0.11
&
0.11
\\
\hline
flnr
&
0.293
&
0.41
&
0.41
&
0.41
&
0.41
&
0.293
&
0.293
&
0.41
\\
\hline
fcur
&
1
&
1
&
1
&
1
&
1
&
1
&
1
&
1
\\
\hline
\end{tabulary}
\par
\sphinxattableend\end{savenotes}

Notes: \(Date_{planting}^{min}\) and \(Date_{planting}^{max}\) are
the minimum and maximum planting date in the Northern Hemisphere, the corresponding dates
in the Southern Hemisphere apply 6 months later.
\(T_{p}\) and \(T_{p}^{ min }\) are crop-specific coldest planting temperatures.
\({GDD}_{min}\) is the lowest (for planting) 20-year running mean growing degree-days based
on the base temperature threshold in the 7$^{\text{th}}$ row, tracked from April to September (NH).
\({GDD}_{mat}\) is a crop’s 20-year running mean growing
degree-days needed for vegetative and physiological maturity. Harvest
occurs at 100\%\({GDD}_{mat}\) or when the days past planting
reach the number in the 11$^{\text{th}}$ row. Crop growth phases
are described in the text. \(z_{top}^{\max }\)  is the maximum
top-of-canopy height of a crop, \sphinxstyleemphasis{SLA} is specific leaf area. \(\chi _{L}\) is the leaf
orientation index, equals -1 for vertical, 0 for
random, and 1 for horizontal leaf orientation.
grperc is the growth respiration factor. flnr is the fraction of leaf N in the Rubisco enzyme.
fcur is the fraction of allocation that goes to currently displayed growth.


\subsubsection{Allocation}
\label{\detokenize{tech_note/Crop_Irrigation/CLM50_Tech_Note_Crop_Irrigation:allocation}}\label{\detokenize{tech_note/Crop_Irrigation/CLM50_Tech_Note_Crop_Irrigation:id9}}
Allocation changes based on the crop phenology phases phenology (section \hyperref[\detokenize{tech_note/Crop_Irrigation/CLM50_Tech_Note_Crop_Irrigation:phenology}]{\ref{\detokenize{tech_note/Crop_Irrigation/CLM50_Tech_Note_Crop_Irrigation:phenology}}}).
Simulated C assimilation begins every year upon leaf emergence in phase
2 and ends with harvest at the end of phase 3; therefore, so does the
allocation of such C to the crop’s leaf, live stem, fine root, and
reproductive pools.

Typically, C:N ratios in plant tissue vary throughout the growing season and
tend to be lower during early growth stages and higher in later growth stages.
In order to account for this seasonal change, two sets of C:N
ratios are established in CLM for the leaf, stem, and fine root of
crops: one during the leaf emergence phase (phenology phase 2), and a second during
grain fill phase (phenology phase 3). This modified C:N ratio approach accounts for the nitrogen
retranslocation that occurs during the grain fill phase (phase 3) of crop growth. Leaf, stem, and root
C:N ratios for phase 2 are calculated
using the new CLM5 carbon and nitrogen allocation scheme
(Chapter \hyperref[\detokenize{tech_note/CN_Allocation/CLM50_Tech_Note_CN_Allocation:rst-cn-allocation}]{\ref{\detokenize{tech_note/CN_Allocation/CLM50_Tech_Note_CN_Allocation:rst-cn-allocation}}}), which provides a target C:N value
(\hyperref[\detokenize{tech_note/Crop_Irrigation/CLM50_Tech_Note_Crop_Irrigation:table-crop-allocation-parameters}]{Table \ref{\detokenize{tech_note/Crop_Irrigation/CLM50_Tech_Note_Crop_Irrigation:table-crop-allocation-parameters}}}) and allows C:N to vary through time.
During grain fill (phase 3) of the crop growth cycle, a portion of the
nitrogen in the plant tissues is moved to a storage pool to fulfill
nitrogen demands of organ (reproductive pool) development, such that the
resulting C:N ratio of the plant tissue is reflective of measurements at
harvest. All C:N ratios were determined by calibration process, through
comparisons of model output versus observations of plant carbon
throughout the growing season.

The BGC part of the model keeps track of a term representing excess
maintenance respiration, which supplies the carbon required for maintenance respiration during periods of
low photosynthesis (Chapter \hyperref[\detokenize{tech_note/Plant_Respiration/CLM50_Tech_Note_Plant_Respiration:rst-plant-respiration}]{\ref{\detokenize{tech_note/Plant_Respiration/CLM50_Tech_Note_Plant_Respiration:rst-plant-respiration}}}).
Carbon supply for excess maintenance respiration
cannot continue to happen after harvest for annual crops, so at harvest
the excess respiration pool is turned into a flux that extracts
CO$_{\text{2}}$ directly from the atmosphere. This way
any excess maintenance respiration remaining at harvest is eliminated as if such
respiration had not taken place.


\paragraph{Leaf emergence}
\label{\detokenize{tech_note/Crop_Irrigation/CLM50_Tech_Note_Crop_Irrigation:id10}}\label{\detokenize{tech_note/Crop_Irrigation/CLM50_Tech_Note_Crop_Irrigation:leaf-emergence-to-grain-fill}}
During phase 2, the allocation coefficients (fraction of available C) to
each C pool are defined as:
\phantomsection\label{\detokenize{tech_note/Crop_Irrigation/CLM50_Tech_Note_Crop_Irrigation:equation-25.4}}\begin{equation}\label{equation:tech_note/Crop_Irrigation/CLM50_Tech_Note_Crop_Irrigation:25.4}
\begin{split}\begin{array}{l} {a_{repr} =0} \\ {a_{froot} =a_{froot}^{i} -(a_{froot}^{i} -a_{froot}^{f} )\frac{GDD_{T_{{\rm 2m}} } }{GDD_{{\rm mat}} } {\rm \; \; \; where\; \; \; }\frac{GDD_{T_{{\rm 2m}} } }{GDD_{{\rm mat}} } \le 1} \\ {a_{leaf} =(1-a_{froot} )\cdot \frac{a_{leaf}^{i} (e^{-b} -e^{-b\frac{GDD_{T_{{\rm 2m}} } }{h} } )}{e^{-b} -1} {\rm \; \; \; where\; \; \; }b=0.1} \\ {a_{livestem} =1-a_{repr} -a_{froot} -a_{leaf} } \end{array}\end{split}
\end{equation}
where \(a_{leaf}^{i}\) , \(a_{froot}^{i}\) , and
\(a_{froot}^{f}\)  are initial and final values of these
coefficients (\hyperref[\detokenize{tech_note/Crop_Irrigation/CLM50_Tech_Note_Crop_Irrigation:table-crop-allocation-parameters}]{Table \ref{\detokenize{tech_note/Crop_Irrigation/CLM50_Tech_Note_Crop_Irrigation:table-crop-allocation-parameters}}}), and \sphinxstyleemphasis{h} is a heat unit threshold defined in
section \hyperref[\detokenize{tech_note/Crop_Irrigation/CLM50_Tech_Note_Crop_Irrigation:grain-fill}]{\ref{\detokenize{tech_note/Crop_Irrigation/CLM50_Tech_Note_Crop_Irrigation:grain-fill}}}. At a crop-specific maximum leaf area index,
\({L}_{max}\) (\hyperref[\detokenize{tech_note/Crop_Irrigation/CLM50_Tech_Note_Crop_Irrigation:table-crop-allocation-parameters}]{Table \ref{\detokenize{tech_note/Crop_Irrigation/CLM50_Tech_Note_Crop_Irrigation:table-crop-allocation-parameters}}}), carbon allocation is directed
exclusively to the fine roots.


\paragraph{Grain fill}
\label{\detokenize{tech_note/Crop_Irrigation/CLM50_Tech_Note_Crop_Irrigation:id11}}\label{\detokenize{tech_note/Crop_Irrigation/CLM50_Tech_Note_Crop_Irrigation:grain-fill-to-harvest}}
The calculation of \(a_{froot}\)  remains the same from phase 2 to
phase 3. During grain fill (phase 3), other allocation coefficients change to:
\phantomsection\label{\detokenize{tech_note/Crop_Irrigation/CLM50_Tech_Note_Crop_Irrigation:equation-25.5}}\begin{equation}\label{equation:tech_note/Crop_Irrigation/CLM50_Tech_Note_Crop_Irrigation:25.5}
\begin{split}\begin{array}{ll}
a_{leaf} =a_{leaf}^{i,3} & {\rm when} \quad a_{leaf}^{i,3} \le a_{leaf}^{f} \quad {\rm else} \\
a_{leaf} =a_{leaf} \left(1-\frac{GDD_{T_{{\rm 2m}} } -h}{GDD_{{\rm mat}} d_{L} -h} \right)^{d_{alloc}^{leaf} } \ge a_{leaf}^{f} & {\rm where} \quad \frac{GDD_{T_{{\rm 2m}} } -h}{GDD_{{\rm mat}} d_{L} -h} \le 1 \\
 \\
a_{livestem} =a_{livestem}^{i,3} & {\rm when} \quad a_{livestem}^{i,3} \le a_{livestem}^{f} \quad {\rm else} \\
a_{livestem} =a_{livestem} \left(1-\frac{GDD_{T_{{\rm 2m}} } -h}{GDD_{{\rm mat}} d_{L} -h} \right)^{d_{alloc}^{stem} } \ge a_{livestem}^{f} & {\rm where} \quad \frac{GDD_{T_{{\rm 2m}} } -h}{GDD_{{\rm mat}} d_{L} -h} \le 1 \\
 \\
a_{repr} =1-a_{froot} -a_{livestem} -a_{leaf}
\end{array}\end{split}
\end{equation}
where \(a_{leaf}^{i,3}\)  and \(a_{livestem}^{i,3}\)  (initial
values) equal the last \(a_{leaf}\)  and \(a_{livestem}\)
calculated in phase 2, \(d_{L}\) , \(d_{alloc}^{leaf}\)  and
\(d_{alloc}^{stem}\)  are leaf area index and leaf and stem
allocation decline factors, and \(a_{leaf}^{f}\)  and
\(a_{livestem}^{f}\)  are final values of these allocation
coefficients (\hyperref[\detokenize{tech_note/Crop_Irrigation/CLM50_Tech_Note_Crop_Irrigation:table-crop-allocation-parameters}]{Table \ref{\detokenize{tech_note/Crop_Irrigation/CLM50_Tech_Note_Crop_Irrigation:table-crop-allocation-parameters}}}).


\paragraph{Nitrogen retranslocation for crops}
\label{\detokenize{tech_note/Crop_Irrigation/CLM50_Tech_Note_Crop_Irrigation:id12}}\label{\detokenize{tech_note/Crop_Irrigation/CLM50_Tech_Note_Crop_Irrigation:nitrogen-retranslocation-for-crops}}
Nitrogen retranslocation in crops occurs when nitrogen that was used for
tissue growth of leaves, stems, and fine roots during the early growth
season is remobilized and used for grain development ({\hyperref[\detokenize{tech_note/References/CLM50_Tech_Note_References:pollmeretal1979}]{\sphinxcrossref{\DUrole{std,std-ref}{Pollmer et al. 1979}}}}, {\hyperref[\detokenize{tech_note/References/CLM50_Tech_Note_References:crawfordetal1982}]{\sphinxcrossref{\DUrole{std,std-ref}{Crawford et al. 1982}}}}, {\hyperref[\detokenize{tech_note/References/CLM50_Tech_Note_References:simpsonetal1983}]{\sphinxcrossref{\DUrole{std,std-ref}{Simpson et al. 1983}}}}, {\hyperref[\detokenize{tech_note/References/CLM50_Tech_Note_References:taweiland1992}]{\sphinxcrossref{\DUrole{std,std-ref}{Ta and Weiland 1992}}}}, {\hyperref[\detokenize{tech_note/References/CLM50_Tech_Note_References:barbottinetal2005}]{\sphinxcrossref{\DUrole{std,std-ref}{Barbottin et al. 2005}}}},
{\hyperref[\detokenize{tech_note/References/CLM50_Tech_Note_References:gallaisetal2006}]{\sphinxcrossref{\DUrole{std,std-ref}{Gallais et al. 2006}}}}, {\hyperref[\detokenize{tech_note/References/CLM50_Tech_Note_References:gallaisetal2007}]{\sphinxcrossref{\DUrole{std,std-ref}{Gallais et al. 2007}}}}). Nitrogen allocation
for crops follows that of natural vegetation, is supplied in CLM by the
soil mineral nitrogen pool, and depends on C:N ratios for leaves, stems,
roots, and organs. Nitrogen demand during organ development is fulfilled
through retranslocation from leaves, stems, and roots. Nitrogen
retranslocation is initiated at the beginning of the grain fill stage
for all crops except soybean, for which retranslocation is after LAI decline.
Nitrogen stored in the leaf and stem is moved into a storage
retranslocation pool for all crops, and for wheat and rice, nitrogen in roots is also
released into the retranslocation storage pool. The quantity of nitrogen
mobilized depends on the C:N ratio of the plant tissue, and is
calculated as
\phantomsection\label{\detokenize{tech_note/Crop_Irrigation/CLM50_Tech_Note_Crop_Irrigation:equation-25.6}}\begin{equation}\label{equation:tech_note/Crop_Irrigation/CLM50_Tech_Note_Crop_Irrigation:25.6}
\begin{split}leaf\_ to\_ retransn=N_{leaf} -\frac{C_{leaf} }{CN_{leaf}^{f} }\end{split}
\end{equation}\phantomsection\label{\detokenize{tech_note/Crop_Irrigation/CLM50_Tech_Note_Crop_Irrigation:equation-25.7}}\begin{equation}\label{equation:tech_note/Crop_Irrigation/CLM50_Tech_Note_Crop_Irrigation:25.7}
\begin{split}stemn\_ to\_ retransn=N_{stem} -\frac{C_{stem} }{CN_{stem}^{f} }\end{split}
\end{equation}\phantomsection\label{\detokenize{tech_note/Crop_Irrigation/CLM50_Tech_Note_Crop_Irrigation:equation-25.8}}\begin{equation}\label{equation:tech_note/Crop_Irrigation/CLM50_Tech_Note_Crop_Irrigation:25.8}
\begin{split}frootn\_ to\_ retransn=N_{froot} -\frac{C_{froot} }{CN_{froot}^{f} }\end{split}
\end{equation}
where \({C}_{leaf}\), \({C}_{stem}\), and \({C}_{froot}\) is the carbon in the plant leaf, stem, and fine
root, respectively, \({N}_{leaf}\), \({N}_{stem}\), and \({N}_{froot}\)
is the nitrogen in the plant leaf, stem, and fine root, respectively, and \(CN^f_{leaf}\),
\(CN^f_{stem}\), and \(CN^f_{froot}\) is the post-grain fill C:N
ratio of the leaf, stem, and fine root respectively (\hyperref[\detokenize{tech_note/Crop_Irrigation/CLM50_Tech_Note_Crop_Irrigation:table-crop-allocation-parameters}]{Table \ref{\detokenize{tech_note/Crop_Irrigation/CLM50_Tech_Note_Crop_Irrigation:table-crop-allocation-parameters}}}). Since
C:N measurements are often taken from mature crops, pre-grain development C:N
ratios for leaves, stems, and roots in the model are optimized to allow maximum
nitrogen accumulation for later use during organ development, and post-grain
fill C:N ratios are assigned the same as crop residue. After
nitrogen is moved into the retranslocated pool,
the nitrogen in this pool is used to meet plant
nitrogen demand by assigning the available nitrogen from the
retranslocated pool equal to the plant nitrogen demand for each organ (\({CN_{[organ]}^{f} }\) in \hyperref[\detokenize{tech_note/Crop_Irrigation/CLM50_Tech_Note_Crop_Irrigation:table-crop-allocation-parameters}]{Table \ref{\detokenize{tech_note/Crop_Irrigation/CLM50_Tech_Note_Crop_Irrigation:table-crop-allocation-parameters}}}). Once the
retranslocation pool is depleted, soil mineral nitrogen pool is used to
fulfill plant nitrogen demands.


\paragraph{Harvest}
\label{\detokenize{tech_note/Crop_Irrigation/CLM50_Tech_Note_Crop_Irrigation:harvest-to-food-and-seed}}\label{\detokenize{tech_note/Crop_Irrigation/CLM50_Tech_Note_Crop_Irrigation:id13}}
Variables track the flow of grain C and
N to food and of all other plant pools, including live stem C and N, to litter. Putting live
stem C and N into the litter pool is in contrast to the approach for unmanaged PFTs which
puts live stem C and N into dead stem pools first. Leaf and root C and N pools
are routed to the litter pools in the same manner as natural vegetation. Whereas food C and N
was formerly transferred to the litter pool, CLM5 routes food C and N
to a grain product pool where the C and N decay to the atmosphere over one year,
similar in structure to the wood product pools.
Additionally, CLM5 accounts for the C and N required for crop seeding by removing the seed C and N from the grain
product pool during harvest. The crop seed pool is then used to seed crops in the subsequent year.
Calcuating the crop yields (Equation \eqref{equation:tech_note/Crop_Irrigation/CLM50_Tech_Note_Crop_Irrigation:25.9}) requires that you sum the GRAINC\_TO\_FOOD variable
for each year, and must account for the proportion of C in the dry crop weight.
Here, we assume that grain C is 45\% of the total dry weight. Additionally, harvest is not typically 100\% efficient, so
analysis needs to assume that harvest efficiency is less. We assume a harvest
efficiency of 85\%.
\phantomsection\label{\detokenize{tech_note/Crop_Irrigation/CLM50_Tech_Note_Crop_Irrigation:equation-25.9}}\begin{equation}\label{equation:tech_note/Crop_Irrigation/CLM50_Tech_Note_Crop_Irrigation:25.9}
\begin{split}  Grain\ yield(g.m^{-2})=\frac{\sum(GRAINC\_ TO\_ FOOD)*0.85}{0.45}\end{split}
\end{equation}

\begin{savenotes}\sphinxattablestart
\centering
\sphinxcapstartof{table}
\sphinxcaption{Crop allocation parameters for the active crop plant functional types (pfts) in CLM5BGCCROP. Numbers in the first row correspond to the list of pfts in \hyperref[\detokenize{tech_note/Crop_Irrigation/CLM50_Tech_Note_Crop_Irrigation:table-crop-plant-functional-types}]{Table \ref{\detokenize{tech_note/Crop_Irrigation/CLM50_Tech_Note_Crop_Irrigation:table-crop-plant-functional-types}}}.}\label{\detokenize{tech_note/Crop_Irrigation/CLM50_Tech_Note_Crop_Irrigation:table-crop-allocation-parameters}}\label{\detokenize{tech_note/Crop_Irrigation/CLM50_Tech_Note_Crop_Irrigation:id22}}
\sphinxaftercaption
\begin{tabulary}{\linewidth}[t]{|T|T|T|T|T|T|T|T|T|}
\hline


&\sphinxstylethead{\sphinxstyletheadfamily 
temperate corn
\unskip}\relax &\sphinxstylethead{\sphinxstyletheadfamily 
spring wheat
\unskip}\relax &\sphinxstylethead{\sphinxstyletheadfamily 
temperatue soybean
\unskip}\relax &\sphinxstylethead{\sphinxstyletheadfamily 
cotton
\unskip}\relax &\sphinxstylethead{\sphinxstyletheadfamily 
rice
\unskip}\relax &\sphinxstylethead{\sphinxstyletheadfamily 
sugarcane
\unskip}\relax &\sphinxstylethead{\sphinxstyletheadfamily 
tropical corn
\unskip}\relax &\sphinxstylethead{\sphinxstyletheadfamily 
tropical soybean
\unskip}\relax \\
\hline
IVT
&
17, 18
&
19, 20
&
23, 24
&
41, 42
&
61, 62
&
67, 68
&
75, 76
&
77, 78
\\
\hline
\(a_{leaf}^{i}\)
&
0.8
&
0.9
&
0.85
&
0.85
&
0.75
&
0.8
&
0.8
&
0.85
\\
\hline
\({L}_{max}\) (m $^{\text{2}}$  m $^{\text{-2}}$)
&
5
&
7
&
6
&
6
&
7
&
5
&
5
&
6
\\
\hline
\(a_{froot}^{i}\)
&
0.4
&
0.1
&
0.2
&
0.2
&
0.1
&
0.4
&
0.4
&
0.2
\\
\hline
\(a_{froot}^{f}\)
&
0.05
&
0
&
0.2
&
0.2
&
0
&
0.05
&
0.05
&
0.2
\\
\hline
\(a_{leaf}^{f}\)
&
0
&
0
&
0
&
0
&
0
&
0
&
0
&
0
\\
\hline
\(a_{livestem}^{f}\)
&
0
&
0.05
&
0.3
&
0.3
&
0.05
&
0
&
0
&
0.3
\\
\hline
\(d_{L}\)
&
1.05
&
1.05
&
1.05
&
1.05
&
1.05
&
1.05
&
1.05
&
1.05
\\
\hline
\(d_{alloc}^{stem}\)
&
2
&
1
&
5
&
5
&
1
&
2
&
2
&
5
\\
\hline
\(d_{alloc}^{leaf}\)
&
5
&
3
&
2
&
2
&
3
&
5
&
5
&
2
\\
\hline
\({CN}_{leaf}\)
&
25
&
20
&
20
&
20
&
20
&
25
&
25
&
20
\\
\hline
\({CN}_{stem}\)
&
50
&
50
&
50
&
50
&
50
&
50
&
50
&
50
\\
\hline
\({CN}_{froot}\)
&
42
&
42
&
42
&
42
&
42
&
42
&
42
&
42
\\
\hline
\(CN^f_{leaf}\)
&
65
&
65
&
65
&
65
&
65
&
65
&
65
&
65
\\
\hline
\(CN^f_{stem}\)
&
120
&
100
&
130
&
130
&
100
&
120
&
120
&
130
\\
\hline
\(CN^f_{froot}\)
&
0
&
40
&
0
&
0
&
40
&
0
&
0
&
0
\\
\hline
\({CN}_{grain}\)
&
50
&
50
&
50
&
50
&
50
&
50
&
50
&
50
\\
\hline
\end{tabulary}
\par
\sphinxattableend\end{savenotes}

Notes: Crop growth phases and corresponding variables are described throughout
the text. \({CN}_{leaf}\), \({CN}_{stem}\), and \({CN}_{froot}\) are
the target C:N ratios used during the leaf emergence phase (phase 2).


\subsubsection{Other Features}
\label{\detokenize{tech_note/Crop_Irrigation/CLM50_Tech_Note_Crop_Irrigation:other-features}}\label{\detokenize{tech_note/Crop_Irrigation/CLM50_Tech_Note_Crop_Irrigation:id14}}

\paragraph{Physical Crop Characteristics}
\label{\detokenize{tech_note/Crop_Irrigation/CLM50_Tech_Note_Crop_Irrigation:id15}}\label{\detokenize{tech_note/Crop_Irrigation/CLM50_Tech_Note_Crop_Irrigation:physical-crop-characteristics}}
Leaf area index (\sphinxstyleemphasis{L}) is calculated as a function of specific leaf area
(SLA, \hyperref[\detokenize{tech_note/Crop_Irrigation/CLM50_Tech_Note_Crop_Irrigation:table-crop-phenology-parameters}]{Table \ref{\detokenize{tech_note/Crop_Irrigation/CLM50_Tech_Note_Crop_Irrigation:table-crop-phenology-parameters}}}) and leaf C.
Stem area index (\sphinxstyleemphasis{S}) is equal to 0.1\sphinxstyleemphasis{L} for temperate and tropical corn and sugarcane and 0.2\sphinxstyleemphasis{L} for
other crops, as in AgroIBIS. All live
C and N pools go to 0 after crop harvest, but the \sphinxstyleemphasis{S} is kept at 0.25 to
simulate a post-harvest “stubble” on the ground.

Crop heights at the top and bottom of the canopy, \({z}_{top}\)
and \({z}_{bot}\) (m), come from the AgroIBIS formulation:
\phantomsection\label{\detokenize{tech_note/Crop_Irrigation/CLM50_Tech_Note_Crop_Irrigation:equation-25.10}}\begin{equation}\label{equation:tech_note/Crop_Irrigation/CLM50_Tech_Note_Crop_Irrigation:25.10}
\begin{split}\begin{array}{l}
{z_{top} =z_{top}^{\max } \left(\frac{L}{L_{\max } -1} \right)^{2} \ge 0.05{\rm \; where\; }\frac{L}{L_{\max } -1} \le 1} \\
{z_{bot} =0.02{\rm m}}
\end{array}\end{split}
\end{equation}
where \(z_{top}^{\max }\) is the maximum top-of-canopy height of the crop (\hyperref[\detokenize{tech_note/Crop_Irrigation/CLM50_Tech_Note_Crop_Irrigation:table-crop-phenology-parameters}]{Table \ref{\detokenize{tech_note/Crop_Irrigation/CLM50_Tech_Note_Crop_Irrigation:table-crop-phenology-parameters}}})
and \(L_{\max }\) is the maximum leaf area index (\hyperref[\detokenize{tech_note/Crop_Irrigation/CLM50_Tech_Note_Crop_Irrigation:table-crop-allocation-parameters}]{Table \ref{\detokenize{tech_note/Crop_Irrigation/CLM50_Tech_Note_Crop_Irrigation:table-crop-allocation-parameters}}}).


\paragraph{Interactive Fertilization}
\label{\detokenize{tech_note/Crop_Irrigation/CLM50_Tech_Note_Crop_Irrigation:interactive-fertilization}}\label{\detokenize{tech_note/Crop_Irrigation/CLM50_Tech_Note_Crop_Irrigation:id16}}
CLM simulates fertilization by adding nitrogen directly to the soil mineral nitrogen pool to meet
crop nitrogen demands using both industrial fertilizer and manure application. CLM’s separate crop land unit ensures that
natural vegetation will not access the fertilizer applied to crops.
Fertilizer in CLM5BGCCROP is prescribed by crop functional types and varies spatially
for each year based on the LUMIP land use and land cover change
time series (LUH2 for historical and SSPs for future) ({\hyperref[\detokenize{tech_note/References/CLM50_Tech_Note_References:lawrenceetal2016}]{\sphinxcrossref{\DUrole{std,std-ref}{Lawrence et al. 2016}}}}).
One of two fields is used to prescribe industrial fertilizer based on the type of simulation.
For non-transient simulations, annual fertilizer application in g N/m$^{\text{2}}$/yr
is specified on the land surface data set by the field CONST\_FERTNITRO\_CFT.
In transient simulations, annual fertilizer application is specified on the land use time series
file by the field FERTNITRO\_CFT, which is also in g N/m$^{\text{2}}$/yr.
The values for both of these fields come from the LUMIP time series for each year.
In addition to the industrial fertilizer, background manure fertilizer is specified
on the parameter file by the field ‘manunitro’. For the current CLM5BGCCROP,
manure N is applied at a rate of 0.002 kg N/m$^{\text{2}}$/yr. Because previous versions
of CLM (e.g., CLM4) had rapid denitrification rates, fertilizer is applied slowly
to minimize N loss (primarily through denitrification) and maximize plant uptake.
The current implementation of CLM5 inherits this legacy, although denitrification rates
are slower in the current version of the model ({\hyperref[\detokenize{tech_note/References/CLM50_Tech_Note_References:kovenetal2013}]{\sphinxcrossref{\DUrole{std,std-ref}{Koven et al. 2013}}}}). As such,
fertilizer application begins during the leaf emergence phase of crop
development (phase 2) and continues for 20 days, which helps reduce large losses
of nitrogen from leaching and denitrification during the early stage of
crop development. The 20-day period is chosen as an optimization to
limit fertilizer application to the emergence stage. A fertilizer
counter in seconds, \sphinxstyleemphasis{f}, is set as soon as the leaf emergence phase for crops
initiates:
\phantomsection\label{\detokenize{tech_note/Crop_Irrigation/CLM50_Tech_Note_Crop_Irrigation:equation-25.11}}\begin{equation}\label{equation:tech_note/Crop_Irrigation/CLM50_Tech_Note_Crop_Irrigation:25.11}
\begin{split} f = n \times 86400\end{split}
\end{equation}
where \sphinxstyleemphasis{n} is set to 20 fertilizer application days and 86400 is the number of seconds per day. When the crop enters
phase 2 (leaf emergence) of its growth
cycle, fertilizer application begins by initializing fertilizer amount
to the total fertilizer at each column within the grid cell divided by the initialized \sphinxstyleemphasis{f}.
Fertilizer is applied and \sphinxstyleemphasis{f} is decremented each time step until a zero balance on
the counter is reached.


\paragraph{Biological nitrogen fixation for soybeans}
\label{\detokenize{tech_note/Crop_Irrigation/CLM50_Tech_Note_Crop_Irrigation:biological-nitrogen-fixation-for-soybeans}}\label{\detokenize{tech_note/Crop_Irrigation/CLM50_Tech_Note_Crop_Irrigation:id17}}
Biological N fixation for soybeans is calculated by the fixation and uptake of
nitrogen module (Chapter \hyperref[\detokenize{tech_note/FUN/CLM50_Tech_Note_FUN:rst-fun}]{\ref{\detokenize{tech_note/FUN/CLM50_Tech_Note_FUN:rst-fun}}}) and is the same as N fixation in natural vegetation. Unlike natural
vegetation, where a fraction of each pft are N fixers, all soybeans
are treated as N fixers.


\paragraph{Latitudinal variation in base growth tempereature}
\label{\detokenize{tech_note/Crop_Irrigation/CLM50_Tech_Note_Crop_Irrigation:latitudinal-variation-in-base-growth-tempereature}}\label{\detokenize{tech_note/Crop_Irrigation/CLM50_Tech_Note_Crop_Irrigation:latitude-vary-base-tempereature-for-growing-degree-days}}
For most crops, \(GDD_{T_{{\rm 2m}} }\) (growing degree days since planting)
is the same in all locations. However,
the for both rainfed and irrigated spring wheat and sugarcane, the calculation of
\(GDD_{T_{{\rm 2m}} }\) allows for latitudinal variation:
\phantomsection\label{\detokenize{tech_note/Crop_Irrigation/CLM50_Tech_Note_Crop_Irrigation:equation-25.12}}\begin{equation}\label{equation:tech_note/Crop_Irrigation/CLM50_Tech_Note_Crop_Irrigation:25.12}
\begin{split}latitudinal\ variation\ in\ base\ T = \left\{
\begin{array}{lr}
baset +12 - 0.4 \times latitude &\qquad 0 \le latitude \le 30 \\
baset +12 + 0.4 \times latitude &\qquad -30 \le latitude \le 0
\end{array} \right\}\end{split}
\end{equation}
where \(baset\) is the \sphinxstyleemphasis{base temperature for GDD} (7$^{\text{th}}$ row) in \hyperref[\detokenize{tech_note/Crop_Irrigation/CLM50_Tech_Note_Crop_Irrigation:table-crop-phenology-parameters}]{Table \ref{\detokenize{tech_note/Crop_Irrigation/CLM50_Tech_Note_Crop_Irrigation:table-crop-phenology-parameters}}}.
Such latitudinal variation in base growth temperature could increase the base temperature, slow down \(GDD_{T_{{\rm 2m}} }\)
accumulation, and extend the growing season for regions within 30ºS to 30ºN for spring wheat
and sugarcane.


\paragraph{Separate reproductive pool}
\label{\detokenize{tech_note/Crop_Irrigation/CLM50_Tech_Note_Crop_Irrigation:id18}}\label{\detokenize{tech_note/Crop_Irrigation/CLM50_Tech_Note_Crop_Irrigation:separate-reproductive-pool}}
One notable difference between natural vegetation and crops is the
presence of reproductive carbon and nitrogen pools. Accounting
for the reproductive pools helps determine whether crops are performing
reasonably through yield calculations.
The reproductive pool is maintained similarly to the leaf, stem,
and fine root pools, but allocation of carbon and nitrogen does not
begin until the grain fill stage of crop development. Equation \eqref{equation:tech_note/Crop_Irrigation/CLM50_Tech_Note_Crop_Irrigation:25.5} describes the
carbon and nitrogen allocation coefficients to the reproductive pool.
In CLM5BGCCROP, as allocation declines in stem, leaf, and root pools (see section \hyperref[\detokenize{tech_note/Crop_Irrigation/CLM50_Tech_Note_Crop_Irrigation:grain-fill-to-harvest}]{\ref{\detokenize{tech_note/Crop_Irrigation/CLM50_Tech_Note_Crop_Irrigation:grain-fill-to-harvest}}})
during the grain fill stage of growth, increasing amounts of carbon and
nitrogen are available for grain development.


\subsection{The irrigation model}
\label{\detokenize{tech_note/Crop_Irrigation/CLM50_Tech_Note_Crop_Irrigation:id19}}\label{\detokenize{tech_note/Crop_Irrigation/CLM50_Tech_Note_Crop_Irrigation:the-irrigation-model}}
The CLM includes the option to irrigate cropland areas that are equipped
for irrigation. The application of irrigation responds dynamically to
the soil moisture conditions simulated by the CLM. This irrigation
algorithm is based loosely on the implementation of
{\hyperref[\detokenize{tech_note/References/CLM50_Tech_Note_References:ozdoganetal2010}]{\sphinxcrossref{\DUrole{std,std-ref}{Ozdogan et al. (2010)}}}}.

When irrigation is enabled, the crop areas of each grid cell are divided
into irrigated and rainfed fractions according to a dataset of areas
equipped for irrigation ({\hyperref[\detokenize{tech_note/References/CLM50_Tech_Note_References:portmannetal2010}]{\sphinxcrossref{\DUrole{std,std-ref}{Portmann et al. 2010}}}}).
Irrigated and rainfed crops are placed on separate soil columns, so that
irrigation is only applied to the soil beneath irrigated crops.

In irrigated croplands, a check is made once per day to determine
whether irrigation is required on that day. This check is made in the
first time step after 6 AM local time. Irrigation is required if crop
leaf area \(>\) 0, and the available soil water is below a specified
threshold.

The soil moisture deficit \(D_{irrig}\) is
\phantomsection\label{\detokenize{tech_note/Crop_Irrigation/CLM50_Tech_Note_Crop_Irrigation:equation-25.61}}\begin{equation}\label{equation:tech_note/Crop_Irrigation/CLM50_Tech_Note_Crop_Irrigation:25.61}
\begin{split}D_{irrig} = \left\{
\begin{array}{lr}
w_{thresh} - w_{avail} &\qquad w_{thresh} > w_{avail} \\
0 &\qquad w_{thresh} \le w_{avail}
\end{array} \right\}\end{split}
\end{equation}
where \(w_{thresh}\) is the irrigation moisture threshold (mm) and
\(w_{avail}\) is the available moisture (mm).  The moisture threshold
is
\phantomsection\label{\detokenize{tech_note/Crop_Irrigation/CLM50_Tech_Note_Crop_Irrigation:equation-25.62}}\begin{equation}\label{equation:tech_note/Crop_Irrigation/CLM50_Tech_Note_Crop_Irrigation:25.62}
\begin{split}w_{thresh} = f_{thresh} \left(w_{target} - w_{wilt}\right) + w_{wilt}\end{split}
\end{equation}
where \(w_{target}\) is the irrigation target soil moisture (mm)
\phantomsection\label{\detokenize{tech_note/Crop_Irrigation/CLM50_Tech_Note_Crop_Irrigation:equation-25.63}}\begin{equation}\label{equation:tech_note/Crop_Irrigation/CLM50_Tech_Note_Crop_Irrigation:25.63}
\begin{split}w_{target} = \sum_{j=1}^{N_{irr}} \theta_{target} \Delta z_{j} \ ,\end{split}
\end{equation}
\(w_{wilt}\) is the wilting point soil moisture (mm)
\phantomsection\label{\detokenize{tech_note/Crop_Irrigation/CLM50_Tech_Note_Crop_Irrigation:equation-25.64}}\begin{equation}\label{equation:tech_note/Crop_Irrigation/CLM50_Tech_Note_Crop_Irrigation:25.64}
\begin{split}w_{wilt} = \sum_{j=1}^{N_{irr}} \theta_{wilt} \Delta z_{j} \ ,\end{split}
\end{equation}
and \(f_{thresh}\) is a tuning parameter.  The available moisture in
the soil is
\phantomsection\label{\detokenize{tech_note/Crop_Irrigation/CLM50_Tech_Note_Crop_Irrigation:equation-25.65}}\begin{equation}\label{equation:tech_note/Crop_Irrigation/CLM50_Tech_Note_Crop_Irrigation:25.65}
\begin{split}w_{avail} = \sum_{j=1}^{N_{irr}} \theta_{j} \Delta z_{j} \ ,\end{split}
\end{equation}
\(N_{irr}\) is the index of the soil layer corresponding to a specified
depth \(z_{irrig}\) (\hyperref[\detokenize{tech_note/Crop_Irrigation/CLM50_Tech_Note_Crop_Irrigation:table-irrigation-parameters}]{Table \ref{\detokenize{tech_note/Crop_Irrigation/CLM50_Tech_Note_Crop_Irrigation:table-irrigation-parameters}}}) and
\(\Delta z_{j}\) is the thickness of the soil layer in layer \(j\) (section
\hyperref[\detokenize{tech_note/Ecosystem/CLM50_Tech_Note_Ecosystem:vertical-discretization}]{\ref{\detokenize{tech_note/Ecosystem/CLM50_Tech_Note_Ecosystem:vertical-discretization}}}).  \(\theta_{j}\) is the
volumetric soil moisture in layer \(j\) (section \hyperref[\detokenize{tech_note/Hydrology/CLM50_Tech_Note_Hydrology:soil-water}]{\ref{\detokenize{tech_note/Hydrology/CLM50_Tech_Note_Hydrology:soil-water}}}).
\(\theta_{target}\) and
\(\theta_{wilt}\) are the target and wilting point volumetric
soil moisture values, respectively, and are determined by inverting
\eqref{equation:tech_note/Hydrology/CLM50_Tech_Note_Hydrology:7.94} using soil matric
potential parameters \(\Psi_{target}\) and \(\Psi_{wilt}\)
(\hyperref[\detokenize{tech_note/Crop_Irrigation/CLM50_Tech_Note_Crop_Irrigation:table-irrigation-parameters}]{Table \ref{\detokenize{tech_note/Crop_Irrigation/CLM50_Tech_Note_Crop_Irrigation:table-irrigation-parameters}}}). After the soil moisture deficit
\(D_{irrig}\) is calculated, irrigation in an amount equal to
\(\frac{D_{irrig}}{T_{irrig}}\) (mm/s) is applied uniformly over
the irrigation period \(T_{irrig}\) (s).  Irrigation water is applied
directly to the ground surface, bypassing canopy interception (i.e.,
added to  \({q}_{grnd,liq}\): section \hyperref[\detokenize{tech_note/Hydrology/CLM50_Tech_Note_Hydrology:canopy-water}]{\ref{\detokenize{tech_note/Hydrology/CLM50_Tech_Note_Hydrology:canopy-water}}}).

To conserve mass, irrigation is removed from river water storage (Chapter \hyperref[\detokenize{tech_note/MOSART/CLM50_Tech_Note_MOSART:rst-river-transport-model-rtm}]{\ref{\detokenize{tech_note/MOSART/CLM50_Tech_Note_MOSART:rst-river-transport-model-rtm}}}).
When river water storage is inadequate to meet irrigation demand,
there are two options: 1) the additional water can be removed from the
ocean model, or 2) the irrigation demand can be reduced such that
river water storage is maintained above a specified threshold.


\begin{savenotes}\sphinxattablestart
\centering
\sphinxcapstartof{table}
\sphinxcaption{Irrigation parameters}\label{\detokenize{tech_note/Crop_Irrigation/CLM50_Tech_Note_Crop_Irrigation:table-irrigation-parameters}}\label{\detokenize{tech_note/Crop_Irrigation/CLM50_Tech_Note_Crop_Irrigation:id23}}
\sphinxaftercaption
\begin{tabulary}{\linewidth}[t]{|T|T|}
\hline
\sphinxstylethead{\sphinxstyletheadfamily 
Parameter
\unskip}\relax &\sphinxstylethead{\sphinxstyletheadfamily \unskip}\relax \\
\hline
\(f_{thresh}\)
&
1.0
\\
\hline
\(z_{irrig}\)       (m)
&
0.6
\\
\hline
\(\Psi_{target}\)   (mm)
&
-3400
\\
\hline
\(\Psi_{wilt}\)     (mm)
&
-150000
\\
\hline
\end{tabulary}
\par
\sphinxattableend\end{savenotes}


\section{Transient Land Use and Land Cover Change}
\label{\detokenize{tech_note/Transient_Landcover/CLM50_Tech_Note_Transient_Landcover:rst-transient-landcover-change}}\label{\detokenize{tech_note/Transient_Landcover/CLM50_Tech_Note_Transient_Landcover::doc}}\label{\detokenize{tech_note/Transient_Landcover/CLM50_Tech_Note_Transient_Landcover:transient-land-use-and-land-cover-change}}
CLM includes a treatment of mass and energy fluxes associated with
prescribed temporal land use and land cover change (LULCC). The model uses an annual
time series of the spatial distribution of the natural and crop land units
of each grid cell, in combination with the distribution of PFTs and CFTs
that exist in those land units. Additional land use is prescribed through annual
crop specific management of nitrogen fertilizer and irrigation described further
in Chapter 25, and through wood harvest on tree PFTs. For changes in the distributions
of natural and crop vegetation CLM diagnoses the change in area of the PFTs and CFTs
on January 1st of each model year and then performs mass and energy balance accounting
necessary to represent the expansion and contraction of the PFT and CFT areas. The
biogeophysical impacts of LULCC are simulated through changes
in surface properties which in turn impact the surface albedo, hydrology, and roughness
which then impact fluxes of energy, moisture and momentum to the atmosphere under the
altered properties. Additionally changes in energy and moisture associated with changes
in the natural and crop vegetation distribution are accounted for through small
fluxes to the atmosphere. The biogeochemical impacts of LULCC
are simulated through changes in CLM carbon pools and fluxes as shown in Figure xx.x and
described further in Chapter 16.


\subsection{Annual Transient Land Use and Land Cover Data and Time Interpolation}
\label{\detokenize{tech_note/Transient_Landcover/CLM50_Tech_Note_Transient_Landcover:annual-transient-land-use-and-land-cover-data-and-time-interpolation}}
The changes in area over time associated with changes in natural and crop
vegetation and the land use on that vegetation are prescribed through a forcing dataset,
referred to here as the \sphinxstyleemphasis{landuse.timeseries} dataset. The \sphinxstyleemphasis{landuse.timeseries} dataset
consists of an annual time series of global grids, where each annual time slice describes
the fractional area occupied by all PFTs and CFTs along with the nitrogen fertilizer and
irrigation fraction of each crop CFT, and the annual wood harvest applied to tree PFTs.
Changes in area of PFTs and CFTs are performed annually on the first time step of January
1st of the year. Fertilizer application, irrigation and wood harvest for each PFT and CFT
are performed at each model time step depending on rules from the crop and natural vegetation
phenology models. The irrigation fraction is set annually however fertlizer application and
wood harvest are set from a time-interpolation of the application rates from the two bracketing
annual time slices in the \sphinxstyleemphasis{landuse.timeseries} dataset.

As a special case, when the time dimension of the \sphinxstyleemphasis{landuse.timeseries} dataset
starts at a later year than the current model time step, the first time
slice from the \sphinxstyleemphasis{landuse.timeseries} dataset is used to represent the current time
step PFT and CFT fractional area distributions. Similarly, when the time
dimension of the \sphinxstyleemphasis{landuse.timeseries} dataset stops at an earlier year than the
current model time step, the last time slice of the \sphinxstyleemphasis{landuse.timeseries} dataset is
used. Thus, the simulation will have invariant representations of PFT and CFT
distributions through time for the periods prior to and following the
time duration of the \sphinxstyleemphasis{landuse.timeseries} dataset, with transient PFT and CFT distributions
during the period covered by the \sphinxstyleemphasis{landuse.timeseries} dataset.

The following equations capture this logic, where \(year_{cur}\)  is
the calendar year for the current timestep,
\(landuse.timeseries\_ year(1)\) and
\(landuse.timeseries\_ year(nyears)\) are the first and last calendar years in
the \sphinxstyleemphasis{landuse.timeseries} dataset, respectively, \(nyears\) is the number of
years in the \sphinxstyleemphasis{landuse.timeseries} dataset, \(nt_{1}\)  and \(nt_{2}\)
\({}_{ }\)are the two bracketing years used in the interpolation
algorithm, and \(n\) is the index value for the
\(landuse.timeseries\_ year\) array corresponding to
\(landuse.timeseries\_ year(n)=year_{cur}\) :
\phantomsection\label{\detokenize{tech_note/Transient_Landcover/CLM50_Tech_Note_Transient_Landcover:equation-26.1)}}\begin{equation}\label{equation:tech_note/Transient_Landcover/CLM50_Tech_Note_Transient_Landcover:26.1)}
\begin{split}nt_{1} =\left\{\begin{array}{l} {1\qquad {\rm for}\qquad year_{cur} <landuse.timeseries\_ year(1)} \\ {n\qquad {\rm for}\qquad landuse.timeseries\_ year(1)\le year_{cur} <landuse.timeseries\_ year(nyears)} \\ {nyears\qquad {\rm for}\qquad year_{cur} \ge landuse.timeseries\_ year(nyears)} \end{array}\right\}\end{split}
\end{equation}\phantomsection\label{\detokenize{tech_note/Transient_Landcover/CLM50_Tech_Note_Transient_Landcover:equation-26.2)}}\begin{equation}\label{equation:tech_note/Transient_Landcover/CLM50_Tech_Note_Transient_Landcover:26.2)}
\begin{split}nt_{2} =\left\{\begin{array}{l} {1\qquad {\rm for}\qquad year_{cur} <landuse.timeseries\_ year(1)} \\ {n+1\qquad {\rm for}\qquad landuse.timeseries\_ year(1)\le year_{cur} <landuse.timeseries\_ year(nyears)} \\ {nyears\qquad {\rm for}\qquad year_{cur} \ge landuse.timeseries\_ year(nyears)} \end{array}\right\}\end{split}
\end{equation}
Interpolation of fertilizer and wood harvest rates between annual time slices in the \sphinxstyleemphasis{landuse.timeseries}
dataset uses a simple linear algorithm, based on the conversion of the
current time step information into a floating-point value for the number
of calendar days since January 1 of the current model year
(\(cday\)). The interpolation weight for the current time step
\(tw_{cday}\) \({}_{ }\)is
\phantomsection\label{\detokenize{tech_note/Transient_Landcover/CLM50_Tech_Note_Transient_Landcover:equation-26.3)}}\begin{equation}\label{equation:tech_note/Transient_Landcover/CLM50_Tech_Note_Transient_Landcover:26.3)}
\begin{split}tw_{cday} =\frac{366-cday}{365}\end{split}
\end{equation}
where the numerator is 366 instead of 365 because the time manager
function for CLM returns a value of \(cday=1.0\) for midnight
Greenwich mean time on January 1. With weights \(w_{p} (nt_{1} )\)
and \(w_{p} (nt_{2} )\)obtained from the \sphinxstyleemphasis{landuse.timeseries} dataset for fertilizer and wood harvest
\sphinxstyleemphasis{p} at the bracketing annual time slices
\(nt_{1}\) \({}_{ }\)and \(nt_{2}\) , the interpolated
application rate for the current time step (\(w_{p,t}\) ) is
\phantomsection\label{\detokenize{tech_note/Transient_Landcover/CLM50_Tech_Note_Transient_Landcover:equation-26.4)}}\begin{equation}\label{equation:tech_note/Transient_Landcover/CLM50_Tech_Note_Transient_Landcover:26.4)}
\begin{split}w_{p,t} =tw_{cday} \left[w_{p} \left(nt_{1} \right)-w_{p} \left(nt_{2} \right)\right]+w_{p} \left(nt_{2} \right)\end{split}
\end{equation}
The form of this equation is designed to improve roundoff accuracy
performance, and guarantees \(w_{p,t}\)  stays in the range {[}0,1{]}.
Note that values for \(w_{p} (nt_{1} )\), \(w_{p} (nt_{2} )\),
and \(w_{p,t}\) \({}_{ }\)are fractional weights at the
column level of the subgrid hierarchy.

The change in weight for a fertilizer or wood harvest rate between the current and previous time
steps (\(\Delta w_{p}\) ) is
\phantomsection\label{\detokenize{tech_note/Transient_Landcover/CLM50_Tech_Note_Transient_Landcover:equation-26.5)}}\begin{equation}\label{equation:tech_note/Transient_Landcover/CLM50_Tech_Note_Transient_Landcover:26.5)}
\begin{split}\Delta w_{p} =w_{p}^{n} -w_{p}^{n-1}\end{split}
\end{equation}
where \sphinxstyleemphasis{n} denotes the current time step. The rate of application
increases for \(\Delta w_{p} >0\) and decreases for
\(\Delta w_{p} <0\).


\subsection{Mass and Energy Conservation}
\label{\detokenize{tech_note/Transient_Landcover/CLM50_Tech_Note_Transient_Landcover:mass-and-energy-conservation}}
Mass conservation is maintained across PFT and CFT weight transitions by
summing up all the carbon, nitrogen, water and energy state variables to get the total vegetated land
units value before (\(W_{tot,1}\) ) and after
(\(W_{tot,2}\) ) the new PFT and CFT weights are calculated. Transitions are performed on above ground
variables first and then at the land unit level for below ground variables second. For example the hydrological
balance is calculated,
\(W_{tot,1}\)  is
\phantomsection\label{\detokenize{tech_note/Transient_Landcover/CLM50_Tech_Note_Transient_Landcover:equation-26.6)}}\begin{equation}\label{equation:tech_note/Transient_Landcover/CLM50_Tech_Note_Transient_Landcover:26.6)}
\begin{split}W_{tot,1} =W_{a} +W_{sno} +\sum _{i=1}^{N_{levgrnd} }\left(w_{liq,i} +w_{ice,i} \right) +\sum _{j=1}^{npft}\left(W_{can,j} wt_{j,1} \right)\end{split}
\end{equation}
where \(W_{a}\)  is the aquifer water, \(W_{sno}\)  is the snow
water, \(w_{liq,i}\)  and \(w_{ice,i}\) are the liquid and ice
soil water contents, \(W_{can,j}\) is the canopy water content for
PFT and CFT \(j\), and \(wt_{j,1}\)  is the PFT or CFT weight for
\(j\). For the situation where PFT and CFT weights are changing, any difference
between \(W_{tot,1}\)  and \(W_{tot,2}\) are due to
differences in the total canopy water before and after the PFT and CFT weight
change. To ensure conservation, the typically very small
difference between \(W_{tot,2}\) and \(W_{tot,1}\)  is
subtracted from the grid cell runoff
\phantomsection\label{\detokenize{tech_note/Transient_Landcover/CLM50_Tech_Note_Transient_Landcover:equation-26.7)}}\begin{equation}\label{equation:tech_note/Transient_Landcover/CLM50_Tech_Note_Transient_Landcover:26.7)}
\begin{split}R_{liq} =R_{liq} +W_{tot,2} -W_{tot,1} .\end{split}
\end{equation}
Total energy is unperturbed in this case and therefore an energy
conservation treatment is not required. Changing the area of natural and crop land units
in association with the change in PFTs and CFTs results in changes in the soil/snow columns
and land unit area. To address these additional changes, conservation of mass and
energy among the soil/snow columns and land units is performed as a secondary calculation once
all above ground PFT and CFT changes have been done.


\subsection{Annual Transient Land Cover Dataset Development}
\label{\detokenize{tech_note/Transient_Landcover/CLM50_Tech_Note_Transient_Landcover:annual-transient-land-cover-dataset-development}}
This section describes the development of the \sphinxstyleemphasis{landuse.timeseries} dataset.
Development of this dataset involves the translation of
harmonized datasets of LULCC for the historical period and
for the different Shared Socioeconomic Pathway (SSP) - Representative
Concentration Pathway (RCP) scenarios. Additionally, LULCC time
series are to be generated for the Last Millennium and the extension beyond 2100 experiments
of CMIP6.


\subsubsection{LUH2 Transient Land Use and Land Cover Change Dataset}
\label{\detokenize{tech_note/Transient_Landcover/CLM50_Tech_Note_Transient_Landcover:luh2-transient-land-use-and-land-cover-change-dataset}}
To coordinate the processing and consistency of LULCC data between
the historical period (1850-2015) and the six
SSP-RCP (2016-2100) scenarios derived from Integrated
Assessment Models (IAM), the University of Maryland and the University of New Hampshire
research groups (Louise Chini, George Hurtt, Steve
Frolking and Ritvik Sahajpal; luh.umd.edu) produced a new version of the Land Use Harmonized version 2
(LUH2) transient datasets for use with Earth System Model simulations. The new data sets
are the product of the Land Use Model Intercomparison Project (LUMIP; \sphinxurl{https://cmip.ucar.edu/lumip})
as part of the Coupled Model Intercomparison Project 6 (CMIP6). The historical component of the
transient LULCC dataset has agriculture and urban
land use based on HYDE 3.2 with wood harvest based on FAO, Landsat and other sources, for the period 850-2015.
The SSP-RCP transient LULCC components (2015-2100) are
referred to as the LUH2 Future Scenario datasets. The LULCC information is provided at 0.25 degree grid resolution and includes
fractional grid cell coverage by the 12 land units of:

Primary Forest, Secondary Forest, Primary Non-Forest, Secondary Non-Forest,

Pasture, Rangeland, Urban,

C3 Annual Crop, C4 Annual Crop, C3 Perennial Crop, C4 Perennial Crop, and C3 Nitrogen Fixing Crop.

The new land unit format is an improvement on the CMIP5 LULCC
datasets as they: provide Forest and Non Forest information in combination with Primary and Secondary
land; differentiate between Pasture and Rangelands for grazing livestock; and specify annual details
on the types of Crops grown and management practices applied in each grid cell. Like the CMIP5 LULCC datasets Primary vegetation
represents the fractional area of a grid cell with vegetation undisturbed by human activities. Secondary
vegetation represents vegetated areas that have recovered from some human disturbance; this could include
re-vegetation of pasture and crop areas as well as primary vegetation areas that have been logged.
In this manner the land units can change through deforestation from Forested to Non Forested land and in the
opposite direction from Non Forested to Forested land through reforestation or afforestation without going
through the Crop, Pasture or Rangeland states.

The LUH2 dataset provides a time series of land cover states as well as a transition matrices that describes
the annual fraction of land that is transformed from one land unit category to
another (e.g. Primary Forest to C3 Annual Crop, Pasture to C3 Perrenial Crop, etc.; Lawrence et al.
2016). Included in these transition matrices is the total conversion of one land cover type to another referred to
as Gross LULCC. This value can be larger than the sum of the changes in the state of a land unit from one time period
to the next known as the Net LULCC. This difference is possible as land unit changes can occur both from the land unit
and to the land unit at the same time. An example of this difference occurs with shifting cultivation where Secondary Forest
can be converted to C3 Annual Crop at the same time as C3 Annual Crop is abandoned to Secondary Forest.

The transition matrices also provide harmonized prescriptions of wood harvest both in area of the grid cell harvested
and in the amount of biomass carbon harvested. The wood harvest biomass amount includes a 30\% slash component inline with
the CMIP5 LULCC data described in (Hurtt et al. 2011). The harvest area and carbon amounts are prescribed for the five classes of:
Primary Forest, Primary Non-Forest,
Secondary Mature Forest, Secondary
Young Forest, and Secondary
Non-Forest.

Additional land use management is prescribed on the Crop land units for
nitrogen fertilization and irrigation equipped land. The fertilizer application and the the irrigation fraction is
prescribed for each Crop land unit in a grid cell individually for each year of the time series. The wood harvest
and crop management are both prescribed spatially on the same 0.25 degree grid as the land use class transitions.


\subsubsection{Representing LUH2 Land Use and Land Cover Change in CLM5}
\label{\detokenize{tech_note/Transient_Landcover/CLM50_Tech_Note_Transient_Landcover:representing-luh2-land-use-and-land-cover-change-in-clm5}}
To represent the LUH2 transient LULCC dataset in CLM5, the annual fractional
composition of the twelve land units specified in the dataset needs to be
faithfully represented with a corresponding PFT and CFT mosaics of CLM.
CLM5 represents the land surface as a hierarchy of sub-grid types:
glacier; lake; wetland; urban; vegetated land; and crop land. The vegetated land is
further divided into a mosaic of Plant Functional Types (PFTs), while the crop land
is divided into a mosaic of Crop Functional Types (CFTs).

To support this translation task the CLM5 Land Use Data tool has been built that extends the
methods described in Lawrence et al (2012) to include all the new functionality of CMIP6 and CLM5 LULCC.
The tool translates each of the LUH2 land units for a given year into fractional PFT and CFT values based on
the current day CLM5 data for the land unit in that grid cell. The current day land unit descriptions are generated from
from 1km resolution MODIS, MIRCA2000, ICESAT, AVHRR, SRTM, and CRU climate data products combined with reference year
LUH2 land unit data, usually set to 2005. Where the land unit does not exist in a grid cell for the current
day, the land unit description is generated from nearest neighbors with an inverse distance weighted search
algorithm.

The Land Use Data tool produces raw vegetation, crop, and management data files which are combined with
other raw land surface data to produce the CLM5 initial surface dataset and the dynamic
\sphinxstyleemphasis{landuse.timeseries} dataset with the CLM5 mksurfdata\_map tool. The schematic of this entire process from
LUH2 time series and high resolution current day data to the output of CLM5 surface datasets from the
mksurfdata\_map tool is shown in Figure 21.2.

The methodology for creating the CLM5 transient PFT and CFT dataset is based on four
steps which are applied across all of the historical and future time series.
The first step involves generating the current day descriptions of natural and managed vegetation PFTs at
1km resolution from the global source datasets, and the current day description of crop CFTs at the 10km resolution
from the MIRCA 2000 datasets. The second step combines the current day (2005) LUH2 land units with the current
day CLM5 PFT and CFT distributions to get CLM5 land unit descriptions in either PFTs or CFTs at the LUH2 resolution of
0.25 degrees. The third step involves combining the LUH2 land unit time series with the CLM5 PFT and CFT descriptions
for that land unit to generate the CLM5 raw PFT and CFT time series in the \sphinxstyleemphasis{landuse.timeseries} file. At this point in the process
management information in terms of fertilizer, irrigation and wood harvest are added to the CLM5 PFT and CFT data
to complete the CLM5 raw PFT and CFT files. The final step is to combine these files with the other raw CLM5 surface
data files in the mksurfdata\_map tool.

\begin{figure}[htbp]
\centering
\capstart

\noindent\sphinxincludegraphics{{image18}.png}
\caption{Schematic of land cover change impacts on CLM carbon pools and fluxes.}\label{\detokenize{tech_note/Transient_Landcover/CLM50_Tech_Note_Transient_Landcover:figure-schematic-of-land-cover-change}}\label{\detokenize{tech_note/Transient_Landcover/CLM50_Tech_Note_Transient_Landcover:id1}}\end{figure}

\begin{figure}[htbp]
\centering
\capstart

\noindent\sphinxincludegraphics{{image22}.png}
\caption{Schematic of translation of annual LUH2 land units to CLM5 plant and crop functional types.}\label{\detokenize{tech_note/Transient_Landcover/CLM50_Tech_Note_Transient_Landcover:figure-schematic-of-translation-of-annual-luh2-land-units}}\label{\detokenize{tech_note/Transient_Landcover/CLM50_Tech_Note_Transient_Landcover:id2}}\end{figure}

\begin{figure}[htbp]
\centering
\capstart

\noindent\sphinxincludegraphics{{image3}.png}
\caption{Workflow of CLM5 Land Use Data Tool and Mksurfdata\_map Tool}\label{\detokenize{tech_note/Transient_Landcover/CLM50_Tech_Note_Transient_Landcover:figure-workflow-of-clm5-land-use-data-tool-and-mksurfdata-map-tool}}\label{\detokenize{tech_note/Transient_Landcover/CLM50_Tech_Note_Transient_Landcover:id3}}\end{figure}


\section{Dynamic Global Vegetation}
\label{\detokenize{tech_note/DGVM/CLM50_Tech_Note_DGVM:dynamic-global-vegetation}}\label{\detokenize{tech_note/DGVM/CLM50_Tech_Note_DGVM::doc}}\label{\detokenize{tech_note/DGVM/CLM50_Tech_Note_DGVM:rst-dynamic-global-vegetation-model}}

\subsection{What has changed}
\label{\detokenize{tech_note/DGVM/CLM50_Tech_Note_DGVM:what-has-changed}}\begin{itemize}
\item {} 
Deprecation of the dynamic global vegetation model (DGVM): The CLM5.0 model contains the legancy ‘CNDV’ code, which runs the CLM4(CN) model in combination with the LPJ-derived dynamics vegetation model introduced in CLM3. While this capacity has not technically been removed from the model, the DGVM has not been tested in the development of CLM5 and is no longer scientifically supported.

\item {} 
Introduction of FATES: The Functionally Assembled Terrestrial Ecosystem Simulator (FATES) is the actively developed DGVM for the CLM5. See

\end{itemize}


\section{Technical Documentation for FATES}
\label{\detokenize{tech_note/DGVM/CLM50_Tech_Note_DGVM:rst-fates}}\label{\detokenize{tech_note/DGVM/CLM50_Tech_Note_DGVM:technical-documentation-for-fates}}
FATES is the “Functionally Assembled Terrestrial Ecosystem Simulator”. It is an external module which can run within a given “Host Land Model” (HLM). Currently (November 2017) implementations are supported in both the Community Land Model(CLM) and in the land model of the E3SM Dept. of Energy Earth System Model.

FATES was derived from the CLM Ecosystem Demography model (CLM(ED)), which was documented in:
\begin{description}
\item[{Fisher RA, Muszala S, Verteinstein M, Lawrence P, Xu C, McDowell NG,}] \leavevmode
Knox RG, Koven C, Holm J, Rogers BM, Lawrence D. Taking off the
training wheels: the properties of a dynamic vegetation model without
climate envelopes. Geoscientific Model Development Discussions. 2015
Apr 1;8(4).

\end{description}

and this technical note was first published as an appendix to that paper.

\sphinxurl{https://pdfs.semanticscholar.org/396c/b9f172cb681421ed78325a2237bfb428eece.pdf}


\subsection{Introduction}
\label{\detokenize{tech_note/DGVM/CLM50_Tech_Note_DGVM:introduction}}
The Ecosystem Demography (‘ED’), concept within FATES is derived from the work of {\hyperref[\detokenize{tech_note/References/CLM50_Tech_Note_References:mc-2001}]{\sphinxcrossref{\DUrole{std,std-ref}{Moorcroft et al. (2001)}}}}

and is a cohort model of vegetation competition and co-existence, allowing a representation of the biosphere which accounts for the division of the land surface into successional stages, and for competition for light between height structured cohorts of representative trees of various plant functional types.

The implementation of the Ecosystem Demography
concept within FATES links the surface flux and canopy physiology concepts in the CLM/E3SM
with numerous additional developments necessary to accommodate the new
model also documented here. These include a version of the SPITFIRE
(Spread and InTensity of Fire) model of {\hyperref[\detokenize{tech_note/References/CLM50_Tech_Note_References:thonicke2010}]{\sphinxcrossref{\DUrole{std,std-ref}{Thonicke et al. (2010)}}}}, and an adoption of the concept of
\sphinxtitleref{Perfect Plasticity Approximation} approach of
{\hyperref[\detokenize{tech_note/References/CLM50_Tech_Note_References:purves2008}]{\sphinxcrossref{\DUrole{std,std-ref}{Purves et al. 2008}}}}, {\hyperref[\detokenize{tech_note/References/CLM50_Tech_Note_References:lichstein2011}]{\sphinxcrossref{\DUrole{std,std-ref}{Lichstein et al. 2011}}}} and {\hyperref[\detokenize{tech_note/References/CLM50_Tech_Note_References:weng2014}]{\sphinxcrossref{\DUrole{std,std-ref}{Weng et al. 2014}}}}, in accounting
for the spatial arrangement of crowns. Novel algorithms accounting for
the fragmentation of coarse woody debris into chemical litter streams,
for the physiological optimisation of canopy thickness, for the
accumulation of seeds in the seed bank, for multi-layer multi-PFT
radiation transfer, for drought-deciduous and cold-deciduous phenology,
for carbon storage allocation, and for tree mortality under carbon
stress, are also included and presented here.

Numerous other implementations of the
Ecosystem Demography concept exist (See Fisher et al. 2017 for a review of these) Therefore, to avoid confusion between the
concept of ‘Ecosystem Demography’ and the implementation of this concept
in different models, the CLM(ED) implementation described by Fisher et al. (2015) will hereafter be called ‘FATES’ (the Functionally Assembled Terrestrial Ecosystem Simulator).


\subsection{The representation of ecosystem heterogeneity in FATES}
\label{\detokenize{tech_note/DGVM/CLM50_Tech_Note_DGVM:the-representation-of-ecosystem-heterogeneity-in-fates}}
The terrestrial surface of the Earth is heterogeneous for many reasons, driven
by variations in climate, edaphic history, ecological variability,
geological forcing and human interventions. Land surface models
represent this variability first by introducing a grid structure to the
land surface, allowing different atmospheric forcings to operate in each
grid cell, and subsequently by representing ‘sub-grid’ variability in
the surface properties. In the CLM, the land surface is divided into
numerous ‘landunits’ corresponding to the underlying condition of the
surface (e.g. soils, ice, lakes, bare ground) and then ‘columns’
referring to elements of the surface that share below ground resources
(water \& nutrients). Within the soil landunit, for example, there are
separate columns for crops, and for natural vegetation, as these are
assumed to use separate resource pools. The FATES model at present
only operates on the naturally vegetated column. The soil column is
sub-divided into numerous tiles, that correspond to statistical
fractions of the potentially vegetated land area. In the CLM 4.5 (and
all previous versions of the model), sub-grid tiling operates on the
basis of plant functional types (PFTs). That is, each piece of land is
assumed to be occupied by only one plant functional type, with multiple
PFT-specific tiles sharing a common soil water and nutrient pool. This
PFT-based tiling structure is the standard method used by most land
surface models deployed in climate prediction.

The introduction of the Ecosystem Demography concept introduces
significant alterations to the representation of the land surface in the
CLM. In FATES, the tiling structure represents the disturbance
history of the ecosystem. Thus, some fraction of the land surface is
characterized as ‘recently disturbed’, some fraction has escaped
disturbance for a long time, and other areas will have intermediate
disturbances. Thus the ED concept essentially discretizes the trajectory
of succession from disturbed ground to ‘mature’ ecosystems. Within
FATES, each “disturbance history class” is referred to as a ‘patch’.
The word “patch”  has many possible interpretations, so it is important
to note that: \sphinxstylestrong{there is no spatial location associated with the concept
of a ‘patch’ . It refers to a fraction of the potential vegetated area
consisting of all parts of the ecosystem with similar disturbance
history.}

The ‘patch’ organizational structure in CLM thus replaces the previous
‘PFT’ structure in the organization heirarchy. The original hierarchical
land surface organizational structure of CLM as described in
{\hyperref[\detokenize{tech_note/References/CLM50_Tech_Note_References:oleson2013}]{\sphinxcrossref{\DUrole{std,std-ref}{Oleson et al. 2013}}}} may be depicted as:
\begin{equation*}
\begin{split}\mathbf{gridcell} \left\{
\begin{array}{cc}
\mathbf{landunit} &   \\
\mathbf{landunit} &\left\{
\begin{array}{ll}
\mathbf{column}&\\
\mathbf{column}&\left\{
\begin{array}{ll}
\mathbf{pft}&\\
\mathbf{pft}&\\
\mathbf{pft}&\\
\end{array}\right.\\
\mathbf{column}&\\
\end{array}\right.\\
\mathbf{landunit} &   \\
\end{array}\right.\end{split}
\end{equation*}
and the new structure is altered to the following:
\begin{equation*}
\begin{split}\mathbf{gridcell} \left\{
\begin{array}{cc}
\mathbf{landunit} &   \\
\mathbf{landunit} &\left\{
\begin{array}{ll}
\mathbf{column}&\\
\mathbf{column}&\left\{
\begin{array}{ll}
\mathbf{patch}&\\
\mathbf{patch}&\\
\mathbf{patch}&\\
\end{array}\right.\\
\mathbf{column}&\\
\end{array}\right.\\
\mathbf{landunit} &   \\
\end{array}\right.\end{split}
\end{equation*}
Thus, each gridcell becomes a matrix of ‘patches’ that are
conceptualized by their ‘age since disturbance’ in years. This is the
equivalent of grouping together all those areas of a gridcell that are
‘canopy gaps’, into a single entity, and all those areas that are
‘mature forest’ into a single entity.


\subsubsection{Cohortized representation of tree populations}
\label{\detokenize{tech_note/DGVM/CLM50_Tech_Note_DGVM:cohortized-representation-of-tree-populations}}
Each common-disturbance-history patch is a notional ecosystem that might
in reality contain numerous individual plants which vary in their
physiological attributes, in height and in spatial position. One way of
addressing this heterogeneity is to simulate a forest of specific
individuals, and to monitor their behavior through time. This is the
approach taken by “gap” and individual-based models
({\hyperref[\detokenize{tech_note/References/CLM50_Tech_Note_References:smith2001}]{\sphinxcrossref{\DUrole{std,std-ref}{Smith et al. 2001}}}}, {\hyperref[\detokenize{tech_note/References/CLM50_Tech_Note_References:sato2007}]{\sphinxcrossref{\DUrole{std,std-ref}{Sato et al. 2007}}}}, {\hyperref[\detokenize{tech_note/References/CLM50_Tech_Note_References:uriarte2009}]{\sphinxcrossref{\DUrole{std,std-ref}{Uriarte et al. 2009}}}}, {\hyperref[\detokenize{tech_note/References/CLM50_Tech_Note_References:fyllas2014}]{\sphinxcrossref{\DUrole{std,std-ref}{Fyllas et al. 2014}}}}). The
depiction of individuals typically implies that the outcome of the model
is stochastic. This is because we lack the necessary detailed knowledge
to simulate the individual plant’s fates. Thus gap models imply both
stochastic locations and mortality of plants. Thus, (with a genuinely
random seed) each model outcome is different, and an ensemble of model
runs is required to generate an average representative solution. Because
the random death of large individual trees can cause significant
deviations from the mean trajectory for a small plot (a typical
simulated plot size is 30m x 30 m) the number of runs required to
minimize these deviations is large and computationally expensive. For
this reason, models that resolve individual trees typically use a
physiological timestep of one day or longer (e.g.  {\hyperref[\detokenize{tech_note/References/CLM50_Tech_Note_References:smith2001}]{\sphinxcrossref{\DUrole{std,std-ref}{Smith et al. 2001}}}}, {\hyperref[\detokenize{tech_note/References/CLM50_Tech_Note_References:xiaodong2005}]{\sphinxcrossref{\DUrole{std,std-ref}{Xiaidong et al. 2005}}}}, {\hyperref[\detokenize{tech_note/References/CLM50_Tech_Note_References:sato2007}]{\sphinxcrossref{\DUrole{std,std-ref}{Sato et al. 2007}}}}

The approach introduced by the Ecosystem Demography model
{\hyperref[\detokenize{tech_note/References/CLM50_Tech_Note_References:mc-2001}]{\sphinxcrossref{\DUrole{std,std-ref}{Moorcroft et al. 2001}}}} is to group the hypothetical population
of plants into “cohorts”. In the notional ecosystem, after the
land-surface is divided into common-disturbance-history patches, the
population in each patch is divided first into plant functional types
(the standard approach to representing plant diversity in large scale
vegetation models), and then each plant type is represented as numerous
height classes. Importantly, \sphinxstylestrong{for each PFT/height class bin, we model
*one* representative individual plant, which tracks the average
properties of this {}`cohort{}` of individual plants.} Thus, each
common-disturbance-history patch is typically occupied by a set of
cohorts of different plant functional types, and different height
classes within those plant functional types. Each cohort is associated
with a number of identical trees, \(n_{coh}\) (where \({coh}\)
denotes the identification or index number for a given cohort)..

The complete hierarchy of elements in FATES is therefore now
described as follows:
\begin{equation*}
\begin{split}\mathbf{gridcell}\left\{
\begin{array}{cc}
\mathbf{landunit} &   \\
\mathbf{landunit} &\left\{
\begin{array}{ll}
\mathbf{column}&\\
\mathbf{column}&\left\{
\begin{array}{ll}
\mathbf{patch}&\\
\mathbf{patch}&\left\{
\begin{array}{ll}
\mathbf{cohort}&\\
\mathbf{cohort}&\\
\mathbf{cohort}&\\
\end{array}\right.\\
\mathbf{patch}&\\
\end{array}\right.\\
\mathbf{column}&\\
\end{array}\right.\\
\mathbf{landunit} &   \\
\end{array}\right.\end{split}
\end{equation*}

\subsubsection{Discretization of cohorts and patches}
\label{\detokenize{tech_note/DGVM/CLM50_Tech_Note_DGVM:discretization-of-cohorts-and-patches}}
Newly disturbed land and newly recruited seedlings can in theory be
generated at each new model timestep as the result of germination and
disturbance processes. If the new patches and cohorts established at
\sphinxstyleemphasis{every} timestep were tracked by the model structure, the computational
load would of course be extremely high (and thus equivalent to an
individual-based approach). A signature feature of the ED model is the
system by which \sphinxtitleref{functionally equivalent} patches and cohorts are fused
into single model entities to save memory and computational time.

%
\begin{footnote}[1]\sphinxAtStartFootnote
This description covers algorithms in the ‘fuse\_cohorts’ subroutine.
%
\end{footnote} This functionality requires that criteria are established for the
meaning of \sphinxtitleref{functional equivalence}, which are by necessity slightly
subjective, as they represent ways of abstracting reality into a more
tractable mathematical representation. As an example of this, for
height-structured cohorts, we calculate the relativized differences in
height (\(h_{coh}\), m) between two cohorts of the same pft,
\(p\) and \(q\) as
\begin{equation*}
\begin{split}d_{hite,p,q} = \frac{\mathrm{abs}.(h_{p-}h_{q})}{\frac{1}{2}(h_{p}+h_{q})}\end{split}
\end{equation*}
If \(d_{hite,p,q}\) is smaller than some threshold \(t_{ch}\),
and they are of the same plant functional type, the two cohorts are
considered equivalent and merged to form a third cohort \(r\), with
the properties of cohort \(p\) and \(q\) averaged such that they
conserve mass. The model parameter \(t_{ch}\) can be adjusted to
adjust the trade-off between simulation accuracy and computational load.
There is no theoretical optimal value for this threshold but it may be
altered to have finer or coarser model resolutions as needed.

%
\begin{footnote}[2]\sphinxAtStartFootnote
This description covers algorithms in the ‘fuse\_patches’ subroutine.
%
\end{footnote} Similarly, for common-disturbance-history patches, we again assign
a threshold criteria, which is then compared to the difference between
patches \(m\) and \(n\), and if the difference is less than some
threshold value (\(t_{p}\)) then patches are merged together,
otherwise they are kept separate. However, in contrast with
height-structured cohorts, where the meaning of the difference criteria
is relatively clear, how the landscape should be divided into
common-disturbance-history units is less clear. Several alternative
criteria are possible, including Leaf Area Index, total biomass and
total stem basal area.

In this implementation of FATES we assess the amount of
above-ground biomass in each PFT/plant diameter bin. Biomass is first
grouped into fixed diameter bins for each PFT (\(ft\)) and a
significant difference in any bin will cause patches to remain
separated. This means that if two patches have similar total biomass,
but differ in the distribution of that biomass between diameter classes
or plant types, they remain as separate entities. Thus
\begin{equation*}
\begin{split}B_{profile,m,dc,ft} = \sum_{d_{c,min}}^{d_{c,max}} (B_{ag,coh}n_{coh})\end{split}
\end{equation*}
\(B_{profile,m,dc,ft}\) is the binned above-ground biomass profile
for patch \(m\),\(d_{c}\) is the diameter class.
\(d_{c,min}\) and \(d_{c,max}\) are the lower and upper
boundaries for the \(d_{c}\) diameter class. \(B_{ag,coh}\) and
\(n_{coh}\) depict the biomass (KgC m\(^{-2}\)) and the number
of individuals of each cohort respectively. A difference matrix between
patches \(m\) and \(n\) is thus calculated as
\begin{equation*}
\begin{split}d_{biomass,mn,dc,ft} = \frac{\rm{abs}\it(B_{profile,m,hc,ft}-B_{profile,n,hc,ft})}{\frac{1}{2}(B_{profile,m,hc,ft}+B_{profile,n,hc,ft})}\end{split}
\end{equation*}
If all the values of \(d_{biomass,mn,hc,ft}\) are smaller than the
threshold, \(t_{p}\), then the patches \(m\) and \(n\) are
fused together to form a new patch \(o\).

To increase computational efficiency and to simplify the coding
structure of the model, the maximum number of patches is capped at
\(P_{no,max}\). To force the fusion of patches down to this number,
the simulation begins with a relatively sensitive discretization of
patches (\(t_{p}\) = 0.2) but if the patch number exceeds the
maximum, the fusion routine is repeated iteratively until the two most
similar patches reach their fusion threshold. This approach maintains an
even discretization along the biomass gradient, in contrast to, for
example, simply fusing the oldest or youngest patches together.

%
\begin{footnote}[3]\sphinxAtStartFootnote
This description covers algorithms in the ‘fuse\_2\_patches’
subroutine.
%
\end{footnote} The area of the new patch (\(A_{patch,o}\), m\(^{2}\))
is the sum of the area of the two existing patches,
\begin{equation*}
\begin{split}A_{patch,o}  = A_{patch,n}  + A_{patch,m}\end{split}
\end{equation*}
and the cohorts ‘belonging’ to patches \(m\) and \(n\) now
co-occupy patch \(o\). The state properties of \(m\) and
\(n\) (litter, seed pools, etc. ) are also averaged in accordance
with mass conservation .


\subsubsection{Linked Lists: the general code structure of FATES}
\label{\detokenize{tech_note/DGVM/CLM50_Tech_Note_DGVM:linked-lists-the-general-code-structure-of-fates}}
%
\begin{footnote}[4]\sphinxAtStartFootnote
This description covers the structure of code in all modules in
clm4\_5 that are located in ‘ED’ subdirectories
%
\end{footnote} The number of patches in each natural vegetation column and the
number of cohorts in any given patch are variable through time because
they are re-calculated for each daily timestep of the model. The more
complex an ecosystem, the larger the number of patches and cohorts. For
a slowly growing ecosystem, where maximum cohort size achieved between
disturbance intervals is low, the number of cohorts is also low. For
fast-growing ecosystems where many plant types are viable and maximum
heights are large, more cohorts are required to represent the ecosystem
with adequate complexity.

In terms of variable structure, the creation of an array whose size
could accommodate every possible cohort would mean defining the maximum
potential number of cohorts for every potential patch, which would
result in very large amounts of wasted allocated memory, on account of
the heterogeneity in the number of cohorts between complex and simple
ecosystems (n.b. this does still happen for some variables at restart
timesteps). To resolve this, the cohort structure in FATES model
does not use an array system for internal calculations. Instead it uses
a system of \sphinxstyleemphasis{linked lists} where each cohort structure is linked to the
cohorts larger than and smaller than itself using a system of pointers.
The shortest cohort in each patch has a ‘shorter’ pointer that points to
the \sphinxstyleemphasis{null} value, and the tallest cohort has a ‘taller’ pointer that
points to the null value.

Instead of iterating along a vector indexed by \(coh\), the code
structures typically begin at the tallest cohort in a given patch, and
iterate until a null pointer is encountered.

Using this structure, it is therefore possible to have an unbounded upper limit on cohort number, and also to easily alter the ordering of  cohorts if, for example, a cohort of one functional type begins to  grow faster than a competitor of another functional type, and the cohort list can easily be re-ordered by altering the pointer structure. Each cohort has \sphinxtitleref{pointers} indicating to which patch and gridcell it belongs. The patch system is analogous to the cohort system, except that patches are ordered in terms of their relative age, with pointers to older and younger patches where cp\(_1\) is the oldest:


\subsubsection{Indices used in FATES}
\label{\detokenize{tech_note/DGVM/CLM50_Tech_Note_DGVM:indices-used-in-fates}}
Some of the indices used in FATES are similar to those used in the
standard CLM4.5 model; column (\(c\)), land unit(\(l\)), grid
cell(\(g\)) and soil layer (\(j\)). On account of the
additional complexity of the new representation of plant function,
several additional indices are introduced that describe the
discritization of plant type, fuel type, litter type, plant height,
canopy identity, leaf vertical structure and fuel moisture
characteristics. To provide a reference with which to interpret the
equations that follow, they are listed here.

\bigskip

\captionof{table}{Table of subscripts used in this document  }


\begin{savenotes}\sphinxattablestart
\centering
\begin{tabulary}{\linewidth}[t]{|T|T|}
\hline
\sphinxstylethead{\sphinxstyletheadfamily 
Parameter Symbol
\unskip}\relax &\sphinxstylethead{\sphinxstyletheadfamily 
Parameter Name
\unskip}\relax \\
\hline
\sphinxstyleemphasis{ft}
&
Plant Functional Type
\\
\hline
\sphinxstyleemphasis{fc}
&
Fuel Class
\\
\hline
\sphinxstyleemphasis{lsc}
&
Litter Size Class
\\
\hline
\sphinxstyleemphasis{coh}
&
Cohort Index
\\
\hline
\sphinxstyleemphasis{patch}
&
Patch Index
\\
\hline
\sphinxstyleemphasis{Cl}
&
Canopy Layer
\\
\hline
\sphinxstyleemphasis{z}
&
Leaf Layer
\\
\hline
\sphinxstyleemphasis{mc}
&
Moisture Class
\\
\hline
\end{tabulary}
\par
\sphinxattableend\end{savenotes}

\bigskip


\subsubsection{Cohort State Variables}
\label{\detokenize{tech_note/DGVM/CLM50_Tech_Note_DGVM:cohort-state-variables}}
The unit of allometry in the ED model is the cohort. Each cohort
represents a group of plants with similar functional types and heights
that occupy portions of column with similar disturbance histories. The
state variables of each cohort therefore consist of several pieces of
information that fully describe the growth status of the plant and its
position in the ecosystem structure, and from which the model can be
restarted. The state variables of a cohort are as follows:

\bigskip

\captionof{table}{State Variables of  `cohort' sructure}


\begin{savenotes}\sphinxattablestart
\centering
\begin{tabulary}{\linewidth}[t]{|T|T|T|T|}
\hline
\sphinxstylethead{\sphinxstyletheadfamily 
Quantity
\unskip}\relax &\sphinxstylethead{\sphinxstyletheadfamily 
Variable name
\unskip}\relax &\sphinxstylethead{\sphinxstyletheadfamily 
Units
\unskip}\relax &\sphinxstylethead{\sphinxstyletheadfamily 
Notes
\unskip}\relax \\
\hline
Plant
Functional Type
&
\({\it{ft}
_{coh}}\)
&
integer
&\\
\hline
Number of
Individuals
&
\(n_{coh}\)
&
n per
10000m:math:{}`
\textasciicircum{}\{-2\}{}`
&\\
\hline
Height
&
\(h_{coh}\)
&
m
&\\
\hline
Diameter
&
\(\it{dbh_
{coh}}\)
&
cm
&\\
\hline
Structural
Biomass
&
\({b_{stru
c,coh}}\)
&
KgC
plant\(^
{-1}\)
&
Stem wood
(above and
below ground)
\\
\hline
Alive Biomass
&
\({b_{aliv
e,coh}}\)
&
KgC
plant\(^
{-1}\)
&
Leaf, fine root
and sapwood
\\
\hline
Stored Biomass
&
\({b_{stor
e,coh}}\)
&
KgC
plant\(^
{-1}\)
&
Labile carbon
reserve
\\
\hline
Leaf memory
&
\({l_{memo
ry,coh}}\)
&
KgC
plant\(^
{-1}\)
&
Leaf mass when
leaves are
dropped
\\
\hline
Canopy Layer
&
\({C_{l,co
h}}\)
&
integer
&
1 = top layer
\\
\hline
Phenological
Status
&
\({S_{phen
,coh}}\)
&
integer
&
1=leaves off.
2=leaves on
\\
\hline
Canopy trimming
&
\(C_{trim,
coh}\)
&
fraction
&
1.0=max leaf
area
\\
\hline
Patch Index
&
\({p_{coh}
}\)
&
integer
&
To which patch
does this
cohort belong?
\\
\hline
\end{tabulary}
\par
\sphinxattableend\end{savenotes}


\subsubsection{Patch State Variables}
\label{\detokenize{tech_note/DGVM/CLM50_Tech_Note_DGVM:patch-state-variables}}
A patch, as discuss earlier, is a fraction of the landscape which
contains ecosystems with similar structure and disturbance history. A
patch has no spatial location. The state variables, which are
‘ecosystem’ rather than ‘tree’ scale properties, from which the model
can be restarted, are as follows

\bigskip

\captionof{table}{State variables of `patch' structure}


\begin{savenotes}\sphinxattablestart
\centering
\begin{tabular}[t]{|*{5}{\X{1}{5}|}}
\hline
\sphinxstylethead{\sphinxstyletheadfamily 
Quantity
\unskip}\relax &\sphinxstylethead{\sphinxstyletheadfamily 
Variable
name
\unskip}\relax &\sphinxstylethead{\sphinxstyletheadfamily 
Units
\unskip}\relax &\sphinxstylethead{\sphinxstyletheadfamily 
Indexed By
\unskip}\relax &\sphinxstylethead{\sphinxstyletheadfamily \unskip}\relax \\
\hline
Area
&
\(\it{
A_{patch}}\)
&
m\(^
{2}\)
&\begin{itemize}
\item {} 
\end{itemize}
&\\
\hline
Age
&
\(age_
{patch}\)
&
years
&\begin{itemize}
\item {} 
\end{itemize}
&\\
\hline
Seed
&
\(seed_
{patch}\)
&
KgC
m\(^
{-2}\)
&
\(ft\)
&\\
\hline
Leaf Litter
&
\(l_{l
itter,patch
}\)
&
KgC
m\(^
{-2}\)
&
\(ft\)
&\\
\hline
Root Litter
&
\(r_{l
itter,patch
}\)
&
KgC
m\(^
{-2}\)
&
\(ft\)
&\\
\hline
AG Coarse
Woody
Debris
&
:math:{}`
\{CWD\}\_\{A
G,patch\}{}`
&
KgC
m\(^
{-2}\)
&
Size Class
(lsc)
&\\
\hline
BG Coarse
Woody
Debris
&
:math:{}`
\{CWD\}\_\{B
G,patch\}{}`
&
KgC
m\(^
{-2}\)
&
Size Class
(lsc)
&\\
\hline
Canopy
Spread
&
\(S_{c
,patch}\)
&\begin{itemize}
\item {} 
\end{itemize}
&
Canopy
Layer
&\\
\hline
Column
Index
&
\({l_{
patch}}\)
&
integer
&\begin{itemize}
\item {} 
\end{itemize}
&\\
\hline
\end{tabular}
\par
\sphinxattableend\end{savenotes}


\subsubsection{Model Structure}
\label{\detokenize{tech_note/DGVM/CLM50_Tech_Note_DGVM:model-structure}}
Code concerned with the Ecosystem Demography model interfaces with the
CLM model in four ways: i) During initialization, ii) During the
calculation of surface processes (albedo, radiation absorption, canopy
fluxes) each model time step (typically half-hourly), iii) During the
main invokation of the ED model code at the end of each day. Daily
cohort-level NPP is used to grow plants and alter the cohort structures,
disturbance processes (fire and mortality) operate to alter the patch
structures, and all fragmenting carbon pool dynamics are calculated. iv)
during restart reading and writing. The net assimilation (NPP) fluxes
attributed to each cohort are accumulated throughout each daily cycle
and passed into the ED code as the major driver of vegetation dynamics.


\subsection{Initialization of vegetation from bare ground}
\label{\detokenize{tech_note/DGVM/CLM50_Tech_Note_DGVM:initialization-of-vegetation-from-bare-ground}}
{[}5{]}\_If the model is restarted from a bare ground state (as opposed to a
pre-existing vegetation state), the state variables above are
initialized as follows. First, the number of plants per PFT is allocated
according to the initial seeding density (\(S_{init}\), individuals
per m\(^{2}\)) and the area of the patch \(A_{patch}\), which
in the first timestep is the same as the area of the notional ecosystem
\(A_{tot}\). The model has no meaningful spatial dimension, but we
assign a notional area such that the values of ‘\(n_{coh}\)’ can be
attributed. The default value of \(A_{tot}\) is one hectare (10,000
m\(^{2}\)), but the model will behave identically irrepective of
the value of this parameter.
\begin{equation*}
\begin{split}n_{coh,0} = S_{init}A_{patch}\end{split}
\end{equation*}
Each cohort is initialized at the minimum canopy height
\(h_{min,ft}\), which is specified as a parameter for each plant
functional type and denotes the smallest size of plant which is tracked
by the model. Smaller plants are not considered, and their emergence
from the recruitment processes is unresolved and therefore implicitly
parameterized in the seedling establishment model.. The diameter of each
cohort is then specified using the log-linear allometry between stem
diameter and canopy height
\begin{equation*}
\begin{split}\mathit{dbh}_{coh} = 10^{\frac{\log_{10}(h_{coh}) - c_{allom}}{m_{allom}}  }\end{split}
\end{equation*}
where the slope of the log-log relationship, \(m_{allom}\) is 0.64
and the intercept \(c_{allom}\) is 0.37. The structural biomass
associated with a plant of this diameter and height is given (as a
function of wood density, \(\rho\), g cm\(^{-3}\))
\begin{equation*}
\begin{split}b_{struc,coh} =c_{str}h_{coh}^{e_{str,hite}} dbh_{coh}^{e_{str,dbh}} \rho_{ft}^{e_{str,dens}}\end{split}
\end{equation*}
taken from the original ED1.0 allometry
{\hyperref[\detokenize{tech_note/References/CLM50_Tech_Note_References:mc-2001}]{\sphinxcrossref{\DUrole{std,std-ref}{Moorcroft et al. 2001}}}} (values of the allometric constants in
Table {\hyperref[\detokenize{tech_note/DGVM/CLM50_Tech_Note_DGVM:table:allom}]{\emph{{[}table:allom{]}}}}. The maximum amount of leaf
biomass associated with this diameter of tree is calculated according to
the following allometry
\begin{equation*}
\begin{split}b_{max,leaf,coh} =c_{leaf}\it{dbh}_{coh}^{e_{leaf,dbh}} \rho_{ft}^{e_{leaf,dens}}\end{split}
\end{equation*}
from this quantity, we calculate the active/fine root biomass
\(b_{root,coh}\) as
\begin{equation*}
\begin{split}b_{root,coh} =  b_{max,leaf,coh}\cdot f_{frla}\end{split}
\end{equation*}
where \(f_{frla}\) is the fraction of fine root biomass to leaf
biomass, assigned per PFT

\captionof{table}{Parameters needed for model initialization.}


\begin{savenotes}\sphinxattablestart
\centering
\begin{tabulary}{\linewidth}[t]{|T|T|T|T|}
\hline
\sphinxstylethead{\sphinxstyletheadfamily 
Parameter
Symbol
\unskip}\relax &\sphinxstylethead{\sphinxstyletheadfamily 
Parameter Name
\unskip}\relax &\sphinxstylethead{\sphinxstyletheadfamily 
Units
\unskip}\relax &\sphinxstylethead{\sphinxstyletheadfamily 
Default Value
\unskip}\relax \\
\hline
\(h_{min}\)
&
Minimum plant
height
&
m
&
1.5
\\
\hline
\(S_{init}\)
&
Initial
Planting
density
&
Individuals
m\(^{-2}\)
&\\
\hline
\(A_{tot}\)
&
Model area
&
m\(^{2}\)
&
10,000
\\
\hline
\end{tabulary}
\par
\sphinxattableend\end{savenotes}

{[}table:init{]}


\subsection{Allocation of biomass}
\label{\detokenize{tech_note/DGVM/CLM50_Tech_Note_DGVM:allocation-of-biomass}}
{[}6{]}\_Total live biomass \(b_{alive}\) is the state variable of the model
that describes the sum of the three live biomass pools leaf
\(b_{leaf}\), root \(b_{root}\) and sapwood \(b_{sw}\) (all
in kGC individual\(^{-1}\)). The quantities are constrained by the
following
\begin{equation*}
\begin{split}b_{alive} = b_{leaf} + b_{root} + b_{sw}\end{split}
\end{equation*}
Sapwood volume is a function of tree height and leaf biomass
\begin{equation*}
\begin{split}b_{sw} = b_{leaf}\cdot h_{coh}\cdot f_{swh}\end{split}
\end{equation*}
where \(f_{swh}\) is the ratio of sapwood mass (kgC) to leaf mass
per unit tree height (m). Also, root mass is a function of leaf mass
\begin{equation*}
\begin{split}b_{root} = b_{leaf}\cdot f_{swh}\end{split}
\end{equation*}
Thus
\begin{equation*}
\begin{split}b_{alive} = b_{leaf} + b_{leaf}\cdot f_{frla} + b_{leaf}\cdot h_{coh}\cdot f_{swh}\end{split}
\end{equation*}
Rearranging gives the fraction of biomass in the leaf pool
\(f_{leaf}\) as
\begin{equation*}
\begin{split}f_{leaf} = \frac{1}{1+h_{coh}\cdot f_{swh}+f_{frla} }\end{split}
\end{equation*}
Thus, we can determine the leaf fraction from the height at the tissue
ratios, and the phenological status of the cohort \(S_{phen,coh}\).
\begin{equation*}
\begin{split}b_{leaf} = b_{alive} \cdot l _{frac}\end{split}
\end{equation*}
To divide the live biomass pool at restart, or whenever it is
recalculated, into its consituent parts, we first
\begin{equation*}
\begin{split}b_{leaf} = \left\{ \begin{array}{ll}
b_{alive} \cdot l _{frac}&\textrm{for } S_{phen,coh} = 1\\
&\\
0&\textrm{for } S_{phen,coh} = 0\\
\end{array} \right.\end{split}
\end{equation*}
Because sometimes the leaves are dropped, using leaf biomass as a
predictor of root and sapwood would produce zero live biomass in the
winter. To account for this, we add the LAI memory variable
\(l_{memory}\) to the live biomass pool to account for the need to
maintain root biomass when leaf biomass is zero. Thus, to calculated the
root biomass, we use
\begin{equation*}
\begin{split}b_{root} = (b_{alive}+l_{memory})\cdot l_{frac} \cdot f_{frla}\end{split}
\end{equation*}
To calculated the sapwood biomass, we use
\begin{equation*}
\begin{split}b_{sw} = (b_{alive}+l_{memory})\cdot l_{frac} \cdot f_{swh} \cdot h_{coh}\end{split}
\end{equation*}
\captionof{table}{Allometric Constants}


\begin{savenotes}\sphinxattablestart
\centering
\begin{tabular}[t]{|*{4}{\X{1}{4}|}}
\hline
\sphinxstylethead{\sphinxstyletheadfamily 
Parameter
Symbol
\unskip}\relax &\sphinxstylethead{\sphinxstyletheadfamily 
Parameter Name
\unskip}\relax &\sphinxstylethead{\sphinxstyletheadfamily 
Units
\unskip}\relax &\sphinxstylethead{\sphinxstyletheadfamily 
Default Value
\unskip}\relax \\
\hline
\(c_{allom
}\)
&
Allometry
intercept
&&
0.37
\\
\hline
\(m_{allom
}\)
&
Allometry slope
&&
0.64
\\
\hline
\(c_{str}\)
&
Structural
biomass
multiplier
&&
0.06896
\\
\hline
\(e_{str,d
bh}\)
&
Structural
Biomass dbh
exponent
&&
1.94
\\
\hline
\(e_{str,h
ite}\)
&
Structural
Biomass height
exponent
&&
0.572
\\
\hline
\(e_{str,d
ens}\)
&
Structural
Biomass density
exponent
&&
0.931
\\
\hline
\(c_{leaf}\)
&
Leaf biomass
multiplier
&&
0.0419
\\
\hline
\(e_{leaf,
dbh}\)
&
Leaf biomass
dbh exponent
&&
1.56
\\
\hline
\(e_{leaf,
dens}\)
&
Leaf biomass
density
exponent
&&
0.55
\\
\hline
\(f_{swh}\)
&
Ratio of
sapwood mass to
height
&
m\(^{-1}\)
&\\
\hline
\(f_{frla}\)
&
Ratio of fine
root mass to
leaf mass
&\begin{itemize}
\item {} 
\end{itemize}
&
1.0
\\
\hline
\end{tabular}
\par
\sphinxattableend\end{savenotes}

{[}table:allom{]}


\subsection{Canopy Structure and the Perfect Plasticity Approximation}
\label{\detokenize{tech_note/DGVM/CLM50_Tech_Note_DGVM:canopy-structure-and-the-perfect-plasticity-approximation}}
{[}7{]}\_During initialization and every subsequent daily ED timestep, the canopy
structure model is called to determine how the leaf area of the
different cohorts is arranged relative to the incoming radiation, which
will then be used to drive the radiation and photosynthesis
calculations. This task requires that some assumptions are made about 1)
the shape and depth of the canopy within which the plant leaves are
arranged and 2) how the leaves of different cohorts are arranged
relative to each other. This set of assumptions are critical to model
performance in ED-like cohort based models, since they determine how
light resources are partitioned between competing plants of varying
heights, which has a very significant impact on how vegetation
distribution emerges from competition
{\hyperref[\detokenize{tech_note/References/CLM50_Tech_Note_References:fisher2010}]{\sphinxcrossref{\DUrole{std,std-ref}{Fisher et al. 2010}}}}.

The standard ED1.0 model makes a simple ‘flat disk’ assumption, that the
leaf area of each cohort is spread in an homogenous layer at one exact
height across entire the ground area represented by each patch. FATES has diverged from this representation due to (at least) two problematic emergent properties that we identified as generating unrealistic behaviours espetially for large-area patches.

1. Over-estimation of light competition . The vertical stacking of
cohorts which have all their leaf area at the same nominal height means
that when one cohort is only very slightly taller than it’s competitor,
it is completely shaded by it. This means that any small advantage in
terms of height growth translates into a large advantage in terms of
light competition, even at the seedling stage. This property of the
model artificially exaggerates the process of light competition. In
reality, trees do not compete for light until their canopies begin to
overlap and canopy closure is approached.

2. Unrealistic over-crowding. The ‘flat-disk’ assumption has no
consideration of the spatial extent of tree crowns. Therefore it has no
control on the packing density of plants in the model. Given a mismatch
between production and mortality, entirely unrealistic tree densities
are thus possible for some combinations of recruitment, growth and
mortality rates.

To account for the filling of space in three dimensions using the
one-dimensional representation of the canopy employed by CLM, we
implement a new scheme derived from that of
{\hyperref[\detokenize{tech_note/References/CLM50_Tech_Note_References:purves2008}]{\sphinxcrossref{\DUrole{std,std-ref}{Purves et al. 2008}}}}. Their argument follows the development
of an individual-based variant of the SORTIE model, called SHELL, which
allows the location of individual plant crowns to be highly flexible in
space. Ultimately, the solutions of this model possess an emergent
property whereby the crowns of the plants simply fill all of the
available space in the canopy before forming a distinct understorey.

Purves et al. developed a model that uses this feature, called the
‘perfect plasticity approximation’, which assumes the plants are able to
perfectly fill all of the available canopy space. That is, at canopy
closure, all of the available horizontal space is filled, with
negligible gaps, owing to lateral tree growth and the ability of tree
canopies to grow into the available gaps (this is of course, an
over-simplified but potential useful ecosystem property). The ‘perfect
plasticity approximation’ (PPA) implies that the community of trees is
subdivided into discrete canopy layers, and by extension, each cohort
represented by FATES model is assigned a canopy layer status flag,
\(C_L\). In this version, we set the maximum number of canopy layers
at 2 for simplicity, although is possible to have a larger number of
layers in theory. \(C_{L,coh}\) = 1 means that all the trees of
cohort \(coh\) are in the upper canopy (overstory), and
\(C_{L,coh}\) = 2 means that all the trees of cohort \(coh\) are
in the understorey.

In this model, all the trees in the canopy experience full light on
their uppermost leaf layer, and all trees in the understorey experience
the same light (full sunlight attenuated by the average LAI of the upper
canopy) on their uppermost leaves, as described in the radiation
transfer section (more nuanced versions of this approach may be
investigated in future model versions). The canopy is assumed to be
cylindrical, the lower layers of which experience self-shading by the
upper layers.

To determine whether a second canopy layer is required, the model needs
to know the spatial extent of tree crowns. Crown area,
\(A_{crown}\), m\(^{2}\), is defined as
\begin{equation*}
\begin{split}A_{crown,coh}  = \pi (dbh_{coh} S_{c,patch,Cl})^{1.56}\end{split}
\end{equation*}
where \(A_{crown}\) is the crown area of a single tree canopy
(m:math:\sphinxtitleref{\textasciicircum{}\{-2\}}) and \(S_{c,patch,Cl}\) is the ‘canopy spread’
parameter (m cm\textasciicircum{}-1) of this canopy layer, which is assigned as a
function of canopy space filling, discussed below. In contrast to
{\hyperref[\detokenize{tech_note/References/CLM50_Tech_Note_References:purves2008}]{\sphinxcrossref{\DUrole{std,std-ref}{Purves et al. 2008}}}} , we use an exponent, identical to that
for leaf biomass, of 1.56, not 2.0, such that tree leaf area index does
not change as a function of diameter.

To determine whether the canopy is closed, we calculate the total canopy
area as:
\begin{equation*}
\begin{split}A_{canopy} = \sum_{coh=1}^{nc,patch}{A_{crown,coh}.n_{coh}}\end{split}
\end{equation*}
where \(nc_{patch}\) is the number of cohorts in a given patch. If
the area of all crowns \(A_{canopy}\) (m:math:\sphinxtitleref{\textasciicircum{}\{-2\}}) is larger
than the total ground area of a patch (\(A_{patch}\)), then some
fraction of each cohort is demoted to the understorey.

Under these circumstances, the \sphinxtitleref{extra} crown area \(A_{loss}\)
(i.e., \(A_{canopy}\) - \(A_p\)) is moved into the understorey.
For each cohort already in the canopy, we determine a fraction of trees
that are moved from the canopy (\(L_c\)) to the understorey.
\(L_c\) is calculated as {\hyperref[\detokenize{tech_note/References/CLM50_Tech_Note_References:fisher2010}]{\sphinxcrossref{\DUrole{std,std-ref}{Fisher et al. 2010}}}}
\begin{equation*}
\begin{split}L_{c}= \frac{A_{loss,patch} w_{coh}}{\sum_{coh=1}^{nc,patch}{w_{coh}}} ,\end{split}
\end{equation*}
where \(w_{coh}\) is a weighting of each cohort determined by basal
diameter \(dbh\) (cm) and the competitive exclusion coefficient
\(C_{e}\)
\begin{equation*}
\begin{split}w_{coh}=dbh_{coh}C_{e}.\end{split}
\end{equation*}
The higher the value of \(C_e\) the greater the impact of tree
diameter on the probability of a given tree obtaining a position in the
canopy layer. That is, for high \(C_e\) values, competition is
highly deterministic. The smaller the value of \(C_e\), the greater
the influence of random factors on the competitive exclusion process,
and the higher the probability that slower growing trees will get into
the canopy. Appropriate values of \(C_e\) are poorly constrained but
alter the outcome of competitive processes.

The process by which trees are moved between canopy layers is complex
because 1) the crown area predicted for a cohort to lose may be larger
than the total crown area of the cohort, which requires iterative
solutions, and 2) on some occasions (e.g. after fire), the canopy may
open up and require ‘promotion’ of cohorts from the understorey, and 3)
canopy area may change due to the variations of canopy spread values (
\(S_{c,patch,Cl}\), see the section below for details) when
fractions of cohorts are demoted or promoted. Further details can be
found in the code references in the footnote.


\subsubsection{Horizontal Canopy Spread}
\label{\detokenize{tech_note/DGVM/CLM50_Tech_Note_DGVM:horizontal-canopy-spread}}
%
\begin{footnote}[8]\sphinxAtStartFootnote
This description relates to algorithms in the canopy\_spread
subroutine
%
\end{footnote}{\hyperref[\detokenize{tech_note/References/CLM50_Tech_Note_References:purves2008}]{\sphinxcrossref{\DUrole{std,std-ref}{Purves et al. 2008}}}} estimated the ratio between canopy and
stem diameter \(c_{p}\) as 0.1 m cm\(^{-1}\) for canopy trees
in North American forests, but this estimate was made on trees in closed
canopies, whose shape is subject to space competition from other
individuals. Sapling trees have no constraints in their horizontal
spatial structure, and as such, are more likely to display their leaves
to full sunlight. Also, prior to canopy closure, light interception by
leaves on the sides of the canopy is also higher than it would be in a
closed canopy forest. If the ‘canopy spread’ parameter is constant for
all trees, then we simulate high levels of self-shading for plants in
unclosed canopies, which is arguably unrealistic and can lower the
productivity of trees in areas of unclosed canopy (e.g. low productivity
areas of boreal or semi-arid regions where LAI and canopy cover might
naturally be low). We here interpret the degree of canopy spread,
\(S_{c}\) as a function of how much tree crowns interfere with each
other in space, or the total canopy area \(A_{canopy}\). However
\(A_{canopy}\) itself is a function of \(S_{c}\), leading to a
circularity. \(S_{c}\) is thus solved iteratively through time.

Each daily model step, \(A_{canopy}\) and the fraction of the
gridcell occupied by tree canopies in each canopy layer
(\(A_{f,Cl}\) = \(A_{canopy,Cl}\)/\(A_{patch}\)) is
calculated based on \(S_{c}\) from the previous timestep. If
\(A_{f}\) is greater than a threshold value \(A_{t}\),
\(S_{c}\) is increased by a small increment \(i\). The threshold
\(A_{t}\) is, hypothetically, the canopy fraction at which light
competition begins to impact on tree growth. This is less than 1.0 owing
to the non-perfect spatial spacing of tree canopies. If \(A_{f,Cl}\)
is greater than \(A_{t}\), then \(S_{c}\) is reduced by an
increment \(i\), to reduce the spatial extent of the canopy, thus.
\begin{equation*}
\begin{split}S_{c,patch,Cl,t+1} = \left\{ \begin{array}{ll}
S_{c,patch,Cl,t} + i& \textrm{for $A_{f,Cl} < A_{t}$}\\
&\\
S_{c,patch,Cl,t} - i& \textrm{for $A_{f,Cl} > A_{t}$}\\
\end{array} \right.\end{split}
\end{equation*}
The values of \(S_{c}\) are bounded to upper and lower limits. The
lower limit corresponds to the observed canopy spread parameter for
canopy trees \(S_{c,min}\) and the upper limit corresponds to the
largest canopy extent \(S_{c,max}\)
\begin{equation*}
\begin{split}S_{c,patch,Cl} = \left\{ \begin{array}{ll}
S_{c,min}& \textrm{for } S_{c,patch,Cl}< S_{c,\rm{min}}\\
&\\
S_{c,max}& \textrm{for } S_{c,patch,Cl} > S_{c,\rm{max}}\\
\end{array} \right.\end{split}
\end{equation*}
This iterative scheme requires two additional parameters (\(i\) and
\(A_{t}\)). \(i\) affects the speed with which canopy spread
(and hence leaf are index) increase as canopy closure is neared.
However, the model is relatively insensitive to the choice of either
\(i\) or \(A_{t}\).


\subsubsection{Definition of Leaf and Stem Area Profile}
\label{\detokenize{tech_note/DGVM/CLM50_Tech_Note_DGVM:definition-of-leaf-and-stem-area-profile}}
{[}9{]}\_Within each patch, the model defines and tracks cohorts of multiple
plant functional types that exist either in the canopy or understorey.
Light on the top leaf surface of each cohort in the canopy is the same,
and the rate of decay through the canopy is also the same for each PFT.
Therefore, we accumulate all the cohorts of a given PFT together for the
sake of the radiation and photosynthesis calculations (to avoid separate
calculations for every cohort).

Therefore, the leaf area index for each patch is defined as a
three-dimensional array \(\mathit{lai}_{Cl,ft,z}\) where \(C_l\)
is the canopy layer, \(ft\) is the functional type and \(z\) is
the leaf layer within each canopy. This three-dimensional structure is
the basis of the radiation and photosynthetic models. In addition to a
leaf area profile matrix, we also define, for each patch, the area which
is covered by leaves at each layer as \(\mathit{carea}_{Cl,ft,z}\).

Each plant cohort is already defined as a member of a single canopy
layer and functional type. This means that to generate the
\(x_{Cl,ft,z}\) matrix, it only remains to divide the leaf area of
each cohort into leaf layers. First, we determine how many leaf layers
are occupied by a single cohort, by calculating the ‘tree LAI’ as the
total leaf area of each cohort divided by its crown area (both in
m\(^{2}\))
\begin{equation*}
\begin{split}\mathit{tree}_{lai,coh} = \frac{b_{leaf,coh}\cdot\mathrm{sla}_{ft}}{A_{crown,coh}}\end{split}
\end{equation*}
where \(\mathrm{sla}_{ft}\) is the specific leaf area in
m\(^{2}\) KgC\(^{-1}\) and \(b_{leaf}\) is in kGC per
plant.

Stem area index (SAI) is ratio of the total area of all woody stems on a
plant to the area of ground covered by the plant. During winter in
deciduous areas, the extra absorption by woody stems can have a
significant impact on the surface energy budget. However, in previous
\sphinxtitleref{big leaf} versions of the CLM, computing the circumstances under which
stem area was visible in the absence of leaves was difficult and the
algorithm was largely heuristic as a result. Given the multi-layer
canopy introduced for FATES, we can determine the leaves in the higher
canopy layers will likely shade stem area in the lower layers when
leaves are on, and therefore stem area index can be calculated as a
function of woody biomass directly.

Literature on stem area index is particularly poor, as it’s estimation
is complex and not particularly amenable to the use of, for example,
assumptions of random distribution in space that are typically used to
calculate leaf area from light interception.
{\hyperref[\detokenize{tech_note/References/CLM50_Tech_Note_References:kucharik1998}]{\sphinxcrossref{\DUrole{std,std-ref}{Kucharik et al. 1998}}}} estimated that SAI visible from an
LAI2000 sensor was around 0.5 m\textasciicircum{}2 m\textasciicircum{}-2. \DUrole{xref,std,std-ref}{Low et al. 2001}
estimate that the wood area index for Ponderosa Pine forest is
0.27-0.33. The existing CLM(CN) algorithm sets the minimum SAI at 0.25
to match MODIS observations, but then allows SAI to rise as a function
of the LAI lost, meaning than in some places, predicted SAI can reach
value of 8 or more. Clearly, greater scientific input on this quantity
is badly needed. Here we determine that SAI is a linear function of
woody biomass, to at very least provide a mechanistic link between the
existence of wood and radiation absorbed by it. The non-linearity
between how much woody area exists and how much radiation is absorbed is
provided by the radiation absorption algorithm. Specifically, the SAI of
an individual cohort (\(\mathrm{tree}_{sai,coh}\), m\(^{2}\)
m\(^{-2}\)) is calculated as follows,
\begin{equation*}
\begin{split}\mathrm{tree}_{sai,coh} = k_{sai}\cdot b_{struc,coh} ,\end{split}
\end{equation*}
where \(k_{sai}\) is the coefficient linking structural biomass to
SAI. The number of occupied leaf layers for cohort \(coh\)
(\(n_{z,coh}\)) is then equal to the rounded up integer value of the
tree SAI (\({tree}_{sai,coh}\)) and LAI (\({tree}_{lai,coh}\))
divided by the layer thickness (i.e., the resolution of the canopy layer
model, in units of vegetation index (\(lai\)+\(sai\)) with a
default value of 1.0, \(\delta _{vai}\) ),
\begin{equation*}
\begin{split}n_{z,coh} = {\frac{\mathrm{tree}_{lai,coh}+\mathrm{tree}_{sai,coh}}{\delta_{vai}}}.\end{split}
\end{equation*}
The fraction of each layer that is leaf (as opposed to stem) can then be
calculated as
\begin{equation*}
\begin{split}f_{leaf,coh} = \frac{\mathrm{tree}_{lai,coh}}{\mathrm{tree}_{sai,coh}+\mathrm{tree}_{lai,coh}}.\end{split}
\end{equation*}
Finally, the leaf area in each leaf layer pertaining to this cohort is
thus
\begin{align*}\!\begin{aligned}
\mathit{lai}_{z,coh}  = \left\{ \begin{array}{ll}
   \delta_{vai} \cdot f_{leaf,coh} \frac{A_{canopy,coh}}{A_{canopy,patch}}& \textrm{for $i=1,..., i=n_{z,coh}-1$}\\
  &\\
   \delta_{vai} \cdot f_{leaf,coh} \frac{A_{canopy,coh}}{A_{canopy,patch}}\cdot r_{vai}& \textrm{for $i=n_{z,coh}$}\\
  \end{array} \right.\\
and the stem area index is\\
\end{aligned}\end{align*}\begin{equation*}
\begin{split}\mathit{sai}_{z,coh}  = \left\{ \begin{array}{ll}
 \delta_{vai} \cdot (1-f_{leaf,coh})\frac{A_{canopy,coh}}{A_{canopy,patch}}& \textrm{for $i=1,..., i=n_{z,coh}-1$}\\
&\\
 \delta_{vai} \cdot (1-f_{leaf,coh}) \frac{A_{canopy,coh}}{A_{canopy,patch}}\cdot r_{vai}& \textrm{for $i=n_{z,coh}$}\\
\end{array} \right.\end{split}
\end{equation*}
where \(r_{vai}\) is the remainder of the canopy that is below the
last full leaf layer
\begin{equation*}
\begin{split}r_{vai} =(\mathrm{tree}_{lai,coh} + \mathrm{tree}_{sai,coh}) - (\delta _{vai} \cdot (n_{z,coh} -1)).\end{split}
\end{equation*}
\(A_{canopy,patch}\) is the total canopy area occupied by plants in
a given patch (m:math:\sphinxtitleref{\textasciicircum{}\{2\}}) and is calculated as follows,
\begin{equation*}
\begin{split}A_{canopy,patch} = \textrm{min}\left( \sum_{coh=1}^{coh = ncoh}A_{canopy,coh}, A_{patch}  \right).\end{split}
\end{equation*}
The canopy is conceived as a cylinder, although this assumption could be
altered given sufficient evidence that canopy shape was an important
determinant of competitive outcomes, and the area of ground covered by
each leaf layer is the same through the cohort canopy. With the
calculated SAI and LAI, we are able to calculate the complete canopy
profile. Specifically, the relative canopy area for the cohort
\({coh}\) is calculated as
\begin{equation*}
\begin{split}\mathit{area}_{1:nz,coh}  =  \frac{A_{crown,coh}}{A_{canopy,patch}}.\end{split}
\end{equation*}
The total occupied canopy area for each canopy layer (\(Cl\)), plant
functional type (\(ft\)) and leaf layer (\(z\)) bin is thus
\begin{equation*}
\begin{split}\mathit{c}_{area,Cl,ft,z} = \sum_{coh=1}^{coh=ncoh} area_{1:nz,coh}\end{split}
\end{equation*}
where \(ft_{coh}=ft\)  and  \(Cl_{coh} = Cl.\)

All of these quantities are summed across cohorts to give the complete
leaf and stem area profiles,
\begin{equation*}
\begin{split}\mathit{lai} _{Cl,ft,z} = \sum_{coh=1}^{coh=ncoh} \mathit{lai}_{z,coh}\end{split}
\end{equation*}\begin{equation*}
\begin{split}\mathit{sai}_{Cl,ft,z} = \sum_{coh=1}^{coh=ncoh} \mathit{sai}_{z,coh}\end{split}
\end{equation*}

\subsubsection{Burial of leaf area by snow}
\label{\detokenize{tech_note/DGVM/CLM50_Tech_Note_DGVM:burial-of-leaf-area-by-snow}}
The calculations above all pertain to the total leaf and stem area
indices which charecterize the vegetation structure. In addition, the
model must know when the vegetation is covered by snow, and by how much,
so that the albedo and energy balance calculations can be adjusted
accordingly. Therefore, we calculated a ‘total’ and ‘exposed’
\(lai\) and \(sai\) profile using a representation of the bottom
and top canopy heights, and the depth of the average snow pack. For each
leaf layer \(z\) of each cohort, we calculate an ‘exposed fraction
\(f_{exp,z}\) via consideration of the top and bottom heights of
that layer \(h_{top,z}\) and \(h_{bot,z}\) (m),
\begin{equation*}
\begin{split}\begin{array}{ll}
h_{top,z} = h_{coh} - h_{coh}\cdot f_{crown,ft}\cdot\frac{z}{n_{z,coh}}& \\
&\\
h_{bot,z} = h_{coh} - h_{coh}\cdot f_{crown,ft}\cdot\frac{z+1}{n_{z,coh}}&\\
\end{array}\end{split}
\end{equation*}
where \(f_{crown,ft}\) is the plant functional type (\(ft\))
specific fraction of the cohort height that is occupied by the crown.
Specifically, the ‘exposed fraction \(f_{exp,z}\) is calculated as
follows,
\begin{equation*}
\begin{split}f_{exp,z}\left\{ \begin{array}{ll}
= 1.0 &  h_{bot,z}> d_{snow}\\
&\\
= \frac{d_{snow} -h_{bot,z}}{h_{top,z}-h_{bot,z}}  & h_{top,z}> d_{snow}, h_{bot,z}< d_{snow}\\
&\\
= 0.0 & h_{top,z}< d_{snow}\\
\end{array} \right.\end{split}
\end{equation*}
The resulting exposed (\(elai, esai\)) and total
(\(tlai, tsai\)) leaf and stem area indicies are calculated as
\begin{equation*}
\begin{split}\begin{array}{ll}
\mathit{elai} _{Cl,ft,z} &= \mathit{lai} _{Cl,ft,z} \cdot f_{exp,z}\\
\mathit{esai} _{Cl,ft,z} &= \mathit{sai} _{Cl,ft,z} \cdot f_{exp,z}\\
\mathit{tlai} _{Cl,ft,z} &= \mathit{lai} _{Cl,ft,z}\\
\mathit{tsai} _{Cl,ft,z} &= \mathit{sai} _{Cl,ft,z} \
\end{array} ,\end{split}
\end{equation*}
and are used in the radiation interception and photosynthesis algorithms
described later.


\begin{savenotes}\sphinxattablestart
\centering
\begin{tabulary}{\linewidth}[t]{|T|T|T|T|T|}
\hline
\sphinxstylethead{\sphinxstyletheadfamily 
Parameter
Symbol
\unskip}\relax &\sphinxstylethead{\sphinxstyletheadfamily 
Parameter
Name
\unskip}\relax &\sphinxstylethead{\sphinxstyletheadfamily 
Units
\unskip}\relax &\sphinxstylethead{\sphinxstyletheadfamily 
Notes
\unskip}\relax &\sphinxstylethead{\sphinxstyletheadfamily 
Indexed by
\unskip}\relax \\
\hline
:math:{}`
delta\_
\{vai\}{}`
&
Thickness
of single
canopy
layer
&
m\(^
{-2}\)m:
math:\sphinxtitleref{\textasciicircum{}\{-2\}}
&&\\
\hline
\(C_e\)
&
Competitive
Exclusion
Parameter
&
none
&&\\
\hline
\(c_{p
,min}\)
&
Minimum
canopy
spread
&
m\(^
{2}\)
cm:math:{}`
\textasciicircum{}\{-1\}{}`
&&\\
\hline
\(c_{p
,max}\)
&
Competitive
Exclusion
Parameter
&
m\(^
{2}\)
cm:math:{}`
\textasciicircum{}\{-1\}{}`
&&\\
\hline
\(i\)
&
Incremental
change in
\(c_p\)
&
m\(^
{2}\)
cm:math:{}`
\textasciicircum{}\{-1\}{}`
y\(^
{-1}\)
&&\\
\hline
\(A_t\)
&
Threshold
canopy
closure
&
none
&&\\
\hline
\(f_{c
rown,ft}\)
&
Crown
fraction
&
none
&&
\(ft\)
\\
\hline
\(k_{s
ai}\)
&
Stem area
per unit
woody
biomass
&
m\textasciicircum{}2 KgC\textasciicircum{}-1
&&\\
\hline
\end{tabulary}
\par
\sphinxattableend\end{savenotes}


\subsection{Radiation Transfer}
\label{\detokenize{tech_note/DGVM/CLM50_Tech_Note_DGVM:radiation-transfer}}

\subsubsection{Fundamental Radiation Transfer Theory}
\label{\detokenize{tech_note/DGVM/CLM50_Tech_Note_DGVM:fundamental-radiation-transfer-theory}}
{[}10{]}\_The first interaction of the land surface with the properties of
vegetation concerns the partitioning of energy into that which is
absorbed by vegetation, reflected back into the atmosphere, and absorbed
by the ground surface. Older versions of the CLM have utilized a
“two-stream” approximation
{\hyperref[\detokenize{tech_note/References/CLM50_Tech_Note_References:sellers1985}]{\sphinxcrossref{\DUrole{std,std-ref}{Sellers 1985}}}}, {\hyperref[\detokenize{tech_note/References/CLM50_Tech_Note_References:sellers1996}]{\sphinxcrossref{\DUrole{std,std-ref}{Sellers et al. 1986}}}} that provided an
empirical solution for the radiation partitioning of a multi-layer
canopy for two streams, of diffuse and direct light. However,
implementation of the Ecosystem Demography model requires a) the
adoption of an explicit multiple layer canopy b) the implementation of a
multiple plant type canopy and c) the distinction of canopy and
under-storey layers, in-between which the radiation streams are fully
mixed. The radiation mixing between canopy layers is necessary as the
position of different plants in the under-storey is not defined
spatially or relative to the canopy trees above. In this new scheme, we
thus implemented a one-dimensional scheme that traces the absorption,
transmittance and reflectance of each canopy layer and the soil,
iterating the upwards and downwards passes of radiation through the
canopy until a pre-defined accuracy tolerance is reached. This approach
is based on the work of {\hyperref[\detokenize{tech_note/References/CLM50_Tech_Note_References:norman1979}]{\sphinxcrossref{\DUrole{std,std-ref}{Norman 1979}}}}.

Here we describe the basic theory of the radiation transfer model for
the case of a single homogenous canopy, and in the next section we
discuss how this is applied to the multi layer multi PFT canopy in the
FATES implementation. The code considers the fractions of a single
unit of incoming direct and a single unit of incoming diffuse light,
that are absorbed at each layer of the canopy for a given solar angle
(\(\alpha_{s}\), radians). Direct radiation is extinguished through
the canopy according to the coefficient \(k_{dir}\) that is
calculated from the incoming solar angle and the dimensionless leaf
angle distribution parameter (\(\chi\)) as
\begin{equation*}
\begin{split}k_{dir} = g_{dir} / \sin(\alpha_s)\\\end{split}
\end{equation*}
where
\begin{equation*}
\begin{split}g_{dir} = \phi_1 + \phi_2 \cdot \sin(\alpha_s)\\\end{split}
\end{equation*}
and
\begin{align*}\!\begin{aligned}
\begin{array} {l}
\phi_1 = 0.5 - 0.633\chi_{l} - 0.33\chi_l ^2\\
\phi_2 =0.877 (1 - 2\phi_1)\\\\
\end{array}\\
\end{aligned}\end{align*}
The leaf angle distribution is a descriptor of how leaf surfaces are
arranged in space. Values approaching 1.0 indicate that (on average) the
majority of leaves are horizontally arranged with respect to the ground.
Values approaching -1.0 indicate that leaves are mostly vertically
arranged, and a value of 0.0 denotes a canopy where leaf angle is random
(a ‘spherical’ distribution).

According to Beer’s Law, the fraction of light that is transferred
through a single layer of vegetation (leaves or stems) of thickness
\(\delta_{vai}\), without being intercepted by any surface, is
\begin{equation*}
\begin{split}\mathit{tr}_{dir} = e^{-k_{dir}  \delta_{vai}}\end{split}
\end{equation*}
and the incident direct radiation transmitted to each layer of the
canopy (\(dir_{tr,z}\)) is thus calculated from the cumulative leaf
area ( \(L_{above}\) ) shading each layer (\(z\)):
\begin{equation*}
\begin{split}\mathit{dir}_{tr,z} = e^{-k_{dir}  L_{above,z}}\end{split}
\end{equation*}
The fraction of the leaves \(f_{sun}\) that are exposed to direct
light is also calculated from the decay coefficient \(k_{dir}\).
\begin{equation*}
\begin{split}\begin{array}{l}
f_{sun,z} = e^{-k_{dir}  L_{above,z}}\\
 \rm{and}
\\ f_{shade,z} = 1-f_{sun,z}
\end{array}\end{split}
\end{equation*}
where \(f_{shade,z}\) is the fraction of leaves that are shaded
from direct radiation and only receive diffuse light.

Diffuse radiation, by definition, enters the canopy from a spectrum of
potential incident directions, therefore the un-intercepted transfer
(\(tr_{dif}\)) through a leaf layer of thickness \(\delta_l\) is
calculated as the mean of the transfer rate from each of 9 different
incident light directions (\(\alpha_{s}\)) between 0 and 180 degrees
to the horizontal.
\begin{equation*}
\begin{split}\mathit{tr}_{dif} = \frac{1}{9} \sum\limits_{\alpha_s=5\pi/180}^{\alpha_s=85\pi/180} e^{-k_{dir,l} \delta_{vai}} \\ \\\end{split}
\end{equation*}\begin{equation*}
\begin{split}tr_{dif}= \frac{1}{9} \pi \sum_{\alpha s=0}^{ \pi / 2}  \frac{e^{-gdir} \alpha_s}{\delta_{vai} \cdot \rm{sin}(\alpha_s) \rm{sin}(\alpha_s) \rm{cos}(\alpha_s)}\end{split}
\end{equation*}
The fraction (1-\(tr_{dif}\)) of the diffuse radiation is
intercepted by leaves as it passes through each leaf layer. Of this,
some fraction is reflected by the leaf surfaces and some is transmitted
through. The fractions of diffuse radiation reflected from
(\(\mathit{refl}_{dif}\)) and transmitted though
(\(\mathit{tran}_{dif}\)) each layer of leaves are thus,
respectively
\begin{equation*}
\begin{split}\begin{array}{l}
\mathit{refl_{dif}} = (1 - tr_{dif})  \rho_{l,ft}\\
\mathit{tran}_{dif} = (1 - tr_{dif})  \tau_{l,ft} + tr_{dif}
\end{array}\end{split}
\end{equation*}
where \(\rho_{l,ft}\) and \(\tau_{l,ft}\) are the fractions of
incident light reflected and transmitted by individual leaf surfaces.

Once we know the fractions of light that are transmitted and reflected
by each leaf layer, we begin the process of distributing light through
the canopy. Starting with the first leaf layer (\(z\)=1), where
the incident downwards diffuse radiation (\(\mathit{dif}_{down}\))
is 1.0, we work downwards for \(n_z\) layers, calculating the
radiation in the next layer down (\(z+1\)) as:
\begin{equation*}
\begin{split}\mathit{dif}_{down,z+1} = \frac{\mathit{dif}_{down,z} \mathit{tran}_{dif} }    {1 - \mathit{r}_{z+1}  \mathit{refl}_{dif}}\end{split}
\end{equation*}
Here, \(\mathit{dif}_{down,z} \mathit{tran}_{dif}\) calculates the
fraction of incoming energy transmitted downwards onto layer
\(z+1\). This flux is then increased by the additional radiation
\(r_z\) that is reflected upwards from further down in the canopy to
layer \(z\), and then is reflected back downwards according to the
reflected fraction \(\mathit{refl_{dif}}\). The more radiation in
\(\mathit{r}_{z+1}  \mathit{refl}_{dif}\), the smaller the
denominator and the larger the downwards flux. \(r\) is also
calculated sequentially, starting this time at the soil surface layer
(where \(z = n_z+1\))
\begin{equation*}
\begin{split}r_{nz+1} = alb_s\end{split}
\end{equation*}
where \(alb_s\) is the soil albedo characteristic. The upwards
reflected fraction \(r_z\) for each leaf layer, moving upwards, is
then {\hyperref[\detokenize{tech_note/References/CLM50_Tech_Note_References:norman1979}]{\sphinxcrossref{\DUrole{std,std-ref}{Norman 1979}}}}
\begin{equation*}
\begin{split}r_z  = \frac{r_{z+1}  \times \mathit{tran}_{dif}  ^{2} }{ (1 - r_{z+1}  \mathit{refl_{dif}}) + \mathit{refl_{dif}}}.\end{split}
\end{equation*}
The corresponding upwards diffuse radiation flux is therefore the
fraction of downwards radiation that is incident on a particular layer,
multiplied by the fraction that is reflected from all the lower layers:
\begin{equation*}
\begin{split}\mathit{dif}_{up,z} = r_z \mathit{dif}_{down,z+1}\end{split}
\end{equation*}
Now we have initial conditions for the upwards and downwards diffuse
fluxes, these must be modified to account for the fact that, on
interception with leaves, direct radiation is transformed into diffuse
radiation. In addition, the initial solutions to the upwards and
downwards radiation only allow a single ‘bounce’ of radiation through
the canopy, so some radiation which might be intercepted by leaves
higher up is potentially lost. Therefore, the solution to this model is
iterative. The iterative solution has upwards and a downwards components
that calculate the upwards and downwards fluxes of total radiation at
each leaf layer (\(rad_{dn, z}\) and \(rad_{up, z}\)) . The
downwards component begins at the top canopy layer (\(z=1\)). Here
we define the incoming solar diffuse and direct radiation
(\(\it{solar}_{dir}\) and \(\it{solar}_{dir}\) respectively).
\begin{equation*}
\begin{split}\begin{array}{l}
 \mathit{dif}_{dn,1} =  \it{solar}_{dif} \\
\mathit{rad}_{dn, z+1} = \mathit{dif}_{dn,z} \cdot  \mathit{tran}_{dif}  +\mathit{dif}_{up,z+1}   \cdot  \mathit{refl}_{dif}   + \mathit{solar}_{dir}  \cdot  dir_{tr,z}  (1- tr_{dir})  \tau_l.
\end{array}\end{split}
\end{equation*}
The first term of the right-hand side deals with the diffuse radiation
transmitted downwards, the second with the diffuse radiation travelling
upwards, and the third with the direct radiation incoming at each layer
(\(dir_{tr,z}\)) that is intercepted by leaves
(\(1-  tr_{dir}\)) and then transmitted through through the leaf
matrix as diffuse radiation (\(\tau_l\)). At the bottom of the
canopy, the light reflected off the soil surface is calculated as
\begin{equation*}
\begin{split}rad _{up, nz} =  \rm{\it{dif}}_{down,z}  \cdot  salb_{dif} +\it{solar}_{dir} \cdot dir_{tr,z} salb_{dir}.\end{split}
\end{equation*}
The upwards propagation of the reflected radiation is then
\begin{equation*}
\begin{split}rad_{up, z} = \mathit{dif}_{up,z+1} \cdot  \mathit{tran}_{dif}  +\mathit{dif}_{dn,z}   \cdot  \mathit{refl}_{dif}   + \it{solar}_{dir}  \cdot  dir_{tr,z}  (1- tr_{dir})  \rho_l.\end{split}
\end{equation*}
Here the first two terms deal with the diffuse downwards and upwards
fluxes, as before, and the third deals direct beam light that is
intercepted by leaves and reflected upwards. These upwards and downwards
fluxes are computed for multiple iterations, and at each iteration,
\(rad_{up, z}\) and \(rad_{down, z}\) are compared to their
values in the previous iteration. The iteration scheme stops once the
differences between iterations for all layers is below a predefined
tolerance factor, (set here at \(10^{-4}\)). Subsequently, the
fractions of absorbed direct (\(abs_{dir,z}\)) and diffuse
(\(abs_{dif,z}\)) radiation for each leaf layer then
\begin{equation*}
\begin{split}abs_{dir,z} = \it{solar}_{dir}   \cdot dir_{tr,z} \cdot (1- tr_{dir}) \cdot (1 - \rho_l-\tau_l)\end{split}
\end{equation*}\begin{equation*}
\begin{split}abs_{dif,z} = (\mathit{dif}_{dn,z} +  \mathit{dif}_{up,z+1} ) \cdot (1 - tr_{dif}) \cdot (1 - \rho_l-\tau_l).\end{split}
\end{equation*}
and, the radiation energy absorbed by the soil for the diffuse and
direct streams is is calculated as
\begin{equation*}
\begin{split}\it{abs}_{soil} = \mathit{dif}_{down,nz+1} \cdot (1 -  salb_{dif}) +\it{solar}_{dir}   \cdot dir_{tr,nz+1} \cdot (1-  salb_{dir}).\end{split}
\end{equation*}
Canopy level albedo is denoted as the upwards flux from the top leaf
layer
\begin{equation*}
\begin{split}\it{alb}_{canopy}=  \frac{\mathit{dif}_{up,z+1}  }{  \it{solar}_{dir} + \it{solar}_{dif}}\end{split}
\end{equation*}
and the division of absorbed energy into sunlit and shaded leaf
fractions, (required by the photosynthesis calculations), is
\begin{equation*}
\begin{split}abs_{sha,z} = abs_{dif,z} \cdot f_{sha}\end{split}
\end{equation*}\begin{equation*}
\begin{split}abs_{sun,z} =  abs_{dif,z} \cdot f_{sun}+ abs_{dir,z}\end{split}
\end{equation*}

\subsubsection{Resolution of radiation transfer theory within the FATES canopy structure}
\label{\detokenize{tech_note/DGVM/CLM50_Tech_Note_DGVM:resolution-of-radiation-transfer-theory-within-the-fates-canopy-structure}}
The radiation transfer theory above, was described with reference to a
single canopy of one plant functional type, for the sake of clarity of
explanation. The FATES model, however, calculates radiative and
photosynthetic fluxes for a more complex hierarchical structure within
each patch/time-since-disturbance class, as described in the leaf area
profile section. Firstly, we denote two or more canopy layers (denoted
\(C_l\)). The concept of a ‘canopy layer’ refers to the idea that
plants are organized into discrete over and under-stories, as predicted
by the Perfect Plasticity Approximation
({\hyperref[\detokenize{tech_note/References/CLM50_Tech_Note_References:purves2008}]{\sphinxcrossref{\DUrole{std,std-ref}{Purves et al. 2008}}}}, {\hyperref[\detokenize{tech_note/References/CLM50_Tech_Note_References:fisher2010}]{\sphinxcrossref{\DUrole{std,std-ref}{Fisher et al. 2010}}}}). Within each canopy layer
there potentially exist multiple cohorts of different plant functional
types and heights. Within each canopy layer, \(C_l\), and functional
type, \(ft\), the model resolves numerous leaf layers \(z\),
and, for some processes, notably photosynthesis, each leaf layer is
split into a fraction of sun and shade leaves, \(f_{sun}\) and
\(f_{sha}\), respectively.

The radiation scheme described in Section is solved explicitly for this
structure, for both the visible and near-infrared wavebands, according
to the following assumptions.
\begin{itemize}
\item {} 
A \sphinxstyleemphasis{canopy layer} (\(C_{L}\)) refers to either the over or understorey

\item {} 
A \sphinxstyleemphasis{leaf layer} (\(z\)) refers to the discretization of the LAI
within the canopy of a given plant functional type.

\item {} 
All PFTs in the same canopy layer have the same solar radiation
incident on the top layer of the canopy

\item {} 
Light is transmitted through the canopy of each plant functional type independently

\item {} 
Between canopy layers, the light streams from different plant
functional types are mixed, such that the (undefined) spatial
location of plants in lower canopy layers does not impact the amount
of light received.

\item {} 
Where understorey layers fill less area than the overstorey layers,
radiation is directly transferred to the soil surface.

\item {} 
All these calculations pertain to a single patch, so we omit the
\sphinxtitleref{patch} subscript for simplicity in the following discussion.

\end{itemize}

Within this framework, the majority of the terms in the radiative
transfer scheme are calculated with indices of \(C_L\),
\(\it{ft}\) and \(z\). In the following text, we revisit the
simplified version of the radiation model described above, and explain
how it is modified to account for the more complex canopy structure used
by FATES.

Firstly, the light penetration functions, \(k_{dir}\) and
\(g_{dir}\) are described as functions of \(\it{ft}\), because
the leaf angle distribution, \(\chi_l\), is a pft-specific
parameter. Thus, the diffuse irradiance transfer rate, \(tr_{dif}\)
is also \(\it{ft}\) specific because \(g_{dir}\), on which it
depends, is a function of \(\chi_l\).

The amount of direct light reaching each leaf layer is a function of the
leaves existing above the layer in question. If a leaf layer ‘\(z\)’
is in the top canopy layer (the over-storey), it is only shaded by
leaves of the same PFT so \(k_{dir}\) is unchanged from equation. If
there is more than one canopy layer (\(C_{l,max}>1\)), then the
amount of direct light reaching the top leaf surfaces of the
second/lower layer is the weighted average of the light attenuated by
all the parallel tree canopies in the canopy layer above, thus.
\begin{equation*}
\begin{split}dir_{tr,Cl,:,1} =\sum_{ft=1}^{npft}{(dir_{tr,Cl,ft,z_{max}} \cdot c_{area,Cl-1,ft,z_{max}})}\end{split}
\end{equation*}
where \(\it{pft}_{wt}\) is the areal fraction of each canopy layer
occupied by each functional type and \(z_{max}\) is the index of the
bottom canopy layer of each pft in each canopy layer (the subscripts
\(C_l\) and \(ft\) are implied but omitted from all
\(z_{max}\) references to avoid additional complications)

Similarly, the sunlit fraction for a leaf layer ‘\(z\)’ in the
second canopy layer (where \(C_l > 1\)) is
\begin{equation*}
\begin{split}f_{sun,Cl,ft,z} = W_{sun,Cl} \cdot e^{k_{dir,ft,laic,z}}\end{split}
\end{equation*}
where \(W_{sun,Cl}\) is the weighted average sunlit fraction in the
bottom layer of a given canopy layer.
\begin{equation*}
\begin{split}W_{sun,Cl} = \sum_{ft=1}^{npft}{(f_{sun,Cl-1,ft,zmax} \cdot  c_{area,Cl-1,ft,zmax})}\end{split}
\end{equation*}
Following through the sequence of equations for the simple single pft
and canopy layer approach above, the \(\mathit{refl}_{dif}\) and
\(\mathit{tran}_{dif}\) fluxes are also indexed by \(C_l\),
\(\it{ft}\), and \(z\). The diffuse radiation reflectance ratio
\(r_z\) is also calculated in a manner that homogenizes fluxes
between canopy layers. For the canopy layer nearest the soil
(\(C_l\) = \(C_{l,max}\)). For the top canopy layer
(\(C_l\)=1), a weighted average reflectance from the lower layers
is used as the baseline, in lieu of the soil albedo. Thus:
\begin{equation*}
\begin{split}r_{z,Cl,:,1} =  \sum_{ft=1}^{npft}{(r_{z,Cl-1,ft,1}   \it{pft}_{wt,Cl-1,ft,1})}\end{split}
\end{equation*}
For the iterative flux resolution, the upwards and downwards fluxes are
also averaged between canopy layers, thus where \(C_l>1\)
\begin{equation*}
\begin{split}rad_{dn, Cl,ft,1} = \sum_{ft=1}^{npft}{(rad_{dn, Cl-1,ft,zmax} \cdot  \it{pft}_{wt,Cl-1,ft,zmax})}\end{split}
\end{equation*}
and where \(C_l\) =1, and \(C_{l,max}>1\)
\begin{equation*}
\begin{split}rad_{up,Cl,ft,zmax} = \sum_{ft=1}^{npft}{(rad_{up, Cl+1,ft,1} \cdot  \it{pft}_{wt,Cl+1,ft,1})}\end{split}
\end{equation*}
The remaining terms in the radiation calculations are all also indexed
by \(C_l\), \(ft\) and \(z\) so that the fraction of
absorbed radiation outputs are termed \(abs_{dir,Cl,ft,z}\) and
\(abs_{dif,Cl,ft,z}\). The sunlit and shaded absorption rates are
therefore
\begin{equation*}
\begin{split}abs_{sha,Cl,ft,z} = abs_{dif,Cl,ft,z}\cdot f_{sha,Cl,ft,z}\end{split}
\end{equation*}
and
\begin{equation*}
\begin{split}abs_{sun,Cl,ft,z} =  abs_{dif,Cl,ft,z} \cdot f_{sun,Cl,ft,z}+ abs_{dir,Cl,ft,z}\end{split}
\end{equation*}
The albedo of the mixed pft canopy is calculated as the weighted average
of the upwards radiation from the top leaf layer of each pft where
\(C_l\)=1:
\begin{equation*}
\begin{split}\it{alb}_{canopy}=  \sum_{ft=1}^{npft}{\frac{\mathit{dif}_{up,1,ft,1}    \it{pft}_{wt,1,ft,1}} {\it{solar}_{dir} + \it{solar}_{dif}}}\end{split}
\end{equation*}
The radiation absorbed by the soil after passing through through
under-storey vegetation is:
\begin{equation*}
\begin{split}\it{abs}_{soil}=  \sum_{ft=1}^{npft}{ \it{pft}_{wt,1,ft,1}( \mathit{dif}_{down,nz+1} (1 -  salb_{dif}) +\it{solar}_{dir}   dir_{tr,nz+1}  (1-  salb_{dir}))}\end{split}
\end{equation*}
to which is added the diffuse flux coming directly from the upper
canopy and hitting no understorey vegetation.
\begin{equation*}
\begin{split}\it{abs}_{soil}=  \it{abs}_{soil}+dif_{dn,2,1}  (1-  \sum_{ft=1}^{npft}{\it{pft}_{wt,1,ft,1}})  (1 -  salb_{dif})\end{split}
\end{equation*}
and the direct flux coming directly from the upper canopy and hitting
no understorey vegetation.
\begin{equation*}
\begin{split}\it{abs}_{soil}=  \it{abs}_{soil}+\it{solar}_{dir} dir_{tr,2,1}(1-  \sum_{ft=1}^{npft}{\it{pft}_{wt,1,ft,1}})  (1 -  salb_{dir})\end{split}
\end{equation*}
These changes to the radiation code are designed to be structurally
flexible, and the scheme may be collapsed down to only include on canopy
layer, functional type and pft for testing if necessary.

\captionof{table}{Parameters needed for radiation transfer model. }


\begin{savenotes}\sphinxattablestart
\centering
\begin{tabulary}{\linewidth}[t]{|T|T|T|T|}
\hline
\sphinxstylethead{\sphinxstyletheadfamily 
Parameter
Symbol
\unskip}\relax &\sphinxstylethead{\sphinxstyletheadfamily 
Parameter Name
\unskip}\relax &\sphinxstylethead{\sphinxstyletheadfamily 
Units
\unskip}\relax &\sphinxstylethead{\sphinxstyletheadfamily 
indexed by
\unskip}\relax \\
\hline
\(\chi\)
&
Leaf angle
distribution
parameter
&
none
&
\sphinxstyleemphasis{ft}
\\
\hline
\(\rho_l\)
&
Fraction of
light reflected
by leaf surface
&
none
&
\sphinxstyleemphasis{ft}
\\
\hline
\(\tau_l\)
&
Fraction of
light
transmitted by
leaf surface
&
none
&
\sphinxstyleemphasis{ft}
\\
\hline
\(alb_s\)
&
Fraction of
light reflected
by soil
&
none
&
direct vs
diffuse
\\
\hline
\end{tabulary}
\par
\sphinxattableend\end{savenotes}

\bigskip


\subsection{Photosynthesis}
\label{\detokenize{tech_note/DGVM/CLM50_Tech_Note_DGVM:photosynthesis}}

\subsubsection{Fundamental photosynthetic physiology theory}
\label{\detokenize{tech_note/DGVM/CLM50_Tech_Note_DGVM:fundamental-photosynthetic-physiology-theory}}
{[}11{]}\_In this section we describe the physiological basis of the
photosynthesis model before describing its application to the FATES
canopy structure. This description in this section is largely repeated
from the Oleson et al. CLM4.5 technical note but included here for
comparison with its implementation in FATES. Photosynthesis in C3
plants is based on the model of {\hyperref[\detokenize{tech_note/References/CLM50_Tech_Note_References:farquhar1980}]{\sphinxcrossref{\DUrole{std,std-ref}{Farquhar 1980}}}} as
modified by {\hyperref[\detokenize{tech_note/References/CLM50_Tech_Note_References:collatz1991}]{\sphinxcrossref{\DUrole{std,std-ref}{Collatz et al. 1991}}}}. Photosynthetic assimilation
in C4 plants is based on the model of {\hyperref[\detokenize{tech_note/References/CLM50_Tech_Note_References:collatz1991}]{\sphinxcrossref{\DUrole{std,std-ref}{Collatz et al. 1991}}}}.
In both models, leaf photosynthesis, \(\textrm{gpp}\)
(\(\mu\)mol CO\(_2\) m\(^{-2}\) s\(^{-1}\)) is
calculated as the minimum of three potentially limiting fluxes,
described below:
\begin{equation*}
\begin{split}\textrm{gpp} = \rm{min}(w_{j}, w_{c},w_{p}).\end{split}
\end{equation*}
The RuBP carboxylase (Rubisco) limited rate of carboxylation
\(w_{c}\) (\(\mu\)mol CO\(_{2}\) m\(^{-2}\)
s\(^{-1}\)) is determined as
\begin{equation*}
\begin{split}w_{c}=  \left\{ \begin{array}{ll}
\frac{V_{c,max}(c_{i} - \Gamma_*)}{ci+K_{c}(1+o_{i}/K_{o})} & \textrm{for $C_{3}$ plants}\\
&\\
V_{c,max}& \textrm{for $C_{4}$ plants}\\
\end{array} \right.
c_{i}-\Gamma_*\ge 0\end{split}
\end{equation*}
where \(c_{i}\) is the internal leaf CO\(_{2}\) partial
pressure (Pa) and \(o_i (0.209P_{atm}\)) is the O\(_{2}\)
partial pressure (Pa). \(K_{c}\) and \(K_{o}\) are the
Michaelis-Menten constants (Pa) for CO\(_{2}\) and
O\(_{2}\). These vary with vegetation temperature \(T_v\)
(\(^{o}\)C) according to an Arrhenious function described in
{\hyperref[\detokenize{tech_note/References/CLM50_Tech_Note_References:oleson2013}]{\sphinxcrossref{\DUrole{std,std-ref}{Oleson et al. 2013}}}}. \(V_{c,max}\) is the leaf layer
photosynthetic capacity (\(\mu\) mol CO\(_2\) m\(^{-2}\)
s\(^{-1}\)).

The maximum rate of carboxylation allowed by the capacity to regenerate
RuBP (i.e., the light-limited rate) \(w_{j}\) (\(\mu\)mol
CO\(_2\) m\(^{-2}\) s\(^{-1}\)) is
\begin{equation*}
\begin{split}w_j=  \left\{ \begin{array}{ll}
\frac{J(c_i - \Gamma_*)}{4ci+8\Gamma_*} & \textrm{for C$_3$ plants}\\
&\\
4.6\phi\alpha & \textrm{for C$_4$ plants}\\
\end{array} \right.
c_i-\Gamma_*\ge 0\end{split}
\end{equation*}
To find \(J\), the electron transport rate (\(\mu\) mol
CO\(_2\) m\(^{-2}\) s\(^{-1}\)), we solve the
following quadratic term and take its smaller root,
\begin{equation*}
\begin{split}\Theta_{psII}J^{2}-(I_{psII} +J_{max})J+I_{psII}J_{max} =0\end{split}
\end{equation*}
where \(J_{max}\) is the maximum potential rate of electron
transport (\(\mu\)mol m\(_{-2}\) s\(^{-1}\)),
\(I_{PSII}\) is the is the light utilized in electron transport by
photosystem II (\(\mu\)mol m\(_{-2}\) s\(^{-1}\)) and
\(\Theta_{PSII}\) is is curvature parameter. \(I_{PSII}\) is
determined as
\begin{equation*}
\begin{split}I_{PSII} =0.5 \Phi_{PSII}(4.6\phi)\end{split}
\end{equation*}
where \(\phi\) is the absorbed photosynthetically active radiation
(Wm:math:\sphinxtitleref{\textasciicircum{}\{-2\}}) for either sunlit or shaded leaves (\(abs_{sun}\)
and \(abs_{sha}\)). \(\phi\) is converted to photosynthetic
photon flux assuming 4.6 \(\mu\)mol photons per joule. Parameter
values are \(\Phi_{PSII}\) = 0.7 for C3 and \(\Phi_{PSII}\) =
0.85 for C4 plants.

The export limited rate of carboxylation for C3 plants and the PEP
carboxylase limited rate of carboxylation for C4 plants \(w_e\)
(also in \(\mu\)mol CO\(_2\) m\(^{-2}\)
s\(^{-1}\)) is
\begin{equation*}
\begin{split}w_e=  \left\{ \begin{array}{ll}
3 T_{p,0} & \textrm{for $C_3$ plants}\\
&\\
k_{p} \frac{c_i}{P_{atm}}& \textrm{for $C_4$ plants}.\\
\end{array} \right.\end{split}
\end{equation*}
\(T_{p}\) is the triose-phosphate limited rate of photosynthesis,
which is equal to \(0.167 V_{c,max0}\). \(k_{p}\) is the initial
slope of C4 CO\(_{2}\) response curve. The Michaelis-Menten
constants \(K_{c}\) and \(K_{o}\) are modeled as follows,
\begin{equation*}
\begin{split}K_{c} = K_{c,25}(a_{kc})^{\frac{T_v-25}{10}},\end{split}
\end{equation*}\begin{equation*}
\begin{split}K_{o} = K_{o,25}(a_{ko})^{\frac{T_v-25}{10}},\end{split}
\end{equation*}
where \(K_{c,25}\) = 30.0 and \(K_{o,25}\) = 30000.0 are values
(Pa) at 25 \(^{o}\)C, and \(a_{kc}\) = 2.1 and \(a_{ko}\)
=1.2 are the relative changes in \(K_{c,25}\) and \(K_{o,25}\)
respectively, for a 10\(^{o}\)C change in temperature. The
CO\(_{2}\) compensation point \(\Gamma_{*}\) (Pa) is
\begin{equation*}
\begin{split}\Gamma_* = \frac{1}{2} \frac{K_c}{K_o}0.21o_i\end{split}
\end{equation*}
where the term 0.21 represents the ratio of maximum rates of oxygenation
to carboxylation, which is virtually constant with temperature
{\hyperref[\detokenize{tech_note/References/CLM50_Tech_Note_References:farquhar1980}]{\sphinxcrossref{\DUrole{std,std-ref}{Farquhar, 1980}}}}.


\subsubsection{Resolution of the photosynthesis theory within the FATES canopy structure.}
\label{\detokenize{tech_note/DGVM/CLM50_Tech_Note_DGVM:resolution-of-the-photosynthesis-theory-within-the-fates-canopy-structure}}
The photosynthesis scheme is modified from the CLM4.5 model to give
estimates of photosynthesis, respiration and stomatal conductance for a
three dimenstional matrix indexed by canopy level (\(C_l\)), plant
functional type (\(ft\)) and leaf layer (\(z\)). We conduct the
photosynthesis calculations at each layer for both sunlit and shaded
leaves. Thus, the model also generates estimates of \(w_{c},w_{j}\)
and \(w_{e}\) indexed in the same three dimensional matrix. In this
implementation, some properties (stomatal conductance parameters,
top-of-canopy photosynthetic capacity) vary with plant functional type,
and some vary with both functional type and canopy depth (absorbed
photosynthetically active radiation, nitrogen-based variation in
photosynthetic properties). The remaining drivers of photosynthesis
(\(P_{atm}\), \(K_c\), \(o_i\), \(K_o\), temperature,
atmospheric CO\(_2\)) remain the same throughout the canopy. The
rate of gross photosynthesis (\(gpp_{Cl,ft,z}\))is the smoothed
minimum of the three potentially limiting processes (carboxylation,
electron transport, export limitation), but calculated independently for
each leaf layer:
\begin{equation*}
\begin{split}\textrm{gpp}_{Cl,ft,z} = \rm{min}(w_{c,Cl,ft,z},w_{j,Cl,ft,z},w_{e,Cl,ft,z}).\end{split}
\end{equation*}
For \(w_{c,Cl,ft,z},\), we use
\begin{equation*}
\begin{split}w_{c,Cl,ft,z}=  \left\{ \begin{array}{ll}
\frac{V_{c,max,Cl,ft,z}(c_{i,Cl,ft,z}- \Gamma_*)}{c_{i,Cl,ft,z}+K_c(1+o_i/K_o)} & \textrm{for $C_3$ plants}\\
&\\
V_{c,max,Cl,ft,z}& \textrm{for $C_4$ plants}\\
\end{array} \right.
c_{i,Cl,ft,z}-\Gamma_*\ge 0\end{split}
\end{equation*}
where \(V_{c,max}\) now varies with PFT, canopy depth and layer
(see below). Internal leaf \(CO_{2}\) (\(c_{i,Cl,ft,z})\) is
tracked seperately for each leaf layer. For the light limited rate
\(w_j\), we use
\begin{equation*}
\begin{split}w_j=  \left\{ \begin{array}{ll}
\frac{J(c_i - \Gamma_*)4.6\phi\alpha}{4ci+8\Gamma_*} & \textrm{for C$_3$ plants}\\
&\\
4.6\phi\alpha & \textrm{for C$_4$ plants}\\
\end{array} \right.\end{split}
\end{equation*}
where \(J\) is calculated as above but based on the absorbed
photosynthetically active radiation( \(\phi_{Cl,ft,z}\)) for either
sunlit or shaded leaves in Wm\(^{-2}\). Specifically,
\begin{equation*}
\begin{split}\phi_{Cl,ft,z}=  \left\{ \begin{array}{ll}
abs_{sun,Cl,ft,z}& \textrm{for sunlit leaves}\\
&\\
abs_{sha,Cl,ft,z}& \textrm{for shaded leaves}\\
\end{array} \right.\end{split}
\end{equation*}
The export limited rate of carboxylation for C3 plants and the PEP
carboxylase limited rate of carboxylation for C4 plants \(w_c\)
(also in \(\mu\)mol CO\(_2\) m\(^{-2}\)
s\(^{-1}\)) is calculated in a similar fashion,
\begin{equation*}
\begin{split}w_{e,Cl,ft,z}=  \left\{ \begin{array}{ll}
0.5V_{c,max,Cl,ft,z} & \textrm{for $C_3$ plants}\\
&\\
4000 V_{c,max,Cl,ft,z} \frac{c_{i,Cl,ft,z}}{P_{atm}}& \textrm{for $C_4$ plants}.\\
\end{array} \right.\end{split}
\end{equation*}

\subsubsection{Variation in plant physiology with canopy depth}
\label{\detokenize{tech_note/DGVM/CLM50_Tech_Note_DGVM:variation-in-plant-physiology-with-canopy-depth}}
Both \(V_{c,max}\) and \(J_{max}\) vary with vertical depth in
the canopy on account of the well-documented reduction in canopy
nitrogen through the leaf profile, see {\hyperref[\detokenize{tech_note/References/CLM50_Tech_Note_References:bonan2012}]{\sphinxcrossref{\DUrole{std,std-ref}{Bonan et al. 2012}}}} for
details). Thus, both \(V_{c,max}\) and \(J_{max}\) are indexed
by by \(C_l\), \(ft\) and \(z\) according to the nitrogen
decay coefficient \(K_n\) and the amount of vegetation area shading
each leaf layer \(V_{above}\),
\begin{equation*}
\begin{split}\begin{array}{ll}
V_{c,max,Cl,ft,z} & = V_{c,max0,ft} e^{-K_{n,ft}V_{above,Cl,ft,z}},\\
J_{max,Cl,ft,z} & = J_{max0,ft} e^{-K_{n,ft}V_{above,Cl,ft,z}},\\
\end{array}\end{split}
\end{equation*}
where \(V_{c,max,0}\) and \(J_{max,0}\) are the top-of-canopy
photosynthetic rates. \(V_{above}\) is the sum of exposed leaf area
index (\(\textrm{elai}_{Cl,ft,z}\)) and the exposed stem area index
(\(\textrm{esai}_{Cl,ft,z}\))( m\(^{2}\) m\(^{-2}\) ).
Namely,
\begin{equation*}
\begin{split}V_{Cl,ft,z} = \textrm{elai}_{Cl,ft,z} + \textrm{esai}_{Cl,ft,z}.\end{split}
\end{equation*}
The vegetation index shading a particular leaf layer in the top canopy
layer is equal to
\begin{equation*}
\begin{split}\begin{array}{ll}
V_{above,Cl,ft,z}= \sum_{1}^{z} V_{Cl,ft,z} & \textrm{for $Cl= 1$. }
\end{array}\end{split}
\end{equation*}
For lower canopy layers, the weighted average vegetation index of the
canopy layer above (\(V_{canopy}\)) is added to this within-canopy
shading. Thus,
\begin{equation*}
\begin{split}\begin{array}{ll}
V_{above,Cl,ft,z}=  \sum_{1}^{z}  V_{Cl,ft,z} + V_{canopy,Cl-1} & \textrm{for $Cl >1$, }\\
\end{array}\end{split}
\end{equation*}
where \(V_{canopy}\) is calculated as
\begin{equation*}
\begin{split}V_{canopy,Cl} =  \sum_{ft=1}^{\emph{npft}} {\sum_{z=1}^{nz(ft)} (V_{Cl,ft,z} \cdot  \it{pft}_{wt,Cl,ft,1}).}\end{split}
\end{equation*}
\(K_{n}\) is the coefficient of nitrogen decay with canopy depth.
The value of this parameter is taken from the work of
{\hyperref[\detokenize{tech_note/References/CLM50_Tech_Note_References:lloyd2010}]{\sphinxcrossref{\DUrole{std,std-ref}{Lloyd et al. 2010}}}} who determined, from 204 vertical profiles
of leaf traits, that the decay rate of N through canopies of tropical
rainforests was a function of the \(V_{cmax}\) at the top of the
canopy. They obtain the following term to predict \(K_{n}\),
\begin{equation*}
\begin{split}K_{n,ft} = e^{0.00963 V_{c,max0,ft} - 2.43},\end{split}
\end{equation*}
where \(V_{cmax}\) is again in \(\mu\)mol CO\(_2\)
m\(^{-2}\) s\(^{-1}\).


\subsubsection{Water Stress on gas exchange}
\label{\detokenize{tech_note/DGVM/CLM50_Tech_Note_DGVM:water-stress-on-gas-exchange}}
%
\begin{footnote}[12]\sphinxAtStartFootnote
This description relates to algorithms in the ED\_btran subroutine
%
\end{footnote} The top of canopy leaf photosynthetic capacity, \(V_{c,max0}\), is
also adjusted for the availability of water to plants as
\begin{equation*}
\begin{split}V_{c,max0,25} = V_{c,max0,25}  \beta_{sw},\end{split}
\end{equation*}
where the adjusting factor \(\beta_{sw}\) ranges from one when the
soil is wet to zero when the soil is dry. It depends on the soil water
potential of each soil layer, the root distribution of the plant
functional type, and a plant-dependent response to soil water stress,
\begin{equation*}
\begin{split}\beta_{sw} = \sum_{j=1}^{nj}w_{j}r_{j},\end{split}
\end{equation*}
where \(w_{j}\) is a plant wilting factor for layer \(j\) and
\(r_{j}\) is the fraction of roots in layer \(j\).The plant
wilting factor \(w_{j}\) is
\begin{equation*}
\begin{split}w_{j}=  \left\{ \begin{array}{ll}
\frac{\psi_c-\psi_{j}}{\psi_c - \psi_o} (\frac{\theta_{sat,j} - \theta_{ice,j}}{\theta_{sat,j}})& \textrm{for $T_i >$-2C}\\
&\\
0 & \textrm{for $T_{j} \ge$-2C}\\
\end{array} \right.\end{split}
\end{equation*}
where \(\psi_{i}\) is the soil water matric potential (mm) and
\(\psi_{c}\) and \(\psi_{o}\) are the soil water potential (mm)
when stomata are fully closed or fully open, respectively. The term in
brackets scales \(w_{i}\) the ratio of the effective porosity (after
accounting for the ice fraction) relative to the total porosity.
\(w_{i}\) = 0 when the temperature of the soil layer (\(T_{i}\)
) is below some threshold (-2:math:\sphinxtitleref{\textasciicircum{}\{o\}}C) or when there is no liquid
water in the soil layer (\(\theta_{liq,i} \le 0\)). For more details
on the calculation of soil matric potential, see the CLM4.5 technical
note.


\paragraph{Variation of water stress and water uptake within tiles}
\label{\detokenize{tech_note/DGVM/CLM50_Tech_Note_DGVM:variation-of-water-stress-and-water-uptake-within-tiles}}
The remaining drivers of the photosynthesis model remain constant
(atmospheric CO\(_2\) and O\(^2\) and canopy temperature)
throughout the canopy, except for the water stress index
\(\beta_{sw}\). \(\beta_{sw}\) must be indexed by \(ft\),
because plants of differing functional types have the capacity to have
varying root depth, and thus access different soil moisture profile and
experience differing stress functions. Thus, the water stress function
applied to gas exchange calculation is now calculated as
\begin{equation*}
\begin{split}\beta_{sw,ft} = \sum_{j=1}^{nj}w_{j,ft} r_{j,ft},\end{split}
\end{equation*}
where \(w_{j}\) is the water stress at each soil layer \(j\)
and \(r_{j,ft}\) is the root fraction of each PFT’s root mass in
layer \(j\). Note that this alteration of the \(\beta_{sw}\)
parameter also necessitates recalculation of the vertical water
extraction profiles. In the original model, the fraction of extraction
from each layer (\(r_{e,j,patch}\)) is the product of a single root
distribution, because each patch only has one plant functional type. In
FATES, we need to calculate a new weighted patch effective rooting
depth profile \(r_{e,j,patch}\) as the weighted average of the
functional-type level stress functions and their relative contributions
to canopy conductance. Thus for each layer \(j\), the extraction
fraction is summed over all PFTs as
\begin{equation*}
\begin{split}r_{e,j,patch} =  \sum_{ft=1}^{ft=npft} \frac{w_{j,ft}}{\sum_{j=1}^{=nj} w_{j,ft} }\frac{G_{s,ft}}{G_{s,canopy}},\end{split}
\end{equation*}
where \(nj\) is the number of soil layers, \(G_{s,canopy}\)is
the total canopy (see section 9 for details) and \(G_{s,ft}\) is the
canopy conductance for plant functional type \(ft\),
\begin{equation*}
\begin{split}G_{s,ft}= \sum_{1}w_{ncoh,ft} {gs_{can,coh} n_{coh} }.\end{split}
\end{equation*}

\subsubsection{Aggregation of assimilated carbon into cohorts}
\label{\detokenize{tech_note/DGVM/CLM50_Tech_Note_DGVM:aggregation-of-assimilated-carbon-into-cohorts}}
The derivation of photosynthetic rates per leaf layer, as above, give us
the estimated rate of assimilation for a unit area of leaf at a given
point in the canopy in \(\mu\)mol CO\(_2\) m\(^{-2}\)
s\(_{-1}\). To allow the integration of these rates into fluxes
per individual tree, or cohort of trees (gCO:math:\sphinxtitleref{\_2}
tree\(^{-1}\) s\(^{-1}\)), they must be multiplied by the
amount of leaf area placed in each layer by each cohort. Each cohort is
described by a single functional type, \(ft\) and canopy layer
\(C_l\) flag, so the problem is constrained to integrating these
fluxes through the vertical profile (\(z\)).

We fist make a weighted average of photosynthesis rates from sun
(\(\textrm{gpp}_{sun}\), \(\mu\)mol CO\(_2\)
m\(^{-2}\) s\(^{-1}\)) and shade leaves (
\(\textrm{gpp}_{shade}\), \(\mu\)mol CO\(_2\)
m\(^{-2}\) s\(^{-1}\)) as
\begin{equation*}
\begin{split}\textrm{gpp}_{Cl,ft,z} =\textrm{gpp}_{sun,Cl,ft,z} f_{sun,Cl,ft,z}+ \textrm{gpp}_{sha,Cl,ft,z}(1-f_{sun,Cl,ft,z}).\end{split}
\end{equation*}
The assimilation per leaf layer is then accumulated across all the leaf
layers in a given cohort (\sphinxstyleemphasis{coh}) to give the cohort-specific gross
primary productivity (\(\mathit{GPP}_{coh}\)),
\begin{equation*}
\begin{split}\textit{GPP}_{coh} = 12\times 10^{-9}\sum_{z=1}^{nz(coh)}gpp_{Cl,ft,z} A_{crown,coh} \textrm{elai}_{Cl,ft,z}\end{split}
\end{equation*}
The \(\textrm{elai}_{l,Cl,ft,z}\) is the exposed leaf area which is
present in each leaf layer in m\(^{2}\) m\(^{-2}\). (For all
the leaf layers that are completely occupied by a cohort, this is the
same as the leaf fraction of \(\delta_{vai}\)). The fluxes are
converted from \(\mu\)mol into mol and then multiplied by 12 (the
molecular weight of carbon) to give units for GPP\(_{coh}\) of KgC
cohort\(^{-1}\) s\(^{-1}\). These are integrated for each
timestep to give KgC cohort\(^{-1}\) day\(^{-1}\)

\captionof{table}{Parameters needed for photosynthesis model.}


\begin{savenotes}\sphinxattablestart
\centering
\begin{tabulary}{\linewidth}[t]{|T|T|T|T|}
\hline
\sphinxstylethead{\sphinxstyletheadfamily 
Parameter
Symbol
\unskip}\relax &\sphinxstylethead{\sphinxstyletheadfamily 
Parameter Name
\unskip}\relax &\sphinxstylethead{\sphinxstyletheadfamily 
Units
\unskip}\relax &\sphinxstylethead{\sphinxstyletheadfamily 
indexed by
\unskip}\relax \\
\hline
\(V_{c,max
0}\)
&
Maximum
carboxylation
capacity
&
\(\mu\) mol
CO \(_2\)
m \(^{-2}\)
s \(^{-1}\)
&
\sphinxstyleemphasis{ft}
\\
\hline
\(r_b\)
&
Base Rate of
Respiration
&
gC
gN\(^{-1
} s^{-1}\))
&\\
\hline
\(q_{10}\)
&
Temp. Response
of stem and
root
respiration
&&\\
\hline
\(R_{cn,le
af,ft}\)
&
CN ratio of
leaf matter
&
gC/gN
&
\sphinxstyleemphasis{ft}
\\
\hline
\(R_{cn,ro
ot,ft}\)
&
CN ratio of
root matter
&
gC/gN
&
\sphinxstyleemphasis{ft}
\\
\hline
\(f_{gr}\)
&
Growth
Respiration
Fraction
&
none
&\\
\hline
\(\psi_c\)
&
Water content
when stomata
close
&
Pa
&
\sphinxstyleemphasis{ft}
\\
\hline
\(\psi_o\)
&
Water content
above which
stomata are
open
&
Pa
&
\sphinxstyleemphasis{ft}
\\
\hline
\end{tabulary}
\par
\sphinxattableend\end{savenotes}

\bigskip


\subsection{Plant respiration}
\label{\detokenize{tech_note/DGVM/CLM50_Tech_Note_DGVM:plant-respiration}}
{[}13{]}\_Plant respiration per individual \(R_{plant,coh}\) (KgC individual
\(^{-1}\) s\(^{-1}\)) is the sum of two terms, growth and
maintenance respiration \(R_{g,coh}\) and \(R_{m,coh}\)
\begin{equation*}
\begin{split}R_{plant} = R_{g,coh}+ R_{m,coh}\end{split}
\end{equation*}
Maintenance respiration is the sum of the respiration terms from four
different plant tissues, leaf, \(R_{m,leaf,coh}\), fine root
\(R_{m,froot,coh}\), coarse root \(R_{m,croot,coh}\)and stem
\(R_{m,stem,coh}\), all also in (KgC individual \(^{-1}\)
s\(^{-1}\)) .
\begin{equation*}
\begin{split}R_{m,coh} = R_{m,leaf,coh}+ R_{m,froot,coh}+R_{m,croot,coh}+R_{m,stem,coh}\end{split}
\end{equation*}
To calculate canopy leaf respiration, which varies through we canopy, we
first determine the top-of-canopy leaf respiration rate
(\(r_{m,leaf,ft,0}\), gC s\(^{-1}\) m\(^{-2}\)) is
calculated from a base rate of respiration per unit leaf nitrogen
derived from {\hyperref[\detokenize{tech_note/References/CLM50_Tech_Note_References:ryan1991}]{\sphinxcrossref{\DUrole{std,std-ref}{Ryan et al. 1991}}}}. The base rate for leaf
respiration (\(r_{b}\)) is 2.525 gC/gN s\(^{-1}\),
\begin{equation*}
\begin{split}r_{m,leaf,ft,0} = r_{b} N_{a,ft}(1.5^{(25-20)/10})\end{split}
\end{equation*}
where \(r_b\) is the base rate of metabolism (2.525 x
10\(^6\) gC/gN s\(^{-1}\). This base rate is adjusted
assuming a Q\(_{10}\) of 1.5 to scale from the baseline of 20C to
the CLM default base rate temperature of 25C. For use in the
calculations of net photosynthesis and stomatal conductance, leaf
respiration is converted from gC s\(^{-1}\) m\(^{-2}\), into
\(\mu\)mol CO\(_2\) m\(^{-2}\) s\(^{-1}\)
(\(/12\cdot 10^{-6}\)).

This top-of-canopy flux is scaled to account for variation in
\(N_a\) through the vertical canopy, in the same manner as the
\(V_{c,max}\) values are scaled using \(V_{above}\).
\begin{equation*}
\begin{split}r_{leaf,Cl,ft,z}  = r_{m,leaf,ft,0} e^{-K_{n,ft}V_{above,Cl,ft,z}}\beta_{ft}f(t)\end{split}
\end{equation*}
Leaf respiration is also adjusted such that it is reduced by drought
stress, \(\beta_{ft}\), and canopy temperature, \(f(t_{veg})\).
For details of the temperature functions affecting leaf respiration see
the CLM4 technical note, Section 8, Equations 8.13 and 8.14. The
adjusted leaf level fluxes are scaled to individual-level (gC individual
\(^{-1}\) s\(^{-1}\)) in the same fashion as the
\(\rm{GPP}_{coh}\) calculations
\begin{equation*}
\begin{split}\rm{R}_{m,leaf,coh} = 12\times 10^{-9}\sum_{z=1}^{nz(coh)}r_{leaf,Cl,ft,z} A_{crown} \textrm{elai}_{Cl,ft,z}\end{split}
\end{equation*}
The stem and the coarse-root respiration terms are derived using the
same base rate of respiration per unit of tissue Nitrogen.
\begin{equation*}
\begin{split}R_{m,croot,coh} =  10^{-3}r_b t_c \beta_{ft} N_{\rm{livecroot,coh}}\end{split}
\end{equation*}\begin{equation*}
\begin{split}R_{m,stem,coh} =   10^{-3}r_b t_c \beta_{ft} N_{\rm{stem,coh}}\end{split}
\end{equation*}
Here, \(t_c\) is a temperature relationship based on a
\(q_{10}\) value of 1.5, where \(t_v\) is the vegetation
temperature. We use a base rate of 20 here as, again, this is the
baseline temperature used by {\hyperref[\detokenize{tech_note/References/CLM50_Tech_Note_References:ryan1991}]{\sphinxcrossref{\DUrole{std,std-ref}{Ryan et al. 1991}}}}. The
10\(^{-3}\) converts from gC invididual\(^{-1}\)
s\(^{-1}\) to KgC invididual\(^{-1}\) s\(^{-1}\)
\begin{equation*}
\begin{split}t_c=q_{10}^{(t_{v} - 20)/10}\end{split}
\end{equation*}
The tissue N contents for live sapwood are derived from the leaf CN
ratios, and for fine roots from the root CN ratio as:
\begin{equation*}
\begin{split}N_{\rm{stem,coh}}  = \frac{B_{\rm{sapwood,coh}}}{ R_{cn,leaf,ft}}\end{split}
\end{equation*}
and
\begin{equation*}
\begin{split}N_{\rm{livecroot,coh}}  = \frac{ B_{\rm{root,coh}}w_{frac,ft}}{R_{cn,root,ft}}\end{split}
\end{equation*}
where \(B_{\rm{sapwood,coh}}\) and \(B_{\rm{root,coh}}\) are
the biomass pools of sapwood and live root biomass respectively (KgC
individual) and \(w_{frac,ft}\) is the fraction of coarse root
tissue in the root pool (0.5 for woody plants, 0.0 for grasses and
crops). We assume here that stem CN ratio is the same as the leaf C:N
ratio, for simplicity. The final maintenance respiration term is derived
from the fine root respiration, which accounts for gradients of
temperature in the soil profile and thus calculated for each soil layer
\(j\) as follows:
\begin{equation*}
\begin{split}R_{m,froot,j } = \frac{(1 - w_{frac,ft})B_{\rm{root,coh}}b_r\beta_{ft}}{10^3R_{cn,leaf,ft}}   \sum_{j=1}^{nj}t_{c,soi,j} r_{i,ft,j}\end{split}
\end{equation*}
\(t_{c,soi}\) is a function of soil temperature in layer \(j\)
that has the same form as that for stem respiration, but uses vertically
resolved soil temperature instead of canopy temperature. In the CLM4.5,
only coarse and not fine root respriation varies as a function of soil
depth, and we maintain this assumption here, although it may be altered
in later versions. The growth respiration, \(R_{g,coh}\) is a fixed
fraction \(f_{gr}\) of the carbon remaining after maintenance
respiration has occurred.
\begin{equation*}
\begin{split}R_{g,coh}=\textrm{max}(0,GPP_{g,coh} - \it R\rm_{m,coh})f_{gr}\end{split}
\end{equation*}
\captionof{table}{Parameters needed for plant respiration model.  }


\begin{savenotes}\sphinxattablestart
\centering
\begin{tabular}[t]{|*{4}{\X{1}{4}|}}
\hline
\sphinxstylethead{\sphinxstyletheadfamily 
Parameter
Symbol
\unskip}\relax &\sphinxstylethead{\sphinxstyletheadfamily 
Parameter Name
\unskip}\relax &\sphinxstylethead{\sphinxstyletheadfamily 
Units
\unskip}\relax &\sphinxstylethead{\sphinxstyletheadfamily 
indexed by
\unskip}\relax \\
\hline
\(-K_{n,ft
}\)
&
Rate of
reduction of N
through the
canopy
&
none
&\begin{itemize}
\item {} 
\end{itemize}
\\
\hline
\(r_b\)
&
Base Rate of
Respiration
&
gC
gN\(^{-1
} s^{-1}\))
&\\
\hline
\(q_{10}\)
&
Temp. Response
of stem and
root
respiration
&&\\
\hline
\(R_{cn,le
af,ft}\)
&
CN ratio of
leaf matter
&
gC/gN
&
\sphinxstyleemphasis{ft}
\\
\hline
\(R_{cn,ro
ot,ft}\)
&
CN ratio of
root matter
&
gC/gN
&
\sphinxstyleemphasis{ft}
\\
\hline
\(f_{gr}\)
&
Growth
Respiration
Fraction
&
none
&\\
\hline
\end{tabular}
\par
\sphinxattableend\end{savenotes}

\bigskip


\subsection{Stomatal Conductance}
\label{\detokenize{tech_note/DGVM/CLM50_Tech_Note_DGVM:stomatal-conductance}}

\subsubsection{Fundamental stomatal conductance theory}
\label{\detokenize{tech_note/DGVM/CLM50_Tech_Note_DGVM:fundamental-stomatal-conductance-theory}}
{[}14{]}\_Stomatal conductance is unchanged in concept from the CLM4.5 approach.
Leaf stomatal resistance is calculated from the Ball-Berry conductance
model as described by {\hyperref[\detokenize{tech_note/References/CLM50_Tech_Note_References:collatz1991}]{\sphinxcrossref{\DUrole{std,std-ref}{Collatz et al. 1991}}}} and implemented in
a global climate model by {\hyperref[\detokenize{tech_note/References/CLM50_Tech_Note_References:sellers1996}]{\sphinxcrossref{\DUrole{std,std-ref}{Sellers et al. 1996}}}}. The model
relates stomatal conductance (i.e., the inverse of resistance) to net
leaf photosynthesis, scaled by the relative humidity at the leaf surface
and the CO\(_2\) concentration at the leaf surface. The primary
difference between the CLM implementation and that used by
{\hyperref[\detokenize{tech_note/References/CLM50_Tech_Note_References:collatz1991}]{\sphinxcrossref{\DUrole{std,std-ref}{Collatz et al. 1991}}}} and {\hyperref[\detokenize{tech_note/References/CLM50_Tech_Note_References:sellers1996}]{\sphinxcrossref{\DUrole{std,std-ref}{Sellers et al. 1996}}}} is
that they used net photosynthesis (i.e., leaf photosynthesis minus leaf
respiration) instead of gross photosynthesis. As implemented here,
stomatal conductance equals the minimum conductance (\(b\)) when
gross photosynthesis (\(A\)) is zero. Leaf stomatal resistance is
\begin{equation*}
\begin{split}\frac{1}{r_{s}} = m_{ft} \frac{A}{c_s}\frac{e_s}{e_i}P_{atm}+b_{ft} \beta_{sw}\end{split}
\end{equation*}
where \(r_{s}\) is leaf stomatal resistance (s m\(^2\)
\(\mu\)mol\(^{-1}\)), \(b_{ft}\) is a plant functional
type dependent parameter equivalent to \(g_{0}\) in the Ball-Berry
model literature. This parameter is also scaled by the water stress
index \(\beta_{sw}\). Similarly, \(m_{ft}\) is the slope of the
relationship between the assimilation, \(c_s\) and humidty dependant
term and the stomatal conductance, and so is equivalent to the
\(g_{1}\) term in the stomatal literature. \(A\) is leaf
photosynthesis (\(\mu\)mol CO\(_2\) m\(^{-2}\)
s\(^{-1}\)), \(c_s\) is the CO\(_2\) partial pressure at
the leaf surface (Pa), \(e_s\) is the vapor pressure at the leaf
surface (Pa), \(e_i\) is the saturation vapor pressure (Pa) inside
the leaf at the vegetation temperature conductance (\(\mu\)mol
m\(^{-2}\) s\(^{-1}\)) when \(A\) = 0 . Typical values
are \(m_{ft}\) = 9 for C\(_3\) plants and \(m_{ft}\) = 4
for C\(_4\) plants (
{\hyperref[\detokenize{tech_note/References/CLM50_Tech_Note_References:collatz1991}]{\sphinxcrossref{\DUrole{std,std-ref}{Collatz et al. 1991}}}}, {\hyperref[\detokenize{tech_note/References/CLM50_Tech_Note_References:collatz1992}]{\sphinxcrossref{\DUrole{std,std-ref}{Collatz, 1992}}}}, \DUrole{xref,std,std-ref}{Sellers et al 1996\textless{}sellers1996\textgreater{})}.
{\hyperref[\detokenize{tech_note/References/CLM50_Tech_Note_References:sellers1996}]{\sphinxcrossref{\DUrole{std,std-ref}{Sellers et al. 1996}}}} used \(b\) = 10000 for C\(_3\)
plants and \(b\) = 40000 for C\(_4\) plants. Here, \(b\)
was chosen to give a maximum stomatal resistance of 20000 s
m\(^{-1}\). These terms are nevertheless plant strategy dependent,
and have been found to vary widely with plant type
{\hyperref[\detokenize{tech_note/References/CLM50_Tech_Note_References:medlyn2011}]{\sphinxcrossref{\DUrole{std,std-ref}{Medlyn et al. 2001}}}}.

Resistance is converted from units of s m\(^2 \mu\)
mol\(^{-1}\) to s m\(^{-1}\) as: 1 s m\(^{-1}\) =
\(1\times 10^{-9}\)R\(_{\rm{gas}} \theta_{\rm{atm}}P_{\rm{atm}}\)
(\(\mu\)mol\(^{-1}\) m\(^{2}\) s), where
R\(_{gas}\) is the universal gas constant (J K\(^{-1}\)
kmol\(^{-1}\)) and \(\theta_{atm}\) is the atmospheric
potential temperature (K).


\subsubsection{Resolution of stomatal conductance theory in the FATES canopy structure}
\label{\detokenize{tech_note/DGVM/CLM50_Tech_Note_DGVM:resolution-of-stomatal-conductance-theory-in-the-fates-canopy-structure}}
The stomatal conductance is calculated, as with photosynthesis, for each
canopy, PFT and leaf layer. The CLM code requires a single canopy
conductance estimate to be generated from the multi-layer multi-PFT
array. In previous iterations of the CLM, sun and shade-leaf specific
values have been reported and then averaged by their respective leaf
areas. In this version, the total canopy condutance
\(G_{s,canopy}\), is calculated as the sum of the cohort-level
conductance values.
\begin{equation*}
\begin{split}G_{s,canopy} =  \sum{ \frac{gs_{can,coh} n_{coh} }{A_{patch}}}\end{split}
\end{equation*}
Cohort conductance is the sum of the inverse of the leaf resistances at
each canopy layer (\(r_{s,z}\) ) multipled by the area of each
cohort.
\begin{equation*}
\begin{split}gs_{can,coh} =\sum_{z=1}^{z=nv,coh}{\frac{ A_{crown,coh}}{r_{s,cl,ft,z}+r_{b}}}\end{split}
\end{equation*}
\captionof{table}{Parameters needed for stomatal conductance model.  }


\begin{savenotes}\sphinxattablestart
\centering
\begin{tabulary}{\linewidth}[t]{|T|T|T|T|}
\hline
\sphinxstylethead{\sphinxstyletheadfamily 
Parameter Symbol
\unskip}\relax &\sphinxstylethead{\sphinxstyletheadfamily 
Parameter Name
\unskip}\relax &\sphinxstylethead{\sphinxstyletheadfamily 
Units
\unskip}\relax &\sphinxstylethead{\sphinxstyletheadfamily 
indexed by
\unskip}\relax \\
\hline
\(b_{ft}\)
&
Slope of Ball-Berry term
&
none
&
\sphinxstyleemphasis{ft}
\\
\hline
\(m_{ft}\)
&
Slope of Ball-Berry term
&
none
&
\sphinxstyleemphasis{ft}
\\
\hline
\end{tabulary}
\par
\sphinxattableend\end{savenotes}


\subsection{Allocation and Growth}
\label{\detokenize{tech_note/DGVM/CLM50_Tech_Note_DGVM:allocation-and-growth}}
{[}15{]}\_Total assimilation carbon enters the ED model each day as a
cohort-specific Net Primary Productivity \(\mathit{NPP}_{coh}\),
which is calculated as
\begin{equation*}
\begin{split}\mathit{NPP}_{coh} = \mathit{GPP}_{coh} - R_{plant,coh}\end{split}
\end{equation*}
This flux of carbon is allocated between the demands of tissue turnover,
of carbohydrate storage and of growth (increase in size of one or many
plant organs). Priority is explicitly given to maintenance respiration,
followed by tissue maintenance and storage, then allocation to live
biomass and then to the expansion of structural and live biomass pools.
All fluxes here are first converted into in KgC
individual\(^{-1}\) year\(^{-1}\) and ultimately integrated
using a timesteps of 1/365 years for each day.


\subsubsection{Tissue maintenance demand}
\label{\detokenize{tech_note/DGVM/CLM50_Tech_Note_DGVM:tissue-maintenance-demand}}
We calculate a ‘tissue maintenance’ flux. The magnitude of this flux is
such that the quantity of biomass in each pool will remain constant,
given background turnover rates. For roots, this maintenenace demand is
simply
\begin{equation*}
\begin{split}r_{md,coh}  = b_{root}\cdot\alpha_{root,ft}\end{split}
\end{equation*}
Where \(\alpha_{root,ft}\) is the root turnover rate in y\textasciicircum{}-1. Given
that, for deciduous trees, loss of leaves is assumed to happen only one
per growing season, the algorithm is dependent on phenological habit
(whether or not this PFT is evergreen), thus
\begin{equation*}
\begin{split}l_{md,coh} = \left\{ \begin{array}{ll}
 b_{leaf}\cdot\alpha_{leaf,ft}&\textrm{for } P_{evergreen}= 1\\
&\\
0&\textrm{for }  P_{evergreen}= 0\\
\end{array} \right.\end{split}
\end{equation*}
Leaf litter resulting from deciduous senescence is handled in the
phenology section. The total quantity of maintenance demand
(\(t_{md,coh}\). KgC individual y\(^{-1}\)) is therefore
\begin{equation*}
\begin{split}t_{md,coh}  = l_{md,coh} + r_{md,coh}\end{split}
\end{equation*}

\subsubsection{Allocation to storage and turnover}
\label{\detokenize{tech_note/DGVM/CLM50_Tech_Note_DGVM:allocation-to-storage-and-turnover}}
The model must now determine whether the NPP input is sufficient to meet
the maintenance demand and keep tissue levels constant. To determine
this, we introduce the idea of ‘carbon balance’ \(C_{bal,coh}\) (KgC
individual\(^{-1}\)) where
\begin{equation*}
\begin{split}C_{bal,coh}= \mathit{NPP}_{coh} - t_{md,coh}\cdot f_{md,min,ft}\end{split}
\end{equation*}
where \(f_{md,min,ft}\) is the minimum fraction of the maintenance
demand that the plant must meet each timestep, which is indexed by \sphinxstyleemphasis{ft}
and represents a life-history-strategy decision concerning whether
leaves should remain on in the case of low carbon uptake (a risky
strategy) or not be replaced (a conservative strategy). Subsequently, we
determine a flux to the storage pool, where the flux into the pool, as a
fraction of \(C_{bal,coh}\), is proportional to the discrepancy
between the target pool size and the actual pool size \(f_{tstore}\)
where
\begin{equation*}
\begin{split}f_{tstore} =  \mathrm{max}\left(0,\frac{b_{store}}{b_{leaf}\cdot S_{cushion}}\right)\end{split}
\end{equation*}
The allocation to storage is a fourth power function of
\(f_{tstore}\) to mimic the qualitative behaviour found for carbon
allocation in arabidopsis by {\hyperref[\detokenize{tech_note/References/CLM50_Tech_Note_References:smith2007}]{\sphinxcrossref{\DUrole{std,std-ref}{Smith et al. 2007}}}}.
\begin{equation*}
\begin{split}\frac{\delta b_{store}}{\delta t} = \left\{ \begin{array}{ll}
C_{bal,coh} \cdot e^{-f_{tstore}^{4}} &\textrm{for }C_{bal,coh}>0\\
&\\
C_{bal,coh} &\textrm{for }C_{bal,coh}\leq0\\
\end{array} \right.\end{split}
\end{equation*}
If the carbon remaining after the storage and minimum turnover fluxes
have been met, the next priority is the remaining flux to leaves
\(t_{md}\cdot(1-f_{md,min})\). If the quantity of carbon left
\((C_{bal,coh}-\frac{\delta b_{store}}{\delta t})\) is insufficient
to supply this amount of carbon, then the store of alive carbon is
depleted (to represent those leaves that have fallen off and not been
replaced)
\begin{equation*}
\begin{split}\frac{\delta b_{alive}}{\delta t} = \left\{ \begin{array}{ll}
0 &\textrm{ for } (C_{bal,coh}-\frac{\delta b_{store}}{\delta t}) > t_{md}\cdot(1-f_{md,min})\\
&\\
t_{md}\cdot(1-f_{md,min}) - \left(C_{bal,coh}-\frac{\delta b_{store}}{\delta t}\right)&\textrm{ for } (C_{bal,coh}-\frac{\delta b_{store}}{\delta t}) \leq t_{md}\cdot(1-f_{md,min})\\
\end{array} \right.\end{split}
\end{equation*}
correspondingly, the carbon left over for growth (\(C_{growth}\):
(KgC individual\(^{-1}\) year\(^{-1}\)) is therefore
\begin{equation*}
\begin{split}C_{growth} = \left\{ \begin{array}{ll}
C_{bal,coh}-\frac{\delta b_{store}}{\delta t} &\textrm{ for } (C_{bal,coh}-\frac{\delta b_{store}}{\delta t}) > 0\\
&\\
0&\textrm{ for } (C_{bal,coh}-\frac{\delta b_{store}}{\delta t}) \leq 0\\
\end{array} \right.\end{split}
\end{equation*}
to allocate the remaining carbon (if there is any), we first ascertain
whether the live biomass pool is at its target, or whether is has been
depleted by previous low carbon timesteps. Thus
\begin{equation*}
\begin{split}\begin{array}{lll}
b_{alive,target}&= b_{leaf,target}  (1+ f_{frla}+f_{swh}h_{coh}) &\textrm{for } S_{phen,coh} = 2\\
b_{alive,target}&= b_{leaf,target}  ( f_{frla}+f_{swh}h_{coh})&\textrm{for } S_{phen,coh} = 1\\
\end{array}\end{split}
\end{equation*}
where the target leaf biomass \(b_{leaf.target}\) ((Kg C
individual\(^{-1}\))) is the allometric relationship between dbh
and leaf biomass, ameliorated by the leaf trimming fraction (see
‘control of leaf area’ below)
\begin{equation*}
\begin{split}b_{leaf.target} = c_{leaf}\cdot dbh_{coh}^{e_{leaf,dbh}} \rho_{ft} ^{e_{leaf,dens}}\cdot C_{trim,coh}\end{split}
\end{equation*}
\(\rho_{ft}\) is the wood density, in g cm\(^{3}\).


\subsubsection{Allocation to Seeds}
\label{\detokenize{tech_note/DGVM/CLM50_Tech_Note_DGVM:allocation-to-seeds}}
The fraction remaining for growth (expansion of live and structural
tissues) \(f_{growth}\) is 1 minus that allocated to seeds.
\begin{equation*}
\begin{split}f_{growth,coh} = 1 - f_{seed,coh}\end{split}
\end{equation*}
Allocation to seeds only occurs if the alive biomass is not below its
target, and then is a predefined fixed fraction of the carbon remaining
for growth. Allocation to clonal reproduction (primarily for grasses)
occurs when \(\textrm{max}_{dbh}\) is achieved.
\begin{equation*}
\begin{split}f_{seed,coh} = \left\{ \begin{array}{ll}
R_{frac,ft}&\textrm{ for } \textrm{max}_{dbh} < dbh_{coh} \\
&\\
\left( R_{frac,ft}+C_{frac,ft} \right) &\textrm{ for } \textrm{max}_{dbh} \geq dbh_{coh} \\
\end{array} \right.\end{split}
\end{equation*}
the total amount allocated to seed production (\(p_{seed,coh}\) in
KgC individual \(^{-1}\) y\(^{-1}\)) is thus
\begin{equation*}
\begin{split}p_{seed,coh} = C_{growth}\cdot f_{seed,coh}\end{split}
\end{equation*}

\subsubsection{Allocation to growing pools}
\label{\detokenize{tech_note/DGVM/CLM50_Tech_Note_DGVM:allocation-to-growing-pools}}
%
\begin{footnote}[16]\sphinxAtStartFootnote
This description relates to algorithms in the ED\_GrowthFunctions
subroutine
%
\end{footnote} The carbon is then partitioned into carbon available to grow the
\(b_{alive}\) and \(b_{struc}\) pools. A fraction \(v_{a}\)
is available to live biomass pools, and a fraction \(v_{s}\) is
available to structural pools.
\begin{equation*}
\begin{split}\frac{\delta b_{alive}}{\delta t} = C_{growth}\cdot  f_{growth} v_{a}\end{split}
\end{equation*}\begin{equation*}
\begin{split}\frac{\delta b_{struc}}{\delta t} = C_{growth}\cdot  f_{growth} v_{s}\end{split}
\end{equation*}
If the alive biomass is lower than its ideal target, all of the
available carbon is directed into that pool. Thus:
\begin{equation*}
\begin{split}v_{a}= \left\{ \begin{array}{ll}
\frac{1}{1+u}&\textrm{ for } b_{alive} \geq b_{alive,target} \\
&\\
1.0&\textrm{ for } b_{alive} <  b_{alive,target} \\
\end{array} \right.\end{split}
\end{equation*}\begin{equation*}
\begin{split}v_{s}= \left\{ \begin{array}{ll}
\frac{u}{1+u}&\textrm{ for } b_{alive} \geq b_{alive,target} \\
&\\
0.0&\textrm{ for } b_{alive} <  b_{alive,target} \\
\end{array} \right.\end{split}
\end{equation*}
In this case, the division of carbon between the live and structural
pools \(u\) is derived as the inverse of the sum of the rates of
change in live biomass with respect to structural:
\begin{equation*}
\begin{split}u = \frac{1}{\frac{\delta b_{leaf}}{ \delta b_{struc} } + \frac{\delta b_{root}}{ \delta b_{struc} } +\frac{\delta b_{sw}}{ \delta b_{struc} } }\end{split}
\end{equation*}
To calculate all these differentials, we first start with
\(\delta b_{leaf}/\delta b_{struc}\), where
\begin{equation*}
\begin{split}\frac{\delta b_{leaf}}{ \delta b_{struc}}= \frac{\frac{\delta \mathrm{dbh}}{\delta b_{struc}}}   {\frac{\delta \mathrm{dbh} }{\delta b_{leaf}} }\end{split}
\end{equation*}
The rates of change of dbh with respect to leaf and structural biomass
are the differentials of the allometric equations linking these terms to
each other. Hence,
\begin{equation*}
\begin{split}\frac{\delta \mathrm{dbh} }{\delta b_{leaf}}=\frac{1}{b_{trim,coh}}\cdot (e_{leaf,dbh}-1)\exp  {\big(c_{leaf} \mathrm{dbh}^{(e_{leaf,dbh})-1} \rho_{ft}^{e_{leaf,dens}} \big)}\end{split}
\end{equation*}
and where \(\mathrm{dbh}_{coh} >   \mathrm{dbh}_{max}\)
\begin{equation*}
\begin{split}\frac{\delta b_{struc}}{\delta \mathrm{dbh}}  = e_{str,dbh} \cdot c_{str}\cdot e_{str,hite} h_{coh}^{e_{str,dbh}-1}   \mathrm{dbh}_{coh}^{e_{str,dbh}} \rho_{ft}^{e_{str,dens}}\end{split}
\end{equation*}
If \(\mathrm{dbh}_{coh} \leq   \mathrm{dbh}_{max}\) then we must
also account for allocation for growing taller as:
\begin{equation*}
\begin{split}\frac{\delta b_{struc}}{\delta \mathrm{dbh}} =  \frac{\delta b_{struc}}{\delta \mathrm{dbh}} + \frac{\delta h}{\delta \mathrm{dbh}} \cdot  \frac{\delta b_{struc}}{\delta \mathrm{dbh} }\end{split}
\end{equation*}
where
\begin{equation*}
\begin{split}\frac{\delta h}{\delta \mathrm{dbh}}= 1.4976  \mathrm{dbh}_{coh}^{m_{allom}-1}\end{split}
\end{equation*}\begin{equation*}
\begin{split}\frac{\delta  \mathrm{dbh} }{\delta b_{struc}} =\frac{1}{ \frac{\delta b_{struc}}{\delta \mathrm{dbh}} }\end{split}
\end{equation*}
Once we have the \(\delta b_{leaf}/\delta b_{struc}\), we calculate
\(\delta b_{root}/\delta b_{struc}\) as
\begin{equation*}
\begin{split}\frac{\delta  b_{root}}{\delta b_{struc}} =\frac{\delta b_{leaf}}{\delta b_{struc}}\cdot f_{frla}\end{split}
\end{equation*}
and the sapwood differential as
\begin{equation*}
\begin{split}\frac{\delta  b_{sw}}{\delta b_{struc}} = f_{swh}\left( h_{coh} \frac{\delta b_{leaf}}{ \delta b_{struc}} + b_{leaf,coh}\frac{\delta h}{\delta b_{struc}} \right)\end{split}
\end{equation*}
where
\begin{equation*}
\begin{split}\frac{\delta h}{\delta b_{struc}} =  \frac{1}{c_{str}\times e_{str,hite} h_{coh}^{e_{str,dbh}-1}   \mathrm{dbh}_{coh}^{e_{str,dbh}} \rho_{ft}^{e_{str,dens}}}\end{split}
\end{equation*}
In all of the above terms, height in in m, \(\mathrm{dbh}\) is in
cm, and all biomass pools are in KgCm\(^{-2}\). The allometric
terms for the growth trajectory are all taken from the ED1.0 model, but
could in theory be altered to accomodate alternative allometric
relationships. Critically, the non-linear relationships between live and
structural biomass pools are maintained in this algorithm, which
diverges from the methodology currently deployed in the CLM4.5.


\subsubsection{Integration of allocated fluxes}
\label{\detokenize{tech_note/DGVM/CLM50_Tech_Note_DGVM:integration-of-allocated-fluxes}}
All of the flux calculations generate differential of the biomass state
variables against time (in years). To integrate these differential rates
into changes in the state variables, we use a simple simple forward
Euler integration. Other methods exist (e.g. ODEINT solvers, Runge Kutta
methods etc.), but they are more prone to errors that become difficult
to diagnose, and the typically slow rates of change of carbon pools mean
that these are less important than they might be in strongly non-linear
systems (soil drainage, energy balance, etc.)
\begin{equation*}
\begin{split}b_{alive,t+1} = \textrm{min}\left( 0,b_{alive,t} +  \frac{\delta b_{alive}}{\delta t}  \delta t \right)\end{split}
\end{equation*}\begin{equation*}
\begin{split}b_{struc,t+1} = \textrm{min}\left(0, b_{struc,t} +  \frac{\delta b_{struc}}{\delta t}  \delta t \right)\end{split}
\end{equation*}\begin{equation*}
\begin{split}b_{store,t+1} = \textrm{min}\left(0, b_{store,t} +  \frac{\delta b_{store}}{\delta t}  \delta t \right)\end{split}
\end{equation*}
In this case, \(\delta t\) is set to be one day
(\(\frac{1}{365}\) years).

\bigskip

\captionof{table}{Parameters needed for allocation model. }


\begin{savenotes}\sphinxattablestart
\centering
\begin{tabulary}{\linewidth}[t]{|T|T|T|T|}
\hline
\sphinxstylethead{\sphinxstyletheadfamily 
Parameter
Symbol
\unskip}\relax &\sphinxstylethead{\sphinxstyletheadfamily 
Parameter Name
\unskip}\relax &\sphinxstylethead{\sphinxstyletheadfamily 
Units
\unskip}\relax &\sphinxstylethead{\sphinxstyletheadfamily 
indexed by
\unskip}\relax \\
\hline
S
&
Target stored
biomass as
fraction of
\(b_{leaf}\)
&
none
&
\sphinxstyleemphasis{ft}
\\
\hline
f
&
Minimum
fraction of
turnover that
must be met
&
none
&
\sphinxstyleemphasis{ft}
\\
\hline
R
&
Fraction
allocated to
seeds
&
none
&
\sphinxstyleemphasis{ft}
\\
\hline
C
&
Fraction
allocated to
clonal
reproduction
&
none
&
\sphinxstyleemphasis{ft}
\\
\hline
\(\textrm{
max}_{dbh}\)
&
Diameter at
which maximum
height is
achieved
&
m
&
\sphinxstyleemphasis{ft}
\\
\hline
P
&
Does this
cohort have an
evergreen
phenological
habit?
&
1=yes, 0=no
&
\sphinxstyleemphasis{ft}
\\
\hline
\end{tabulary}
\par
\sphinxattableend\end{savenotes}

\bigskip


\subsection{Control of Leaf Area Index}
\label{\detokenize{tech_note/DGVM/CLM50_Tech_Note_DGVM:control-of-leaf-area-index}}
{[}17{]}\_The leaf area \(A_{leaf}\) (m:math:\sphinxtitleref{\textasciicircum{}\{-2\}}) of each cohort is
calculated from leaf biomass \(b_{leaf,coh}\) (kgC
individual\(^{-1}\)) and specific leaf area (SLA, m\(^2\) kg
C\(^{-1}\))
\begin{equation*}
\begin{split}A_{leaf,coh} = b_{leaf,coh} \cdot SLA_{ft}\end{split}
\end{equation*}
For a given tree allometry, leaf biomass is determined from basal area
using the function used by {\hyperref[\detokenize{tech_note/References/CLM50_Tech_Note_References:mc-2001}]{\sphinxcrossref{\DUrole{std,std-ref}{Moorcroft et al. 2001}}}} where \(d_w\)
is wood density in g cm\(^{-3}\).
\begin{equation*}
\begin{split}b_{leaf,coh} = c_{leaf} \cdot dbh_{coh}^{e_{leaf,dbh}} \rho_{ft}^{e_{leaf,dens}}\end{split}
\end{equation*}
However, using this model, where leaf area and crown area are both
functions of diameter, the leaf area index of each tree in a closed
canopy forest is always the same (where \(S_{c,patch}\) =
\(S_{c,min}\) , irrespective of the growth conditions. To allow
greater plasticity in tree canopy structure, and for tree leaf area
index to adapt to prevailing conditions, we implemented a methodology
for removing those leaves in the canopy that exist in negative carbon
balance. That is, their total annual assimilation rate is insufficient
to pay for the turnover and maintenance costs associated with their
supportive root and stem tissue, plus the costs of growing the leaf. The
tissue turnover maintenance cost (KgC m\(^{-2} y^{-1}\) of leaf is
the total maintenance demand divided by the leaf area:
\begin{equation*}
\begin{split}L_{cost,coh} = \frac{t_{md,coh}} {b_{leaf,coh} \cdot \textrm{SLA}}\end{split}
\end{equation*}
The net uptake for each leaf layer \(U_{net,z}\) in (KgC
m\(^{-2}\) year\(^{-1}\)) is
\begin{equation*}
\begin{split}U_{net,coh,z} = g_{coh,z}-r_{m,leaf,coh,z}\end{split}
\end{equation*}
where \(g_{z}\) is the GPP of each layer of leaves in each tree (KgC
m\(^{-2}\) year\(^{-1}\)), \(r_{m,leaf,z}\) is the rate
of leaf dark respiration (also KgC m\(^{-2}\)
year\(^{-1}\)). We use an iterative scheme to define the cohort
specific canopy trimming fraction \(C_{trim,coh}\), on an annual
time-step, where
\begin{equation*}
\begin{split}b_{leaf,coh} =   C_{trim} \times 0.0419  dbh_{coh}^{1.56} d_w^{0.55}\end{split}
\end{equation*}
If the annual maintenance cost of the bottom layer of leaves (KgC m-2
year-1) is less than then the canopy is trimmed by an increment \(\iota_l\)(0.01), which is applied until the end of  next calander year. Because this is an optimality model, there is an
issue of the timescale over which net assimilation is evaluated, the
timescale of response, and the plasticity of plants to respond to these
pressures. These properties should be investigated further in future
efforts.
\begin{equation*}
\begin{split}C_{trim,y+1}  = \left\{ \begin{array}{ll}
\rm{max}(C_{trim,y}-\iota_l,1.0)&\rm{for} (L_{cost,coh} > U_{net,coh,nz})\\
&\\
\rm{min}(C_{trim,y}+\iota_l,L_{trim,min})&\rm{for} (L_{cost,coh} < U_{net,coh,nz})\\
\end{array} \right.\end{split}
\end{equation*}
We impose an arbitrary minimum value on the scope of canopy trimming of
\(L_{trim,min}\) (0.5). If plants are able simply to drop all of
their canopy in times of stress, with no consequences, then tree
mortality from carbon starvation is much less likely to occur because of
the greatly reduced maintenance and turnover requirements.

\bigskip

\captionof{table}{Parameters needed for leaf area control model.  }


\begin{savenotes}\sphinxattablestart
\centering
\begin{tabular}[t]{|*{4}{\X{1}{4}|}}
\hline
\sphinxstylethead{\sphinxstyletheadfamily 
Parameter
Symbol
\unskip}\relax &\sphinxstylethead{\sphinxstyletheadfamily 
Parameter Name
\unskip}\relax &\sphinxstylethead{\sphinxstyletheadfamily 
Units
\unskip}\relax &\sphinxstylethead{\sphinxstyletheadfamily 
indexed by
\unskip}\relax \\
\hline
\(\iota_l\)
&
Fraction by
which leaf mass
is reduced next
year
&
none
&\begin{itemize}
\item {} 
\end{itemize}
\\
\hline
\(L_{trim,
min}\)
&
Minimum
fraction to
which leaf mass
can be reduced
&\begin{itemize}
\item {} 
\end{itemize}
&\\
\hline
\end{tabular}
\par
\sphinxattableend\end{savenotes}

\bigskip


\subsection{Phenology}
\label{\detokenize{tech_note/DGVM/CLM50_Tech_Note_DGVM:phenology}}

\subsubsection{Cold Deciduous Phenology}
\label{\detokenize{tech_note/DGVM/CLM50_Tech_Note_DGVM:cold-deciduous-phenology}}

\paragraph{Cold Leaf-out timing}
\label{\detokenize{tech_note/DGVM/CLM50_Tech_Note_DGVM:cold-leaf-out-timing}}
%
\begin{footnote}[18]\sphinxAtStartFootnote
This description relates to algorithms in the phenology subroutine
%
\end{footnote}. The phenology model of {\hyperref[\detokenize{tech_note/References/CLM50_Tech_Note_References:botta2000}]{\sphinxcrossref{\DUrole{std,std-ref}{Botta et al. 2000}}}} is used in
FATES to determine the leaf-on timing. The Botta et al. model was
verified against satellite data and is one of the only globally verified
and published models of leaf-out phenology. This model differs from the
phenology model in the CLM4.5. The model simulates leaf-on date as a
function of the number of growing degree days (GDD), defined by the sum
of mean daily temperatures (\(T_{day}\) \(^{o}\)C) above a
given threshold \(T_{g}\) (0 \(^{o}\)C).
\begin{equation*}
\begin{split}GDD=\sum \textrm{max}(T_{day}-T_{g},0)\end{split}
\end{equation*}
Budburst occurs when \(GDD\) exceeds a threshold
(\(GDD_{crit}\)). The threshold is modulated by the number of
chilling days experienced (NCD) where the mean daily temperature falls
below a threshold determined by \sphinxtitleref{Botta et al. 2000\textless{}botta2000\textgreater{}} as
5\(^{o}\)C. A greater number of chilling days means that fewer
growing degree days are required before budburst:
\begin{equation*}
\begin{split}GDD_{crit}=a+be^{c.NCD}\end{split}
\end{equation*}
where a = -68, b= 638 and c=-0.01 \sphinxtitleref{Botta et al. 2000\textless{}botta2000\textgreater{}}. In the
Northern Hemisphere, counting of degree days begins on 1st January, and
of chilling days on 1st November. The calendar opposite of these dates
is used for points in the Southern Hemisphere.

If the growing degree days exceed the critical threshold, leaf-on is
triggered by a change in the gridcell phenology status flag
\(S_{phen,grid}\) where ‘2’ indicates that leaves should come on and
‘1’ indicates that they should fall.
\begin{equation*}
\begin{split}\begin{array}{ll}
S_{phen,grid} = 2
&\textrm{ if } S_{phen,grid} = 1\textrm{ and } GDD_{grid} \ge GDD_{crit} \\
\end{array}\end{split}
\end{equation*}

\paragraph{Cold Leaf-off timing}
\label{\detokenize{tech_note/DGVM/CLM50_Tech_Note_DGVM:cold-leaf-off-timing}}
The leaf-off model is taken from the Sheffield Dynamic Vegetation Model
(SDGVM) and is similar to that for LPJ
{\hyperref[\detokenize{tech_note/References/CLM50_Tech_Note_References:sitch2003}]{\sphinxcrossref{\DUrole{std,std-ref}{Sitch et al. 2003}}}} and IBIS
{\hyperref[\detokenize{tech_note/References/CLM50_Tech_Note_References:foley1996}]{\sphinxcrossref{\DUrole{std,std-ref}{Foley et al. 1996}}}} models. The average daily
temperatures of the previous 10 day period are stored. Senescence is
triggered when the number of days with an average temperature below
7.5\(^{o}\) (\(n_{colddays}\)) rises above a threshold values
\(n_{crit,cold}\), set at 5 days.
\begin{equation*}
\begin{split}\begin{array}{ll}
S_{phen,grid} = 1
&\textrm{ if } S_{phen,grid} = 2\textrm{ and } n_{colddays} \ge n_{crit,cold} \\
\end{array}\end{split}
\end{equation*}

\paragraph{Global implementation modifications}
\label{\detokenize{tech_note/DGVM/CLM50_Tech_Note_DGVM:global-implementation-modifications}}
Because of the global implementation of the cold-deciduous phenology
scheme, adjustments must be made to account for the possibility of
cold-deciduous plants experiencing situations where no chilling period
triggering leaf-off ever happens. If left unaccounted for, these leaves
will last indefinitely, resulting in highly unrealistic behaviour.
Therefore, we implement two additional rules. Firstly, if the number of
days since the last senescence event was triggered is larger than 364,
then leaf-off is triggered on that day. Secondly, if no chilling days
have occured during the winter accumulation period, then leaf-on is not
triggered. This means that in effect, where there are no cold periods,
leaves will fall off and not come back on, meaning that cold-deciduous
plants can only grow in places where there is a cold season.

Further to this rule, we introduce a ‘buffer’ time periods after leaf-on
of 30 days, so that cold-snap periods in the spring cannot trigger a
leaf senescence. The 30 day limit is an arbitrary limit. In addition, we
constrain growing degree day accumulation to the second half of the year
(Jult onwards in the Northern hemisphere, or Jan-June in the Southern)
and only allow GDD accumulation while the leaves are off.


\subsubsection{Drought-deciduous Phenology: TBD}
\label{\detokenize{tech_note/DGVM/CLM50_Tech_Note_DGVM:drought-deciduous-phenology-tbd}}
In the current version of the model, a drought deciduous algorithm
exists, but is not yet operational, due to issue detected in the existing
CN and soil moisture modules, which also affect the behaviour of the
native ED drought deciduous model. This is a priority to address before
the science tag is released.


\subsubsection{Carbon Dynamics of deciduous plants}
\label{\detokenize{tech_note/DGVM/CLM50_Tech_Note_DGVM:carbon-dynamics-of-deciduous-plants}}
{[}19{]}\_In the present version, leaf expansion and senescence happen over the
course of a single day. This is clearly not an empirically robust
representation of leaf behaviour, whereby leaf expansion occurs over a
period of 10-14 days, and senescence over a similar period. This will be
incorporated in later versions. When the cold or drought phenological
status of the gridcell status changes (\(S_{phen,grid}\)) from ‘2’
to ‘1’, and the leaves are still on (\(S_{phen,coh}\) =2 ), the leaf
biomass at this timestep is ’remembered’ by the model state variable
\(l_{memory,coh}\). This provides a ‘target’ biomass for leaf onset
at the beginning of the next growing season (it is a target, since
depletion of stored carbon in the off season may render achieving the
target impossible).
\begin{equation*}
\begin{split}l_{memory,coh} = b_{leaf,coh}\end{split}
\end{equation*}
Leaf carbon is then added to the leaf litter flux \(l_{leaf,coh}\)
(KgC individual\(^{-1}\))
\begin{equation*}
\begin{split}l_{leaf,coh} = b_{leaf,coh}\end{split}
\end{equation*}
The alive biomass is depleted by the quantity of leaf mass lost, and the
leaf biomass is set to zero
\begin{equation*}
\begin{split}b_{alive,coh} = b_{alive,coh} - b_{leaf,coh}\end{split}
\end{equation*}\begin{equation*}
\begin{split}b_{leaf,coh} = 0\end{split}
\end{equation*}
Finally, the status \(S_{phen,coh}\) is set to 1, indicating that
the leaves have fallen off.

For bud burst, or leaf-on, the same occurs in reverse. If the leaves are
off (\(S_{phen,coh}\)=1) and the phenological status triggers
budburst (\(S_{phen,grid}\)=2) then the leaf mass is set the
maximum of the leaf memory and the available store
\begin{equation*}
\begin{split}b_{leaf,coh} =  \textrm{max}\left(l_{memory,coh}, b_{store,coh}\right.)\end{split}
\end{equation*}
this amount of carbon is removed from the store
\begin{equation*}
\begin{split}b_{store,coh} = b_{store,coh}  - b_{leaf,coh}\end{split}
\end{equation*}
and the new leaf biomass is added to the alive pool
\begin{equation*}
\begin{split}b_{alive,coh} = b_{alive,coh}  + b_{leaf,coh}\end{split}
\end{equation*}
Lastly, the leaf memory variable is set to zero and the phenological
status of the cohort back to ‘2’. No parameters are currently required
for this carbon accounting scheme.

\bigskip

\captionof{table}{Parameters needed for phenology model.  }


\begin{savenotes}\sphinxattablestart
\centering
\begin{tabular}[t]{|*{4}{\X{1}{4}|}}
\hline
\sphinxstylethead{\sphinxstyletheadfamily 
Parameter
Symbol
\unskip}\relax &\sphinxstylethead{\sphinxstyletheadfamily 
Parameter Name
\unskip}\relax &\sphinxstylethead{\sphinxstyletheadfamily 
Units
\unskip}\relax &\sphinxstylethead{\sphinxstyletheadfamily 
indexed by
\unskip}\relax \\
\hline
\(n_{crit,
cold}\)
&
Threshold of
cold days for
senescence
&
none
&\begin{itemize}
\item {} 
\end{itemize}
\\
\hline
\(T_{g}\)
&
Threshold for
counting
growing degree
days
&
\(^{o}\)C
&\\
\hline
\end{tabular}
\par
\sphinxattableend\end{savenotes}

\bigskip


\subsection{Seed Dynamics and Recruitment}
\label{\detokenize{tech_note/DGVM/CLM50_Tech_Note_DGVM:seed-dynamics-and-recruitment}}
{[}20{]}\_The production of seeds and their subsequent germination is a process
that must be captured explicitly or implicitly in vegetation models. FATES contains a seed bank model designed to allow the dynamics of
seed production and germination to be simulated independently. In the
ED1.0 model, seed recruitment occurs in the same timestep as allocation
to seeds, which prohibits the survival of a viable seed bank through a
period of disturbance or low productivity (winter, drought). In FATES, a plant functional type specific seed bank is tracked in
each patch (\(Seeds_{patch}\) KgC m\(^{-2}\)), whose rate of
change (KgC m\(^{-2}\) y\(^{-1}\)) is the balance of inputs,
germination and decay:
\begin{equation*}
\begin{split}\frac{\delta Seeds_{FT}}{\delta t } = Seed_{in,ft} - Seed_{germ,ft} - Seed_{decay,ft}\end{split}
\end{equation*}
where \(Seed_{in}\), \(Seed_{germ}\) and \(Seed_{decay}\)
are the production, germination and decay (or onset of inviability) of
seeds, all in KgC m\(^{-2}\) year\(^{-1}\).

Seeds are assumed to be distributed evenly across the site (in this
version of the model), so the total input to the seed pool is therefore
the sum of all of the reproductive output of all the cohorts in each
patch of the correct PFT type.
\begin{equation*}
\begin{split}Seed_{in,ft} =  \frac{\sum_{p=1}^{n_{patch}}\sum_{i=1}^{n_{coh}}p_{seed,i}.n_{coh}}{area_{site}}\end{split}
\end{equation*}
Seed decay is the sum of all the processes that reduce the number of
seeds, taken from {\hyperref[\detokenize{tech_note/References/CLM50_Tech_Note_References:lischke2006}]{\sphinxcrossref{\DUrole{std,std-ref}{Lischke et al. 2006}}}}. Firstly, the rate at
which seeds become inviable is described as a constant rate \(\phi\)
(y:math:\sphinxtitleref{\textasciicircum{}\{-1\}}) which is set to 0.51, the mean of the parameters used
by {\hyperref[\detokenize{tech_note/References/CLM50_Tech_Note_References:lischke2006}]{\sphinxcrossref{\DUrole{std,std-ref}{Lischke et al. 2006}}}}.
\begin{equation*}
\begin{split}Seed_{decay,ft} = Seeds_{FT}.\phi\end{split}
\end{equation*}
The seed germination flux is also prescribed as a fraction of the
existing pool (\(\alpha_{sgerm}\)), but with a cap on maximum
germination rate \(\beta_{sgerm}\), to prevent excessive dominance
of one plant functional type over the seed pool.
\begin{equation*}
\begin{split}Seed_{germ,ft} = \textrm{max}(Seeds_{FT}\cdot \alpha_{sgerm},\beta_{sgerm})\end{split}
\end{equation*}
\bigskip

\captionof{table}{Parameters needed for seed model.  }


\begin{savenotes}\sphinxattablestart
\centering
\begin{tabular}[t]{|*{4}{\X{1}{4}|}}
\hline
\sphinxstylethead{\sphinxstyletheadfamily 
Parameter
Symbol
\unskip}\relax &\sphinxstylethead{\sphinxstyletheadfamily 
Parameter Name
\unskip}\relax &\sphinxstylethead{\sphinxstyletheadfamily 
Units
\unskip}\relax &\sphinxstylethead{\sphinxstyletheadfamily 
indexed by
\unskip}\relax \\
\hline
\(K_s\)
&
Maximum seed
mass
&
kgC m\(^{-2}\)
&\\
\hline
\(\alpha_{
sgerm}\)
&
Proportional
germination
rate
&\begin{itemize}
\item {} 
\end{itemize}
&\\
\hline
\(\beta_{s
germ}\)
&
Maximum
germination
rate
&
KgC
m\(^{-2}\)

y\(^{-1}\)
&\\
\hline
\(\phi\)
&
Decay rate of
viable seeds
&
none
&
FT
\\
\hline
\(R_{frac,
ft}\)
&
Fraction of
\(C_{bal}\)
devoted to
reproduction
&
none
&
FT
\\
\hline
\end{tabular}
\par
\sphinxattableend\end{savenotes}

\bigskip


\subsection{Litter Production and Fragmentation}
\label{\detokenize{tech_note/DGVM/CLM50_Tech_Note_DGVM:litter-production-and-fragmentation}}
\begin{DUlineblock}{0em}
\item[] The original CLM4.5 model contains streams of carbon pertaining to
different chemical properties of litter (lignin, cellulose and labile
streams, specifically). In FATES model, the fire simulation
scheme in the SPITFIRE model requires that the model tracks the pools
of litter pools that differ with respect to their propensity to burn
(surface area-volume ratio, bulk density etc.). Therefore, this model
contains more complexity in the representation of coarse woody debris.
We also introduce the concept of ’fragmenting’ pools, which are pools
that can be burned, but are not available for decomposition or
respiration. In this way, we can both maintain above-ground pools that
affect the rate of burning, and the lag between tree mortality and
availability of woody material for decomposition.
\item[] FATES recognizes four classes of litter. Above- and below-ground
coarse woody debris (\(CWD_{AG}\), \(CWD_{BG}\)) and leaf
litter (\(l_{leaf}\) and fine root litter \(l_{root}\)). All
pools are represented per patch, and with units of kGC
m\({^-2}\). Further to this, \(CWD_{AG}\), \(CWD_{BG}\)
are split into four litter size classes (\(lsc\)) for the purposes
of proscribing this to the SPITFIRE fire model (seed ’Fuel Load’
section for more detail. 1-hour (twigs), 10-hour (small branches),
100-hour (large branches) and 1000-hour(boles or trunks). 4.5 \%, 7.5\%,
21 \% and 67\% of the woody biomass (\(b_{store,coh} + b_{sw,coh}\))
is partitioned into each class, respectively.
\end{DUlineblock}

\(l_{leaf}\) and \(l_{root}\) are indexed by plant functional
type (\(ft\)). The rational for indexing leaf and fine root by PFT
is that leaf and fine root matter typically vary in their
carbon:nitrogen ratio, whereas woody pools typically do not.

Rates of change of litter, all in kGC m\({^-2}\)
year\(^{-1}\), are calculated as
\begin{equation*}
\begin{split}\frac{\delta CWD_{AG,out,lsc}}{ \delta t }= CWD_{AG,in,lsc} - CWD_{AG,out,lsc}\end{split}
\end{equation*}\begin{equation*}
\begin{split}\frac{\delta CWD_{BG,out,lsc}}{ \delta t } = CWD_{BG,in,lsc} - CWD_{BG,in,lsc}\end{split}
\end{equation*}\begin{equation*}
\begin{split}\frac{\delta l_{leaf,out,ft} }{ \delta t } = l_{leaf,in,ft} -  l_{leaf,out,ft}\end{split}
\end{equation*}\begin{equation*}
\begin{split}\frac{\delta l_{root,out,ft} }{ \delta t } = l_{root,in,ft} - l_{root,out,ft}\end{split}
\end{equation*}

\subsubsection{Litter Inputs}
\label{\detokenize{tech_note/DGVM/CLM50_Tech_Note_DGVM:litter-inputs}}
{[}21{]}\_Inputs into the litter pools come from tissue turnover, mortality of
canopy trees, mortality of understorey trees, mortality of seeds, and
leaf senescence of deciduous plants.
\begin{equation*}
\begin{split}l_{leaf,in,ft} =\Big(\sum_{i=1}^{n_{coh,ft}} n_{coh}(l_{md,coh}  + l_{leaf,coh}) + M_{t,coh}.b_{leaf,coh}\Big)/\sum_{p=1}^{n_{pat}}A_{patch}\end{split}
\end{equation*}
where \(l_{md,coh}\) is the leaf turnover rate for evergreen trees
and \(l_{leaf,coh}\) is the leaf loss from phenology in that
timestep (KgC \(m^{-2}\). \(M_{t,coh}\) is the total mortality
flux in that timestep (in individuals). For fine root input.
\(n_{coh,ft}\) is the number of cohorts of functional type
‘\(FT\)’ in the current patch.
\begin{equation*}
\begin{split}l_{root,in,ft} =\Big(\sum_{i=1}^{n_{coh,ft}} n_{coh}(r_{md,coh} ) + M_{t,coh}.b_{root,coh}\Big)/\sum_{p=1}^{n_{pat}}A_p\end{split}
\end{equation*}
where \(r_{md,coh}\) is the root turnover rate. For coarse woody
debris input (\(\mathit{CWD}_{AG,in,lsc}\) , we first calculate the
sum of the mortality \(M_{t,coh}.(b_{struc,coh}+b_{sw,coh})\) and
turnover \(n_{coh}(w_{md,coh}\)) fluxes, then separate these into
size classes and above/below ground fractions using the fixed fractions
assigned to each (\(f_{lsc}\) and \(f_{ag}\))
\begin{equation*}
\begin{split}\mathit{CWD}_{AG,in,lsc} =\Big(f_{lsc}.f_{ag}\sum_{i=1}^{n_{coh,ft}}n_{coh}w_{md,coh}  + M_{t,coh}.(b_{struc,coh}+b_{sw,coh})\Big)/\sum_{p=1}^{n_{pat}}A_p\end{split}
\end{equation*}\begin{equation*}
\begin{split}\mathit{CWD}_{BG,in,lsc} =\Big(f_{lsc}.(1-f_{ag})\sum_{i=1}^{n_{coh,ft}}n_{coh}w_{md,coh}  + M_{t,coh}.(b_{struc,coh}+b_{sw,coh})\Big)/\sum_{p=1}^{n_{pat}}A_p\end{split}
\end{equation*}

\subsubsection{Litter Outputs}
\label{\detokenize{tech_note/DGVM/CLM50_Tech_Note_DGVM:litter-outputs}}
{[}22{]}\_The fragmenting litter pool is available for burning but not for
respiration or decomposition. Fragmentation rates are calculated
according to a maximum fragmentation rate (\(\alpha_{cwd,lsc}\) or
\(\alpha_{litter}\)) which is ameliorated by a temperature and water
dependent scalar \(S_{tw}\). The form of the temperature scalar is
taken from the existing CLM4.5BGC decomposition cascade calculations).
The water scaler is equal to the water limitation on photosynthesis
(since the CLM4.5BGC water scaler pertains to the water potential of
individual soil layers, which it is difficult to meaningfully average,
given the non-linearities in the impact of soil moisture). The scaler
code is modular, and new functions may be implemented trivially. Rate
constants for the decay of the litter pools are extremely uncertain in
literature, as few studies either separate litter into size classes, nor
examine its decomposition under non-limiting moisture and temperature
conditions. Thus, these parameters should be considered as part of
sensitivity analyses of the model outputs.
\begin{equation*}
\begin{split}\mathit{CWD}_{AG,out,lsc} = CWD_{AG,lsc}. \alpha_{cwd,lsc}.S_{tw}\end{split}
\end{equation*}\begin{equation*}
\begin{split}\mathit{CWD}_{BG,out,lsc} = CWD_{BG,lsc} .\alpha_{cwd,lsc}.S_{tw}\end{split}
\end{equation*}\begin{equation*}
\begin{split}l_{leaf,out,ft} = l_{leaf,ft}.\alpha_{litter}.S_{tw}\end{split}
\end{equation*}\begin{equation*}
\begin{split}l_{root,out,ft} = l_{root,ft}.\alpha_{root,ft}.S_{tw}\end{split}
\end{equation*}

\subsubsection{Flux into decompsition cascade}
\label{\detokenize{tech_note/DGVM/CLM50_Tech_Note_DGVM:flux-into-decompsition-cascade}}
{[}23{]}\_Upon fragmentation and release from the litter pool, carbon is
transferred into the labile, lignin and cellulose decomposition pools.
These pools are vertically resolved in the biogeochemistry model. The
movement of carbon into each vertical layer is obviously different for
above- and below-ground fragmenting pools. For each layer \(z\) and
chemical litter type \(i\), we derive a flux from ED into the
decomposition cascade as \(ED_{lit,i,z}\) (kGC m\(^{-2}\)
s\(^{-1}\))

where \(t_c\) is the time conversion factor from years to seconds,
\(f_{lab,l}\), \(f_{cel,l}\) and \(f_{lig,l}\) are the
fractions of labile, cellulose and lignin in leaf litter, and
\(f_{lab,r}\), \(f_{cel,r}\) and \(f_{lig,r}\) are their
counterparts for root matter. Similarly, \(l_{prof}\),
\(r_{f,prof}\)and \(r_{c,prof}\) are the fractions of leaf,
coarse root and fine root matter that are passed into each vertical soil
layer \(z\), derived from the CLM(BGC) model.

\bigskip

\captionof{table}{Parameters needed for litter model.  }


\begin{savenotes}\sphinxattablestart
\centering
\begin{tabulary}{\linewidth}[t]{|T|T|T|T|}
\hline
\sphinxstylethead{\sphinxstyletheadfamily 
Parameter
Symbol
\unskip}\relax &\sphinxstylethead{\sphinxstyletheadfamily 
Parameter Name
\unskip}\relax &\sphinxstylethead{\sphinxstyletheadfamily 
Units
\unskip}\relax &\sphinxstylethead{\sphinxstyletheadfamily 
indexed by
\unskip}\relax \\
\hline
\(\alpha_{
cwd,lsc}\)
&
Maximum
fragmentation
rate of CWD
&
y\(^{-1}\)
&\\
\hline
\(\alpha_{
litter}\)
&
Maximum
fragmentation
rate of leaf
litter
&
y\(^{-1}\)
&\\
\hline
\(\alpha_{
root}\)
&
Maximum
fragmentation
rate of fine
root litter
&
y\(^{-1}\)
&\\
\hline
\(f_{lab,l
}\)
&
Fraction of
leaf mass in
labile carbon
pool
&
none
&\\
\hline
\(f_{cel,l
}\)
&
Fraction of
leaf mass in
cellulose
carbon pool
&
none
&\\
\hline
\(f_{lig,l
}\)
&
Fraction of
leaf mass in
lignin carbon
pool
&
none
&\\
\hline
\(f_{lab,r
}\)
&
Fraction of
root mass in
labile carbon
pool
&
none
&\\
\hline
\(f_{cel,r
}\)
&
Fraction of
root mass in
cellulose
carbon pool
&
none
&\\
\hline
\(f_{lig,r
}\)
&
Fraction of
root mass in
lignin carbon
pool
&
none
&\\
\hline
\(l_{prof,
z}\)
&
Fraction of
leaf matter
directed to
soil layer z
&
none
&
soil layer
\\
\hline
\(r_{c,pro
f,z}\)
&
Fraction of
coarse root
matter directed
to soil layer z
&
none
&
soil layer
\\
\hline
\(r_{f,pro
f,z}\)
&
Fraction of
fine root
matter directed
to soil layer z
&
none
&
soil layer
\\
\hline
\end{tabulary}
\par
\sphinxattableend\end{savenotes}

\bigskip


\subsection{Plant Mortality}
\label{\detokenize{tech_note/DGVM/CLM50_Tech_Note_DGVM:plant-mortality}}
Total plant mortality per cohort \(M_{t,coh}\), (fraction
year\(^{-1}\)) is simulated as the sum of four additive terms,
\begin{equation*}
\begin{split}M_{t,coh}= M_{b,coh} + M_{cs,coh} + M_{hf,coh} + M_{f,coh},\end{split}
\end{equation*}
where \(M_b\) is the background mortality that is unaccounted by
any of the other mortality rates and is fixed at 0.014. \(M_{cs}\)
is the carbon starvation derived mortality, which is a function of the
non-structural carbon storage term \(b_{store,coh}\) and the
PFT-specific ‘target’ carbon storage, \(l_{targ,ft}\), as follows:
\begin{equation*}
\begin{split}M_{cs,coh}= \rm{max} \left(0.0, S_{m,ft} \left(0.5 -  \frac{b_{store,coh}}{l_{targ,ft}b_{leaf}}\right)\right)\end{split}
\end{equation*}
where \(S_{m,ft}\) is the \sphinxtitleref{stress mortality} parameter, or the
fraction of trees in a landscape that die when the mean condition of a
given cohort triggers mortality. This parameter is needed to scale from
individual-level mortality simulation to grid-cell average conditions.

Mechanistic simulation of hydraulic failure is not undertaken on account
of it’s mechanistic complexity (see \DUrole{xref,std,std-ref}{McDowell et al. 2013\textless{}mcdowell2013\textgreater{}{}`for
details). Instead, we use a proxy for hydraulic failure induced
mortality (:math:{}`M\_\{hf,coh\}}) that uses a water potential threshold
beyond mortality is triggered, such that the tolerance of low water
potentials is a function of plant functional type (as expressed via the
\(\psi_c\) parameter). For each day that the aggregate water
potential falls below a threshold value, a set fraction of the trees are
killed. The aggregation of soil moisture potential across the root zone
is expressed using the \(\beta\) function. We thus determine plant
mortality caused by extremely low water potentials as
\begin{equation*}
\begin{split}M_{hf,coh} = \left\{ \begin{array}{ll}
S_{m,ft}& \textrm{for } \beta_{ft} < 10^{-6}\\
&\\
0.0& \textrm{for } \beta_{ft}>= 10^{-6}.\\
\end{array} \right.\end{split}
\end{equation*}
The threshold value of 10\(^{-6}\) represents a state where the
average soil moisture potential is within 10\(^{-6}\) of the
wilting point (a PFT specific parameter \(\theta_{w,ft}\)).

\(M_{hf,coh}\) is the fire-induced mortality, as described in the
fire modelling section.

\bigskip

\captionof{table}{Parameters needed for mortality model.  }


\begin{savenotes}\sphinxattablestart
\centering
\begin{tabulary}{\linewidth}[t]{|T|T|T|T|}
\hline
\sphinxstylethead{\sphinxstyletheadfamily 
Parameter Symbol
\unskip}\relax &\sphinxstylethead{\sphinxstyletheadfamily 
Parameter Name
\unskip}\relax &\sphinxstylethead{\sphinxstyletheadfamily 
Units
\unskip}\relax &\sphinxstylethead{\sphinxstyletheadfamily 
indexed by
\unskip}\relax \\
\hline
\(S_{m,ft}\)
&
Stress Mortality Scaler
&
none
&\\
\hline
\(l_{targ,ft}\)
&
Target carbon storage fraction
&
none
&
ft
\\
\hline
\end{tabulary}
\par
\sphinxattableend\end{savenotes}

\bigskip


\subsection{Fire (SPITFIRE)}
\label{\detokenize{tech_note/DGVM/CLM50_Tech_Note_DGVM:fire-spitfire}}
{[}24{]}\_The influence of fire on vegetation is estimated using the SPITFIRE
model, which has been modified for use in ED following it’s original
implementation in the LPJ-SPITFIRE model
(\DUrole{xref,std,std-ref}{Thonicke et al. 2010}). This model as
described is substantially different from the existing CLM4.5 fire model
{\hyperref[\detokenize{tech_note/References/CLM50_Tech_Note_References:li2012}]{\sphinxcrossref{\DUrole{std,std-ref}{Li et al. 2012}}}}, however, further developments are
intended to increase the merging of SPITFIRE’s natural vegetation fire
scheme with the fire suppression, forest-clearing and peat fire
estimations in the existing model. The coupling to the ED model allows
fires to interact with vegetation in a size-structured manner, so
small fires can burn only understorey vegetation. Also, the patch
structure and representation of succession in the ED model allows the
model to track the impacts of fire on different forest stands, therefore
removing the problem of area-averaging implicit in area-based DGVMs. The
SPITFIRE approach has also been coupled to the LPJ-GUESS
individual-based model (Forrest et al. in prep) and so this is not the
only implementation of this type of scheme in existence.

The SPITFIRE model operates at a daily timestep and at the patch level,
meaning that different litter pools and vegetation charecteristics of
open and closed forests can be represented effectively (we omit the
\sphinxtitleref{patch} subscript throughout for simplicity).


\subsubsection{Properties of fuel load}
\label{\detokenize{tech_note/DGVM/CLM50_Tech_Note_DGVM:properties-of-fuel-load}}
Many fire processes are impacted by the properties of the litter pool in
the SPITFIRE model. There are one live (live grasses) and five dead fuel
categories (dead leaf litter and four pools of coarse woody debris).
Coarse woody debris is classified into 1h, 10h, 100h, and 1000h fuels,
defined by the order of magnitude of time required for fuel to lose
(or gain) 63\% of the difference between its current moisture content and
the equilibrium moisture content under defined atmospheric conditions.
{\hyperref[\detokenize{tech_note/References/CLM50_Tech_Note_References:thonicke2010}]{\sphinxcrossref{\DUrole{std,std-ref}{Thonicke et al. 2010}}}}. For the purposes of describing
the behaviour of fire, we introduce a new index ‘fuel class’ \sphinxstyleemphasis{fc}, the
values of which correspond to each of the six possible fuel categories
as follows.


\begin{savenotes}\sphinxattablestart
\centering
\begin{tabulary}{\linewidth}[t]{|T|T|T|}
\hline
\sphinxstylethead{\sphinxstyletheadfamily 
\sphinxstyleemphasis{fc} index
\unskip}\relax &\sphinxstylethead{\sphinxstyletheadfamily 
Fuel type
\unskip}\relax &\sphinxstylethead{\sphinxstyletheadfamily 
Drying Time
\unskip}\relax \\
\hline
1
&
dead grass
&
n/a
\\
\hline
2
&
twigs
&
1h fuels
\\
\hline
3
&
small branches
&
10h fuel
\\
\hline
4
&
large branches
&
100h fuel
\\
\hline
5
&
stems and trunks
&
1000h fuel
\\
\hline
6
&
live grasses
&
n/a
\\
\hline
\end{tabulary}
\par
\sphinxattableend\end{savenotes}

\bigskip


\subsubsection{Nesterov Index}
\label{\detokenize{tech_note/DGVM/CLM50_Tech_Note_DGVM:nesterov-index}}
Dead fuel moisture (\(\emph{moist}_{df,fc}\)), and several other
properties of fire behaviour, are a function of the ‘Nesterov Index’
(\(N_{I}\)) which is an accumulation over time of a function of
temperature and humidity (Eqn 5, {\hyperref[\detokenize{tech_note/References/CLM50_Tech_Note_References:thonicke2010}]{\sphinxcrossref{\DUrole{std,std-ref}{Thonicke et al. 2010}}}}).
\begin{equation*}
\begin{split}N_{I}=\sum{\textrm{max}(T_{d}(T_{d}-D),0)}\end{split}
\end{equation*}
where \(T_{d}\) is the daily mean temperature in \(^{o}\)C and
\(D\) is the dew point calculated as .
\begin{equation*}
\begin{split}\begin{aligned}
\upsilon&=&\frac{17.27T_{d}}{237.70+T_{d}}+\log(RH/100)\\
D&=&\frac{237.70\upsilon}{17.27-\upsilon}\end{aligned}\end{split}
\end{equation*}
where \(RH\) is the relative humidity (\%).


\subsubsection{Fuel properties}
\label{\detokenize{tech_note/DGVM/CLM50_Tech_Note_DGVM:fuel-properties}}
Total fuel load \(F_{tot,patch}\) for a given patch is the sum of
the above ground coarse woody debris and the leaf litter, plus the alive
grass leaf biomass \(b_{l,grass}\) multiplied by the non-mineral
fraction (1-\(M_{f}\)).
\begin{equation*}
\begin{split}F_{tot,patch}=\left(\sum_{fc=1}^{fc=5}  CWD_{AG,fc}+l_{litter}+b_{l,grass}\right)(1-M_{f})\end{split}
\end{equation*}
Many of the model behaviours are affected by the patch-level weighted
average properties of the fuel load. Typically, these are calculated in
the absence of 1000-h fuels because these do not contribute greatly to
fire spread properties.


\paragraph{Dead Fuel Moisture Content}
\label{\detokenize{tech_note/DGVM/CLM50_Tech_Note_DGVM:dead-fuel-moisture-content}}
Dead fuel moisture is calculated as
\begin{equation*}
\begin{split}\emph{moist}_{df,fc}=e^{-\alpha_{fmc,fc}N_{I}}\end{split}
\end{equation*}
where \(\alpha_{fmc,fc}\) is a parameter defining how fuel moisture
content varies between the first four dead fuel classes.


\paragraph{Live grass moisture Content}
\label{\detokenize{tech_note/DGVM/CLM50_Tech_Note_DGVM:live-grass-moisture-content}}
The live grass fractional moisture content(\(\emph{moist}_{lg}\))
is a function of the soil moisture content. (Equation B2 in
{\hyperref[\detokenize{tech_note/References/CLM50_Tech_Note_References:thonicke2010}]{\sphinxcrossref{\DUrole{std,std-ref}{Thonicke et al. 2010}}}})
\begin{equation*}
\begin{split}\emph{moist}_{lg}=\textrm{max}(0.0,\frac{10}{9}\theta_{30}-\frac{1}{9})\end{split}
\end{equation*}
where \(\theta_{30}\) is the fractional moisture content of the top
30cm of soil.


\paragraph{Patch Fuel Moisture}
\label{\detokenize{tech_note/DGVM/CLM50_Tech_Note_DGVM:patch-fuel-moisture}}
The total patch fuel moisture is based on the weighted average of the
different moisture contents associated with each of the different live
grass and dead fuel types available (except 1000-h fuels).
\begin{equation*}
\begin{split}F_{m,patch}=\sum_{fc=1}^{fc=4}  \frac{F_{fc}}{F_{tot}}\emph{moist}_{df,fc}+\frac{b_{l,grass}}{F_{tot}}\emph{moist}_{lg}\end{split}
\end{equation*}

\paragraph{Effective Fuel Moisture Content}
\label{\detokenize{tech_note/DGVM/CLM50_Tech_Note_DGVM:effective-fuel-moisture-content}}
Effective Fuel Moisture Content is used for calculations of fuel
consumed, and is a function of the ratio of dead fuel moisture content
\(M_{df,fc}\) and the moisture of extinction factor,
\(m_{ef,fc}\)
\begin{equation*}
\begin{split}E_{moist,fc}=\frac{\emph{moist}_{fc}}{m_{ef,fc}}\end{split}
\end{equation*}
where the \(m_{ef}\) is a function of surface-area to volume ratio.
\begin{equation*}
\begin{split}m_{ef,fc}=0.524-0.066\log_{10}{\sigma_{fc}}\end{split}
\end{equation*}

\paragraph{Patch Fuel Moisture of Extinction}
\label{\detokenize{tech_note/DGVM/CLM50_Tech_Note_DGVM:patch-fuel-moisture-of-extinction}}
The patch ‘moisture of extinction’ factor (\(F_{mef}\)) is the
weighted average of the \(m_{ef}\) of the different fuel classes
\begin{equation*}
\begin{split}F_{mef,patch}=\sum_{fc=1}^{fc=5}  \frac{F_{fc}}{F_{tot}}m_{ef,fc}+\frac{b_{l,grass}}{F_{tot}}m_{ef,grass}\end{split}
\end{equation*}

\paragraph{Patch Fuel Bulk Density}
\label{\detokenize{tech_note/DGVM/CLM50_Tech_Note_DGVM:patch-fuel-bulk-density}}
The patch fuel bulk density is the weighted average of the bulk density
of the different fuel classes (except 1000-h fuels).
\begin{equation*}
\begin{split}F_{bd,patch}=\sum_{fc=1}^{fc=4} \frac{F_{fc}}{F_{tot}}\beta_{fuel,fc}+\frac{b_{l,grass}}{F_{tot}}\beta_{fuel,lgrass}\end{split}
\end{equation*}
where \(\beta_{fuel,fc}\) is the bulk density of each fuel size
class (kG m\(^{-3}\))


\paragraph{Patch Fuel Surface Area to Volume}
\label{\detokenize{tech_note/DGVM/CLM50_Tech_Note_DGVM:patch-fuel-surface-area-to-volume}}
The patch surface area to volume ratio (\(F_{\sigma}\)) is the
weighted average of the surface area to volume ratios
(\(\sigma_{fuel}\)) of the different fuel classes (except 1000-h
fuels).
\begin{equation*}
\begin{split}F_{\sigma}=\sum_{fc=1}^{fc=4}  \frac{F_{fc}}{F_{tot}}\sigma_{fuel,fc}+\frac{b_{l,grass}}{F_{tot}}\sigma_{fuel,grass}\end{split}
\end{equation*}

\subsubsection{Forward rate of spread}
\label{\detokenize{tech_note/DGVM/CLM50_Tech_Note_DGVM:forward-rate-of-spread}}
For each patch and each day, we calculate the rate of forward spread of
the fire \sphinxstyleemphasis{ros}\(_{f}\) (nominally in the direction of the wind).
\begin{equation*}
\begin{split}\emph{ros}_{f}=\frac{i_{r}x_{i}(1-\phi_{w})}{F_{bd,patch}e_{ps}q_{ig}}\end{split}
\end{equation*}
\(e_{ps}\) is the effective heating number
(\(e^{\frac{-4.528}{F_{\sigma,patch}}}\)). \(q_{ig}\) is the
heat of pre-ignition (\(581+2594F_{m}\)). \(x_{i}\) is the
propagating flux calculated as (see {\hyperref[\detokenize{tech_note/References/CLM50_Tech_Note_References:thonicke2010}]{\sphinxcrossref{\DUrole{std,std-ref}{Thonicke et al. 2010}}}}
Appendix A).
\begin{equation*}
\begin{split}x_{i}= \left\{ \begin{array}{ll}
0.0& \textrm{for $F_{\sigma,patch}<0.00001$}\\
\frac{e^{0.792+3.7597F_{\sigma,patch}^{0.5}(\frac{F_{bd,patch}}{p_{d}}+0.1)}}{192+7.9095F_{\sigma,patch}} & \textrm{for $F_{\sigma,patch}\geq 0.00001$}\\
\end{array} \right.\end{split}
\end{equation*}
\(\phi_{w}\) is the influence of windspeed on rate of spread.
\begin{equation*}
\begin{split}\phi_{w}=cb_{w}^{b}.\beta^{-e}\end{split}
\end{equation*}
Where \(b\), \(c\) and \(e\) are all functions of
surface-area-volume ratio \(F_{\sigma,patch}\):
\(b=0.15988F_{\sigma,patch}^{0.54}\),
\(c=7.47e^{-0.8711F_{\sigma,patch}^{0.55}}\),
\(e=0.715e^{-0.01094F_{\sigma,patch}}\). \(b_{w}=196.86W\) where
\(W\) is the the windspeed in ms\(^{-1}\), and
\(\beta=\frac{F_{bd}/p_{d}}{0.200395F_{\sigma,patch}^{-0.8189}}\)
where \(p_{d}\) is the particle density (513).

\(i_{r}\) is the reaction intensity, calculated using the following
set of expressions (from {\hyperref[\detokenize{tech_note/References/CLM50_Tech_Note_References:thonicke2010}]{\sphinxcrossref{\DUrole{std,std-ref}{Thonicke et al. 2010}}}} Appendix A).:
\begin{equation*}
\begin{split}\begin{aligned}
i_{r}&=&\Gamma_{opt}F_{tot}Hd_{moist}d_{miner}\\
d_{moist}&=&\textrm{max}\Big(0.0,(1-2.59m_{w}+5.11m_{w}^{2}-3.52m_{w}^{3})\Big)\\
m_{w}&=&\frac{F_{m,patch}}{F_{mef,patch}}\\
\Gamma _{opt}&=&\Gamma_{max}\beta^{a}\lambda\\
\Gamma _{max}&=&\frac{1}{0.0591+2.926F_{\sigma,patch}^{-1.5}}\\
\lambda&=&e^{a(1-\beta)}\\
a&=&8.9033F_{\sigma,patch}^{-0.7913}\end{aligned}\end{split}
\end{equation*}
\(\Gamma_{opt}\) is the residence time of the fire, and
\(d_{miner}\) is the mineral damping coefficient
(=0.174:math:\sphinxtitleref{S\_e\textasciicircum{}\{-0.19\}} , where \(S_e\) is 0.01 and so =
\(d_{miner}\) 0.41739).


\subsubsection{Fuel Consumption}
\label{\detokenize{tech_note/DGVM/CLM50_Tech_Note_DGVM:fuel-consumption}}
The fuel consumption (fraction of biomass pools) of each dead biomass
pool in the area affected by fire on a given day (\(f_{c,dead,fc}\))
is a function of effective fuel moisture \(E_{moist,fc}\) and size
class \sphinxstyleemphasis{fc} (Eqn B1, B4 and B5, {\hyperref[\detokenize{tech_note/References/CLM50_Tech_Note_References:thonicke2010}]{\sphinxcrossref{\DUrole{std,std-ref}{Thonicke et al. 2010}}}}). The
fraction of each fuel class that is consumed decreases as its moisture
content relative to its moisture of extinction (\(E_{moist,fc}\))
increases.
\begin{equation*}
\begin{split}f_{cdead,fc}=\textrm{max}\left(0,\textrm{min}(1,m_{int,mc,fc}-m_{slope,mc,fc}E_{moist,fc})\Big)\right.\end{split}
\end{equation*}
\(m_{int}\) and \(m_{slope}\) are parameters, the value of which
is modulated by both size class \(fc\) and by the effective fuel
moisture class \(mc\), defined by \(E_{moist,fc}\).
\(m_{int}\) and \(m_{slope}\) are defined for low-, mid-, and
high-moisture conditions, the boundaries of which are also functions of
the litter size class following {\hyperref[\detokenize{tech_note/References/CLM50_Tech_Note_References:peterson1986}]{\sphinxcrossref{\DUrole{std,std-ref}{Peterson and Ryan 1986}}}} (page
802). The fuel burned, \(f_{cground,fc}\) (Kg m\(^{-2}\)
day\(^{-1}\)) iscalculated from \(f_{cdead,fc}\) for each fuel
class:
\begin{equation*}
\begin{split}f_{cground,fc}=f_{c,dead,fc}(1-M_{f})\frac{F_{fc}}{0.45}\end{split}
\end{equation*}
Where 0.45 converts from carbon to biomass. The total fuel consumption,
\(f_{ctot,patch}\)(Kg m\(^{-2}\)), used to calculate fire
intensity, is then given by
\begin{equation*}
\begin{split}f_{ctot,patch}=\sum_{fc=1}^{fc=4} f_{c,ground,fc} +  f_{c,ground,lgrass}\end{split}
\end{equation*}
There is no contribution from the 1000 hour fuels to the patch-level
\(f_{ctot,patch}\) used in the fire intensity calculation.


\subsubsection{Fire Intensity}
\label{\detokenize{tech_note/DGVM/CLM50_Tech_Note_DGVM:fire-intensity}}
Fire intensity at the front of the burning area (\(I_{surface}\), kW
m\(^{-2}\)) is a function of the total fuel consumed
(\(f_{ctot,patch}\)) and the rate of spread at the front of the
fire, \(\mathit{ros}_{f}\) (m min\(^{-1}\)) (Eqn 15
{\hyperref[\detokenize{tech_note/References/CLM50_Tech_Note_References:thonicke2010}]{\sphinxcrossref{\DUrole{std,std-ref}{Thonicke et al. 2010}}}})
\begin{equation*}
\begin{split}I_{surface}=\frac{0.001}{60}f_{energy} f_{ctot,patch}\mathit{ros}_{f}\end{split}
\end{equation*}
where \(f_{energy}\) is the energy content of fuel (kJ/kG - the
same for alll fuel classes). Fire intensity is used to define whether an
ignition is successful. If the fire intensity is greater than 50kw/m
then the ignition is successful.


\subsubsection{Fire Duration}
\label{\detokenize{tech_note/DGVM/CLM50_Tech_Note_DGVM:fire-duration}}
Fire duration is a function of the fire danger index with a maximum
length of \(F_{dur,max}\) (240 minutes in
{\hyperref[\detokenize{tech_note/References/CLM50_Tech_Note_References:thonicke2010}]{\sphinxcrossref{\DUrole{std,std-ref}{Thonicke et al. 2010}}}} Eqn 14, derived from Canadian Forest
Fire Behaviour Predictions Systems)
\begin{equation*}
\begin{split}D_{f}=\textrm{min}\Big(F_{dur,max},\frac{F_{dur,max}}{1+F_{dur,max}e^{-11.06fdi}}\Big)\end{split}
\end{equation*}

\subsubsection{Fire Danger Index}
\label{\detokenize{tech_note/DGVM/CLM50_Tech_Note_DGVM:fire-danger-index}}
Fire danger index (\sphinxstyleemphasis{fdi}) is a representation of the effect of
meteorological conditions on the likelihood of a fire. It is calculated
for each gridcell as a function of the Nesterov Index .
\(\emph{fdi}\) is calculated from \(NI\) as
\begin{equation*}
\begin{split}\emph{fdi}=1-e^{\alpha N_{I}}\end{split}
\end{equation*}
where \(\alpha\) = 0.00037 following
{\hyperref[\detokenize{tech_note/References/CLM50_Tech_Note_References:venevsky2002}]{\sphinxcrossref{\DUrole{std,std-ref}{Venevsky et al. 2002}}}}.


\subsubsection{Area Burned}
\label{\detokenize{tech_note/DGVM/CLM50_Tech_Note_DGVM:area-burned}}
Total area burnt is assumed to be in the shape of an ellipse, whose
major axis \(f_{length}\) (m) is determined by the forward and
backward rates of spread (\(ros_{f}\) and \(ros_{b}\)
respectively).
\begin{equation*}
\begin{split}f_{length}=F_{d}(ros_{b}+ros_{f})\end{split}
\end{equation*}
\(ros_{b}\) is a function of \(ros_{f}\) and windspeed (Eqn 10
{\hyperref[\detokenize{tech_note/References/CLM50_Tech_Note_References:thonicke2010}]{\sphinxcrossref{\DUrole{std,std-ref}{Thonicke et al. 2010}}}})
\begin{equation*}
\begin{split}ros_{b}=ros_{f}e^{-0.72W}\end{split}
\end{equation*}
The minor axis to major axis ratio \(l_{b}\) of the ellipse is
determined by the windspeed. If the windspeed (\(W\)) is less than
16.67 ms\(^{-1}\) then \(l_{b}=1\). Otherwise (Eqn 12 and 13,
{\hyperref[\detokenize{tech_note/References/CLM50_Tech_Note_References:thonicke2010}]{\sphinxcrossref{\DUrole{std,std-ref}{Thonicke et al. 2010}}}})
\begin{equation*}
\begin{split}l_{b}=\textrm{min}\Big(8,f_{tree}(1.0+8.729(1.0-e^{-0.108W})^{2.155})+(f_{grass}(1.1+3.6W^{0.0464}))\Big)\end{split}
\end{equation*}
\(f_{grass}\) and \(f_{tree}\) are the fractions of the patch
surface covered by grass and trees respectively.

The total area burned (\(A_{burn}\) in m\(^{2}\)) is therefore
(Eqn 11, {\hyperref[\detokenize{tech_note/References/CLM50_Tech_Note_References:thonicke2010}]{\sphinxcrossref{\DUrole{std,std-ref}{Thonicke et al. 2010}}}})
\begin{equation*}
\begin{split}A_{burn}=\frac{n_{f}\frac{3.1416}{4l_{b}}(f_{length}^{2}))}{10000}\end{split}
\end{equation*}
where \(n_{f}\) is the number of fires.


\subsubsection{Crown Damage}
\label{\detokenize{tech_note/DGVM/CLM50_Tech_Note_DGVM:crown-damage}}
\(c_{k}\) is the fraction of the crown which is consumed by the
fire. This is calculated from scorch height \(H_{s}\), tree height
\(h\) and the crown fraction parameter \(F_{crown}\) (Eqn 17
{\hyperref[\detokenize{tech_note/References/CLM50_Tech_Note_References:thonicke2010}]{\sphinxcrossref{\DUrole{std,std-ref}{Thonicke et al. 2010}}}}):
\begin{equation*}
\begin{split}c_{k} = \left\{ \begin{array}{ll}
0 & \textrm{for $H_{s}<(h-hF_{crown})$}\\
1-\frac{h-H_{s}}{h-F_{crown}}& \textrm{for $h>H_{s}>(h-hF_{crown})$}\\
1 & \textrm{for $H_{s}>h$ }
\end{array} \right.\end{split}
\end{equation*}
The scorch height \(H_{s}\) (m) is a function of the fire intensity,
following {\hyperref[\detokenize{tech_note/References/CLM50_Tech_Note_References:byram1959}]{\sphinxcrossref{\DUrole{std,std-ref}{Byram, 1959}}}}, and is proportional to a plant
functional type specific parameter \(\alpha_{s,ft}\) (Eqn 16
{\hyperref[\detokenize{tech_note/References/CLM50_Tech_Note_References:thonicke2010}]{\sphinxcrossref{\DUrole{std,std-ref}{Thonicke et al. 2010}}}}):
\begin{equation*}
\begin{split}H_{s}=\sum_{FT=1}^{NPFT}{\alpha_{s,p}\cdot f_{biomass,ft}} I_{surface}^{0.667}\end{split}
\end{equation*}
where \(f_{biomass,ft}\) is the fraction of the above-ground
biomass in each plant functional type.


\subsubsection{Cambial Damage and Kill}
\label{\detokenize{tech_note/DGVM/CLM50_Tech_Note_DGVM:cambial-damage-and-kill}}
The cambial kill is a function of the fuel consumed \(f_{c,tot}\),
the bark thickness \(t_{b}\), and \(\tau_{l}\), the duration of
cambial heating (minutes) (Eqn 8, {\hyperref[\detokenize{tech_note/References/CLM50_Tech_Note_References:peterson1986}]{\sphinxcrossref{\DUrole{std,std-ref}{Peterson and Ryan 1986}}}}):
\begin{equation*}
\begin{split}\tau_{l}=\sum_{fc=1}^{fc=5}39.4F_{p,c}\frac{10000}{0.45}(1-(1-f_{c,dead,fc})^{0.5})\end{split}
\end{equation*}
Bark thickness is a linear function of tree diameter \(dbh_{coh}\),
defined by PFT-specific parameters \(\beta_{1,bt}\) and
\(\beta_{2,bt}\) (Eqn 21 {\hyperref[\detokenize{tech_note/References/CLM50_Tech_Note_References:thonicke2010}]{\sphinxcrossref{\DUrole{std,std-ref}{Thonicke et al. 2010}}}}):
\begin{equation*}
\begin{split}t_{b,coh}=\beta_{1,bt,ft}+\beta_{2,bt,ft}dbh_{coh}\end{split}
\end{equation*}
The critical time for cambial kill, \(\tau_{c}\) (minutes) is given
as (Eqn 20 {\hyperref[\detokenize{tech_note/References/CLM50_Tech_Note_References:thonicke2010}]{\sphinxcrossref{\DUrole{std,std-ref}{Thonicke et al. 2010}}}}):
\begin{equation*}
\begin{split}\tau_{c}=2.9t_{b}^{2}\end{split}
\end{equation*}
The mortality rate caused by cambial heating \(\tau_{pm}\) of trees
within the area affected by fire is a function of the ratio between
\(\tau_{l}\) and \(\tau_{c}\) (Eqn 19,
{\hyperref[\detokenize{tech_note/References/CLM50_Tech_Note_References:thonicke2010}]{\sphinxcrossref{\DUrole{std,std-ref}{Thonicke et al. 2010}}}}):
\begin{equation*}
\begin{split}\tau_{pm} = \left\{ \begin{array}{ll}
1.0 & \textrm{for } \tau_{1}/\tau_{c}\geq \textrm{2.0}\\
0.563(\tau_{l}/\tau_{c}))-0.125 & \textrm{for } \textrm{2.0} > \tau_{1}/\tau_{c}\ge \textrm{0.22}\\
0.0 & \textrm{for } \tau_{1}/\tau_{c}< \textrm{0.22}\\
\end{array} \right.\end{split}
\end{equation*}
\bigskip

\captionof{table}{Parameters needed for fire model.  }


\begin{savenotes}\sphinxattablestart
\centering
\begin{tabulary}{\linewidth}[t]{|T|T|T|T|}
\hline
\sphinxstylethead{\sphinxstyletheadfamily 
Parameter
Symbol
\unskip}\relax &\sphinxstylethead{\sphinxstyletheadfamily 
Parameter Name
\unskip}\relax &\sphinxstylethead{\sphinxstyletheadfamily 
Units
\unskip}\relax &\sphinxstylethead{\sphinxstyletheadfamily 
indexed by
\unskip}\relax \\
\hline
\(\beta_{1
,bt}\)
&
Intercept of
bark thickness
function
&
mm
&
\sphinxstyleemphasis{FT}
\\
\hline
\(\beta_{2
,bt}\)
&
Slope of bark
thickness
function
&
mm
cm\(^{-1
}\)
&
\sphinxstyleemphasis{FT}
\\
\hline
\(F_{crown
}\)
&
Ratio of crown
height to total
height
&
none
&
\sphinxstyleemphasis{FT}
\\
\hline
\(\alpha_{
fmc}\)
&
Fuel moisture
parameter
&
\({^o}\)C
\(^{-2}\)
&
\sphinxstyleemphasis{fc}
\\
\hline
\(\beta_{f
uel}\)
&
Fuel Bulk
Density
&
kG
m\(^{-3}\)
&
\sphinxstyleemphasis{fc}
\\
\hline
\(\sigma_{
fuel,fc}\)
&
Surface area to
volume ratio
&
cm
\(^{-1}\)
&
\sphinxstyleemphasis{fc}
\\
\hline
\(m_{int}\)
&
Intercept of
fuel burned
&
none
&
\(fc\),
moisture class
\\
\hline
\(m_{slope
}\)
&
Slope of fuel
burned
&
none
&
\(fc\),
moisture class
\\
\hline
\(M_f\)
&
Fuel Mineral
Fraction
&&\\
\hline
\(F_{dur,m
ax}\)
&
Maximum
Duration of
Fire
&
Minutes
&\\
\hline
\(f_{energ
y}\)
&
Energy content
of fuel
&
kJ/kG
&\\
\hline
\(\alpha_{
s}\)
&
Flame height
parameter
&&
\sphinxstyleemphasis{FT}
\\
\hline
\end{tabulary}
\par
\sphinxattableend\end{savenotes}

\bigskip

\begin{DUlineblock}{0em}
\item[] 
\end{DUlineblock}


\section{Biogenic Volatile Organic Compounds (BVOCs)}
\label{\detokenize{tech_note/BVOCs/CLM50_Tech_Note_BVOCs:rst-biogenic-volatile-organic-compounds-bvocs}}\label{\detokenize{tech_note/BVOCs/CLM50_Tech_Note_BVOCs::doc}}\label{\detokenize{tech_note/BVOCs/CLM50_Tech_Note_BVOCs:biogenic-volatile-organic-compounds-bvocs}}
This chapter briefly describes the biogenic volatile organic compound
(BVOC) emissions model implemented in CLM. The CLM3 version (Levis et
al. 2003; Oleson et al. 2004) was based on Guenther et al. (1995). Heald
et al. (2008) updated this scheme in CLM4 based on Guenther et al.
(2006). The current version was implemented in CLM4.5 and is based on MEGAN2.1 discussed in
detail in Guenther et al. (2012). This update of MEGAN incorporates four
main features: 1) expansion to 147 chemical compounds, 2) the treatment of the
light-dependent fraction (LDF) for each compound, 3) inclusion of the
inhibition of isoprene emission by atmospheric CO$_{\text{2}}$ and
4) emission factors mapped to the specific PFTs of the CLM.

MEGAN2.1 now describes the emissions of speciated monoterpenes,
sesquiterpenes, oxygenated VOCs as well as isoprene. A flexible scheme
has been implemented in the CLM to specify a subset of emissions. This
allows for additional flexibility in grouping chemical compounds to form
the lumped species frequently used in atmospheric chemistry. The mapping
or grouping is therefore defined through a namelist parameter in
drv\_flds\_in, e.g. megan\_specifier = ’ISOP = isoprene’, ’BIGALK =
pentane + hexane + heptane + tricyclene’.

Terrestrial BVOC emissions from plants to the atmosphere are expressed
as a flux, \(F_{i}\) (\(\mu\) g C m$^{\text{-2}}$ ground area h$^{\text{-1}}$), for emission of chemical compound
\(i\)
\phantomsection\label{\detokenize{tech_note/BVOCs/CLM50_Tech_Note_BVOCs:equation-ZEqnNum964222}}\begin{equation}\label{equation:tech_note/BVOCs/CLM50_Tech_Note_BVOCs:ZEqnNum964222}
\begin{split}F_{i} =\gamma _{i} \rho \sum _{j}\varepsilon _{i,j}  \left(wt\right)_{j}\end{split}
\end{equation}
where \(\gamma _{i}\)  is the emission activity factor accounting
for responses to meteorological and phenological conditions,
\(\rho\)  is the canopy loss and production factor also known as
escape efficiency (set to 1), and \(\varepsilon _{i,\, j}\)
(\(\mu\) g C m$^{\text{-2}}$ ground area h$^{\text{-1}}$) is
the emission factor at standard conditions of light, temperature, and
leaf area for plant functional type \sphinxstyleemphasis{j} with fractional coverage
\(\left(wt\right)_{j}\)  (Guenther et al. 2012). The emission
activity factor \(\gamma _{i}\)  depends on plant functional type,
temperature, LAI, leaf age, and soil moisture (Guenther et al. 2012).
For isoprene only, the effect of CO$_{\text{2}}$ inhibition is now
included as described by Heald et al. (2009). Previously, only isoprene
was treated as a light-dependent emission. In MEGAN2.1, each chemical
compound is assigned a LDF (ranging from 1.0 for isoprene to 0.2 for
some monoterpenes, VOCs and acetone). The activity factor for the light
response of emissions is therefore estimated as:
\phantomsection\label{\detokenize{tech_note/BVOCs/CLM50_Tech_Note_BVOCs:equation-28.2)}}\begin{equation}\label{equation:tech_note/BVOCs/CLM50_Tech_Note_BVOCs:28.2)}
\begin{split}\gamma _{P,\, i} =\left(1-LDF_{i} \right)+\gamma _{P\_ LDF} LDF_{i}\end{split}
\end{equation}
where the LDF activity factor (\(\gamma _{P\_ LDF}\) ) is specified
as a function of PAR as in previous versions of MEGAN.

The values for each emission factor \(\epsilon _{i,\, j}\)  are
now available for each of the plant functional types in the CLM and
each chemical compound. This information is distributed through an
external file, allowing for more frequent and easier updates.


\section{Dust Model}
\label{\detokenize{tech_note/Dust/CLM50_Tech_Note_Dust:rst-dust-model}}\label{\detokenize{tech_note/Dust/CLM50_Tech_Note_Dust::doc}}\label{\detokenize{tech_note/Dust/CLM50_Tech_Note_Dust:dust-model}}
Atmospheric dust is mobilized from the land by wind in the CLM. The most
important factors determining soil erodibility and dust emission include
the wind friction speed, the vegetation cover, and the soil moisture.
The CLM dust mobilization scheme ({\hyperref[\detokenize{tech_note/References/CLM50_Tech_Note_References:mahowaldetal2006}]{\sphinxcrossref{\DUrole{std,std-ref}{Mahowald et al. 2006}}}})
accounts for these factors based on the DEAD (Dust Entrainment and Deposition)
model of {\hyperref[\detokenize{tech_note/References/CLM50_Tech_Note_References:zenderetal2003}]{\sphinxcrossref{\DUrole{std,std-ref}{Zender et al. (2003)}}}}. Please refer to the
{\hyperref[\detokenize{tech_note/References/CLM50_Tech_Note_References:zenderetal2003}]{\sphinxcrossref{\DUrole{std,std-ref}{Zender et al. (2003)}}}} article for additional
information regarding the equations presented in this section.

The total vertical mass flux of dust, \(F_{j}\)
(kg m$^{\text{-2}}$ s$^{\text{-1}}$), from the ground into transport bin
\(j\) is given by
\phantomsection\label{\detokenize{tech_note/Dust/CLM50_Tech_Note_Dust:equation-29.1}}\begin{equation}\label{equation:tech_note/Dust/CLM50_Tech_Note_Dust:29.1}
\begin{split}F_{j} =TSf_{m} \alpha Q_{s} \sum _{i=1}^{I}M_{i,j}\end{split}
\end{equation}
where \(T\) is a global factor that compensates for the DEAD model’s
sensitivity to horizontal and temporal resolution and equals 5 x
10$^{\text{-4}}$ in the CLM instead of 7 x 10$^{\text{-4}}$ in
{\hyperref[\detokenize{tech_note/References/CLM50_Tech_Note_References:zenderetal2003}]{\sphinxcrossref{\DUrole{std,std-ref}{Zender et al. (2003)}}}}. \(S\) is the source
erodibility factor set to 1 in the CLM and serves as a place holder
at this time.

The grid cell fraction of exposed bare soil suitable for dust
mobilization \(f_{m}\)  is given by
\phantomsection\label{\detokenize{tech_note/Dust/CLM50_Tech_Note_Dust:equation-29.2}}\begin{equation}\label{equation:tech_note/Dust/CLM50_Tech_Note_Dust:29.2}
\begin{split}f_{m} =\left(1-f_{lake} -f_{wetl} \right)\left(1-f_{sno} \right)\left(1-f_{v} \right)\frac{w_{liq,1} }{w_{liq,1} +w_{ice,1} }\end{split}
\end{equation}
where \(f_{lake}\)  and \(f_{wetl}\)  and \(f_{sno}\)
are the CLM grid cell fractions of lake and wetland (section
\hyperref[\detokenize{tech_note/Ecosystem/CLM50_Tech_Note_Ecosystem:surface-data}]{\ref{\detokenize{tech_note/Ecosystem/CLM50_Tech_Note_Ecosystem:surface-data}}}) and snow cover (section
\hyperref[\detokenize{tech_note/Snow_Hydrology/CLM50_Tech_Note_Snow_Hydrology:snow-covered-area-fraction}]{\ref{\detokenize{tech_note/Snow_Hydrology/CLM50_Tech_Note_Snow_Hydrology:snow-covered-area-fraction}}}), all ranging from zero to one. Not mentioned
by {\hyperref[\detokenize{tech_note/References/CLM50_Tech_Note_References:zenderetal2003}]{\sphinxcrossref{\DUrole{std,std-ref}{Zender et al. (2003)}}}}, \(w_{liq,\, 1}\)  and
\({}_{w_{ice,\, 1} }\) are the CLM top soil layer liquid water and
ice contents (mm) entered as a ratio expressing the decreasing ability
of dust to mobilize from increasingly frozen soil. The grid cell
fraction of vegetation cover,\({}_{f_{v} }\), is defined as
\phantomsection\label{\detokenize{tech_note/Dust/CLM50_Tech_Note_Dust:equation-29.3}}\begin{equation}\label{equation:tech_note/Dust/CLM50_Tech_Note_Dust:29.3}
\begin{split}0\le f_{v} =\frac{L+S}{\left(L+S\right)_{t} } \le 1{\rm \; \; \; \; where\; }\left(L+S\right)_{t} =0.3{\rm \; m}^{2} {\rm m}^{-2}\end{split}
\end{equation}
where equation applies only for dust mobilization and is not related to
the plant functional type fractions prescribed from the CLM input data
or simulated by the CLM dynamic vegetation model (Chapter 22). \(L\)
and \(S\) are the CLM leaf and stem area index values
(m $^{\text{2}}$ m$^{\text{-2}}$) averaged at the land unit level so
as to include all the pfts and the bare ground present in a vegetated
land unit. \(L\) and \(S\) may be prescribed from the CLM
input data (section \hyperref[\detokenize{tech_note/Ecosystem/CLM50_Tech_Note_Ecosystem:phenology-and-vegetation-burial-by-snow}]{\ref{\detokenize{tech_note/Ecosystem/CLM50_Tech_Note_Ecosystem:phenology-and-vegetation-burial-by-snow}}})
or simulated by the CLM biogeochemistry model (Chapter
\hyperref[\detokenize{tech_note/Vegetation_Phenology_Turnover/CLM50_Tech_Note_Vegetation_Phenology_Turnover:rst-vegetation-phenology-and-turnover}]{\ref{\detokenize{tech_note/Vegetation_Phenology_Turnover/CLM50_Tech_Note_Vegetation_Phenology_Turnover:rst-vegetation-phenology-and-turnover}}}).

The sandblasting mass efficiency \(\alpha\)  (m $^{\text{-1}}$) is
calculated as
\phantomsection\label{\detokenize{tech_note/Dust/CLM50_Tech_Note_Dust:equation-29.4}}\begin{equation}\label{equation:tech_note/Dust/CLM50_Tech_Note_Dust:29.4}
\begin{split}\alpha =100e^{\left(13.4M_{clay} -6.0\right)\ln 10} {\rm \; \; }\left\{\begin{array}{l} {M_{clay} =\% clay\times 0.01{\rm \; \; \; 0}\le \% clay\le 20} \\ {M_{clay} =20\times 0.01{\rm \; \; \; \; \; \; \; \; 20<\% }clay\le 100} \end{array}\right.\end{split}
\end{equation}
where \(M_{clay}\) is the mass fraction of clay
particles in the soil and \%clay is determined from the surface dataset
(section \hyperref[\detokenize{tech_note/Ecosystem/CLM50_Tech_Note_Ecosystem:surface-data}]{\ref{\detokenize{tech_note/Ecosystem/CLM50_Tech_Note_Ecosystem:surface-data}}}). \(M_{clay} =0\) corresponds to sand and
\(M_{clay} =0.2\) to sandy loam.

\(Q_{s}\)  is the total horizontally saltating mass flux (kg
m$^{\text{-1}}$ s$^{\text{-1}}$) of “large” particles (\hyperref[\detokenize{tech_note/Dust/CLM50_Tech_Note_Dust:table-dust-mass-fraction}]{Table \ref{\detokenize{tech_note/Dust/CLM50_Tech_Note_Dust:table-dust-mass-fraction}}}),
also referred to as the vertically integrated streamwise mass flux
\phantomsection\label{\detokenize{tech_note/Dust/CLM50_Tech_Note_Dust:equation-29.5}}\begin{equation}\label{equation:tech_note/Dust/CLM50_Tech_Note_Dust:29.5}
\begin{split}Q_{s} = \left\{
\begin{array}{lr}
\frac{c_{s} \rho _{atm} u_{*s}^{3} }{g} \left(1-\frac{u_{*t} }{u_{*s} } \right)\left(1+\frac{u_{*t} }{u_{*s} } \right)^{2} {\rm \; } & \qquad {\rm for\; }u_{*t} <u_{*s}  \\
0{\rm \; \; \; \; \; \; \; \; \; \; \; \; \; \; \; \; \; \; \; \; \; \; \; \; \; \; \; \; \; \; \; \; \; \; \; \; \; \; \; \; } & \qquad {\rm for\; }u_{*t} \ge u_{*s}
\end{array}\right.\end{split}
\end{equation}
where the saltation constant \(c_{s}\) equals 2.61 and
\(\rho _{atm}\)  is the atmospheric density (kg m$^{\text{-3}}$)
(\hyperref[\detokenize{tech_note/Ecosystem/CLM50_Tech_Note_Ecosystem:table-atmospheric-input-to-land-model}]{Table \ref{\detokenize{tech_note/Ecosystem/CLM50_Tech_Note_Ecosystem:table-atmospheric-input-to-land-model}}}), \(g\) the acceleration of gravity (m
s$^{\text{-2}}$) (\hyperref[\detokenize{tech_note/Ecosystem/CLM50_Tech_Note_Ecosystem:table-physical-constants}]{Table \ref{\detokenize{tech_note/Ecosystem/CLM50_Tech_Note_Ecosystem:table-physical-constants}}}). The threshold wind friction speed for saltation \(u_{*t}\)  (m s$^{\text{-1}}$) is
\phantomsection\label{\detokenize{tech_note/Dust/CLM50_Tech_Note_Dust:equation-29.6}}\begin{equation}\label{equation:tech_note/Dust/CLM50_Tech_Note_Dust:29.6}
\begin{split}u_{*t} =f_{z} \left[Re_{*t}^{f} \rho _{osp} gD_{osp} \left(1+\frac{6\times 10^{-7} }{\rho _{osp} gD_{osp}^{2.5} } \right)\right]^{\frac{1}{2} } \rho _{atm} ^{-\frac{1}{2} } f_{w}\end{split}
\end{equation}
where \(f_{z}\)  is a factor dependent on surface roughness but set
to 1 as a place holder for now, \(\rho _{osp}\)  and
\(D_{osp}\)  are the density (2650 kg m$^{\text{-3}}$) and
diameter (75 x 10\({}^{-6}\) m) of optimal saltation particles,
and \(f_{w}\)  is a factor dependent on soil moisture:
\phantomsection\label{\detokenize{tech_note/Dust/CLM50_Tech_Note_Dust:equation-29.7}}\begin{equation}\label{equation:tech_note/Dust/CLM50_Tech_Note_Dust:29.7}
\begin{split}f_{w} =\left\{\begin{array}{l} {1{\rm \; \; \; \; \; \; \; \; \; \; \; \; \; \; \; \; \; \; \; \; \; \; \; \; \; \; \; \; \; \; \; \; \; \; \; \; \; \; \; \; \; \; for\; }w\le w_{t} } \\ {\sqrt{1+1.21\left[100\left(w-w_{t} \right)\right]^{0.68} } {\rm \; \; for\; }w>w_{t} } \end{array}\right.\end{split}
\end{equation}
where
\phantomsection\label{\detokenize{tech_note/Dust/CLM50_Tech_Note_Dust:equation-29.8}}\begin{equation}\label{equation:tech_note/Dust/CLM50_Tech_Note_Dust:29.8}
\begin{split}w_{t} =a\left(0.17M_{clay} +0.14M_{clay}^{2} \right){\rm \; \; \; \; \; \; 0}\le M_{clay} =\% clay\times 0.01\le 1\end{split}
\end{equation}
and
\phantomsection\label{\detokenize{tech_note/Dust/CLM50_Tech_Note_Dust:equation-29.9}}\begin{equation}\label{equation:tech_note/Dust/CLM50_Tech_Note_Dust:29.9}
\begin{split}w=\frac{\theta _{1} \rho _{liq} }{\rho _{d,1} }\end{split}
\end{equation}
where \(a=M_{clay}^{-1}\)  for tuning purposes,
\(\theta _{1}\)  is the volumetric soil moisture in the top soil
layer (m \({}^{3 }\)m$^{\text{-3}}$) (section \hyperref[\detokenize{tech_note/Hydrology/CLM50_Tech_Note_Hydrology:soil-water}]{\ref{\detokenize{tech_note/Hydrology/CLM50_Tech_Note_Hydrology:soil-water}}}),
\(\rho _{liq}\)  is the density of liquid water (kg
m$^{\text{-3}}$) (\hyperref[\detokenize{tech_note/Ecosystem/CLM50_Tech_Note_Ecosystem:table-physical-constants}]{Table \ref{\detokenize{tech_note/Ecosystem/CLM50_Tech_Note_Ecosystem:table-physical-constants}}}), and \(\rho _{d,\, 1}\)
is the bulk density of soil in the top soil layer (kg m$^{\text{-3}}$)
defined as in section \hyperref[\detokenize{tech_note/Soil_Snow_Temperatures/CLM50_Tech_Note_Soil_Snow_Temperatures:soil-and-snow-thermal-properties}]{\ref{\detokenize{tech_note/Soil_Snow_Temperatures/CLM50_Tech_Note_Soil_Snow_Temperatures:soil-and-snow-thermal-properties}}}
rather than as in {\hyperref[\detokenize{tech_note/References/CLM50_Tech_Note_References:zenderetal2003}]{\sphinxcrossref{\DUrole{std,std-ref}{Zender et al. (2003)}}}}.
\(Re_{*t}^{f}\)  from equation is the threshold friction Reynolds
factor
\phantomsection\label{\detokenize{tech_note/Dust/CLM50_Tech_Note_Dust:equation-29.10}}\begin{equation}\label{equation:tech_note/Dust/CLM50_Tech_Note_Dust:29.10}
\begin{split}Re_{*t}^{f} =\left\{\begin{array}{l} {\frac{0.1291^{2} }{-1+1.928Re_{*t} } {\rm \; \; \; \; \; \; \; \; \; \; \; \; \; \; \; \; \; \; \; \; \; \; \; \; \; \; for\; 0.03}\le Re_{*t} \le 10} \\ {0.12^{2} \left(1-0.0858e^{-0.0617(Re_{*t} -10)} \right)^{2} {\rm \; for\; }Re_{*t} >10} \end{array}\right.\end{split}
\end{equation}
and \(Re_{*t}\)  is the threshold friction Reynolds number
approximation for optimally sized particles
\phantomsection\label{\detokenize{tech_note/Dust/CLM50_Tech_Note_Dust:equation-29.11}}\begin{equation}\label{equation:tech_note/Dust/CLM50_Tech_Note_Dust:29.11}
\begin{split}Re_{*t} =0.38+1331\left(100D_{osp} \right)^{1.56}\end{split}
\end{equation}
In \eqref{equation:tech_note/Dust/CLM50_Tech_Note_Dust:29.5} , \(u_{*s}\) is defined as the wind friction speed
(m s$^{\text{-1}}$) accounting for the Owen effect ({\hyperref[\detokenize{tech_note/References/CLM50_Tech_Note_References:owen1964}]{\sphinxcrossref{\DUrole{std,std-ref}{Owen 1964}}}})
\phantomsection\label{\detokenize{tech_note/Dust/CLM50_Tech_Note_Dust:equation-29.12}}\begin{equation}\label{equation:tech_note/Dust/CLM50_Tech_Note_Dust:29.12}
\begin{split}u_{\*s} = \left\{
\begin{array}{lr}
u_{\*} & \quad {\rm \; for \;} U_{10} <U_{10,t}  \\
u_{*} +0.003\left(U_{10} -U_{10,t} \right)^{2} & \quad {\rm \; for\; }U_{10} \ge U_{10,t}
\end{array}\right.\end{split}
\end{equation}
where \(u_{*}\)  is the CLM wind friction speed (m s$^{\text{-1}}$),
also known as friction velocity (section \hyperref[\detokenize{tech_note/Fluxes/CLM50_Tech_Note_Fluxes:monin-obukhov-similarity-theory}]{\ref{\detokenize{tech_note/Fluxes/CLM50_Tech_Note_Fluxes:monin-obukhov-similarity-theory}}}),
\(U_{10}\) is the 10-m wind speed (m s$^{\text{-1}}$)
calculated as the wind speed at the top of the canopy in section 4.3 of
{\hyperref[\detokenize{tech_note/References/CLM50_Tech_Note_References:bonan1996}]{\sphinxcrossref{\DUrole{std,std-ref}{Bonan (1996)}}}} but here for 10 m above the ground, and
\(U_{10,\, t}\)  is the threshold wind speed at 10 m (m
s$^{\text{-1}}$)
\phantomsection\label{\detokenize{tech_note/Dust/CLM50_Tech_Note_Dust:equation-29.13}}\begin{equation}\label{equation:tech_note/Dust/CLM50_Tech_Note_Dust:29.13}
\begin{split}U_{10,t} =u_{*t} \frac{U_{10} }{u_{*} }\end{split}
\end{equation}
In equation we sum \(M_{i,\, j}\)  over \(I=3\) source modes
\(i\) where \(M_{i,\, j}\)  is the mass fraction of each source
mode \(i\) carried in each of \sphinxstyleemphasis{:math:{}`J=4{}`} transport bins \(j\)
\phantomsection\label{\detokenize{tech_note/Dust/CLM50_Tech_Note_Dust:equation-29.14}}\begin{equation}\label{equation:tech_note/Dust/CLM50_Tech_Note_Dust:29.14}
\begin{split}M_{i,j} =\frac{m_{i} }{2} \left[{\rm erf}\left(\frac{\ln {\textstyle\frac{D_{j,\max } }{\tilde{D}_{v,i} }} }{\sqrt{2} \ln \sigma _{g,i} } \right)-{\rm erf}\left(\frac{\ln {\textstyle\frac{D_{j,\min } }{\tilde{D}_{v,i} }} }{\sqrt{2} \ln \sigma _{g,i} } \right)\right]\end{split}
\end{equation}
where \(m_{i}\) , \(\tilde{D}_{v,\, i}\) , and
\(\sigma _{g,\, i}\)  are the mass fraction, mass median diameter,
and geometric standard deviation assigned to each particle source mode
\(i\) (\hyperref[\detokenize{tech_note/Dust/CLM50_Tech_Note_Dust:table-dust-mass-fraction}]{Table \ref{\detokenize{tech_note/Dust/CLM50_Tech_Note_Dust:table-dust-mass-fraction}}}), while \(D_{j,\, \min }\)  and
\(D_{j,\, \max }\)  are the minimum and maximum diameters (m) in
each transport bin \(j\) (\hyperref[\detokenize{tech_note/Dust/CLM50_Tech_Note_Dust:table-dust-minimum-and-maximum-particle-diameters}]{Table \ref{\detokenize{tech_note/Dust/CLM50_Tech_Note_Dust:table-dust-minimum-and-maximum-particle-diameters}}}).


\begin{savenotes}\sphinxattablestart
\centering
\sphinxcapstartof{table}
\sphinxcaption{Mass fraction \(m_{i}\) , mass median diameter \(\tilde{D}_{v,\, i}\) , and geometric standard deviation \(\sigma _{g,\, i}\) , per dust source mode \(i\)}\label{\detokenize{tech_note/Dust/CLM50_Tech_Note_Dust:table-dust-mass-fraction}}\label{\detokenize{tech_note/Dust/CLM50_Tech_Note_Dust:id1}}
\sphinxaftercaption
\begin{tabulary}{\linewidth}[t]{|T|T|T|T|}
\hline

\(i\)
&\sphinxstylethead{\sphinxstyletheadfamily 
\(m_{i}\)  (fraction)
\unskip}\relax &\sphinxstylethead{\sphinxstyletheadfamily 
\(\tilde{D}_{v,\, i}\)  (m)
\unskip}\relax &
\(\sigma _{g,\, i}\)
\\
\hline
1
&
0.036
&
0.832 x 10\({}^{-6}\)
&
2.1
\\
\hline
2
&
0.957
&
4.820 x 10\({}^{-6}\)
&
1.9
\\
\hline
3
&
0.007
&
19.38 x 10\({}^{-6}\)
&
1.6
\\
\hline
\end{tabulary}
\par
\sphinxattableend\end{savenotes}


\begin{savenotes}\sphinxattablestart
\centering
\sphinxcapstartof{table}
\sphinxcaption{Minimum and maximum particle diameters in each dust transport bin \(j\)}\label{\detokenize{tech_note/Dust/CLM50_Tech_Note_Dust:table-dust-minimum-and-maximum-particle-diameters}}\label{\detokenize{tech_note/Dust/CLM50_Tech_Note_Dust:id2}}
\sphinxaftercaption
\begin{tabulary}{\linewidth}[t]{|T|T|T|}
\hline

\(j\)
&\sphinxstylethead{\sphinxstyletheadfamily 
\(D_{j,\, \min }\)  (m)
\unskip}\relax &\sphinxstylethead{\sphinxstyletheadfamily 
\(D_{j,\, \max }\)  (m)
\unskip}\relax \\
\hline
1
&
0.1 x 10\({}^{-6}\)
&
1.0 x 10\({}^{-6}\)
\\
\hline
2
&
1.0 x 10\({}^{-6}\)
&
2.5 x 10\({}^{-6}\)
\\
\hline
3
&
2.5 x 10\({}^{-6}\)
&
5.0 x 10\({}^{-6}\)
\\
\hline
4
&
5.0 x 10\({}^{-6}\)
&
10.0 x 10\({}^{-6}\)
\\
\hline
\end{tabulary}
\par
\sphinxattableend\end{savenotes}


\section{Carbon Isotopes}
\label{\detokenize{tech_note/Isotopes/CLM50_Tech_Note_Isotopes:carbon-isotopes}}\label{\detokenize{tech_note/Isotopes/CLM50_Tech_Note_Isotopes::doc}}\label{\detokenize{tech_note/Isotopes/CLM50_Tech_Note_Isotopes:rst-carbon-isotopes}}
CLM includes a fully prognostic representation of the fluxes, storage,
and isotopic discrimination of the carbon isotopes $^{\text{13}}$C
and $^{\text{14}}$C. The implementation of the C isotopes capability
takes advantage of the CLM hierarchical data structures, replicating the
carbon state and flux variable structures at the column and PFT level to
track total carbon and both C isotopes separately (see description of
data structure hierarchy in Chapter 2). For the most part, fluxes and
associated updates to carbon state variables for $^{\text{13}}$C are
calculated directly from the corresponding total C fluxes. Separate
calculations are required in a few special cases, such as where isotopic
discrimination occurs, or where the necessary isotopic ratios are
undefined. The general approach for $^{\text{13}}$C flux and state
variable calculation is described here, followed by a description of all
the places where special calculations are required.


\subsection{General Form for Calculating $^{\text{13}}$C and $^{\text{14}}$C Flux}
\label{\detokenize{tech_note/Isotopes/CLM50_Tech_Note_Isotopes:general-form-for-calculating-13c-and-14c-flux}}
In general, the flux of $^{\text{13}}$C and corresponding to a given
flux of total C (\({CF}_{13C}\) and \({CF}_{totC}\),
respectively) is determined by \({CF}_{totC}\), the masses of
$^{\text{13}}$C and total C in the upstream pools
(\({CS}_{13C\_up}\) and \({CS}_{totC\_up}\),
respectively, i.e. the pools \sphinxstyleemphasis{from which} the fluxes of
$^{\text{13}}$C and total C originate), and a fractionation factor,
\({f}_{frac}\):
\phantomsection\label{\detokenize{tech_note/Isotopes/CLM50_Tech_Note_Isotopes:equation-ZEqnNum629812}}\begin{equation}\label{equation:tech_note/Isotopes/CLM50_Tech_Note_Isotopes:ZEqnNum629812}
\begin{split}CF_{13C} =\left\{\begin{array}{l} {CF_{totC} \frac{CS_{13C\_ up} }{CS_{totC\_ up} } f_{frac} \qquad {\rm for\; }CS_{totC} \ne 0} \\ {0\qquad {\rm for\; }CS_{totC} =0} \end{array}\right\}\end{split}
\end{equation}
If the \({f}_{frac}\) = 1.0 (no fractionation), then the fluxes
\({CF}_{13C}\) and  \({CF}_{totC}\) will be in simple
proportion to the masses \({CS}_{13C\_up}\) and
\({CS}_{totC\_up}\). Values of \({f}_{frac} < 1.0\) indicate a discrimination against the heavier isotope
($^{\text{13}}$C) in the flux-generating process, while
\({f}_{frac}\) \(>\) 1.0 would indicate a preference for the
heavier isotope. Currently, in all cases where Eq. is used to calculate
a $^{\text{13}}$C flux, \({f}_{frac}\) is set to 1.0.

For $^{\text{1}}$$^{\text{4}}$C, no fractionation is used in
either the initial photosynthetic step, nor in subsequent fluxes from
upstream to downstream pools; as discussed below, this is because
observations of $^{\text{1}}$$^{\text{4}}$C are typically
described in units that implicitly correct out the fractionation of
$^{\text{1}}$$^{\text{4}}$C by referencing them to
$^{\text{1}}$$^{\text{3}}$C ratios.


\subsection{Isotope Symbols, Units, and Reference Standards}
\label{\detokenize{tech_note/Isotopes/CLM50_Tech_Note_Isotopes:isotope-symbols-units-and-reference-standards}}
Carbon has two primary stable isotopes, $^{\text{12}}$C and
$^{\text{13}}$C. $^{\text{12}}$C is the most abundant, comprising
about 99\% of all carbon. The isotope ratio of a compound,
\({R}_{A}\), is the mass ratio of the rare isotope to the abundant isotope
\phantomsection\label{\detokenize{tech_note/Isotopes/CLM50_Tech_Note_Isotopes:equation-30.2)}}\begin{equation}\label{equation:tech_note/Isotopes/CLM50_Tech_Note_Isotopes:30.2)}
\begin{split}R_{A} =\frac{{}^{13} C_{A} }{{}^{12} C_{A} } .\end{split}
\end{equation}
Carbon isotope ratios are often expressed using delta notation,
\(\delta\). The \(\delta^{13}\)C value of a
compound A, \(\delta^{13}\)C$_{\text{A}}$, is the
difference between the isotope ratio of the compound,
\({R}_{A}\), and that of the Pee Dee Belemnite standard, \({R}_{PDB}\), in parts per thousand
\phantomsection\label{\detokenize{tech_note/Isotopes/CLM50_Tech_Note_Isotopes:equation-30.3)}}\begin{equation}\label{equation:tech_note/Isotopes/CLM50_Tech_Note_Isotopes:30.3)}
\begin{split}\delta ^{13} C_{A} =\left(\frac{R_{A} }{R_{PDB} } -1\right)\times 1000\end{split}
\end{equation}
where \({R}_{PDB}\) = 0.0112372, and units of \(\delta\) are per mil (‰).

Isotopic fractionation can be expressed in several ways. One expression
of the fractionation factor is with alpha (\(\alpha\)) notation.
For example, the equilibrium fractionation between two reservoirs A and
B can be written as:
\phantomsection\label{\detokenize{tech_note/Isotopes/CLM50_Tech_Note_Isotopes:equation-30.4)}}\begin{equation}\label{equation:tech_note/Isotopes/CLM50_Tech_Note_Isotopes:30.4)}
\begin{split}\alpha _{A-B} =\frac{R_{A} }{R_{B} } =\frac{\delta _{A} +1000}{\delta _{B} +1000} .\end{split}
\end{equation}
This can also be expressed using epsilon notation (\(\epsilon\)), where
\phantomsection\label{\detokenize{tech_note/Isotopes/CLM50_Tech_Note_Isotopes:equation-30.5)}}\begin{equation}\label{equation:tech_note/Isotopes/CLM50_Tech_Note_Isotopes:30.5)}
\begin{split}\alpha _{A-B} =\frac{\varepsilon _{A-B} }{1000} +1\end{split}
\end{equation}
In other words, if \({\epsilon }_{A-B} = 4.4\) ‰ , then \({\alpha}_{A-B} =1.0044\).

In addition to the stable isotopes $^{\text{1}}$$^{\text{2}}$C and $^{\text{1}}$$^{\text{3}}$C, the unstable isotope
$^{\text{1}}$$^{\text{4}}$C is included in CLM. $^{\text{1}}$$^{\text{4}}$C can also be described using the delta notation:
\phantomsection\label{\detokenize{tech_note/Isotopes/CLM50_Tech_Note_Isotopes:equation-30.6)}}\begin{equation}\label{equation:tech_note/Isotopes/CLM50_Tech_Note_Isotopes:30.6)}
\begin{split}\delta ^{14} C=\left(\frac{A_{s} }{A_{abs} } -1\right)\times 1000\end{split}
\end{equation}
However, observations of $^{\text{1}}$$^{\text{4}}$C are typically
fractionation-corrected using the following notation:
\phantomsection\label{\detokenize{tech_note/Isotopes/CLM50_Tech_Note_Isotopes:equation-30.7)}}\begin{equation}\label{equation:tech_note/Isotopes/CLM50_Tech_Note_Isotopes:30.7)}
\begin{split}\Delta {}^{14} C=1000\times \left(\left(1+\frac{\delta {}^{14} C}{1000} \right)\frac{0.975^{2} }{\left(1+\frac{\delta {}^{13} C}{1000} \right)^{2} } -1\right)\end{split}
\end{equation}
where \(\delta^{14}\)C is the measured isotopic
fraction and \(\mathrm{\Delta}^{14}\)C corrects for
mass-dependent isotopic fractionation processes (assumed to be 0.975 for
fractionation of $^{\text{13}}$C by photosynthesis). CLM assumes a
background preindustrial atmospheric $^{\text{14}}$C /C ratio of 10$^{\text{-12}}$, which is used for A:sub::\sphinxtitleref{abs}.
For the reference standard A\({}_{abs}\), which is a plant tissue and has
a \(\delta^{13}\)C value is \(\mathrm{-}\)25 ‰ due to photosynthetic discrimination,
\(\delta\)$^{\text{14}}$C = \(\mathrm{\Delta}\)$^{\text{14}}$C. For CLM, in order to use
the $^{\text{14}}$C model independently of the $^{\text{13}}$C
model, for the $^{\text{14}}$C calculations, this fractionation is
set to zero, such that the 0.975 term becomes 1, the
\(\delta^{13}\)C term (for the calculation of
\(\delta^{14}\)C only) becomes 0, and thus
\(\delta^{14}\)C = \(\mathrm{\Delta}\)$^{\text{14}}$C.


\subsection{Carbon Isotope Discrimination During Photosynthesis}
\label{\detokenize{tech_note/Isotopes/CLM50_Tech_Note_Isotopes:carbon-isotope-discrimination-during-photosynthesis}}
Photosynthesis is modeled in CLM as a two-step process: diffusion of
CO$_{\text{2}}$ into the stomatal cavity, followed by enzymatic
fixation (Chapter \hyperref[\detokenize{tech_note/Photosynthesis/CLM50_Tech_Note_Photosynthesis:rst-stomatal-resistance-and-photosynthesis}]{\ref{\detokenize{tech_note/Photosynthesis/CLM50_Tech_Note_Photosynthesis:rst-stomatal-resistance-and-photosynthesis}}}). Each step is associated with a kinetic isotope
effect. The kinetic isotope effect during diffusion of
CO$_{\text{2}}$ through the stomatal opening is 4.4‰. The kinetic
isotope effect during fixation of CO$_{\text{2}}$ with Rubisco is
\(\sim\)30‰; however, since about 5-10\% of carbon in C3 plants
reacts with phosphoenolpyruvate carboxylase (PEPC) (Melzer and O’Leary,
1987), the net kinetic isotope effect during fixation is
\(\sim\)27‰ for C3 plants. In C4 plant photosynthesis, only the
diffusion effect is important. The fractionation factor equations for C3
and C4 plants are given below:

For C4 plants,
\phantomsection\label{\detokenize{tech_note/Isotopes/CLM50_Tech_Note_Isotopes:equation-30.8)}}\begin{equation}\label{equation:tech_note/Isotopes/CLM50_Tech_Note_Isotopes:30.8)}
\begin{split}\alpha _{psn} =1+\frac{4.4}{1000}\end{split}
\end{equation}
For C3 plants,
\phantomsection\label{\detokenize{tech_note/Isotopes/CLM50_Tech_Note_Isotopes:equation-30.9)}}\begin{equation}\label{equation:tech_note/Isotopes/CLM50_Tech_Note_Isotopes:30.9)}
\begin{split}\alpha _{psn} =1+\frac{4.4+22.6\frac{c_{i}^{*} }{pCO_{2} } }{1000}\end{split}
\end{equation}
where \({\alpha }_{psn}\) is the fractionation factor, and
\(c^*_i\) and pCO$_{\text{2}}$ are the revised intracellular and
atmospheric CO$_{\text{2}}$ partial pressure, respectively.

As can be seen from the above equation, kinetic isotope effect during
fixation of CO$_{\text{2}}$ is dependent on the intracellular
CO$_{\text{2}}$ concentration, which in turn depends on the net
carbon assimilation. That is calculated during the photosynthesis
calculation as follows:
\phantomsection\label{\detokenize{tech_note/Isotopes/CLM50_Tech_Note_Isotopes:equation-30.10)}}\begin{equation}\label{equation:tech_note/Isotopes/CLM50_Tech_Note_Isotopes:30.10)}
\begin{split}c_{i} =pCO_{2} -a_{n} p\frac{\left(1.4g_{s} \right)+\left(1.6g_{b} \right)}{g_{b} g_{s} }\end{split}
\end{equation}
where \(a_n\) is net carbon assimilation during photosynthesis, \(p\) is
atmospheric pressure, \(g_b\) is leaf boundary layer conductance,
and \(g_s\) is leaf stomatal conductance.

Isotopic fractionation code is compatible with multi-layered canopy
parameterization; i.e., it is possible to calculate varying
discrimination rates for each layer of a multi-layered canopy. However,
as with the rest of the photosynthesis model, the number of canopy
layers is currently set to one by default.


\subsection{$^{\text{14}}$C radioactive decay and historical atmospheric $^{\text{14}}$C and $^{\text{13}}$C concentrations}
\label{\detokenize{tech_note/Isotopes/CLM50_Tech_Note_Isotopes:c-radioactive-decay-and-historical-atmospheric-14c-and-13c-concentrations}}
In the preindustrial biosphere, radioactive decay of $^{\text{14}}$C
in carbon pools allows dating of long-term age since photosynthetic
uptake; while over the 20\({}^{th}\) century, radiocarbon in the
atmosphere was first diluted by radiocarbon-free fossil fuels and then
enriched by aboveground thermonuclear testing to approximately double
its long-term mean concentration. CLM includes both of these processes
to allow comparison of carbon that may vary on multiple timescales with
observed values.

For radioactive decay, at each timestep all $^{\text{14}}$C pools are
reduced at a rate of \textendash{}log/\(\tau\), where \(\tau\) is the
half-life (Libby half-life value of 5568 years). In order to rapidly
equilibrate the long-lived pools during accelerated decomposition
spinup, the radioactive decay of the accelerated pools is also
accelerated by the same degree as the decomposition, such that the
$^{\text{14}}$C value of these pools is in equilibrium when taken out
of the spinup mode.

For variation of atmospheric $^{\text{14}}$C and $^{\text{13}}$C over the historical
period, \(\mathrm{\Delta}\)$^{\text{14}}$C and \(\mathrm{\Delta}\):sup:\sphinxtitleref{13}C values can be set to
either fixed concentrations
or time-varying concentrations read in from a file. A default file is provided that spans the historical period ({\hyperref[\detokenize{tech_note/References/CLM50_Tech_Note_References:gravenetal2017}]{\sphinxcrossref{\DUrole{std,std-ref}{Graven et al., 2017}}}}).  For
\(\mathrm{\Delta}\)$^{\text{14}}$C, values are provided and read in for three latitude bands (30 $^{\text{o}}$N-90 $^{\text{o}}$N, 30 $^{\text{o}}$S-30 $^{\text{o}}$N, and 30 $^{\text{o}}$S-90 $^{\text{o}}$S).


\section{Land-Only Mode}
\label{\detokenize{tech_note/Land-Only_Mode/CLM50_Tech_Note_Land-Only_Mode:rst-land-only-mode}}\label{\detokenize{tech_note/Land-Only_Mode/CLM50_Tech_Note_Land-Only_Mode:land-only-mode}}\label{\detokenize{tech_note/Land-Only_Mode/CLM50_Tech_Note_Land-Only_Mode::doc}}
In land-only mode (uncoupled to an atmospheric model), the atmospheric
forcing required by CLM (\hyperref[\detokenize{tech_note/Ecosystem/CLM50_Tech_Note_Ecosystem:table-atmospheric-input-to-land-model}]{Table \ref{\detokenize{tech_note/Ecosystem/CLM50_Tech_Note_Ecosystem:table-atmospheric-input-to-land-model}}})
is supplied by observed datasets.  The standard forcing provided with the
model is a 110-year (1901-2010) dataset provided by the Global Soil Wetness
Project (GSWP3; NEED A REFERENCE). The GSWP3 dataset has a spatial resolution of
0.5$^{\text{o}}$ X 0.5$^{\text{o}}$ and a temporal resolution of three
hours.

An alternative forcing dataset is also available, CRUNCEP, a 110-year (1901-2010) dataset
(CRUNCEP; {\hyperref[\detokenize{tech_note/References/CLM50_Tech_Note_References:viovy2011}]{\sphinxcrossref{\DUrole{std,std-ref}{Viovy 2011}}}}) that is a combination of two existing datasets;
the CRU TS3.2 0.5$^{\text{o}}$ X 0.5$^{\text{o}}$ monthly data covering the period
1901 to 2002 ({\hyperref[\detokenize{tech_note/References/CLM50_Tech_Note_References:mitchelljones2005}]{\sphinxcrossref{\DUrole{std,std-ref}{Mitchell and Jones 2005}}}})
and the NCEP reanalysis 2.5$^{\text{o}}$ X 2.5$^{\text{o}}$
6-hourly data covering the period 1948 to 2010. The CRUNCEP dataset has
been used to force CLM for studies of vegetation growth,
evapotranspiration, and gross primary production ({\hyperref[\detokenize{tech_note/References/CLM50_Tech_Note_References:maoetal2012}]{\sphinxcrossref{\DUrole{std,std-ref}{Mao et al. 2012}}}},
{\hyperref[\detokenize{tech_note/References/CLM50_Tech_Note_References:maoetal2013}]{\sphinxcrossref{\DUrole{std,std-ref}{Mao et al. 2013}}}}, {\hyperref[\detokenize{tech_note/References/CLM50_Tech_Note_References:shietal2013}]{\sphinxcrossref{\DUrole{std,std-ref}{Shi et al. 2013}}}})
and for the TRENDY (trends in net land-atmosphere carbon exchange over the period
1980-2010) project ({\hyperref[\detokenize{tech_note/References/CLM50_Tech_Note_References:piaoetal2012}]{\sphinxcrossref{\DUrole{std,std-ref}{Piao et al. 2012}}}}). Version 7 is available
here ({\hyperref[\detokenize{tech_note/References/CLM50_Tech_Note_References:viovy2011}]{\sphinxcrossref{\DUrole{std,std-ref}{Viovy 2011}}}}).

Here, the GSWP3 dataset, which does not include data for particular fields over oceans,
lakes, and Antarctica is modified. This missing data is filled with
{\hyperref[\detokenize{tech_note/References/CLM50_Tech_Note_References:qianetal2006}]{\sphinxcrossref{\DUrole{std,std-ref}{Qian et al. (2006)}}}} data from 1948 that is interpolated by the data atmosphere
model to the 0.5$^{\text{o}}$ GSWP3 grid. This allows the model
to be run over Antarctica and ensures data is available along coastlines
regardless of model resolution.

The forcing data is ingested into a data atmosphere model in three
“streams”; precipitation (\(P\)) (mm s$^{\text{-1}}$), solar
radiation (\(S_{atm}\) ) (W m$^{\text{-2}}$), and four other
fields {[}atmospheric pressure \(P_{atm}\)  (Pa), atmospheric specific
humidity \(q_{atm}\)  (kg kg$^{\text{-1}}$), atmospheric
temperature \(T_{atm}\)  (K), and atmospheric wind \(W_{atm}\)
(m s$^{\text{-1}}$){]}. These are separate streams because they are
handled differently according to the type of field. In the GSWP3
dataset, the precipitation stream is provided at three hour intervals and
the data atmosphere model prescribes the same precipitation rate for
each model time step within the three hour period. The four fields that
are grouped together in another stream (pressure, humidity, temperature,
and wind) are provided at three hour intervals and the data atmosphere
model linearly interpolates these fields to the time step of the model.

The total solar radiation is also provided at three hour intervals. The
data is fit to the model time step using a diurnal function that depends
on the cosine of the solar zenith angle \(\mu\)  to provide a
smoother diurnal cycle of solar radiation and to ensure that all of the
solar radiation supplied by the three-hourly forcing data is actually
used. The solar radiation at model time step \(t_{M}\)  is
\phantomsection\label{\detokenize{tech_note/Land-Only_Mode/CLM50_Tech_Note_Land-Only_Mode:equation-31.1}}\begin{equation}\label{equation:tech_note/Land-Only_Mode/CLM50_Tech_Note_Land-Only_Mode:31.1}
\begin{split}\begin{array}{lr}
S_{atm} \left(t_{M} \right)=\frac{\frac{\Delta t_{FD} }{\Delta t_{M} } S_{atm} \left(t_{FD} \right)\mu \left(t_{M} \right)}{\sum _{i=1}^{\frac{\Delta t_{FD} }{\Delta t_{M} } }\mu \left(t_{M_{i} } \right) } & \qquad {\rm for\; }\mu \left(t_{M} \right)>0.001 \\
S_{atm} \left(t_{M} \right)=0 & \qquad {\rm for\; }\mu \left(t_{M} \right)\le 0.001
\end{array}\end{split}
\end{equation}
where \(\Delta t_{FD}\)  is the time step of the forcing data (3
hours \(\times\)  3600 seconds hour$^{\text{-1}}$ = 10800
seconds), \(\Delta t_{M}\)  is the model time step (seconds),
\(S_{atm} \left(t_{FD} \right)\) is the three-hourly solar radiation
from the forcing data (W m$^{\text{-2}}$), and
\(\mu \left(t_{M} \right)\) is the cosine of the solar zenith angle
at model time step \(t_{M}\)  (section \hyperref[\detokenize{tech_note/Surface_Albedos/CLM50_Tech_Note_Surface_Albedos:solar-zenith-angle}]{\ref{\detokenize{tech_note/Surface_Albedos/CLM50_Tech_Note_Surface_Albedos:solar-zenith-angle}}}). The term in the
denominator of equation (1) is the sum of the cosine of the solar zenith
angle for each model time step falling within the three hour period. For
numerical purposes, \(\mu \left(t_{M_{i} } \right)\ge 0.001\).

The total incident solar radiation \(S_{atm}\)  at the model time
step \(t_{M}\)  is then split into near-infrared and visible
radiation and partitioned into direct and diffuse according to factors
derived from one year’s worth of hourly CAM output from CAM version
cam3\_5\_55 as
\phantomsection\label{\detokenize{tech_note/Land-Only_Mode/CLM50_Tech_Note_Land-Only_Mode:equation-31.2}}\begin{equation}\label{equation:tech_note/Land-Only_Mode/CLM50_Tech_Note_Land-Only_Mode:31.2}
\begin{split}S_{atm} \, \downarrow _{vis}^{\mu } =R_{vis} \left(\alpha S_{atm} \right)\end{split}
\end{equation}\phantomsection\label{\detokenize{tech_note/Land-Only_Mode/CLM50_Tech_Note_Land-Only_Mode:equation-31.3}}\begin{equation}\label{equation:tech_note/Land-Only_Mode/CLM50_Tech_Note_Land-Only_Mode:31.3}
\begin{split}S_{atm} \, \downarrow _{nir}^{\mu } =R_{nir} \left[\left(1-\alpha \right)S_{atm} \right]\end{split}
\end{equation}\phantomsection\label{\detokenize{tech_note/Land-Only_Mode/CLM50_Tech_Note_Land-Only_Mode:equation-31.4}}\begin{equation}\label{equation:tech_note/Land-Only_Mode/CLM50_Tech_Note_Land-Only_Mode:31.4}
\begin{split}S_{atm} \, \downarrow _{vis} =\left(1-R_{vis} \right)\left(\alpha S_{atm} \right)\end{split}
\end{equation}\phantomsection\label{\detokenize{tech_note/Land-Only_Mode/CLM50_Tech_Note_Land-Only_Mode:equation-31.5}}\begin{equation}\label{equation:tech_note/Land-Only_Mode/CLM50_Tech_Note_Land-Only_Mode:31.5}
\begin{split}S_{atm} \, \downarrow _{nir} =\left(1-R_{nir} \right)\left[\left(1-\alpha \right)S_{atm} \right].\end{split}
\end{equation}
where \(\alpha\) , the ratio of visible to total incident solar
radiation, is assumed to be
\phantomsection\label{\detokenize{tech_note/Land-Only_Mode/CLM50_Tech_Note_Land-Only_Mode:equation-31.6}}\begin{equation}\label{equation:tech_note/Land-Only_Mode/CLM50_Tech_Note_Land-Only_Mode:31.6}
\begin{split}\alpha =\frac{S_{atm} \, \downarrow _{vis}^{\mu } +S_{atm} \, \downarrow _{vis}^{} }{S_{atm} } =0.5.\end{split}
\end{equation}
The ratio of direct to total incident radiation in the visible
\(R_{vis}\)  is
\phantomsection\label{\detokenize{tech_note/Land-Only_Mode/CLM50_Tech_Note_Land-Only_Mode:equation-31.7}}\begin{equation}\label{equation:tech_note/Land-Only_Mode/CLM50_Tech_Note_Land-Only_Mode:31.7}
\begin{split}R_{vis} =a_{0} +a_{1} \times \alpha S_{atm} +a_{2} \times \left(\alpha S_{atm} \right)^{2} +a_{3} \times \left(\alpha S_{atm} \right)^{3} \qquad 0.01\le R_{vis} \le 0.99\end{split}
\end{equation}
and in the near-infrared \(R_{nir}\)  is
\phantomsection\label{\detokenize{tech_note/Land-Only_Mode/CLM50_Tech_Note_Land-Only_Mode:equation-31.8}}\begin{equation}\label{equation:tech_note/Land-Only_Mode/CLM50_Tech_Note_Land-Only_Mode:31.8}
\begin{split}R_{nir} =b_{0} +b_{1} \times \left(1-\alpha \right)S_{atm} +b_{2} \times \left[\left(1-\alpha \right)S_{atm} \right]^{2} +b_{3} \times \left[\left(1-\alpha \right)S_{atm} \right]^{3} \qquad 0.01\le R_{nir} \le 0.99\end{split}
\end{equation}
where
\(a_{0} =0.17639,\, a_{1} =0.00380,\, a_{2} =-9.0039\times 10^{-6} ,\, a_{3} =8.1351\times 10^{-9}\)
and
\(b_{0} =0.29548,b_{1} =0.00504,b_{2} =-1.4957\times 10^{-5} ,b_{3} =1.4881\times 10^{-8}\)
are coefficients from polynomial fits to the CAM data.

The additional atmospheric forcing variables required by \hyperref[\detokenize{tech_note/Ecosystem/CLM50_Tech_Note_Ecosystem:table-atmospheric-input-to-land-model}]{Table \ref{\detokenize{tech_note/Ecosystem/CLM50_Tech_Note_Ecosystem:table-atmospheric-input-to-land-model}}} are
derived as follows. The atmospheric reference height \(z'_{atm}\)
(m) is set to 30 m. The directional wind components are derived as
\(u_{atm} =v_{atm} ={W_{atm} \mathord{\left/ {\vphantom {W_{atm}  \sqrt{2} }} \right. \kern-\nulldelimiterspace} \sqrt{2} }\) .
The potential temperature \(\overline{\theta _{atm} }\) (K) is set
to the atmospheric temperature \(T_{atm}\) . The atmospheric
longwave radiation \(L_{atm} \, \downarrow\)  (W m$^{\text{-2}}$)
is derived from the atmospheric vapor pressure \(e_{atm}\)  and
temperature \(T_{atm}\)  ({\hyperref[\detokenize{tech_note/References/CLM50_Tech_Note_References:idso1981}]{\sphinxcrossref{\DUrole{std,std-ref}{Idso 1981}}}}) as
\phantomsection\label{\detokenize{tech_note/Land-Only_Mode/CLM50_Tech_Note_Land-Only_Mode:equation-31.9}}\begin{equation}\label{equation:tech_note/Land-Only_Mode/CLM50_Tech_Note_Land-Only_Mode:31.9}
\begin{split}L_{atm} \, \downarrow =\left[0.70+5.95\times 10^{-5} \times 0.01e_{atm} \exp \left(\frac{1500}{T_{atm} } \right)\right]\sigma T_{atm}^{4}\end{split}
\end{equation}
where
\phantomsection\label{\detokenize{tech_note/Land-Only_Mode/CLM50_Tech_Note_Land-Only_Mode:equation-31.10}}\begin{equation}\label{equation:tech_note/Land-Only_Mode/CLM50_Tech_Note_Land-Only_Mode:31.10}
\begin{split}e_{atm} =\frac{P_{atm} q_{atm} }{0.622+0.378q_{atm} }\end{split}
\end{equation}
and \(\sigma\)  is the Stefan-Boltzmann constant (W m$^{\text{-2}}$ K$^{\text{-4}}$)
(\hyperref[\detokenize{tech_note/Ecosystem/CLM50_Tech_Note_Ecosystem:table-physical-constants}]{Table \ref{\detokenize{tech_note/Ecosystem/CLM50_Tech_Note_Ecosystem:table-physical-constants}}}). The fraction of
precipitation \(P\) (mm s$^{\text{-1}}$) falling as rain and/or
snow is
\phantomsection\label{\detokenize{tech_note/Land-Only_Mode/CLM50_Tech_Note_Land-Only_Mode:equation-31.11}}\begin{equation}\label{equation:tech_note/Land-Only_Mode/CLM50_Tech_Note_Land-Only_Mode:31.11}
\begin{split}q_{rain} =P\left(f_{P} \right),\end{split}
\end{equation}\phantomsection\label{\detokenize{tech_note/Land-Only_Mode/CLM50_Tech_Note_Land-Only_Mode:equation-31.12}}\begin{equation}\label{equation:tech_note/Land-Only_Mode/CLM50_Tech_Note_Land-Only_Mode:31.12}
\begin{split}q_{snow} =P\left(1-f_{P} \right)\end{split}
\end{equation}
where
\phantomsection\label{\detokenize{tech_note/Land-Only_Mode/CLM50_Tech_Note_Land-Only_Mode:equation-31.13}}\begin{equation}\label{equation:tech_note/Land-Only_Mode/CLM50_Tech_Note_Land-Only_Mode:31.13}
\begin{split}f_{P} =0<0.5\left(T_{atm} -T_{f} \right)<1.\end{split}
\end{equation}
The aerosol deposition rates \(D_{sp}\)  (14 rates as described in
\hyperref[\detokenize{tech_note/Ecosystem/CLM50_Tech_Note_Ecosystem:table-atmospheric-input-to-land-model}]{Table \ref{\detokenize{tech_note/Ecosystem/CLM50_Tech_Note_Ecosystem:table-atmospheric-input-to-land-model}}}) are provided by a
time-varying, globally-gridded aerosol deposition file developed by
{\hyperref[\detokenize{tech_note/References/CLM50_Tech_Note_References:lamarqueetal2010}]{\sphinxcrossref{\DUrole{std,std-ref}{Lamarque et al. (2010)}}}}.

If the user wishes to provide atmospheric forcing data from another
source, the data format outlined above will need to be followed with the
following exceptions. The data atmosphere model will accept a
user-supplied relative humidity \(RH\) (\%) and derive specific
humidity \(q_{atm}\)  (kg kg$^{\text{-1}}$) from
\phantomsection\label{\detokenize{tech_note/Land-Only_Mode/CLM50_Tech_Note_Land-Only_Mode:equation-31.14}}\begin{equation}\label{equation:tech_note/Land-Only_Mode/CLM50_Tech_Note_Land-Only_Mode:31.14}
\begin{split}q_{atm} =\frac{0.622e_{atm} }{P_{atm} -0.378e_{atm} }\end{split}
\end{equation}
where the atmospheric vapor pressure \(e_{atm}\)  (Pa) is derived
from the water (\(T_{atm} >T_{f}\) ) or ice
(\(T_{atm} \le T_{f}\) ) saturation vapor pressure
\(e_{sat}^{T_{atm} }\)  as
\(e_{atm} =\frac{RH}{100} e_{sat}^{T_{atm} }\)  where \(T_{f}\)
is the freezing temperature of water (K) (\hyperref[\detokenize{tech_note/Ecosystem/CLM50_Tech_Note_Ecosystem:table-physical-constants}]{Table \ref{\detokenize{tech_note/Ecosystem/CLM50_Tech_Note_Ecosystem:table-physical-constants}}}), and
\(P_{atm}\)  is the pressure at height \(z_{atm}\)  (Pa). The
data atmosphere model will also accept a user-supplied dew point
temperature \(T_{dew}\)  (K) and derive specific humidity
\(q_{atm}\)  from
\phantomsection\label{\detokenize{tech_note/Land-Only_Mode/CLM50_Tech_Note_Land-Only_Mode:equation-31.15}}\begin{equation}\label{equation:tech_note/Land-Only_Mode/CLM50_Tech_Note_Land-Only_Mode:31.15}
\begin{split}q_{atm} = \frac{0.622e_{sat}^{T_{dew} } }{P_{atm} -0.378e_{sat}^{T_{dew} } } .\end{split}
\end{equation}
Here, \(e_{sat}^{T}\) , the saturation vapor pressure as a function
of temperature, is derived from {\hyperref[\detokenize{tech_note/References/CLM50_Tech_Note_References:lowe1977}]{\sphinxcrossref{\DUrole{std,std-ref}{Lowe’s (1977)}}}} polynomials. If not
provided by the user, the atmospheric pressure \(P_{atm}\)  (Pa) is
set equal to the standard atmospheric pressure \(P_{std} =101325\)
Pa, and surface pressure \(P_{srf}\)  (Pa) is set equal
to\(P_{atm}\) .

The user may provide the total direct and diffuse solar radiation,
\(S_{atm} \, \downarrow ^{\mu }\)  and
\(S_{atm} \, \downarrow\) . These will be time-interpolated using
the procedure described above and then each term equally apportioned
into the visible and near-infrared wavebands (e.g.,
\(S_{atm} \, \downarrow _{vis}^{\mu } =0.5S_{atm} \, \downarrow ^{\mu }\) ,
\(S_{atm} \, \downarrow _{nir}^{\mu } =0.5S_{atm} \, \downarrow ^{\mu }\) ).


\subsection{Anomaly Forcing}
\label{\detokenize{tech_note/Land-Only_Mode/CLM50_Tech_Note_Land-Only_Mode:id1}}\label{\detokenize{tech_note/Land-Only_Mode/CLM50_Tech_Note_Land-Only_Mode:anomaly-forcing}}
The ‘Anomaly Forcing’ atmospheric forcing mode provides a means to drive
CLM with projections of future climate conditions without the need for
large, high-frequency datasets.  From an existing climate simulation
spanning both the historical and future time periods, a set of anomalies
are created by removing a climatological seasonal cycle based on the end
of the historical period from each year of the future time period of the
simulation.  These anomalies can then be applied to a repeating
high-frequency forcing dataset of finite duration (e.g. 10 years).  State
and flux forcing variables are adjusted using additive and multiplicative
anomalies, respectively:
\phantomsection\label{\detokenize{tech_note/Land-Only_Mode/CLM50_Tech_Note_Land-Only_Mode:equation-31.16}}\begin{equation}\label{equation:tech_note/Land-Only_Mode/CLM50_Tech_Note_Land-Only_Mode:31.16}
\begin{split}\begin{array}{lr}
S^{'} = S + k_{anomaly} & \quad {\rm state \ variable} \\
F^{'} = f \times k_{anomaly} & \quad {\rm flux \ variable}
\end{array}\end{split}
\end{equation}
where \(S^{'}\) is the adjusted atmospheric state variable, \(S\)
is the state variable from the high-frequency reference atmospheric
forcing dataset, and \(k_{anomaly}\) is an additive anomaly.
Similarly, \(F^{'}\) is the adjusted atmospheric flux variable,
\(F\) is the flux variable from the high-frequency reference
atmospheric forcing dataset, and \(k_{anomaly}\) is a
multiplicative anomaly.  State variables are temperature \(T_{atm}\),
pressure \(P_{atm}\), humidity \(q_{atm}\), and wind
\(W_{atm}\).  Flux variables are precipitation \(P\), atmospheric
shortwave radiation \(S_{atm} \, \downarrow\), and atmospheric
longwave radiation \(L_{atm} \, \downarrow\).
\phantomsection\label{\detokenize{tech_note/References/CLM50_Tech_Note_References:rst-references}}

\section{References}
\label{\detokenize{tech_note/References/CLM50_Tech_Note_References:references}}\label{\detokenize{tech_note/References/CLM50_Tech_Note_References::doc}}\phantomsection\label{\detokenize{tech_note/References/CLM50_Tech_Note_References:aberetal1990}}
\\

Aber, J.D., Melillo, J.M. and McClaugherty, C.A., 1990. Predicting
long-term patterns of mass loss, nitrogen dynamics, and soil organic
matter formation from initial fime litter chemistry in temperate forest
ecosystems. Canadian Journal of Botany, 68: 2201-2208.
\phantomsection\label{\detokenize{tech_note/References/CLM50_Tech_Note_References:aberetal2003}}
\\

Aber, J.D., Goodale, C.L., Ollinger, S.V., Smith, M.-L., Magill, A.H.,
Martin, M.E., Hallett, R.A., and Stoddard, J.L. 2003. Is nitrogen
deposition altering the nitrogen status of northeastern forests?
BioScience 53:375-389.
\phantomsection\label{\detokenize{tech_note/References/CLM50_Tech_Note_References:alietal2016}}
\\

Ali, A. A., C. Xu, A. Rogers, R. A. Fisher, S. D. Wullschleger, E. Massoud, J. A. Vrugt, J. D. Muss, N. McDowell,
and J. Fisher, 2016: A global scale mechanistic model of
photosynthetic capacity (LUNA V1. 0). Geosci. Mod. Dev., 9:587-606.
\phantomsection\label{\detokenize{tech_note/References/CLM50_Tech_Note_References:allenetal2005}}
\\

Allen, C.B., Will, R.E., and Jacobson, M.A. 2005. Production efficiency
and radiation use efficiency of four tree species receiving irrigation
and fertilization. Forest Science 51:556-569.
\phantomsection\label{\detokenize{tech_note/References/CLM50_Tech_Note_References:anderson1976}}
\\

Anderson, E.A. 1976. A point energy and mass balance model of a snow
cover. NOAA Technical Report NWS 19, Office of Hydrology, National
Weather Service, Silver Spring, MD.
\phantomsection\label{\detokenize{tech_note/References/CLM50_Tech_Note_References:andreetal1986}}
\\

André, J.-C., Goutorbe, J.-P., and Perrier, A. 1986. HAPEX-MOBILHY: A
hydrologic atmosphere experiment for the study of water budget and
evaporation flux at the climatic scale. Bull. Amer. Meteor. Soc.
67:138-144.
\phantomsection\label{\detokenize{tech_note/References/CLM50_Tech_Note_References:andrenpaustian1987}}
\\

Andrén, O. and Paustian, K., 1987. Barley straw decomposition in the
field: a comparison of models. Ecology 68:1190-1200.
\phantomsection\label{\detokenize{tech_note/References/CLM50_Tech_Note_References:arahstephen1998}}
\\

Arah, J.R.M. and Stephen, K.D., 1998. A model of the processes leading
to methane emission from peatland. Atmos. Environ. 32:3257-3264.
\phantomsection\label{\detokenize{tech_note/References/CLM50_Tech_Note_References:arahvinten1995}}
\\

Arah, J. and Vinten, A., 1995. Simplified models of anoxia and
denitrification in aggregated and simple-structured soils. European
Journal of Soil Science 46:507-517.
\phantomsection\label{\detokenize{tech_note/References/CLM50_Tech_Note_References:arendtetal2012}}
\\

Arendt, A., et al. 2012. Randolph Glacier Inventory: A Dataset of Global
Glacier Outlines Version: 1.0, Global Land Ice Measurements from Space,
Boulder Colorado, USA. Digital Media.
\phantomsection\label{\detokenize{tech_note/References/CLM50_Tech_Note_References:aroraboer2005}}
\\

Arora, V.K. and Boer, G.J. 2005. Fire as an interactive component of
dynamic vegetation models. J. Geophys. Res. 110:G02008.
DOI:10.1029/2005JG000042.
\phantomsection\label{\detokenize{tech_note/References/CLM50_Tech_Note_References:arya2001}}
\\

Arya, S.P. 2001. Introduction to Meteorology. Academic Press, San Diego,
CA.
\phantomsection\label{\detokenize{tech_note/References/CLM50_Tech_Note_References:asneretal1998}}
\\

Asner, G.P., Wessman, C.A., Schimel, D.S., and Archer, S. 1998.
Variability in leaf and litter optical properties: implications for BRDF
model inversions using AVHRR, MODIS, and MISR. Remote Sens. Environ.
63:243-257.
\phantomsection\label{\detokenize{tech_note/References/CLM50_Tech_Note_References:axelssonaxelsson1986}}
\\

Axelsson, E., and Axelsson, B. 1986. Changes in carbon allocation
patterns in spruce and pine trees following irrigation and
fertilization. Tree Phys. 2:189-204.
\phantomsection\label{\detokenize{tech_note/References/CLM50_Tech_Note_References:badgeranddirmeyer2015}}
\\

Badger, A.M., and Dirmeyer, P.A., 2015. Climate response to Amazon forest
replacement by heterogeneous crop cover. Hydrol. Earth. Syst. Sci. 19:4547-
4557.
\phantomsection\label{\detokenize{tech_note/References/CLM50_Tech_Note_References:bairdetal2004}}
\\

Baird, A.J., Beckwith, C.W., Waldron, S. and Waddington, J.M., 2004.
Ebullition of methane-containing gas bubbles from near-surface Sphagnum
peat. Geophys. Res. Lett. 31. DOI:10.1029/2004GL021157.
\phantomsection\label{\detokenize{tech_note/References/CLM50_Tech_Note_References:baldocchietal2001}}
\\

Baldocchi, D., et al. 2001. FLUXNET: A new tool to study the temporal
and spatial variability of ecosystem-scale carbon dioxide, water vapor,
and energy flux densities. Bull. Amer. Meteor. Soc. 82:2415-2433.
\phantomsection\label{\detokenize{tech_note/References/CLM50_Tech_Note_References:barbottinetal2005}}
\\

Barbottin, A., Lecomte, C., Bouchard, C., and Jeuffroy, M.-H. 2005.
Nitrogen remobilization during grain filling in wheat: Genotypic and
environmental effects. Crop Sci. 45:1141-1150.
\phantomsection\label{\detokenize{tech_note/References/CLM50_Tech_Note_References:batjes2006}}
\\

Batjes, N.H., 2006. ISRIC-WISE derived soil properties on a 5 by 5
arc-minutes global grid. Report 2006/02 (available through :
\sphinxurl{http://www.isric.org})
\phantomsection\label{\detokenize{tech_note/References/CLM50_Tech_Note_References:berger1978a}}
\\

Berger, A.L. 1978a. Long-term variations of daily insolation and
quaternary climatic changes. J. Atmos. Sci. 35:2362-2367.
\phantomsection\label{\detokenize{tech_note/References/CLM50_Tech_Note_References:berger1978b}}
\\

Berger, A.L. 1978b. A simple algorithm to compute long-term variations
of daily or monthly insolation. Contribution de l’Institut d’Astronomie
et de Géophysique, Université Catholique de Louvain, Louvain-la-Neuve,
No. 18.
\phantomsection\label{\detokenize{tech_note/References/CLM50_Tech_Note_References:bergeretal1993}}
\\

Berger, A., Loutre, M.-F., and Tricot, C. 1993. Insolation and Earth’s
orbital periods. J. Geophys. Res. 98:10341-10362.
\phantomsection\label{\detokenize{tech_note/References/CLM50_Tech_Note_References:berkowitzbalberg1992}}
\\

Berkowitz, B., and Balberg, I. 1992. Percolation approach to the problem
of hydraulic conductivity in porous media. Transport in Porous Media
9:275\textendash{}286.
\phantomsection\label{\detokenize{tech_note/References/CLM50_Tech_Note_References:bevenkirkby1979}}
\\

Beven, K.J., and Kirkby, M.J. 1979. A physically based variable
contributing area model of basin hydrology. Hydrol. Sci. Bull. 24:43-69.
\phantomsection\label{\detokenize{tech_note/References/CLM50_Tech_Note_References:bohrenhuffman1983}}
\\

Bohren, C. F., and Huffman, D. R. 1983. Absorption and scattering of
light by small particles. John Wiley \& Sons, New York, NY.
\phantomsection\label{\detokenize{tech_note/References/CLM50_Tech_Note_References:bonan1996}}
\\

Bonan, G.B. 1996. A land surface model (LSM version 1.0) for ecological,
hydrological, and atmospheric studies: Technical description and user’s
guide. NCAR Technical Note NCAR/TN-417+STR, National Center for
Atmospheric Research, Boulder, CO, 150 pp.
\phantomsection\label{\detokenize{tech_note/References/CLM50_Tech_Note_References:bonan1998}}
\\

Bonan, G.B. 1998. The land surface climatology of the NCAR Land Surface
Model coupled to the NCAR Community Climate Model. J. Climate
11:1307-1326.
\phantomsection\label{\detokenize{tech_note/References/CLM50_Tech_Note_References:bonan2002}}
\\

Bonan, G.B. 2002. Ecological Climatology: Concepts and Applications.
Cambridge University Press.
\phantomsection\label{\detokenize{tech_note/References/CLM50_Tech_Note_References:bonanetal2002a}}
\\

Bonan, G.B., Oleson, K.W., Vertenstein, M., Levis, S., Zeng, X., Dai,
Y., Dickinson, R.E., and Yang, Z.-L. 2002a. The land surface climatology
of the Community Land Model coupled to the NCAR Community Climate Model.
J. Climate 15: 3123-3149.
\phantomsection\label{\detokenize{tech_note/References/CLM50_Tech_Note_References:bonanetal2002b}}
\\

Bonan, G.B., Levis, S., Kergoat, L., and Oleson, K.W. 2002b. Landscapes
as patches of plant functional types: An integrating concept for climate
and ecosystem models. Global Biogeochem. Cycles 16: 5.1-5.23.
\phantomsection\label{\detokenize{tech_note/References/CLM50_Tech_Note_References:bonanlevis2006}}
\\

Bonan, G.B., and Levis, S. 2006. Evaluating aspects of the Community
Land and Atmosphere Models (CLM3 and CAM3) using a dynamic global
vegetation model. J. Climate 19:2290-2301.
\phantomsection\label{\detokenize{tech_note/References/CLM50_Tech_Note_References:bonanetal2011}}
\\

Bonan, G.B., Lawrence P.J., Oleson K.W., Levis S., Jung M., Reichstein
M., Lawrence, D.M., and Swenson, S.C. 2011. Improving canopy processes
in the Community Land Model (CLM4) using global flux fields empirically
inferred from FLUXNET data. J. Geophys. Res. 116, G02014.
DOI:10.1029/2010JG001593.
\phantomsection\label{\detokenize{tech_note/References/CLM50_Tech_Note_References:bonanetal2012}}
\\

Bonan, G. B., Oleson, K.W., Fisher, R.A., Lasslop, G., and Reichstein,
M. 2012. Reconciling leaf physiological traits and canopy flux data: Use
of the TRY and FLUXNET databases in the Community Land Model version 4,
J. Geophys. Res., 117, G02026. DOI:10.1029/2011JG001913.
\phantomsection\label{\detokenize{tech_note/References/CLM50_Tech_Note_References:bonanetal2014}}
\\

Bonan, G.B., Williams, M., Fisher, R.A., and Oleson, K.W. 2014. Modeling
stomatal conductance in the earth system: linking leaf water-use
efficiency and water transport along the soil\textendash{}plant\textendash{}atmosphere continuum,
Geosci. Model Dev., 7, 2193-2222, doi:10.5194/gmd-7-2193-2014.
\phantomsection\label{\detokenize{tech_note/References/CLM50_Tech_Note_References:brun1989}}
\\

Brun, E. 1989. Investigation of wet-snow metamorphism in respect of
liquid water content. Ann. Glaciol. 13:22-26.
\phantomsection\label{\detokenize{tech_note/References/CLM50_Tech_Note_References:brunkeetal2016}}
\\

Brunke, M. A., P. Broxton, J. Pelletier, D. Gochis, P. Hazenberg, D. M. Lawrence, L. R. Leung, G.-Y. Niu, P. A. Troch, and X. Zeng, 2016: Implementing and Evaluating Variable Soil Thickness in the Community Land Model, Version 4.5 (CLM4.5). J. Clim. 29:3441-3461.
\phantomsection\label{\detokenize{tech_note/References/CLM50_Tech_Note_References:brzosteketal2014}}
\\

Brzostek, E. R., J. B. Fisher, and R. P. Phillips, 2014. Modeling the carbon cost of plant nitrogen acquisition: Mycorrhizal trade-offs and multipath resistance uptake improve predictions of retranslocation. J. Geophys. Res. Biogeosci., 119, 1684\textendash{}1697, doi:10.1002/2014JG002660.
\phantomsection\label{\detokenize{tech_note/References/CLM50_Tech_Note_References:bugmannsolomon2000}}
\\

Bugmann, H., and Solomon, A.M. 2000. Explaining forest composition and
biomass across multiple biogeographical regions. Ecol. Appl. 10:95-114.
\phantomsection\label{\detokenize{tech_note/References/CLM50_Tech_Note_References:busing2005}}
\\

Busing, R.T. 2005. Tree mortality, canopy turnover, and woody detritus
in old cove forests of the southern Appalachians. Ecology 86:73-84.
\phantomsection\label{\detokenize{tech_note/References/CLM50_Tech_Note_References:buzanetal2015}}
\\

Buzan, J.R., Oleson, K., and Huber, M. 2015: Implementation and
comparison of a suite of heat stress metrics within the Community Land
Model version 4.5, Geosci. Model Dev., 8, 151-170, doi:10.5194/gmd-8-151-2015.
\phantomsection\label{\detokenize{tech_note/References/CLM50_Tech_Note_References:campbellnorman1998}}
\\

Campbell, G.S., and Norman, J.M. 1998. An Introduction to Environmental
Biophysics (2:math:\sphinxtitleref{\{\}\textasciicircum{}\{nd\}} edition). Springer-Verlag, New York.
\phantomsection\label{\detokenize{tech_note/References/CLM50_Tech_Note_References:castilloetal2012}}
\\

Castillo, G., Kendra, C., Levis, S., and Thornton, P. 2012. Evaluation
of the new CNDV option of the Community Land Model: effects of dynamic
vegetation and interactive nitrogen on CLM4 means and variability. J.
Climate 25:3702\textendash{}3714.
\phantomsection\label{\detokenize{tech_note/References/CLM50_Tech_Note_References:caoetal1996}}
\\

Cao, M., Marshall, S. and Gregson, K., 1996. Global carbon exchange and
methane emissions from natural wetlands: Application of a process-based
model. J. Geophys. Res. 101(D9):14,399-14,414.
\phantomsection\label{\detokenize{tech_note/References/CLM50_Tech_Note_References:chuangetal2006}}
\\

Chuang Y.L., Oren R., Bertozzi A.L, Phillips N., Katul G.G. 2006. The
porous media model for the hydraulic system of a conifer tree: Linking
sap flux data to transpiration rate, Ecological Modelling, 191, 447-468,
doi:10.1016/j.ecolmodel.2005.03.027.
\phantomsection\label{\detokenize{tech_note/References/CLM50_Tech_Note_References:churkinaetal2003}}
\\

Churkina, G. et al., 2003. Analyzing the ecosystem carbon dynamics of
four European coniferous forests using a biogeochemistry model.
Ecosystems, 6: 168-184.
\phantomsection\label{\detokenize{tech_note/References/CLM50_Tech_Note_References:ciesin2005}}
\\

CIESIN: Gridded population of the world version 3 (GPWv3), 2005.
Population density grids, Technical report, Socioeconomic Data and
Applications Center (SEDAC), Columbia University, Palisades, New York,
USA.
\phantomsection\label{\detokenize{tech_note/References/CLM50_Tech_Note_References:clapphornberger1978}}
\\

Clapp, R.B., and Hornberger, G.M. 1978. Empirical equations for some
soil hydraulic properties. Water Resour. Res. 14:601-604.
\phantomsection\label{\detokenize{tech_note/References/CLM50_Tech_Note_References:clauserhuenges1995}}
\\

Clauser, C., and Huenges, E. 1995. Thermal conductivity of rocks and
minerals. pp. 105-126. In: T. J. Ahrens (editor) Rock Physics and Phase
Relations: A Handbook of Physical Constants. Washington, D.C.
\phantomsection\label{\detokenize{tech_note/References/CLM50_Tech_Note_References:clevelandetal1999}}
\\

Cleveland, C.C., Townsend, A.R., Schimel, D.S., Fisher, H., Howarth,
R.W., Hedin, L.O., Perakis, S.S., Latty, E.F., Von Fischer, J.C.,
Elseroad, A., and Wasson, M.F. 1999. Global patterns of terrestrial
biological nitrogen (N2) fixation in natural ecosystems. Global
Biogeochem. Cycles 13:623-645.
\phantomsection\label{\detokenize{tech_note/References/CLM50_Tech_Note_References:collatzetal1991}}
\\

Collatz, G.J., Ball, J.T., Grivet, C., and Berry, J.A. 1991.
Physiological and environmental regulation of stomatal conductance,
photosynthesis, and transpiration: A model that includes a laminar
boundary layer. Agric. For. Meteor. 54:107-136.
\phantomsection\label{\detokenize{tech_note/References/CLM50_Tech_Note_References:collatzetal1992}}
\\

Collatz, G.J., Ribas-Carbo, M., and Berry, J.A. 1992. Coupled
photosynthesis-stomatal conductance model for leaves of
C\({}_{4}\) plants. Aust. J. Plant Physiol. 19:519-538.
\phantomsection\label{\detokenize{tech_note/References/CLM50_Tech_Note_References:colmer2003}}
\\

Colmer, T.D., 2003. Long-distance transport of gases in plants: a
perspective on internal aeration and radial oxygen loss from roots.
Plant Cell and Environment 26:17-36.
\phantomsection\label{\detokenize{tech_note/References/CLM50_Tech_Note_References:conwayetal1996}}
\\

Conway, H., Gades, A., and Raymond, C.F. 1996. Albedo of dirty snow
during conditions of melt. Water Resour. Res. 32:1713-1718.
\phantomsection\label{\detokenize{tech_note/References/CLM50_Tech_Note_References:cosbyetal1984}}
\\

Cosby, B.J., Hornberger, G.M., Clapp, R.B., and Ginn, T.R. 1984. A
statistical exploration of the relationships of soil moisture
characteristics to the physical properties of soils. Water Resour. Res.
20:682-690.
\phantomsection\label{\detokenize{tech_note/References/CLM50_Tech_Note_References:crawfordetal1982}}
\\

Crawford, T. W., Rendig, V. V., and Broadent, F. E. 1982. Sources,
fluxes, and sinks of nitrogen during early reproductive growth of maize
(Zea mays L.). Plant Physiol. 70:1645-1660.
\phantomsection\label{\detokenize{tech_note/References/CLM50_Tech_Note_References:dahlinetal2015}}
\\

Dahlin, K., R. Fisher, and P. Lawrence, 2015: Environmental drivers of drought deciduous phenology in the Community Land Model. Biogeosciences, 12:5061-5074.
\phantomsection\label{\detokenize{tech_note/References/CLM50_Tech_Note_References:daizeng1997}}
\\

Dai, Y., and Zeng, Q. 1997. A land surface model (IAP94) for climate
studies. Part I: formulation and validation in off-line experiments.
Adv. Atmos. Sci. 14:433-460.
\phantomsection\label{\detokenize{tech_note/References/CLM50_Tech_Note_References:daietal2001}}
\\

Dai, Y., et al. 2001. Common Land Model: Technical documentation and
user’s guide {[}Available online at
\sphinxurl{http://climate.eas.gatech.edu/dai/clmdoc.pdf}{]}.
\phantomsection\label{\detokenize{tech_note/References/CLM50_Tech_Note_References:daietal2003}}
\\

Dai, Y., Zeng, X., Dickinson, R.E., Baker, I., Bonan, G.B., Bosilovich,
M.G., Denning, A.S., Dirmeyer, P.A., Houser, P.R., Niu, G., Oleson,
K.W., Schlosser, C.A., and Yang, Z.-L. 2003. The Common Land Model.
Bull. Amer. Meteor. Soc. 84:1013-1023.
\phantomsection\label{\detokenize{tech_note/References/CLM50_Tech_Note_References:daietal2004}}
\\

Dai, Y., Dickinson, R.E., and Wang, Y.-P. 2004. A two-big-leaf model for
canopy temperature, photosynthesis, and stomatal conductance. J. Climate
17:2281-2299.
\phantomsection\label{\detokenize{tech_note/References/CLM50_Tech_Note_References:daitrenberth2002}}
\\

Dai, A., and Trenberth, K.E. 2002. Estimates of freshwater discharge
from continents: Latitudinal and seasonal variations. J. Hydrometeor.
3:660-687.
\phantomsection\label{\detokenize{tech_note/References/CLM50_Tech_Note_References:defriesetal2000}}
\\

DeFries, R.S., Hansen, M.C., Townshend, J.R.G., Janetos, A.C., and
Loveland, T.R. 2000. A new global 1-km dataset of percentage tree cover
derived from remote sensing. Global Change Biol. 6:247-254.
\phantomsection\label{\detokenize{tech_note/References/CLM50_Tech_Note_References:degenssparling1996}}
\\

Degens, B. and Sparling, G., 1996. Changes in aggregation do not
correspond with changes in labile organic C fractions in soil amended
with \({}^{14}\)C-glucose. Soil Biology and Biochemistry, 28(4/5):
453-462.
\phantomsection\label{\detokenize{tech_note/References/CLM50_Tech_Note_References:devries1963}}
\\

de Vries, D.A. 1963. Thermal Properties of Soils. In: W.R. van Wijk
(editor) Physics of the Plant Environment. North-Holland, Amsterdam.
\phantomsection\label{\detokenize{tech_note/References/CLM50_Tech_Note_References:dickinson1983}}
\\

Dickinson, R.E. 1983. Land surface processes and climate-surface albedos
and energy balance. Adv. Geophys. 25:305-353.
\phantomsection\label{\detokenize{tech_note/References/CLM50_Tech_Note_References:dickinsonetal1993}}
\\

Dickinson, R.E., Henderson-Sellers, A., and Kennedy, P.J. 1993.
Biosphere-Atmosphere Transfer Scheme (BATS) version 1e as coupled to the
NCAR Community Climate Model. NCAR Technical Note NCAR/TN-387+STR.
National Center for Atmospheric Research, Boulder, CO.
\phantomsection\label{\detokenize{tech_note/References/CLM50_Tech_Note_References:dickinsonetal2006}}
\\

Dickinson, R.E., Oleson, K.W., Bonan, G., Hoffman, F., Thornton, P.,
Vertenstein, M., Yang, Z.-L., and Zeng, X. 2006. The Community Land
Model and its climate statistics as a component of the Community Climate
System Model. J. Climate 19:2302-2324.
\phantomsection\label{\detokenize{tech_note/References/CLM50_Tech_Note_References:dingman2002}}
\\

Dingman, S.L. 2002. Physical Hydrology. Second Edition. Prentice Hall,
NJ.
\phantomsection\label{\detokenize{tech_note/References/CLM50_Tech_Note_References:dirmeyeretal1999}}
\\

Dirmeyer, P.A., Dolman, A.J., and Sato, N. 1999. The pilot phase of the
Global Soil Wetness Project. Bull. Amer. Meteor. Soc. 80:851-878.
\phantomsection\label{\detokenize{tech_note/References/CLM50_Tech_Note_References:dobsonetal2000}}
\\

Dobson, J.E., Bright, E.A., Coleman, P.R., Durfee, R.C., and Worley,
B.A. 2000. LandScan: A global population database for estimating
populations at risk. Photogramm. Eng. Rem. Sens. 66:849-857.
\phantomsection\label{\detokenize{tech_note/References/CLM50_Tech_Note_References:dormansellers1989}}
\\

Dorman, J.L., and Sellers, P.J. 1989. A global climatology of albedo,
roughness length and stomatal resistance for atmospheric general
circulation models as represented by the simple biosphere model (SiB).
J. Appl. Meteor. 28:833-855.
\phantomsection\label{\detokenize{tech_note/References/CLM50_Tech_Note_References:doughertyetal1994}}
\\

Dougherty, R.L., Bradford, J.A., Coyne, P.I., and Sims, P.L. 1994.
Applying an empirical model of stomatal conductance to three C4 grasses.
Agric. For. Meteor. 67:269-290.
\phantomsection\label{\detokenize{tech_note/References/CLM50_Tech_Note_References:drewniaketal2013}}
\\

Drewniak, B., Song, J., Prell, J., Kotamarthi, V.R., and Jacob, R. 2013.
Modeling agriculture in the Community Land Model. Geosci. Model Dev.
6:495-515. DOI:10.5194/gmd-6-495-2013.
\phantomsection\label{\detokenize{tech_note/References/CLM50_Tech_Note_References:dunfieldetal1993}}
\\

Dunfield, P., Knowles, R., Dumont, R. and Moore, T.R., 1993. Methane
Production and Consumption in Temperate and Sub-Arctic Peat Soils -
Response to Temperature and Ph. Soil Biology \& Biochemistry 25:321-326.
\phantomsection\label{\detokenize{tech_note/References/CLM50_Tech_Note_References:entekhabieagleson1989}}
\\

Entekhabi, D., and Eagleson, P.S. 1989. Land surface hydrology
parameterization for atmospheric general circulation models including
subgrid scale spatial variability. J. Climate 2:816-831.
\phantomsection\label{\detokenize{tech_note/References/CLM50_Tech_Note_References:fangstefan1996}}
\\

Fang, X. and Stefan, H.G., 1996. Long-term lake water temperature and
ice cover simulations/measurements. Cold Regions Science and Technology
24:289-304.
\phantomsection\label{\detokenize{tech_note/References/CLM50_Tech_Note_References:farouki1981}}
\\

Farouki, O.T. 1981. The thermal properties of soils in cold regions.
Cold Regions Sci. and Tech. 5:67-75.
\phantomsection\label{\detokenize{tech_note/References/CLM50_Tech_Note_References:farquharetal1980}}
\\

Farquhar, G.D., von Caemmerer, S., and Berry, J.A. 1980. A biochemical
model of photosynthetic CO$_{\text{2}}$ assimilation in leaves of
C\({}_{3}\) species. Planta 149:78-90.
\phantomsection\label{\detokenize{tech_note/References/CLM50_Tech_Note_References:farquharvoncaemmerer1982}}
\\

Farquhar, G.D., and von Caemmerer, S. 1982. Modeling of photosynthetic
response to environmental conditions. pp. 549-587. In: O.L. Lange, P.S.
Nobel, C.B. Osmond, and H. Zeigler (editors) Encyclopedia of Plant
Physiology. Vol. 12B. Physiological Plant Ecology. II. Water Relations
and Carbon Assimilation. Springer-Verlag, New York.
\phantomsection\label{\detokenize{tech_note/References/CLM50_Tech_Note_References:ferrari1999}}
\\

Ferrari, J.B., 1999. Fine-scale patterns of leaf litterfall and nitrogen
cycling in an old-growth forest. Canadian Journal of Forest Research,
29: 291-302.
\phantomsection\label{\detokenize{tech_note/References/CLM50_Tech_Note_References:firestonedavidson1989}}
\\

Firestone, M.K. and Davidson, E.A. 1989. Exchange of Trace Gases between
Terrestrial Ecosystems and the Atmosphere. In: M.O. Andreae and D.S.
Schimel (Editors). John Wiley and Sons, pp. 7-21.
\phantomsection\label{\detokenize{tech_note/References/CLM50_Tech_Note_References:fisheretal2015}}
\\

Fisher, R. A., S. Muszala, M. Verteinstein, P. Lawrence, C. Xu, N. G. McDowell, R. G. Knox, C. Koven, J. Holm, B. M. Rogers, A. Spessa, D. Lawrence, and G. Bonan, 2015: Taking off the training wheels: the properties of a dynamic vegetation model without climate envelopes, CLM4.5(ED). Geosci. Model Dev., 8: 3593-3619, doi:10.5194/gmd-8-3593-2015.
\phantomsection\label{\detokenize{tech_note/References/CLM50_Tech_Note_References:fisheretal2010}}
\\

Fisher, J. B., S. Sitch, Y. Malhi, R. A. Fisher, C. Huntingford, and S.-Y. Tan, 2010. Carbon cost of plant nitrogen acquisition: A mechanistic, globally applicable model of plant nitrogen uptake, retranslocation, and fixation. Global Biogeochem. Cycles, 24, GB1014, doi:10.1029/2009GB003621.
\phantomsection\label{\detokenize{tech_note/References/CLM50_Tech_Note_References:flannerzender2005}}
\\

Flanner, M.G., and Zender. C.S. 2005. Snowpack radiative heating:
Influence on Tibetan Plateau climate. Geophys. Res. Lett. 32:L06501.
DOI:10.1029/2004GL022076.
\phantomsection\label{\detokenize{tech_note/References/CLM50_Tech_Note_References:flannerzender2006}}
\\

Flanner, M.G., and Zender, C.S. 2006. Linking snowpack microphysics and
albedo evolution. J. Geophys. Res. 111:D12208. DOI:10.1029/2005JD006834.
\phantomsection\label{\detokenize{tech_note/References/CLM50_Tech_Note_References:flanneretal2007}}
\\

Flanner, M.G., Zender, C.S., Randerson, J.T., and Rasch, P.J. 2007.
Present day climate forcing and response from black carbon in snow. J.
Geophys. Res. 112:D11202. DOI:10.1029/2006JD008003.
\phantomsection\label{\detokenize{tech_note/References/CLM50_Tech_Note_References:flatauetal1992}}
\\

Flatau, P.J., Walko, R.L., and Cotton, W.R. 1992. Polynomial fits to
saturation vapor pressure. J. Appl. Meteor. 31:1507-1513.
\phantomsection\label{\detokenize{tech_note/References/CLM50_Tech_Note_References:friedl-etal2002}}
\\

Friedl, M.A., McIver, D.K., Hodges, J.C.F., Zhang, X.Y., Muchoney, D.,
Strahler, A.H., Woodcock, C.E., Gopal, S., Schneider, A., Cooper, A.,
Baccini, A., Gao, F., and Schaaf, C. 2002. Global land cover mapping
from MODIS: algorithms and early results. Remote Sens. Environ.
83:287-302.
\phantomsection\label{\detokenize{tech_note/References/CLM50_Tech_Note_References:frolkingetal2001}}
\\

Frolking, S., et al. 2001. Modeling Northern Peatland Decomposition and
Peat Accumulation. Ecosystems. 4:479-498.
\phantomsection\label{\detokenize{tech_note/References/CLM50_Tech_Note_References:gallaisetal2006}}
\\

Gallais, A., Coque, M. Quillere, I., Prioul, J., and Hirel, B. 2006.
Modeling postsilking nitrogen fluxes in maize (Zea mays) using
15N-labeling field experiments. New Phytologist 172:696-707.
\phantomsection\label{\detokenize{tech_note/References/CLM50_Tech_Note_References:gallaisetal2007}}
\\

Gallais, A., Coque, M., Gouis, J. L., Prioul, J. L., Hirel, B., and
Quillere, I. 2007. Estimating the proportion of nitrogen remobilization
and of postsilking nitrogen uptake allocated to maize kernels by
Nitrogen-15 labeling. Crop Sci. 47:685-693.
\phantomsection\label{\detokenize{tech_note/References/CLM50_Tech_Note_References:gallowayetal2004}}
\\

Galloway, J.N., et al. 2004. Nitrogen cycles: past, present, and future.
Biogeochem. 70:153-226.
\phantomsection\label{\detokenize{tech_note/References/CLM50_Tech_Note_References:garciaetal1988}}
\\

Garcia, R.L., Kanemasu, E.T., Blad, B.L., Bauer, A., Hatfield, J.L.,
Major, D.A., Reginato, R.J., and Hubbard, K.G. 1988. Interception and
use efficiency of light in winter wheat under different nitrogen
regimes. Agric. For. Meteor. 44:175-186.
\phantomsection\label{\detokenize{tech_note/References/CLM50_Tech_Note_References:gardner1960}}
\\

Gardner, W. R. 1960. Dynamic aspects of water availability to plants,
Soil Sci., 89, 63\textendash{}73.
\phantomsection\label{\detokenize{tech_note/References/CLM50_Tech_Note_References:gashetal1996}}
\\

Gash, J.H.C., Nobre, C.A., Roberts, J.M., and Victoria, R.L. 1996. An
overview of ABRACOS. pp. 1-14. In: J.H.C. Gash, C.A. Nobre, J.M.
Roberts, and R.L. Victoria (editors) Amazonian Deforestation and
Climate. John Wiley and Sons, Chichester, England.
\phantomsection\label{\detokenize{tech_note/References/CLM50_Tech_Note_References:getiranaetal2012}}
\\

Getirana, A. C. V., A. Boone, D. Yamazaki, B. Decharme, F. Papa, and
N. Mognard. 2012. The hydrological modeling and analysis platform
(HyMAP): Evaluation in the Amazon basin, J. Hydrometeorol., 13, 1641-1665.
\phantomsection\label{\detokenize{tech_note/References/CLM50_Tech_Note_References:ghimireetal2016}}
\\

Ghimire, B., W. J. Riley, C. D. Koven, M. Mu, and J. T. Randerson, 2016: Representing leaf and root physiological traits in CLM improves global carbon and nitrogen cycling predictions. J. Adv. Mod. Earth Sys. 8: 598-613.
\phantomsection\label{\detokenize{tech_note/References/CLM50_Tech_Note_References:gholzetal1985}}
\\

Gholz, H.L., Perry, C.S., Cropper, W.P., Jr. and Hendry, L.C., 1985.
Litterfall, decomposition, and nitrogen and phosphorous dynamics in a
chronosequence of slash pine (\sphinxstyleemphasis{Pinus elliottii}) plantations. Forest
Science, 31: 463-478.
\phantomsection\label{\detokenize{tech_note/References/CLM50_Tech_Note_References:giglioetal2006}}
\\

Giglio, L., Csiszar, I., and Justice, C.O. 2006. Global distribution and
seasonality of active fires as observed with the Terra and Aqua Moderate
Resolution Imaging Spectroradiometer (MODIS) sensors. J. Geophys. Res.
111:G02016. DOI:10.1029/2005JG000142.
\phantomsection\label{\detokenize{tech_note/References/CLM50_Tech_Note_References:globalsoildatatask2000}}
\\

Global Soil Data Task 2000. Global soil data products CD-ROM (IGBP-DIS).
International Geosphere-Biosphere Programme-Data and Information
Available Services {[}Available online at \sphinxurl{http://www.daac.ornl.gov}{]}.
\phantomsection\label{\detokenize{tech_note/References/CLM50_Tech_Note_References:gomesetal2003}}
\\

Gomes, E.P.C., Mantovani, W., and Kageyama, P.Y. 2003. Mortality and
recruitment of trees in a secondary montane rain forest in southeastern
Brazil. Brazilian Journal of Biology 63:47-60.
\phantomsection\label{\detokenize{tech_note/References/CLM50_Tech_Note_References:goszetal1973}}
\\

Gosz, J.R., Likens, G.E., and Bormann, F.H. 1973. Nutrient release from
decomposing leaf and branch litter in the Hubbard Brook Forest, New
Hampshire. Ecological Monographs 43:173-191.
\phantomsection\label{\detokenize{tech_note/References/CLM50_Tech_Note_References:gotangcocastilloetal2012}}
\\

Gotangco Castillo C., Levis S., and Thornton P. 2012. Evaluation of the
new CNDV option of the Community Land Model: Effects of dynamic
vegetation and interactive nitrogen on CLM4 means and variability. J.
Climate 25:3702-3714. DOI:10.1175/JCLID-11-00372.1.
\phantomsection\label{\detokenize{tech_note/References/CLM50_Tech_Note_References:grahametal1999}}
\\

Graham, S.T., Famiglietti, J.S., and Maidment, D.R. 1999. Five-minute,
1/2º, and 1º data sets of continental watersheds and river networks for
use in regional and global hydrologic and climate system modeling
studies. Water Resour. Res. 35:583-587.
\phantomsection\label{\detokenize{tech_note/References/CLM50_Tech_Note_References:gravenetal2017}}
\\

Graven, H., C. E. Allison, D. M. Etheridge, S. Hammer, R. F. Keeling, I. Levin, H. A. J. Meijer, M. Rubino, P. P. Tans, C. M. Trudinger, B. H. Vaughn and J. W. C. White (2017) Compiled records of carbon isotopes in atmospheric CO2 for historical simulations in CMIP6, Geoscientific Model Development, in review. doi: 10.5194/gmd-2017-166.
\phantomsection\label{\detokenize{tech_note/References/CLM50_Tech_Note_References:grenfellwarren1999}}
\\

Grenfell, T.C., and Warren, S.G. 1999. Representation of a nonspherical
ice particle by a collection of independent spheres for scattering and
absorption of radiation. J. Geophys. Res. 104(D24):37697-37709.
\phantomsection\label{\detokenize{tech_note/References/CLM50_Tech_Note_References:delgrossoetal2000}}
\\

del Grosso, S.J., et al. 2000. General model for N2O and N2 gas
emissions from soils due to dentrification. Global Biogeochem. Cycles
14:1045-1060.
\phantomsection\label{\detokenize{tech_note/References/CLM50_Tech_Note_References:guentheretal1995}}
\\

Guenther, A., Hewitt, C.N., Erickson, D., Fall, R., Geron, C., Graedel,
T., Harley, P., Klinger, L., Lerdau, M., McKay, W.A., Pierce, T.,
Scholes, B., Steinbrecher, R., Tallamraju, R., Taylor, J., and
Zimmerman, P. 1995. A global model of natural volatile organic compound
emissions. J. Geophys. Res. 100:8873-8892.
\phantomsection\label{\detokenize{tech_note/References/CLM50_Tech_Note_References:guentheretal2006}}
\\

Guenther, A., Karl, T., Harley, P., Wiedinmyer, C., Palmer. P.I., and
Geron, C. 2006. Estimates of global terrestrial isoprene emissions using
MEGAN (Model of Emissions of Gases and Aerosols from Nature). Atmos.
Chem. Phys. 6:3181\textendash{}3210.
\phantomsection\label{\detokenize{tech_note/References/CLM50_Tech_Note_References:guentheretal2012}}
\\

Guenther, A. B., Jiang, X., Heald, C. L., Sakulyanontvittaya, T., Duhl,
T., Emmons, L. K., \& Wang, X., 2012. The Model of Emissions of Gases and
Aerosols from Nature version 2.1 (MEGAN2.1): an extended and updated
framework for modeling biogenic emissions, Geosci. Model Dev., 5,
1471\textendash{}1492. DOI:10.5194.
\phantomsection\label{\detokenize{tech_note/References/CLM50_Tech_Note_References:hacketal2006}}
\\

Hack, J.J., Caron, J.M., Yeager, S.G., Oleson, K.W., Holland, M.M.,
Truesdale, J.E., and Rasch, P.J. 2006. Simulation of the global
hydrological cycle in the CCSM Community Atmosphere Model version 3
(CAM3): mean features. J. Climate 19:2199-2221.
\phantomsection\label{\detokenize{tech_note/References/CLM50_Tech_Note_References:hansenetal2003}}
\\

Hansen, M., DeFries, R.S., Townshend, J.R.G., Carroll, M., Dimiceli, C.,
and Sohlberg, R.A. 2003. Global percent tree cover at a spatial
resolution of 500 meters: first results of the MODIS vegetation
continuous fields algorithm. Earth Interactions 7:1-15.
\phantomsection\label{\detokenize{tech_note/References/CLM50_Tech_Note_References:hastingsetal1999}}
\\

Hastings, D.A., Dunbar, P.K., Elphingstone, G.M., Bootz, M., Murakami,
H., Maruyama, H., Masaharu, H., Holland, P., Payne, J., Bryant, N.A.,
Logan, T.L., Muller, J.-P., Schreier, G., and MacDonald, J.S., eds.,
1999. The Global Land One-kilometer Base Elevation (GLOBE) Digital
Elevation Model, Version 1.0. National Oceanic and Atmospheric
Administration, National Geophysical Data Center, 325 Broadway, Boulder,
Colorado 80305-3328, U.S.A.
\phantomsection\label{\detokenize{tech_note/References/CLM50_Tech_Note_References:healdetal2008}}
\\

Heald, C.L., Henze, D.K., Horowitz, L.W., Feddema, J., Lamarque, J.-F.,
Guenther, A., Hess, P.G., Vitt, F., Seinfeld, J.H., Goldstein, A.H., and
Fung, I. 2008. Predicted change in global secondary organic aerosol
concentrations in response to future climate, emissions, and land use
change. J. Geophys. Res. 113:D05211. DOI:10.1029/2007JD009092.
\phantomsection\label{\detokenize{tech_note/References/CLM50_Tech_Note_References:healdetal2009}}
\\

Heald, C.L., Wilkinson, M.J., Monson, R.K., Alo, C.A., Wang, G.L., and
Guenther, A. 2009. Response of isoprene emission to ambient
CO$_{\text{2}}$ changes and implications for global budgets. Global
Change Biol. 15:1127-1140. DOI:10.1111/j.1365-2486.2008.01802.x
\phantomsection\label{\detokenize{tech_note/References/CLM50_Tech_Note_References:henderson-sellers1985}}
\\

Henderson-Sellers, B. 1985. New formulation of eddy diffusion
thermocline models. Appl. Math. Modelling 9:441-446.
\phantomsection\label{\detokenize{tech_note/References/CLM50_Tech_Note_References:henderson-sellers1986}}
\\

Henderson-Sellers, B. 1986. Calculating the surface energy balance for
lake and reservoir modeling: A review. Rev. Geophys. 24:625-649.
\phantomsection\label{\detokenize{tech_note/References/CLM50_Tech_Note_References:henderson-sellersetal1993}}
\\

Henderson-Sellers, A., Yang, Z.-L., and Dickinson, R.E. 1993. The
project for intercomparison of land-surface parameterization schemes.
Bull. Amer. Meteor. Soc. 74: 1335-1349.
\phantomsection\label{\detokenize{tech_note/References/CLM50_Tech_Note_References:hostetlerbartlein1990}}
\\

Hostetler, S.W., and Bartlein, P.J. 1990. Simulation of lake evaporation
with application to modeling lake level variations of Harney-Malheur
Lake, Oregon. Water Resour. Res. 26:2603-2612.
\phantomsection\label{\detokenize{tech_note/References/CLM50_Tech_Note_References:hostetleretal1993}}
\\

Hostetler, S.W., Bates, G.T., and Giorgi, F. 1993. Interactive coupling
of a lake thermal model with a regional climate model. J. Geophys. Res.
98:5045-5057.
\phantomsection\label{\detokenize{tech_note/References/CLM50_Tech_Note_References:hostetleretal1994}}
\\

Hostetler, S.W., Giorgi, F., Bates, G.T., and Bartlein, P.J. 1994.
Lake-atmosphere feedbacks associated with paleolakes Bonneville and
Lahontan. Science 263:665-668.
\phantomsection\label{\detokenize{tech_note/References/CLM50_Tech_Note_References:houetal2012}}
\\

Hou, Z., Huang, M., Leung, L.R., Lin, G., and Ricciuto, D.M. 2012.
Sensitivity of surface flux simulations to hydrologic parameters based
on an uncertainty quantification framework applied to the Community Land
Model. J. Geophys. Res. 117:D15108.
\phantomsection\label{\detokenize{tech_note/References/CLM50_Tech_Note_References:houltonetal2008}}
\\

Houlton, B.Z., Wang, Y.P., Vitousek, P.M. and Field, C.B., 2008. A unifying framework for dinitrogen fixation in the terrestrial biosphere. Nature, 454(7202), p.327.
\phantomsection\label{\detokenize{tech_note/References/CLM50_Tech_Note_References:huangliang2006}}
\\

Huang, M., and Liang, X. 2006. On the assessment of the impact of
reducing parameters and identification of parameter uncertainties for a
hydrologic model with applications to ungauged basins. J. Hydrol.
320:37-61.
\phantomsection\label{\detokenize{tech_note/References/CLM50_Tech_Note_References:hugeliusetal2012}}
\\

Hugelius, G., C. Tarnocai, G. Broll, J.G. Canadell, P. Kuhry, adn D.K.
Swanson, 2012. The Northern Circumpolar Soil Carbon Database: spatially
distributed datasets of soil coverage and soil carbon storage in the
northern permafrost regions. Earth Syst. Sci. Data Discuss., 5, 707-733
(available online at (\sphinxurl{http://dev1.geo.su.se/bbcc/dev/ncscd/}).
\phantomsection\label{\detokenize{tech_note/References/CLM50_Tech_Note_References:huntetal1988}}
\\

Hunt, H.W., Ingham, E.R., Coleman, D.C., Elliott, E.T., and Reid, C.P.P.
1988. Nitrogen limitation of production and decomposition in prairie,
mountain meadow, and pine forest. Ecology 69:1009-1016.
\phantomsection\label{\detokenize{tech_note/References/CLM50_Tech_Note_References:huntrunning1992}}
\\

Hunt, E.R., Jr. and Running, S.W., 1992. Simulated dry matter yields for
aspen and spruce stands in the north american boreal forest. Canadian
Journal of Remote Sensing, 18: 126-133.
\phantomsection\label{\detokenize{tech_note/References/CLM50_Tech_Note_References:huntetal1996}}
\\

Hunt, E.R., Jr. et al., 1996. Global net carbon exchange and
intra-annual atmospheric CO$_{\text{2}}$ concentrations predicted by
an ecosystem process model and three-dimensional atmospheric transport
model. Global Biogeochemical Cycles, 10: 431-456.
\phantomsection\label{\detokenize{tech_note/References/CLM50_Tech_Note_References:hurttetal2006}}
\\

Hurtt, G.C., Frolking, S., Fearon, M.G., Moore, B., Shevliakova, E.,
Malyshev, S., Pacala, S.W., and Houghton, R.A. 2006. The underpinnings
of land-use history: three centuries of global gridded land-use
transitions, wood-harvest activity, and resulting secondary lands.
Global Change Biol. 12:1208-1229.
\phantomsection\label{\detokenize{tech_note/References/CLM50_Tech_Note_References:hurttetal2011}}
\\

Hurtt, G.C., et al. 2011. Harmonization of land-use scenarios for the
period 1500-2100: 600 years of global gridded annual land-use
transitions, wood harvest, and resulting secondary lands. Climatic
Change 109:117-161. DOI:10.1007/s10584-011-0153-2.
\phantomsection\label{\detokenize{tech_note/References/CLM50_Tech_Note_References:idso1981}}
\\

Idso, S.B. 1981. A set of equations for full spectrum and 8- to
14-\(\mu\) m and 10.5- to 12.5-\(\mu\) m thermal
radiation from cloudless skies. Water Resour. Res. 17:295-304.
\phantomsection\label{\detokenize{tech_note/References/CLM50_Tech_Note_References:iiyamahasegawa2005}}
\\

Iiyama, I. and Hasegawa, S., 2005. Gas diffusion coefficient of
undisturbed peat soils. Soil Science and Plant Nutrition 51:431-435.
\phantomsection\label{\detokenize{tech_note/References/CLM50_Tech_Note_References:jacksonetal1996}}
\\

Jacksonetal1996:
E., and Schulze, E. D. 1996. A global analysis of root distributions for
terrestrial biomes Oecologia 108:389\textendash{}411. DOI:10.1007/BF00333714.
\phantomsection\label{\detokenize{tech_note/References/CLM50_Tech_Note_References:jacksonetal2010}}
\\

Jackson, T.L., Feddema, J.J., Oleson, K.W., Bonan, G.B., and Bauer, J.T.
2010. Parameterization of urban characteristics for global climate
modeling. Annals of the Association of American Geographers.
100:848-865.
\phantomsection\label{\detokenize{tech_note/References/CLM50_Tech_Note_References:jenkinsoncoleman2008}}
\\

Jenkinson, D. and Coleman, K. 2008. The turnover of organic carbon in
subsoils. Part 2. Modelling carbon turnover. European Journal of Soil
Science 59:400-413.
\phantomsection\label{\detokenize{tech_note/References/CLM50_Tech_Note_References:jordan1991}}
\\

Jordan, R. 1991. A One-dimensional Temperature Model for a Snow Cover:
Technical Documentation for SNTHERM.89. U.S. Army Cold Regions Research
and Engineering Laboratory, Special Report 91-16.
\phantomsection\label{\detokenize{tech_note/References/CLM50_Tech_Note_References:kattgeknorr2007}}
\\

Kattge, J., and Knorr, W. 2007. Temperature acclimation in a biochemical
model of photosynthesis: a reanalysis of data from 36 species. Plant
Cell Environ. 30:1176-1190. DOI:10.1111/j.1365-3040.2007.01690.x.
\phantomsection\label{\detokenize{tech_note/References/CLM50_Tech_Note_References:kattgeetal2009}}
\\

Kattge, J., Knorr, W., Raddatz, T., and Wirth C. 2009: Quantifying
photosynthetic capacity and its relationship to leaf nitrogen content
for global\textendash{}scale terrestrial biosphere models. Global Change Biol.
15:976\textendash{}991.
\phantomsection\label{\detokenize{tech_note/References/CLM50_Tech_Note_References:kavetskietal2002}}
\\

Kavetski, D., Binning, P. and Sloan, S.W., 2002. Noniterative time
stepping schemes with  adaptive truncation error control for the
solution of Richards equation. Water Resources  Research, 38(10).
\phantomsection\label{\detokenize{tech_note/References/CLM50_Tech_Note_References:kelleretal2004}}
\\

Keller, M., Palace, M., Asner, G.P., Pereira, R., Jr. and Silva, J.N.M.,
2004. Coarse woody debris in undisturbed and logged forests in the
eastern Brazilian Amazon. Global Change Biology, 10: 784-795.
\phantomsection\label{\detokenize{tech_note/References/CLM50_Tech_Note_References:kellneretal2006}}
\\

Kellner, E., Baird, A.J., Oosterwoud, M., Harrison, K. and Waddington,
J.M., 2006. Effect of temperature and atmospheric pressure on methane
(CH4) ebullition from near-surface peats. Geophys. Res. Lett. 33.
DOI:10.1029/2006GL027509.
\phantomsection\label{\detokenize{tech_note/References/CLM50_Tech_Note_References:kimballetal1997}}
\\

Kimball, J.S., Thornton, P.E., White, M.A. and Running, S.W. 1997.
Simulating forest productivity and surface-atmosphere exchange in the
BOREAS study region. Tree Physiology 17:589-599.
\phantomsection\label{\detokenize{tech_note/References/CLM50_Tech_Note_References:kohyamaetal2001}}
\\

Kohyama, T., Suzuki, E., Partomihardjo, T., and Yamada, T. 2001. Dynamic
steady state of patch-mosaic tree size structure of a mixed diptocarp
forest regulated by local crowding. Ecological Research 16:85-98.
\phantomsection\label{\detokenize{tech_note/References/CLM50_Tech_Note_References:kourzeneva2009}}
\\

Kourzeneva, E., 2009. Global dataset for the parameterization of lakes
in Numerical Weather Prediction and Climate modeling. ALADIN Newsletter,
No 37, July-December, 2009, F. Bouttier and C. Fischer, Eds.,
Meteo-France, Toulouse, France, 46-53.
\phantomsection\label{\detokenize{tech_note/References/CLM50_Tech_Note_References:kourzeneva2010}}
\\

Kourzeneva, E., 2010: External data for lake parameterization in
Numerical Weather Prediction and climate modeling. Boreal Environment
Research, 15, 165-177.
\phantomsection\label{\detokenize{tech_note/References/CLM50_Tech_Note_References:kourzenevaetal2012}}
\\

Kourzeneva, E., Asensio, H., Martin, E. and Faroux, S., 2012. Global
gridded dataset of lake coverage and lake depth for use in numerical
weather prediction and climate modelling. Tellus A 64.
\phantomsection\label{\detokenize{tech_note/References/CLM50_Tech_Note_References:kovenetal2009}}
\\

Koven, C., et al. 2009. On the formation of high-latitude soil carbon
stocks: The effects of cryoturbation and insulation by organic matter in
a land surface model. Geophys. Res. Lett. 36: L21501.
\phantomsection\label{\detokenize{tech_note/References/CLM50_Tech_Note_References:kovenetal2011}}
\\

Koven, C.D., et al. 2011. Permafrost carbon-climate feedbacks accelerate
global warming. Proceedings of the National Academy of Sciences
108:14769-14774.
\phantomsection\label{\detokenize{tech_note/References/CLM50_Tech_Note_References:kovenetal2013}}
\\

Koven, C.D. et al. 2013. The effect of vertically-resolved soil
biogeochemistry and alternate soil C and N models on C dynamics of CLM4.
Biogeosciences Discussions 10:7201-7256.
\phantomsection\label{\detokenize{tech_note/References/CLM50_Tech_Note_References:kovenetal2015}}
\\

Koven, C.D. et al. 2015. Permafrost carbon-climate feedback is
sensitive to deep soil carbon decomposability but not deep soil
nitrogen dynamics. Proceedings of the National Academies of Science,
112, 12, 3752-3757, doi:10.1073/pnas.1415123112
\phantomsection\label{\detokenize{tech_note/References/CLM50_Tech_Note_References:kovenetal2017}}
\\

Koven, C.D., G. Hugelius, D.M. Lawrence, and W.R. Wieder, 2017: Higher climatological temperature sensitivity of soil carbon in cold than warm climates. Nature Clim. Change, 7, doi:10.1038/nclimate3421.
\phantomsection\label{\detokenize{tech_note/References/CLM50_Tech_Note_References:kuchariketal2000}}
\\

Kucharik, C.J., Foley, J.A., Delire, C., Fisher, V.A., Coe, M.T.,
Lenters, J.D., Young-Molling, C., and Ramankutty, N. 2000. Testing the
performance of a dynamic global ecosystem model: water balance, carbon
balance, and vegetation structure. Global Biogeochem. Cycles 14:
795\textendash{}825.
\phantomsection\label{\detokenize{tech_note/References/CLM50_Tech_Note_References:kucharikbrye2003}}
\\

Kucharik, C.J., and Brye, K.R. 2003. Integrated BIosphere Simulator
(IBIS) yield and nitrate loss predictions for Wisconsin maize receiving
varied amounts of nitrogen fertilizer. Journal of Environmental Quality
32: 247\textendash{}268.
\phantomsection\label{\detokenize{tech_note/References/CLM50_Tech_Note_References:laddetal2992}}
\\

Ladd, J.N., Jocteur-Monrozier, L. and Amato, M., 1992. Carbon turnover
and nitrogen transformations in an alfisol and vertisol amended with
{[}U-\({}^{14}\)C{]} glucose and {[}\({}^{15}\)N{]} ammonium
sulfate. Soil Biology and Biochemistry, 24: 359-371.
\phantomsection\label{\detokenize{tech_note/References/CLM50_Tech_Note_References:lamarqueetal2010}}
\\

Lamarque, J.-F., et al. 2010. Historical (1850-2000) gridded
anthropogenic and biomass burning emissions of reactive gases and
aerosols: methodology and application. Atmos. Chem. Phys. Discuss.
10:4963-5019. DOI:10.5194/acpd-10-4963-2010.
\phantomsection\label{\detokenize{tech_note/References/CLM50_Tech_Note_References:larcher1995}}
\\

Larcher, W. 1995. Physiological Plant Ecology, Springer-Verlag, Berlin
Heidelberg.
\phantomsection\label{\detokenize{tech_note/References/CLM50_Tech_Note_References:lavigneryan1997}}
\\

Lavigne, M.B., and Ryan, M.G. 1997. Growth and maintenance respiration
rates of aspen, black spruce, and jack pine stems at northern and
southern BOREAS sites. Tree Phys. 17:543-551.
\phantomsection\label{\detokenize{tech_note/References/CLM50_Tech_Note_References:lawetal2003}}
\\

Law, B.E., Sun, O.J., Campbell, J., Van Tuyl, S. and Thornton, P.E.
2003. Changes in carbon storage and fluxes in a chronosequence of
ponderosa pine. Global Change Biology, 9: 510-514.
\phantomsection\label{\detokenize{tech_note/References/CLM50_Tech_Note_References:lawrenceetal2007}}
\\

Lawrence, D.M., Thornton, P.E., Oleson, K.W., and Bonan, G.B. 2007. The
partitioning of evapotranspiration into transpiration, soil evaporation,
and canopy evaporation in a GCM: Impacts on land-atmosphere interaction.
J. Hydrometeor. 8:862-880.
\phantomsection\label{\detokenize{tech_note/References/CLM50_Tech_Note_References:lawrenceslater2008}}
\\

Lawrence, D.M., and Slater, A.G. 2008. Incorporating organic soil into a
global climate model. Clim. Dyn. 30. DOI:10.1007/s00382-007-0278-1.
\phantomsection\label{\detokenize{tech_note/References/CLM50_Tech_Note_References:lawrenceetal2008}}
\\

Lawrence, D.M., Slater, A.G., Romanovsky, V.E., and Nicolsky, D.J. 2008.
The sensitivity of a model projection of near-surface permafrost
degradation to soil column depth and inclusion of soil organic matter.
J. Geophys. Res. 113:F02011. DOI:10.1029/2007JF000883.
\phantomsection\label{\detokenize{tech_note/References/CLM50_Tech_Note_References:lawrenceetal2011}}
\\

Lawrence, D.M., K.W. Oleson, M.G. Flanner, P.E. Thornton, S.C. Swenson,
P.J. Lawrence, X. Zeng, Z.-L. Yang, S. Levis, K. Sakaguchi, G.B. Bonan,
and A.G. Slater, 2011. Parameterization improvements and functional and
structural advances in version 4 of the Community Land Model. J. Adv.
Model. Earth Sys. 3. DOI:10.1029/2011MS000045.
\phantomsection\label{\detokenize{tech_note/References/CLM50_Tech_Note_References:lawrenceetal2016}}
\\

Lawrence, D.M., Hurtt, G.C., Arneth, A., Brovkin, V., Calvin, K.V.,
Jones, A.D., Jones, C.D., Lawrence, P.J., de Noblet-Ducoudré, N., Pongratz,
J., Seneviratne, S.I., and Shevliakova, E. 2016. The Land Use Model
Intercomparison Project (LUMIP) contribution to CMIP6: rationale
and experimental design. Geosci. Model Dev. 9:2973-2998.
DOI:10.5194/gmd-9-2973-2016.
\phantomsection\label{\detokenize{tech_note/References/CLM50_Tech_Note_References:lawrencechase2007}}
\\

Lawrence, P.J., and Chase, T.N. 2007. Representing a MODIS consistent
land surface in the Community Land Model (CLM 3.0). J. Geophys. Res.
112:G01023. DOI:10.1029/2006JG000168.
\phantomsection\label{\detokenize{tech_note/References/CLM50_Tech_Note_References:lawrencechase2010}}
\\

Lawrence, P.J., and Chase, T.N. 2010. Investigating the climate impacts
of global land cover change in the Community Climate System Model. Int.
J. Climatol. 30:2066-2087. DOI:10.1002/joc.2061.
\phantomsection\label{\detokenize{tech_note/References/CLM50_Tech_Note_References:lawrenceetal2012}}
\\

Lawrence, P.J., et al. 2012. Simulating the biogeochemical and
biogeophysical impacts of transient land cover change and wood harvest
in the Community Climate System Model (CCSM4) from 1850 to 2100. J.
Climate 25:3071-3095. DOI:10.1175/JCLI-D-11-00256.1.
\phantomsection\label{\detokenize{tech_note/References/CLM50_Tech_Note_References:lehnerdoll2004}}
\\

Lehner, B. and Döll, P., 2004. Development and validation of a global
database of lakes, reservoirs and wetlands, J. Hydrol., 296, 1\textendash{}22.
\phantomsection\label{\detokenize{tech_note/References/CLM50_Tech_Note_References:lehneretal2008}}
\\

Lehner, B., Verdin, K. and Jarvis, A., 2008. New global hydrograhy
derived from spaceborne elevation data. Eos Trans., AGU, 89, 93 \textendash{} 94.
\phantomsection\label{\detokenize{tech_note/References/CLM50_Tech_Note_References:lepageetal2010}}
\\

Le Page, Y., van der Werf, G.R., Morton, D.C., and Pereira, J.M.C. 2010.
Modeling fire-driven deforestation potential in Amazonia under current
and projected climate conditions. J. Geophys. Res. 115:G03012.
DOI:10.1029/2009JG001190.
\phantomsection\label{\detokenize{tech_note/References/CLM50_Tech_Note_References:lerman1979}}
\\

Lerman, A., 1979. Geochemical processes: Water and sediment
environments. John Wiley and Sons, New York, N.Y.
\phantomsection\label{\detokenize{tech_note/References/CLM50_Tech_Note_References:lettsetal2000}}
\\

Letts, M.G., Roulet, N.T., Comer, N.T., Skarupa, M.R., and Verseghy,
D.L. 2000. Parametrization of peatland hydraulic properties for the
Canadian Land Surface Scheme. Atmos.-Ocean 38:141-160.
\phantomsection\label{\detokenize{tech_note/References/CLM50_Tech_Note_References:levisetal2003}}
\\

Levis, S., Wiedinmyer, C., Bonan, G.B., and Guenther, A. 2003.
Simulating biogenic volatile organic compound emissions in the Community
Climate System Model. J. Geophys. Res. 108:4659.
DOI:10.1029/2002JD003203.
\phantomsection\label{\detokenize{tech_note/References/CLM50_Tech_Note_References:levisetal2004}}
\\

Levis, S., Bonan, G.B., Vertenstein, M., and Oleson, K.W. 2004. The
community land model’s dynamic global vegetation model (CLM-DGVM):
technical description and user’s guide. NCAR Technical Note
NCAR/TN-459+STR. National Center for Atmospheric Research, Boulder,
Colorado. 50 pp.
\phantomsection\label{\detokenize{tech_note/References/CLM50_Tech_Note_References:levisetal2009}}
\\

Levis, S., Thornton, P., Bonan, G., and Kucharik, C. 2009. Modeling land
use and land management with the Community Land Model. iLeaps
newsletter, No. 7.
\phantomsection\label{\detokenize{tech_note/References/CLM50_Tech_Note_References:levisetal2012}}
\\

Levis, S., Bonan, G., Kluzek, E., Thornton, P., Jones, A., Sacks, W.,
and Kucharik, C 2012. Interactive crop management in the Community Earth
System Model (CESM1): Seasonal influences on land-atmosphere fluxes. J.
Climate 25: 4839-4859. DOI:10.1175/JCLI-D-11-00446.1.
\phantomsection\label{\detokenize{tech_note/References/CLM50_Tech_Note_References:levisetal2016}}
\\

Levis, S., Badger, A., Drewniak, B., Nevison, C., Ren, X. 2016. CLMcrop
yields and water requirements: avoided impacts by choosing RCP 4.5 over 8.5.
Climatic Change. DOI:10.1007/s10584-016-1654-9.
\phantomsection\label{\detokenize{tech_note/References/CLM50_Tech_Note_References:lietal2000}}
\\

Li, C., Aber, J., Stange, F., Butterbach-Bahl, K. and Papen, H. 2000. A
process-oriented model of N2O and NO emissions from forest soils: 1.
Model development. J. Geophys. Res. 105(D4):4369-4384.
\phantomsection\label{\detokenize{tech_note/References/CLM50_Tech_Note_References:lietal2012a}}
\\

Li, F., Zeng, X.-D., and Levis, S. 2012a. A process-based fire
parameterization of intermediate complexity in a Dynamic Global
Vegetation Model. Biogeosciences 9:2761-2780.
\phantomsection\label{\detokenize{tech_note/References/CLM50_Tech_Note_References:lietal2012b}}
\\

Li, F., Zeng, X. D., and Levis, S. 2012b. Corrigendum to “A
process-based fire parameterization of intermediate complexity in a
Dynamic Global Vegetation Model” published in Biogeosciences, 9,
2761\textendash{}2780, 2012”. Biogeosciences 9: 4771-4772.
\phantomsection\label{\detokenize{tech_note/References/CLM50_Tech_Note_References:lietal2013a}}
\\

Li, F., Levis, S., and Ward, D. S. 2013a. Quantifying the role of fire
in the Earth system \textendash{} Part 1: Improved global fire modeling in the
Community Earth System Model (CESM1). Biogeosciences 10:2293-2314.
\phantomsection\label{\detokenize{tech_note/References/CLM50_Tech_Note_References:lilawrence2017}}
\\

Li, F., and Lawrence, D. 2017. Role of fire in the global land water
budget during the 20th century through changing ecosystems.
J. Clim. 30: 1894-1908.
\phantomsection\label{\detokenize{tech_note/References/CLM50_Tech_Note_References:lietal2013b}}
\\

Li, H.-Y., Huang, M., Tesfa, T., Ke, Y., Sun, Y., Liu, Y., and Leung, L.
R. 2013b. A subbasin-based framework to represent land surface processes
in an Earth System Model, Geosci. Model Dev. Discuss. 6:2699-2730.
DOI:10.5194/gmdd-6-2699-2013.
\phantomsection\label{\detokenize{tech_note/References/CLM50_Tech_Note_References:lietal2011}}
\\

Li, H., Huang, M., Wigmosta, M.S., Ke, Y., Coleman, A.M., Leung, L.R.,
Wang, A., and Ricciuto, D.M. 2011. Evaluating runoff simulations from
the Community Land Model 4.0 using observations from flux towers and a
mountainous watershed. J. Geophys. Res. 116:D24120.
DOI:10.1029/2011JD016276.
\phantomsection\label{\detokenize{tech_note/References/CLM50_Tech_Note_References:lietal2015a}}
\\

Li, H., L. Leung, A. Getirana, M. Huang, H. Wu, Y. Xu, J. Guo and
N. Voisin. 2015a. Evaluating global streamflow simulations by a
physically-based routing model coupled with the Community Land Model,
J. of Hydromet., 16(2):948-971, doi: 10.1175/JHM-D-14-0079.1
\phantomsection\label{\detokenize{tech_note/References/CLM50_Tech_Note_References:lietal2015b}}
\\

Li, H., L. Leung, T. Tesfa, N. Voisin, M. Hejazi, L. Liu, Y. Liu,
J. Rice, H. Wu, and X. Yang. 2015. Modeling stream temperature in the
Anthropocene: An earth system modeling approach, J. Adv. Model.
Earth Syst., 7, doi:10.1002/2015MS000471.
\phantomsection\label{\detokenize{tech_note/References/CLM50_Tech_Note_References:liangetal1994}}
\\

Liang, X., Lettenmaier, D.P., Wood, E.F., and Burges, S.J. 1994. A
simple hydrologically based model of land surface water and energy
fluxes for GSMs. J. Geophys. Res. 99(D7):14,415\textendash{}14,428.
\phantomsection\label{\detokenize{tech_note/References/CLM50_Tech_Note_References:lipscombsacks2012}}
\\

Lipscomb, W., and Sacks, W. 2012. The CESM land ice model documentation
and user’s guide. 46 pp. {[}Available online at
\sphinxurl{http://www.cesm.ucar.edu/models/cesm1.1/cism/}{]}.
\phantomsection\label{\detokenize{tech_note/References/CLM50_Tech_Note_References:lloydtaylor1994}}
\\

Lloyd, J. and Taylor, J.A., 1994. On the temperature dependence of soil
respiration. Functional Ecology, 8: 315-323.
\phantomsection\label{\detokenize{tech_note/References/CLM50_Tech_Note_References:lloydetal2010}}
\\

Lloyd, J., et al. 2010. Optimisation of photosynthetic carbon gain and
within-canopy gradients of associated foliar traits for Amazon forest
trees. Biogeosci. 7:1833-1859. DOI:10.5194/bg-7-1833-2010.
\phantomsection\label{\detokenize{tech_note/References/CLM50_Tech_Note_References:lobelletal2006}}
\\

Lobell, D.B., Bala, G., and Duffy, P.B. 2006. Biogeophysical impacts of
cropland management changes on climate. Geophys. Res. Lett. 33:L06708.
DOI:10.1029/2005GL025492.
\phantomsection\label{\detokenize{tech_note/References/CLM50_Tech_Note_References:lombardozzietal2015}}
\\

Lombardozzi, D.L., Bonan, G.B., Smith, N.G., Dukes, J.S. 2015. Temperature
acclimation of photosynthesis and respiration: A key uncertainty in the
carbon cycle-climate feedback. Geophys. Res. Lett. 42:8624-8631.
\phantomsection\label{\detokenize{tech_note/References/CLM50_Tech_Note_References:lovelandetal2000}}
\\

Loveland, T.R., Reed, B.C., Brown, J.F., Ohlen, D.O., Zhu, Z., Yang, L.,
and Merchant, J.W. 2000. Development of a global land cover
characteristics database and IGBP DISCover from 1 km AVHRR data. Int. J.
Remote Sens. 21:1303-1330.
\phantomsection\label{\detokenize{tech_note/References/CLM50_Tech_Note_References:lowe1977}}
\\

Lowe, P.R. 1977. An approximating polynomial for the computation of
saturation vapor pressure. J. Appl. Meteor. 16:100-103.
\phantomsection\label{\detokenize{tech_note/References/CLM50_Tech_Note_References:luoetal2006}}
\\

Luo, Y., Hui, D., and Zhang, D. 2006. Elevated CO2 stimulates net
accumulations of carbon and nitrogen in land ecosystems: a
meta-analysis. Ecology 87:53-63.
\phantomsection\label{\detokenize{tech_note/References/CLM50_Tech_Note_References:magilletal1997}}
\\

Magill, A.H. et al., 1997. Biogeochemical response of forest ecosystems
to simulated chronic nitrogen deposition. Ecological Applications, 7:
402-415.
\phantomsection\label{\detokenize{tech_note/References/CLM50_Tech_Note_References:mahowaldetal2006}}
\\

Mahowald, N.M., Muhs, D.R., Levis, S., Rasch, P.J., Yoshioka, M.,
Zender, C.S., and Luo, C. 2006. Change in atmospheric mineral aerosols
in response to climate: last glacial period, pre-industrial, modern and
doubled CO$_{\text{2}}$ climates. J. Geophys. Res\sphinxstyleemphasis{.} 111:D10202.
DOI:10.1029/2005JD006653.
\phantomsection\label{\detokenize{tech_note/References/CLM50_Tech_Note_References:makela2002}}
\\

Makela, A. 2002. Derivation of stem taper from the pipe model theory in
a carbon balance framework. Tree Phys. 22:891-905.
\phantomsection\label{\detokenize{tech_note/References/CLM50_Tech_Note_References:maoetal2012}}
\\

Mao, J., Thornton, P.E., Shi, X., Zhao, M., and Post, W.M. 2012. Remote
sensing evaluation of CLM4 GPP for the period 2000 to 2009. J. Climate
25:5327-5342.
\phantomsection\label{\detokenize{tech_note/References/CLM50_Tech_Note_References:maoetal2013}}
\\

Mao, J., Shi, X., Thornton, P.E., Hoffman, F.M., Zhu, Z., and Ranga B.
Myneni, R.B. 2013. Global latitudinal-asymmetric vegetation growth
trends and their driving mechanisms: 1982-2009. Remote Sensing
5:1484-1497.
\phantomsection\label{\detokenize{tech_note/References/CLM50_Tech_Note_References:martinetal1980}}
\\

Martin, J.P., Haider, K. and Kassim, G., 1980. Biodegradation and
stabilization after 2 years of specific crop, lignin, and polysaccharide
carbons in soils. Soil Science Society of America Journal 44:1250-1255.
\phantomsection\label{\detokenize{tech_note/References/CLM50_Tech_Note_References:maryetal1993}}
\\

Mary, B., Fresneau, C., Morel, J.L. and Mariotti, A., 1993. C and N
cycling during decomposition of root mucilage, roots and glucose in
soil. Soil Biology and Biochemistry 25:1005-1014.
\phantomsection\label{\detokenize{tech_note/References/CLM50_Tech_Note_References:mcguireetal1992}}
\\

McGuire, A.D., Melillo, J.M., Joyce, L.A., Kicklighter, D.W., Grace,
A.L., Moore III, B., and Vorosmarty, C.J. 1992. Interactions between
carbon and nitrogen dynamics in estimating net primary productivity for
potential vegetation in North America. Global Biogeochem. Cycles
6:101-124.
\phantomsection\label{\detokenize{tech_note/References/CLM50_Tech_Note_References:medlynetal2011}}
\\

Medlyn, B.E., Duursma, R.A., Eamus, D., Ellsworth, D.S., Prentice, I.C.,
Barton, C.V.M., Crous, K.Y., De Angelis, P., Freeman, M., and
Wingate, L. (2011), Reconciling the optimal and empirical approaches to
modelling stomatal conductance. Global Change Biology, 17: 2134\textendash{}2144.
doi:10.1111/j.1365-2486.2010.02375.x
\phantomsection\label{\detokenize{tech_note/References/CLM50_Tech_Note_References:melzeroleary1987}}
\\

Melzer, E., and O’Leary, M.H. 1987. Anapleurotic CO2 Fixation by
Phosphoenolpyruvate Carboxylase in C3 Plants. Plant. Physiol. 84:58.
\phantomsection\label{\detokenize{tech_note/References/CLM50_Tech_Note_References:milleretal1994}}
\\

Miller, J.R., Russell, G.L., and Caliri, G. 1994. Continental-scale
river flow in climate models. J. Climate 7:914-928.
\phantomsection\label{\detokenize{tech_note/References/CLM50_Tech_Note_References:millingtonquirk1961}}
\\

Millington, R. and Quirk, J.P., 1961. Permeability of Porous Solids.
Transactions of the Faraday Society 57:1200-1207.
\phantomsection\label{\detokenize{tech_note/References/CLM50_Tech_Note_References:mironovetal2010}}
\\

Mironov, D. et al., 2010. Implementation of the lake parameterisation
scheme FLake into the numerical weather prediction model COSMO. Boreal
Environment Research 15:218-230.
\phantomsection\label{\detokenize{tech_note/References/CLM50_Tech_Note_References:mitchelljones2005}}
\\

Mitchell, T.D., and Jones, P.D. 2005. An improved method of constructing
a database of monthly climate observations and associated
high-resolution grids. Int. J. Climatol. 25:693-712.
\phantomsection\label{\detokenize{tech_note/References/CLM50_Tech_Note_References:moldrupetal2003}}
\\

Moldrup, P. et al. 2003. Modeling diffusion and reaction in soils: X. A
unifying model for solute and gas diffusivity in unsaturated soil. Soil
Science 168:321-337.
\phantomsection\label{\detokenize{tech_note/References/CLM50_Tech_Note_References:mynenietal2002}}
\\

Myneni, R.B., et al. 2002. Global products of vegetation leaf area and
fraction absorbed PAR from year one of MODIS data. Remote Sens. Environ.
83:214-231.
\phantomsection\label{\detokenize{tech_note/References/CLM50_Tech_Note_References:neffetal2005}}
\\

Neff, J.C., Harden, J.W. and Gleixner, G. 2005. Fire effects on soil
organic matter content, composition, and nutrients in boreal interior
Alaska. Canadian Journal of Forest Research-Revue Canadienne De
Recherche Forestiere 35:2178-2187.
\phantomsection\label{\detokenize{tech_note/References/CLM50_Tech_Note_References:neitschetal2005}}
\\

Neitsch, S.L., Arnold, J.G., Kiniry, J.R., and Williams J.R. 2005. Soil
and Water Assessment Tool, Theoretical Documentation: Version 2005.
Temple, TX. USDA Agricultural Research Service and Texas A\&M Blackland
Research Center.
\phantomsection\label{\detokenize{tech_note/References/CLM50_Tech_Note_References:negronjuarezetal2015}}
\\

Negron-Juarez, R. Koven, C.D., Riley, W.J., Knox, R.G., Chambers, J.Q.
2015. Environmental Research Letters 10:064017. DOI:10.1088/1748-9326/10/6/064017.
\phantomsection\label{\detokenize{tech_note/References/CLM50_Tech_Note_References:nemanirunning1996}}
\\

Nemani, R.R., and Running, S.W. 1996. Implementation of a hierarchical
global vegetation classification in ecosystem function models. J. Veg.
Sci. 7:337-346.
\phantomsection\label{\detokenize{tech_note/References/CLM50_Tech_Note_References:niinemetstal1998}}
\\

Niinemets, U., Kull, O., and Tenhunen, J.D. 1998. An analysis of light
effects on foliar morphology, physiology, and light interception in
temperate deciduous woody species of contrasting shade tolerance. Tree
Phys. 18:681-696.
\phantomsection\label{\detokenize{tech_note/References/CLM50_Tech_Note_References:niuetal2005}}
\\

Niu, G.-Y., Yang, Z.-L., Dickinson, R.E., and Gulden, L.E. 2005. A
simple TOPMODEL-based runoff parameterization (SIMTOP) for use in global
climate models. J. Geophys. Res. 110:D21106. DOI:10.1029/2005JD006111.
\phantomsection\label{\detokenize{tech_note/References/CLM50_Tech_Note_References:niuyang2006}}
\\

Niu, G.-Y., and Yang, Z.-L. 2006. Effects of frozen soil on snowmelt
runoff and soil water storage at a continental scale. J. Hydrometeor.
7:937-952.

Niu, G.-Y., Yang, Z.-L., Dickinson, R.E., Gulden, L.E., and Su, H. 2007.
Development of a simple groundwater model for use in climate models and
evaluation with Gravity Recovery and Climate Experiment data. J.
Geophys. Res. 112:D07103. DOI:10.1029/2006JD007522.

Niu, G.-Y., and Yang, Z.-L. 2007. An observation-based formulation of
snow cover fraction and its evaluation over large North American river
basins. J. Geophys. Res. 112:D21101. DOI:10.1029/2007JD008674.
\phantomsection\label{\detokenize{tech_note/References/CLM50_Tech_Note_References:oikawaetal2005}}
\\

Oikawa, S., Hikosaka, K. and Hirose, T., 2005. Dynamics of leaf area and
nitrogen in the canopy of an annual herb, Xanthium canadense. Oecologia,
143: 517-526.
\phantomsection\label{\detokenize{tech_note/References/CLM50_Tech_Note_References:oke1987}}
\\

Oke, T. 1987. Boundary Layer Climates (2:math:\sphinxtitleref{\{\}\textasciicircum{}\{nd\}} edition).
Routledge, London and New York.
\phantomsection\label{\detokenize{tech_note/References/CLM50_Tech_Note_References:olesonbonan2000}}
\\

Oleson, K.W., and Bonan, G.B. 2000. The effects of remotely-sensed plant
functional type and leaf area index on simulations of boreal forest
surface fluxes by the NCAR land surface model. J. Hydrometeor.
1:431-446.
\phantomsection\label{\detokenize{tech_note/References/CLM50_Tech_Note_References:olesonetal2004}}
\\

Oleson, K.W., Dai, Y., Bonan, G., Bosilovich, M., Dickinson, R.,
Dirmeyer, P., Hoffman, F., Houser, P., Levis, S., Niu, G.-Y., Thornton,
P., Vertenstein, M., Yang, Z.-L., and Zeng. X. 2004. Technical
description of the Community Land Model (CLM). NCAR Technical Note
NCAR/TN-461+STR. National Center for Atmospheric Research, Boulder,
Colorado. 173 pp.
\phantomsection\label{\detokenize{tech_note/References/CLM50_Tech_Note_References:olesonetal2008a}}
\\

Oleson, K.W., Niu, G.-Y., Yang, Z.-L., Lawrence, D.M., Thornton, P.E.,
Lawrence, P.J., Stöckli, R., Dickinson, R.E., Bonan, G.B., Levis, S.,
Dai, A., and Qian, T. 2008a. Improvements to the Community Land Model
and their impact on the hydrological cycle. J. Geophys. Res. 113:G01021.
DOI:10.1029/2007JG000563.
\phantomsection\label{\detokenize{tech_note/References/CLM50_Tech_Note_References:olesonetal2008b}}
\\

Oleson, K.W., Bonan, G.B., Feddema, J., Vertenstein, M., and Grimmond,
C.S.B. 2008b. An urban parameterization for a global climate model. 1.
Formulation and evaluation for two cities. J. Appl. Meteor. Clim.
47:1038-1060.
\phantomsection\label{\detokenize{tech_note/References/CLM50_Tech_Note_References:olesonetal2008c}}
\\

Oleson, K.W., Bonan, G.B., Feddema, J., and Vertenstein, M. 2008c. An
urban parameterization for a global climate model. 2. Sensitivity to
input parameters and the simulated urban heat island in offline
simulations. J. Appl. Meteor. Clim. 47:1061-1076.
\phantomsection\label{\detokenize{tech_note/References/CLM50_Tech_Note_References:olesonetal2010a}}
\\

Oleson, K.W., et al. 2010a. Technical description of version 4.0 of the
Community Land model (CLM). NCAR Technical Note NCAR/TN-478+STR,
National Center for Atmospheric Research, Boulder, CO, 257 pp.
\phantomsection\label{\detokenize{tech_note/References/CLM50_Tech_Note_References:olesonetal2010b}}
\\

Oleson, K.W., Bonan, G.B., Feddema, J., Vertenstein, M., and Kluzek, E.
2010b. Technical description of an urban parameterization for the
Community Land Model (CLMU). NCAR Technical Note NCAR/TN-480+STR,
National Center for Atmospheric Research, Boulder, CO, 169 pp.
\phantomsection\label{\detokenize{tech_note/References/CLM50_Tech_Note_References:olesonetal2013}}
\\

Oleson, K.W., et al. 2013. Technical description of version 4.5 of the
Community Land Model (CLM). NCAR Technical Note NCAR/TN-503+STR,
National Center for Atmospheric Research, Boulder, CO, 420 pp.
\phantomsection\label{\detokenize{tech_note/References/CLM50_Tech_Note_References:olson1963}}
\\

Olson, J.S., 1963. Energy storage and the balance of producers and
decomposers in ecological systems. Ecology 44:322-331.
\phantomsection\label{\detokenize{tech_note/References/CLM50_Tech_Note_References:olsonetal2001}}
\\

Olson, D.M., Dinerstein, E., Wikramanayake, E.D., Burgess, N.D., Powell,
G.V.N., Underwood, E.C., D’Amico, J.A., Itoua, I., Strand, H. E.,
Morrison, J. C., Loucks, C. J., Allnutt, T. F., Ricketts, T. H., Kura,
Y., Lamoreux, J. F., Wettengel, W. W., Heda, P., and Kassem, K. R.,
2001. Terrestrial ecoregions of the world a new map of life on earth,
Bioscience, 51, 933\textendash{}938.
\phantomsection\label{\detokenize{tech_note/References/CLM50_Tech_Note_References:orchardcook1983}}
\\

Orchard, V.A. and Cook, F.J., 1983. Relationship between soil
respiration and soil moisture. Soil Biology and Biochemistry, 15:
447-453.
\phantomsection\label{\detokenize{tech_note/References/CLM50_Tech_Note_References:owen1964}}
\\

Owen, P.R. 1964. Saltation of uniform grains in air. J. Fluid Mech\sphinxstyleemphasis{.}
20:225-242.
\phantomsection\label{\detokenize{tech_note/References/CLM50_Tech_Note_References:ozdoganetal2010}}
\\

Ozdogan, M., Rodell, M., Beaudoing, H.K., and Toll, D.L. 2010.
Simulating the effects of irrigation over the United States in a land
surface model based on satellite-derived agricultural data. Journal of
Hydrometeorology 11:171-184.
\phantomsection\label{\detokenize{tech_note/References/CLM50_Tech_Note_References:pageetal2002}}
\\

Page, S.E., Siegert, F., Rieley, J.O., Boehm, H-D.V., Jaya, A., and
Limin, S. 2002. The amount of carbon released from peat and forest fires
in Indonesia in 1997. Nature 420:61-65.
\phantomsection\label{\detokenize{tech_note/References/CLM50_Tech_Note_References:panofskydutton1984}}
\\

Panofsky, H.A., and Dutton, J.A. 1984. Atmospheric Turbulence: Models
and Methods for Engineering Applications. John Wiley and Sons, New York.
\phantomsection\label{\detokenize{tech_note/References/CLM50_Tech_Note_References:partonetal1988}}
\\

Parton, W., Stewart, J. and Cole, C., 1988. Dynamics of C, N, P And S in
Grassland Soils - A Model. Biogeochemistry 5:109-131.
\phantomsection\label{\detokenize{tech_note/References/CLM50_Tech_Note_References:partonetal1993}}
\\

Parton, W.J., et al. 1993. Observations and modeling of biomass and soil
organic matter dynamics for the grassland biome worlwide. Global
Biogeochemical Cycles 7:785-809.
\phantomsection\label{\detokenize{tech_note/References/CLM50_Tech_Note_References:partonetal1996}}
\\

Parton, W. et al. 1996. Generalized model for N2 and N2O production from
nitrification and denitrification. Global Biogeochemical Cycles
10:401-412.
\phantomsection\label{\detokenize{tech_note/References/CLM50_Tech_Note_References:partonetal2001}}
\\

Parton, W.J. et al. 2001. Generalized model for NOx and N2O emissions
from soils. J. Geophys. Res. 106(D15):17403-17419.
\phantomsection\label{\detokenize{tech_note/References/CLM50_Tech_Note_References:paterson1994}}
\\

Paterson, W.S.B., 1994. The Physics of Glaciers. Elsevier Science Inc.,
New York, 480 pp.
\phantomsection\label{\detokenize{tech_note/References/CLM50_Tech_Note_References:pelletieretal2016}}
\\

Pelletier, J. D., P. D. Broxton, P. Hazenberg, X. Zeng, P. A. Troch, G. Y. Niu, Z. Williams, M. A. Brunke, and D. Gochis, 2016: A gridded global data set of soil, intact regolith, and sedimentary deposit thicknesses for regional and global land surface modeling. J. Adv. Mod. Earth Sys. 8:41-65.
\phantomsection\label{\detokenize{tech_note/References/CLM50_Tech_Note_References:petrescuetal2010}}
\\

Petrescu, A.M.R. et al. 2010. Modeling regional to global CH4 emissions
of boreal and arctic wetlands. Global Biogeochemical Cycles, 24(GB4009).
\phantomsection\label{\detokenize{tech_note/References/CLM50_Tech_Note_References:philip1957}}
\\

Philip, J.R. 1957. Evaporation, and moisture and heat fields in the
soil. J. Meteor. 14:354-366.
\phantomsection\label{\detokenize{tech_note/References/CLM50_Tech_Note_References:piaoetal2012}}
\\

Piao, S.L., et al. 2012. The carbon budget of terrestrial ecosystems in
East Asia over the last two decades. Biogeosciences 9:3571-3586.
\phantomsection\label{\detokenize{tech_note/References/CLM50_Tech_Note_References:pivovarov1972}}
\\

Pivovarov, A.A., 1972. Thermal Conditions in Freezing Lakes and
Reservoirs. John Wiley, New York.
\phantomsection\label{\detokenize{tech_note/References/CLM50_Tech_Note_References:pollmeretal1979}}
\\

Pollmer, W.G., Eberhard, D., Klein, D., and Dhillon, B.S. 1979. Genetic
control of nitrogen uptake and translocation in maize. Crop Sci.
19:82-86.
\phantomsection\label{\detokenize{tech_note/References/CLM50_Tech_Note_References:pomeroyetal1998}}
\\

Pomeroy, J. W., D. M. Gray, K. R. Shook, B. Toth, R. L. H. Essery,
A. Pietroniro, and N. Hedstrom. 1998. An evaluation of snow accumulation
and ablation processes for land surface modelling. Hydrol. Process. 12:2339\textendash{}2367.
\phantomsection\label{\detokenize{tech_note/References/CLM50_Tech_Note_References:portmannetal2010}}
\\

Portmann, F.T., Siebert, S., and Döll, P. 2010. MIRCA2000 - Global
monthly irrigated and rainfed crop areas around the year 2000: A new
high-resolution data set for agricultural and hydrological modeling.
Global Biogeochem. Cycles. 24, GB1011. DOI:10.1029/2008GB003435.
\phantomsection\label{\detokenize{tech_note/References/CLM50_Tech_Note_References:pressetal1992}}
\\

Press, W.H., Teukolsky, S.A., Vetterling, W.T., and Flannery, B.P. 1992.
Numerical Recipes in FORTRAN: The Art of Scientific Computing. Cambridge
University Press, New York.
\phantomsection\label{\detokenize{tech_note/References/CLM50_Tech_Note_References:prigentetal2007}}
\\

Prigent, C., Papa, F., Aires, F., Rossow, W.B. and Matthews, E. 2007.
Global inundation dynamics inferred from multiple satellite
observations, 1993-2000. J. Geophys. Res. 112(D12).
\phantomsection\label{\detokenize{tech_note/References/CLM50_Tech_Note_References:pritchardetal2008}}
\\

Pritchard, M.S., Bush, A.B.G., and Marshall, S.J. 2008. Neglecting
ice-atmosphere interactions underestimates ice sheet melt in
millennial-scale deglaciation simulations. Geophys. Res. Lett. **
35:L01503. DOI:10.1029/2007GL031738.
\phantomsection\label{\detokenize{tech_note/References/CLM50_Tech_Note_References:qianetal2006}}
\\

Qian, T., Dai, A., Trenberth, K.E., and Oleson, K.W. 2006. Simulation of
global land surface conditions from 1948 to 2004: Part I: Forcing data
and evaluations. J. Hydrometeor. 7:953-975.
\phantomsection\label{\detokenize{tech_note/References/CLM50_Tech_Note_References:ramankuttyfoley1998}}
\\

Ramankutty, N., and Foley, J. A., 1998. Characterizing patterns of
global land use: An analysis of global croplands data. Global
Biogeochemical Cycles, 12, 667-685.
\phantomsection\label{\detokenize{tech_note/References/CLM50_Tech_Note_References:ramankuttyetal2008}}
\\

Ramankutty, N., Evan, A., Monfreda, C., and Foley, J.A. 2008. Farming
the Planet. Part 1: The Geographic Distribution of Global Agricultural
Lands in the Year 2000. Global Biogeochem. Cycles. 22:GB1003.
DOI:10.1029/2007GB002952.
\phantomsection\label{\detokenize{tech_note/References/CLM50_Tech_Note_References:randlettetal1996}}
\\

Randlett, D.L., Zak, D.R., Pregitzer, K.S., and Curtis, P.S. 1996.
Elevated atmospheric carbon dioxide and leaf litter chemistry:
Influences on microbial respiration and net nitrogen mineralization.
Soil Sci. Soc. Am. J. 60:1571-1577.
\phantomsection\label{\detokenize{tech_note/References/CLM50_Tech_Note_References:rastetteretal1991}}
\\

Rastetter, E.B., Ryan, M.G., Shaver, G.R., Melillo, J.M., Nadelhoffer,
K.J., Hobbie, J.E., and Aber, J.D. 1991. A general biogeochemical model
describing the responses of the C and N cycles in terrestrial ecosystems
to changes in CO2, climate and N deposition. Tree Phys. 9:101-126.
\phantomsection\label{\detokenize{tech_note/References/CLM50_Tech_Note_References:rastneretal2012}}
\\

Rastner, P., Bolch, T., Mölg, N., Machguth, H., and Paul, F., 2012. The
first complete glacier inventory for the whole of Greenland, The
Cryosphere Discuss., 6, 2399-2436, 10.5194/tcd-6-2399-2012.
\phantomsection\label{\detokenize{tech_note/References/CLM50_Tech_Note_References:rileyetal2011a}}
\\

Riley, W. J., Z. M. Subin, D. M. Lawrence, S. C. Swenson, M. S. Torn, L.
Meng, N. Mahowald, and P. Hess, 2011a. Barriers to predicting global
terrestrial methane fluxes: Analyses using a methane biogeochemistry
model integrated in CESM. Biogeosciences, 8, 1925\textendash{}1953.
DOI:10.5194/bg-8-1925-2011.
\phantomsection\label{\detokenize{tech_note/References/CLM50_Tech_Note_References:rileyetal2011b}}
\\

Riley, W.J. et al. 2011b. CLM4Me, a Methane Biogeochemistry Model
Integrated in CESM, Land and Biogeochemistry Model Working Group
Meeting, Boulder, CO.
\phantomsection\label{\detokenize{tech_note/References/CLM50_Tech_Note_References:roeschetal2001}}
\\

Roesch, A., M. Wild, H. Gilgen, and A. Ohmura. 2001. A new snow cover
fraction parametrization for the ECHAM4 GCM, Clim. Dyn., 17:933\textendash{}946.
\phantomsection\label{\detokenize{tech_note/References/CLM50_Tech_Note_References:rogersetal2017}}
\\

Rogers, A., B. E. Medlyn, J. S. Dukes, G. Bonan, S. Caemmerer, M. C. Dietze, J. Kattge, A. D. Leakey, L. M. Mercado, and U. Niinemets, 2017: A roadmap for improving the representation of photosynthesis in Earth system models. New Phytologist, 213:22-42.
\phantomsection\label{\detokenize{tech_note/References/CLM50_Tech_Note_References:ryan1991}}
\\

Ryan, M. G. 1991. A simple method for estimating gross carbon budgets
for vegetation in forest ecosystems. Tree Phys. 9:255-266.
\phantomsection\label{\detokenize{tech_note/References/CLM50_Tech_Note_References:runningcoughlan1988}}
\\

Running, S.W. and Coughlan, J.C., 1988. A general model of forest
ecosystem processes for regional applications. I. Hydrological balance,
canopy gas exchange and primary production processes. Ecological
Modelling, 42: 125-154.
\phantomsection\label{\detokenize{tech_note/References/CLM50_Tech_Note_References:runningetal1989}}
\\

Running, S.W. et al., 1989. Mapping regional forest evapotranspiration
and photosynthesis by coupling satellite data with ecosystem simlation.
Ecology, 70: 1090-1101.
\phantomsection\label{\detokenize{tech_note/References/CLM50_Tech_Note_References:runninggower1991}}
\\

Running, S.W. and Gower, S.T., 1991. FOREST BGC, A general model of
forest ecosystem processes for regional applications. II. Dynamic carbon
allocation and nitrogen budgets. Tree Physiology, 9: 147-160.
\phantomsection\label{\detokenize{tech_note/References/CLM50_Tech_Note_References:runninghunt1993}}
\\

Running, S.W. and Hunt, E.R., Jr., 1993. Generalization of a forest
ecosystem process model for other biomes, BIOME-BGC, and an
applicationfor global-scale models. In: J.R. Ehleringer and C. Field
(Editors), Scaling Physiological Processes: Leaf to Globe. Academic
Press, San Diego, CA, pp. 141-158.
\phantomsection\label{\detokenize{tech_note/References/CLM50_Tech_Note_References:sacksetal2009}}
\\

Sacks, W. J., Cook, B. I., Buenning, N., Levis, S., and Helkowski, J. H.
2009. Effects of global irrigation on the near-surface climate. Climate
Dyn., 33, 159\textendash{}175. DOI:10.1007/s00382-008-0445-z.
\phantomsection\label{\detokenize{tech_note/References/CLM50_Tech_Note_References:saggaretal1994}}
\\

Saggar, S., Tate, K.R., Feltham, C.W., Childs, C.W. and Parshotam, A.,
1994. Carbon turnover in a range of allophanic soils amended with
\({}^{14}\)C-labelled glucose. Soil Biology and Biochemistry, 26:
1263-1271.

Sakaguchi, K., and Zeng, X. 2009. Effects of soil wetness, plant litter,
and under-canopy atmospheric stability on ground evaporation in the
Community Land Model (CLM3.5). J. Geophys. Res. 114:D01107.
DOI:10.1029/2008JD010834.
\phantomsection\label{\detokenize{tech_note/References/CLM50_Tech_Note_References:schaafetal2002}}
\\

Schaaf, C.B., Gao, F., Strahler, A.H., Lucht, W., Li, X., Tsang, T.,
Strugnell, N.C., Zhang, X., Jin, Y., and Muller, J.-P. 2002. First
operational BRDF, albedo nadir reflectance products from MODIS. Remote
Sens. Environ. 83:135-148.
\phantomsection\label{\detokenize{tech_note/References/CLM50_Tech_Note_References:schlesinger1997}}
\\

Schlesinger, W.H., 1997. Biogeochemistry: an analysis of global change.
Academic Press, London, 588 pp.
\phantomsection\label{\detokenize{tech_note/References/CLM50_Tech_Note_References:schnellking1996}}
\\

Schnell, S. and King, G.M., 1996. Responses of methanotrophic activity
in soils and cultures to water stress. Applied and Environmental
Microbiology 62:3203-3209.
\phantomsection\label{\detokenize{tech_note/References/CLM50_Tech_Note_References:segers1998}}
\\

Segers, R., 1998. Methane production and methane consumption: a review
of processes underlying wetland methane fluxes. Biogeochemistry
41:23-51.
\phantomsection\label{\detokenize{tech_note/References/CLM50_Tech_Note_References:sellers1985}}
\\

Sellers, P.J. 1985. Canopy reflectance, photosynthesis and
transpiration. Int. J. Remote Sens. 6:1335-1372.
\phantomsection\label{\detokenize{tech_note/References/CLM50_Tech_Note_References:sellersetal1986}}
\\

Sellers, P.J., Mintz, Y., Sud, Y.C., and Dalcher, A. 1986. A simple
biosphere model (SiB) for use within general circulation models. J.
Atmos. Sci. 43:505-531.
\phantomsection\label{\detokenize{tech_note/References/CLM50_Tech_Note_References:sellersetal1988}}
\\

Sellers, P.J., Hall, F.G., Asrar, G., Strebel, D.E., and Murphy, R.E.
1988. The First ISLSCP Field Experiment (FIFE). Bull. Amer. Meteor. Soc.
69:22-27.
\phantomsection\label{\detokenize{tech_note/References/CLM50_Tech_Note_References:sellersetal1992}}
\\

Sellers, P.J., Berry, J.A., Collatz, G.J., Field, C.B., and Hall, F.G.
1992. Canopy reflectance, photosynthesis, and transpiration. III. A
reanalysis using improved leaf models and a new canopy integration
scheme. Remote Sens. Environ. 42:187-216.
\phantomsection\label{\detokenize{tech_note/References/CLM50_Tech_Note_References:sellersetal1995}}
\\

Sellers, P.J., et al. 1995. The Boreal Ecosystem-Atmosphere Study
(BOREAS): An overview and early results from the 1994 field year. Bull.
Amer. Meteor. Soc. 76:1549-1577.
\phantomsection\label{\detokenize{tech_note/References/CLM50_Tech_Note_References:sellersetal1996}}
\\

Sellers, P.J., Randall, D.A., Collatz, G.J., Berry, J.A., Field, C.B.,
Dazlich, D.A., Zhang, C., Collelo, G.D., and Bounoua, L. 1996. A revised
land surface parameterization (SiB2) for atmospheric GCMs. Part I: Model
formulation. J. Climate 9:676-705.
\phantomsection\label{\detokenize{tech_note/References/CLM50_Tech_Note_References:shietal2013}}
\\

Shi, X., Mao, J., Thornton, P.E., and Huang, M. 2013. Spatiotemporal
patterns of evapotranspiration in response to multiple environmental
factors simulated by the Community Land Model. Environ. Res. Lett.
8:024012.
\phantomsection\label{\detokenize{tech_note/References/CLM50_Tech_Note_References:shietal2016}}
\\

Shi, M., J. B. Fisher, E. R. Brzostek, and R. P. Phillips, 2016: Carbon cost of plant nitrogen acquisition: global carbon cycle impact from an improved plant nitrogen cycle in the Community Land Model. Glob. Change Biol., 22:1299-1314.
\phantomsection\label{\detokenize{tech_note/References/CLM50_Tech_Note_References:shiklomanov2000}}
\\

Shiklomanov, I.A. 2000. Appraisal and assessment of world water
resources. Water International 25:11-32.
\phantomsection\label{\detokenize{tech_note/References/CLM50_Tech_Note_References:siebertetal2005}}
\\

Siebert, S., Döll, P., Hoogeveen, J., Faures, J.M., Frenken, K., Feick,
S., 2005. Development and validation of the global map of irrigation
areas. Hydrol Earth Syst Sc 9:535\textendash{}547
\phantomsection\label{\detokenize{tech_note/References/CLM50_Tech_Note_References:simardetal2011}}
\\

Simard, M., Pinto, N., Fisher, J.B., and Baccini, A. (2011), Mapping
forest canopy height globally with spaceborne lidar.
J. Geophys. Res., 116, G04021, doi:10.1029/2011JG001708.
\phantomsection\label{\detokenize{tech_note/References/CLM50_Tech_Note_References:simpsonetal1983}}
\\

Simpson, R.J., Lambers, H., and Dalling, M.J. 1983. Nitrogen
redistribution during grain growth in wheat (Triticum avestivum L.).
Plant Physiol. 71:7-14.
\phantomsection\label{\detokenize{tech_note/References/CLM50_Tech_Note_References:sivak2013}}
\\

Sivak, M. 2013. Air conditioning versus heating: climate control is more
energy demanding in Minneapolis than in Miami. Environ. Res. Lett., 8,
doi:10.1088/1748-9326/8/1/014050.
\phantomsection\label{\detokenize{tech_note/References/CLM50_Tech_Note_References:smithetal2005}}
\\

Smith, A.M.S., Wooster, M.J., Drake, N.A., Dipotso, F.M. and Perry,
G.L.W., 2005. Fire in African savanna: Testing the impact of incomplete
combustion on pyrogenic emissions estimates. Ecological Applications,
15: 1074-1082.
\phantomsection\label{\detokenize{tech_note/References/CLM50_Tech_Note_References:sollins1982}}
\\

Sollins, P., 1982. Input and decay of coarse woody debris in coniferous
stands in western Oregon and Washington. Canadian Journal of Forest
Research, 12: 18-28.
\phantomsection\label{\detokenize{tech_note/References/CLM50_Tech_Note_References:songower1991}}
\\

Son, Y. and Gower, S.T., 1991. Aboveground nitrogen and phosphorus use
by five plantation-grown trees with different leaf longevities.
Biogeochemistry, 14: 167-191.
\phantomsection\label{\detokenize{tech_note/References/CLM50_Tech_Note_References:sorensen1981}}
\\

Sørensen, L.H., 1981. Carbon-nitrogen relationships during the
humification of cellulose in soils containing different amounts of clay.
Soil Biology and Biochemistry, 13: 313-321.
\phantomsection\label{\detokenize{tech_note/References/CLM50_Tech_Note_References:sperryetal1998}}
\\

Sperry, J.S., Adler, F.R., Campbell, G.S. and Comstock, J.P. 1998.
Limitation of plant water use by rhizosphere and xylem conductance:
results from a model. Plant, Cell \& Environment, 21: 347\textendash{}359.
doi:10.1046/j.1365-3040.1998.00287.x
\phantomsection\label{\detokenize{tech_note/References/CLM50_Tech_Note_References:sperryandlove2015}}
\\

Sperry, J.S. and Love, D.M. 2015. What plant hydraulics can tell us
about responses to climate-change droughts. New Phytol, 207: 14\textendash{}27.
doi:10.1111/nph.13354
\phantomsection\label{\detokenize{tech_note/References/CLM50_Tech_Note_References:sprugeletal1995}}
\\

Sprugel, D.G., Ryan, M.G., Brooks, J.R., Vogt, K.A., and Martin, T.A.
1995. Respiration from the organ level to stand level. pp. 255-299. In:
W. K. Smith and T. M. Hinkley (editors) Resource Physiology of Conifers.
Academic Press, San Diego,CA.
\phantomsection\label{\detokenize{tech_note/References/CLM50_Tech_Note_References:staufferaharony1994}}
\\

Stauffer, D., and Aharony, A. 1994. Introduction to Percolation Theory.
Taylor and Francis, London.
\phantomsection\label{\detokenize{tech_note/References/CLM50_Tech_Note_References:stilletal2003}}
\\

Still, C.J., Berry, J.A., Collatz, G.J., and DeFries, R.S. 2003. Global
distribution of C3 and C4 vegetation: carbon cycle implications. Global
Biogeochem. Cycles 17:1006. DOI:10.1029/2001GB001807.
\phantomsection\label{\detokenize{tech_note/References/CLM50_Tech_Note_References:stocklietal2008}}
\\

Stöckli, R., Lawrence, D.M., Niu, G.-Y., Oleson, K.W., Thornton, P.E.,
Yang, Z.-L., Bonan, G.B., Denning, A.S., and Running, S.W. 2008. Use of
FLUXNET in the Community Land Model development. J. Geophys. Res.
113:G01025. DOI:10.1029/2007JG000562.
\phantomsection\label{\detokenize{tech_note/References/CLM50_Tech_Note_References:stracketal2006}}
\\

Strack, M., Kellner, E. and Waddington, J.M., 2006. Effect of entrapped
gas on peatland surface level fluctuations. Hydrological Processes
20:3611-3622.
\phantomsection\label{\detokenize{tech_note/References/CLM50_Tech_Note_References:strahleretal1999}}
\\

Strahler, A.H., Muchoney, D., Borak, J., Friedl, M., Gopal, S., Lambin,
E., and Moody. A. 1999. MODIS Land Cover Product: Algorithm Theoretical
Basis Document (Version 5.0). Boston University, Boston.
\phantomsection\label{\detokenize{tech_note/References/CLM50_Tech_Note_References:stull1988}}
\\

Stull, R.B. 1988. An Introduction to Boundary Layer Meteorology. Kluwer
Academic Publishers, Dordrecht.
\phantomsection\label{\detokenize{tech_note/References/CLM50_Tech_Note_References:subinetal2012a}}
\\

Subin, Z.M., Riley, W.J. and Mironov, D. 2012a. Improved lake model for
climate simulations, J. Adv. Model. Earth Syst., 4, M02001.
DOI:10.1029/2011MS000072.
\phantomsection\label{\detokenize{tech_note/References/CLM50_Tech_Note_References:subinetal2012b}}
\\

Subin, Z.M., Murphy, L.N., Li, F., Bonfils, C. and Riley, W.J., 2012b.
Boreal lakes moderate seasonal and diurnal temperature variation and
perturb atmospheric circulation: analyses in the Community Earth System
Model 1 (CESM1). Tellus A, North America, 64.
\phantomsection\label{\detokenize{tech_note/References/CLM50_Tech_Note_References:sunetal2012}}
\\

Sun, Y., Gu, L., and Dickinson, R. E. 2012. A numerical issue in
calculating the coupled carbon and water fluxes in a climate model, J.
Geophys. Res., 117, D22103. DOI:10.1029/2012JD018059.
\phantomsection\label{\detokenize{tech_note/References/CLM50_Tech_Note_References:swensonetal2012}}
\\

Swenson, S.C., Lawrence, D.M., and Lee, H. 2012. Improved Simulation of
the Terrestrial Hydrological Cycle in Permafrost Regions by the
Community Land Model. JAMES, 4, M08002. DOI:10.1029/2012MS000165.
\phantomsection\label{\detokenize{tech_note/References/CLM50_Tech_Note_References:swensonlawrence2012}}
\\

Swenson, S.C. and Lawrence, D.M. 2012. A New Fractional Snow Covered
Area Parameterization for the Community Land Model and its Effect on the
Surface Energy Balance. JGR, 117, D21107. DOI:10.1029/2012JD018178.
\phantomsection\label{\detokenize{tech_note/References/CLM50_Tech_Note_References:swensonlawrence2014}}
\\

Swenson, S.C., and D. M. Lawrence. 2014. Assessing a dry surface
layer-based soil resistance parameterization for the Community Land Model
using GRACE and FLUXNET-MTE data. JGR, 119, 10, 299\textendash{}10,312,
DOI:10.1002/2014JD022314.
\phantomsection\label{\detokenize{tech_note/References/CLM50_Tech_Note_References:swensonlawrence2015}}
\\

Swenson, S.C., and D. M. Lawrence. 2015. A GRACE-based assessment of
interannual groundwater dynamics in the Community Land Model. WRR, 51,
doi:10.1002/2015WR017582.
\phantomsection\label{\detokenize{tech_note/References/CLM50_Tech_Note_References:taweiland1992}}
\\

Ta, C.T. and Weiland, R.T. 1992. Nitrogen partitioning in maize during
ear development. Crop Sci. 32:443-451.
\phantomsection\label{\detokenize{tech_note/References/CLM50_Tech_Note_References:tangriley2013}}
\\

Tang, J.Y. and Riley, W.J. 2013. A new top boundary condition for
modeling surface diffusive exchange of a generic volatile tracer:
Theoretical analysis and application to soil evaporation. Hydrol. Earth
Syst. Sci. 17:873-893.
\phantomsection\label{\detokenize{tech_note/References/CLM50_Tech_Note_References:tarnocaietal2011}}
\\

Tarnocai, C., Kettles, I. M., and Lacelle, B., 2011. Peatlands of
Canada, Geological Survey of Canada, Open File 6561, CD-ROM.
DOI:10.495/288786.
\phantomsection\label{\detokenize{tech_note/References/CLM50_Tech_Note_References:tayloretal1989}}
\\

Taylor, B.R., Parkinson, D. and Parsons, W.F.J., 1989. Nitrogen and
lignin content as predictors of litter decay rates: A microcosm test.
Ecology, 70: 97-104.
\phantomsection\label{\detokenize{tech_note/References/CLM50_Tech_Note_References:thomasetal2015}}
\\

Thomas R.Q., Brookshire E.N., Gerber S. 2015. Nitrogen limitation on land: how can it occur in Earth system models? Global Change Biology, 21, 1777-1793, doi:10.1111/gcb.12813.
\phantomsection\label{\detokenize{tech_note/References/CLM50_Tech_Note_References:thonickeetal2001}}
\\

Thonicke, K., Venevsky, S., Sitch, S., and Cramer, W. 2001. The role of
fire disturbance for global vegetation dynamics: coupling fire into a
Dynamic Global Vegetation Model. Global Ecology and Biogeography
10:661-667.
\phantomsection\label{\detokenize{tech_note/References/CLM50_Tech_Note_References:thornton1998}}
\\

Thornton, P.E., 1998. Regional ecosystem simulation: combining surface-
and satellite-based observations to study linkages between terrestrial
energy and mass budgets. Ph.D. Thesis, The University of Montana,
Missoula, 280 pp.
\phantomsection\label{\detokenize{tech_note/References/CLM50_Tech_Note_References:thorntonetal2002}}
\\

Thornton, P.E., Law, B.E., Gholz, H.L., Clark, K.L., Falge, E.,
Ellsworth, D.S., Goldstein, A.H., Monson, R.K., Hollinger, D., Falk, M.,
Chen, J., and Sparks, J.P. 2002. Modeling and measuring the effects of
disturbance history and climate on carbon and water budgets in evergreen
needleleaf forests. Agric. For. Meteor. 113:185-222.
\phantomsection\label{\detokenize{tech_note/References/CLM50_Tech_Note_References:thorntonrosenbloom2005}}
\\

Thornton, P.E., and Rosenbloom, N.A. 2005. Ecosystem model spin-up:
estimating steady state conditions in a coupled terrestrial carbon and
nitrogen cycle model. Ecological Modelling 189:25-48.
\phantomsection\label{\detokenize{tech_note/References/CLM50_Tech_Note_References:thorntonzimmermann2007}}
\\

Thornton, P.E., and Zimmermann, N.E. 2007. An improved canopy
integration scheme for a land surface model with prognostic canopy
structure. J. Climate 20:3902-3923.
\phantomsection\label{\detokenize{tech_note/References/CLM50_Tech_Note_References:thorntonetal2007}}
\\

Thornton, P.E., Lamarque, J.-F., Rosenbloom, N.A., and Mahowald, N.M.
2007. Influence of carbon-nitrogen cycle coupling on land model response
to CO$_{\text{2}}$ fertilization and climate variability. Global
Biogeochem. Cycles 21:GB4018.
\phantomsection\label{\detokenize{tech_note/References/CLM50_Tech_Note_References:thorntonetal2009}}
\\

Thornton, P.E., Doney, S.C., Lindsay, K., Moore, J.K., Mahowald, N.,
Randerson, J.T., Fung, I., Lamarque, J.F., Feddema, J.J., and Lee, Y.H.
2009. Carbon-nitrogen interactions regulate climate-carbon cycle
feedbacks: results from an atmosphere-ocean general circulation model.
Biogeosci. 6:2099-2120.
\phantomsection\label{\detokenize{tech_note/References/CLM50_Tech_Note_References:tianetal2010}}
\\

Tian, H. et al. 2010. Spatial and temporal patterns of CH4 and N2O
fluxes in terrestrial ecosystems of North America during 1979-2008:
application of a global biogeochemistry model. Biogeosciences
7:2673-2694.
\phantomsection\label{\detokenize{tech_note/References/CLM50_Tech_Note_References:toonetal1989}}
\\

Toon, O.B., McKay, C.P., Ackerman, T.P., and Santhanam, K. 1989. Rapid
calculation of radiative heating rates and photodissociation rates in
inhomogeneous multiple scattering atmospheres. J. Geophys. Res.
94(D13):16,287-16,301.
\phantomsection\label{\detokenize{tech_note/References/CLM50_Tech_Note_References:turetskyetal2002}}
\\

Turetsky, M.R., Wieder, R.K., Halsey, L.A., and Vitt, D.H. 2002. Current
disturbance and the diminishing peatland carbon sink. Geophys. Res.
Lett. 29:1526. DOI:10.1029/2001GL014000.
\phantomsection\label{\detokenize{tech_note/References/CLM50_Tech_Note_References:turetskyetal2004}}
\\

Turetsky, M.R., Amiro, B.D., Bosch, E., and Bhatti, J.S. 2004. Historical
burn area in western Canadian peatlands and its relationship to fire
weather indices. Global Biogeochem. Cycles 18:GB4014.
DOI:10.1029/2004GB002222.
\phantomsection\label{\detokenize{tech_note/References/CLM50_Tech_Note_References:tyeetal2005}}
\\

Tye, A.M., et al. 2005. The fate of N-15 added to high Arctic tundra to
mimic increased inputs of atmospheric nitrogen released from a melting
snowpack. Global Change Biology 11:1640-1654.
\phantomsection\label{\detokenize{tech_note/References/CLM50_Tech_Note_References:unlandetal1996}}
\\

Unland, H.E., Houser, P.R., Shuttleworth, W.J., and Yang, Z.-L. 1996.
Surface flux measurement and modeling at a semi-arid Sonoran Desert
site. Agric. For. Meteor. 82:119-153.
\phantomsection\label{\detokenize{tech_note/References/CLM50_Tech_Note_References:unstat2005}}
\\

UNSTAT, 2005. National Accounts Main Aggregates Database, United Nations
Statistics Division.
\phantomsection\label{\detokenize{tech_note/References/CLM50_Tech_Note_References:vallanosparks2007}}
\\

Vallano, D.M. and Sparks, J.P. 2007. Quantifying foliar uptake of
gaseous itrogen dioxide using enriched foliar
\(\delta^{15}\) N values. New Phytologist
177:946-955.
\phantomsection\label{\detokenize{tech_note/References/CLM50_Tech_Note_References:vanderwerfetal2010}}
\\

van der Werf, G.R., Randerson, J.T., Giglio, L., Collatz, G.J., Mu, M.,
Kasibhatla, S.P., Morton, D.C., DeFries, R.S., Jin, Y., van Leeuwen,
T.T. 2010. Global fire emissions and the contribution of deforestation,
savanna, forest, agricultural, and peat fires (1997-2009) Atmos. Chem.
Phys. 10:11707-11735.
\phantomsection\label{\detokenize{tech_note/References/CLM50_Tech_Note_References:van-veenetal1984}}
\\

van Veen, J.A., Ladd, J.N. and Frissel, M.J., 1984. Modelling C and N
turnover through the microbial biomass in soil. Plant and Soil, 76:
257-274.
\phantomsection\label{\detokenize{tech_note/References/CLM50_Tech_Note_References:vankampenhoutetal2017}}
\\

van Kampenhout, L., J.T.M. Lenaerts, W.H. Lipscomb, W.J. Sacks, D.M.
Lawrence, A.G. Slater, and M.R. van den Broeke, 2017.
Improving the representation of polar snow and firn in the
Community Earth System Model, submitted.
\phantomsection\label{\detokenize{tech_note/References/CLM50_Tech_Note_References:vanvuurenetal2006}}
\\

Van Vuuren, D.P., Lucas, P.S., and Hilderink, H.B.M., 2006. Downscaling
drivers of global environmental change: enabling use of global SRES
scenarios at the national and grid levels, Report 550025001, Netherlands
Environmental Assessment Agency, 45 pp.
\phantomsection\label{\detokenize{tech_note/References/CLM50_Tech_Note_References:vanninenmakela2005}}
\\

Vanninen, P., and Makela, A. 2005. Carbon budget for Scots pine trees:
effects of size, competition and site fertility on growth allocation and
production. Tree Phys. 25:17-30.
\phantomsection\label{\detokenize{tech_note/References/CLM50_Tech_Note_References:verdingreenlee1996}}
\\

Verdin, K. L., and S. K. Greenlee, 1996. Development of continental
scale digital elevation models and extraction of hydrographic features,
paper presented at the Third International Conference/Workshop on
Integrating GIS and Environmental Modeling, Santa Fe, New Mexico, 21\textendash{}26
January, Natl. Cent. for Geogr. Inf. and Anal., Santa Barbara, Calif.
\phantomsection\label{\detokenize{tech_note/References/CLM50_Tech_Note_References:viovy2011}}
\\

Viovy, N. 2011. CRUNCEP dataset. {[}Description available at
\sphinxurl{http://dods.extra.cea.fr/data/p529viov/cruncep/readme.htm}. Data
available at
\sphinxurl{http://dods.extra.cea.fr/store/p529viov/cruncep/V4\_1901\_2011/}{]}.
\phantomsection\label{\detokenize{tech_note/References/CLM50_Tech_Note_References:vitousekhowarth1991}}
\\

Vitousek, P.M., and Howarth, R.W. 1991. Nitrogen limitation on land and
in the sea: How can it occur? Biogeochem. 13:87-115.
\phantomsection\label{\detokenize{tech_note/References/CLM50_Tech_Note_References:walteretal2001}}
\\

Walter, B.P., Heimann, M. and Matthews, E., 2001. Modeling modern
methane emissions from natural wetlands 1. Model description and
results. J. Geophys. Res. 106(D24):34189-34206.
\phantomsection\label{\detokenize{tech_note/References/CLM50_Tech_Note_References:waniaetal2009}}
\\

Wania, R., Ross, I. and Prentice, I.C. 2009. Integrating peatlands and
permafrost into a dynamic global vegetation model: 2. Evaluation and
sensitivity of vegetation and carbon cycle processes. Global Biogeochem.
Cycles 23.
\phantomsection\label{\detokenize{tech_note/References/CLM50_Tech_Note_References:waniaetal2010}}
\\

Wania, R., Ross, I. and Prentice, I.C. 2010. Implementation and
evaluation of a new methane model within a dynamic global vegetation
model LPJ-WHyMe v1.3. Geoscientific Model Development Discussions
3:1-59.
\phantomsection\label{\detokenize{tech_note/References/CLM50_Tech_Note_References:wangzeng2009}}
\\

Wang, A., and Zeng, X. 2009. Improving the treatment of vertical snow
burial fraction over short vegetation in the NCAR CLM3. Adv. Atmos. Sci.
26:877-886. DOI:10.1007/s00376-009-8098-3.
\phantomsection\label{\detokenize{tech_note/References/CLM50_Tech_Note_References:whiteetal1997}}
\\

White, M.A., Thornton, P.E., and Running, S.W. 1997. A continental
phenology model for monitoring vegetation responses to interannual
climatic variability. Global Biogeochem. Cycles 11:217-234.
\phantomsection\label{\detokenize{tech_note/References/CLM50_Tech_Note_References:whiteetal2000}}
\\

White, M.A., Thornton, P.E., Running, S.W., and Nemani, R.R. 2000.
Parameterization and sensitivity analysis of the Biome-BGC terrestrial
ecosystem model: net primary production controls. Earth Interactions
4:1-85.
\phantomsection\label{\detokenize{tech_note/References/CLM50_Tech_Note_References:wiederetal2015}}
\\

Wieder, W. R., Cleveland, C. C., Lawrence, D. M., and Bonan, G. B. 2015.
Effects of model structural uncertainty on carbon cycle projections:
biological nitrogen fixation as a case study. Environmental Research
Letters, 10(4), 044016.
\phantomsection\label{\detokenize{tech_note/References/CLM50_Tech_Note_References:williamsetal1996}}
\\

Williams, M., Rastetter, E.B., Fernandes, D.N., Goulden, M.L.,
Wofsy, S.C., Shaver, G.R., Melillo, J.M., Munger, J.W., Fan, S.M.
and Nadelhoffer, K.J. 1996. Modelling the soil-plant-atmosphere
continuum in a Quercus\textendash{}Acer stand at Harvard Forest: the regulation
of stomatal conductance by light, nitrogen and soil/plant hydraulic
properties. Plant, Cell \& Environment, 19: 911\textendash{}927.
doi:10.1111/j.1365-3040.1996.tb00456.x
\phantomsection\label{\detokenize{tech_note/References/CLM50_Tech_Note_References:wiscombewarren1980}}
\\

Wiscombe, W.J., and Warren, S.G. 1980. A model for the spectral albedo
of snow. I. Pure snow. J. Atmos. Sci. 37:2712-2733.
\phantomsection\label{\detokenize{tech_note/References/CLM50_Tech_Note_References:woodetal1992}}
\\

Wood, E.F., Lettenmaier, D.P., and Zartarian, V.G. 1992. A land-surface
hydrology parameterization with subgrid variability for general
circulation models. J. Geophys. Res. 97(D3):2717\textendash{}2728.
DOI:10.1029/91JD01786.
\phantomsection\label{\detokenize{tech_note/References/CLM50_Tech_Note_References:worldbank2004}}
\\

World Bank, 2004. World development indicators 2004, Oxford University
Press, New York, 416 pp.
\phantomsection\label{\detokenize{tech_note/References/CLM50_Tech_Note_References:wuetal2011}}
\\

Wu, H., J. S. Kimball, N. Mantua, and J. Stanford, 2011: Automated
upscaling of river networks for macroscale hydrological modeling.
Water Resour. Res., 47, W03517, doi:10.1029/2009WR008871.
\phantomsection\label{\detokenize{tech_note/References/CLM50_Tech_Note_References:wuetal2012}}
\\

Wu, H., J. S. Kimball, H. Li, M. Huang, L. R. Leung, and R. F. Adler
(2012), A New Global River Network Database for Macroscale Hydrologic
modeling, Water Resour. Res., 48, W09701, doi:10.1029/2012WR012313.
\phantomsection\label{\detokenize{tech_note/References/CLM50_Tech_Note_References:xuetal2012}}
\\

Xu, C., R. Fisher, S. D. Wullschleger, C. J. Wilson, M. Cai, and N. G. McDowell, 2012: Toward a mechanistic modeling of nitrogen limitation on vegetation dynamics. PloS one, 7:e37914.
\phantomsection\label{\detokenize{tech_note/References/CLM50_Tech_Note_References:yang1998}}
\\

Yang, Z.-L. 1998. Technical note of a 10-layer soil moisture and
temperature model. Unpublished manuscript.
\phantomsection\label{\detokenize{tech_note/References/CLM50_Tech_Note_References:zenderetal2003}}
\\

Zender, C.S., Bian, H., and Newman, D. 2003. Mineral dust entrainment
and deposition (DEAD) model: Description and 1990s dust climatology. **
J. Geophys. Res\sphinxstyleemphasis{.} 108(D14):4416. DOI:10.1029/2002JD002775.
\phantomsection\label{\detokenize{tech_note/References/CLM50_Tech_Note_References:zengdickinson1998}}
\\

Zeng, X., and Dickinson, R.E. 1998. Effect of surface sublayer on
surface skin temperature and fluxes. J.Climate 11:537-550.
\phantomsection\label{\detokenize{tech_note/References/CLM50_Tech_Note_References:zengetal1998}}
\\

Zeng, X., Zhao, M., and Dickinson, R.E. 1998. Intercomparison of bulk
aerodynamic algorithms for the computation of sea surface fluxes using
the TOGA COARE and TAO data. J. Climate 11:2628-2644.
\phantomsection\label{\detokenize{tech_note/References/CLM50_Tech_Note_References:zeng2001}}
\\

Zeng, X. 2001. Global vegetation root distribution for land modeling. J.
Hydrometeor. 2:525-530.
\phantomsection\label{\detokenize{tech_note/References/CLM50_Tech_Note_References:zengetal2002}}
\\

Zeng, X., Shaikh, M., Dai, Y., Dickinson, R.E., and Myneni, R. 2002.
Coupling of the Common Land Model to the NCAR Community Climate Model.
J. Climate 15:1832-1854.
\phantomsection\label{\detokenize{tech_note/References/CLM50_Tech_Note_References:zengetal2005}}
\\

Zeng, X., Dickinson, R.E., Barlage, M., Dai, Y., Wang, G., and Oleson,
K. 2005. Treatment of under-canopy turbulence in land models. J. Climate
18:5086-5094.
\phantomsection\label{\detokenize{tech_note/References/CLM50_Tech_Note_References:zengwang2007}}
\\

Zeng, X., and Wang, A. 2007. Consistent parameterization of roughness
length and displacement height for sparse and dense canopies in land
models. J. Hydrometeor. 8:730-737.

Zeng, X., and Decker, M. 2009. Improving the numerical solution of soil
moisture-based Richards equation for land models with a deep or shallow
water table. J. Hydrometeor. 10:308-319.
\phantomsection\label{\detokenize{tech_note/References/CLM50_Tech_Note_References:zengetal2008}}
\\

Zeng, X., Zeng, X., and Barlage, M. 2008. Growing temperate shrubs over
arid and semiarid regions in the Community Land Model - Dynamic Global
Vegetation Model. Global Biogeochem. Cycles 22:GB3003.
DOI:10.1029/2007GB003014.
\phantomsection\label{\detokenize{tech_note/References/CLM50_Tech_Note_References:zhangetal2002}}
\\

Zhang, Y., Li, C.S., Trettin, C.C., Li, H. and Sun, G., 2002. An
integrated model of soil, hydrology, and vegetation for carbon dynamics
in wetland ecosystems. Global Biogeochemical Cycles 16.
DOI:10.1029/2001GB001838.
\phantomsection\label{\detokenize{tech_note/References/CLM50_Tech_Note_References:zhuangetal2004}}
\\

Zhuang, Q., et al. 2004. Methane fluxes between terrestrial ecosystems
and the atmosphere at northern high latitudes during the past century: A
retrospective analysis with a process-based biogeochemistry model.
Global Biogeochemical Cycles 18. DOI:10.1029/2004GB002239.
\phantomsection\label{\detokenize{tech_note/References/CLM50_Tech_Note_References:zilitinkevich1970}}
\\

Zilitinkevich, S.S. 1970. Dynamics of the Atmospheric Boundary Layer.
Leningrad Gidrometeor.
\phantomsection\label{\detokenize{tech_note/References/CLM50_Tech_Note_References:bonan2012}}
\\

Bonan, Gordon B et al. (2012). \sphinxtitleref{Reconciling leaf physiological traits and canopy flux data: Use of the TRY and FLUXNET databases in the Community Land Model version 4}. Journal of Geophysical Research: Biogeosciences (2005-2012) 117.G2.
\phantomsection\label{\detokenize{tech_note/References/CLM50_Tech_Note_References:botta2000}}
\\

Botta, A et al. (2000). \sphinxtitleref{A global prognostic scheme of leaf onset using satellite data}. Global Change Biology 6.7, pp. 709-725.
\phantomsection\label{\detokenize{tech_note/References/CLM50_Tech_Note_References:byram1959}}
\\

Byram, GM (1959). Combustion of forest fuels. In Forest fire: control and use.(Ed. KP Davis) pp. 61-89.
\phantomsection\label{\detokenize{tech_note/References/CLM50_Tech_Note_References:collatz1991}}
\\

Collatz, G James et al. (1991). \sphinxtitleref{Physiological and environmental regulation of stomatal conductance, photosynthesis and transpiration: a model that includes a laminar boundary layer} Agricultural and Forest Meteorology 54.2, pp. 107-136.
\phantomsection\label{\detokenize{tech_note/References/CLM50_Tech_Note_References:collatz1992}}
\\

Collatz, Go J, M Ribas-Carbo, and JA Berry (1992). \sphinxtitleref{Coupled photosynthesis-stomatal conductance model for leaves of C4 plants}. Functional Plant Biology 19.5, pp. 519-538.
\phantomsection\label{\detokenize{tech_note/References/CLM50_Tech_Note_References:farquhar1980}}
\\

Farquhar, GD, S von von Caemmerer, and JA Berry (1980). \sphinxtitleref{A biochemical model of photosynthetic CO2 assimilation in leaves of C3 species}. Planta 149.1, pp. 780-90.
\phantomsection\label{\detokenize{tech_note/References/CLM50_Tech_Note_References:fisher2010}}
\\

Fisher, JB et al. (2010). \sphinxtitleref{Carbon cost of plant nitrogen acquisition: A mechanistic, globally applicable model of plant nitrogen uptake, retranslocation, and fixation}. Global Biogeochemical Cycles 24.1.
\phantomsection\label{\detokenize{tech_note/References/CLM50_Tech_Note_References:foley1996}}
\\

Foley, Jonathan A et al. (1996). \sphinxtitleref{An integrated biosphere model of land surface processes, terrestrial carbon balance, and vegetation dynamics}. Global Biogeochemical Cycles 10.4, pp. 603-628.
\phantomsection\label{\detokenize{tech_note/References/CLM50_Tech_Note_References:fyllas2014}}
\\

Fyllas, NM et al. (2014). \sphinxtitleref{Analysing Amazonian forest productivity using a new individual and trait- based model (TFS v. 1)}. Geoscientific Model Development 7.4, pp. 1251-1269.
\phantomsection\label{\detokenize{tech_note/References/CLM50_Tech_Note_References:kucharik1998}}
\\

Kucharik, Christopher J, John M Norman, and Stith T Gower (1998). \sphinxtitleref{Measurements of branch area and adjusting leaf area index indirect measurements}. Agricultural and Forest Meteorology 91.1, pp. 69-88.
\phantomsection\label{\detokenize{tech_note/References/CLM50_Tech_Note_References:li2012}}
\\

Li, F, XD Zeng, and S Levis (2012). \sphinxtitleref{A process-based fire parameterization of intermediate complexity in a Dynamic Global Vegetation Model}. Biogeosciences 9.7, pp. 2761-2780.
\phantomsection\label{\detokenize{tech_note/References/CLM50_Tech_Note_References:lichstein2011}}
\\

Lichstein, Jeremy W and Stephen W Pacala (2011). \sphinxtitleref{Local diversity in heterogeneous landscapes: quantitative assessment with a height-structured forest metacommunity model}. Theoretical Ecology 4.2, pp. 269-281.
\phantomsection\label{\detokenize{tech_note/References/CLM50_Tech_Note_References:lischke2006}}
\\

Lischke, Heike et al. (2006). \sphinxtitleref{TreeMig: a forest-landscape model for simulating spatio-temporal patterns from stand to landscape scale}. Ecological Modelling 199.4, pp. 409-420. 41
\phantomsection\label{\detokenize{tech_note/References/CLM50_Tech_Note_References:lloyd2010}}
\\

Lloyd, J et al. (2010). \sphinxtitleref{Optimisation of photosynthetic carbon gain and within-canopy gradients of associated foliar traits for Amazon forest trees}. Biogeosciences 7.6, pp. 1833-1859.
\phantomsection\label{\detokenize{tech_note/References/CLM50_Tech_Note_References:mcdowell2013}}
\\

McDowell, Nate G et al. (2013). \sphinxtitleref{Evaluating theories of drought-induced vegetation mortality using a multimodel experiment framework}. New Phytologist 200.2, pp. 304-321.
\phantomsection\label{\detokenize{tech_note/References/CLM50_Tech_Note_References:medlyn2011}}
\\

Medlyn, Belinda E et al. (2011). \sphinxtitleref{Reconciling the optimal and empirical approaches to modelling stom- atal conductance}. Global Change Biology 17.6, pp. 2134-2144.
\phantomsection\label{\detokenize{tech_note/References/CLM50_Tech_Note_References:mc-2001}}
\\

Moorcroft, PR, GC Hurtt, and Stephen W Pacala (2001). \sphinxtitleref{A method for scaling vegetation dynamics: the ecosystem demography model ED}. Ecological monographs 71.4, pp. 557-586.
\phantomsection\label{\detokenize{tech_note/References/CLM50_Tech_Note_References:norman1979}}
\\

Norman, JM (1979). \sphinxtitleref{Modeling the complete crop canopy}. Modification of the Aerial Environment of Crops, pp. 249-280.
\phantomsection\label{\detokenize{tech_note/References/CLM50_Tech_Note_References:oleson2013}}
\\

Oleson, KW et al. (2013). \sphinxtitleref{Technical description of version 4.5 of the Community Land Model (CLM)}.
\phantomsection\label{\detokenize{tech_note/References/CLM50_Tech_Note_References:peterson1986}}
\\

Peterson, David L and Kevin C Ryan (1986). \sphinxtitleref{Modeling postfire conifer mortality for long-range planning}. Environmental Management 10.6, pp. 797-808.
\phantomsection\label{\detokenize{tech_note/References/CLM50_Tech_Note_References:pfeiffer2013}}
\\

Pfeiffer, M, A Spessa, and JO Kaplan (2013). \sphinxtitleref{A model for global biomass burning in preindustrial time: LPJ-LMfire (v1. 0)}. Geoscientific Model Development 6.3, pp. 643-685.
\phantomsection\label{\detokenize{tech_note/References/CLM50_Tech_Note_References:purves2008}}
\\

Purves, Drew W et al. (2008). \sphinxtitleref{Predicting and understanding forest dynamics using a simple tractable model}. Proceedings of the National Academy of Sciences 105.44, pp. 17018-17022.
\phantomsection\label{\detokenize{tech_note/References/CLM50_Tech_Note_References:sato2007}}
\\

Sato, Hisashi, Akihiko Itoh, and Takashi Kohyama (2007). \sphinxtitleref{SEIB\textendash{}DGVM: A new Dynamic Global Vegetation Model using a spatially explicit individual-based approach}. Ecological Modelling 200.3, pp. 2793307.
\phantomsection\label{\detokenize{tech_note/References/CLM50_Tech_Note_References:sellers1996}}
\\

Sellers, Piers J et al. (1996). \sphinxtitleref{A revised land surface parameterization (SiB2) for atmospheric GCMs. Part II: The generation of global fields of terrestrial biophysical parameters from satellite data}. Journal of climate 9.4, pp. 706-737.
\phantomsection\label{\detokenize{tech_note/References/CLM50_Tech_Note_References:sitch2003}}
\\

Sitch, S et al. (2003). \sphinxtitleref{Evaluation of ecosystem dynamics, plant geography and terrestrial carbon cycling in the LPJ dynamic global vegetation model}. Global Change Biology 9.2, pp. 161-185.
\phantomsection\label{\detokenize{tech_note/References/CLM50_Tech_Note_References:smith2007}}
\\

Smith, Alison M and Mark Stitt (2007). \sphinxtitleref{Coordination of carbon supply and plant growth}. Plant, cell \& environment 30.9, pp. 1126-1149.
\phantomsection\label{\detokenize{tech_note/References/CLM50_Tech_Note_References:smith2001}}
\\

Smith, Benjamin, I Colin Prentice, and Martin T Sykes (2001). \sphinxtitleref{Representation of vegetation dynamics in the modelling of terrestrial ecosystems: comparing two contrasting approaches within European climate space}. Global Ecology and Biogeography 10.6, pp. 621-637.
\phantomsection\label{\detokenize{tech_note/References/CLM50_Tech_Note_References:thonicke2010}}
\\

Thonicke, K et al. (2010). \sphinxtitleref{The influence of vegetation, fire spread and fire behaviour on biomass burning and trace gas emissions: results from a process-based model}. Biogeosciences 7.6, pp. 1991-2011.
\phantomsection\label{\detokenize{tech_note/References/CLM50_Tech_Note_References:uriarte2009}}
\\

Uriarte, Maria et al. (2009). \sphinxtitleref{Natural disturbance and human land use as determinants of tropical forest dynamics: results from a forest simulator}. Ecological Monographs 79.3, pp. 423-443.
\phantomsection\label{\detokenize{tech_note/References/CLM50_Tech_Note_References:venevsky2002}}
\\

Venevsky, Sergey et al. (2002). \sphinxtitleref{Simulating fire regimes in human-dominated ecosystems: Iberian Peninsula case study}. Global Change Biology 8.10, pp. 984-998.
\phantomsection\label{\detokenize{tech_note/References/CLM50_Tech_Note_References:weng2014}}
\\

Weng, ES et al. (2014). \sphinxtitleref{Scaling from individuals to ecosystems in an Earth System Model using a mathematically tractable model of height-structured competition for light}. Biogeosciences Discussions 11.12, pp. 17757-17860.
\phantomsection\label{\detokenize{tech_note/References/CLM50_Tech_Note_References:xiaodong2005}}
\\

Xiaodong, Yan and HH Shugart (2005). \sphinxtitleref{FAREAST: a forest gap model to simulate dynamics and patterns of eastern Eurasian forests}. Journal of Biogeography 32.9, pp. 1641-1658.


\chapter{Indices and tables}
\label{\detokenize{index:indices-and-tables}}\begin{itemize}
\item {} 
\DUrole{xref,std,std-ref}{genindex}

\item {} 
\DUrole{xref,std,std-ref}{modindex}

\item {} 
\DUrole{xref,std,std-ref}{search}

\end{itemize}



\renewcommand{\indexname}{Index}
\printindex
\end{document}